\documentclass[10pt,b5j]{tarticle} % B6 縦書き
% \documentclass[10pt,b5j]{tarticle} % B6 縦書き
\AtBeginDvi{\special{papersize=128mm,182mm}} % B6 用用紙サイズ
\usepackage{otf} % Unicode で字を入力するのに必要なパッケージ
\usepackage[size=b6j]{bxpapersize} % B6 用紙サイズを指定
\usepackage[dvipdfmx]{graphicx} % 画像を挿入するためのパッケージ
\usepackage[dvipdfmx]{color} % 色をつけるためのパッケージ
\usepackage{pxrubrica} % ルビを振るためのパッケージ
\usepackage{multicol} % 複数段組を作るためのパッケージ
\setlength{\topmargin}{14mm} % 上下方向のマージン
\addtolength{\topmargin}{-1in} % 
\setlength{\oddsidemargin}{11mm} % 左右方向のマージン
\addtolength{\oddsidemargin}{-1in} % 
\setlength{\textwidth}{154mm} % B6 用
\setlength{\textheight}{108mm} % B6 用
\setlength{\headsep}{0mm} % 
\setlength{\headheight}{0mm} % 
\setlength{\topskip}{0mm} % 
\setlength{\parskip}{0pt} % 
\def\labelenumi{\theenumi、} % 箇条書きのフォーマット
\parindent = 0pt % 段落下げしない

 % B6 用テンプレート読み込み

\begin{document}
% begin header
%%%%% タイトルと作者 ここから %%%%%
\begin{minipage}[c]{0.7\hsize} % タイトルは上から 7 割
    \begin{center}
    % begin title
        {\LARGE
            恵迪小唄 % タイトルを入れる
        }
        {\small 
            (平成十九年度寮歌) % 年などを入れる
        }
    % end title
    \end{center}
\end{minipage}
\begin{minipage}[c]{0.3\hsize} % 作歌作曲は上から 3 割
    \begin{flushright} % 下寄せにする
        % begin name
        井関俊介君 作歌\\八城雄太君 作曲 % 作歌・作曲者
        % end name
    \end{flushright}
\end{minipage}
%%%%% タイトルと作者 ここまで %%%%%
% (1,2,3,4 繰り返しなし)
% end header

% begin length
\vspace{1.5em} % タイトル, 作者と歌詞の間に隙間を設ける
\newcommand{\linespace}{0.5em} % 行間の設定
\newcommand{\blocksize}{0.5\hsize} % 段組間の設定
\newcommand{\itemmargin}{3em} % 曲番の位置調整の長さ
% end length
% begin body
%%%%% 歌詞 ここから %%%%%
\begin{enumerate} % 番号の箇条書き ここから
    \setlength{\itemindent}{\itemmargin} % 曲番の位置調整
    \begin{minipage}[c]{\blocksize}
    
        \vspace{\linespace}
        \item~\\
        % 1.
        \ruby{金}{きむ}がないのが\ruby{最初}{さいしょ}の\ruby{縁}{えん}で\\
        \ruby{入}{はい}ってみたのは\ruby{良}{よ}いけれど\\
        すみかはボロ\ruby{屋}{や}に\\
        \ruby{得体}{えたい}の\ruby{知}{し}れぬ\\
        \ruby{上}{うえ}の\ruby{年}{ねん}\ruby{目}{め}が\ruby{一}{いち}\ruby{絡}{から}げ ヤレ\\
        \ruby{想}{おも}えば\ruby{遠}{とお}くへ\ruby{来}{き}たもんだ
        
    \end{minipage}
    \begin{minipage}[c]{\blocksize}
        
        \vspace{\linespace}
        \item~\\
        % 2.
        \ruby{大志}{たいし}\ruby{抱}{いだ}きて\ruby{北都}{ほくと}へ\ruby{来}{き}たが\\
        \ruby{気付}{きづ}けば\ruby{朝寝}{あさね}に\ruby{高}{たか}いびき\\
        \ruby{自分}{じぶん}は\ruby{違}{ちが}うと\ruby{言}{い}ってはみたが\\
        サァ \ruby{明日}{あした}からがんばるぞ ヤレ\\
        \ruby{朱}{しゅ}に\ruby{交}{まじ}われば\ruby{朱}{しゅ}くなる
        
    \end{minipage}
    \begin{minipage}[c]{\blocksize}
        
        \vspace{\linespace}
        \item~\\
        % 3.
        \ruby{酒}{さけ}を\ruby{飲}{の}み\ruby{飲}{の}み\ruby{話}{はなし}もすれば\\
        \ruby{\ruby{突}{つ}然}{とつぜん}ドンパと\ruby{突}{つ}っ\ruby{張}{ぱ}り\ruby{合}{あ}い\\
        \ruby{時}{とき}には\ruby{突}{つ}き\ruby{上}{あ}げ\ruby{時}{とき}には\ruby{日和}{ひより}り\\
        \ruby{奴}{やつ}より\ruby{俺}{おれ}の\ruby{方}{ほう}が\ruby{上}{うえ} ヤレ\\
        \ruby{同}{おな}じ\ruby{団栗}{どんぐり}せいくらべ
        
    \end{minipage}
    \begin{minipage}[c]{\blocksize}
        
        \vspace{\linespace}
        \item~\\
        % 4.
        \ruby{先}{さき}は\ruby{長}{なが}いと\ruby{思}{おも}っていても\\
        \ruby{時間}{じかん}の\ruby{経}{た}つのは\ruby{早}{はや}いもの\\
        \ruby{苦楽}{くらく}を\ruby{共}{とも}に\ruby{住}{す}んではいたが\\
        \ruby{避}{さ}けては\ruby{通}{とお}れぬ\ruby{別}{わか}れ\ruby{道}{みち} ヤレ\\
        \ruby{縁}{えん}は\ruby{異}{いなものあじ}なもの\ruby{味}{}なもの
    
    \end{minipage}
\end{enumerate} % 番号の箇条書き ここまで
%%%%% 歌詞 ここまで %%%%%
% end body

\end{document}

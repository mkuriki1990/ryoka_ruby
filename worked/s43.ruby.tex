\documentclass[10pt,b5j]{tarticle} % B6 縦書き
% \documentclass[10pt,b5j]{tarticle} % B6 縦書き
\AtBeginDvi{\special{papersize=128mm,182mm}} % B6 用用紙サイズ
\usepackage{otf} % Unicode で字を入力するのに必要なパッケージ
\usepackage[size=b6j]{bxpapersize} % B6 用紙サイズを指定
\usepackage[dvipdfmx]{graphicx} % 画像を挿入するためのパッケージ
\usepackage[dvipdfmx]{color} % 色をつけるためのパッケージ
\usepackage{pxrubrica} % ルビを振るためのパッケージ
\usepackage{plext} % 漢数字の enumerate を使うためのパッケージ
\usepackage{multicol} % 複数段組を作るためのパッケージ
\setlength{\topmargin}{14mm} % 上下方向のマージン
\addtolength{\topmargin}{-1in} % 
\setlength{\oddsidemargin}{11mm} % 左右方向のマージン
\addtolength{\oddsidemargin}{-1in} % 
\setlength{\textwidth}{154mm} % B6 用
\setlength{\textheight}{108mm} % B6 用
\setlength{\headsep}{0mm} % 
\setlength{\headheight}{0mm} % 
\setlength{\topskip}{0mm} % 
\setlength{\parskip}{0pt} % 
\def\theenumi{\Kanji{enumi}} % 箇条書きのフォーマットを漢数字に変更
\parindent = 0pt % 段落下げしない
\pagestyle{empty} % すべてのページ番号を消去
% \renewcommand{\baselinestretch}{0.9} % 行間の倍率
 % B6 用テンプレート読み込み

\begin{document}
% begin header
%%%%% タイトルと作者 ここから %%%%%
\begin{minipage}[c]{0.7\hsize} % タイトルは上から 7 割
    \begin{center}
    % begin title
        {\LARGE
            樹梢霧海に % タイトルを入れる
        }
        {\small 
            (昭和四十三年寮歌) % 年などを入れる
        }
    % end title
    \end{center}
\end{minipage}
\begin{minipage}[c]{0.3\hsize} % 作歌作曲は上から 3 割
    \begin{flushright} % 下寄せにする
        % begin name
        新橋登君 作歌\\佐藤菊男君 作曲 % 作歌・作曲者
        % end name
    \end{flushright}
\end{minipage}
%%%%% タイトルと作者 ここまで %%%%%
% (1,4,転句 繰り返しなし)
% end header

% begin length
\vspace{1.5em} % タイトル, 作者と歌詞の間に隙間を設ける
\newcommand{\linespace}{0.5em} % 行間の設定
\newcommand{\blocksize}{0.33\hsize} % 段組間の設定
\newcommand{\itemmargin}{3em} % 曲番の位置調整の長さ
% end length
% begin body
%%%%% 歌詞 ここから %%%%%
\begin{enumerate} % 番号の箇条書き ここから
    \setlength{\itemindent}{\itemmargin} % 曲番の位置調整
    \begin{minipage}[c]{\blocksize}
    
        \vspace{\linespace}
        \item~\\
        % 1.
        \ruby{樹梢}{じゅ|しょう}\ruby{霧海}{む|かい}に\ruby{消}{き}え\ruby{入}{い}りて\\
        \ruby{北溟}{ほく|めい}\ruby{牙城}{が|じょう}の\ruby{夏}{なつ}の\ruby{宵}{よい}\\
        \ruby{難攻不落}{なん|こう|ふ|らく}を\ruby{誇}{ほこ}りしも\\
        \ruby{時}{とき}\ruby{凋衰}{ちょう|すい}の\ruby{風}{かぜ}\ruby{強}{つよ}し
        
        \vspace{\linespace}
        \item~\\
        % 2.
        \ruby[g]{伝統}{つたへ}の\ruby{石}{いし}に\ruby{佇}{たたず}みて\\
        \ruby[g]{古昔}{むかし}の\ruby{意気}{い|き}に\ruby{涙}{なみだ}する\\
        \ruby{秋}{あき}の\ruby[g]{今宵}{こよい}の\ruby{宴}{うたげ}にも\\
        \ruby{貧}{ひん}\ruby{交行}{こう|こう}の\ruby{風}{かぜ}\ruby{寒}{さむ}し
        
    \end{minipage}
    \begin{minipage}[c]{\blocksize}
        
        \vspace{\linespace}
        \item[転句]~\\
        % \ruby{転}{てん}\ruby{句}{く}.
        \ruby{楡陵}{ゆ|りょう}の\ruby{二春}{に|しゅん}に\ruby{宿}{やど}せる\ruby{白露}{しら|つゆ}の\\
        \ruby[g]{生命}{いのち}\ruby[g]{短命}{みじか}にして\ruby{吉}{よ}しとする\\
        さにあらば\ruby{吾等}{われ|ら}が\ruby{友}{とも}よ\\
        \ruby[g]{久遠}{とわ}なる\ruby{星}{ほし}に\\
        \ruby{崇厳}{すう|げん}に\ruby{大志}{たい|し}を\ruby{告}{つ}げるべく\\
        \ruby{今}{いま}\ruby{高}{たか}らかに\ruby{誓}{ちか}いけん
        
    \end{minipage}
    \begin{minipage}[c]{\blocksize}
        
        \vspace{\linespace}
        \item~\\
        % 3.
        \ruby{白雪}{はく|せつ}\ruby{深}{ふか}き\ruby{北国}{きた|ぐに}に\\
        \ruby{迪}{みち}をたずねる\ruby{旅人}{たび|びと}よ\\
        \ruby{朔風}{さく|ふう}\ruby[g]{如何}{いか}に\ruby[g]{荒吹}{すさぶ}とも\\
        \ruby[g]{真理}{まこと}の\ruby{郷}{さと}は\ruby{遠}{とお}からじ
        
        \vspace{\linespace}
        \item~\\
        % 4.
        いざ\ruby[g]{寮友}{ともだち}ようたわなん\\
        あすの\ruby[g]{生命}{いのち}を\ruby{闘}{たたか}うと\\
        \ruby{万花}{ばん|か}\ruby{乱}{みだ}るる\ruby{春}{はる}の\ruby{日}{ひ}に\\
        \ruby[g]{高遠}{たか}き\ruby[g]{大望}{のぞみ}を\ruby{目指}{め|ざ}さんや
    
    \end{minipage}
\end{enumerate} % 番号の箇条書き ここまで
%%%%% 歌詞 ここまで %%%%%
% end body

\end{document}

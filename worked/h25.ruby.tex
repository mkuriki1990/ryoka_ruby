\documentclass[10pt,b5j]{tarticle} % B6 縦書き
% \documentclass[10pt,b5j]{tarticle} % B6 縦書き
\AtBeginDvi{\special{papersize=128mm,182mm}} % B6 用用紙サイズ
\usepackage{otf} % Unicode で字を入力するのに必要なパッケージ
\usepackage[size=b6j]{bxpapersize} % B6 用紙サイズを指定
\usepackage[dvipdfmx]{graphicx} % 画像を挿入するためのパッケージ
\usepackage[dvipdfmx]{color} % 色をつけるためのパッケージ
\usepackage{pxrubrica} % ルビを振るためのパッケージ
\usepackage{multicol} % 複数段組を作るためのパッケージ
\setlength{\topmargin}{14mm} % 上下方向のマージン
\addtolength{\topmargin}{-1in} % 
\setlength{\oddsidemargin}{11mm} % 左右方向のマージン
\addtolength{\oddsidemargin}{-1in} % 
\setlength{\textwidth}{154mm} % B6 用
\setlength{\textheight}{108mm} % B6 用
\setlength{\headsep}{0mm} % 
\setlength{\headheight}{0mm} % 
\setlength{\topskip}{0mm} % 
\setlength{\parskip}{0pt} % 
\def\labelenumi{\theenumi、} % 箇条書きのフォーマット
\parindent = 0pt % 段落下げしない

 % B6 用テンプレート読み込み

\begin{document}
% begin header
%%%%% タイトルと作者 ここから %%%%%
\begin{minipage}[c]{0.7\hsize} % タイトルは上から 7 割
    \begin{center}
    % begin title
        {\LARGE
            北溟の我らぞ % タイトルを入れる
        }
        {\small 
            (平成二十五年新々寮三十周年記念寮歌) % 年などを入れる
        }
    % end title
    \end{center}
\end{minipage}
\begin{minipage}[c]{0.3\hsize} % 作歌作曲は上から 3 割
    \begin{flushright} % 下寄せにする
        % begin name
        森貝聡恵君 作歌\\菊池玄之介君 作曲 % 作歌・作曲者
        % end name
    \end{flushright}
\end{minipage}
%%%%% タイトルと作者 ここまで %%%%%
% (1,2,3 繰り返しなし)
% end header

% begin length
\vspace{1.5em} % タイトル, 作者と歌詞の間に隙間を設ける
\newcommand{\linespace}{0.5em} % 行間の設定
\newcommand{\blocksize}{0.5\hsize} % 段組間の設定
\newcommand{\itemmargin}{3em} % 曲番の位置調整の長さ
% end length
% begin body
%%%%% 歌詞 ここから %%%%%
\begin{enumerate} % 番号の箇条書き ここから
    \setlength{\itemindent}{\itemmargin} % 曲番の位置調整
    \begin{minipage}[c]{\blocksize}
    
        \vspace{\linespace}
        \item~\\
        % 1.
        \ruby{君}{きみ}\ruby{何故}{な|ぜ}\ruby{来}{き}たるこの\ruby[g]{北溟}{きた}の\ruby{地}{ち}に\\
        かの\ruby{師}{し}の\ruby{教}{おし}へ\ruby{受}{う}け\ruby{継}{つ}ぎし\ruby{地}{ち}に\\
        \ruby{貴}{とうと}き\ruby{野心}{や|しん}ゆめ\ruby{忘}{わす}るまじ\\
        ひたと\ruby{気高}{け|だか}く\\
        いざ\ruby{踏}{ふ}み\ruby{出}{だ}さむ\ruby{新}{あら}たなる\ruby{夢}{ゆめ}へ
        
        \vspace{\linespace}
        \item~\\
        % 2.
        \ruby{野心}{や|しん}は\ruby{強}{つよ}く\ruby{熱}{あつ}くとも\\
        \ruby{道草}{みち|くさ}\ruby{恋}{こい}しき\ruby{時}{とき}もあるかな\\
        \ruby{青}{あお}き\ruby{春}{はる}の\ruby{夜}{よる}\ruby{友}{とも}らが\ruby{宴}{うたげ}\\
        \ruby{語}{かた}り\ruby{合}{あ}おうぞ\\
        \ruby{我}{われ}らが\ruby{大}{おお}き\ruby{大}{おお}き\ruby{夢}{ゆめ}の\ruby{芽}{め}
        
    \end{minipage}
    \begin{minipage}[c]{\blocksize}
        
        \vspace{\linespace}
        \item~\\
        % 3.
        \ruby{我}{われ}らが\ruby[g]{未来}{さき}はいかなるものか\\
        \ruby{曇}{くも}り\ruby{澱}{よど}みし\ruby{過去}{か|こ}ではあれど\\
        これより\ruby{先}{さき}は\ruby{我}{われ}らが\ruby{拓}{ひら}かむ\\
        \ruby{先達}{せん|だつ}に\ruby{続}{つづ}け\\
        \ruby[g]{大和}{やまと}の\ruby{栄}{は}えをば\ruby{担}{にな}うは\ruby{我}{われ}らぞ
    
    \end{minipage}
\end{enumerate} % 番号の箇条書き ここまで
%%%%% 歌詞 ここまで %%%%%
% end body

\end{document}

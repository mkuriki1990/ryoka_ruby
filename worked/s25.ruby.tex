\documentclass[10pt,b5j]{tarticle} % B6 縦書き
% \documentclass[10pt,b5j]{tarticle} % B6 縦書き
\AtBeginDvi{\special{papersize=128mm,182mm}} % B6 用用紙サイズ
\usepackage{otf} % Unicode で字を入力するのに必要なパッケージ
\usepackage[size=b6j]{bxpapersize} % B6 用紙サイズを指定
\usepackage[dvipdfmx]{graphicx} % 画像を挿入するためのパッケージ
\usepackage[dvipdfmx]{color} % 色をつけるためのパッケージ
\usepackage{pxrubrica} % ルビを振るためのパッケージ
\usepackage{multicol} % 複数段組を作るためのパッケージ
\setlength{\topmargin}{14mm} % 上下方向のマージン
\addtolength{\topmargin}{-1in} % 
\setlength{\oddsidemargin}{11mm} % 左右方向のマージン
\addtolength{\oddsidemargin}{-1in} % 
\setlength{\textwidth}{154mm} % B6 用
\setlength{\textheight}{108mm} % B6 用
\setlength{\headsep}{0mm} % 
\setlength{\headheight}{0mm} % 
\setlength{\topskip}{0mm} % 
\setlength{\parskip}{0pt} % 
\def\labelenumi{\theenumi、} % 箇条書きのフォーマット
\parindent = 0pt % 段落下げしない

 % B6 用テンプレート読み込み

\begin{document}
% begin header
%%%%% タイトルと作者 ここから %%%%%
\begin{minipage}[c]{0.7\hsize} % タイトルは上から 7 割
    \begin{center}
    % begin title
        {\LARGE
            悠遠き日にあこがれて % タイトルを入れる
        }
        {\small 
            (昭和二十五年寮歌) % 年などを入れる
        }
    % end title
    \end{center}
\end{minipage}
\begin{minipage}[c]{0.3\hsize} % 作歌作曲は上から 3 割
    \begin{flushright} % 下寄せにする
        % begin name
        高倉和昭君 作歌\\金井倶光君 作曲 % 作歌・作曲者
        % end name
    \end{flushright}
\end{minipage}
%%%%% タイトルと作者 ここまで %%%%%
% (1 繰り返しなし)
% end header

% begin length
\vspace{1.0em} % タイトル, 作者と歌詞の間に隙間を設ける
\newcommand{\linespace}{0.3em} % 行間の設定
\newcommand{\blocksize}{0.5\hsize} % 段組間の設定
\newcommand{\itemmargin}{3em} % 曲番の位置調整の長さ
% end length
% begin body
%%%%% 歌詞 ここから %%%%%
\begin{enumerate} % 番号の箇条書き ここから
    \setlength{\itemindent}{\itemmargin} % 曲番の位置調整
    \begin{minipage}[c]{\blocksize}
    
        \vspace{\linespace}
        \item~\\
        % 1.
        \ruby[g]{悠遠}{とほ}き\ruby{日}{ひ}にあこがれて\ruby{吾}{われ}は\ruby{来}{き}たりぬ\\
        \ruby{北国}{きた|ぐに}の\ruby{詩}{うた}の\ruby{都}{みやこ}ぞ\\
        やはらかき\ruby{緑}{みどり}の\ruby{芝生}{しば|ふ}\\
        \ruby{美}{うる}はしき\ruby{小川}{を|がは}の\ruby{畔}{ほとり}\\
        \ruby{清明}{せい|めい}の\ruby{森蔭}{もり|かげ}\ruby{深}{ふか}く\ruby{訪}{たづ}ね\ruby{来}{き}て\\
        \ruby{新}{あた}らしき\ruby{喜}{よろこ}びに\ruby{満}{み}つ
        
        \vspace{\linespace}
        \item~\\
        % 2.
        \ruby{讃}{たた}へなむ\ruby{石狩}{いし|かり}の\ruby[g]{曠野}{の}に\ruby{打}{うち}\ruby{建}{た}てし\\
        \ruby[g]{雄大}{おほ}いなる\ruby{先人}{せん|じん}が\ruby[g]{足跡}{あと}\\
        \ruby[g]{四十三回}{よそみたび}\ruby[g]{記念祭}{まつり}\ruby{巡}{めぐ}りて\\
        \ruby[g]{光栄}{ひかり}あれ\ruby[g]{伝統}{つたへ}の\ruby[g]{法燈}{ともし}\\
        \ruby[g]{星辰}{ほし}\ruby{清}{きよ}きエルムの\ruby[g]{学園}{その}に\ruby{甦}{か}へりたる\\
        \ruby{鐘}{かね}の\ruby{音}{ね}は\ruby{高}{たか}く\ruby{鳴}{な}るなり
        
    \end{minipage}
    \begin{minipage}[c]{\blocksize}
        
        \vspace{\linespace}
        \item~\\
        % 3.
        あかつきは\ruby{紫}{むらさき}の\ruby{夢}{ゆめ}にけむれり\\
        \ruby{雪解}{ゆき|げ}なる\ruby{陵}{をか}にのぼりて\\
        \ruby{恋}{こ}ひ\ruby{慕}{した}ふ\ruby{意気}{い|き}と\ruby{血汐}{ち|しほ}の\\
        \ruby{花}{はな}\ruby{香}{かを}る\ruby{青史}{せい|し}の\ruby[g]{光栄}{はえ}よ\\
        \ruby{二春}{ふた|はる}を\ruby{魂}{こころ}の\ruby[g]{故郷}{さと}に\ruby{契}{ちぎ}りては\\
        \ruby{培}{つちか}はん\ruby{尊}{とうと}き\ruby[g]{遺訓}{をしへ}
        
        \vspace{\linespace}
        \item~\\
        % 4.
        \ruby{仰}{あふ}ぎ\ruby{見}{み}よ\ruby{秀}{ひい}でたる\ruby{久遠}{く|をん}の\ruby{山河}{さん|が}\\
        \ruby{悠久}{いう|きう}の\ruby{時}{とき}の\ruby{移}{うつ}ろひ\\
        \ruby{森蔭}{もり|かげ}に\ruby[g]{心情}{こころ}は\ruby{燃}{も}えて\\
        \ruby{恵}{もと}むなり\ruby[g]{真理}{まこと}の\ruby{秘奥}{ひ|あう}\\
        \ruby[g]{青春}{わかきひ}の\ruby[g]{高遠}{たか}き\ruby[g]{理想}{おもひ}を\ruby{抱}{いだ}きては\\
        \ruby{進}{すす}まなむ\ruby{厳}{きび}しかる\ruby{道}{みち}
    
    \end{minipage}
\end{enumerate} % 番号の箇条書き ここまで
%%%%% 歌詞 ここまで %%%%%
% end body

\end{document}

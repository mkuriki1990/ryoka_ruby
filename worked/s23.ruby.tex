\documentclass[10pt,b5j]{tarticle} % B6 縦書き
% \documentclass[10pt,b5j]{tarticle} % B6 縦書き
\AtBeginDvi{\special{papersize=128mm,182mm}} % B6 用用紙サイズ
\usepackage{otf} % Unicode で字を入力するのに必要なパッケージ
\usepackage[size=b6j]{bxpapersize} % B6 用紙サイズを指定
\usepackage[dvipdfmx]{graphicx} % 画像を挿入するためのパッケージ
\usepackage[dvipdfmx]{color} % 色をつけるためのパッケージ
\usepackage{pxrubrica} % ルビを振るためのパッケージ
\usepackage{multicol} % 複数段組を作るためのパッケージ
\setlength{\topmargin}{14mm} % 上下方向のマージン
\addtolength{\topmargin}{-1in} % 
\setlength{\oddsidemargin}{11mm} % 左右方向のマージン
\addtolength{\oddsidemargin}{-1in} % 
\setlength{\textwidth}{154mm} % B6 用
\setlength{\textheight}{108mm} % B6 用
\setlength{\headsep}{0mm} % 
\setlength{\headheight}{0mm} % 
\setlength{\topskip}{0mm} % 
\setlength{\parskip}{0pt} % 
\def\labelenumi{\theenumi、} % 箇条書きのフォーマット
\parindent = 0pt % 段落下げしない

 % B6 用テンプレート読み込み

\begin{document}
% begin header
%%%%% タイトルと作者 ここから %%%%%
\begin{minipage}[c]{0.7\hsize} % タイトルは上から 7 割
    \begin{center}
    % begin title
        {\LARGE
            饗宴の杯に % タイトルを入れる
        }
        {\small 
            (昭和二十三年寮歌) % 年などを入れる
        }
    % end title
    \end{center}
\end{minipage}
\begin{minipage}[c]{0.3\hsize} % 作歌作曲は上から 3 割
    \begin{flushright} % 下寄せにする
        % begin name
        中坪清八君 作歌\\堀井洵君 作曲 % 作歌・作曲者
        % end name
    \end{flushright}
\end{minipage}
%%%%% タイトルと作者 ここまで %%%%%
% (1,6 了あり)
% end header

% begin length
\vspace{1.5em} % タイトル, 作者と歌詞の間に隙間を設ける
\newcommand{\linespace}{0.5em} % 行間の設定
\newcommand{\blocksize}{0.33\hsize} % 段組間の設定
\newcommand{\itemmargin}{3em} % 曲番の位置調整の長さ
% end length
% begin body
%%%%% 歌詞 ここから %%%%%
\begin{enumerate} % 番号の箇条書き ここから
    \setlength{\itemindent}{\itemmargin} % 曲番の位置調整
    \begin{minipage}[c]{\blocksize}
    
        \vspace{\linespace}
        \item~\\
        % 1.
        \ruby[g]{饗宴}{うたげ}の\ruby{杯}{つき}に\ruby{淡}{うす}れゆく\\
        \ruby{手稲}{て|いね}の\ruby{峰}{みね}に\ruby{今}{いま}しばし\\
        \ruby[g]{追憶}{おもひ}\ruby{止}{とど}めて\ruby{涙}{なみだ}する\\
        \ruby{逝}{ゆ}く\ruby{水}{みづ}はやき\ruby[g]{三春秋}{みつとせ}の\\
        \ruby{絵巻}{ゑ|まき}はやがて\ruby{尽}{つ}きざらん\\
        \ruby{優}{ゆか}しき\ruby[g]{薫香}{かをり}\ruby{遺}{のこ}しつつ
        
        \vspace{\linespace}
        \item~\\
        % 2.
        \ruby[g]{真理}{まこと}の\ruby{道}{みち}の\ruby[g]{彷徨}{あこがれ}に\\
        \ruby{遊子}{いう|し}は\ruby{尋}{と}めぬ\ruby{人性}{ひと|さが}を\\
        \ruby[g]{真紅}{あけ}に\ruby{輝}{かがや}く\ruby{森蔭}{もり|かげ}に\\
        \ruby{榾火}{ほた|び}\ruby{廻}{めぐ}りて\ruby{歌}{うた}へども\\
        \ruby{琥珀}{こ|はく}の\ruby{酒}{さけ}を\ruby{酌}{く}みしかど\\
        \ruby{贏}{かちえ}しものは\ruby{何}{なに}ならん
        
    \end{minipage}
    \begin{minipage}[c]{\blocksize}
        
        \vspace{\linespace}
        \item~\\
        % 3.
        \ruby[g]{原始林}{もり}の\ruby[g]{濃緑}{みどり}のまどろみに\\
        \ruby[g]{高夢}{ゆめ}は\ruby{結}{むす}びぬ\ruby{先人}{せん|じん}の\\
        \ruby[g]{遺訓}{さとし}の\ruby{蔭}{かげ}に\ruby{泪}{なみだ}あり\\
        \ruby{孤雁}{こ|がん}\ruby{一}{ひと}たび\ruby[g]{大地}{ち}に\ruby{啼}{な}きて\\
        \ruby{驚}{おどろ}き\ruby{醒}{さ}むる\ruby{邯鄲}{かん|たん}の\\
        \ruby[g]{草野}{の}に\ruby{夕陽}{せき|やう}は\ruby{既}{すで}に\ruby{没}{お}つ
        
        \vspace{\linespace}
        \item~\\
        % 4.
        \ruby{秋}{あき}の\ruby[g]{哀愁}{あはれ}は\ruby{旅}{たび}の\ruby{子}{こ}に\\
        ひとしほ\ruby{沁}{し}みる\ruby{夜半}{や|は}の\ruby{月}{つき}\\
        \ruby{悲恋}{ひ|れん}の\ruby[g]{苦悩}{うれひ}\ruby{胸}{むね}に\ruby{秘}{ひ}め\\
        \ruby{北斗}{ほく|と}の\ruby[g]{光影}{かげ}に\ruby{嘯}{うそぶ}けば\\
        \ruby{若}{わか}き\ruby[g]{情熱}{ちしほ}の\ruby{高鳴}{たか|な}りて\\
        \ruby{凋落}{てう|らく}の\ruby{世}{よ}に\ruby{響}{ひび}くなり
        
    \end{minipage}
    \begin{minipage}[c]{\blocksize}
        
        \vspace{\linespace}
        \item~\\
        % 5.
        \ruby{狂}{くる}ふ\ruby[g]{吹雪}{ふぶき}に\ruby{我}{わ}が\ruby[g]{思索}{おもひ}\\
        \ruby{託}{たく}して\ruby{進}{すす}む\ruby{三百}{さん|びゃく}の\\
        \ruby{児等}{こ|ら}の\ruby[g]{生命}{いのち}はみはるかす\\
        \ruby{北溟}{ほく|めい}の\ruby[g]{曠野}{の}にこだまして\\
        \ruby{東}{ひがし}の\ruby{空}{そら}は\ruby[g]{暁紅}{あけ}に\ruby{染}{し}み\\
        \ruby{高}{たか}き\ruby{理想}{り|さう}の\ruby[g]{旭日}{ひ}は\ruby{出}{い}でぬ
        
        \vspace{\linespace}
        \item~\\
        % 6.
        \ruby{楡}{エルム}の\ruby[g]{鐘声}{かね}に\ruby{逝}{ゆ}く\ruby[g]{青春}{はる}の\\
        \ruby[g]{神秘}{くしび}を\ruby{解}{と}かん\ruby{花}{はな}\ruby{莚}{むしろ}\\
        \ruby{朝}{あした}はろけき\ruby{旅}{たび}を\ruby{行}{ゆ}く\\
        \ruby{郭公}{くわく|こう}\ruby{鳥}{どり}よ\ruby[g]{永遠}{とこしへ}に\\
        \ruby[g]{黒百合}{くろゆり}\ruby{咲}{さ}ける\ruby{石狩}{いし|かり}の\\
        \ruby{汝}{な}が\ruby[g]{故郷}{ふるさと}を\ruby{憶}{おぼ}えよや
    
    \end{minipage}
\end{enumerate} % 番号の箇条書き ここまで
%%%%% 歌詞 ここまで %%%%%
% end body

\end{document}

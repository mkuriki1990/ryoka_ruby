\documentclass[10pt,b5j]{tarticle} % B6 縦書き
% \documentclass[10pt,b5j]{tarticle} % B6 縦書き
\AtBeginDvi{\special{papersize=128mm,182mm}} % B6 用用紙サイズ
\usepackage{otf} % Unicode で字を入力するのに必要なパッケージ
\usepackage[size=b6j]{bxpapersize} % B6 用紙サイズを指定
\usepackage[dvipdfmx]{graphicx} % 画像を挿入するためのパッケージ
\usepackage[dvipdfmx]{color} % 色をつけるためのパッケージ
\usepackage{pxrubrica} % ルビを振るためのパッケージ
\usepackage{plext} % 漢数字の enumerate を使うためのパッケージ
\usepackage{multicol} % 複数段組を作るためのパッケージ
\setlength{\topmargin}{14mm} % 上下方向のマージン
\addtolength{\topmargin}{-1in} % 
\setlength{\oddsidemargin}{11mm} % 左右方向のマージン
\addtolength{\oddsidemargin}{-1in} % 
\setlength{\textwidth}{154mm} % B6 用
\setlength{\textheight}{108mm} % B6 用
\setlength{\headsep}{0mm} % 
\setlength{\headheight}{0mm} % 
\setlength{\topskip}{0mm} % 
\setlength{\parskip}{0pt} % 
\def\theenumi{\Kanji{enumi}} % 箇条書きのフォーマットを漢数字に変更
\parindent = 0pt % 段落下げしない
\pagestyle{empty} % すべてのページ番号を消去
% \renewcommand{\baselinestretch}{0.9} % 行間の倍率
 % B6 用テンプレート読み込み

\renewcommand{\baselinestretch}{0.90} % 行間の倍率

\begin{document}
% begin header
%%%%% タイトルと作者 ここから %%%%%
\begin{minipage}[c]{0.7\hsize} % タイトルは上から 7 割
    \begin{center}
    % begin title
        {\LARGE
            魂の故郷 % タイトルを入れる
        }
        {\small 
            (昭和十二年寮歌) % 年などを入れる
        }
    % end title
    \end{center}
\end{minipage}
\begin{minipage}[c]{0.3\hsize} % 作歌作曲は上から 3 割
    \begin{flushright} % 下寄せにする
        % begin name
        山崎善陽君 作歌\\平城鷹雄君 作曲 % 作歌・作曲者
        % end name
    \end{flushright}
\end{minipage}
%%%%% タイトルと作者 ここまで %%%%%
% (1,5 繰り返しなし)
% end header

% begin length
\vspace{0.5em} % タイトル, 作者と歌詞の間に隙間を設ける
\newcommand{\linespace}{0.3em} % 行間の設定
\newcommand{\blocksize}{0.33\hsize} % 段組間の設定
\newcommand{\itemmargin}{3em} % 曲番の位置調整の長さ
% end length
% begin body
%%%%% 歌詞 ここから %%%%%
\begin{enumerate} % 番号の箇条書き ここから
    \setlength{\itemindent}{\itemmargin} % 曲番の位置調整
    \begin{minipage}[c]{\blocksize}
    
        \vspace{\linespace}
        \item~\\
        % 1.
        \ruby{魂}{たましい}の\ruby[g]{故郷}{ふるさと}に\ruby{立}{た}つ\\
        \ruby{星}{ほし}\ruby{清}{きよ}き\ruby{楡}{エルム}の\ruby{園}{その}よ\\
        \ruby{花芳}{はな|かほ}る\ruby{三春}{さん|しゅん}の\ruby{夢}{ゆめ}\\
        \ruby{感激}{かん|げき}の\ruby{涙}{なみだ}あふれて\\
        \ruby[g]{原始林}{もり}\ruby{蔭}{かげ}に\ruby{盃}{さかづき}かはす\\
        \ruby[g]{青春}{わか}き\ruby{日}{ひ}の\ruby[g]{記念}{かたみ}の\ruby{宴}{うたげ}\\
        \ruby{歌}{うた}ふなり\\
        \ruby{自治}{じ|ち}と\ruby{自由}{じ|ゆう}の\ruby{高}{たか}き\ruby{誇}{ほこり}を
        
        \vspace{\linespace}
        \item~\\
        % 2.
        \ruby[g]{六十年}{むそとせ}の\ruby{青史}{せい|し}は\ruby{薫}{かを}り\\
        \ruby{郭公}{かっ|こう}の\ruby[g]{啼声}{こえ}もはるかに\\
        \ruby{紺青}{こん|じょう}の\ruby{入相}{いり|あひ}の\ruby{空}{そら}\\
        \ruby{魂}{たましい}は\ruby{虚空}{こ|くう}に\ruby{走}{は}せて\\
        \ruby[g]{住昔}{ふるきよ}の\ruby[g]{意気}{こころ}を\ruby{慕}{した}ふ\\
        \ruby{尽}{つ}きるなき\ruby{川}{かわ}のせせらぎ\\
        \ruby{夢}{ゆめ}ふかし\\
        \ruby{残春}{ざん|しゅん}あはきポプラ\ruby{並木}{なみ|き}よ
        
    \end{minipage}
    \begin{minipage}[c]{\blocksize}
        
        \vspace{\linespace}
        \item~\\
        % 3.
        いで\ruby{湯}{ゆ}\ruby{湧}{わ}く\ruby{郷}{さと}の\ruby{宴}{うたげ}は\\
        \ruby{夜}{よ}もすがら\ruby[g]{感激}{なみだ}はてなき\\
        \ruby{絢爛}{けん|らん}たる\ruby[g]{瞬間}{ひととき}の\ruby{夢}{ゆめ}\\
        \ruby[g]{落葉松}{からまつ}の\ruby{林}{はやし}\ruby[g]{時雨}{しぐ}れて\\
        \ruby{{\CID{18947}}々}{しゅう|しゅう}の\ruby{悲歌}{ひ|か}の\ruby{調}{しら}べは\\ % CID 颼
        \ruby[g]{楡鐘}{かね}の\ruby{響}{ね}と\ruby{闇}{やみ}にきえゆく\\
        さびしらに\\
        \ruby{秋}{あき}\ruby{深}{ふか}みゆく\ruby[g]{静寂}{しじま}の\ruby{都}{みやこ}
        
        \vspace{\linespace}
        \item~\\
        % 4.
        \ruby{{\CID{18945}}々}{へう|へう}の\ruby[g]{暴風}{あらし}おさまり\\ % CID 颷
        \ruby[g]{際涯}{はてし}なき\ruby{雪}{ゆき}の\ruby{荒野}{こう|や}に\\
        \ruby{皎々}{こう|こう}と\ruby[g]{月光}{つきかげ}\ruby{冴}{さ}ゆる\\
        \ruby{橇}{そり}の\ruby{音}{ね}の\ruby[g]{玻窓}{まど}にこほりて\\
        \ruby{限}{かぎ}りなき\ruby[g]{瞑想}{おもひ}をさそふ\\
        \ruby{悠久}{ゆう|きゅう}の\ruby{時}{とき}の\ruby[g]{流転}{うつろひ}\\
        \ruby{人}{ひと}の\ruby{世}{よ}の\\
        \ruby{悲}{かな}しき\ruby[g]{運命}{さだめ}ぞ\ruby[g]{明日}{あす}の\ruby{旅路}{たび|じ}は
        
    \end{minipage}
    \begin{minipage}[c]{\blocksize}
        
        \vspace{\linespace}
        \item~\\
        % 5.
        \ruby[g]{曠野}{の}に\ruby[g]{高嘯}{うた}ふ\ruby{恵迪}{けい|てき}の\ruby{健児}{こ|ら}\\
        \ruby{毅然}{き|ぜん}たり\ruby{若}{わか}き\ruby[g]{生命}{いのち}よ\\
        \ruby{先人}{せん|じん}の\ruby{崇}{たか}き\ruby[g]{訓戒}{おしへ}に\\
        \ruby{大}{おほ}いなる\ruby[g]{野心}{こころ}\ruby{育}{はぐく}む\\
        \ruby{慨世}{がい|せい}の\ruby{憂}{うれひ}はあれど\\
        ここ\ruby{暫}{しば}し\ruby[g]{休息}{いこひ}もとめて\\
        いざ\ruby[g]{寮友}{とも}よ\\
        のこりの\ruby{春}{はる}を\ruby{惜}{お}しまざらめや
    
    \end{minipage}
\end{enumerate} % 番号の箇条書き ここまで
%%%%% 歌詞 ここまで %%%%%
% end body

\end{document}

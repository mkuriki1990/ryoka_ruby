\documentclass[10pt,b5j]{tarticle} % B6 縦書き
% \documentclass[10pt,b5j]{tarticle} % B6 縦書き
\AtBeginDvi{\special{papersize=128mm,182mm}} % B6 用用紙サイズ
\usepackage{otf} % Unicode で字を入力するのに必要なパッケージ
\usepackage[size=b6j]{bxpapersize} % B6 用紙サイズを指定
\usepackage[dvipdfmx]{graphicx} % 画像を挿入するためのパッケージ
\usepackage[dvipdfmx]{color} % 色をつけるためのパッケージ
\usepackage{pxrubrica} % ルビを振るためのパッケージ
\usepackage{multicol} % 複数段組を作るためのパッケージ
\setlength{\topmargin}{14mm} % 上下方向のマージン
\addtolength{\topmargin}{-1in} % 
\setlength{\oddsidemargin}{11mm} % 左右方向のマージン
\addtolength{\oddsidemargin}{-1in} % 
\setlength{\textwidth}{154mm} % B6 用
\setlength{\textheight}{108mm} % B6 用
\setlength{\headsep}{0mm} % 
\setlength{\headheight}{0mm} % 
\setlength{\topskip}{0mm} % 
\setlength{\parskip}{0pt} % 
\def\labelenumi{\theenumi、} % 箇条書きのフォーマット
\parindent = 0pt % 段落下げしない

 % B6 用テンプレート読み込み

\begin{document}
% begin header
%%%%% タイトルと作者 ここから %%%%%
\begin{minipage}[c]{0.7\hsize} % タイトルは上から 7 割
    \begin{center}
    % begin title
        {\LARGE
            春来にけらし % タイトルを入れる
        }
        {\small 
            (昭和十七年寮歌) % 年などを入れる
        }
    % end title
    \end{center}
\end{minipage}
\begin{minipage}[c]{0.3\hsize} % 作歌作曲は上から 3 割
    \begin{flushright} % 下寄せにする
        % begin name
        橋爪秀雄君 作歌\\杢子一雄君 作曲 % 作歌・作曲者
        % end name
    \end{flushright}
\end{minipage}
%%%%% タイトルと作者 ここまで %%%%%
% (1,2,3,4 繰り返しなし)
% end header

% begin length
\vspace{1.0em} % タイトル, 作者と歌詞の間に隙間を設ける
\newcommand{\linespace}{0.3em} % 行間の設定
\newcommand{\blocksize}{0.5\hsize} % 段組間の設定
\newcommand{\itemmargin}{3em} % 曲番の位置調整の長さ
% end length
% begin body
%%%%% 歌詞 ここから %%%%%
\begin{enumerate} % 番号の箇条書き ここから
    \setlength{\itemindent}{\itemmargin} % 曲番の位置調整
    \begin{minipage}[c]{\blocksize}
    
        \vspace{\linespace}
        \item~\\
        % 1.
        \ruby{春来}{はる|き}にけらし\ruby{白雪}{しら|ゆき}の\\
        \ruby{厚}{あつ}き\ruby{衣}{ころも}や\ruby{重}{おも}からん\\
        \ruby{綾羅}{りょう|ら}の\ruby{糸}{いと}も\ruby{綻}{ほこ}ろびて\\
        \ruby{朧々}{ろう|ろう}\ruby{深}{ふか}き\ruby[g]{五月}{さつき}\ruby{闇}{やみ}\\
        \ruby{楡影}{ゆ|えい}\ruby{揺}{ゆら}めく\ruby[g]{{\CID{15373}}鼓}{ここ}の\ruby{音}{ね}に\\ % CID 鼙
        \ruby{夜霧}{よ|ぎり}に\ruby{蒸}{む}せる\ruby{緑酒}{りょく|しゅ}\ruby{汲}{く}み\\
        \ruby{挙}{こぞ}りて\ruby{踊}{おど}る\ruby{楡}{にれ}の\ruby{精}{せい}
        
        \vspace{\linespace}
        \item~\\
        % 2.
        \ruby{草茅}{さう|ばう}しげき\ruby[g]{原始林}{もり}かげに\\
        \ruby{聖}{きよ}き\ruby[g]{{\CID{7644}}}{ほのほ}を\ruby{囲}{かこ}みつつ\\ % CID 焰
        \ruby{若}{わか}き\ruby[g]{情熱}{こころ}は\ruby{求}{もと}むれど\\
        \ruby{人生}{じん|せい}\ruby{誰}{たれ}かよく\ruby{解}{と}かん\\
        ただ\ruby{真}{まこと}なる\ruby{愛}{あい}に\ruby{泣}{な}く\\
        \ruby[g]{寮友}{とも}の\ruby{姿}{すがた}の\ruby{清}{きよ}ければ\\
        \ruby[g]{春{\CID{13831}}}{しゅんせう}の\ruby{罪}{つみ}と\ruby{誰}{たれ}か\ruby{言}{い}ふ % CID 宵󠄁
        
    \end{minipage}
    \begin{minipage}[c]{\blocksize}
        
        \vspace{\linespace}
        \item~\\
        % 3.
        \ruby{春秋}{はる|あき}\ruby{糸}{いと}も\ruby{限}{かぎ}りなく\\
        \ruby[g]{文月}{ふづき}の\ruby{夢}{ゆめ}は\ruby[g]{織女星}{おりひめ}の\\
        あはれ\ruby{手稲}{て|いね}の\ruby{衣}{ころも}かな\\
        \ruby{山}{やま}の\ruby{端}{は}\ruby{深}{ふか}くたそがれて\\
        \ruby[g]{今{\CID{13831}}}{こよひ}\ruby{銀河}{ぎん|が}の\ruby{祭日}{まつり|び}の\\ % CID 宵󠄁
        \ruby{永劫}{えい|ごふ}の\ruby{空}{そら}を\ruby{眺}{なが}むれば\\
        \ruby{天空}{てん|くう}\ruby{流}{なが}る\ruby{星}{ほし}\ruby{一}{ひと}つ
        
        \vspace{\linespace}
        \item~\\
        % 4.
        \ruby{雨月}{う|げつ}の\ruby[g]{濁流}{ながれ}\ruby{滔々}{たう|たう}と\\
        \ruby{豊川}{ほう|せん}に\ruby{聞}{き}く\ruby{世}{よ}の\ruby{憂}{うれひ}\\
        \ruby{泥潦}{でい|らう}\ruby{沈}{しず}み\ruby{真清水}{ま|し|みづ}の\\
        \ruby{流}{なが}るる\ruby{秋}{とき}は\ruby{見}{み}ざるとも\\
        \ruby{墳墓}{ふん|ぼ}の\ruby{土}{つち}を\ruby{清}{きよ}くせん\\
        \ruby{戦}{いくさ}の\ruby{庭}{には}を\ruby{高}{たか}らかに\\
        \ruby{七}{なな}つの\ruby{海}{うみ}の\ruby{潮音}{てう|おん}よ
    
    \end{minipage}
\end{enumerate} % 番号の箇条書き ここまで
%%%%% 歌詞 ここまで %%%%%
% end body

\end{document}

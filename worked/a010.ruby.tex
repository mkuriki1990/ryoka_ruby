\documentclass[10pt,b5j]{tarticle} % B6 縦書き
% \documentclass[10pt,b5j]{tarticle} % B6 縦書き
\AtBeginDvi{\special{papersize=128mm,182mm}} % B6 用用紙サイズ
\usepackage{otf} % Unicode で字を入力するのに必要なパッケージ
\usepackage[size=b6j]{bxpapersize} % B6 用紙サイズを指定
\usepackage[dvipdfmx]{graphicx} % 画像を挿入するためのパッケージ
\usepackage[dvipdfmx]{color} % 色をつけるためのパッケージ
\usepackage{pxrubrica} % ルビを振るためのパッケージ
\usepackage{multicol} % 複数段組を作るためのパッケージ
\setlength{\topmargin}{14mm} % 上下方向のマージン
\addtolength{\topmargin}{-1in} % 
\setlength{\oddsidemargin}{11mm} % 左右方向のマージン
\addtolength{\oddsidemargin}{-1in} % 
\setlength{\textwidth}{154mm} % B6 用
\setlength{\textheight}{108mm} % B6 用
\setlength{\headsep}{0mm} % 
\setlength{\headheight}{0mm} % 
\setlength{\topskip}{0mm} % 
\setlength{\parskip}{0pt} % 
\def\labelenumi{\theenumi、} % 箇条書きのフォーマット
\parindent = 0pt % 段落下げしない

 % B6 用テンプレート読み込み

\begin{document}
% begin header
%%%%% タイトルと作者 ここから %%%%%
\begin{minipage}[c]{0.7\hsize} % タイトルは上から 7 割
    \begin{center}
    % begin title
        {\LARGE
            ストームの歌 % タイトルを入れる
        }
        {\small 
             % 年などを入れる
        }
    % end title
    \end{center}
\end{minipage}
\begin{minipage}[c]{0.3\hsize} % 作歌作曲は上から 3 割
    \begin{flushright} % 下寄せにする
        % begin name
         % 作歌・作曲者
        % end name
    \end{flushright}
\end{minipage}
%%%%% タイトルと作者 ここまで %%%%%
% % end header

% begin length
\vspace{1.5em} % タイトル, 作者と歌詞の間に隙間を設ける
\newcommand{\linespace}{0.5em} % 行間の設定
\newcommand{\blocksize}{0.5\hsize} % 段組間の設定
\newcommand{\itemmargin}{3em} % 曲番の位置調整の長さ
% end length
% begin body
%%%%% 歌詞 ここから %%%%%
\begin{minipage}[c]{\blocksize}
    \begin{center}
        
        ‐\ruby{醒}{さ}めよ\ruby{迷}{まよ}ひの\ruby{夢}{ゆめ}\ruby{醒}{さ}めよ\\
        \ruby{醒}{さ}めよ\ruby{迷}{まよ}ひの\ruby{夢}{ゆめ}\ruby{醒}{さ}めよ‐
        
    \end{center}
\end{minipage}
\begin{enumerate} % 番号の箇条書き ここから
    \setlength{\itemindent}{\itemmargin} % 曲番の位置調整
    \begin{minipage}[c]{\blocksize}
        
        \vspace{\linespace}
        \item~\\
        % 1.
        \ruby{札幌}{さっ|ぽろ}\ruby{農学校}{のう|がっ|こう}は\ruby{蝦夷ヶ島}{え|ぞ|が|しま}~
        \ruby{熊}{くま}が\ruby{棲}{す}む\\
        \ruby{荒野}{あれ|の}に\ruby{建}{た}てたる\ruby{大}{だい}\ruby{校舎}{こう|しゃ}~コチャ\\
        エルムの\ruby{樹影}{こ|かげ}で\ruby{真理}{しん|り}\ruby{解}{と}く~
        コチャエ コチャエ
        
        \vspace{\linespace}
        \item~\\
        % 2.
        \ruby{札幌}{さっ|ぽろ}\ruby{農学校}{のう|がっ|こう}は\ruby{蝦夷ヶ島}{え|ぞ|が|しま}~
        \ruby{手稲山}{て|いね|やま}\\
        \ruby{夕焼}{ゆう|や}け\ruby{小焼}{こ|や}けのするところ~コチャ\\
        \ruby{牧草}{ぼく|そう}\ruby{方}{かた}\ruby{敷}{し}き\ruby{詩集}{し|しゅう}\ruby{読}{よ}む~
        コチャエ コチャエ
        
    % \end{minipage}
    % \begin{minipage}[c]{\blocksize}
        
        \vspace{\linespace}
        \item~\\
        % 3.
        \ruby{札幌}{さっ|ぽろ}\ruby{農学校}{のう|がっ|こう}は\ruby{蝦夷ヶ島}{え|ぞ|が|しま}~
        クラーク\ruby{氏}{し}\\
        ビーアンビシァスボーイズと~コチャ\\
        \ruby{学府}{がく|ふ}の\ruby{基}{もとい}を\ruby{残}{のこ}し\ruby{行}{ゆ}く~
        コチャエ コチャエ
    \end{minipage}
\begin{minipage}[c]{\blocksize}
    \setlength{\itemindent}{0em} % 曲番の位置調整

        \item[※]
        色部米作君が、明治三十八年頃、 一番の\\
        歌詞を作った。 明治四十三年九月に\\
        加藤茂雄君が二番を、\\
        出納陽一君が三番をそれぞれ作詞した。
    \end{minipage}
\end{enumerate} % 番号の箇条書き ここまで
% \begin{minipage}[c]{\blocksize}
%     \begin{itemize}
% 
% 
%     \end{itemize}
% \end{minipage}
%%%%% 歌詞 ここまで %%%%%
% end body

\end{document}

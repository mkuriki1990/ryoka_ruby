\documentclass[10pt,b5j]{tarticle} % B6 縦書き
% \documentclass[10pt,b5j]{tarticle} % B6 縦書き
\AtBeginDvi{\special{papersize=128mm,182mm}} % B6 用用紙サイズ
\usepackage{otf} % Unicode で字を入力するのに必要なパッケージ
\usepackage[size=b6j]{bxpapersize} % B6 用紙サイズを指定
\usepackage[dvipdfmx]{graphicx} % 画像を挿入するためのパッケージ
\usepackage[dvipdfmx]{color} % 色をつけるためのパッケージ
\usepackage{pxrubrica} % ルビを振るためのパッケージ
\usepackage{plext} % 漢数字の enumerate を使うためのパッケージ
\usepackage{multicol} % 複数段組を作るためのパッケージ
\setlength{\topmargin}{14mm} % 上下方向のマージン
\addtolength{\topmargin}{-1in} % 
\setlength{\oddsidemargin}{11mm} % 左右方向のマージン
\addtolength{\oddsidemargin}{-1in} % 
\setlength{\textwidth}{154mm} % B6 用
\setlength{\textheight}{108mm} % B6 用
\setlength{\headsep}{0mm} % 
\setlength{\headheight}{0mm} % 
\setlength{\topskip}{0mm} % 
\setlength{\parskip}{0pt} % 
\def\theenumi{\Kanji{enumi}} % 箇条書きのフォーマットを漢数字に変更
\parindent = 0pt % 段落下げしない
\pagestyle{empty} % すべてのページ番号を消去
% \renewcommand{\baselinestretch}{0.9} % 行間の倍率
 % B6 用テンプレート読み込み

\begin{document}
% begin header
%%%%% タイトルと作者 ここから %%%%%
\begin{minipage}[c]{0.7\hsize} % タイトルは上から 7 割
    \begin{center}
    % begin title
        {\LARGE
            新な燈火 % タイトルを入れる
        }
        {\small 
            (昭和五十二年寮歌) % 年などを入れる
        }
    % end title
    \end{center}
\end{minipage}
\begin{minipage}[c]{0.3\hsize} % 作歌作曲は上から 3 割
    \begin{flushright} % 下寄せにする
        % begin name
        石川徹君 作歌\\元辻毅君 作曲 % 作歌・作曲者
        % end name
    \end{flushright}
\end{minipage}
%%%%% タイトルと作者 ここまで %%%%%
% (1,2,3,4 繰り返しなし)
% end header

% begin length
\vspace{1.5em} % タイトル, 作者と歌詞の間に隙間を設ける
\newcommand{\linespace}{0.5em} % 行間の設定
\newcommand{\blocksize}{0.5\hsize} % 段組間の設定
\newcommand{\itemmargin}{3em} % 曲番の位置調整の長さ
% end length
% begin body
%%%%% 歌詞 ここから %%%%%
\begin{enumerate} % 番号の箇条書き ここから
    \setlength{\itemindent}{\itemmargin} % 曲番の位置調整
    \begin{minipage}[c]{\blocksize}
    
        \vspace{\linespace}
        \item~\\
        % 1.
        \ruby{北国}{きた|ぐに}の\ruby{荒}{すさ}ぶ\ruby[g]{吹雪}{ふぶき}に\\
        \ruby{楡}{にれ}の\ruby{木}{き}の\ruby{高}{たか}く\ruby{聳}{そび}える\\
        \ruby[g]{原始林}{もり}の\ruby{中}{なか}\ruby{果}{は}てる\ruby{事}{こと}なく\\
        \ruby{雄々}{お|お}しくて\ruby{人}{ひと}の\ruby{瞼}{まぶた}に\\
        \ruby[g]{何時}{いつ}\ruby{迄}{まで}も\ruby{鮮}{あざ}やかに\ruby{刻}{きざ}む\\
        \ruby{其}{そ}の\ruby{姿}{すがた}を\ruby{恵迪寮}{けい|てき|りょう}は
        
        \vspace{\linespace}
        \item~\\
        % 2.
        \ruby{憂愁}{ゆう|しゅう}と\ruby{理想}{り|そう}を\ruby{胸}{むね}に\\
        \ruby{爽}{さわ}やかに\ruby[g]{寮友}{とも}は\ruby{去}{さ}り\ruby{行}{ゆ}く\\
        \ruby{夜}{よ}を\ruby{徹}{とお}し\ruby{未来}{み|らい}の\ruby{事}{こと}を\\
        \ruby{御互}{お|たがい}に\ruby{語}{かた}った\ruby{部屋}{へ|や}に\\
        \ruby{思出}{おもい|で}の\ruby{言葉}{こと|ば}を\ruby{残}{のこ}し\\
        \ruby{懐}{なつ}かしい\ruby{恵迪寮}{けい|てき|りょう}を
        
    \end{minipage}
    \begin{minipage}[c]{\blocksize}
        
        \vspace{\linespace}
        \item~\\
        % 3.
        \ruby{年月}{とし|つき}に\ruby{傾}{かたぶ}く\ruby{姿}{すがた}\\
        \ruby{痛}{いた}ましく\ruby{懐}{おも}いの\ruby{残}{のこ}る\\
        \ruby{部屋}{へ|や}の\ruby{壁}{かべ}\ruby{崩}{くず}れ\ruby{落}{お}ちて\\
        \ruby{昔}{むかし}から\ruby{点}{とも}る\ruby{燈火}{ともし|び}\\
        \ruby{今}{いま}はもう\ruby{細}{ほそ}くなり\ruby{行}{ゆ}く\\
        \ruby{我々}{われ|われ}の\ruby{恵迪寮}{けい|てき|りょう}の
        
        \vspace{\linespace}
        \item~\\
        % 4.
        \ruby{先人}{せん|じん}の\ruby{残}{のこ}した\ruby{燈火}{ともし|び}\\
        \ruby{心}{こころ}\ruby{有}{あ}る\ruby[g]{寮友}{とも}よ\ruby{絶}{た}やさず\\
        \ruby{思}{おも}い\ruby{見}{み}て\ruby{新}{あらた}な\ruby{燈火}{ともし|び}\\
        \ruby{今}{いま}こそ\ruby{探}{さが}し\ruby{求}{もと}めて\\
        \ruby{点}{とも}そう\ruby{絶}{た}やす\ruby{事}{こと}なく\\
        \ruby[g]{何時}{いつ}\ruby{迄}{まで}も\ruby{恵迪寮}{けい|てき|りょう}に
    
    \end{minipage}
\end{enumerate} % 番号の箇条書き ここまで
%%%%% 歌詞 ここまで %%%%%
% end body

\end{document}

\documentclass[10pt,b5j]{tarticle} % B6 縦書き
% \documentclass[10pt,b5j]{tarticle} % B6 縦書き
\AtBeginDvi{\special{papersize=128mm,182mm}} % B6 用用紙サイズ
\usepackage{otf} % Unicode で字を入力するのに必要なパッケージ
\usepackage[size=b6j]{bxpapersize} % B6 用紙サイズを指定
\usepackage[dvipdfmx]{graphicx} % 画像を挿入するためのパッケージ
\usepackage[dvipdfmx]{color} % 色をつけるためのパッケージ
\usepackage{pxrubrica} % ルビを振るためのパッケージ
\usepackage{multicol} % 複数段組を作るためのパッケージ
\setlength{\topmargin}{14mm} % 上下方向のマージン
\addtolength{\topmargin}{-1in} % 
\setlength{\oddsidemargin}{11mm} % 左右方向のマージン
\addtolength{\oddsidemargin}{-1in} % 
\setlength{\textwidth}{154mm} % B6 用
\setlength{\textheight}{108mm} % B6 用
\setlength{\headsep}{0mm} % 
\setlength{\headheight}{0mm} % 
\setlength{\topskip}{0mm} % 
\setlength{\parskip}{0pt} % 
\def\labelenumi{\theenumi、} % 箇条書きのフォーマット
\parindent = 0pt % 段落下げしない

 % B6 用テンプレート読み込み

\begin{document}
% begin header
%%%%% タイトルと作者 ここから %%%%%
\begin{minipage}[c]{0.7\hsize} % タイトルは上から 7 割
    \begin{center}
    % begin title
        {\LARGE
            津軽の海 % タイトルを入れる
        }
        {\small 
            (昭和九年寮歌) % 年などを入れる
        }
    % end title
    \end{center}
\end{minipage}
\begin{minipage}[c]{0.3\hsize} % 作歌作曲は上から 3 割
    \begin{flushright} % 下寄せにする
        % begin name
        星勇君 作歌\\白石祐義君 作曲 % 作歌・作曲者
        % end name
    \end{flushright}
\end{minipage}
%%%%% タイトルと作者 ここまで %%%%%
% (1,4,7 了あり)
% end header

% begin length
\vspace{1.5em} % タイトル, 作者と歌詞の間に隙間を設ける
\newcommand{\linespace}{0.5em} % 行間の設定
\newcommand{\blocksize}{0.25\hsize} % 段組間の設定
\newcommand{\itemmargin}{3em} % 曲番の位置調整の長さ
% end length
% begin body
%%%%% 歌詞 ここから %%%%%
\begin{enumerate} % 番号の箇条書き ここから
    \setlength{\itemindent}{\itemmargin} % 曲番の位置調整
    \begin{minipage}[c]{\blocksize}
    
        \vspace{\linespace}
        \item~\\
        % 1.
        \ruby{津軽}{つ|がる}の\ruby{海}{み}\ruby{渦}{うず}\ruby{巻}{ま}ける\ruby{奥}{おく}\\
        オホツクの\ruby{寒潮}{かん|ちょう}\ruby[g]{咆哮}{ほ}えて\\
        \ruby[g]{雄健}{たけ}き\ruby{名}{な}ぞ\ruby{蝦夷}{え|ぞ}が\ruby{島根}{しま|ね}に\\
        \ruby{年}{とし}\ruby{古}{ふ}りし\ruby{恵迪}{けい|てき}の\ruby{寮}{りょう}\\
        \ruby{旅寝}{たび|ね}とな\ruby{言}{い}ひし\ruby[g]{三年}{みとせ}を\\
        \ruby{揺籃}{よう|らん}の\ruby[g]{高夢}{ゆめ}を\ruby{追}{お}ふなり
        
        \vspace{\linespace}
        \item~\\
        % 2.
        \ruby[g]{寂寥}{さびしら}の\ruby[g]{歩行}{あゆみ}はこびて\\
        \ruby{茂}{し}みさぶる\ruby{森}{もり}に\ruby{仰臥}{ぎょう|が}し\\
        \ruby{先人}{せん|じん}の\ruby{詩}{ふみ}になぞらへ\\
        \ruby{陳腐}{ちん|ぷ}なる\ruby{歌}{うた}を\ruby{恥}{は}ぢらふ\\
        ただ\ruby{仰}{あお}げ\ruby{自然}{し|ぜん}の\ruby{姿}{すがた}\\
        そは\ruby{深}{ふか}き\ruby[g]{黙示}{さとし}をきざむ
        
    \end{minipage}
    \begin{minipage}[c]{\blocksize}
        
        \vspace{\linespace}
        \item~\\
        % 3.
        \ruby{清明}{せい|めい}の\ruby{水}{みず}に\ruby{浮}{うか}べる\\
        \ruby[g]{宵󠄁月}{よいづき}の\ruby{影}{かげ}はさやけし\\
        \ruby{酒觴}{さか|づき}をめぐらしかさね\\
        \ruby[g]{羆熊}{くま}の\ruby{声}{こえ}\ruby{聞}{き}くもすべなし\\
        たぎりゆく\ruby{若}{わか}き\ruby{血潮}{ち|しお}に\\
        \ruby{限}{かぎ}りなき\ruby[g]{感激}{おもひ}をしたふ
        
        \vspace{\linespace}
        \item~\\
        % 4.
        \ruby[g]{六十}{むそじ}にも\ruby{齢}{よわい}うつろひ\\
        \ruby{集}{つど}ひたる\ruby[g]{寮友}{とも}は\ruby[g]{兄弟}{はらから}\\
        \ruby{伝統}{でん|とう}の\ruby[g]{永遠}{とは}の\ruby[g]{記念}{かたみ}と\\
        \ruby{感激}{かん|げき}の\ruby{寮史}{りょう|し}も\ruby{成}{な}りぬ\\
        \ruby[g]{情懐}{むね}\ruby{深}{ふか}く\ruby{唯}{ただ}\ruby{魂}{たましい}が\\
        \ruby{魂}{たましい}と\ruby{結}{むす}び\ruby{輝}{かがや}く
        
    \end{minipage}
    \begin{minipage}[c]{\blocksize}
        
        \vspace{\linespace}
        \item~\\
        % 5.
        \ruby{恵迪}{けい|てき}の\ruby{館}{やかた}を\ruby{訪}{と}ひし\\
        \ruby{竜田姫}{たつ|た|ひめ}\ruby{佐保神}{さ|ほ|がみ}\ruby{三}{み}たび\\
        \ruby[g]{若人}{わこうど}の\ruby[g]{生命}{いのち}\ruby{捧}{ささ}げし\\
        \ruby{想}{おも}ひ\ruby{出}{で}の\ruby{自由}{じ|ゆう}の\ruby[g]{宴遊}{うたげ}\\
        \ruby{永劫}{えい|ごう}に\ruby{若}{わか}き\ruby{一日}{ひと|ひ}の\\
        \ruby{夢}{ゆめ}とせむ\ruby[g]{楡鐘}{かね}の\ruby{調}{しら}べを
        
        \vspace{\linespace}
        \item~\\
        % 6.
        \ruby{黎明}{れい|めい}は\ruby{曠野}{こう|や}の\ruby[g]{際涯}{みきり}\\
        \ruby{雄叫}{お|たけ}びと\ruby{共}{とも}に\ruby{来}{きた}れり\\
        \ruby{満蒙}{まん|もう}の\ruby{長夜}{なが|よ}の\ruby{闇}{やみ}も\\
        \ruby{晴}{は}れんとす\ruby{起}{た}てよ\ruby[g]{寮友}{ともどち}\\
        \ruby{青春}{せい|しゅん}の\ruby{象牙}{ぞう|げ}の\ruby{塔}{とう}を\\
        いざ\ruby{出}{い}でむ\ruby{時}{とき}は\ruby{到}{いた}れり
        
    \end{minipage}
    \begin{minipage}[c]{\blocksize}
        
        \vspace{\linespace}
        \item~\\
        % 7.
        \ruby{北溟}{ほく|めい}の\ruby{自治}{じ|ち}の\ruby{牙城}{が|じょう}を\\
        \ruby[g]{蒼穹}{そら}\ruby{高}{たか}く\ruby{巣立}{す|だ}つ\ruby[g]{寮友}{ともひと}\\
        \ruby{澆季}{ぎょう|き}の\ruby{世}{よ}\ruby{救}{すく}はんは\ruby{汝}{な}れ\\
        \ruby{済世}{さい|せい}の\ruby[g]{烽火}{のろし}あぐべし\\
        \ruby{忘}{わす}れ\ruby{得}{え}ぬ\ruby{恵迪}{けい|てき}の\ruby{歌}{うた}\\
        \ruby[g]{高唱}{うた}ひゆけ\ruby{正義}{せい|ぎ}の\ruby[g]{大道}{みち}を
    
    \end{minipage}
\end{enumerate} % 番号の箇条書き ここまで
%%%%% 歌詞 ここまで %%%%%
% end body

\end{document}

\documentclass[10pt,b5j]{tarticle} % B6 縦書き
% \documentclass[10pt,b5j]{tarticle} % B6 縦書き
\AtBeginDvi{\special{papersize=128mm,182mm}} % B6 用用紙サイズ
\usepackage{otf} % Unicode で字を入力するのに必要なパッケージ
\usepackage[size=b6j]{bxpapersize} % B6 用紙サイズを指定
\usepackage[dvipdfmx]{graphicx} % 画像を挿入するためのパッケージ
\usepackage[dvipdfmx]{color} % 色をつけるためのパッケージ
\usepackage{pxrubrica} % ルビを振るためのパッケージ
\usepackage{multicol} % 複数段組を作るためのパッケージ
\setlength{\topmargin}{14mm} % 上下方向のマージン
\addtolength{\topmargin}{-1in} % 
\setlength{\oddsidemargin}{11mm} % 左右方向のマージン
\addtolength{\oddsidemargin}{-1in} % 
\setlength{\textwidth}{154mm} % B6 用
\setlength{\textheight}{108mm} % B6 用
\setlength{\headsep}{0mm} % 
\setlength{\headheight}{0mm} % 
\setlength{\topskip}{0mm} % 
\setlength{\parskip}{0pt} % 
\def\labelenumi{\theenumi、} % 箇条書きのフォーマット
\parindent = 0pt % 段落下げしない

 % B6 用テンプレート読み込み

\begin{document}
% begin header
%%%%% タイトルと作者 ここから %%%%%
\begin{minipage}[c]{0.7\hsize} % タイトルは上から 7 割
    \begin{center}
    % begin title
        {\LARGE
            雪の白さに % タイトルを入れる
        }
        {\small 
            (昭和五十九年寮歌) % 年などを入れる
        }
    % end title
    \end{center}
\end{minipage}
\begin{minipage}[c]{0.3\hsize} % 作歌作曲は上から 3 割
    \begin{flushright} % 下寄せにする
        % begin name
        濱田和雄君 作歌\\青木毅君 作曲 % 作歌・作曲者
        % end name
    \end{flushright}
\end{minipage}
%%%%% タイトルと作者 ここまで %%%%%
% (1,2,3,4 了あり)
% end header

% begin length
\vspace{1.5em} % タイトル, 作者と歌詞の間に隙間を設ける
\newcommand{\linespace}{0.5em} % 行間の設定
\newcommand{\blocksize}{0.5\hsize} % 段組間の設定
\newcommand{\itemmargin}{3em} % 曲番の位置調整の長さ
% end length
% begin body
%%%%% 歌詞 ここから %%%%%
\begin{enumerate} % 番号の箇条書き ここから
    \setlength{\itemindent}{\itemmargin} % 曲番の位置調整
    \begin{minipage}[c]{\blocksize}
    
        \vspace{\linespace}
        \item~\\
        % 1.
        \ruby{雪}{ゆき}の\ruby{白}{しろ}さに\ruby{映}{は}ゆる\ruby{我等}{われ|ら}が\ruby{恵迪寮}{けい|てき|りょう}\\
        \ruby[g]{吹雪}{ふぶき}\ruby{逆巻}{さか|ま}く\ruby{日}{ひ}もあれど\\
        \ruby{正義}{せい|ぎ}の\ruby{迪}{みち}を\ruby{見定}{み|さだ}めて\\
        \ruby[g]{真実}{まこと}\ruby{求}{もと}むは\ruby{風}{かぜ}の\ruby{教}{おし}へなり
        
        \vspace{\linespace}
        \item~\\
        % 2.
        \ruby{土}{つち}の\ruby{黒}{くろ}さに\ruby{萌}{も}ゆる\ruby{新}{あら}たな\ruby{芽}{め}が\ruby{一}{ひと}つ\\
        \ruby{雨風}{あめ|かぜ}\ruby{寒}{さむ}さに\ruby{怯}{おび}ゆるとも\\
        \ruby{宴}{うたげ}\ruby{討論}{とう|ろん}\ruby{酔}{よ}ひしれて\\
        \ruby[g]{恵迪}{りょう}に\ruby{根}{ね}づくは\ruby{土}{つち}の\ruby{教}{おし}へなり
        
    \end{minipage}
    \begin{minipage}[c]{\blocksize}
        
        \vspace{\linespace}
        \item~\\
        % 3.
        \ruby{空}{そら}の\ruby{青}{あお}さに\ruby{育}{そだ}つみんなの\ruby{自治}{じ|ち}\ruby{意識}{い|しき}\\
        \ruby{熱風}{ねっ|ぷう}\ruby[g]{日干}{ひでり}の\ruby{害}{がい}あれど\\
        \ruby{理想}{り|そう}\ruby{高}{たか}く\ruby{足}{あし}は\ruby[g]{大地}{ち}につきて\\
        \ruby{汗}{あせ}を\ruby{流}{なが}すは\ruby{陽}{ひ}の\ruby{教}{おし}へなり
        
        \vspace{\linespace}
        \item~\\
        % 4.
        \ruby{秋}{あき}の\ruby[g]{疾風}{はやて}に\ruby{聳}{そび}ゆ\ruby{大}{おお}きな\ruby{林檎}{りん|ご}の\ruby{木}{き}\\
        \ruby[g]{頂上}{いただき}の\ruby{実}{み}が\ruby{墜}{お}つるとも\\
        その\ruby[g]{精神}{こころ}もて\ruby{糧}{かて}として\\
        \ruby{自律}{じ|りつ}\ruby{目指}{め|ざ}すは\ruby[g]{生命}{いのち}の\ruby{教}{おし}へなり
    
    \end{minipage}
\end{enumerate} % 番号の箇条書き ここまで
%%%%% 歌詞 ここまで %%%%%
% end body

\end{document}

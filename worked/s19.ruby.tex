\documentclass[10pt,b5j]{tarticle} % B6 縦書き
% \documentclass[10pt,b5j]{tarticle} % B6 縦書き
\AtBeginDvi{\special{papersize=128mm,182mm}} % B6 用用紙サイズ
\usepackage{otf} % Unicode で字を入力するのに必要なパッケージ
\usepackage[size=b6j]{bxpapersize} % B6 用紙サイズを指定
\usepackage[dvipdfmx]{graphicx} % 画像を挿入するためのパッケージ
\usepackage[dvipdfmx]{color} % 色をつけるためのパッケージ
\usepackage{pxrubrica} % ルビを振るためのパッケージ
\usepackage{multicol} % 複数段組を作るためのパッケージ
\setlength{\topmargin}{14mm} % 上下方向のマージン
\addtolength{\topmargin}{-1in} % 
\setlength{\oddsidemargin}{11mm} % 左右方向のマージン
\addtolength{\oddsidemargin}{-1in} % 
\setlength{\textwidth}{154mm} % B6 用
\setlength{\textheight}{108mm} % B6 用
\setlength{\headsep}{0mm} % 
\setlength{\headheight}{0mm} % 
\setlength{\topskip}{0mm} % 
\setlength{\parskip}{0pt} % 
\def\labelenumi{\theenumi、} % 箇条書きのフォーマット
\parindent = 0pt % 段落下げしない

 % B6 用テンプレート読み込み

\begin{document}
% begin header
%%%%% タイトルと作者 ここから %%%%%
\begin{minipage}[c]{0.7\hsize} % タイトルは上から 7 割
    \begin{center}
    % begin title
        {\LARGE
            雪解の楡陵の % タイトルを入れる
        }
        {\small 
            (昭和十九年寮歌) % 年などを入れる
        }
    % end title
    \end{center}
\end{minipage}
\begin{minipage}[c]{0.3\hsize} % 作歌作曲は上から 3 割
    \begin{flushright} % 下寄せにする
        % begin name
        鈴木信夫君 作歌\\竹山賢治君 作曲 % 作歌・作曲者
        % end name
    \end{flushright}
\end{minipage}
%%%%% タイトルと作者 ここまで %%%%%
% (1,2,3 了あり)
% end header

% begin length
\vspace{1.5em} % タイトル, 作者と歌詞の間に隙間を設ける
\newcommand{\linespace}{0.5em} % 行間の設定
\newcommand{\blocksize}{0.33\hsize} % 段組間の設定
\newcommand{\itemmargin}{3em} % 曲番の位置調整の長さ
% end length
% begin body
%%%%% 歌詞 ここから %%%%%
\begin{enumerate} % 番号の箇条書き ここから
    \setlength{\itemindent}{\itemmargin} % 曲番の位置調整
    \begin{minipage}[c]{\blocksize}
    
        \vspace{\linespace}
        \item~\\
        % 1.
        \ruby{雪解}{ゆき|げ}の\ruby[g]{楡陵}{をか}の\ruby[g]{一流}{ひとすぢ}や\\
        \ruby{岸辺}{きし|べ}に\ruby{憩}{いこ}ふ\ruby{水鳥}{みづ|とり}の\\
        \ruby[g]{孤影}{すがた}ぞしばし\ruby{春}{はる}の\ruby[g]{水面}{みお}\\
        ああ\ruby{石狩}{いし|かり}の\ruby[g]{天空}{そら}\ruby{晴}{は}れて\\
        \ruby{轟}{とどろ}け\ruby{謳}{うた}ふ\ruby{恵迪}{けい|てき}の\\
        \ruby{児等}{こ|ら}が\ruby[g]{生命}{いのち}や\ruby{聖}{きよ}からん
        
        \vspace{\linespace}
        \item~\\
        % 2.
        \ruby{歓喜}{くわん|き}\ruby{憂苦}{いう|く}を\ruby{共}{とも}にせむ\\
        \ruby{結}{むす}ぶ\ruby{契}{ちぎり}の\ruby{盃}{さかづき}に\\
        \ruby{松}{まつ}の\ruby{枝}{え}\ruby{漏}{も}るる\ruby{月影}{つき|かげ}や\\
        \ruby{人生}{じん|せい}\ruby{意気}{い|き}に\ruby{感}{かん}じてか\\
        \ruby{集}{つど}ひし\ruby{雁}{かり}の\ruby{行}{ゆ}く\ruby{手稲}{て|いね}\\
        \ruby[g]{青雲}{せいうん}の\ruby[g]{峯}{みね}\ruby{巍峨}{ぎ|が}の\ruby{峯}{みね}
        
    \end{minipage}
    \begin{minipage}[c]{\blocksize}
        
        \vspace{\linespace}
        \item~\\
        % 3.
        いざや\ruby[g]{伝統}{つたへ}の\ruby[g]{聖火}{ひ}を\ruby{翳}{かざ}し\\
        \ruby{先人}{せん|じん}の\ruby[g]{絢夢}{ゆめ}\ruby{偲}{しの}びつつ\\
        \ruby[g]{寮祭}{まつり}の\ruby{庭}{にわ}に\ruby[g]{四十回}{よそたび}の\\
        \ruby{春風}{はる|かぜ}\ruby[g]{頬涙}{ほほ}を\ruby{乾}{ほ}すなれば\\
        \ruby{散}{ち}りゆく\ruby[g]{夜迷雲}{くも}のかげ\ruby{消}{き}えて\\
        \ruby{声}{こゑ}を\ruby{限}{かぎ}りの\ruby[g]{感激}{おもひ}かな
        
        \vspace{\linespace}
        \item~\\
        % 4.
        \ruby{南}{みなみ}の\ruby{海}{うみ}の\ruby{有明}{あり|あけ}に\\
        \ruby{燦}{きらめ}く\ruby[g]{星辰}{ほし}の\ruby{消}{き}え\ruby{果}{は}てて\\
        \ruby{散}{ち}りぬる\ruby[g]{若桜}{はな}もあるぞかし\\
        いかで\ruby{我等}{われ|ら}の\ruby[g]{蹶起}{たた}ざらん\\
        \ruby{義憤}{ぎ|ふん}が\ruby{胸}{むね}にほのぼのと\\
        \ruby{染}{そ}め\ruby{映}{は}えにしか\ruby{朝日}{あさ|ひ}\ruby{影}{かげ}
        
    \end{minipage}
    \begin{minipage}[c]{\blocksize}
        
        \vspace{\linespace}
        \item~\\
        % 5.
        \ruby{噫}{ああ}\ruby{世}{よ}は\ruby[g]{変遷}{うつ}り\ruby{人}{ひと}\ruby{変}{かは}り\\
        \ruby{舘}{やかた}の\ruby[g]{原始林}{もり}は\ruby{愁}{うれ}へども\\
        \ruby{剛毅}{がう|き}の\ruby{大{\CID{19431}}}{たい|はい}\ruby{仰}{あふ}ぎてし\\ % CID 斾
        \ruby{熱血}{ねっ|けつ}\ruby{燃}{も}ゆる\ruby{益良夫}{ます|ら|を}が\\
        \ruby{皇国}{み|くに}の\ruby{道}{みち}に\ruby[g]{挺身}{すす}まんと\\
        \ruby{誓}{ちか}ひし\ruby{眸}{まみ}に\ruby[g]{光輝}{ひかり}あれ
    
    \end{minipage}
\end{enumerate} % 番号の箇条書き ここまで
%%%%% 歌詞 ここまで %%%%%
% end body

\end{document}

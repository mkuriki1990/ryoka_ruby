\documentclass[10pt,b5j]{tarticle} % B6 縦書き
% \documentclass[10pt,b5j]{tarticle} % B6 縦書き
\AtBeginDvi{\special{papersize=128mm,182mm}} % B6 用用紙サイズ
\usepackage{otf} % Unicode で字を入力するのに必要なパッケージ
\usepackage[size=b6j]{bxpapersize} % B6 用紙サイズを指定
\usepackage[dvipdfmx]{graphicx} % 画像を挿入するためのパッケージ
\usepackage[dvipdfmx]{color} % 色をつけるためのパッケージ
\usepackage{pxrubrica} % ルビを振るためのパッケージ
\usepackage{multicol} % 複数段組を作るためのパッケージ
\setlength{\topmargin}{14mm} % 上下方向のマージン
\addtolength{\topmargin}{-1in} % 
\setlength{\oddsidemargin}{11mm} % 左右方向のマージン
\addtolength{\oddsidemargin}{-1in} % 
\setlength{\textwidth}{154mm} % B6 用
\setlength{\textheight}{108mm} % B6 用
\setlength{\headsep}{0mm} % 
\setlength{\headheight}{0mm} % 
\setlength{\topskip}{0mm} % 
\setlength{\parskip}{0pt} % 
\def\labelenumi{\theenumi、} % 箇条書きのフォーマット
\parindent = 0pt % 段落下げしない

 % B6 用テンプレート読み込み

\renewcommand{\baselinestretch}{0.90} % 行間の倍率

\begin{document}
% begin header
%%%%% タイトルと作者 ここから %%%%%
\begin{minipage}[c]{0.7\hsize} % タイトルは上から 7 割
    \begin{center}
    % begin title
        {\LARGE
            漕艇部部歌 % タイトルを入れる
        }
        {\large 
            ‐春三月の(茨戸の歌)‐
        }
        {\small 
            (昭和三十年) % 年などを入れる
        }
    % end title
    \end{center}
\end{minipage}
\begin{minipage}[c]{0.3\hsize} % 作歌作曲は上から 3 割
    \begin{flushright} % 下寄せにする
        % begin name
        木原慎一君 作歌・作曲 % 作歌・作曲者
        % end name
    \end{flushright}
\end{minipage}
%%%%% タイトルと作者 ここまで %%%%%
% % end header

% begin length
\vspace{0.5em} % タイトル, 作者と歌詞の間に隙間を設ける
\newcommand{\linespace}{0.3em} % 行間の設定
\newcommand{\blocksize}{0.33\hsize} % 段組間の設定
\newcommand{\itemmargin}{3em} % 曲番の位置調整の長さ
% end length
% begin body
%%%%% 歌詞 ここから %%%%%
\begin{enumerate} % 番号の箇条書き ここから
    \setlength{\itemindent}{\itemmargin} % 曲番の位置調整
    \begin{minipage}[c]{\blocksize}
    
        \vspace{\linespace}
        \item~\\
        % 1.
        \ruby{春}{はる}\ruby{三月}{さん|がつ}の\ruby[g]{蝦夷島}{えぞがしま}\\
        \ruby{長}{なが}き\ruby{眠}{ねむ}りにとざされし\\
        \ruby{茨戸}{ばら|と}\ruby{河畔}{か|はん}の\ruby{雪}{ゆき}とけて\\
        とく\ruby{待}{ま}ちわびし\ruby{水}{みず}の\ruby{子}{こ}の\\
        \ruby{喜}{よろこ}び\ruby{笑}{わら}ふ\ruby{声}{こえ}すなり
        
        \vspace{\linespace}
        \item~\\
        % 2.
        \ruby{岸}{きし}の\ruby{辺}{べ}\ruby{近}{ちか}く\ruby{郭公}{かっ|こう}の\\
        \ruby{啼}{な}く\ruby{音}{ね}うれしく\ruby{聞}{き}き\ruby{初}{そ}めね\\
        \ruby{漕}{こ}ぎ\ruby{来}{こ}し\ruby{方}{かた}を\ruby{眺}{なが}むれば\\
        \ruby{霞}{かすみ}にとける\ruby{野}{の}の\ruby{煙}{けむり}\\
        \ruby{水郷}{すい|ごう}の\ruby{春}{はる}の\ruby{昼}{ひる}\ruby{閑}{のど}か
        
        \vspace{\linespace}
        \item~\\
        % 3.
        \ruby[g]{岩燕}{つばめ}は\ruby{去}{さ}りて\ruby{風}{かぜ}\ruby{熱}{あつ}き\\
        \ruby{夏}{なつ}たけなはの\ruby{候}{こう}となる\\
        \ruby{運河}{うん|が}\ruby{一発}{いっ|ぱつ}\ruby{引}{ひ}き\ruby{抜}{ぬ}きて\\
        しばし\ruby{憩}{いこ}はむ\ruby{土手}{ど|て}の\ruby{上}{うえ}\\
        \ruby{羊}{ひつじ}も\ruby{寄}{よ}りて\ruby{草}{くさ}を\ruby{食}{は}む
        
    \end{minipage}
    \begin{minipage}[c]{\blocksize}
        
        \vspace{\linespace}
        \item~\\
        % 4.
        いつか\ruby{炎暑}{えん|しょ}の\ruby{日}{ひ}はゆきて\\
        \ruby{光}{ひかり}のどけき\ruby{茨戸河}{ばら|と|がわ}\\
        \ruby{青}{あお}き\ruby{水}{み}の\ruby{面}{も}に\ruby{波立}{なみ|た}たず\\
        こよなき\ruby{季節}{き|せつ}\ruby{訪}{おとず}れぬ\\
        \ruby{心}{こころ}ゆくまで\ruby{漕}{こ}がむかな
        
        \vspace{\linespace}
        \item~\\
        % 5.
        \ruby{手稲}{て|いね}は\ruby{紅}{あか}く\ruby{空高}{そら|たか}く\\
        \ruby{秋}{あき}の\ruby{気}{き}\ruby{深}{ふか}くなりにけり\\
        かい\ruby{先}{さき}\ruby{近}{ちか}くぼらはねて\\
        \ruby{夕}{ゆうべ}\ruby{練習}{れん|しゅう}\ruby{終}{お}へるころ\\
        \ruby{陽}{ひ}はくれないに\ruby{没}{ぼっ}したり
        
        \vspace{\linespace}
        \item~\\
        % 6.
        \ruby{河}{かわ}\ruby{霧}{きり}\ruby{深}{ふか}くたちこめて\\
        \ruby{霜}{しも}\ruby{結}{むす}ぶ\ruby{朝}{あさ}\ruby{艇}{ふね}\ruby{出}{いだ}す\\
        みぎわの\ruby{木々}{き|ぎ}は\ruby{枯}{か}れはてて\\
        \ruby{冬}{ふゆ}もま\ruby{近}{ぢか}となりぬれば\\
        \ruby{惜}{お}しみて\ruby{漕}{こ}がむ\ruby{残}{のこ}る\ruby{日々}{ひ|び}
        
    \end{minipage}
    \begin{minipage}[c]{\blocksize}
        
        \vspace{\linespace}
        \item~\\
        % 7.
        \ruby{北風}{きた|かぜ}すさび\ruby{雪}{ゆき}は\ruby{舞}{ま}ひ\\
        ふぶきに\ruby{暮}{く}れる\ruby{冬}{ふゆ}の\ruby{河}{かわ}\\
        \ruby[g]{今日}{きょう}ぞわれらが\ruby{漕}{こ}ぎ\ruby{納}{おさ}め\\
        いざわが\ruby{友}{とも}よ\ruby{胸}{むね}\ruby{深}{ふか}く\\
        また\ruby{来}{こ}む\ruby{年}{とし}の\ruby{幸}{さち}\ruby{思}{おも}へ
    
    \end{minipage}
\end{enumerate} % 番号の箇条書き ここまで
%%%%% 歌詞 ここまで %%%%%
% end body

\end{document}

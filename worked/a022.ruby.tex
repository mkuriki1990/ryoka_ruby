\documentclass[10pt,b5j]{tarticle} % B6 縦書き
% \documentclass[10pt,b5j]{tarticle} % B6 縦書き
\AtBeginDvi{\special{papersize=128mm,182mm}} % B6 用用紙サイズ
\usepackage{otf} % Unicode で字を入力するのに必要なパッケージ
\usepackage[size=b6j]{bxpapersize} % B6 用紙サイズを指定
\usepackage[dvipdfmx]{graphicx} % 画像を挿入するためのパッケージ
\usepackage[dvipdfmx]{color} % 色をつけるためのパッケージ
\usepackage{pxrubrica} % ルビを振るためのパッケージ
\usepackage{plext} % 漢数字の enumerate を使うためのパッケージ
\usepackage{multicol} % 複数段組を作るためのパッケージ
\setlength{\topmargin}{14mm} % 上下方向のマージン
\addtolength{\topmargin}{-1in} % 
\setlength{\oddsidemargin}{11mm} % 左右方向のマージン
\addtolength{\oddsidemargin}{-1in} % 
\setlength{\textwidth}{154mm} % B6 用
\setlength{\textheight}{108mm} % B6 用
\setlength{\headsep}{0mm} % 
\setlength{\headheight}{0mm} % 
\setlength{\topskip}{0mm} % 
\setlength{\parskip}{0pt} % 
\def\theenumi{\Kanji{enumi}} % 箇条書きのフォーマットを漢数字に変更
\parindent = 0pt % 段落下げしない
\pagestyle{empty} % すべてのページ番号を消去
% \renewcommand{\baselinestretch}{0.9} % 行間の倍率
 % B6 用テンプレート読み込み

\begin{document}
% begin header
%%%%% タイトルと作者 ここから %%%%%
\begin{minipage}[c]{0.7\hsize} % タイトルは上から 7 割
    \begin{center}
    % begin title
        {\LARGE
            ラグビー部部歌 % タイトルを入れる
        }
        {\small 
             % 年などを入れる
        }
    % end title
    \end{center}
\end{minipage}
\begin{minipage}[c]{0.3\hsize} % 作歌作曲は上から 3 割
    \begin{flushright} % 下寄せにする
        % begin name
        坂井稔君 作歌\\飯田毅君 作曲 % 作歌・作曲者
        % end name
    \end{flushright}
\end{minipage}
%%%%% タイトルと作者 ここまで %%%%%
% % end header

% begin length
\vspace{1.5em} % タイトル, 作者と歌詞の間に隙間を設ける
\newcommand{\linespace}{0.5em} % 行間の設定
\newcommand{\blocksize}{0.5\hsize} % 段組間の設定
\newcommand{\itemmargin}{3em} % 曲番の位置調整の長さ
% end length
% begin body
%%%%% 歌詞 ここから %%%%%
\begin{enumerate} % 番号の箇条書き ここから
    \setlength{\itemindent}{\itemmargin} % 曲番の位置調整
    \begin{minipage}[c]{\blocksize}
    
        \vspace{\linespace}
        \item~\\
        % 1.
        \ruby{黄塵}{こう|じん}はるか \ruby{隔}{へだ}てたる\\
        ここ\ruby{北溟}{ほく|めい}の \ruby{原始林}{げん|し|りん}\\
        \ruby{巣立}{す|だ}ちし \ruby{若}{わか}き\ruby{荒鷲}{あら|わし}の\\
        \ruby{意気}{い|き}ぞ\ruby{満}{み}てる フェアプレー\\
        \ruby{見}{み}よ \ruby{意気}{い|き}の \ruby{北大}{ほく|だい}\ruby{予科}{よ|か}ラガー\\
        ファイト ファイト\\
        ファイト ファイト ファイト\\
        オンワーズ ヴィクトリー
        
    \end{minipage}
    \begin{minipage}[c]{\blocksize}
        
        \vspace{\linespace}
        \item~\\
        % 2.
            \ruby{残陽}{ざん|よう}\ruby{西}{にし}に \ruby{茜}{あかね}して\\
        \ruby{五彩}{ご|さい}の\ruby{雲}{くも}の はゆるとき\\
        ダーク・グリーンの \ruby{若武者}{わか|む|しゃ}の\\
        \ruby{熱}{ねつ}ぞ\ruby{満}{み}てる フェアプレー\\
        \ruby{見}{み}よ \ruby{熱}{ねつ}の \ruby{北大}{ほく|だい}\ruby{予科}{よ|か}ラガー\\
        ファイト ファイト\\
        ファイト ファイト ファイト\\
        オンワーズ ヴィクトリー
    
    \end{minipage}
\end{enumerate} % 番号の箇条書き ここまで
%%%%% 歌詞 ここまで %%%%%
% end body

\end{document}

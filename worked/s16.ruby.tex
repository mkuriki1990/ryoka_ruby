\documentclass[10pt,b5j]{tarticle} % B6 縦書き
% \documentclass[10pt,b5j]{tarticle} % B6 縦書き
\AtBeginDvi{\special{papersize=128mm,182mm}} % B6 用用紙サイズ
\usepackage{otf} % Unicode で字を入力するのに必要なパッケージ
\usepackage[size=b6j]{bxpapersize} % B6 用紙サイズを指定
\usepackage[dvipdfmx]{graphicx} % 画像を挿入するためのパッケージ
\usepackage[dvipdfmx]{color} % 色をつけるためのパッケージ
\usepackage{pxrubrica} % ルビを振るためのパッケージ
\usepackage{multicol} % 複数段組を作るためのパッケージ
\setlength{\topmargin}{14mm} % 上下方向のマージン
\addtolength{\topmargin}{-1in} % 
\setlength{\oddsidemargin}{11mm} % 左右方向のマージン
\addtolength{\oddsidemargin}{-1in} % 
\setlength{\textwidth}{154mm} % B6 用
\setlength{\textheight}{108mm} % B6 用
\setlength{\headsep}{0mm} % 
\setlength{\headheight}{0mm} % 
\setlength{\topskip}{0mm} % 
\setlength{\parskip}{0pt} % 
\def\labelenumi{\theenumi、} % 箇条書きのフォーマット
\parindent = 0pt % 段落下げしない

 % B6 用テンプレート読み込み

\begin{document}
% begin header
%%%%% タイトルと作者 ここから %%%%%
\begin{minipage}[c]{0.7\hsize} % タイトルは上から 7 割
    \begin{center}
    % begin title
        {\LARGE
            湖に星の散るなり % タイトルを入れる
        }
        {\small 
            (昭和十六年寮歌) % 年などを入れる
        }
    % end title
    \end{center}
\end{minipage}
\begin{minipage}[c]{0.3\hsize} % 作歌作曲は上から 3 割
    \begin{flushright} % 下寄せにする
        % begin name
        切替辰哉君 作歌\\岡田和雄君 作曲 % 作歌・作曲者
        % end name
    \end{flushright}
\end{minipage}
%%%%% タイトルと作者 ここまで %%%%%
% (1 繰り返しなし)
% end header

% begin length
\vspace{1.5em} % タイトル, 作者と歌詞の間に隙間を設ける
\newcommand{\linespace}{0.5em} % 行間の設定
\newcommand{\blocksize}{0.5\hsize} % 段組間の設定
\newcommand{\itemmargin}{3em} % 曲番の位置調整の長さ
% end length
% begin body
%%%%% 歌詞 ここから %%%%%
\begin{enumerate} % 番号の箇条書き ここから
    \setlength{\itemindent}{\itemmargin} % 曲番の位置調整
    \begin{minipage}[c]{\blocksize}
    
        \vspace{\linespace}
        \item~\\
        % 1.
        \ruby{湖}{みずうみ}に\ruby{星}{ほし}の\ruby{散}{ち}るなり\ruby{幽}{かそ}けさよ
        \ruby{松}{まつ}の\ruby{火}{ひ}\ruby{燃}{も}えて\\
        \ruby{漕}{こ}ぎ\ruby{出}{い}づる\ruby[g]{愛奴}{あいぬ}の\ruby[g]{漁舟}{ふね}の
        \ruby{岸辺}{きし|べ}\ruby{佇}{た}ち\ruby{沁々}{しみ|じみ}\ruby{眺}{なが}む\\
        \ruby{旅}{たび}の\ruby{日}{ひ}ははや\ruby{暮}{く}れゆきぬ
        \ruby{夢}{ゆめ}に\ruby{酔}{よ}ひ\ruby{夢}{ゆめ}にぞ\ruby{歎}{な}かん\\
        \ruby{汚}{けが}れなき\ruby{心}{こころ}を\ruby{慕}{した}ふ
        \ruby{大}{おお}いなる\ruby{支笏}{し|こつ}の\ruby{湖}{うみ}よ\\
        \ruby{花}{はな}\ruby{若}{わか}く\ruby{我}{われ}\ruby{汝}{な}が\ruby{許}{もと}に
        \ruby[g]{希望}{のぞみ}\ruby{満}{み}ち\ruby[g]{今宵}{こよい}\ruby{宿}{やど}らん
        
        \vspace{\linespace}
        \item~\\
        % 2.
        \ruby{轟}{とどろ}けるかの\ruby{雄叫}{を|たけ}びよ\ruby{創造}{そう|ぞう}の
        \ruby{歴程}{れき|てい}\ruby{一路}{いち|ろ}\\\ruby{新}{あたら}しき\ruby{使命}{し|めい}に\ruby{捧}{ささ}ぐ
        \ruby{幸}{しあわせ}の\ruby[g]{今日}{きょう}にしあれば\\
        \ruby{忍苦}{にん|く}して\ruby[g]{欣求}{もと}むるところ
        \ruby{得}{う}べくして\ruby{得}{う}べからざりし\\
        \ruby[g]{秀麗}{うる}はしきまことの\ruby{道}{みち}ぞ
        \ruby{近}{ちか}くして\ruby{遙}{はる}かなる\ruby{哉}{かな}\\
        \ruby{若}{わか}き\ruby{世}{よ}の\ruby{秩序}{ちつ|じょ}を\ruby{背負}{せ|お}ふ
        \ruby{洋々}{よう|よう}の\ruby{日}{ひ}と\ruby{倶}{とも}にゆかなむ
        
    \end{minipage}
    \begin{minipage}[c]{\blocksize}
        
        \vspace{\linespace}
        \item~\\
        % 3.
        \ruby[g]{乾坤}{けんこん}に\ruby{伏}{ふ}し\ruby{祈}{いの}るなり\ruby[g]{栄光}{さかえ}あれ
        \ruby{祖国}{そ|こく}の\ruby[g]{生命}{いのち}\\\ruby{決意}{けつ|い}する
        \ruby{光}{ひかり}\ruby{眩}{まば}ゆく\ruby{手}{て}に\ruby{取}{と}りぬ\ruby{楡}{にれ}の\ruby{嫩葉}{わか|ば}\\
        \ruby{葉脈}{よう|みゃく}の\ruby{強}{つよ}きを\ruby{讃}{たた}ふ
        \ruby{草々}{くさ|ぐさ}のたふれ\ruby{生}{うま}れて\\
        \ruby{春}{はる}\ruby{青}{あお}み\ruby[g]{辛夷}{こぶし}\ruby{咲}{さ}くなり
        \ruby[g]{逍遙}{さすらひ}の\ruby[g]{原始林}{もり}\ruby{蔭}{かげ}\ruby{清}{きよ}く\\
        \ruby{暢}{の}び\ruby{行}{ゆ}かん\ruby{我}{わ}が\ruby{民族}{みん|ぞく}の
        \ruby{逞}{たくま}しき\ruby[g]{息吹}{いぶ}き\ruby{感}{かん}じぬ
        
        \vspace{\linespace}
        \item~\\
        % 4.
        \ruby{立}{た}て\ruby{歩}{あゆ}め\ruby{光}{ひかり}の\ruby{中}{なか}を\ruby{国民}{くに|たみ}の
        \ruby{重}{おも}き\ruby[g]{責任}{せめ}\ruby{負}{お}ひ\\\ruby{燦}{きら}めきの
        \ruby[g]{星辰}{ほし}は\ruby{語}{かた}らひ\ruby{微香}{ほの|かを}る\ruby[g]{大地}{つち}\ruby{囁}{ささや}きぬ\\
        \ruby[g]{甦生}{たちか}へる\ruby{征覇}{せい|は}のいくさ\ruby[g]{祝歌}{はぎうた}ふ
        \ruby{吾等}{われ|ら}が\ruby[g]{双頬}{ほほ}に\\
        \ruby{失}{うしな}はじ\ruby{高}{たか}きが\ruby[g]{矜持}{ほこり}\ruby{護}{まも}り\ruby{来}{こ}し
        \ruby[g]{伝統}{つたへ}の\ruby[g]{法火}{ともし}\\
        \ruby{浄}{きよ}らかに\ruby{燃}{も}え\ruby{熾}{さか}る\ruby{刻}{とき}
        \ruby{継}{つ}ぎ\ruby{行}{ゆ}かな\ruby{来}{こ}ん\ruby[g]{若人}{わこうど}に
    
    \end{minipage}
\end{enumerate} % 番号の箇条書き ここまで
%%%%% 歌詞 ここまで %%%%%
% end body

\end{document}

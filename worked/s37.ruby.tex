\documentclass[10pt,b5j]{tarticle} % B6 縦書き
% \documentclass[10pt,b5j]{tarticle} % B6 縦書き
\AtBeginDvi{\special{papersize=128mm,182mm}} % B6 用用紙サイズ
\usepackage{otf} % Unicode で字を入力するのに必要なパッケージ
\usepackage[size=b6j]{bxpapersize} % B6 用紙サイズを指定
\usepackage[dvipdfmx]{graphicx} % 画像を挿入するためのパッケージ
\usepackage[dvipdfmx]{color} % 色をつけるためのパッケージ
\usepackage{pxrubrica} % ルビを振るためのパッケージ
\usepackage{multicol} % 複数段組を作るためのパッケージ
\setlength{\topmargin}{14mm} % 上下方向のマージン
\addtolength{\topmargin}{-1in} % 
\setlength{\oddsidemargin}{11mm} % 左右方向のマージン
\addtolength{\oddsidemargin}{-1in} % 
\setlength{\textwidth}{154mm} % B6 用
\setlength{\textheight}{108mm} % B6 用
\setlength{\headsep}{0mm} % 
\setlength{\headheight}{0mm} % 
\setlength{\topskip}{0mm} % 
\setlength{\parskip}{0pt} % 
\def\labelenumi{\theenumi、} % 箇条書きのフォーマット
\parindent = 0pt % 段落下げしない

 % B6 用テンプレート読み込み

\begin{document}
% begin header
%%%%% タイトルと作者 ここから %%%%%
\begin{minipage}[c]{0.7\hsize} % タイトルは上から 7 割
    \begin{center}
    % begin title
        {\LARGE
            壁歌は語る % タイトルを入れる
        }
        {\small 
            (昭和三十七年寮歌) % 年などを入れる
        }
    % end title
    \end{center}
\end{minipage}
\begin{minipage}[c]{0.3\hsize} % 作歌作曲は上から 3 割
    \begin{flushright} % 下寄せにする
        % begin name
        執行洋視君 作歌\\助川秀三郎君 作曲 % 作歌・作曲者
        % end name
    \end{flushright}
\end{minipage}
%%%%% タイトルと作者 ここまで %%%%%
% (1,2,3 了あり)
% end header

% begin length
\vspace{1.5em} % タイトル, 作者と歌詞の間に隙間を設ける
\newcommand{\linespace}{0.5em} % 行間の設定
\newcommand{\blocksize}{0.5\hsize} % 段組間の設定
\newcommand{\itemmargin}{3em} % 曲番の位置調整の長さ
% end length
% begin body
%%%%% 歌詞 ここから %%%%%
\begin{enumerate} % 番号の箇条書き ここから
    \setlength{\itemindent}{\itemmargin} % 曲番の位置調整
    \begin{minipage}[c]{\blocksize}
    
        \vspace{\linespace}
        \item~\\
        % 1.
        \ruby{壁歌}{へき|か}は\ruby{語}{かた}る\ruby{幾星霜}{いく|せい|そう}\\
        \ruby{集}{あつま}り\ruby{散}{さん}ず\ruby[g]{若人}{わこうど}が\\
        \ruby{夜々}{よ|よ}に\ruby{語}{の}ったる\ruby{苦悩}{く|のう}の\ruby{記}{き}\\
        \ruby{日々}{ひ|び}に\ruby{語}{の}ったる\ruby{歓喜}{かん|き}の\ruby{記}{き}\\
        ああその\ruby{意気}{い|き}は\ruby[g]{永遠}{とわ}に\ruby{栄}{さか}えん
        
        \vspace{\linespace}
        \item~\\
        % 2.
        \ruby{壁歌}{へき|か}は\ruby{続}{つづ}く\ruby[g]{百年}{ももとせ}に\\
        \ruby{美辞}{び|じ}をば\ruby{嫌}{きら}いし\ruby[g]{若人}{わこうど}が\\
        \ruby{好機}{こう|き}に\ruby{変}{か}えたる\ruby[g]{時流}{とき}の\ruby{言}{げん}\\
        \ruby{好機}{こう|き}に\ruby{乗}{の}りし\ruby[g]{時流}{とき}の\ruby{波}{なみ}\\
        ああその\ruby[g]{思出}{おもい}いつか\ruby{崩}{くず}れん
        
    \end{minipage}
    \begin{minipage}[c]{\blocksize}
        
        \vspace{\linespace}
        \item~\\
        % 3.
        \ruby{壁歌}{へき|か}は\ruby{残}{のこ}る\ruby{千代}{せん|だい}に\\
        \ruby{日夜}{にち|や}ひもとき\ruby{探索}{たん|さく}に\\
        \ruby{我}{われ}が\ruby{捨}{す}てたる\ruby[g]{邪道}{よこしま}よ\\
        \ruby{我}{われ}が\ruby{容}{い}れたる\ruby[g]{真理}{まごころ}よ\\
        ああその\ruby[g]{純情}{こころ}\ruby{後}{のち}に\ruby{偲}{しの}ばん
    
    \end{minipage}
\end{enumerate} % 番号の箇条書き ここまで
%%%%% 歌詞 ここまで %%%%%
% end body

\end{document}

\documentclass[10pt,b5j]{tarticle} % B6 縦書き
% \documentclass[10pt,b5j]{tarticle} % B6 縦書き
\AtBeginDvi{\special{papersize=128mm,182mm}} % B6 用用紙サイズ
\usepackage{otf} % Unicode で字を入力するのに必要なパッケージ
\usepackage[size=b6j]{bxpapersize} % B6 用紙サイズを指定
\usepackage[dvipdfmx]{graphicx} % 画像を挿入するためのパッケージ
\usepackage[dvipdfmx]{color} % 色をつけるためのパッケージ
\usepackage{pxrubrica} % ルビを振るためのパッケージ
\usepackage{multicol} % 複数段組を作るためのパッケージ
\setlength{\topmargin}{14mm} % 上下方向のマージン
\addtolength{\topmargin}{-1in} % 
\setlength{\oddsidemargin}{11mm} % 左右方向のマージン
\addtolength{\oddsidemargin}{-1in} % 
\setlength{\textwidth}{154mm} % B6 用
\setlength{\textheight}{108mm} % B6 用
\setlength{\headsep}{0mm} % 
\setlength{\headheight}{0mm} % 
\setlength{\topskip}{0mm} % 
\setlength{\parskip}{0pt} % 
\def\labelenumi{\theenumi、} % 箇条書きのフォーマット
\parindent = 0pt % 段落下げしない

 % B6 用テンプレート読み込み

\begin{document}
% begin header
%%%%% タイトルと作者 ここから %%%%%
\begin{minipage}[c]{0.7\hsize} % タイトルは上から 7 割
    \begin{center}
    % begin title
        {\LARGE
            鴉翼の影 % タイトルを入れる
        }
        {\small 
            (令和二年度寮歌) % 年などを入れる
        }
    % end title
    \end{center}
\end{minipage}
\begin{minipage}[c]{0.3\hsize} % 作歌作曲は上から 3 割
    \begin{flushright} % 下寄せにする
        % begin name
        落合海宇君 作歌\\加納央人君 作曲 % 作歌・作曲者
        % end name
    \end{flushright}
\end{minipage}
%%%%% タイトルと作者 ここまで %%%%%
% (1,2,3 繰り返しなし)
% end header

% begin length
\vspace{1.5em} % タイトル, 作者と歌詞の間に隙間を設ける
\newcommand{\linespace}{0.5em} % 行間の設定
\newcommand{\blocksize}{0.5\hsize} % 段組間の設定
\newcommand{\itemmargin}{3em} % 曲番の位置調整の長さ
% end length
% begin body
%%%%% 歌詞 ここから %%%%%
\begin{enumerate} % 番号の箇条書き ここから
    \setlength{\itemindent}{\itemmargin} % 曲番の位置調整
    \begin{minipage}[c]{\blocksize}
    
        \vspace{\linespace}
        \item~\\
        % 1.
        \ruby{雲居}{くも|い}の\ruby{空}{そら}に \ruby{黒銀}{こく|ぎん}の\ruby{羽}{はね}\\
        \ruby{六華}{りっ|か}の\ruby[g]{深緑}{みどり} \ruby{薄}{うす}れゆき\\
        \ruby{二豎}{に|じゅ}の\ruby{魔}{ま} \ruby{北溟}{ほく|めい}の\ruby{地}{ち}を\ruby{蝕}{むしば}みて\\
        \ruby[g]{鴉翼}{からす}は なにをか\ruby{鳴}{な}かん
        
        \vspace{\linespace}
        \item~\\
        % 2.
        \ruby{空}{そら}の\ruby{鏡}{かがみ}に \ruby{夜}{よ}さりの\ruby{事跡}{じ|せき}\\
        \ruby[g]{若人}{わこうど}の\ruby[g]{光迪}{ひかり} \ruby{幽}{かそ}けきものへ\\
        \ruby{闇}{やみ}の\ruby[g]{暗夜}{よ} \ruby{溌溂}{はつ|らつ}たる\ruby{影}{かげ}を\ruby{牽}{ひ}きて\\
        \ruby[g]{鴉翼}{からす}は なにをか\ruby{知}{し}らん
        
    \end{minipage}
    \begin{minipage}[c]{\blocksize}
        
        \vspace{\linespace}
        \item~\\
        % 3.
        \ruby[g]{朝明}{あさけ}の\ruby{風}{かぜ}に たなびく\ruby[g]{黒翼}{つばさ}\\
        \ruby{陽}{ひ}は\ruby{悠揚}{ゆう|よう}と \ruby{手稲}{て|いね}の\ruby[g]{山端}{はし}に\\
        \ruby{春}{はる}の\ruby{芽吹}{め|ぶ}き \ruby[g]{静寂}{しじま}の\ruby{楡林}{ゆ|りん}の\ruby{中}{なか}に\\
        \ruby[g]{鴉翼}{からす}は なにをか\ruby{語}{かた}らん

    \end{minipage}
\end{enumerate} % 番号の箇条書き ここまで
%%%%% 歌詞 ここまで %%%%%
% end body

\end{document}

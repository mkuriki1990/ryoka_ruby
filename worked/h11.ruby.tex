\documentclass[10pt,b5j]{tarticle} % B6 縦書き
% \documentclass[10pt,b5j]{tarticle} % B6 縦書き
\AtBeginDvi{\special{papersize=128mm,182mm}} % B6 用用紙サイズ
\usepackage{otf} % Unicode で字を入力するのに必要なパッケージ
\usepackage[size=b6j]{bxpapersize} % B6 用紙サイズを指定
\usepackage[dvipdfmx]{graphicx} % 画像を挿入するためのパッケージ
\usepackage[dvipdfmx]{color} % 色をつけるためのパッケージ
\usepackage{pxrubrica} % ルビを振るためのパッケージ
\usepackage{multicol} % 複数段組を作るためのパッケージ
\setlength{\topmargin}{14mm} % 上下方向のマージン
\addtolength{\topmargin}{-1in} % 
\setlength{\oddsidemargin}{11mm} % 左右方向のマージン
\addtolength{\oddsidemargin}{-1in} % 
\setlength{\textwidth}{154mm} % B6 用
\setlength{\textheight}{108mm} % B6 用
\setlength{\headsep}{0mm} % 
\setlength{\headheight}{0mm} % 
\setlength{\topskip}{0mm} % 
\setlength{\parskip}{0pt} % 
\def\labelenumi{\theenumi、} % 箇条書きのフォーマット
\parindent = 0pt % 段落下げしない

 % B6 用テンプレート読み込み

\begin{document}
% begin header
%%%%% タイトルと作者 ここから %%%%%
\begin{minipage}[c]{0.7\hsize} % タイトルは上から 7 割
    \begin{center}
    % begin title
        {\LARGE
            清華の誓 % タイトルを入れる
        }
        {\small 
            (平成十一年度寮歌) % 年などを入れる
        }
    % end title
    \end{center}
\end{minipage}
\begin{minipage}[c]{0.3\hsize} % 作歌作曲は上から 3 割
    \begin{flushright} % 下寄せにする
        % begin name
        荒木洋祐君 作歌\\小出隆広君 作曲 % 作歌・作曲者
        % end name
    \end{flushright}
\end{minipage}
%%%%% タイトルと作者 ここまで %%%%%
% (1,2 了あり)
% end header

% begin length
\vspace{1.5em} % タイトル, 作者と歌詞の間に隙間を設ける
\newcommand{\linespace}{0.5em} % 行間の設定
\newcommand{\blocksize}{0.5\hsize} % 段組間の設定
\newcommand{\itemmargin}{3em} % 曲番の位置調整の長さ
% end length
% begin body
%%%%% 歌詞 ここから %%%%%
\begin{enumerate} % 番号の箇条書き ここから
    \setlength{\itemindent}{\itemmargin} % 曲番の位置調整
    \begin{minipage}[c]{\blocksize}
    
        \vspace{\linespace}
        \item~\\
        % 1.
        \ruby{雪}{ゆき}\ruby{舞}{ま}う\ruby{地平}{ち|へい}にひときわ\ruby{映}{は}える\\
        \ruby{六華}{りっ|か}の\ruby{紋}{もん}ぞ\ruby{我等}{われ|ら}が\ruby{砦}{とりで}\\
        \ruby{野心}{や|しん}は\ruby{満}{み}ちて\ruby{冬空}{ふゆ|ぞら}\ruby{焦}{こ}がし\\
        \ruby{樹間}{じゅ|かん}の\ruby{風}{かぜ}は\ruby[g]{情熱}{おもい}を\ruby{運}{はこ}ぶ\\
        \ruby{杯}{さかづき}に\ruby{写}{うつ}る\ruby{未来}{み|らい}をみよう\\
        \ruby{夜明}{よ|あ}かし\ruby{語}{かた}るこの\ruby{今}{とき}にこそ\\
        カペラの\ruby{叡智}{えい|ち}オリオンの\ruby{武勇}{ぶ|ゆう}\\
        \ruby{天}{てん}よ\ruby{闇}{やみ}よ\ruby{我等}{われ|ら}に\ruby{賜}{たま}え
        
    \end{minipage}
    \begin{minipage}[c]{\blocksize}
        
        \vspace{\linespace}
        \item~\\
        % 2.
        \ruby{国}{くに}を\ruby{覆}{おお}い\ruby{地球}{ち|きゅう}を\ruby{揺}{ゆ}るがす\\
        \ruby{四百}{よん|ひゃく}\ruby{志士}{し|し}の\ruby{夢}{ゆめ}よ\ruby{醒}{さ}めよ\\
        \ruby[g]{太平洋}{せかい}にかかる
        \ruby{橋}{はし}にぞなれる\\
        \ruby{我等}{われ|ら}がゆくてに\ruby{光}{ひかり}あり\\
        \ruby{寮}{ここ}で\ruby{培}{つちか}う\ruby{時間}{じ|かん}を\ruby{糧}{かて}に\\
        いざうちつれて\ruby{歩}{あゆ}み\ruby{出}{だ}そう\\
        \ruby{北}{きた}の\ruby{都}{みやこ}に\ruby{世紀}{せい|き}はめぐり\\
        \ruby{清華}{せい|か}の\ruby{誓}{ちかい}\ruby{今}{いま}ここに
    
    \end{minipage}
\end{enumerate} % 番号の箇条書き ここまで
%%%%% 歌詞 ここまで %%%%%
% end body

\end{document}

\documentclass[10pt,b5j]{tarticle} % B6 縦書き
% \documentclass[10pt,b5j]{tarticle} % B6 縦書き
\AtBeginDvi{\special{papersize=128mm,182mm}} % B6 用用紙サイズ
\usepackage{otf} % Unicode で字を入力するのに必要なパッケージ
\usepackage[size=b6j]{bxpapersize} % B6 用紙サイズを指定
\usepackage[dvipdfmx]{graphicx} % 画像を挿入するためのパッケージ
\usepackage[dvipdfmx]{color} % 色をつけるためのパッケージ
\usepackage{pxrubrica} % ルビを振るためのパッケージ
\usepackage{multicol} % 複数段組を作るためのパッケージ
\setlength{\topmargin}{14mm} % 上下方向のマージン
\addtolength{\topmargin}{-1in} % 
\setlength{\oddsidemargin}{11mm} % 左右方向のマージン
\addtolength{\oddsidemargin}{-1in} % 
\setlength{\textwidth}{154mm} % B6 用
\setlength{\textheight}{108mm} % B6 用
\setlength{\headsep}{0mm} % 
\setlength{\headheight}{0mm} % 
\setlength{\topskip}{0mm} % 
\setlength{\parskip}{0pt} % 
\def\labelenumi{\theenumi、} % 箇条書きのフォーマット
\parindent = 0pt % 段落下げしない

 % B6 用テンプレート読み込み

\begin{document}
% begin header
%%%%% タイトルと作者 ここから %%%%%
\begin{minipage}[c]{0.7\hsize} % タイトルは上から 7 割
    \begin{center}
    % begin title
        {\LARGE
            魔神の呪 % タイトルを入れる
        }
        {\small 
            (大正六年寮歌) % 年などを入れる
        }
    % end title
    \end{center}
\end{minipage}
\begin{minipage}[c]{0.3\hsize} % 作歌作曲は上から 3 割
    \begin{flushright} % 下寄せにする
        % begin name
        佐藤惣之助君 作歌\\植村泰二君 作曲 % 作歌・作曲者
        % end name
    \end{flushright}
\end{minipage}
%%%%% タイトルと作者 ここまで %%%%%
% (1,6 了あり)
% end header

% begin length
\vspace{1.5em} % タイトル, 作者と歌詞の間に隙間を設ける
\newcommand{\linespace}{0.5em} % 行間の設定
\newcommand{\blocksize}{0.33\hsize} % 段組間の設定
\newcommand{\itemmargin}{3em} % 曲番の位置調整の長さ
% end length
% begin body
%%%%% 歌詞 ここから %%%%%
\begin{enumerate} % 番号の箇条書き ここから
    \setlength{\itemindent}{\itemmargin} % 曲番の位置調整
    \begin{minipage}[c]{\blocksize}
    
        \vspace{\linespace}
        \item~\\
        % 1.
        \ruby{魔神}{ま|じん}の\ruby{呪}{のろい}アルペンの\\
        \ruby{白雪}{はく|せつ}\ruby[g]{永久}{とは}に\ruby{清}{きよ}からず\\
        \ruby{見}{み}よ\ruby{永劫}{えい|ごう}と\ruby{誓}{ちか}ひけん\\
        \ruby{平和}{へい|わ}の\ruby{春}{はる}は\ruby{短}{みじか}くて\\
        \ruby{吹}{ふ}く\ruby{凋落}{ちょう|らく}の\ruby{秋風}{あき|かぜ}に\\
        \ruby{正義}{せい|ぎ}の\ruby{光}{ひかり}\ruby{影}{かげ}くらし
        
        \vspace{\linespace}
        \item~\\
        % 2.
        されど\ruby{儼然}{げん|ぜん}\ruby{東洋}{とう|よう}に\\
        その\ruby{義}{ぎ}と\ruby{侠}{きょう}を\ruby{胸}{むね}にして\\
        \ruby{燦}{さん}たる\ruby{北斗}{ほく|と}\ruby{北陲}{ほく|すい}の\\
        \ruby{強}{きょう}と\ruby{仰}{あお}がれ\ruby[g]{誇矜}{ほこ}りつつ\\
        \ruby{自治}{じ|ち}を\ruby[g]{精神}{いのち}の\ruby{我}{わが}\ruby{寮}{りょう}は\\
        \ruby[g]{映華}{はえ}ある\ruby{歴史}{れき|し}\ruby{十二年}{じゅう|に|ねん}
        
    \end{minipage}
    \begin{minipage}[c]{\blocksize}
        
        \vspace{\linespace}
        \item~\\
        % 3.
        \ruby{嗚呼}{あ|あ}\ruby{北海}{ほっ|かい}の\ruby{荒}{あら}\ruby[g]{吹雪}{ふぶき}\\
        \ruby{白箭}{はく|せん}\ruby{膚}{はだ}を\ruby{擘}{つんざ}くも\\
        \ruby{胸}{むね}の\ruby{狂瀾}{きょう|らん}\ruby{青春}{せい|しゅん}の\\
        \ruby{血潮}{ち|しお}に\ruby[g]{如何}{いか}で\ruby{比}{ひ}すべきぞ\\
        \ruby{力}{ちから}の\ruby{緒琴}{お|ごと}\ruby{高鳴}{たか|な}りて\\
        \ruby{紅}{くれない}\ruby{燃}{も}ゆる\ruby{悶}{もだ}えあり
        
        \vspace{\linespace}
        \item~\\
        % 4.
        \ruby{残陽}{ざん|よう}\ruby{西}{にし}に\ruby{茜}{あかね}して\\
        \ruby[g]{今日}{きょう}も\ruby{暮}{く}れ\ruby{行}{ゆ}く\ruby{手稲山}{て|いね|やま}\\
        \ruby{雲}{くも}の\ruby{五彩}{ご|さい}を\ruby{眺}{なが}めては\\
        \ruby{思}{おも}ひは\ruby{遠}{とお}く\ruby{渺茫}{べう|ぼう}の\\
        \ruby{彼}{か}の\ruby{海}{うみ}を\ruby{越}{こ}え\ruby{山}{やま}を\ruby{越}{こ}え\\
        \ruby{雄図}{ゆう|と}\ruby{千里}{せん|り}ぞ\ruby{駆}{はし}りゆく
        
    \end{minipage}
    \begin{minipage}[c]{\blocksize}
        
        \vspace{\linespace}
        \item~\\
        % 5.
        \ruby{平和}{へい|わ}の\ruby{流}{なが}れ\ruby{豊平}{とよ|ひら}の\\
        \ruby{狭霧}{さ|ぎり}\ruby{罩}{こ}めたる\ruby{朝}{あさ}ぼらけ\\
        \ruby{東}{ひんがし}\ruby{指}{さ}して\ruby{流}{なが}れ\ruby{行}{ゆ}く\\
        \ruby{淙々}{そう|そう}の\ruby{音}{ね}を\ruby{我}{われ}\ruby{聴}{き}けば\\
        \ruby{瀬々}{せ|せ}の\ruby{河波}{かは|なみ}\ruby{声}{こえ}あげて\\
        \ruby{唄}{うた}ふ「\ruby{自由}{じ|ゆう}」の\ruby{二字}{に|じ}の\ruby{曲}{きょく}
        
        \vspace{\linespace}
        \item~\\
        % 6.
        \ruby[g]{今宵}{こよい}\ruby{楡影}{ゆ|えい}に\ruby[g]{団欒}{まどゐ}して\\
        \ruby{月影}{つき|かげ}に\ruby{酌}{く}む\ruby{自治}{じ|ち}の\ruby{宴}{えん}\\
        \ruby{廻}{めぐ}る\ruby{盃}{さかずき}\ruby{夜}{よ}も\ruby{更}{ふ}けて\\
        \ruby{北斗}{ほく|と}\ruby{傾}{かたむ}く\ruby{玻璃}{は|り}の\ruby{窓}{まど}\\
        いざ\ruby{吾}{わ}が\ruby{友}{とも}よ\ruby[g]{熟睡}{うまい}せむ\\
        \ruby[g]{明日}{あす}は\ruby{人生}{じん|せい}の\ruby{旅}{たび}なれば
    
    \end{minipage}
\end{enumerate} % 番号の箇条書き ここまで
%%%%% 歌詞 ここまで %%%%%
% end body

\end{document}

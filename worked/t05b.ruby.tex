\documentclass[10pt,b5j]{tarticle} % B6 縦書き
% \documentclass[10pt,b5j]{tarticle} % B6 縦書き
\AtBeginDvi{\special{papersize=128mm,182mm}} % B6 用用紙サイズ
\usepackage{otf} % Unicode で字を入力するのに必要なパッケージ
\usepackage[size=b6j]{bxpapersize} % B6 用紙サイズを指定
\usepackage[dvipdfmx]{graphicx} % 画像を挿入するためのパッケージ
\usepackage[dvipdfmx]{color} % 色をつけるためのパッケージ
\usepackage{pxrubrica} % ルビを振るためのパッケージ
\usepackage{multicol} % 複数段組を作るためのパッケージ
\setlength{\topmargin}{14mm} % 上下方向のマージン
\addtolength{\topmargin}{-1in} % 
\setlength{\oddsidemargin}{11mm} % 左右方向のマージン
\addtolength{\oddsidemargin}{-1in} % 
\setlength{\textwidth}{154mm} % B6 用
\setlength{\textheight}{108mm} % B6 用
\setlength{\headsep}{0mm} % 
\setlength{\headheight}{0mm} % 
\setlength{\topskip}{0mm} % 
\setlength{\parskip}{0pt} % 
\def\labelenumi{\theenumi、} % 箇条書きのフォーマット
\parindent = 0pt % 段落下げしない

 % B6 用テンプレート読み込み

\begin{document}
% begin header
%%%%% タイトルと作者 ここから %%%%%
\begin{minipage}[c]{0.7\hsize} % タイトルは上から 7 割
    \begin{center}
    % begin title
        {\LARGE
            荒潮繞る % タイトルを入れる
        }
        {\small 
            (大正五年北寮寮歌) % 年などを入れる
        }
    % end title
    \end{center}
\end{minipage}
\begin{minipage}[c]{0.3\hsize} % 作歌作曲は上から 3 割
    \begin{flushright} % 下寄せにする
        % begin name
        桜井芳次郎君 作歌\\橋本吉郎君 作曲 % 作歌・作曲者
        % end name
    \end{flushright}
\end{minipage}
%%%%% タイトルと作者 ここまで %%%%%
% (1,2,3,4,5,6,7 了あり)
% end header

% begin length
\vspace{1.5em} % タイトル, 作者と歌詞の間に隙間を設ける
\newcommand{\linespace}{0.5em} % 行間の設定
\newcommand{\blocksize}{0.33\hsize} % 段組間の設定
\newcommand{\itemmargin}{3em} % 曲番の位置調整の長さ
% end length
% begin body
%%%%% 歌詞 ここから %%%%%
\begin{enumerate} % 番号の箇条書き ここから
    \setlength{\itemindent}{\itemmargin} % 曲番の位置調整
    \begin{minipage}[c]{\blocksize}
    
        \vspace{\linespace}
        \item~\\
        % 1.
        \ruby{荒潮}{あら|しほ}\ruby{繞}{めぐ}る\ruby{北}{きた}の\ruby{郷}{さと}\\
        \ruby{絢爛}{けん|らん}の\ruby{時}{とき}いと\ruby{高}{たか}く\\
        \ruby{看}{み}よ\ruby{極光}{きょく|こう}に\ruby{照}{て}らされて\\
        \ruby{夢}{ゆめ}にまどろむ\ruby{春}{はる}の\ruby{精}{せい}
        
        \vspace{\linespace}
        \item~\\
        % 2.
        \ruby{嗚呼}{あ|あ}\ruby{感激}{かん|げき}の\ruby[g]{経営}{いとなみ}を\\
        \ruby{矜}{ほこ}る\ruby{血潮}{ち|しほ}に\ruby{求}{もと}め\ruby{来}{き}て\\
        \ruby{十一}{じゅう|いち}の\ruby{年}{とし}の\ruby[g]{旦暮}{あけくれ}は\\
        \ruby{澄明}{ちょう|めい}の\ruby{府}{くに}\ruby{霊}{れい}\ruby{清}{きよ}し
        
        \vspace{\linespace}
        \item~\\
        % 3.
        \ruby{夏}{なつ}の\ruby{日}{ひ}\ruby[g]{悠然}{のどか}に\ruby{石狩}{いし|かり}の\\
        \ruby{浩蕩}{こう|とう}の\ruby{水}{みず}\ruby{煌}{きら}めきて\\
        \ruby{流光}{りゅう|こう}\ruby{高}{たか}く\ruby[g]{際涯}{はてし}なき\\
        \ruby{自然}{し|ぜん}の\ruby{業}{わざ}を\ruby{畏}{おそ}れずや
        
    \end{minipage}
    \begin{minipage}[c]{\blocksize}
        
        \vspace{\linespace}
        \item~\\
        % 4.
        \ruby{夕暮}{ゆう|ぐ}れ\ruby{呼}{よ}ばふ\ruby{閑古鳥}{かん|こ|どり}\\
        \ruby{冥想}{めい|そう}ここに\ruby{始}{はじ}めよと\\
        \ruby{遠}{とほ}\ruby{鳴}{な}くなべも\ruby[g]{紅葉}{もみぢ}しつ\\
        \ruby{稜畳}{りょう|てふ}として\ruby{唐錦}{から|にしき}
        
        \vspace{\linespace}
        \item~\\
        % 5.
        \ruby{北風}{きた|かぜ}\ruby{胡沙}{こ|さ}に\ruby{雪}{ゆき}を\ruby{捲}{ま}き\\
        \ruby{荒}{あ}れ\ruby{狂}{くる}ひたる\ruby[g]{戦場}{こば}の\ruby{跡}{あと}\\
        \ruby{暮}{く}れ\ruby{行}{ゆ}く\ruby[g]{蛮霧}{もや}に\ruby{包}{つつ}まれて\\
        \ruby[g]{白銀}{しろがね}の\ruby{都}{みやこ}\ruby{今}{いま}\ruby{静}{しづ}か
        
        \vspace{\linespace}
        \item~\\
        % 6.
        \ruby{清}{きよ}けき\ruby[g]{永久}{とは}の\ruby{霊泉}{れい|せん}の\\
        \ruby{至福}{し|ふく}の\ruby{水}{みず}を\ruby{掬}{むす}ぶ\ruby{可}{べ}く\\
        \ruby[g]{黄金}{こがね}の\ruby{甕}{もたひ}\ruby{守}{まも}りつつ\\
        \ruby{調}{ふし}\ruby{新}{あたら}しく\ruby{唱}{うた}はなん
        
    \end{minipage}
    \begin{minipage}[c]{\blocksize}
        
        \vspace{\linespace}
        \item~\\
        % 7.
        \ruby{智慧}{ち|え}の\ruby{光}{ひかり}に\ruby{導}{みちび}かれ\\
        \ruby{熱}{ねつ}の\ruby[g]{磅{\UTF{7934}}}{ひそみ}に\ruby{生立}{おい|た}ちて\\ % UTF 礴
        \ruby{潔}{きよ}き\ruby[g]{生活}{いのち}の\ruby{道}{みち}すがら\\
        \ruby{曲}{ふし}\ruby{勇}{いさ}ましく\ruby{唱}{うた}はなむ
    
    \end{minipage}
\end{enumerate} % 番号の箇条書き ここまで
%%%%% 歌詞 ここまで %%%%%
% end body

\end{document}

\documentclass[10pt,b5j]{tarticle} % B6 縦書き
% \documentclass[10pt,b5j]{tarticle} % B6 縦書き
\AtBeginDvi{\special{papersize=128mm,182mm}} % B6 用用紙サイズ
\usepackage{otf} % Unicode で字を入力するのに必要なパッケージ
\usepackage[size=b6j]{bxpapersize} % B6 用紙サイズを指定
\usepackage[dvipdfmx]{graphicx} % 画像を挿入するためのパッケージ
\usepackage[dvipdfmx]{color} % 色をつけるためのパッケージ
\usepackage{pxrubrica} % ルビを振るためのパッケージ
\usepackage{multicol} % 複数段組を作るためのパッケージ
\setlength{\topmargin}{14mm} % 上下方向のマージン
\addtolength{\topmargin}{-1in} % 
\setlength{\oddsidemargin}{11mm} % 左右方向のマージン
\addtolength{\oddsidemargin}{-1in} % 
\setlength{\textwidth}{154mm} % B6 用
\setlength{\textheight}{108mm} % B6 用
\setlength{\headsep}{0mm} % 
\setlength{\headheight}{0mm} % 
\setlength{\topskip}{0mm} % 
\setlength{\parskip}{0pt} % 
\def\labelenumi{\theenumi、} % 箇条書きのフォーマット
\parindent = 0pt % 段落下げしない

 % B6 用テンプレート読み込み

\begin{document}
% begin header
%%%%% タイトルと作者 ここから %%%%%
\begin{minipage}[c]{0.7\hsize} % タイトルは上から 7 割
    \begin{center}
    % begin title
        {\LARGE
            春静寂なる % タイトルを入れる
        }
        {\small 
            (昭和二十三年逍遙歌) % 年などを入れる
        }
    % end title
    \end{center}
\end{minipage}
\begin{minipage}[c]{0.3\hsize} % 作歌作曲は上から 3 割
    \begin{flushright} % 下寄せにする
        % begin name
        中島通雄君 作歌\\佐々木淳君 作曲 % 作歌・作曲者
        % end name
    \end{flushright}
\end{minipage}
%%%%% タイトルと作者 ここまで %%%%%
% (1 了あり)
% end header

% begin length
\vspace{1.5em} % タイトル, 作者と歌詞の間に隙間を設ける
\newcommand{\linespace}{0.5em} % 行間の設定
\newcommand{\blocksize}{0.33\hsize} % 段組間の設定
\newcommand{\itemmargin}{3em} % 曲番の位置調整の長さ
% end length
% begin body
%%%%% 歌詞 ここから %%%%%
\begin{enumerate} % 番号の箇条書き ここから
    \setlength{\itemindent}{\itemmargin} % 曲番の位置調整
    \begin{minipage}[c]{\blocksize}
    
        \vspace{\linespace}
        \item~\\
        % 1.
        \ruby{春}{はる}\ruby[g]{静寂}{しづか}なる\ruby{石狩}{いし|かり}の\\
        \ruby[g]{曠野}{の}に\ruby[g]{漂泊}{さすら}ひて\ruby{人}{ひと}を\ruby{哭}{な}き\\
        \ruby{秋}{あき}\ruby{蕭々}{しょう|しょう}の\ruby[g]{寮窓}{まど}に\ruby{倚}{よ}り\\
        \ruby{夕雲}{ゆう|ぐも}\ruby{遠}{とほ}く\ruby{友}{とも}を\ruby{呼}{よ}ぶ\\
        \ruby{北斗}{ほく|と}の\ruby[g]{啓光}{ひかり}さしそえど\\
        \ruby{哀}{あわ}れ\ruby{悲}{かな}しき\ruby{旅}{たび}ならむ
        
        \vspace{\linespace}
        \item~\\
        % 2.
        \ruby[g]{北溟}{きた}ゆく\ruby{雁}{かり}は\ruby{名}{な}のみにして\\
        \ruby{暮}{くれ}る\ruby{秋風}{あき|かぜ}に\ruby{啼}{な}く\ruby{虫}{むし}か\\
        \ruby[g]{楡梢}{こずえ}に\ruby{喘}{あえ}ぐ\ruby{郭公}{かっ|こう}か\\
        はた\ruby{又}{また}\ruby{魂}{たましい}の\ruby{語}{かた}らひか\\
        \ruby{現}{うつつ}の\ruby[g]{波濤}{なみ}は\ruby{荒}{あら}くとも\\
        \ruby{知}{し}るや\ruby{無象}{む|ぞう}の\ruby{天}{てん}の\ruby{外}{そと}
        
    \end{minipage}
    \begin{minipage}[c]{\blocksize}
        
        \vspace{\linespace}
        \item~\\
        % 3.
        \ruby{十勝}{と|かち}の\ruby{峰}{みね}に\ruby[g]{断雲}{くも}\ruby{怒}{いか}り\\
        \ruby{白銀}{はく|ぎん}\ruby{吼}{ほ}ゆる\ruby{朝風}{あさ|かぜ}も\\
        \ruby{奇}{く}しき\ruby{調}{しらべ}の\ruby{琴}{こと}と\ruby{聴}{き}き\\
        \ruby{燃}{も}ゆる\ruby[g]{理想}{おもひ}に\ruby{悶}{もだ}えつつ\\
        ただひたぶるに\ruby{辿}{たど}りゆく\\
        \ruby{長}{なが}き\ruby[g]{生命}{いのち}の\ruby[g]{斗争}{たたかひ}に
        
        \vspace{\linespace}
        \item~\\
        % 4.
        \ruby{自然}{し|ぜん}の\ruby[g]{芸術}{たくみ}\ruby{変}{かわ}らねど\\
        \ruby[g]{何処}{いづこ}に\ruby[g]{祓所}{ひそ}を\ruby{尋}{と}めゆかむ\\
        ああ\ruby[g]{孤独}{ひとりみ}の\ruby[g]{寂寥}{さびしら}を\\
        \ruby{味}{あじ}はひ\ruby{知}{し}れる\ruby{人}{ひと}ならで\\
        \ruby{誰}{だれ}に\ruby{語}{かた}らん\ruby{入相}{いり|あい}の\\
        \ruby{鐘}{かね}\ruby{鳴}{な}りひびく\ruby[g]{楡陵}{をか}の\ruby{上}{うえ}
        
    \end{minipage}
    \begin{minipage}[c]{\blocksize}
        
        \vspace{\linespace}
        \item~\\
        % 5.
        \ruby{花}{はな}\ruby{咲}{さ}き\ruby{散}{ち}りて\ruby{春秋}{しゅん|じゅう}の\\
        \ruby{遷}{うつ}りてここに\ruby{三星霜}{みつ|せい|そう}\\
        \ruby{逝}{い}にし\ruby[g]{遊宴}{うたげ}の\ruby[g]{宵󠄁}{よい}の\ruby{夢}{ゆめ}\\
        たぎる\ruby[g]{情熱}{なさけ}を\ruby{篝火}{かがり|び}に\\
        \ruby[g]{残恨}{のこん}の\ruby{杯}{つき}を\ruby{汲}{く}み\ruby{交}{か}はし\\
        \ruby[g]{高唱}{うた}はなんかな\ruby{自治}{じ|ち}の\ruby{歌}{うた}
        
        \vspace{\linespace}
        \item~\\
        % 6.
        \ruby{今}{いま}\ruby{逍遥}{しょう|よう}の\ruby[g]{原野}{の}に\ruby{萠}{も}ゆる\\
        \ruby{森}{もり}の\ruby{翠}{みどり}の\ruby{色}{いろ}\ruby{深}{ふか}く\\
        \ruby{行手}{ゆく|て}\ruby{遙}{はろ}けき\ruby{豊平}{とよ|ひら}の\\
        \ruby[g]{清流}{ながれ}に\ruby{泛}{うか}ぶ\ruby[g]{綺花}{はな}の\ruby{影}{かげ}\\
        \ruby{哀}{あわ}れ\ruby{愛}{いと}しき\ruby[g]{絢夢}{ゆめ}なれど\\
        \ruby{我}{わ}が\ruby[g]{生命}{いのち}こそ\ruby{真}{まこと}なれ
    
    \end{minipage}
\end{enumerate} % 番号の箇条書き ここまで
%%%%% 歌詞 ここまで %%%%%
% end body

\end{document}

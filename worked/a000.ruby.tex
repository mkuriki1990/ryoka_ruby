\documentclass[10pt,b5j]{tarticle} % B6 縦書き
% \documentclass[10pt,b5j]{tarticle} % B6 縦書き
\AtBeginDvi{\special{papersize=128mm,182mm}} % B6 用用紙サイズ
\usepackage{otf} % Unicode で字を入力するのに必要なパッケージ
\usepackage[size=b6j]{bxpapersize} % B6 用紙サイズを指定
\usepackage[dvipdfmx]{graphicx} % 画像を挿入するためのパッケージ
\usepackage[dvipdfmx]{color} % 色をつけるためのパッケージ
\usepackage{pxrubrica} % ルビを振るためのパッケージ
\usepackage{multicol} % 複数段組を作るためのパッケージ
\setlength{\topmargin}{14mm} % 上下方向のマージン
\addtolength{\topmargin}{-1in} % 
\setlength{\oddsidemargin}{11mm} % 左右方向のマージン
\addtolength{\oddsidemargin}{-1in} % 
\setlength{\textwidth}{154mm} % B6 用
\setlength{\textheight}{108mm} % B6 用
\setlength{\headsep}{0mm} % 
\setlength{\headheight}{0mm} % 
\setlength{\topskip}{0mm} % 
\setlength{\parskip}{0pt} % 
\def\labelenumi{\theenumi、} % 箇条書きのフォーマット
\parindent = 0pt % 段落下げしない

 % B6 用テンプレート読み込み

\begin{document}
% begin header
%%%%% タイトルと作者 ここから %%%%%
\begin{minipage}[c]{0.7\hsize} % タイトルは上から 7 割
    \begin{center}
    % begin title
        {\normalsize 
        	校歌
        }
        {\LARGE
            永遠の幸 % タイトルを入れる
        }
        {\small 
            (札幌農学校校歌) % 年などを入れる
        }
    % end title
    \end{center}
\end{minipage}
\begin{minipage}[c]{0.3\hsize} % 作歌作曲は上から 3 割
    \begin{flushright} % 下寄せにする
        % begin name
        大和田建樹氏 校閲\\有島武朗君 作歌\\納所弁次郎君 作曲 % 作歌・作曲者
        % end name
    \end{flushright}
\end{minipage}
%%%%% タイトルと作者 ここまで %%%%%
% (1 了あり)
% end header

% begin length
\vspace{1.5em} % タイトル, 作者と歌詞の間に隙間を設ける
\newcommand{\linespace}{0.5em} % 行間の設定
\newcommand{\blocksize}{0.33\hsize} % 段組間の設定
\newcommand{\itemmargin}{3em} % 曲番の位置調整の長さ
% end length
% begin body
%%%%% 歌詞 ここから %%%%%
\begin{enumerate} % 番号の箇条書き ここから
    \setlength{\itemindent}{\itemmargin}
    \begin{minipage}[c]{\blocksize}
    
        \vspace{\linespace}
        \item~\\
        % 1.
        \ruby[g]{永遠}{とこしへ}の\ruby{幸}{さち}
        \ruby{朽}{く}ちざる\ruby{誉}{ほまれ}\\
        つねに\ruby{我等}{われ|ら}がうへにあれ\\
        よるひる\ruby{育}{そだ}て
        あけくれ\ruby{教}{おし}へ\\
        \ruby{人}{ひと}となしし\ruby{我庭}{わが|にわ}に\\
        
    \end{minipage}
    \begin{minipage}[c]{\blocksize}

        \vspace{\linespace}
        \item~\\
        % 2.
        \ruby{北斗}{ほく|と}をつかん
        たかき\ruby[g]{希望}{のぞみ}は\\
        \ruby[g]{時代}{とき}を\ruby{照}{てら}す\ruby{光}{ひかり}なり\\
        \ruby{深雪}{み|ゆき}を\ruby{凌}{しの}ぐ
        \ruby{潔}{きよ}き\ruby[g]{節操}{みさお}は\\
        \ruby{国}{くに}を\ruby{守}{まも}る\ruby{力}{ちから}なり\\
        (※繰り返し)
        
    \end{minipage}
    \begin{minipage}[c]{\blocksize}
        
        \vspace{\linespace}
        \item~\\
        % 3.
        \ruby{山}{やま}は\ruby{裂}{さ}くとも
        \ruby{海}{うみ}はあすとも\\
        \ruby{真理正義}{しん|り|せい|ぎ}おつべしや\\
        \ruby{不朽}{ふ|きゅう}を\ruby{求}{もと}め
        \ruby{意気}{い|き}\ruby{相}{あい}ゆるす\\
        \ruby{我等}{われ|ら}\ruby[g]{丈夫}{ますらお}\ruby{此}{ここ}にあり\\
        (※繰り返し)

    \end{minipage}
    \begin{minipage}[c]{0.5\hsize}

        \vspace{\linespace}
        \item[(※) ]~\\
        イザイザイザ\\
        うちつれて
        \ruby{進}{すす}むは\ruby{今}{いま}ぞ\\
        \ruby{豊平}{とよ|ひら}の\ruby{川}{かわ}
        \ruby{尽}{つき}せぬながれ\\
        \ruby{友}{とも}たれ\ruby{永}{なが}く\ruby{友}{とも}たれ\\

    \end{minipage}
    \begin{minipage}[c]{0.5\hsize}

        \vspace{\linespace}
        (注 有島武郎在学中の明治三十三年の作。\\
        大和田建樹(一八五六 ‐ 一九一〇)は作詞の面で、\\
        納所弁次郎(一八六五 ‐ 一九三六)は作曲の面で、\\
        共に近代日本唱歌史に大きな足跡を残した。)

    \end{minipage}
\end{enumerate} % 番号の箇条書き ここまで
%%%%% 歌詞 ここまで %%%%%
% end body

\end{document}

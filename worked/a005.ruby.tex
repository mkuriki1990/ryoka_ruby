\documentclass[10pt,b5j]{tarticle} % B6 縦書き
% \documentclass[10pt,b5j]{tarticle} % B6 縦書き
\AtBeginDvi{\special{papersize=128mm,182mm}} % B6 用用紙サイズ
\usepackage{otf} % Unicode で字を入力するのに必要なパッケージ
\usepackage[size=b6j]{bxpapersize} % B6 用紙サイズを指定
\usepackage[dvipdfmx]{graphicx} % 画像を挿入するためのパッケージ
\usepackage[dvipdfmx]{color} % 色をつけるためのパッケージ
\usepackage{pxrubrica} % ルビを振るためのパッケージ
\usepackage{multicol} % 複数段組を作るためのパッケージ
\setlength{\topmargin}{14mm} % 上下方向のマージン
\addtolength{\topmargin}{-1in} % 
\setlength{\oddsidemargin}{11mm} % 左右方向のマージン
\addtolength{\oddsidemargin}{-1in} % 
\setlength{\textwidth}{154mm} % B6 用
\setlength{\textheight}{108mm} % B6 用
\setlength{\headsep}{0mm} % 
\setlength{\headheight}{0mm} % 
\setlength{\topskip}{0mm} % 
\setlength{\parskip}{0pt} % 
\def\labelenumi{\theenumi、} % 箇条書きのフォーマット
\parindent = 0pt % 段落下げしない

 % B6 用テンプレート読み込み

\begin{document}
% begin header
%%%%% タイトルと作者 ここから %%%%%
\begin{minipage}[c]{0.7\hsize} % タイトルは上から 7 割
    \begin{center}
    % begin title
        {\LARGE
            瓔珞みがく % タイトルを入れる
        }
        {\small 
            (大正九年桜星会歌) % 年などを入れる
        }
    % end title
    \end{center}
\end{minipage}
\begin{minipage}[c]{0.3\hsize} % 作歌作曲は上から 3 割
    \begin{flushright} % 下寄せにする
        % begin name
        佐藤一雄君 作歌\\置塩寄君 作曲 % 作歌・作曲者
        % end name
    \end{flushright}
\end{minipage}
%%%%% タイトルと作者 ここまで %%%%%
% (1,2,3,4,8 了あり)
% end header

% begin length
\vspace{1.5em} % タイトル, 作者と歌詞の間に隙間を設ける
\newcommand{\linespace}{0.5em} % 行間の設定
\newcommand{\blocksize}{0.33\hsize} % 段組間の設定
\newcommand{\itemmargin}{3em} % 曲番の位置調整の長さ
% end length
% begin body
%%%%% 歌詞 ここから %%%%%
\begin{enumerate} % 番号の箇条書き ここから
    \setlength{\itemindent}{\itemmargin} % 曲番の位置調整
    \begin{minipage}[c]{\blocksize}
    
        \vspace{\linespace}
        \item~\\
        % 1.
        \ruby{瓔珞}{よう|らく}みがく\ruby{石狩}{いし|かり}の\\
        \ruby{源}{みなもと}\ruby{遠}{とほ}く\ruby{訪}{と}ひくれば\\
        \ruby{原始}{げん|し}の\ruby{森}{もり}は\ruby{闇}{くら}くして\\
        \ruby{雪解}{ゆき|げ}の\ruby{泉}{いずみ}\ruby{玉}{たま}と\ruby{湧}{わ}く
        
        \vspace{\linespace}
        \item~\\
        % 2.
        \ruby{浜茄子}{はま|な|す}\ruby{紅}{あか}き\ruby{磯辺}{いそ|べ}にも\\
        \ruby{鈴蘭}{すず|らん}\ruby{薫}{かを}る\ruby{谷間}{たに|ま}にも\\
        \ruby[g]{愛奴}{あいぬ}の\ruby{姿}{すがた}\ruby{薄}{うす}れゆく\\
        \ruby{蝦夷}{え|ぞ}の\ruby{昔}{むかし}を\ruby{懐}{おも}ふかな
        
        \vspace{\linespace}
        \item~\\
        % 3.
        \ruby{今}{いま}\ruby{円山}{まる|やま}の\ruby{桜花}{さくら|ばな}\\
        \ruby{歴史}{れき|し}は\ruby{旧}{ふ}りて\ruby{四十年}{し|じゅう|ねん}\\
        \ruby{我}{わ}が\ruby{学}{まな}び\ruby{舎}{や}の\ruby{先人}{せん|じん}が\\
        \ruby{建}{た}てし\ruby{功}{いさお}はいや\ruby{栄}{さか}ゆ
        
    \end{minipage}
    \begin{minipage}[c]{\blocksize}
        
        \vspace{\linespace}
        \item~\\
        % 4.
        その\ruby{絢爛}{けん|らん}の\ruby{花霞}{はな|がすみ}\\
        \ruby[g]{憧憬}{あこが}れ\ruby{集}{つど}ふ\ruby{四百}{よん|ひゃく}の\\
        \ruby{健児}{けん|じ}が\ruby[g]{希望}{のぞみ}\ruby{深}{ふか}ければ\\
        \ruby{北斗}{ほく|と}に\ruby{強}{つよ}き\ruby{黙示}{もく|し}あり
        
        \vspace{\linespace}
        \item~\\
        % 5.
        \ruby{醜雲}{しこ|くも}\ruby{消}{き}えて\ruby{人}{ひと}の\ruby{世}{よ}に\\
        \ruby[g]{陽光}{ひ}はうららかに\ruby{輝}{かがや}けど\\
        \ruby{風}{かぜ}の\ruby[g]{名残}{なごり}のつきやらで\\
        \ruby{狂瀾}{きょう|らん}さわぐ\ruby{今}{いま}し\ruby{今}{いま}
        
        \vspace{\linespace}
        \item~\\
        % 6.
        \ruby{潮}{うしほ}に\ruby{暮}{く}るる\ruby{西}{にし}の\ruby{空}{そら}\\
        \ruby{月}{つき}も\ruby{凍}{こほ}らむシベリアの\\
        \ruby{吾}{わ}が\ruby[g]{皇軍}{みいくさ}を\ruby{思}{おも}ひては\\
        \ruby{猛}{た}けき\ruby{心}{こころ}の\ruby{躍}{おど}らずや
        
    \end{minipage}
    \begin{minipage}[c]{\blocksize}
        
        \vspace{\linespace}
        \item~\\
        % 7.
        \ruby[g]{白銀}{しろがね}\ruby{狂}{くる}ふ\ruby{埋}{うも}れ\ruby{路}{じ}も\\
        \ruby{踏}{ふ}みて\ruby{拓}{ひら}かむわが\ruby[g]{前途}{ゆくて}\\
        はろけき\ruby[g]{牧場}{まき}に\ruby{嘯}{うそぶ}けば\\
        \ruby{雲}{くも}\ruby{影}{かげ}はやし\ruby{草}{くさ}の\ruby{波}{なみ}
        
        \vspace{\linespace}
        \item~\\
        % 8.
        \ruby{想}{おもひ}を\ruby{秘}{ひ}めし\ruby[g]{若人}{わこうど}が\\
        \ruby{唇}{くちびる}かたくほほゑみつ\\
        \ruby{仰}{あお}げば\ruby{高}{たか}く\ruby{聳}{そび}え\ruby{立}{た}つ\\
        \ruby{羊蹄山}{よう|てい|ざん}に\ruby{雪}{ゆき}\ruby{潔}{きよ}し
    
    \end{minipage}
\end{enumerate} % 番号の箇条書き ここまで
%%%%% 歌詞 ここまで %%%%%
% end body

\end{document}

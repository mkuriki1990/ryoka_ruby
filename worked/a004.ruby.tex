\documentclass[10pt,b5j]{tarticle} % B6 縦書き
% \documentclass[10pt,b5j]{tarticle} % B6 縦書き
\AtBeginDvi{\special{papersize=128mm,182mm}} % B6 用用紙サイズ
\usepackage{otf} % Unicode で字を入力するのに必要なパッケージ
\usepackage[size=b6j]{bxpapersize} % B6 用紙サイズを指定
\usepackage[dvipdfmx]{graphicx} % 画像を挿入するためのパッケージ
\usepackage[dvipdfmx]{color} % 色をつけるためのパッケージ
\usepackage{pxrubrica} % ルビを振るためのパッケージ
\usepackage{multicol} % 複数段組を作るためのパッケージ
\setlength{\topmargin}{14mm} % 上下方向のマージン
\addtolength{\topmargin}{-1in} % 
\setlength{\oddsidemargin}{11mm} % 左右方向のマージン
\addtolength{\oddsidemargin}{-1in} % 
\setlength{\textwidth}{154mm} % B6 用
\setlength{\textheight}{108mm} % B6 用
\setlength{\headsep}{0mm} % 
\setlength{\headheight}{0mm} % 
\setlength{\topskip}{0mm} % 
\setlength{\parskip}{0pt} % 
\def\labelenumi{\theenumi、} % 箇条書きのフォーマット
\parindent = 0pt % 段落下げしない

 % B6 用テンプレート読み込み

\begin{document}
% begin header
%%%%% タイトルと作者 ここから %%%%%
\begin{minipage}[c]{0.7\hsize} % タイトルは上から 7 割
    \begin{center}
    % begin title
        {\LARGE
            流るる光途 % タイトルを入れる
        }
        {\small 
            (大正七年桜星会歌) % 年などを入れる
        }
    % end title
    \end{center}
\end{minipage}
\begin{minipage}[c]{0.3\hsize} % 作歌作曲は上から 3 割
    \begin{flushright} % 下寄せにする
        % begin name
         % 作歌・作曲者
        % end name
    \end{flushright}
\end{minipage}
%%%%% タイトルと作者 ここまで %%%%%
% % end header

% begin length
\vspace{1.5em} % タイトル, 作者と歌詞の間に隙間を設ける
\newcommand{\linespace}{0.5em} % 行間の設定
\newcommand{\blocksize}{0.5\hsize} % 段組間の設定
\newcommand{\itemmargin}{3em} % 曲番の位置調整の長さ
% end length
% begin body
%%%%% 歌詞 ここから %%%%%
\begin{enumerate} % 番号の箇条書き ここから
    \setlength{\itemindent}{\itemmargin} % 曲番の位置調整
    \begin{minipage}[c]{\blocksize}
    
        \vspace{\linespace}
        \item~\\
        % 1.
        \ruby{流}{ながれ}るヽ\ruby{光}{ひかり}\ruby{途}{と}\ruby{重}{かさ}ね\ruby{來}{らい}て\\
        \ruby{星霜}{せいそう}\ruby{此}{この}\ruby{處}{ところ}に\ruby{四十年}{よんじゅうねん,よんじゅーねん}\\
        \ruby{北斗}{ほくと}の\ruby{光}{ひかり}\ruby{眸}{ひとみ}さす\ruby{所}{ところ}\\
        \ruby{櫻}{さくら}かざして\ruby{先人}{せんじん}の\\
        \ruby{樹立}{じゅりつ}し\ruby{歴史}{れきし}を\ruby{偲}{しの}ぶ\ruby{時}{とき}\\
        \ruby{誰}{だれ}か\ruby{血汐}{ちしお}の\ruby{湧}{ゆう}かざらむ
        
    \end{minipage}
    \begin{minipage}[c]{\blocksize}
        
        \vspace{\linespace}
        \item~\\
        % 2.
        \ruby{咽}{むせ}ぶ\ruby{悲憤}{ひふん}の\ruby{誓}{ちかい}より\\
        \ruby{早}{は}や\ruby{七年}{ななねん,ななねん}の\ruby{春}{はる}うつり\\
        \ruby{人}{ひと}は\ruby{変遷}{へんせん}れど\ruby{三}{さん}\ruby{百}{ひゃく}の\\
        \ruby{健兒}{けんじ}\ruby{不滅}{ふめつ}の\ruby{意}{い}\ruby{氣}{}を\ruby{持}{じ}す\\
        いでや\ruby{謳}{}はん\ruby{北州}{ほくしゅう}の\\
        \ruby{精力}{せいりょく}に\ruby{満}{み}ちし\ruby{凱歌}{がいか}を
        
    \end{minipage}
    \begin{minipage}[c]{\blocksize}
        
        \vspace{\linespace}
        \item~\\
        % 3.
        \ruby{陽春}{ようしゅん}の\ruby{光}{ひかり}に\ruby{覆}{くつがえ}\ruby{翼}{つばさ}まれ\\
        \ruby{嫩}{ふたば}\ruby{草}{くさ}\ruby{萠}{めぐむ}ゆる\ruby{北}{きた}の\ruby{郷}{さと}\\
        \ruby{手稲}{ていね}の\ruby{麓}{ふもと}\ruby{健兒}{けんじ}\ruby{等}{とう}が\\
        \ruby{燃}{もゆる}ゆる\ruby{想}{そう}を\ruby{合唱}{がっしょう}せば\\
        \ruby{牧場}{ぼくじょう}の\ruby{彼方}{かなた}\ruby{際涯}{さいがい}しらず\\
        \ruby{高鳴}{たかな}たてヽ\ruby{響}{ひび}きゆく
        
    \end{minipage}
    \begin{minipage}[c]{\blocksize}
        
        \vspace{\linespace}
        \item~\\
        % 4.
        \ruby{豊平川}{とよひらがわ}の\ruby{夏}{なつ}の\ruby{夜}{よる}や\\
        \ruby{玉兎}{ぎょくと}の\ruby{踊}{おど}る\ruby{波}{なみ}の\ruby{上}{うえ}\\
        \ruby{自治}{じち}の\ruby{流}{りゅう}の\ruby{悠久}{ゆうきゅう}を\\
        \ruby{語}{かた}る\ruby{川邊}{かわべ}に\ruby{佇}{たたず}めば\\
        ありし\ruby{往昔}{おうせき}を\ruby{追憶}{ついおく}へとや\\
        \ruby{古塔}{ことう}に\ruby{響}{ひび}く\ruby{時}{とき}の\ruby{音}{おと}
        
    \end{minipage}
    \begin{minipage}[c]{\blocksize}
        
        \vspace{\linespace}
        \item~\\
        % 5.
        こヽ\ruby{石狩}{いしかり}の\ruby{大}{だい}\ruby{沃野}{よくや}\\
        \ruby{静}{せい}けき\ruby{秋}{あき}のめぐり\ruby{來}{らい}て\\
        \ruby{天}{てん}\ruby{紺青}{こんじょう}の\ruby{色}{いろ}ふかく\\
        \ruby{地}{ち}は\ruby{豊穣}{ほうじょう}なる\ruby{平和}{へいわ}\ruby{境}{さかい}\\
        \ruby{人}{ひと}は\ruby{有情}{うじょう}の\ruby{美}{うつく}しき\\
        \ruby{自然}{しぜん}の\ruby{愛}{あい}に\ruby{狎}{}るヽ\ruby{哉}{かな}
        
    \end{minipage}
    \begin{minipage}[c]{\blocksize}
        
        \vspace{\linespace}
        \item~\\
        % 6.
        \ruby{萬里}{まんり}\ruby{茫々}{ぼうぼう}\ruby{雪}{ゆき}の\ruby{海}{うみ}\\
        \ruby{白龍}{はくりゅう}\ruby{怒}{いか}り\ruby{風}{ふう}\ruby{叫}{さけ}ぶ\\
        \ruby{吹雪}{ふぶき}にさめし\ruby{暁}{あかつき}や\\
        \ruby{迷}{まよ}いの\ruby{雲}{くも}をおしひらき\\
        \ruby{常世}{とこよ}の\ruby{幸}{こう}を\ruby{惠}{めぐみ}むなる\\
        おヽ\ruby{紅}{べに}の\ruby{朝日}{あさひ}\ruby{影}{かげ}
        
    \end{minipage}
    \begin{minipage}[c]{\blocksize}
        
        \vspace{\linespace}
        \item~\\
        % 7.
        \ruby{北辰}{ほくしん}\ruby{冴}{さえ}ゆる\ruby{夕}{ゆう}まぐれ\\
        ボーイズ ビイ アンビシャスの\\
        \ruby{崇高}{すうこう}き\ruby{教}{きょう}を\ruby{胸}{むね}に\ruby{秘}{ひ}め\\
        エルムの\ruby{梢}{こずえ}とことばの\\
        \ruby{自由}{じゆう}の\ruby{調}{ちょう}\ruby{聴}{き}くところ\\
        \ruby{若}{わか}き\ruby{生命}{せいめい}を\ruby{誇}{ほこ}らばや
    
    \end{minipage}
\end{enumerate} % 番号の箇条書き ここまで
%%%%% 歌詞 ここまで %%%%%
% end body

\end{document}

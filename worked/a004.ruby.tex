\documentclass[10pt,b5j]{tarticle} % B6 縦書き
% \documentclass[10pt,b5j]{tarticle} % B6 縦書き
\AtBeginDvi{\special{papersize=128mm,182mm}} % B6 用用紙サイズ
\usepackage{otf} % Unicode で字を入力するのに必要なパッケージ
\usepackage[size=b6j]{bxpapersize} % B6 用紙サイズを指定
\usepackage[dvipdfmx]{graphicx} % 画像を挿入するためのパッケージ
\usepackage[dvipdfmx]{color} % 色をつけるためのパッケージ
\usepackage{pxrubrica} % ルビを振るためのパッケージ
\usepackage{multicol} % 複数段組を作るためのパッケージ
\setlength{\topmargin}{14mm} % 上下方向のマージン
\addtolength{\topmargin}{-1in} % 
\setlength{\oddsidemargin}{11mm} % 左右方向のマージン
\addtolength{\oddsidemargin}{-1in} % 
\setlength{\textwidth}{154mm} % B6 用
\setlength{\textheight}{108mm} % B6 用
\setlength{\headsep}{0mm} % 
\setlength{\headheight}{0mm} % 
\setlength{\topskip}{0mm} % 
\setlength{\parskip}{0pt} % 
\def\labelenumi{\theenumi、} % 箇条書きのフォーマット
\parindent = 0pt % 段落下げしない

 % B6 用テンプレート読み込み

\begin{document}
% begin header
%%%%% タイトルと作者 ここから %%%%%
\begin{minipage}[c]{0.7\hsize} % タイトルは上から 7 割
    \begin{center}
    % begin title
        {\LARGE
            流るる光途 % タイトルを入れる
        }
        {\small 
            (大正七年桜星会歌) % 年などを入れる
        }
    % end title
    \end{center}
\end{minipage}
\begin{minipage}[c]{0.3\hsize} % 作歌作曲は上から 3 割
    \begin{flushright} % 下寄せにする
        % begin name
         % 作歌・作曲者
        % end name
    \end{flushright}
\end{minipage}
%%%%% タイトルと作者 ここまで %%%%%
% % end header

% begin length
\vspace{1.5em} % タイトル, 作者と歌詞の間に隙間を設ける
\newcommand{\linespace}{0.5em} % 行間の設定
\newcommand{\blocksize}{0.25\hsize} % 段組間の設定
\newcommand{\itemmargin}{3em} % 曲番の位置調整の長さ
% end length
% begin body
%%%%% 歌詞 ここから %%%%%
\begin{enumerate} % 番号の箇条書き ここから
    \setlength{\itemindent}{\itemmargin} % 曲番の位置調整
    \begin{minipage}[c]{\blocksize}
    
        \vspace{\linespace}
        \item~\\
        % 1.
        \ruby{流}{なが}るヽ\ruby[g]{光途}{ひかり}\ruby{重}{かさ}ね\ruby{來}{き}て\\
        \ruby{星霜}{せい|そう}\ruby[g]{此處}{ここ}に\ruby{四十年}{し|じゅう|ねん}\\
        \ruby{北斗}{ほく|と}の\ruby[g]{光眸}{ひかり}さす\ruby{所}{ところ}\\
        \ruby{櫻}{さくら}かざして\ruby{先人}{せん|じん}の\\
        \ruby[g]{樹立}{たて}し\ruby{歴史}{れき|し}を\ruby{偲}{しの}ぶ\ruby{時}{とき}\\
        \ruby{誰}{だれ}か\ruby{血汐}{ち|しお}の\ruby{湧}{わ}かざらむ
        
        \vspace{\linespace}
        \item~\\
        % 2.
        \ruby{咽}{むせ}ぶ\ruby{悲憤}{ひ|ふん}の\ruby{誓}{ちかひ}より\\
        \ruby{早}{は}や\ruby{七年}{なな|とせ}の\ruby{春}{はる}うつり\\
        \ruby{人}{ひと}は\ruby[g]{変遷}{かわ}れど\ruby{三百}{さん|びゃく}の\\
        \ruby{健兒}{けん|じ}\ruby{不滅}{ふ|めつ}の\ruby{意氣}{い|き}を\ruby{持}{じ}す\\
        いでや\ruby{謳}{うた}はん\ruby{北州}{ほく|しゅう}の\\
        \ruby[g]{精力}{ちから}に\ruby{滿}{み}ちし\ruby[g]{凱歌}{かちうた}を
        
    \end{minipage}
    \begin{minipage}[c]{\blocksize}
        
        \vspace{\linespace}
        \item~\\
        % 3.
        \ruby[g]{陽春}{はる}の\ruby{光}{ひかり}に\ruby[g]{覆翼}{はぐく}まれ\\
        \ruby{嫩草}{わか|くさ}\ruby{萠}{も}ゆる\ruby{北}{きた}の\ruby{郷}{さと}\\
        \ruby{手稲}{て|いね}の\ruby{麓}{ふもと}\ruby{健兒}{けん|じ}\ruby{等}{ら}が\\
        \ruby{燃}{も}ゆる\ruby{想}{おもひ}を\ruby[g]{合唱}{コール}せば\\
        \ruby{牧場}{まき|ば}の\ruby[g]{彼方}{かなた}\ruby[g]{際涯}{はて}しらず\\
        \ruby{高鳴}{たか|なり}たてヽ\ruby{響}{ひび}きゆく
        
        \vspace{\linespace}
        \item~\\
        % 4.
        \ruby{豊平川}{とよ|ひら|がわ}の\ruby{夏}{なつ}の\ruby{夜}{よ}や\\
        \ruby{玉兎}{ぎょく|と}の\ruby{踊}{おど}る\ruby{波}{なみ}の\ruby{上}{うえ}\\
        \ruby{自治}{じ|ち}の\ruby{流}{ながれ}の\ruby{悠久}{ゆう|きゅう}を\\
        \ruby{語}{かた}る\ruby{川邊}{かわ|べ}に\ruby{佇}{たたず}めば\\
        ありし\ruby[g]{往昔}{むかし}を\ruby[g]{追憶}{おも}へとや\\
        \ruby{古塔}{こ|とう}に\ruby{響}{ひび}く\ruby{時}{とき}の\ruby{音}{おと}
        
    \end{minipage}
    \begin{minipage}[c]{\blocksize}
        
        \vspace{\linespace}
        \item~\\
        % 5.
        こヽ\ruby{石狩}{いし|かり}の\ruby{大}{だい}\ruby{沃野}{よく|や}\\
        \ruby{静}{しず}けき\ruby{秋}{あき}のめぐり\ruby{來}{き}て\\
        \ruby{天}{てん}\ruby{紺青}{こん|じょう}の\ruby{色}{いろ}ふかく\\
        \ruby{地}{ち}は\ruby[g]{豊穣}{ゆたか}なる\ruby{平和境}{へい|わ|きょう}\\
        \ruby{人}{ひと}は\ruby{有情}{う|じゃう}の\ruby{美}{うつく}しき\\
        \ruby{自然}{し|ぜん}の\ruby{愛}{あい}に\ruby{狎}{な}るヽ\ruby{哉}{かな}
        
        \vspace{\linespace}
        \item~\\
        % 6.
        \ruby{萬里}{ばん|り}\ruby{茫々}{ぼう|ぼう}\ruby{雪}{ゆき}の\ruby{海}{うみ}\\
        \ruby{白龍}{はく|りゅう}\ruby{怒}{いか}り\ruby{風}{かぜ}\ruby{叫}{さけ}ぶ\\
        \ruby[g]{吹雪}{ふぶき}にさめし\ruby{暁}{あかつき}や\\
        \ruby{迷}{まよひ}の\ruby{雲}{くも}をおしひらき\\
        \ruby{常世}{とこ|よ}の\ruby{幸}{さち}を\ruby{惠}{もと}むなる\\
        おヽ\ruby{紅}{くれなゐ}の\ruby{朝日影}{あさ|ひ|かげ}
        
    \end{minipage}
    \begin{minipage}[c]{\blocksize}
        
        \vspace{\linespace}
        \item~\\
        % 7.
        \ruby{北辰}{ほく|しん}\ruby{冴}{さ}ゆる\ruby{夕}{ゆう}まぐれ\\
        ボーイズ ビイ\\ アンビシァスの\\
        \ruby[g]{崇高}{たか}き\ruby{教}{おしへ}を\ruby{胸}{むね}に\ruby{秘}{ひ}め\\
        エルムの\ruby{梢}{こずえ}とことはの\\
        \ruby{自由}{じ|ゆう}の\ruby{調}{しらべ}\ruby{聽}{き}くところ\\
        \ruby{若}{わか}き\ruby[g]{生命}{いのち}を\ruby{誇}{ほこ}らばや
    
    \end{minipage}
\end{enumerate} % 番号の箇条書き ここまで
%%%%% 歌詞 ここまで %%%%%
% end body

\end{document}

\documentclass[10pt,b5j]{tarticle} % B6 縦書き
% \documentclass[10pt,b5j]{tarticle} % B6 縦書き
\AtBeginDvi{\special{papersize=128mm,182mm}} % B6 用用紙サイズ
\usepackage{otf} % Unicode で字を入力するのに必要なパッケージ
\usepackage[size=b6j]{bxpapersize} % B6 用紙サイズを指定
\usepackage[dvipdfmx]{graphicx} % 画像を挿入するためのパッケージ
\usepackage[dvipdfmx]{color} % 色をつけるためのパッケージ
\usepackage{pxrubrica} % ルビを振るためのパッケージ
\usepackage{multicol} % 複数段組を作るためのパッケージ
\setlength{\topmargin}{14mm} % 上下方向のマージン
\addtolength{\topmargin}{-1in} % 
\setlength{\oddsidemargin}{11mm} % 左右方向のマージン
\addtolength{\oddsidemargin}{-1in} % 
\setlength{\textwidth}{154mm} % B6 用
\setlength{\textheight}{108mm} % B6 用
\setlength{\headsep}{0mm} % 
\setlength{\headheight}{0mm} % 
\setlength{\topskip}{0mm} % 
\setlength{\parskip}{0pt} % 
\def\labelenumi{\theenumi、} % 箇条書きのフォーマット
\parindent = 0pt % 段落下げしない

 % B6 用テンプレート読み込み

\begin{document}
% begin header
%%%%% タイトルと作者 ここから %%%%%
\begin{minipage}[c]{0.7\hsize} % タイトルは上から 7 割
    \begin{center}
    % begin title
        {\LARGE
            タンネの氷柱 % タイトルを入れる
        }
        {\small 
            (昭和八年寮歌) % 年などを入れる
        }
    % end title
    \end{center}
\end{minipage}
\begin{minipage}[c]{0.3\hsize} % 作歌作曲は上から 3 割
    \begin{flushright} % 下寄せにする
        % begin name
        卜部清君 作歌\\白石祐義君 作曲 % 作歌・作曲者
        % end name
    \end{flushright}
\end{minipage}
%%%%% タイトルと作者 ここまで %%%%%
% (1,2,3,4,5 了あり)
% end header

% begin length
\vspace{1.5em} % タイトル, 作者と歌詞の間に隙間を設ける
\newcommand{\linespace}{0.5em} % 行間の設定
\newcommand{\blocksize}{0.33\hsize} % 段組間の設定
\newcommand{\itemmargin}{3em} % 曲番の位置調整の長さ
% end length
% begin body
%%%%% 歌詞 ここから %%%%%
\begin{enumerate} % 番号の箇条書き ここから
    \setlength{\itemindent}{\itemmargin} % 曲番の位置調整
    \begin{minipage}[c]{\blocksize}
    
        \vspace{\linespace}
        \item~\\
        % 1.
        タンネの\ruby[g]{氷柱}{つらら}\ruby{消}{き}ゆる\ruby{頃}{ころ}\\
        \ruby{胡蝶}{こ|ちょう}は\ruby{眠}{ねむ}る\ruby{花}{はな}の\ruby{宿}{やど}\\
        \ruby{牧場}{まき|ば}に\ruby{結}{むす}ぶ\ruby{夢}{ゆめ}\ruby{遙}{はる}か\\
        \ruby{青}{あお}き\ruby[g]{希望}{のぞみ}の\ruby[g]{雪峯}{みね}こえて\\
        \ruby[g]{四海}{しかい}に\ruby{羽振}{は|ぶ}る\ruby{若鵬}{わか|どり}の\\
        \ruby{石狩}{いし|かり}を\ruby{立}{た}つ\ruby{意気}{い|き}をみん
        
        \vspace{\linespace}
        \item~\\
        % 2.
        \ruby{朝里}{あさ|り}の\ruby{丘}{おか}に\ruby{烏頭}{う|づ}\ruby{咲}{さ}けば\\
        \ruby{蝦夷}{え|ぞ}が\ruby{芙蓉}{ふ|よう}の\ruby{雪}{ゆき}とけて\\
        \ruby{千尋}{ち|ひろ}の\ruby[g]{懸崖}{きし}ゆくだけ\ruby{入}{い}る\\
        \ruby[g]{忍路}{おしょろ}の\ruby{沖}{おき}の\ruby{真白帆}{ま|しら|ほ}に\\
        \ruby{万里}{ばん|り}の\ruby{波濤}{は|とう}\ruby{翔}{かけ}らんと\\
        \ruby[g]{白{\CID{7646}}}{かもめ}はしばし\ruby{憩}{いこ}ふなり % CID 鷗
        
    \end{minipage}
    \begin{minipage}[c]{\blocksize}
        
        \vspace{\linespace}
        \item~\\
        % 3.
        \ruby{真紅}{しん|く}の\ruby{夕陽}{ゆう|ひ}\ruby{山}{やま}の\ruby{端}{は}に\\
        もゆる\ruby[g]{紅葉}{もみぢ}をかざしたる\\
        \ruby{友}{とも}がゆくての\ruby{野}{の}を\ruby{遠}{とほ}く\\
        \ruby{幌馬車}{ほろ|ば|しゃ}の\ruby{影}{かげ}\ruby{消}{き}え\ruby{去}{さ}りぬ\\
        \ruby{蓬髪}{ほう|はつ}\ruby[g]{胡風}{かぜ}に\ruby{靡}{なび}けつつ\\
        \ruby[g]{懐情}{おもひ}は\ruby{尽}{つ}きず\ruby{果}{は}てもなく
        
        \vspace{\linespace}
        \item~\\
        % 4.
        \ruby{十勝}{と|かち}の\ruby{峰}{みね}に\ruby{捲}{ま}き\ruby{起}{お}こる\\
        \ruby[g]{吹雪}{ふぶき}\ruby{怒}{いか}りて\ruby{咆}{ほ}ゆる\ruby{夜}{よ}も\\
        \ruby{旭光}{ぎょっ|こう}\ruby{東}{ひがし}に\ruby{色}{いろ}めけば\\
        \ruby{熊}{くま}\ruby{追}{お}ふ\ruby[g]{愛奴}{あいぬ}の\ruby{雄叫}{お|たけ}びに\\
        \ruby{大}{だい}\ruby{雪原}{せつ|げん}の\ruby{霊光}{れい|こう}や\\
        \ruby[g]{無絃琴}{つるなしごと}の\ruby{音}{ね}ぞ\ruby{高}{たか}し
        
    \end{minipage}
    \begin{minipage}[c]{\blocksize}
        
        \vspace{\linespace}
        \item~\\
        % 5.
        \ruby{懸}{かか}る\ruby{垂氷}{たる|ひ}に\ruby{月}{つき}くだけ\\
        \ruby{千々}{ち|ぢ}の\ruby[g]{瞑想}{おもひ}は\ruby{来}{こ}し\ruby{方}{かた}の\\
        \ruby[g]{六十}{むそぢ}の\ruby{秋}{あき}はしるくして\\
        \ruby{緑}{みどり}に\ruby{浮}{うか}ぶ\ruby{白亜城}{はく|あ|じょう}\\
        \ruby{苔}{こけ}むす\ruby[g]{楡鐘}{かね}の\ruby[g]{哀調}{しらべ}きけ\\
        \ruby{若}{わか}き\ruby{力}{ちから}を\ruby{求}{もと}むなり
    
    \end{minipage}
\end{enumerate} % 番号の箇条書き ここまで
%%%%% 歌詞 ここまで %%%%%
% end body

\end{document}

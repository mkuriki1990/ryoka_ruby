\documentclass[10pt,b5j]{tarticle} % B6 縦書き
% \documentclass[10pt,b5j]{tarticle} % B6 縦書き
\AtBeginDvi{\special{papersize=128mm,182mm}} % B6 用用紙サイズ
\usepackage{otf} % Unicode で字を入力するのに必要なパッケージ
\usepackage[size=b6j]{bxpapersize} % B6 用紙サイズを指定
\usepackage[dvipdfmx]{graphicx} % 画像を挿入するためのパッケージ
\usepackage[dvipdfmx]{color} % 色をつけるためのパッケージ
\usepackage{pxrubrica} % ルビを振るためのパッケージ
\usepackage{multicol} % 複数段組を作るためのパッケージ
\setlength{\topmargin}{14mm} % 上下方向のマージン
\addtolength{\topmargin}{-1in} % 
\setlength{\oddsidemargin}{11mm} % 左右方向のマージン
\addtolength{\oddsidemargin}{-1in} % 
\setlength{\textwidth}{154mm} % B6 用
\setlength{\textheight}{108mm} % B6 用
\setlength{\headsep}{0mm} % 
\setlength{\headheight}{0mm} % 
\setlength{\topskip}{0mm} % 
\setlength{\parskip}{0pt} % 
\def\labelenumi{\theenumi、} % 箇条書きのフォーマット
\parindent = 0pt % 段落下げしない

 % B6 用テンプレート読み込み

\begin{document}
% begin header
%%%%% タイトルと作者 ここから %%%%%
\begin{minipage}[c]{0.7\hsize} % タイトルは上から 7 割
    \begin{center}
    % begin title
        {\LARGE
            新らたなり天地 % タイトルを入れる
        }
        {\small 
            (昭和二十六年寮歌) % 年などを入れる
        }
    % end title
    \end{center}
\end{minipage}
\begin{minipage}[c]{0.3\hsize} % 作歌作曲は上から 3 割
    \begin{flushright} % 下寄せにする
        % begin name
        長尾久司君 作歌\\小林滋宗君 作曲 % 作歌・作曲者
        % end name
    \end{flushright}
\end{minipage}
%%%%% タイトルと作者 ここまで %%%%%
% (1,2,3 繰り返しなし)
% end header

% begin length
\vspace{1.5em} % タイトル, 作者と歌詞の間に隙間を設ける
\newcommand{\linespace}{0.5em} % 行間の設定
\newcommand{\blocksize}{0.33\hsize} % 段組間の設定
\newcommand{\itemmargin}{3em} % 曲番の位置調整の長さ
% end length
% begin body
%%%%% 歌詞 ここから %%%%%
\begin{enumerate} % 番号の箇条書き ここから
    \setlength{\itemindent}{\itemmargin} % 曲番の位置調整
    \begin{minipage}[c]{\blocksize}
    
        \vspace{\linespace}
        \item~\\
        % 1.
        \ruby{新}{あ}らたなり\ruby{天地}{あめ|つち}\\
        \ruby{光}{ひかり}あり\ruby{北}{きた}の\ruby[g]{学舎}{まなびや}\\
        \ruby[g]{二年}{ふたとせ}を\ruby{心}{こころ}に\ruby{契}{ちぎ}る\\
        \ruby{若}{わか}き\ruby{日}{ひ}の\ruby[g]{生命}{いのち}の\ruby{郷}{さと}に\\
        \ruby{誇}{ほこ}らなん\ruby{自治}{じ|ち}と\ruby{自由}{じ|ゆう}の\\
        \ruby[g]{四十星霜}{よそとせ}の\ruby{高}{たか}き\ruby[g]{伝統}{つたえ}よ\\
        おごそかに\ruby[g]{遺訓}{おしえ}をこめて\\
        \ruby[g]{楡鐘}{かね}は\ruby{響}{ひび}かん
        
    \end{minipage}
    \begin{minipage}[c]{\blocksize}
        
        \vspace{\linespace}
        \item~\\
        % 2.
        \ruby[g]{雄大}{おお}いなり\ruby{天地}{あめ|つち}\\
        \ruby{永劫}{えい|ごう}の\ruby[g]{時潮}{とき}の\ruby{流}{なが}れよ\\
        \ruby{悠久}{ゆう|きゅう}の\ruby[g]{神秘}{くしび}をひめし\\
        うるわしき\ruby{石狩}{いし|かり}の\ruby{野}{の}に\\
        うたわなん\ruby[g]{希望}{のぞみ}のうたを\\
        \ruby{魂}{たま}ゆする\ruby{雄叫}{お|たけ}びの\ruby{日}{ひ}に\\
        あこがれと\ruby{正義}{せい|ぎ}の\ruby{旗}{はた}を\\
        かざし\ruby{進}{すす}まん
        
    \end{minipage}
    \begin{minipage}[c]{\blocksize}
        
        \vspace{\linespace}
        \item~\\
        % 3.
        きびしかる\ruby{天地}{あめ|つち}\\
        \ruby{野}{の}にすさぶ\ruby{試練}{し|れん}の\ruby{嵐}{あらし}\\
        \ruby{苦}{くる}しみを\ruby{越}{こ}えて\ruby{幸}{さち}あり\\
        たゆみなく\ruby{求}{もと}めて\ruby{得}{え}たり\\
        \ruby{輝}{かがや}ける\ruby[g]{久遠}{くおん}の\ruby[g]{真理}{まこと}\\
        よろこびの\ruby{若}{わか}き\ruby{力}{ちから}に\\
        \ruby[g]{創造}{きず}き\ruby{行}{ゆ}く\ruby{恵迪}{けい|てき}の\ruby{寮}{りょう}\\
        とわに\ruby{栄}{さか}えん
    
    \end{minipage}
\end{enumerate} % 番号の箇条書き ここまで
%%%%% 歌詞 ここまで %%%%%
% end body

\end{document}

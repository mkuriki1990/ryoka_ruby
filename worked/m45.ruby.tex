\documentclass[10pt,b5j]{tarticle} % B6 縦書き
% \documentclass[10pt,b5j]{tarticle} % B6 縦書き
\AtBeginDvi{\special{papersize=128mm,182mm}} % B6 用用紙サイズ
\usepackage{otf} % Unicode で字を入力するのに必要なパッケージ
\usepackage[size=b6j]{bxpapersize} % B6 用紙サイズを指定
\usepackage[dvipdfmx]{graphicx} % 画像を挿入するためのパッケージ
\usepackage[dvipdfmx]{color} % 色をつけるためのパッケージ
\usepackage{pxrubrica} % ルビを振るためのパッケージ
\usepackage{multicol} % 複数段組を作るためのパッケージ
\setlength{\topmargin}{14mm} % 上下方向のマージン
\addtolength{\topmargin}{-1in} % 
\setlength{\oddsidemargin}{11mm} % 左右方向のマージン
\addtolength{\oddsidemargin}{-1in} % 
\setlength{\textwidth}{154mm} % B6 用
\setlength{\textheight}{108mm} % B6 用
\setlength{\headsep}{0mm} % 
\setlength{\headheight}{0mm} % 
\setlength{\topskip}{0mm} % 
\setlength{\parskip}{0pt} % 
\def\labelenumi{\theenumi、} % 箇条書きのフォーマット
\parindent = 0pt % 段落下げしない

 % B6 用テンプレート読み込み

\begin{document}
% begin header
%%%%% タイトルと作者 ここから %%%%%
\begin{minipage}[c]{0.7\hsize} % タイトルは上から 7 割
    \begin{center}
    % begin title
        {\LARGE
            都ぞ弥生 % タイトルを入れる
        }
        {\small 
            (明治45年寮歌) % 年などを入れる
        }
    % end title
    \end{center}
\end{minipage}
\begin{minipage}[c]{0.3\hsize} % 作歌作曲は上から 3 割
    \begin{flushright} % 下寄せにする
        % begin name
        横山芳介君 作歌\\赤木顕次君 作曲 % 作歌・作曲者
        % end name
    \end{flushright}
\end{minipage}
%%%%% タイトルと作者 ここまで %%%%%
% (1 了あり)
% end header

% begin body
\vspace{1.5em} % タイトル, 作者と歌詞の間に隙間を設ける
\newcommand{\linespace}{0.5em} % 行間の設定
\newcommand{\blocksize}{0.5\hsize} % 段組間の設定
%%%%% 歌詞 ここから %%%%%
% begin lilycs
\begin{enumerate} % 番号の箇条書き ここから
    \begin{minipage}[c]{\blocksize}
    
        \vspace{\linespace}
        \item
        % 1.
        \ruby{都}{}ぞ\ruby{弥生}{}の\ruby{雲紫}{}に\\
        \ruby{花}{}の\ruby{香漂}{}ふ\ruby{宴遊}{}の\ruby{筵}{}\\
        \ruby{尽}{}きせぬ\ruby{奢}{}に\ruby{濃}{}き\ruby{紅}{}や\\
        その\ruby{春暮}{}れては\ruby{移}{}らふ\ruby{色}{}の\\
        \ruby{夢}{}こそ\ruby{一時青}{}き\ruby{繁}{}みに\\
        \ruby{燃}{}えなん\ruby{我胸想}{}ひを\ruby{載}{}せて\\
        \ruby{星影冴}{}かに\ruby{光}{}れる\ruby{北}{}を\\
        \ruby{人}{}の\ruby{世}{}の\ruby{清}{}き\ruby{国}{}ぞとあこがれぬ
        
        \vspace{\linespace}
        \item
        % 2.
        \ruby{豊}{}かに\ruby{稔}{}れる\ruby{石狩}{}の\ruby{野}{}に\\
        \ruby{雁遙々沈}{}みてゆけば\\
        \ruby{羊群声}{}なく\ruby{牧舎}{}に\ruby{帰}{}り\\
        \ruby{手稲}{}の\ruby{嶺黄昏}{}こめぬ\\
        \ruby{雄々}{}しく\ruby{聳}{}ゆる\ruby{楡}{}の\ruby{梢}{}\\
        \ruby{打振}{}る\ruby{野分}{}に\ruby{破壊}{}の\ruby{葉音}{}の\\
        さやめく\ruby{甍}{}に\ruby{久遠}{}の\ruby{光}{}り\\
        おごそかに\ruby{北極星}{}を\ruby{仰}{}ぐ\ruby{哉}{}
        
        \vspace{\linespace}
        \item
        % 3.
        \ruby{寒月懸}{}れる\ruby{針葉樹林}{}\\
        \ruby{橇}{}の\ruby{音凍}{}りて\ruby{物皆寒}{}く\\
        \ruby{野}{}もせに\ruby{乱}{}るる\ruby{清白}{}の\ruby{雪}{}\\
        \ruby{沈黙}{}の\ruby{暁霏々}{}として\ruby{舞}{}ふ\\
        ああその\ruby{朔風飆々}{}として\\
        \ruby{荒}{}ぶる\ruby{吹雪}{}の\ruby{逆巻}{}くを\ruby{見}{}よ\\
        ああその\ruby{蒼空梢聯}{}ねて\\
        \ruby{樹氷咲}{}く\ruby{壮麗}{}の\ruby{地}{}をここに\ruby{見}{}よ
        
        \vspace{\linespace}
        \item
        % 4.
        \ruby{牧場}{}の\ruby{若草陽炎燃}{}えて\\
        \ruby{森}{}には\ruby{桂}{}の\ruby{新緑萠}{}し\\
        \ruby{雲}{}ゆく\ruby{雲雀}{}に\ruby{延齢草}{}の\\
        \ruby{真白}{}の\ruby{花影}{}さゆらぎて\ruby{立}{}つ\\
        \ruby{今}{}こそ\ruby{溢}{}れぬ\ruby{清和}{}の\ruby{陽光}{}\\
        \ruby{小河}{}の\ruby{潯}{}をさまよひゆけば\\
        うつくしからずや\ruby{咲}{}く\ruby{水芭蕉}{}\\
        \ruby{春}{}の\ruby{日}{}のこの\ruby{北}{}の\ruby{国幸多}{}し
        
        \vspace{\linespace}
        \item
        % 5.
        \ruby{朝雲流}{}れて\ruby{金色}{}に\ruby{照}{}り\\
        \ruby{平原果}{}てなき\ruby{東}{}の\ruby{際}{}\\
        \ruby{連}{}なる\ruby{山脈玲瓏}{}として\\
        \ruby{今}{}しも\ruby{輝}{}く\ruby{紫紺}{}の\ruby{雪}{}に\\
        \ruby{自然}{}の\ruby{藝術}{}を\ruby{懐}{}みつつ\\
        \ruby{高鳴}{}る\ruby{血潮}{}のほとばしりもて\\
        \ruby{貴}{}とき\ruby{野心}{}の\ruby{訓}{}へ\ruby{培}{}い\\
        \ruby{栄}{}え\ruby{行}{}く\ruby{我等}{}が\ruby{寮}{}を\ruby{誇}{}らずや 
    
    \end{minipage}
\end{enumerate} % 番号の箇条書き ここまで
% end lilycs
%%%%% 歌詞 ここまで %%%%%
% end body

\end{document}

\documentclass[10pt,b5j]{tarticle} % B6 縦書き
% \documentclass[10pt,b5j]{tarticle} % B6 縦書き
\AtBeginDvi{\special{papersize=128mm,182mm}} % B6 用用紙サイズ
\usepackage{otf} % Unicode で字を入力するのに必要なパッケージ
\usepackage[size=b6j]{bxpapersize} % B6 用紙サイズを指定
\usepackage[dvipdfmx]{graphicx} % 画像を挿入するためのパッケージ
\usepackage[dvipdfmx]{color} % 色をつけるためのパッケージ
\usepackage{pxrubrica} % ルビを振るためのパッケージ
\usepackage{multicol} % 複数段組を作るためのパッケージ
\setlength{\topmargin}{14mm} % 上下方向のマージン
\addtolength{\topmargin}{-1in} % 
\setlength{\oddsidemargin}{11mm} % 左右方向のマージン
\addtolength{\oddsidemargin}{-1in} % 
\setlength{\textwidth}{154mm} % B6 用
\setlength{\textheight}{108mm} % B6 用
\setlength{\headsep}{0mm} % 
\setlength{\headheight}{0mm} % 
\setlength{\topskip}{0mm} % 
\setlength{\parskip}{0pt} % 
\def\labelenumi{\theenumi、} % 箇条書きのフォーマット
\parindent = 0pt % 段落下げしない

 % B6 用テンプレート読み込み

\begin{document}
% begin header
%%%%% タイトルと作者 ここから %%%%%
\begin{minipage}[c]{0.7\hsize} % タイトルは上から 7 割
    \begin{center}
    % begin title
        {\LARGE
            楡陵に月は % タイトルを入れる
        }
        {\small 
            (昭和四十七年寮歌) % 年などを入れる
        }
    % end title
    \end{center}
\end{minipage}
\begin{minipage}[c]{0.3\hsize} % 作歌作曲は上から 3 割
    \begin{flushright} % 下寄せにする
        % begin name
        加藤秀弘君 作歌\\矢野哲憲君 作曲 % 作歌・作曲者
        % end name
    \end{flushright}
\end{minipage}
%%%%% タイトルと作者 ここまで %%%%%
% (1,6 了あり)
% end header

% begin length
\vspace{1.5em} % タイトル, 作者と歌詞の間に隙間を設ける
\newcommand{\linespace}{0.5em} % 行間の設定
\newcommand{\blocksize}{0.33\hsize} % 段組間の設定
\newcommand{\itemmargin}{3em} % 曲番の位置調整の長さ
% end length
% begin body
%%%%% 歌詞 ここから %%%%%
\begin{enumerate} % 番号の箇条書き ここから
    \setlength{\itemindent}{\itemmargin} % 曲番の位置調整
    \begin{minipage}[c]{\blocksize}
    
        \vspace{\linespace}
        \item~\\
        % 1.
        \ruby{楡陵}{ゆ|りょう}に\ruby{月}{つき}は\ruby{懸}{かか}れども\\
        \ruby{星霜}{せい|そう}\ruby{深}{ふか}き\ruby[g]{原始林}{もり}\ruby{暗}{くら}し\\
        \ruby{蓁萋}{しん|せい}ゆらぐ\ruby{風}{かぜ}\ruby{有}{あ}れど\\
        \ruby{思}{おも}い\ruby{分}{わ}かたん\ruby{術}{すべ}も\ruby{無}{な}し
        
        \vspace{\linespace}
        \item~\\
        % 2.
        \ruby{天空}{てん|くう}\ruby{破}{やぶ}る\ruby[g]{落雷}{ひ}はあれど\\
        そびゆる\ruby[g]{聳天樹}{ポプラ}は\ruby{堂々}{どう|どう}と\\
        \ruby{慟哭}{どう|こく}の\ruby{声}{こえ}\ruby{上}{あ}げらんと\\
        \ruby{意気}{い|き}\ruby{揺籃}{よう|らん}の\ruby{時}{とき}は\ruby{今}{いま}
        
    \end{minipage}
    \begin{minipage}[c]{\blocksize}
        
        \vspace{\linespace}
        \item~\\
        % 3.
        \ruby{銀晶}{ぎん|しょう}ふるう\ruby[g]{雪原}{の}なれども\\
        \ruby{変}{かわ}らぬ\ruby[g]{沈黙}{しじま}\ruby{奇}{く}しきかな\\
        \ruby{黄鶴}{おう|かく}\ruby{消}{き}えて\ruby{姿}{すがた}\ruby{無}{な}し\\
        \ruby{蘇}{よみが}える\ruby{春}{はる}まだ\ruby{遠}{とお}く
        
        \vspace{\linespace}
        \item~\\
        % 4.
        \ruby{鐘}{かね}の\ruby{音}{ね}\ruby{遠}{とお}く\ruby{聞}{きこ}えども\\
        \ruby{雑踏}{ざっ|とう}の\ruby{声}{こえ}さざめきの\\
        \ruby[g]{辛夷}{こぶし}\ruby{花}{はな}\ruby{咲}{さ}く\ruby{黎明}{れい|めい}よ\\
        \ruby{石狩}{いし|かり}の\ruby{野}{の}\ruby{今}{いま}\ruby[g]{何処}{いずこ}
        
    \end{minipage}
    \begin{minipage}[c]{\blocksize}
        
        \vspace{\linespace}
        \item~\\
        % 5.
        \ruby{無尽}{む|じん}の\ruby{星}{ほし}を\ruby{仰}{あお}げども\\
        \ruby{天}{てん}に\ruby{無双}{む|そう}の\ruby{北斗星}{ほく|と|ほし}\\
        \ruby{白亜}{はく|あ}の\ruby{城}{しろ}に\ruby{覚醒}{かく|せい}し\\
        \ruby[g]{永遠}{とわ}の\ruby[g]{生命}{いのち}を\ruby{誦}{うた}わなん
        
        \vspace{\linespace}
        \item~\\
        % 6.
        \ruby{未明}{み|めい}に\ruby{懸}{かか}る\ruby{白}{しろ}き\ruby{月}{つき}\\
        \ruby{夢見}{ゆめ|み}し\ruby{思}{おも}う\ruby[g]{北溟}{きた}の\ruby{海}{うみ}\\
        \ruby{憧}{あこが}れ\ruby{来}{き}しは\ruby[g]{北溟}{きた}の\ruby{峰}{みね}\\
        \ruby{呼々}{あ|あ}\ruby{我}{われ}\ruby[g]{前途}{みち}の\ruby{行}{ゆ}く\ruby{果}{はて}は
    
    \end{minipage}
\end{enumerate} % 番号の箇条書き ここまで
%%%%% 歌詞 ここまで %%%%%
% end body

\end{document}

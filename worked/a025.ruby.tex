\documentclass[10pt,b5j]{tarticle} % B6 縦書き
% \documentclass[10pt,b5j]{tarticle} % B6 縦書き
\AtBeginDvi{\special{papersize=128mm,182mm}} % B6 用用紙サイズ
\usepackage{otf} % Unicode で字を入力するのに必要なパッケージ
\usepackage[size=b6j]{bxpapersize} % B6 用紙サイズを指定
\usepackage[dvipdfmx]{graphicx} % 画像を挿入するためのパッケージ
\usepackage[dvipdfmx]{color} % 色をつけるためのパッケージ
\usepackage{pxrubrica} % ルビを振るためのパッケージ
\usepackage{multicol} % 複数段組を作るためのパッケージ
\setlength{\topmargin}{14mm} % 上下方向のマージン
\addtolength{\topmargin}{-1in} % 
\setlength{\oddsidemargin}{11mm} % 左右方向のマージン
\addtolength{\oddsidemargin}{-1in} % 
\setlength{\textwidth}{154mm} % B6 用
\setlength{\textheight}{108mm} % B6 用
\setlength{\headsep}{0mm} % 
\setlength{\headheight}{0mm} % 
\setlength{\topskip}{0mm} % 
\setlength{\parskip}{0pt} % 
\def\labelenumi{\theenumi、} % 箇条書きのフォーマット
\parindent = 0pt % 段落下げしない

 % B6 用テンプレート読み込み

\begin{document}
% begin header
%%%%% タイトルと作者 ここから %%%%%
\begin{minipage}[c]{0.7\hsize} % タイトルは上から 7 割
    \begin{center}
    % begin title
        {\LARGE
            山岳部部歌 % タイトルを入れる
        }
        {\large 
            \rule[0.0em]{1.0em}{0.05em} 山の四季 \rule[0.0em]{1.0em}{0.05em}
        }
        {\small 
            (昭和十四年頃) % 年などを入れる
        }
    % end title
    \end{center}
\end{minipage}
\begin{minipage}[c]{0.3\hsize} % 作歌作曲は上から 3 割
    \begin{flushright} % 下寄せにする
        % begin name
        朝比奈英三君 作歌\\渡辺良一君 作曲 % 作歌・作曲者
        % end name
    \end{flushright}
\end{minipage}
%%%%% タイトルと作者 ここまで %%%%%
% % end header

% begin length
\vspace{1.0em} % タイトル, 作者と歌詞の間に隙間を設ける
\newcommand{\linespace}{0.3em} % 行間の設定
\newcommand{\blocksize}{0.5\hsize} % 段組間の設定
\newcommand{\itemmargin}{3em} % 曲番の位置調整の長さ
% end length
% begin body
%%%%% 歌詞 ここから %%%%%
\begin{enumerate} % 番号の箇条書き ここから
    \setlength{\itemindent}{\itemmargin} % 曲番の位置調整
    \begin{minipage}[c]{\blocksize}
    
        \vspace{\linespace}
        \item~\\
        % 1.
        ふぶきの\ruby{尾根}{お|ね}も \ruby{風}{かぜ}\ruby{止}{や}みて\\
        \ruby{春}{はる}の\ruby{日}{ひ}ざしの おとずれに\\
        \ruby{沢}{さわ}のなだれも \ruby{静}{しず}まりて\\
        \ruby{雪}{ゆき}げの\ruby{沢}{さわ}の \ruby{歌}{うた}\ruby{楽}{たの}し\\
        いざ\ruby{行}{ゆ}こう \ruby{我}{わ}が\ruby{友}{とも}よ\\
        \ruby{暑寒}{しょ|かん}の\ruby{尾根}{お|ね}に \ruby{芦別}{あし|べつ}に\\
        \ruby{北}{きた}の\ruby{山}{やま}の
        ざらめの\ruby{尾根}{お|ね}を\ruby{飛}{と}ばそうよ
        
        \vspace{\linespace}
        \item~\\
        % 2.
        \ruby{沢}{さわ}を\ruby{登}{のぼ}りて いま\ruby[g]{五日}{いつか}\\
        ワラジも\ruby{足}{あし}に \ruby{親}{した}しみぬ\\
        \ruby[g]{三日}{みっか}\ruby{三晩}{み|ばん}の \ruby{籠城}{ろう|じょう}も\\
        \ruby{過}{す}ぎて\ruby{楽}{たの}しき \ruby{思}{おも}い\ruby{出}{で}よ\\
        いざ\ruby{行}{ゆ}こう \ruby{我}{わ}が\ruby{友}{とも}よ\\
        \ruby{日高}{ひ|だか}の\ruby{山}{やま}に \ruby{夏}{なつ}の\ruby{旅}{たび}に\\
        \ruby{北}{きた}の\ruby{山}{やま}の
        カールの\ruby{中}{なか}に\ruby{眠}{ねむ}ろうよ
        
    \end{minipage}
    \begin{minipage}[c]{\blocksize}
        
        \vspace{\linespace}
        \item~\\
        % 3.
        \ruby{山}{やま}は\ruby[g]{紅葉}{もみじ}に \ruby{色}{いろ}どられ\\
        \ruby{頂}{いただき}\ruby{高}{たか}く \ruby{空}{そら}\ruby{澄}{す}みぬ\\
        \ruby{新雪}{しん|せつ}\ruby{輝}{かがや}く \ruby{山山}{やま|やま}は\\
        いずれも\ruby{親}{した}しき \ruby{友}{とも}だちよ\\
        いざ\ruby{行}{ゆ}こう \ruby{我}{わ}が\ruby{友}{とも}よ\\
        ニセイカウシュペにトムラウシに\\
        \ruby{北}{きた}の\ruby{山}{やま}の
        \ruby{沢}{さわ}のたき\ruby{火}{び}に\ruby{語}{かた}ろうよ
        
        \vspace{\linespace}
        \item~\\
        % 4.
        \ruby[g]{吹雪}{ふぶき}も\ruby{止}{や}んだ \ruby{朝}{あさ}まだき\\
        \ruby{凍}{こお}ったテントを \ruby{起}{お}き\ruby{出}{い}でて\\
        はるかにのぞむ やせ\ruby{尾根}{お|ね}は\\
        \ruby{朝焼}{あさ|や}け\ruby{燃}{も}ゆる ペテガリだ\\
        いざ\ruby{行}{ゆ}こう \ruby{我}{わ}が\ruby{友}{とも}よ\\
        \ruby{氷}{こおり}の\ruby{尾根}{お|ね}に アンザイレン\\
        \ruby{北}{きた}の\ruby{山}{やま}の
        \ruby{聖}{きよ}き\ruby{頂}{いただき}\ruby{目指}{め|ざ}そうよ
    
    \end{minipage}
\end{enumerate} % 番号の箇条書き ここまで
%%%%% 歌詞 ここまで %%%%%
% end body

\end{document}

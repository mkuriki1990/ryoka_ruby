\documentclass[10pt,b5j]{tarticle} % B6 縦書き
% \documentclass[10pt,b5j]{tarticle} % B6 縦書き
\AtBeginDvi{\special{papersize=128mm,182mm}} % B6 用用紙サイズ
\usepackage{otf} % Unicode で字を入力するのに必要なパッケージ
\usepackage[size=b6j]{bxpapersize} % B6 用紙サイズを指定
\usepackage[dvipdfmx]{graphicx} % 画像を挿入するためのパッケージ
\usepackage[dvipdfmx]{color} % 色をつけるためのパッケージ
\usepackage{pxrubrica} % ルビを振るためのパッケージ
\usepackage{plext} % 漢数字の enumerate を使うためのパッケージ
\usepackage{multicol} % 複数段組を作るためのパッケージ
\setlength{\topmargin}{14mm} % 上下方向のマージン
\addtolength{\topmargin}{-1in} % 
\setlength{\oddsidemargin}{11mm} % 左右方向のマージン
\addtolength{\oddsidemargin}{-1in} % 
\setlength{\textwidth}{154mm} % B6 用
\setlength{\textheight}{108mm} % B6 用
\setlength{\headsep}{0mm} % 
\setlength{\headheight}{0mm} % 
\setlength{\topskip}{0mm} % 
\setlength{\parskip}{0pt} % 
\def\theenumi{\Kanji{enumi}} % 箇条書きのフォーマットを漢数字に変更
\parindent = 0pt % 段落下げしない
\pagestyle{empty} % すべてのページ番号を消去
% \renewcommand{\baselinestretch}{0.9} % 行間の倍率
 % B6 用テンプレート読み込み

\begin{document}
% begin header
%%%%% タイトルと作者 ここから %%%%%
\begin{minipage}[c]{0.7\hsize} % タイトルは上から 7 割
    \begin{center}
    % begin title
        {\LARGE
            姫月に重ねて % タイトルを入れる
        }
        {\small 
            (平成二十六年度寮歌) % 年などを入れる
        }
    % end title
    \end{center}
\end{minipage}
\begin{minipage}[c]{0.3\hsize} % 作歌作曲は上から 3 割
    \begin{flushright} % 下寄せにする
        % begin name
        松元一平君 作歌\\寺尾佳隆君 作曲 % 作歌・作曲者
        % end name
    \end{flushright}
\end{minipage}
%%%%% タイトルと作者 ここまで %%%%%
% (1,2,3 繰り返しなし)
% end header

% begin length
\vspace{1.5em} % タイトル, 作者と歌詞の間に隙間を設ける
\newcommand{\linespace}{0.5em} % 行間の設定
\newcommand{\blocksize}{0.5\hsize} % 段組間の設定
\newcommand{\itemmargin}{3em} % 曲番の位置調整の長さ
% end length
% begin body
%%%%% 歌詞 ここから %%%%%
\begin{enumerate} % 番号の箇条書き ここから
    \setlength{\itemindent}{\itemmargin} % 曲番の位置調整
    \begin{minipage}[c]{\blocksize}
    
        \vspace{\linespace}
        \item~\\
        \ruby{観月}{みづき}\ruby{過}{す}ぎゆく\ruby{晩秋}{ばんしゅう}の\ruby{夜}{よる}、\\
        \ruby{穹}{そら}\ruby{蒼}{あお}の\ruby{天空}{てんくう}\ruby{高}{たか}く\\
        \ruby{舞}{まい}ひたる\ruby{月}{つき}は\ruby{今宵}{こよい}\ruby{満}{み}つるかな。\\
        その\ruby{清輝}{きよてる}に\ruby{映}{は}えし\ruby{姫}{ひめ}が\ruby{鏡水}{かがみず}は、\\
        \ruby{鹿}{しか}が\ruby{純}{じゅん}\ruby{瞳}{ひとみ}に\ruby{宿}{やど}らむ。\\
        \ruby{月影}{つきかげ}は\ruby{鹿}{しか}を\ruby{誘}{さそ}ひ\\
        \ruby{来}{き}たりしこの\ruby{神無月}{かみなづき}に\\
        \ruby{何}{なに}をば\ruby{見}{み}せむ。
        
    \end{minipage}
    \begin{minipage}[c]{\blocksize}
        
        \vspace{\linespace}
        \item~\\
        % 1.
        \ruby{時}{とき}\ruby{移}{うつ}ろひて \ruby{人世}{じんせい}は\ruby{変}{か}われども\\
        \ruby{今宵}{こよい}も\ruby{満月}{まんげつ}は\ruby{我}{われ}らを\ruby{照}{て}さむ\\
        \ruby{夜}{よる}の\ruby{邪}{よこしま}\ruby{帳}{ちょう}をはらはむと\\
        \ruby{流}{ながれ}\ruby{歩}{あゆ}む\ruby{汝}{なんじ}は\ruby{楡}{にれ}に\ruby{似}{に}たれど\\
        \ruby{風流}{ふりゅう}を\ruby{掴}{つか}まむ\ruby{芽}{め}に\ruby{感}{かん}ず\\
        \ruby{風習}{ふうしゅう}に\ruby{付}{つけ}\ruby{和}{わ}せし\\
        \ruby{狗}{いぬ}と\ruby{成}{な}らざらめや\\
        さて\ruby{映}{うつ}りこむ \ruby{我}{わ}が\ruby{鏡}{かがみ}\ruby{瞳}{ひとみ}に\\
        \ruby{風習}{ふうしゅう}だに\ruby{愛}{あい}づる その\ruby{気概}{きがい}
        
    \end{minipage}
    \begin{minipage}[c]{\blocksize}
        
        \vspace{\linespace}
        \item~\\
        % 2.
        \ruby{清}{きよし}\ruby{澄}{す}みたる\ruby{想}{そう}ひ \ruby{知}{し}る\ruby{由}{よし}もなく\\
        \ruby{今宵}{こよい}の\ruby{三日月}{みかづき}は\ruby{川面}{かわも}に\ruby{映}{うつ}らむ\\
        かの\ruby{日}{ひ}の\ruby{月影}{つきかげ}とは\ruby{違}{ちがい}へども\\
        \ruby{人世}{じんせい}(よ)に\ruby{充}{み}つ\ruby{解答}{かいとう}を\ruby{自}{おの}ずと\ruby{心得}{こころえ}\\
        \ruby{此}{この}れは\ruby{汝}{なんじ}の\ruby{求}{もとめ}\ruby{望}{もち}にか\\
        \ruby{漲}{みなぎ}る\ruby{想}{そう}ひ などか\ruby{劣}{れつ}らむ\\
        さて\ruby{映}{うつ}りこむ \ruby{我}{わ}が\ruby{鏡}{かがみ}\ruby{瞳}{ひとみ}に\\
        \ruby{身}{み}を\ruby{委}{ゆだ}ねばや その\ruby{清流}{せいりゅう}
        
    \end{minipage}
    \begin{minipage}[c]{\blocksize}
        
        \vspace{\linespace}
        \item~\\
        % 3.
        \ruby{静}{せい}と\ruby{唸}{うな}りし \ruby{雨}{あめ}\ruby{澪}{みお}したたれば\\
        \ruby{今宵}{こよい}も\ruby{我}{が}は\ruby{朧月}{おぼろづき}を\ruby{仰}{ぎょう}がむ\\
        \ruby{姫}{ひめ}が\ruby{麗姿}{れいし}を\ruby{追憶}{ついおく}ふべく\\
        \ruby{汝}{なんじ}が\ruby{想}{そう}ひは\ruby{涙}{なみだ}と\ruby{落}{おち}\ruby{流}{なが}れ\\
        \ruby{透}{す}かし\ruby{斜光}{しゃこう}にさらさるる\\
        \ruby{閉}{と}じなむ\ruby{凌雲}{りょううん}よ こひ\ruby{願}{ねがい}はくば\\
        さて\ruby{映}{うつ}りこむ \ruby{我}{わ}が\ruby{鏡}{かがみ}\ruby{瞳}{ひとみ}に\\
        \ruby{嗚呼}{ああ}\ruby{汲}{く}まれたし その\ruby{厭}{いや}\ruby{心}{しん}\\
        \ruby{悲}{かな}しかりけむ\ruby{晩秋}{ばんしゅう}の\ruby{夜}{よる}は\\
        \ruby{月影}{つきかげ}\ruby{映}{は}えて\ruby{人影}{ひとかげ}も\ruby{追}{つい}ひ\ruby{得}{え}じ
    
    \end{minipage}
\end{enumerate} % 番号の箇条書き ここまで
%%%%% 歌詞 ここまで %%%%%
% end body

\end{document}

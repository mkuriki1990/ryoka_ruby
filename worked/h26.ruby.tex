\documentclass[10pt,b5j]{tarticle} % B6 縦書き
% \documentclass[10pt,b5j]{tarticle} % B6 縦書き
\AtBeginDvi{\special{papersize=128mm,182mm}} % B6 用用紙サイズ
\usepackage{otf} % Unicode で字を入力するのに必要なパッケージ
\usepackage[size=b6j]{bxpapersize} % B6 用紙サイズを指定
\usepackage[dvipdfmx]{graphicx} % 画像を挿入するためのパッケージ
\usepackage[dvipdfmx]{color} % 色をつけるためのパッケージ
\usepackage{pxrubrica} % ルビを振るためのパッケージ
\usepackage{multicol} % 複数段組を作るためのパッケージ
\setlength{\topmargin}{14mm} % 上下方向のマージン
\addtolength{\topmargin}{-1in} % 
\setlength{\oddsidemargin}{11mm} % 左右方向のマージン
\addtolength{\oddsidemargin}{-1in} % 
\setlength{\textwidth}{154mm} % B6 用
\setlength{\textheight}{108mm} % B6 用
\setlength{\headsep}{0mm} % 
\setlength{\headheight}{0mm} % 
\setlength{\topskip}{0mm} % 
\setlength{\parskip}{0pt} % 
\def\labelenumi{\theenumi、} % 箇条書きのフォーマット
\parindent = 0pt % 段落下げしない

 % B6 用テンプレート読み込み

\begin{document}
% begin header
%%%%% タイトルと作者 ここから %%%%%
\begin{minipage}[c]{0.7\hsize} % タイトルは上から 7 割
    \begin{center}
    % begin title
        {\LARGE
            姫月に重ねて % タイトルを入れる
        }
        {\small 
            (平成二十六年度寮歌) % 年などを入れる
        }
    % end title
    \end{center}
\end{minipage}
\begin{minipage}[c]{0.3\hsize} % 作歌作曲は上から 3 割
    \begin{flushright} % 下寄せにする
        % begin name
        松元一平君 作歌\\寺尾佳隆君 作曲 % 作歌・作曲者
        % end name
    \end{flushright}
\end{minipage}
%%%%% タイトルと作者 ここまで %%%%%
% (1,2,3 繰り返しなし)
% end header

% begin length
\vspace{1.5em} % タイトル, 作者と歌詞の間に隙間を設ける
\newcommand{\linespace}{0.5em} % 行間の設定
\newcommand{\blocksize}{0.33\hsize} % 段組間の設定
\newcommand{\itemmargin}{3em} % 曲番の位置調整の長さ
% end length
% begin body
%%%%% 歌詞 ここから %%%%%

\ruby{観月}{かん|げつ}\ruby{過}{す}ぎゆく\ruby{晩秋}{ばん|しゅう}の\ruby{夜}{よる}、
\ruby{穹蒼}{そう|きゅう}の\ruby[g]{天空}{そら}\ruby{高}{たか}く
\ruby{舞}{ま}ひたる\ruby{月}{つき}は\ruby[g]{今宵}{こよい}\ruby{満}{み}つるかな。\\
その\ruby[g]{清輝}{かがやき}に\ruby{映}{は}えし\ruby{姫}{ひめ}が\ruby[g]{鏡水}{かがみず}は、
\ruby{鹿}{わ}が\ruby[g]{純瞳}{ひとみ}に\ruby{宿}{やど}らむ。\\
\ruby{月影}{つき|かげ}は\ruby{鹿}{われ}を\ruby{誘}{さそ}ひ
\ruby{来}{き}たりしこの\ruby[g]{神無月}{とき}に
\ruby{何}{なに}をば\ruby{見}{み}せむ。

\begin{enumerate} % 番号の箇条書き ここから
    \setlength{\itemindent}{\itemmargin} % 曲番の位置調整
    \begin{minipage}[c]{\blocksize}
        
        \vspace{\linespace}
        \item~\\
        % 1.
        \ruby{時}{とき}\ruby{移}{うつ}ろひて \ruby[g]{人世}{よ}は\ruby{変}{か}われども\\
        \ruby[g]{今宵}{こよい}も\ruby[g]{満月}{つき}は\ruby{我}{われ}らを\ruby{照}{うつ}さむ\\
        \ruby{夜}{よる}の\ruby[g]{邪帳}{とばり}をはらはむと\\
        \ruby[g]{流歩}{あゆ}む\ruby{汝}{なんじ}は\ruby{楡}{にれ}に\ruby{似}{に}たれど\\
        \ruby[g]{風流}{かぜ}を\ruby{掴}{つか}まむ\ruby{芽}{め}に\ruby{感}{かん}ず\\
        \ruby[g]{風習}{ならひ}に\ruby{付和}{ふ|わ}せし\\
        \ruby{狗}{く}と\ruby{成}{な}らざらめや\\
        さて\ruby{映}{うつ}りこむ \ruby{我}{わ}が\ruby[g]{鏡瞳}{まなざし}に\\
        \ruby[g]{風習}{ならひ}だに\ruby{愛}{め}づる その\ruby[g]{気概}{おもひ}
        
    \end{minipage}
    \begin{minipage}[c]{\blocksize}
        
        \vspace{\linespace}
        \item~\\
        % 2.
        \ruby[g]{清澄}{す}みたる\ruby{想}{おも}ひ \ruby{知}{し}る\ruby{由}{よし}もなく\\
        \ruby[g]{今宵}{こよい}の\ruby[g]{三日月}{つき}は\ruby{川面}{かわ|も}に\ruby{映}{うつ}らむ\\
        かの\ruby{日}{ひ}の\ruby[g]{月影}{かた}とは\ruby{違}{ちが}へども\\
        \ruby[g]{人世}{よ}に\ruby{充}{み}つ\ruby[g]{解答}{いらへ}を\ruby{自}{おの}ずと\ruby{心得}{こころ|え}\\
        \ruby{此}{こ}れは\ruby{汝}{なんじ}の\ruby[g]{求望}{のぞみ}にか\\
        \ruby{漲}{みなぎ}る\ruby{想}{おも}ひ などか\ruby{劣}{おと}らむ\\
        さて\ruby{映}{うつ}りこむ \ruby{我}{わ}が\ruby[g]{鏡瞳}{まなざし}に\\
        \ruby{身}{み}を\ruby{委}{ゆだ}ねばや その\ruby[g]{清流}{ながれ}
        
    \end{minipage}
    \begin{minipage}[c]{\blocksize}
        
        \vspace{\linespace}
        \item~\\
        % 3.
        \ruby{静}{せい}と\ruby{唸}{うな}りし \ruby[g]{雨澪}{あめ}したたれば\\
        \ruby[g]{今宵}{こよい}も\ruby{我}{われ}は\ruby[g]{朧月}{つき}を\ruby{仰}{あお}がむ\\
        \ruby{姫}{ひめ}が\ruby[g]{麗姿}{すがた}を\ruby[g]{追憶}{とら}ふべく\\
        \ruby{汝}{な}が\ruby{想}{おも}ひは\ruby{涙}{なみだ}と\ruby[g]{落流}{なが}れ\\
        \ruby{透}{す}かし\ruby[g]{斜光}{ひかり}にさらさるる\\
        \ruby{閉}{と}じなむ\ruby[g]{凌雲}{くも}よ こひ\ruby{願}{ねが}はくば\\
        さて\ruby{映}{うつ}りこむ \ruby{我}{わ}が\ruby[g]{鏡瞳}{まなざし}に\\
        \ruby{嗚呼}{あ|あ}\ruby{汲}{く}まれたし その\ruby[g]{厭心}{こころ}\\
        \ruby{悲}{かな}しかりけむ\ruby[g]{晩秋}{あき}の\ruby{夜}{よ}は\\
        \ruby{月影}{つき|かげ}\ruby{映}{は}えて\ruby[g]{人影}{かげ}も\ruby{追}{お}ひ\ruby{得}{え}じ
    
    \end{minipage}
\end{enumerate} % 番号の箇条書き ここまで
%%%%% 歌詞 ここまで %%%%%
% end body

\end{document}

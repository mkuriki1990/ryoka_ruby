\documentclass[10pt,b5j]{tarticle} % B6 縦書き
% \documentclass[10pt,b5j]{tarticle} % B6 縦書き
\AtBeginDvi{\special{papersize=128mm,182mm}} % B6 用用紙サイズ
\usepackage{otf} % Unicode で字を入力するのに必要なパッケージ
\usepackage[size=b6j]{bxpapersize} % B6 用紙サイズを指定
\usepackage[dvipdfmx]{graphicx} % 画像を挿入するためのパッケージ
\usepackage[dvipdfmx]{color} % 色をつけるためのパッケージ
\usepackage{pxrubrica} % ルビを振るためのパッケージ
\usepackage{multicol} % 複数段組を作るためのパッケージ
\setlength{\topmargin}{14mm} % 上下方向のマージン
\addtolength{\topmargin}{-1in} % 
\setlength{\oddsidemargin}{11mm} % 左右方向のマージン
\addtolength{\oddsidemargin}{-1in} % 
\setlength{\textwidth}{154mm} % B6 用
\setlength{\textheight}{108mm} % B6 用
\setlength{\headsep}{0mm} % 
\setlength{\headheight}{0mm} % 
\setlength{\topskip}{0mm} % 
\setlength{\parskip}{0pt} % 
\def\labelenumi{\theenumi、} % 箇条書きのフォーマット
\parindent = 0pt % 段落下げしない

 % B6 用テンプレート読み込み

\begin{document}
% begin header
%%%%% タイトルと作者 ここから %%%%%
\begin{minipage}[c]{0.7\hsize} % タイトルは上から 7 割
    \begin{center}
    % begin title
        {\LARGE
            古城の春は % タイトルを入れる
        }
        {\small 
            (昭和七年寮歌) % 年などを入れる
        }
    % end title
    \end{center}
\end{minipage}
\begin{minipage}[c]{0.3\hsize} % 作歌作曲は上から 3 割
    \begin{flushright} % 下寄せにする
        % begin name
        大槻均君 作歌\\中村小弥太君 作曲 % 作歌・作曲者
        % end name
    \end{flushright}
\end{minipage}
%%%%% タイトルと作者 ここまで %%%%%
% (1,2,5 繰り返しなし)
% end header

% begin length
\vspace{1.5em} % タイトル, 作者と歌詞の間に隙間を設ける
\newcommand{\linespace}{0.5em} % 行間の設定
\newcommand{\blocksize}{0.33\hsize} % 段組間の設定
\newcommand{\itemmargin}{3em} % 曲番の位置調整の長さ
% end length
% begin body
%%%%% 歌詞 ここから %%%%%
\begin{enumerate} % 番号の箇条書き ここから
    \setlength{\itemindent}{\itemmargin} % 曲番の位置調整
    \begin{minipage}[c]{\blocksize}
    
        \vspace{\linespace}
        \item~\\
        % 1.
        \ruby{古城}{こ|じょう}の\ruby{春}{はる}は\ruby{老}{お}い\ruby{易}{やす}く\\
        \ruby{延齢草}{えん|れい|そう}の\ruby{名}{な}に\ruby{問}{と}へど\\
        \ruby{流転}{る|てん}の\ruby{法}{ほう}は\ruby{断}{た}ち\ruby{難}{がた}し\\
        \ruby{友}{とも}よエルムの\ruby{鐘}{かね}を\ruby{聴}{き}け\\
        \ruby{再建}{さい|けん}の\ruby{秋}{とき}\ruby{程}{ほど}なけん\\
        ペルアスペラと\ruby{鳴}{な}り\ruby{響}{ひび}く
        
        \vspace{\linespace}
        \item~\\
        % 2.
        \ruby{今}{いま}\ruby{移}{うつ}り\ruby{来}{こ}し\ruby[g]{原始林}{もり}の\ruby{蔭}{かげ}\\
        \ruby{宿}{やど}るは\ruby{未}{いま}だ\ruby{浅}{あさ}けれど\\
        \ruby{契}{ちぎり}は\ruby{深}{ふか}き\ruby{三百}{さん|びゃく}の\\
        \ruby{心}{こころ}を\ruby{交}{か}はすこの\ruby{宴}{うたげ}\\
        \ruby{暁}{あかつき}かけていざ\ruby{撞}{つ}かん\\
        アドアストラの\ruby{自治}{じ|ち}の\ruby{鐘}{かね}
        
    \end{minipage}
    \begin{minipage}[c]{\blocksize}
        
        \vspace{\linespace}
        \item~\\
        % 3.
        \ruby{妖雲}{よう|うん}\ruby{西}{にし}に\ruby{漾}{ただよ}へど\\
        \ruby{視}{み}よ\ruby{落日}{らく|じつ}の\ruby{悠々}{ゆう|ゆう}と\\
        \ruby{大地}{だい|ち}を\ruby{旋}{めぐ}り\ruby{淪}{しづ}むかな\\
        \ruby{眠}{ねむ}る\ruby{此}{こ}の\ruby{城}{しろ}\ruby{吾}{われ}も\ruby{亦}{また}\\
        \ruby{醒}{さ}めての\ruby[g]{生命}{いのち}\ruby{培}{つちか}はん\\
        \ruby{四大}{し|だい}の\ruby{荒}{すさ}び\ruby[g]{明日}{あす}あれば
        
        \vspace{\linespace}
        \item~\\
        % 4.
        \ruby{厳寒}{げん|かん}\ruby{凍}{こお}る\ruby{極北}{きょく|ほく}に\\
        \ruby{霧}{きり}\ruby{立}{た}ち\ruby{騒}{さわ}ぐ\ruby{曙}{あけぼの}の\\
        \ruby{光}{ひかり}を\ruby{担}{にの}うて\ruby{起}{た}たんとき\\
        \ruby[g]{際涯}{いやはて}もなく\ruby{寄}{よ}せ\ruby{返}{かえ}す\\
        \ruby{世紀}{せい|き}の\ruby[g]{波濤}{なみ}は\ruby{狂}{くる}へども\\
        \ruby{既倒}{き|とう}にかへす\ruby{力}{ちから}あり
        
    \end{minipage}
    \begin{minipage}[c]{\blocksize}
        
        \vspace{\linespace}
        \item~\\
        % 5.
        \ruby{竜舵}{りょう|だ}\ruby{岸}{きし}\ruby{打}{う}つ\ruby{大洋}{たい|よう}の\\
        \ruby{今}{いま}\ruby{人生}{じん|せい}の\ruby{船出}{ふな|で}かな\\
        \ruby{白帆}{はく|はん}\ruby{高}{たか}くはためきて\\
        \ruby{正気}{せい|き}をはらむ\ruby{若人}{わこ|うど}の\\
        \ruby{理想}{り|そう}の\ruby{船}{ふね}は\ruby{不壊}{ふ|ゑ}にして\\
        さかまく\ruby[g]{苦海}{うみ}を\ruby[g]{永遠}{とは}に\ruby{航}{ゆ}く
    
    \end{minipage}
\end{enumerate} % 番号の箇条書き ここまで
%%%%% 歌詞 ここまで %%%%%
% end body

\end{document}

\documentclass[10pt,b5j]{tarticle} % B6 縦書き
% \documentclass[10pt,b5j]{tarticle} % B6 縦書き
\AtBeginDvi{\special{papersize=128mm,182mm}} % B6 用用紙サイズ
\usepackage{otf} % Unicode で字を入力するのに必要なパッケージ
\usepackage[size=b6j]{bxpapersize} % B6 用紙サイズを指定
\usepackage[dvipdfmx]{graphicx} % 画像を挿入するためのパッケージ
\usepackage[dvipdfmx]{color} % 色をつけるためのパッケージ
\usepackage{pxrubrica} % ルビを振るためのパッケージ
\usepackage{plext} % 漢数字の enumerate を使うためのパッケージ
\usepackage{multicol} % 複数段組を作るためのパッケージ
\setlength{\topmargin}{14mm} % 上下方向のマージン
\addtolength{\topmargin}{-1in} % 
\setlength{\oddsidemargin}{11mm} % 左右方向のマージン
\addtolength{\oddsidemargin}{-1in} % 
\setlength{\textwidth}{154mm} % B6 用
\setlength{\textheight}{108mm} % B6 用
\setlength{\headsep}{0mm} % 
\setlength{\headheight}{0mm} % 
\setlength{\topskip}{0mm} % 
\setlength{\parskip}{0pt} % 
\def\theenumi{\Kanji{enumi}} % 箇条書きのフォーマットを漢数字に変更
\parindent = 0pt % 段落下げしない
\pagestyle{empty} % すべてのページ番号を消去
% \renewcommand{\baselinestretch}{0.9} % 行間の倍率
 % B6 用テンプレート読み込み

\begin{document}
% begin header
%%%%% タイトルと作者 ここから %%%%%
\begin{minipage}[c]{0.7\hsize} % タイトルは上から 7 割
    \begin{center}
    % begin title
        {\LARGE
            咲く六華よ % タイトルを入れる
        }
        {\small 
            (平成二十七年度寮歌) % 年などを入れる
        }
    % end title
    \end{center}
\end{minipage}
\begin{minipage}[c]{0.3\hsize} % 作歌作曲は上から 3 割
    \begin{flushright} % 下寄せにする
        % begin name
        鈴木美奈君 作歌\\小松遼貴君 作曲 % 作歌・作曲者
        % end name
    \end{flushright}
\end{minipage}
%%%%% タイトルと作者 ここまで %%%%%
% (1,2,3,4 了あり)
% end header

% begin length
\vspace{1.5em} % タイトル, 作者と歌詞の間に隙間を設ける
\newcommand{\linespace}{0.5em} % 行間の設定
\newcommand{\blocksize}{0.5\hsize} % 段組間の設定
\newcommand{\itemmargin}{3em} % 曲番の位置調整の長さ
% end length
% begin body
%%%%% 歌詞 ここから %%%%%
\begin{enumerate} % 番号の箇条書き ここから
    \setlength{\itemindent}{\itemmargin} % 曲番の位置調整
    \begin{minipage}[c]{\blocksize}
    
        \vspace{\linespace}
        \item~\\
        % 1.
        \ruby{学}{まな}び\ruby{舎}{や}の\ruby{野}{の}に \ruby{咲}{さ}く\ruby[g]{六華}{はな}よ\\
        \ruby{我}{われ}らを\ruby{招}{まね}く \ruby[g]{北寮}{きた}の\ruby{幸}{さち}\\
        \ruby[g]{大望}{のぞみ}\ruby{麗}{うるわ}し この\ruby{道}{みち}に\\
        \ruby{名花}{めい|か}\ruby[g]{丈夫}{ますらお} \ruby{集}{つど}い\ruby{来}{く}る
        
        \vspace{\linespace}
        \item~\\
        % 2.
        \ruby{涼風}{すず|かぜ}に\ruby{舞}{ま}う \ruby{箱柳}{はこ|やなぎ}\\
        \ruby[g]{寮歌}{うた}\ruby{鳴}{な}り\ruby{響}{ひび}く \ruby[g]{夕餉}{ゆうげ}\ruby{時}{どき}\\
        \ruby{先人}{せん|じん}\ruby{継}{つ}ぎし \ruby[g]{一途}{ひとみち}を\\
        \ruby{未}{ま}だ\ruby{踏}{ふ}み\ruby{初}{そ}めし \ruby[g]{寮友}{われら}なり
        
    \end{minipage}
    \begin{minipage}[c]{\blocksize}
        
        \vspace{\linespace}
        \item~\\
        % 3.
        \ruby{楡影}{ゆ|えい}\ruby{傾}{かたぶ}く \ruby{夜}{よ}の\ruby[g]{静寂}{しじま}\\
        \ruby[g]{微睡}{まどろ}み\ruby{知}{し}らぬ \ruby{蔦}{つた}\ruby[g]{住居}{ずまい}\\
        \ruby{憂}{うれ}いの\ruby{醒}{さ}めぬ \ruby{世}{よ}の\ruby{岐}{みち}も\\
        \ruby{満}{み}ち\ruby{行}{ゆ}く\ruby[g]{若月}{つき}が \ruby{照}{て}らすかな
        
        \vspace{\linespace}
        \item~\\
        % 4.
        \ruby{季節}{き|せつ}\ruby{巡}{めぐ}りて \ruby[g]{朔風}{かぜ}は\ruby{凪}{な}ぎ\\
        \ruby[g]{無何有}{むかう}の\ruby{郷}{さと}を \ruby{離}{か}る\ruby{時}{とき}ぞ\\
        \ruby{嗚呼}{あ|あ}\ruby{忘}{わす}るまじき \ruby{我}{わ}が\ruby{迪}{みち}の\\
        \ruby{齢}{よわい}\ruby{延}{の}べたし \ruby{青}{あお}き\ruby{春}{はる}

    
    \end{minipage}
\end{enumerate} % 番号の箇条書き ここまで
%%%%% 歌詞 ここまで %%%%%
% end body

\end{document}

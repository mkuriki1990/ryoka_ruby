\documentclass[10pt,b5j]{tarticle} % B6 縦書き
% \documentclass[10pt,b5j]{tarticle} % B6 縦書き
\AtBeginDvi{\special{papersize=128mm,182mm}} % B6 用用紙サイズ
\usepackage{otf} % Unicode で字を入力するのに必要なパッケージ
\usepackage[size=b6j]{bxpapersize} % B6 用紙サイズを指定
\usepackage[dvipdfmx]{graphicx} % 画像を挿入するためのパッケージ
\usepackage[dvipdfmx]{color} % 色をつけるためのパッケージ
\usepackage{pxrubrica} % ルビを振るためのパッケージ
\usepackage{plext} % 漢数字の enumerate を使うためのパッケージ
\usepackage{multicol} % 複数段組を作るためのパッケージ
\setlength{\topmargin}{14mm} % 上下方向のマージン
\addtolength{\topmargin}{-1in} % 
\setlength{\oddsidemargin}{11mm} % 左右方向のマージン
\addtolength{\oddsidemargin}{-1in} % 
\setlength{\textwidth}{154mm} % B6 用
\setlength{\textheight}{108mm} % B6 用
\setlength{\headsep}{0mm} % 
\setlength{\headheight}{0mm} % 
\setlength{\topskip}{0mm} % 
\setlength{\parskip}{0pt} % 
\def\theenumi{\Kanji{enumi}} % 箇条書きのフォーマットを漢数字に変更
\parindent = 0pt % 段落下げしない
\pagestyle{empty} % すべてのページ番号を消去
% \renewcommand{\baselinestretch}{0.9} % 行間の倍率
 % B6 用テンプレート読み込み

\begin{document}
% begin header
%%%%% タイトルと作者 ここから %%%%%
\begin{minipage}[c]{0.7\hsize} % タイトルは上から 7 割
    \begin{center}
    % begin title
        {\LARGE
            時轍乾坤に % タイトルを入れる
        }
        {\small 
            (大正四年寮歌) % 年などを入れる
        }
    % end title
    \end{center}
\end{minipage}
\begin{minipage}[c]{0.3\hsize} % 作歌作曲は上から 3 割
    \begin{flushright} % 下寄せにする
        % begin name
        沢田退蔵君 作歌・作曲 % 作歌・作曲者
        % end name
    \end{flushright}
\end{minipage}
%%%%% タイトルと作者 ここまで %%%%%
% (1,2,5,6 繰り返しなし)
% end header

% begin length
\vspace{1.5em} % タイトル, 作者と歌詞の間に隙間を設ける
\newcommand{\linespace}{0.5em} % 行間の設定
\newcommand{\blocksize}{0.33\hsize} % 段組間の設定
\newcommand{\itemmargin}{3em} % 曲番の位置調整の長さ
% end length
% begin body
%%%%% 歌詞 ここから %%%%%
\begin{enumerate} % 番号の箇条書き ここから
    \setlength{\itemindent}{\itemmargin} % 曲番の位置調整
    \begin{minipage}[c]{\blocksize}
    
        \vspace{\linespace}
        \item~\\
        % 1.
        \ruby[g]{時轍}{とき}\ruby{乾坤}{けん|こん}に\ruby{回}{めぐ}り\ruby{来}{き}て\\
        \ruby[g]{陽春}{はる}\ruby{駘蕩}{たい|とう}のおぼろよひ\\
        \ruby{紫}{むらさき}\ruby{淡}{あは}く\ruby{霞}{かすみ}\ruby{罩}{こ}め\\
        \ruby{自治}{じ|ち}の\ruby{流}{なが}れは\ruby[g]{永遠}{とこしへ}に\\
        \ruby{若葉}{わか|ば}の\ruby{陰}{かげ}を\ruby{浮}{うか}べつつ\\
        \ruby{吾等}{われ|ら}が\ruby{幸}{さち}を\ruby{祝}{いは}ふらん
        
        \vspace{\linespace}
        \item~\\
        % 2.
        \ruby{胡馬北風}{こ|ば|ほく|ふう}に\ruby{嘶}{いなな}きて\\
        \ruby{越鳥南枝}{ゑっ|てう|なん|し}に\ruby{巣}{す}を\ruby{造}{つく}る\\
        \ruby{世}{よ}の\ruby{濁江}{にごり|え}に\ruby{逆}{さから}へる\\
        \ruby{棹歌}{たう|か}の\ruby{声}{こえ}の\ruby{勇}{いさ}ましき\\
        \ruby[g]{三星霜}{みとせ}の\ruby{春}{はる}のおきふしに\\
        \ruby{深}{ふか}き\ruby[g]{感慨}{おもひ}のなからめや
        
    \end{minipage}
    \begin{minipage}[c]{\blocksize}
        
        \vspace{\linespace}
        \item~\\
        % 3.
        \ruby{紫扉}{さい|ひ}を\ruby{出}{い}でて\ruby{霜}{しも}を\ruby{踏}{ふ}み\\
        \ruby[g]{川流}{ながれ}を\ruby{掬}{むす}び\ruby{薪}{たき}\ruby{樵}{きこ}る\\
        \ruby{崇}{たか}き\ruby[g]{希望}{のぞみ}の\ruby[g]{若人}{わこうど}が\\
        \ruby{歓喜}{かん|き}\ruby{憂苦}{ゆう|く}を\ruby{共}{とも}にせし\\
        \ruby{友悌}{ゆう|てい}\ruby{凋}{しほ}まぬ\ruby{松柏}{しょう|はく}と\\
        \ruby{幾千代}{いく|ち|よ}かけて\ruby{変}{かわ}らざれ
        
        \vspace{\linespace}
        \item~\\
        % 4.
        \ruby{彼}{か}の\ruby{邯鄲}{かん|たん}の\ruby{仮枕}{かり|まくら}\\
        \ruby{栄華}{えい|が}の\ruby{夢}{ゆめ}も\ruby{半}{なかば}にて\\
        \ruby{世}{よ}の\ruby{秋風}{あき|かぜ}に\ruby{驚}{おどろ}かん\\
        \ruby{目}{め}ざす\ruby[g]{真理}{まこと}の\ruby{高殿}{たか|どの}は\\
        \ruby{遠}{とほ}く\ruby{遙}{はろ}けし\ruby[g]{突進}{すす}めいざ\\
        \ruby{心}{こころ}の\ruby{駒}{こま}に\ruby{鞭打}{むち|う}ちて
        
    \end{minipage}
    \begin{minipage}[c]{\blocksize}
        
        \vspace{\linespace}
        \item~\\
        % 5.
        ウラルの\ruby[g]{彼方}{かなた}\ruby{風}{かぜ}\ruby{凄}{すご}く\\
        \ruby{陣雲}{じん|うん}くらき\ruby[g]{八街}{やちまた}は\\
        \ruby{鉄騎}{てっ|き}\ruby{百万}{ひゃく|まん}\ruby{駆}{かけ}りつつ\\
        \ruby{正義}{せい|ぎ}の\ruby{光}{ひかり}\ruby{失}{う}する\ruby{時}{とき}\\
        \ruby{燃}{も}ゆる\ruby{義憤}{ぎ|ふん}を\ruby{胸}{むね}に\ruby{秘}{ひ}め\\
        \ruby{起}{た}て\ruby{自治寮}{じ|ち|りょう}の\ruby{健男児}{けん|だん|じ}
        
        \vspace{\linespace}
        \item~\\
        % 6.
        \ruby{自由}{じ|ゆう}の\ruby{旗}{はた}を\ruby{振}{ふ}り\ruby{翳}{かざ}し\\
        \ruby{平和}{へい|わ}の\ruby{楯}{たて}を\ruby{掻}{か}き\ruby{列}{つら}ね\\
        \ruby{吾等}{われ|ら}\ruby{起}{た}つべき\ruby{時}{とき}は\ruby{来}{き}ぬ\\
        \ruby{見}{み}よや\ruby{獅子王}{し|し|おう}\ruby{一吼}{いっ|こう}して\\
        \ruby{曠野}{あら|の}\ruby{虎狼}{こ|ろう}の\ruby{影}{かげ}もなし\\
        \ruby{祝}{いわ}へ\ruby[g]{今{\CID{13831}}}{こよひ}の\ruby{記念祭}{き|ねん|さい} % CID 宵󠄁
    
    \end{minipage}
\end{enumerate} % 番号の箇条書き ここまで
%%%%% 歌詞 ここまで %%%%%
% end body

\end{document}

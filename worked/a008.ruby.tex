\documentclass[10pt,b5j]{tarticle} % B6 縦書き
% \documentclass[10pt,b5j]{tarticle} % B6 縦書き
\AtBeginDvi{\special{papersize=128mm,182mm}} % B6 用用紙サイズ
\usepackage{otf} % Unicode で字を入力するのに必要なパッケージ
\usepackage[size=b6j]{bxpapersize} % B6 用紙サイズを指定
\usepackage[dvipdfmx]{graphicx} % 画像を挿入するためのパッケージ
\usepackage[dvipdfmx]{color} % 色をつけるためのパッケージ
\usepackage{pxrubrica} % ルビを振るためのパッケージ
\usepackage{plext} % 漢数字の enumerate を使うためのパッケージ
\usepackage{multicol} % 複数段組を作るためのパッケージ
\setlength{\topmargin}{14mm} % 上下方向のマージン
\addtolength{\topmargin}{-1in} % 
\setlength{\oddsidemargin}{11mm} % 左右方向のマージン
\addtolength{\oddsidemargin}{-1in} % 
\setlength{\textwidth}{154mm} % B6 用
\setlength{\textheight}{108mm} % B6 用
\setlength{\headsep}{0mm} % 
\setlength{\headheight}{0mm} % 
\setlength{\topskip}{0mm} % 
\setlength{\parskip}{0pt} % 
\def\theenumi{\Kanji{enumi}} % 箇条書きのフォーマットを漢数字に変更
\parindent = 0pt % 段落下げしない
\pagestyle{empty} % すべてのページ番号を消去
% \renewcommand{\baselinestretch}{0.9} % 行間の倍率
 % B6 用テンプレート読み込み

\renewcommand{\baselinestretch}{0.90} % 行間の倍率

\begin{document}
% begin header
%%%%% タイトルと作者 ここから %%%%%
\begin{minipage}[c]{0.7\hsize} % タイトルは上から 7 割
    \begin{center}
    % begin title
        {\LARGE
            清き郷石狩の % タイトルを入れる
        }
        {\small 
            (昭和十五年桜星会三十周年記念歌) % 年などを入れる
        }
    % end title
    \end{center}
\end{minipage}
\begin{minipage}[c]{0.3\hsize} % 作歌作曲は上から 3 割
    \begin{flushright} % 下寄せにする
        % begin name
        岩崎五郎君 作歌\\呉泰治郎君 作曲 % 作歌・作曲者
        % end name
    \end{flushright}
\end{minipage}
%%%%% タイトルと作者 ここまで %%%%%
% % end header

% begin length
\vspace{0.5em} % タイトル, 作者と歌詞の間に隙間を設ける
\newcommand{\linespace}{0.5em} % 行間の設定
\newcommand{\blocksize}{0.5\hsize} % 段組間の設定
\newcommand{\itemmargin}{3em} % 曲番の位置調整の長さ
% end length
% begin body
%%%%% 歌詞 ここから %%%%%
\begin{enumerate} % 番号の箇条書き ここから
    \setlength{\itemindent}{\itemmargin} % 曲番の位置調整
    \begin{minipage}[c]{\blocksize}
    
        \vspace{\linespace}
        \item~\\
        % 1.
        \ruby{清}{きよ}き\ruby{郷}{くに}\ruby{石狩}{いし|かり}の\ruby[g]{曠野}{の}に\\
        うち\ruby{立}{た}てし\ruby{先人}{せん|じん}が\ruby{跡}{あと}\\
        \ruby{乾坤}{けん|こん}に\ruby[g]{時光}{ひかり}\ruby{流}{なが}れて\\
        \ruby{今}{いま}ぞなる\ruby[g]{三十年}{みそとせ}の\ruby[g]{崇高}{たか}き\ruby{青史}{せい|し}よ\\
        \ruby{讃}{たた}へなん いざ\\
        \ruby{若}{わか}き\ruby{血潮}{ち|しほ} \ruby{燃}{も}ゆる\ruby[g]{理想}{おもひ}\\
        \ruby{世}{よ}を\ruby[g]{覺醒}{さま}し\ruby{世}{よ}を\ruby{導}{みちび}かん\\
        \ruby{傅統}{でん|とう}の\ruby[g]{楡鐘}{かね}\ruby{高}{たか}く\ruby{鳴}{な}るなり
        
        \vspace{\linespace}
        \item~\\
        % 2.
        \ruby{黒}{くろ}き\ruby{雲}{くも}\ruby{世}{よ}に\ruby{狂}{くる}へども\\
        \ruby{守}{まも}り\ruby{來}{こ}し\ruby{正義}{せい|ぎ}の\ruby[g]{精神}{こころ}\\
        \ruby[g]{青春}{わかきひ}の\ruby[g]{生命}{いのち}\ruby{捧}{ささ}げて\\
        \ruby{惠}{たづ}ぬなり\ruby[g]{幽遠}{かすか}なる\ruby[g]{眞理}{まこと}の\ruby[g]{秘奥}{おくか}\\
        \ruby[g]{高唱}{うた}はなん いざ\\
        \ruby{熱}{あつ}き\ruby[g]{感激}{おもひ} たぎる\ruby[g]{憧憬}{のぞみ}\\
        \ruby{美}{うつく}しく\ruby{強}{つよ}く\ruby{生}{い}かばや\\
        \ruby{雄叫}{を|たけ}びは\ruby{高}{たか}く\ruby{湧}{わ}くなり
        
    \end{minipage}
    \begin{minipage}[c]{\blocksize}
        
        \vspace{\linespace}
        \item~\\
        % 3.
        \ruby{天地}{あめ|つち}に\ruby[g]{暴風雨}{あらし}\ruby{吠}{ほ}ゆるも\\
        \ruby[g]{東洋}{ひんがし}に\ruby{夜}{よ}は\ruby[g]{黎明}{あけ}んとす\\
        \ruby[g]{世界}{よ}を\ruby{救}{すく}ふ\ruby{大}{だい}\ruby{理想}{り|そう}もて\\
        うち\ruby{立}{た}てん\ruby{永劫}{えい|ごう}の\ruby{平和}{へい|わ}の\ruby{大旆}{たい|はい}\\
        \ruby{叫}{さけ}ばなん いざ\\
        \ruby{湧}{わ}ける\ruby[g]{激情}{こころ} あがる\ruby[g]{歡喜}{よろこび}\\
        \ruby{楡}{にれ}の\ruby{舎}{や}の\ruby{健兒}{けん|じ}\ruby{我等}{われ|ら}は\\
        \ruby{生}{い}ける\ruby{證}{しるし}に\ruby{胸}{むね}は\ruby{湧}{わ}くなり
        
        \vspace{\linespace}
        \item~\\
        % 4.
        \ruby{悠久}{ゆう|きゅう}の\ruby{時}{とき}の\ruby{移}{うつ}ろひ\\
        \ruby[g]{青春}{わかきひ}のこの\ruby[g]{瞬間}{ひととき}を\\
        \ruby[g]{星辰}{ほし}\ruby{澄}{きよ}きエルムの\ruby{園}{その}に\\
        \ruby{過}{すぐ}すなり\ruby[g]{涯際}{はてし}なき\ruby[g]{神秘}{くしび}の\ruby[g]{懐中}{うち}に\\
        \ruby{仰}{あお}がなん いざ\\
        \ruby{清}{きよ}き\ruby[g]{生命}{いのち} \ruby{高}{たか}き\ruby[g]{意欲}{まこと}\\
        \ruby{先人}{せん|じん}の\ruby{遺}{のこ}せし\ruby[g]{教訓}{おしへ}\\
        \ruby{我等}{われ|ら}が\ruby{魂}{こころ}\ruby{強}{つよ}く\ruby{打}{う}つなり
    
    \end{minipage}
\end{enumerate} % 番号の箇条書き ここまで
%%%%% 歌詞 ここまで %%%%%
% end body

\end{document}

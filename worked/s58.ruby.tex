\documentclass[10pt,b5j]{tarticle} % B6 縦書き
% \documentclass[10pt,b5j]{tarticle} % B6 縦書き
\AtBeginDvi{\special{papersize=128mm,182mm}} % B6 用用紙サイズ
\usepackage{otf} % Unicode で字を入力するのに必要なパッケージ
\usepackage[size=b6j]{bxpapersize} % B6 用紙サイズを指定
\usepackage[dvipdfmx]{graphicx} % 画像を挿入するためのパッケージ
\usepackage[dvipdfmx]{color} % 色をつけるためのパッケージ
\usepackage{pxrubrica} % ルビを振るためのパッケージ
\usepackage{plext} % 漢数字の enumerate を使うためのパッケージ
\usepackage{multicol} % 複数段組を作るためのパッケージ
\setlength{\topmargin}{14mm} % 上下方向のマージン
\addtolength{\topmargin}{-1in} % 
\setlength{\oddsidemargin}{11mm} % 左右方向のマージン
\addtolength{\oddsidemargin}{-1in} % 
\setlength{\textwidth}{154mm} % B6 用
\setlength{\textheight}{108mm} % B6 用
\setlength{\headsep}{0mm} % 
\setlength{\headheight}{0mm} % 
\setlength{\topskip}{0mm} % 
\setlength{\parskip}{0pt} % 
\def\theenumi{\Kanji{enumi}} % 箇条書きのフォーマットを漢数字に変更
\parindent = 0pt % 段落下げしない
\pagestyle{empty} % すべてのページ番号を消去
% \renewcommand{\baselinestretch}{0.9} % 行間の倍率
 % B6 用テンプレート読み込み

\begin{document}
% begin header
%%%%% タイトルと作者 ここから %%%%%
\begin{minipage}[c]{0.7\hsize} % タイトルは上から 7 割
    \begin{center}
    % begin title
        {\LARGE
            寮生の道 % タイトルを入れる
        }
        {\small 
            (昭和五十八年寮歌) % 年などを入れる
        }
    % end title
    \end{center}
\end{minipage}
\begin{minipage}[c]{0.3\hsize} % 作歌作曲は上から 3 割
    \begin{flushright} % 下寄せにする
        % begin name
        泉進介君 作歌\\島倉朝雄君 作曲 % 作歌・作曲者
        % end name
    \end{flushright}
\end{minipage}
%%%%% タイトルと作者 ここまで %%%%%
% (春,夏,秋,冬,まとめ
% end header

% begin length
\vspace{0.5em} % タイトル, 作者と歌詞の間に隙間を設ける
\newcommand{\linespace}{0.7em} % 行間の設定
\newcommand{\blocksize}{0.33\hsize} % 段組間の設定
\newcommand{\itemmargin}{6em} % 曲番の位置調整の長さ
% end length
% begin body
%%%%% 歌詞 ここから %%%%%
\ruby{凍}{い}てつきし\ruby{氷}{こおり}の\ruby{路}{みち}も\ruby{溶}{と}け\ruby{始}{はじ}め、
\ruby{見}{み}はるかす\ruby{山}{やま}に\ruby{白雪}{はく|せつ}\ruby{消}{き}ゆる\ruby{頃}{ころ}\\
\ruby{集}{つど}い\ruby{来}{こ}し\ruby{百}{ひゃく}と\ruby{四十}{し|じゅう}の\ruby{若人}{わこ|うど}は
\ruby{故郷}{こ|きょう}も\ruby{親}{おや}も\ruby{銭}{かね}もなく
\ruby{恃}{たの}むは\ruby{己}{おのれ}の\ruby[g]{仁侠}{おとこぎ}ばかり\\
\ruby{然}{しか}れども\ruby{新}{あら}たな\ruby{舎}{やど}りの\ruby{恵迪}{けい|てき}は
\ruby{五層六刃}{ご|そう|りっ|ぱ}の\ruby{白亜城}{はく|あ|じょう}\\
\ruby{夜}{よる}も\ruby[g]{希望}{のぞみ}の\ruby{灯}{ひ}は\ruby{消}{け}さず、
\ruby{棲}{す}むは\ruby{豪傑}{ごう|けつ}\ruby{酒乱}{しゅ|らん}の\ruby{徒}{と}\\
さあ\ruby{来}{こ}いさあ\ruby{来}{こ}い\ruby{恵迪}{けい|てき}へ
\ruby{北都}{ほく|と}に\ruby{築}{きず}かん\ruby{我等}{われ|ら}が\ruby[g]{自治寮}{とりで}
\begin{enumerate} % 番号の箇条書き ここから
    \setlength{\itemindent}{\itemmargin}
    \begin{minipage}[c]{\blocksize}

        \vspace{\linespace}
        \item[春(四月)]~\\
        % \ruby{春}{はる}(\ruby{四月}{し|がつ}).
        ちょいとそこ\ruby{行}{ゆ}く\ruby[g]{新入寮生}{りょうせい}さん\\
        \ruby{明日}{あ|す}は\ruby{我身}{わが|み}か\ruby{知}{し}らねども\\
        \ruby{大酒}{おお|ざけ}くらって\ruby{逆噴射}{ぎゃく|ふん|しゃ}\\
        これぞ\ruby[g]{寮生}{おとこ}の\ruby{生}{い}きる\ruby{道}{みち}
        
        \vspace{\linespace}
        \item[夏(八月)]~\\
        % \ruby{夏}{なつ}(\ruby{八月}{はち|がつ}).
        ちょいとそこ\ruby{行}{ゆ}く\ruby{寮生}{りょう|せい}さん\\
        \ruby{弊衣破帽}{へい|い|は|ぼう}に\ruby{食糧難}{しょく|りょう|なん}\\
        \ruby[g]{両親}{おや}の\ruby{顔}{かお}が\ruby{眼}{め}に\ruby{浮}{う}かぶ\\
        これぞ\ruby[g]{寮生}{おとこ}の\ruby{生}{い}きる\ruby{道}{みち}
        
    \end{minipage}
    \begin{minipage}[c]{\blocksize}

        \vspace{\linespace}
        \item[秋(十月)] ~\\
        % \ruby{秋}{あき}(\ruby{十月}{じゅう|がつ}).
        ちょいとそこ\ruby{行}{ゆ}く\ruby{寮生}{りょう|せい}さん\\
        \ruby{尻}{しり}に\ruby{赤}{あか}フン\ruby{巻}{ま}きつけて\\
        \ruby{狂喜乱舞}{きょう|き|らん|ぶ}す\ruby{交差点}{こう|さ|てん}\\
        これぞ\ruby[g]{寮生}{おとこ}の\ruby{生}{い}きる\ruby{道}{みち}
        
        \vspace{\linespace}
        \item[冬(二月)]~\\
        % \ruby{冬}{ふゆ}(\ruby{二月}{にがつ}).
        ちょいとそこ\ruby{行}{ゆ}く\ruby{寮生}{りょう|せい}さん\\
        ジャンプ\ruby{大会}{たい|かい}\ruby{変態}{へん|たい}か\\
        \ruby{花}{はな}の\ruby{女子大}{じょ|し|だい}\ruby{赤面}{せき|めん}す\\
        これぞ\ruby[g]{寮生}{おとこ}の\ruby{生}{い}きる\ruby{道}{みち}
        
    \end{minipage}
    \begin{minipage}[c]{\blocksize}

        \vspace{\linespace}
        \item[まとめ]~\\
        % まとめ
        ちょいとそこ\ruby{行}{ゆ}く\ruby{寮生}{りょう|せい}さん\\
        クラーク\ruby{精神}{せい|しん}\ruby{胸}{むね}に\ruby{秘}{ひ}め\\
        \ruby{天下}{てん|か}の\ruby{北大}{ほく|だい}\ruby[g]{恵迪}{りょう}でもつ\\
        これぞ\ruby[g]{寮生}{おとこ}の\ruby{生}{い}きる\ruby{道}{みち}
    
    \end{minipage}
\end{enumerate} % 番号の箇条書き ここまで

\begin{flushright}
    (※ 前口上は島倉朝雄君の作による)
\end{flushright}

%%%%% 歌詞 ここまで %%%%%
% end body

\end{document}

\documentclass[10pt,b5j]{tarticle} % B6 縦書き
% \documentclass[10pt,b5j]{tarticle} % B6 縦書き
\AtBeginDvi{\special{papersize=128mm,182mm}} % B6 用用紙サイズ
\usepackage{otf} % Unicode で字を入力するのに必要なパッケージ
\usepackage[size=b6j]{bxpapersize} % B6 用紙サイズを指定
\usepackage[dvipdfmx]{graphicx} % 画像を挿入するためのパッケージ
\usepackage[dvipdfmx]{color} % 色をつけるためのパッケージ
\usepackage{pxrubrica} % ルビを振るためのパッケージ
\usepackage{plext} % 漢数字の enumerate を使うためのパッケージ
\usepackage{multicol} % 複数段組を作るためのパッケージ
\setlength{\topmargin}{14mm} % 上下方向のマージン
\addtolength{\topmargin}{-1in} % 
\setlength{\oddsidemargin}{11mm} % 左右方向のマージン
\addtolength{\oddsidemargin}{-1in} % 
\setlength{\textwidth}{154mm} % B6 用
\setlength{\textheight}{108mm} % B6 用
\setlength{\headsep}{0mm} % 
\setlength{\headheight}{0mm} % 
\setlength{\topskip}{0mm} % 
\setlength{\parskip}{0pt} % 
\def\theenumi{\Kanji{enumi}} % 箇条書きのフォーマットを漢数字に変更
\parindent = 0pt % 段落下げしない
\pagestyle{empty} % すべてのページ番号を消去
% \renewcommand{\baselinestretch}{0.9} % 行間の倍率
 % B6 用テンプレート読み込み

\begin{document}
% begin header
%%%%% タイトルと作者 ここから %%%%%
\begin{minipage}[c]{0.7\hsize} % タイトルは上から 7 割
    \begin{center}
    % begin title
        {\LARGE
            寒気身を刺す % タイトルを入れる
        }
        {\small 
            (昭和四十二年寮歌) % 年などを入れる
        }
    % end title
    \end{center}
\end{minipage}
\begin{minipage}[c]{0.3\hsize} % 作歌作曲は上から 3 割
    \begin{flushright} % 下寄せにする
        % begin name
        岡田雄三君 作歌\\森田弘彦君 作曲 % 作歌・作曲者
        % end name
    \end{flushright}
\end{minipage}
%%%%% タイトルと作者 ここまで %%%%%
% (1,2,3 了あり)
% end header

% begin length
\vspace{1.5em} % タイトル, 作者と歌詞の間に隙間を設ける
\newcommand{\linespace}{0.5em} % 行間の設定
\newcommand{\blocksize}{0.5\hsize} % 段組間の設定
\newcommand{\itemmargin}{3em} % 曲番の位置調整の長さ
% end length
% begin body
%%%%% 歌詞 ここから %%%%%
\begin{enumerate} % 番号の箇条書き ここから
    \setlength{\itemindent}{\itemmargin} % 曲番の位置調整
    \begin{minipage}[c]{\blocksize}
    
        \vspace{\linespace}
        \item~\\
        % 1.
        \ruby{寒気}{かん|き}\ruby{身}{み}を\ruby{刺}{さ}す\ruby{北国}{きた|ぐに}の\\
        \ruby[g]{永遠}{とわ}に\ruby{名}{な}を\ruby{覇}{は}す\ruby{恵迪寮}{けい|てき|りょう}\\
        \ruby{四百}{よん|ひゃく}\ruby{野人}{や|じん}の\ruby{集}{つど}いしに\\
        \ruby{我等}{われ|ら}が\ruby[g]{理想}{ロマン}\ruby[g]{何時}{いつ}の\ruby{日}{ひ}か\\
        \ruby{成}{な}さざらむとぞ\ruby{意気}{い|き}\ruby{高}{たか}し
        
        \vspace{\linespace}
        \item~\\
        % 2.
        \ruby{窈窕}{よう|ちょう}\ruby{多}{おお}し\ruby{札幌}{さっ|ぽろ}に\\
        \ruby{弊衣破帽}{へい|い|は|ぼう}の\ruby{身}{み}なれども\\
        \ruby{一度}{ひと|たび}\ruby{歌}{うた}わば\ruby{蛮声}{ばん|せい}の\\
        \ruby{遠}{とお}く\ruby{手稲}{て|いね}に\ruby{木霊}{こ|だま}して\\
        \ruby{嗚呼}{あ|あ}\ruby{誰}{だれ}か\ruby{知}{し}る\ruby{吾}{わ}が\ruby{野心}{や|しん}
        
    \end{minipage}
    \begin{minipage}[c]{\blocksize}
        
        \vspace{\linespace}
        \item~\\
        % 3.
        \ruby{燃}{も}ゆる\ruby{紅}{くれない}\ruby{原始林}{げん|し|りん}\\
        \ruby{尽}{つ}きぬ\ruby{想}{おも}いを\ruby[g]{酒杯}{さかずき}に\\
        \ruby{酔}{よ}えば\ruby{肩}{かた}\ruby{取}{と}り\ruby{乱舞}{らん|ぶ}する\\
        \ruby{吾等}{われ|ら}が\ruby[g]{行先}{ゆくて}に\ruby[g]{光明}{ひかり}あり\\
        \ruby{楽}{たの}しからずや\ruby{此}{こ}の\ruby[g]{饗宴}{うたげ}
        
        \vspace{\linespace}
        \item~\\
        % 4.
        \ruby{蒼空}{そう|くう}の\ruby{下}{もと}\ruby{佇}{たたず}みて\\
        \ruby{木}{こ}の\ruby{葉}{は}\ruby{身}{み}に\ruby{降}{ふ}る\ruby{秋}{あき}の\ruby{日}{ひ}に\\
        \ruby{仮}{たと}いこの\ruby{身}{み}は\ruby{一介}{いっ|かい}の\\
        \ruby{卑}{いや}しきものと\ruby{知}{し}るとても\\
        \ruby{吾}{わ}が\ruby{野望}{や|ぼう}は\ruby{永遠}{えい|えん}に
    
    \end{minipage}
\end{enumerate} % 番号の箇条書き ここまで
%%%%% 歌詞 ここまで %%%%%
% end body

\end{document}

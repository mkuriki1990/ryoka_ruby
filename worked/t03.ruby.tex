\documentclass[10pt,b5j]{tarticle} % B6 縦書き
\usepackage[utf8]{inputenc}
% \documentclass[10pt,b5j]{tarticle} % B6 縦書き
\AtBeginDvi{\special{papersize=128mm,182mm}} % B6 用用紙サイズ
\usepackage{otf} % Unicode で字を入力するのに必要なパッケージ
\usepackage[size=b6j]{bxpapersize} % B6 用紙サイズを指定
\usepackage[dvipdfmx]{graphicx} % 画像を挿入するためのパッケージ
\usepackage[dvipdfmx]{color} % 色をつけるためのパッケージ
\usepackage{pxrubrica} % ルビを振るためのパッケージ
\usepackage{multicol} % 複数段組を作るためのパッケージ
\setlength{\topmargin}{14mm} % 上下方向のマージン
\addtolength{\topmargin}{-1in} % 
\setlength{\oddsidemargin}{11mm} % 左右方向のマージン
\addtolength{\oddsidemargin}{-1in} % 
\setlength{\textwidth}{154mm} % B6 用
\setlength{\textheight}{108mm} % B6 用
\setlength{\headsep}{0mm} % 
\setlength{\headheight}{0mm} % 
\setlength{\topskip}{0mm} % 
\setlength{\parskip}{0pt} % 
\def\labelenumi{\theenumi、} % 箇条書きのフォーマット
\parindent = 0pt % 段落下げしない

 % B6 用テンプレート読み込み

\begin{document}
% begin header
%%%%% タイトルと作者 ここから %%%%%
\begin{minipage}[c]{0.7\hsize} % タイトルは上から 7 割
    \begin{center}
    % begin title
        {\LARGE
            我が運命こそ % タイトルを入れる
        }
        {\small 
            (大正三年寮歌) % 年などを入れる
        }
    % end title
    \end{center}
\end{minipage}
\begin{minipage}[c]{0.3\hsize} % 作歌作曲は上から 3 割
    \begin{flushright} % 下寄せにする
        % begin name
        樋口桜五君 作歌\\赤木顕次君 作曲 % 作歌・作曲者
        % end name
    \end{flushright}
\end{minipage}
%%%%% タイトルと作者 ここまで %%%%%
% (1,2,3,4 了あり)
% end header

% begin length
\vspace{1.5em} % タイトル, 作者と歌詞の間に隙間を設ける
\newcommand{\linespace}{0.5em} % 行間の設定
\newcommand{\blocksize}{0.5\hsize} % 段組間の設定
\newcommand{\itemmargin}{3em} % 曲番の位置調整の長さ
% end length
% begin body
%%%%% 歌詞 ここから %%%%%
\begin{enumerate} % 番号の箇条書き ここから
    \setlength{\itemindent}{\itemmargin} % 曲番の位置調整
    \begin{minipage}[c]{\blocksize}
    
        \vspace{\linespace}
        \item~\\
        % 1.
        \ruby{我}{わ}が\ruby[g]{運命}{さだめ}こそ\ruby{青}{あお}\ruby{渦}{うづ}わける\\
        \ruby{千}{ち}ひろの\ruby{海}{うみ}の\ruby{真珠}{ま|たま}\ruby{取}{と}り\\
        \ruby{美}{うまし}\ruby{想}{おもひ}にあこがるる\ruby{身}{み}は\\
        \ruby[g]{驕楽}{おごり}の\ruby{春}{はる}に\ruby{酔}{ゑ}ひしれて\\
        \ruby{戯}{たはる}る\ruby{人}{ひと}を\ruby{夢}{ゆめ}とはみつつ\\
        \ruby{逆}{さか}まく\ruby{波}{なみ}を\ruby{闡}{ひら}きゆく
        
        \vspace{\linespace}
        \item~\\
        % 2.
        \ruby[g]{永遠}{とは}に\ruby{華}{はな}さく\ruby{水底}{みな|そこ}ふかく\\
        \ruby[g]{神秘}{くしび}の\ruby{巌}{いは}に\ruby[g]{嫦娥}{つきひめ}の\\
        \ruby{露}{つゆ}のしづくの\ruby[g]{真珠}{しらたま}またま\\
        \ruby{掌}{て}に\ruby{獲}{え}し\ruby[g]{光栄}{はえ}と\ruby[g]{喜悦}{よろこび}と\\
        \ruby{七重}{なな|へ}の\ruby{潮}{しほ}の\ruby{妙音}{たへ|ね}にひびく\\
        \ruby[g]{美珠}{たま}こそわれの\ruby[g]{生命}{いのち}なれ
        
    \end{minipage}
    \begin{minipage}[c]{\blocksize}
        
        \vspace{\linespace}
        \item~\\
        % 3.
        \ruby{薫}{くゆ}る\ruby{樹陰}{こ|かげ}に\ruby{花}{はな}\ruby{仄}{ほの}みえて\\
        \ruby{朧}{おぼろ}おぼろの\ruby{春}{はる}の\ruby[g]{{\CID{13831}}}{よひ}\\ % CID 宵󠄁
        \ruby{一壺}{いっ|こ}の\ruby{酒}{さけ}の\ruby{汲}{く}む\ruby{夢}{ゆめ}\ruby{淡}{あわ}く\\
        \ruby{心}{こころ}の\ruby{酔}{よひ}に\ruby{舞歌}{まひ|うた}を\\
        \ruby[g]{社会}{ひとのよ}\ruby{高}{たか}くしらべ\ruby{祝}{いは}はむ\\
        \ruby{君}{きみ}\ruby{瑞祥}{ずい|しやう}の\ruby{歳}{とし}なれや
        
        \vspace{\linespace}
        \item~\\
        % 4.
        \ruby{彩雲}{あや|ぐも}\ruby{低}{ひく}く\ruby{恵}{めぐみ}の\ruby{家}{いえ}に\\
        \ruby{幸}{さち}\ruby[g]{漂蕩}{ただよ}ひてゆく\ruby{水}{みづ}や\\
        \ruby{姿}{すがた}うるほす\ruby{柳}{やなぎ}の\ruby{萠黄}{もえ|ぎ}\\
        \ruby{契}{ちぎ}りゆかしき\ruby{春鳥}{はる|とり}の\\
        \ruby[g]{団欒}{まどゐ}の\ruby{音}{ね}をばうつし\ruby{伝}{つた}へむ\\
        \ruby{遠}{とほ}くはるけき\ruby{師}{し}の\ruby{君}{きみ}に
    
    \end{minipage}
\end{enumerate} % 番号の箇条書き ここまで
%%%%% 歌詞 ここまで %%%%%
% end body

\end{document}

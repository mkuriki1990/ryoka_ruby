\documentclass[10pt,b5j]{tarticle} % B6 縦書き
% \documentclass[10pt,b5j]{tarticle} % B6 縦書き
\AtBeginDvi{\special{papersize=128mm,182mm}} % B6 用用紙サイズ
\usepackage{otf} % Unicode で字を入力するのに必要なパッケージ
\usepackage[size=b6j]{bxpapersize} % B6 用紙サイズを指定
\usepackage[dvipdfmx]{graphicx} % 画像を挿入するためのパッケージ
\usepackage[dvipdfmx]{color} % 色をつけるためのパッケージ
\usepackage{pxrubrica} % ルビを振るためのパッケージ
\usepackage{multicol} % 複数段組を作るためのパッケージ
\setlength{\topmargin}{14mm} % 上下方向のマージン
\addtolength{\topmargin}{-1in} % 
\setlength{\oddsidemargin}{11mm} % 左右方向のマージン
\addtolength{\oddsidemargin}{-1in} % 
\setlength{\textwidth}{154mm} % B6 用
\setlength{\textheight}{108mm} % B6 用
\setlength{\headsep}{0mm} % 
\setlength{\headheight}{0mm} % 
\setlength{\topskip}{0mm} % 
\setlength{\parskip}{0pt} % 
\def\labelenumi{\theenumi、} % 箇条書きのフォーマット
\parindent = 0pt % 段落下げしない

 % B6 用テンプレート読み込み

\begin{document}
% begin header
%%%%% タイトルと作者 ここから %%%%%
\begin{minipage}[c]{0.7\hsize} % タイトルは上から 7 割
    \begin{center}
    % begin title
        {\LARGE
            あますなく拓きゆく道 % タイトルを入れる
        }
        {\small 
            (昭和十七年大東亜戦争頌歌) % 年などを入れる
        }
    % end title
    \end{center}
\end{minipage}
\begin{minipage}[c]{0.3\hsize} % 作歌作曲は上から 3 割
    \begin{flushright} % 下寄せにする
        % begin name
        切替辰哉君 作歌\\池田政晴君 作曲 % 作歌・作曲者
        % end name
    \end{flushright}
\end{minipage}
%%%%% タイトルと作者 ここまで %%%%%
% (1,7 繰り返しなし)
% end header

% begin length
\vspace{0.5em} % タイトル, 作者と歌詞の間に隙間を設ける
\newcommand{\linespace}{0.5em} % 行間の設定
\newcommand{\blocksize}{0.25\hsize} % 段組間の設定
\newcommand{\itemmargin}{3em} % 曲番の位置調整の長さ
% end length
% begin body
%%%%% 歌詞 ここから %%%%%
\begin{enumerate} % 番号の箇条書き ここから
    \setlength{\itemindent}{\itemmargin} % 曲番の位置調整
    \begin{minipage}[c]{\blocksize}
    
        \vspace{\linespace}
        \item~\\
        % 1.
        あますなく\ruby{拓}{ひら}きゆく\ruby{道}{みち}\\
        \ruby{天雲}{あま|ぐも}の\ruby{向伏}{むか|ぶ}す\ruby{極}{きわ}み\\
        \ruby{地}{ち}の\ruby{涯}{はて}ゆ、\ruby{征}{ゆ}かむ\ruby{御楯}{み|たて}と\\
        \ruby[g]{大詔}{みこと}もち、\ruby{我等}{われ|ら}\ruby{日}{ひ}の\ruby{族}{ぞう}\\
        \ruby[g]{源泉}{みなもと}のごと\ruby{湧}{わ}きたたむ\\
        \ruby{誇}{ほこ}らかに\ruby{諸声}{もろ|ごえ}に\\
        \ruby{血潮}{ち|しほ}\ruby{流}{なが}さむ
        
        \vspace{\linespace}
        \item~\\
        % 2.
        \ruby[g]{悠久}{いやはて}の\ruby{天詔琴}{あめ|みのり|ごと}\\
        \ruby{今}{いま}ぞ\ruby{時}{とき}、\ruby{轟}{とどろ}き\ruby{赴}{おもむ}きぬ\\
        \ruby{高光}{たか|ひか}り\ruby{剣}{けん}を\ruby{植}{う}ゑて\\
        \ruby{荒魂}{あら|たま}の\ruby{魂}{たま}にぞ\ruby{生}{い}きむ\\
        \ruby{遷}{うつ}るべく\ruby{遷}{うつ}る\ruby{亜細亜}{あ|じ|あ}の\\
        \ruby{峻}{けは}しかる\ruby{大}{おほ}いなる\\
        \ruby{秋}{とき}に\ruby{生}{うま}れし
        
    \end{minipage}
    \begin{minipage}[c]{\blocksize}
        
        \vspace{\linespace}
        \item~\\
        % 3.
        どよめきぬ\ruby[g]{祖霊}{たまおや}の\ruby{行}{わき}\\
        \ruby{六合}{りく|がふ}に\ruby{頸}{つよ}く\ruby{漲}{みな}ぎり\\
        \ruby{天津日}{あま|つ|ひ}は\ruby{紅}{くれなゐ}\ruby{燃}{も}ゆる\\
        \ruby[g]{南方圏}{みんなみ}の\ruby[g]{洋路}{うみぢ}\ruby{遙}{はろ}けく\\
        \ruby[g]{秀麗}{うるは}しき\ruby{創成}{さう|せい}の\ruby{神意}{しん|い}\\
        \ruby{重}{おも}く\ruby{負}{お}ふに\ruby{務}{つと}めして\\
        \ruby[g]{生命}{いのち}たぎちむ
        
        \vspace{\linespace}
        \item~\\
        % 4.
        \ruby{欣求}{ごん|ぐ}の\ruby{宇宙}{う|ちう}\ruby{蝕変}{しょく|へん}\ruby{満}{み}つも\\
        \ruby{東亜}{とう|あ}の\ruby{空}{そら}、\ruby{復円}{ふく|ゑん}\ruby{光}{こ}らん\\
        \ruby{斯}{か}くせずばやまぬ\ruby[g]{宿命}{さだめ}と\\
        \ruby{十億}{じゅう|おく}の\ruby[g]{健剛}{たけき}を\ruby{{\UTF{79B1}}}{の}みて\\ % UTF 禱
        \ruby{国}{くに}\ruby{挙}{こぞ}り\ruby{歩}{あゆ}みゆくなり\\
        \ruby[g]{熱涙}{なみだ}もて\ruby{仰}{あお}がなむ\\
        \ruby[g]{黎明}{あけ}の\ruby{幸星}{こう|せい}
        
    \end{minipage}
    \begin{minipage}[c]{\blocksize}
        
        \vspace{\linespace}
        \item~\\
        % 5.
        \ruby{帰}{かえ}るなき\ruby{発程}{はつ|てい}に\ruby{起}{た}つ\\
        \ruby{眸}{まみ}\ruby{澄}{す}める\ruby{我等}{われ|ら}\ruby[g]{若人}{わこうど}\\
        \ruby{皇国}{こう|こく}の\ruby{道}{みち}に\ruby[g]{挺身}{すす}まん\\
        \ruby{諸共}{もろ|とも}に\ruby{雄叫}{を|たけ}びすれば\\
        \ruby{叫}{さけ}び\ruby{和}{わ}す\ruby{新潮}{にひ|しほ}の\ruby{声}{こえ}\\
        \ruby{抒情}{じょ|じゃう}\ruby{清}{さや}か、\ruby{白鳥}{はく|てふ}の\\
        \ruby{海図}{かい|づ}に\ruby{夢}{ゆめ}む
        
        \vspace{\linespace}
        \item~\\
        % 6.
        \ruby{厳}{おごそ}かの\ruby{時}{とき}の\ruby{流}{なが}れに\\
        \ruby{新}{あたら}しき\ruby{力}{ちから}よ\ruby{躍}{おど}れ\\
        \ruby{鮮}{あざや}けき\ruby{翳}{かけ}りの\ruby{中}{なか}に\\
        \ruby{新}{あたら}しき\ruby{叫}{さけび}よ\ruby{挙}{あ}がれ\\
        \ruby[g]{胸臆}{こと}\ruby{朗}{ほが}ら、\ruby{身}{み}を\ruby{透}{と}けて\ruby{佇}{た}つ\\
        \ruby{揺}{ゆる}ぎなく、\ruby{鍛}{きた}へして\\
        \ruby{先駆}{せん|く}に\ruby{埋}{う}めん
        
    \end{minipage}
    \begin{minipage}[c]{\blocksize}
        
        \vspace{\linespace}
        \item~\\
        % 7.
        ここぞ\ruby{茲}{ここ}、いかで\ruby{忘}{わす}れむ\\
        \ruby{日}{ひ}に\ruby{若}{わか}き、\ruby{恵迪}{けい|てき}の\ruby{児}{こ}よ\\
        たどり\ruby{得}{え}し\ruby{道}{みち}の\ruby[g]{感喜}{よろこび}\\
        \ruby{溢}{あふ}れつつ、ほの\ruby{認}{みと}めけむ\\
        \ruby{仰}{あお}ぎ\ruby{見}{み}る\ruby{銀漢}{ぎん|が}のほとり\\
        \ruby[g]{真実}{まこと}もて、\ruby[g]{弥生}{いやお}ひに\\
        \ruby{継}{つ}ぎて\ruby{行}{ゆ}かなむ
    
    \end{minipage}
\end{enumerate} % 番号の箇条書き ここまで
%%%%% 歌詞 ここまで %%%%%
% end body

\end{document}

\documentclass[10pt,b5j]{tarticle} % B6 縦書き
% \documentclass[10pt,b5j]{tarticle} % B6 縦書き
\AtBeginDvi{\special{papersize=128mm,182mm}} % B6 用用紙サイズ
\usepackage{otf} % Unicode で字を入力するのに必要なパッケージ
\usepackage[size=b6j]{bxpapersize} % B6 用紙サイズを指定
\usepackage[dvipdfmx]{graphicx} % 画像を挿入するためのパッケージ
\usepackage[dvipdfmx]{color} % 色をつけるためのパッケージ
\usepackage{pxrubrica} % ルビを振るためのパッケージ
\usepackage{multicol} % 複数段組を作るためのパッケージ
\setlength{\topmargin}{14mm} % 上下方向のマージン
\addtolength{\topmargin}{-1in} % 
\setlength{\oddsidemargin}{11mm} % 左右方向のマージン
\addtolength{\oddsidemargin}{-1in} % 
\setlength{\textwidth}{154mm} % B6 用
\setlength{\textheight}{108mm} % B6 用
\setlength{\headsep}{0mm} % 
\setlength{\headheight}{0mm} % 
\setlength{\topskip}{0mm} % 
\setlength{\parskip}{0pt} % 
\def\labelenumi{\theenumi、} % 箇条書きのフォーマット
\parindent = 0pt % 段落下げしない

 % B6 用テンプレート読み込み

\begin{document}
% begin header
%%%%% タイトルと作者 ここから %%%%%
\begin{minipage}[c]{0.7\hsize} % タイトルは上から 7 割
    \begin{center}
    % begin title
        {\LARGE
            黒潮鳴れる % タイトルを入れる
        }
        {\small 
            (昭和四年寮歌) % 年などを入れる
        }
    % end title
    \end{center}
\end{minipage}
\begin{minipage}[c]{0.3\hsize} % 作歌作曲は上から 3 割
    \begin{flushright} % 下寄せにする
        % begin name
        須田政美君 作歌\\森忠文君 作曲 % 作歌・作曲者
        % end name
    \end{flushright}
\end{minipage}
%%%%% タイトルと作者 ここまで %%%%%
% (1,2,3,6 繰り返しなし)
% end header

% begin length
\vspace{1.5em} % タイトル, 作者と歌詞の間に隙間を設ける
\newcommand{\linespace}{0.5em} % 行間の設定
\newcommand{\blocksize}{0.5\hsize} % 段組間の設定
\newcommand{\itemmargin}{3em} % 曲番の位置調整の長さ
% end length
% begin body
%%%%% 歌詞 ここから %%%%%
\begin{enumerate} % 番号の箇条書き ここから
    \setlength{\itemindent}{\itemmargin} % 曲番の位置調整
    \begin{minipage}[c]{\blocksize}
    
        \vspace{\linespace}
        \item~\\
        % 1.
        \ruby{黒潮}{くろ|しお}\ruby{鳴}{な}れる\ruby[g]{滄海}{わだつみ}\ruby{越}{こ}えて\\
        \ruby[g]{際限}{きは}\ruby{無}{な}き\ruby{春}{はる}を\ruby{北州}{ほく|しゅう}に\ruby{訪}{と}ふ\\
        \ruby{原始}{げん|し}の\ruby[g]{大森}{もり}に\ruby{八光}{はっ|こう}\ruby{揺}{ゆら}ぎ\\
        \ruby{若草}{わか|くさ}の\ruby[g]{曠野}{の}に\ruby{羊群}{よう|ぐん}\ruby{遊}{あそ}ぶ
        
        \vspace{\linespace}
        \item~\\
        % 2.
        \ruby[g]{情懐}{こころ}は\ruby[g]{朧月}{つき}に\ruby{仄}{ほの}かに\ruby{薫}{かほ}る\\
        アカシヤの\ruby[g]{白花}{はな}\ruby{慕}{した}ひて\ruby{歩}{あゆ}む\\
        \ruby{恋}{こい}ふる\ruby[g]{往昔}{むかし}の\ruby[g]{静寂}{しづ}けき\ruby[g]{名残}{なご}り\\
        \ruby{古塔}{こ|とう}にひびく\ruby{懐}{なつか}しき\ruby{鐘}{かね}
        
        \vspace{\linespace}
        \item~\\
        % 3.
        \ruby[g]{紅光}{くれなゐ}うすくエルムに\ruby{映}{は}えて\\
        \ruby{草笛}{くさ|ぶえ}かそかに\ruby{牧場}{まき|ば}にながる\\
        \ruby[g]{漂泊}{さす}らひ\ruby{行}{ゆ}ける\ruby[g]{白雲}{くも}\ruby{影}{かげ}\ruby{仰}{あお}ぎ\\
        \ruby{無心}{む|しん}の\ruby[g]{若人}{こ}らは\ruby{緑}{みどり}に\ruby{臥}{ふ}せり
        
    \end{minipage}
    \begin{minipage}[c]{\blocksize}
        
        \vspace{\linespace}
        \item~\\
        % 4.
        \ruby{果}{はて}\ruby{無}{な}き\ruby[g]{憧憬}{のぞみ}\ruby{銀河}{ぎん|が}に\ruby{寄}{よ}せて\\
        \ruby{玻璃}{は|り}\ruby{永劫}{えい|ごう}の\ruby{清}{きよ}き\ruby{夜空}{よ|ぞら}を\\
        \ruby[g]{神秘}{くしび}の\ruby[g]{皓翼}{つばさ}\ruby{声}{こえ}なく\ruby{衝}{う}ちつ\\
        \ruby{我等}{われ|ら}が\ruby[g]{高夢}{ゆめ}は\ruby{流}{なが}れゆくかな
        
        \vspace{\linespace}
        \item~\\
        % 5.
        \ruby{淋}{さび}しき\ruby[g]{風声}{こえ}に\ruby[g]{銀雪}{ゆき}は\ruby{乱}{みだ}れつ\\
        \ruby{大空}{おお|ぞら}\ruby{鳴}{な}りて\ruby[g]{渾瞑}{くら}く\ruby{暮}{く}れゆく\\
        \ruby{燦}{きら}めく\ruby[g]{灯影}{ほかげ}\ruby{常春}{とこ|はる}の\ruby[g]{謳歌}{うた}\\
        \ruby{血潮}{ち|しお}と\ruby{共}{とも}に\ruby{尚}{なほ}\ruby{湧}{わ}き\ruby{立}{た}てり
        
        \vspace{\linespace}
        \item~\\
        % 6.
        \ruby[g]{久遠}{くおん}の\ruby[g]{絢夢}{ゆめ}はうづもれゆきて\\
        \ruby[g]{哀愁}{かなしみ}\ruby{時}{とき}にしづかに\ruby{来}{く}れど\\
        \ruby{雄}{たけ}き「\ruby{自然}{し|ぜん}」と「\ruby{血潮}{ち|しお}」の\ruby{人}{ひと}は\\
        \ruby[g]{楡陵}{おか}に\ruby{永}{なが}くうつくしく\ruby{立}{た}つ
    
    \end{minipage}
\end{enumerate} % 番号の箇条書き ここまで
%%%%% 歌詞 ここまで %%%%%
% end body

\end{document}

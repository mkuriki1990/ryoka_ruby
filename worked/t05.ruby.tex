\documentclass[10pt,b5j]{tarticle} % B6 縦書き
% \documentclass[10pt,b5j]{tarticle} % B6 縦書き
\AtBeginDvi{\special{papersize=128mm,182mm}} % B6 用用紙サイズ
\usepackage{otf} % Unicode で字を入力するのに必要なパッケージ
\usepackage[size=b6j]{bxpapersize} % B6 用紙サイズを指定
\usepackage[dvipdfmx]{graphicx} % 画像を挿入するためのパッケージ
\usepackage[dvipdfmx]{color} % 色をつけるためのパッケージ
\usepackage{pxrubrica} % ルビを振るためのパッケージ
\usepackage{multicol} % 複数段組を作るためのパッケージ
\setlength{\topmargin}{14mm} % 上下方向のマージン
\addtolength{\topmargin}{-1in} % 
\setlength{\oddsidemargin}{11mm} % 左右方向のマージン
\addtolength{\oddsidemargin}{-1in} % 
\setlength{\textwidth}{154mm} % B6 用
\setlength{\textheight}{108mm} % B6 用
\setlength{\headsep}{0mm} % 
\setlength{\headheight}{0mm} % 
\setlength{\topskip}{0mm} % 
\setlength{\parskip}{0pt} % 
\def\labelenumi{\theenumi、} % 箇条書きのフォーマット
\parindent = 0pt % 段落下げしない

 % B6 用テンプレート読み込み

\begin{document}
% begin header
%%%%% タイトルと作者 ここから %%%%%
\begin{minipage}[c]{0.7\hsize} % タイトルは上から 7 割
    \begin{center}
    % begin title
        {\LARGE
            穹蒼高く % タイトルを入れる
        }
        {\small 
            (大正五年南寮寮歌) % 年などを入れる
        }
    % end title
    \end{center}
\end{minipage}
\begin{minipage}[c]{0.3\hsize} % 作歌作曲は上から 3 割
    \begin{flushright} % 下寄せにする
        % begin name
        長崎次郎君 作歌\\黒住須賀夫君 作曲 % 作歌・作曲者
        % end name
    \end{flushright}
\end{minipage}
%%%%% タイトルと作者 ここまで %%%%%
% (1,2,3 了あり)
% end header

% begin length
\vspace{1.5em} % タイトル, 作者と歌詞の間に隙間を設ける
\newcommand{\linespace}{0.5em} % 行間の設定
\newcommand{\blocksize}{0.33\hsize} % 段組間の設定
\newcommand{\itemmargin}{3em} % 曲番の位置調整の長さ
% end length
% begin body
%%%%% 歌詞 ここから %%%%%
\begin{enumerate} % 番号の箇条書き ここから
    \setlength{\itemindent}{\itemmargin} % 曲番の位置調整
    \begin{minipage}[c]{\blocksize}
    
        \vspace{\linespace}
        \item~\\
        % 1.
        \ruby[g]{穹蒼}{おほぞら}\ruby{高}{たか}く\ruby{夜}{よ}は\ruby{深}{ふか}く\\
        \ruby[g]{沈黙}{しじま}の\ruby{森}{もり}に\ruby{聳}{そび}えたつ\\
        \ruby{桂}{かつら}の\ruby{梢}{こずゑ}\ruby{指}{さ}すところ\\
        \ruby{北斗}{ほく|と}の\ruby{冴}{さえ}に\ruby{君}{きみ}\ruby{見}{み}ずや\\
        「\ruby{吾}{わ}が\ruby[g]{若人}{わこうど}よ\ruby{汝}{な}が\ruby[g]{野心}{のぞみ}\\
        われにかも\ruby{似}{に}て\ruby{崇}{たか}くあれ」
        
        \vspace{\linespace}
        \item~\\
        % 2.
        \ruby{荒}{すさ}ぶ\ruby[g]{吹雪}{ふぶき}のもだすとき\\
        \ruby{六片}{りっ|ぺん}の\ruby{花}{はな}\ruby{咲}{さ}くところ\\
        \ruby{皎}{こう}たる\ruby{天地}{てん|ち}\ruby{塵}{ちり}\ruby{絶}{た}えて\\
        \ruby{塞}{み}つる\ruby{力}{ちから}を\ruby{君}{きみ}よ\ruby{知}{し}れ\\
        「\ruby{吾}{わ}が\ruby[g]{若人}{わこうど}よ\ruby{北}{きた}の\ruby[g]{曠野}{の}に\\
        \ruby{身}{み}を\ruby{練}{ね}り\ruby{魂}{たま}を\ruby{磨}{みが}かずや」
        
    \end{minipage}
    \begin{minipage}[c]{\blocksize}
        
        \vspace{\linespace}
        \item~\\
        % 3.
        \ruby{谷間}{たに|ま}の\ruby[g]{百合}{ゆり}の\ruby{香}{か}のゆらぎ\\
        \ruby{楡}{にれ}の\ruby{若葉}{わか|ば}に\ruby{陽}{ひ}はこぼる\\
        \ruby{春}{はる}の\ruby[g]{息吹}{いぶき}に\ruby{渡}{わた}り\ruby{行}{ゆ}く\\
        \ruby[g]{時鐘}{かね}の\ruby{響}{ひびき}に\ruby{君}{きみ}よ\ruby{聴}{き}け\\
        「\ruby{吾}{わ}が\ruby[g]{若人}{わこうど}よ\ruby{石狩}{いし|かり}は\\
        \ruby{自由}{じ|ゆう}の\ruby[g]{郷土}{くに}ぞ\ruby{幸}{さち}\ruby{多}{おお}き」
        
        \vspace{\linespace}
        \item~\\
        % 4.
        \ruby{百鳥}{もも|どり}\ruby{歌}{うた}ひ\ruby{花}{はな}は\ruby{笑}{え}む\\
        \ruby{美}{うま}しき\ruby{国}{くに}の\ruby{自治}{じ|ち}の\ruby{家}{や}に\\
        \ruby{十一}{じゅう|いち}の\ruby{春}{はる}\ruby[g]{今日}{けふ}\ruby{来}{きた}る\\
        \ruby[g]{祝歌}{はぎうた}たかく\ruby{君}{きみ}\ruby{歌}{うた}へ\\
        「\ruby{迪}{みち}に\ruby{恵}{したが}ふ\ruby[g]{若人}{わこうど}の\\
        \ruby{住家}{すみ|か}よ\ruby{永}{とは}に\ruby{栄}{さかえ}あれ」
        
    \end{minipage}
    \begin{minipage}[c]{\blocksize}
        
        \vspace{\linespace}
        \item~\\
        % 5.
        \ruby{崇}{たか}きのぞみを\ruby{星}{ほし}に\ruby{懸}{か}け\\
        \ruby{鐘}{かね}に\ruby{自由}{じ|ゆう}を\ruby{学}{まな}びつつ\\
        \ruby[g]{真理}{まこと}を\ruby{求}{もと}むる\ruby{一百}{いっ|ぴゃく}の\\
        \ruby{健児}{けん|じ}が\ruby{行手}{ゆく|て}\ruby{遠}{とほ}けれど\\
        \ruby{吾}{われ}\ruby{若}{わか}き\ruby{力}{ちから}\ruby{強}{つよ}ければ\\
        \ruby{贏}{かちえ}む\ruby{秋}{とき}は\ruby{近}{ちか}からむ\\
        など\ruby{贏}{かちえ}ざる\ruby{事}{こと}あらん
    
    \end{minipage}
\end{enumerate} % 番号の箇条書き ここまで
%%%%% 歌詞 ここまで %%%%%
% end body

\end{document}

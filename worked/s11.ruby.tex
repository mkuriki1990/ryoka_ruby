\documentclass[10pt,b5j]{tarticle} % B6 縦書き
% \documentclass[10pt,b5j]{tarticle} % B6 縦書き
\AtBeginDvi{\special{papersize=128mm,182mm}} % B6 用用紙サイズ
\usepackage{otf} % Unicode で字を入力するのに必要なパッケージ
\usepackage[size=b6j]{bxpapersize} % B6 用紙サイズを指定
\usepackage[dvipdfmx]{graphicx} % 画像を挿入するためのパッケージ
\usepackage[dvipdfmx]{color} % 色をつけるためのパッケージ
\usepackage{pxrubrica} % ルビを振るためのパッケージ
\usepackage{multicol} % 複数段組を作るためのパッケージ
\setlength{\topmargin}{14mm} % 上下方向のマージン
\addtolength{\topmargin}{-1in} % 
\setlength{\oddsidemargin}{11mm} % 左右方向のマージン
\addtolength{\oddsidemargin}{-1in} % 
\setlength{\textwidth}{154mm} % B6 用
\setlength{\textheight}{108mm} % B6 用
\setlength{\headsep}{0mm} % 
\setlength{\headheight}{0mm} % 
\setlength{\topskip}{0mm} % 
\setlength{\parskip}{0pt} % 
\def\labelenumi{\theenumi、} % 箇条書きのフォーマット
\parindent = 0pt % 段落下げしない

 % B6 用テンプレート読み込み

\renewcommand{\baselinestretch}{0.85} % 行間の倍率

\begin{document}
% begin header
%%%%% タイトルと作者 ここから %%%%%
\begin{minipage}[c]{0.7\hsize} % タイトルは上から 7 割
    \begin{center}
    % begin title
        {\LARGE
            嗚呼茫々の % タイトルを入れる
        }
        {\small 
            (昭和十一年寮歌) % 年などを入れる
        }
    % end title
    \end{center}
\end{minipage}
\begin{minipage}[c]{0.3\hsize} % 作歌作曲は上から 3 割
    \begin{flushright} % 下寄せにする
        % begin name
        宍戸昌夫君 作歌\\村岡五郎君 作曲 % 作歌・作曲者
        % end name
    \end{flushright}
\end{minipage}
%%%%% タイトルと作者 ここまで %%%%%
% (1 繰り返しなし)
% end header

% begin length
\vspace{0.2em} % タイトル, 作者と歌詞の間に隙間を設ける
\newcommand{\linespace}{0.2em} % 行間の設定
\newcommand{\blocksize}{0.5\hsize} % 段組間の設定
\newcommand{\itemmargin}{3em} % 曲番の位置調整の長さ
% end length
% begin body
%%%%% 歌詞 ここから %%%%%
\begin{enumerate} % 番号の箇条書き ここから
    \setlength{\itemindent}{9em} % 曲番の位置調整
    
    \item[\ruby{楡陵謳春賦}{ゆ|りょう|おう|しゅん|ふ}]~\\
    \ruby{吾等}{われ|ら}が\ruby{三年}{み|とせ}を\ruby{契}{ちぎ}る\ruby{絢爛}{けん|らん}の
    その\ruby[g]{饗宴}{うたげ}はげに\ruby{過}{す}ぎ\ruby{易}{やす}し。
    \ruby{然}{しか}れども\ruby{見}{み}ずや\ruby{穹北}{きゅう|ほく}に\\
    \ruby{瞬}{またた}く\ruby{星斗}{せい|と}\ruby[g]{永久}{とこしへ}に\ruby{曇}{くも}りなく、
    \ruby{雲}{くも}とまがふ\ruby{万朶}{ばん|だ}の\ruby{桜花}{おう|か}\ruby[g]{久遠}{くおん}に\ruby{萎}{な}えざるを。\\
    \ruby[g]{寮友}{とも}よ\ruby{徒}{いたず}らに\ruby{明日}{あ|す}の\ruby[g]{運命}{さだめ}を
    \ruby{歎}{なげ}かんよりは\ruby{楡林}{ゆ|りん}に\ruby{篝火}{かがり|び}を\ruby{焚}{た}きて、 \ruby{去}{さ}りては\\
    \ruby{再}{ふたた}び\ruby{帰}{かえ}らざる
    \ruby{若}{わか}き\ruby{日}{ひ}の\ruby{感激}{かん|げき}を\ruby[g]{謳歌}{うた}はん。
        
    \setlength{\itemindent}{\itemmargin} % 曲番の位置調整
    \begin{minipage}[c]{\blocksize}
        
        \vspace{\linespace}
        \item~\\
        % 1.
        \ruby{嗚呼}{あ|あ}\ruby{茫々}{ぼう|ぼう}の\ruby{大}{だい}\ruby{曠野}{こう|や}\\
        \ruby{先人}{せん|じん}ここに\ruby{芟}{くさぎ}りて\\
        \ruby{建}{た}てし\ruby{自由}{じ|ゆう}と\ruby{自治}{じ|ち}の\ruby{城}{しろ}\\
        その\ruby{源}{みなもと}は\ruby{遠}{とお}くして\\
        \ruby{濁世}{だく|せい}\ruby{叱咤}{しっ|た}す\ruby[g]{六十年}{むそとせ}の\\
        \ruby{苔}{こけ}むす\ruby{青史}{せい|し}\ruby{誇}{ほこ}りなん
        
        \vspace{\linespace}
        \item~\\
        % 2.
        \ruby[g]{老桜}{さくら}の\ruby{蔭}{かげ}や\ruby[g]{北辰}{ほし}の\ruby{下}{もと}\\
        \ruby[g]{少時}{しばし}\ruby{旅寝}{たび|ね}の\ruby{若}{わか}き\ruby{子}{こ}が\\
        \ruby{自治}{じ|ち}\ruby{燈}{とう}かかげ\ruby[g]{聖鐘}{かね}うちて\\
        \ruby[g]{惰眠}{ねむ}れる\ruby{魂}{たま}を\ruby[g]{覚醒}{さま}すべく\\
        \ruby{降魔}{ごう|ま}の\ruby{剣}{けん}かざすとき\\
        \ruby{狂}{くる}へる\ruby{飃}{かぜ}も\ruby{声}{こえ}ひそむ
        
    \end{minipage}
    \begin{minipage}[c]{\blocksize}
        
        \vspace{\linespace}
        \item~\\
        % 3.
        さはれ\ruby{今宵}{こ|よい}の\ruby{我}{わ}が\ruby{寮}{すみか}\\
        「\ruby{人生}{じん|せい}\ruby{意気}{い|き}」に
        \ruby{集}{つど}い\ruby{来}{こ}し\\
        \ruby{結}{むす}びてとけぬ\ruby{友垣}{とも|がき}が\\
        \ruby[g]{光明}{ひかり}と\ruby[g]{権威}{ちから}\ruby{謳}{うた}ふとき\\
        \ruby{星屑}{ほし|くず}\ruby[g]{原始林}{もり}に\ruby{輝}{かがや}きて\\
        \ruby{流転}{る|てん}の\ruby{相}{そう}を\ruby{示}{しめ}すなり
        
        \vspace{\linespace}
        \item~\\
        % 4.
        ああ\ruby{感激}{かん|げき}の\ruby[g]{美酒}{うまさけ}は\\
        \ruby{廻}{まわ}りて\ruby{早}{はや}きその\ruby{三年}{み|とせ}\\
        \ruby[g]{希望}{のぞみ}の\ruby{光}{ひかり}\ruby{恵}{もと}めては\\
        \ruby{楡林}{ゆ|りん}にかはす\ruby{盃}{さかずき}に\\
        \ruby[g]{啓示}{さとし}の\ruby{翳}{かげ}を\ruby{泛}{うか}べつつ\\
        \ruby{男}{お}の\ruby{子}{こ}の\ruby{眸}{まみ}に\ruby{涙}{なみだ}あり
    
    \end{minipage}
\end{enumerate} % 番号の箇条書き ここまで
%%%%% 歌詞 ここまで %%%%%
% end body

\end{document}

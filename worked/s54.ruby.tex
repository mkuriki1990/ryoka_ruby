\documentclass[10pt,b5j]{tarticle} % B6 縦書き
% \documentclass[10pt,b5j]{tarticle} % B6 縦書き
\AtBeginDvi{\special{papersize=128mm,182mm}} % B6 用用紙サイズ
\usepackage{otf} % Unicode で字を入力するのに必要なパッケージ
\usepackage[size=b6j]{bxpapersize} % B6 用紙サイズを指定
\usepackage[dvipdfmx]{graphicx} % 画像を挿入するためのパッケージ
\usepackage[dvipdfmx]{color} % 色をつけるためのパッケージ
\usepackage{pxrubrica} % ルビを振るためのパッケージ
\usepackage{multicol} % 複数段組を作るためのパッケージ
\setlength{\topmargin}{14mm} % 上下方向のマージン
\addtolength{\topmargin}{-1in} % 
\setlength{\oddsidemargin}{11mm} % 左右方向のマージン
\addtolength{\oddsidemargin}{-1in} % 
\setlength{\textwidth}{154mm} % B6 用
\setlength{\textheight}{108mm} % B6 用
\setlength{\headsep}{0mm} % 
\setlength{\headheight}{0mm} % 
\setlength{\topskip}{0mm} % 
\setlength{\parskip}{0pt} % 
\def\labelenumi{\theenumi、} % 箇条書きのフォーマット
\parindent = 0pt % 段落下げしない

 % B6 用テンプレート読み込み

\begin{document}
% begin header
%%%%% タイトルと作者 ここから %%%%%
\begin{minipage}[c]{0.7\hsize} % タイトルは上から 7 割
    \begin{center}
    % begin title
        {\LARGE
            うす紅の % タイトルを入れる
        }
        {\small 
            (昭和五十四年寮歌) % 年などを入れる
        }
    % end title
    \end{center}
\end{minipage}
\begin{minipage}[c]{0.3\hsize} % 作歌作曲は上から 3 割
    \begin{flushright} % 下寄せにする
        % begin name
        鶴原文孝君 作歌\\高田和重君 作曲 % 作歌・作曲者
        % end name
    \end{flushright}
\end{minipage}
%%%%% タイトルと作者 ここまで %%%%%
% (1,2,3,4,5 繰り返しなし)
% end header

% begin length
\vspace{1.5em} % タイトル, 作者と歌詞の間に隙間を設ける
\newcommand{\linespace}{0.5em} % 行間の設定
\newcommand{\blocksize}{0.33\hsize} % 段組間の設定
\newcommand{\itemmargin}{3em} % 曲番の位置調整の長さ
% end length
% begin body
%%%%% 歌詞 ここから %%%%%
\begin{enumerate} % 番号の箇条書き ここから
    \setlength{\itemindent}{\itemmargin} % 曲番の位置調整
    \begin{minipage}[c]{\blocksize}
    
        \vspace{\linespace}
        \item~\\
        % 1.
        うす\ruby{紅}{くれない}の\ruby{秋}{あき}ゆうぐれに\\
        \ruby{滅}{ほろ}びの\ruby{風}{かぜ}は\ruby{吹}{ふ}き\ruby{荒}{すさ}ぶ\\
        \ruby{斜陽}{しゃ|よう}かげ\ruby{射}{さ}す\ruby{日}{ひ}に\ruby{移}{うつ}ろいて\\
        \ruby{傾}{かたぶ}く\ruby{姿}{すがた}\ruby{痛}{いた}ましく\\
        \ruby{我}{わ}が\ruby{胸}{むね}に\ruby{満}{み}つ\ruby{過}{い}にし\ruby{日}{ひ}の\ruby{映}{は}え\\
        \ruby{懐}{おも}いは\ruby{恵迪}{けい|てき}と\ruby{共}{とも}に
        
        \vspace{\linespace}
        \item~\\
        % 2.
        うす\ruby{紫}{むらさき}の\ruby{冬}{ふゆ}あけどきに\\
        \ruby{透}{す}みわたる\ruby{風}{かぜ}\ruby{底}{そこ}\ruby{凍}{こお}る\\
        もの\ruby{音}{おと}\ruby{絶}{た}えて\ruby{冷}{つめ}たく\ruby{寒}{さむ}く\\
        \ruby{暗}{くら}くも\ruby{映}{はゆ}る\ruby{空}{むな}しさに\\
        \ruby{倒}{たお}れゆくもの\ruby{今}{いま}この\ruby{時}{とき}に\\
        \ruby{想}{おも}いは\ruby{恵迪}{けい|てき}と\ruby{共}{とも}に
        
    \end{minipage}
    \begin{minipage}[c]{\blocksize}
        
        \vspace{\linespace}
        \item~\\
        % 3.
        うす\ruby{靄}{もや}けぶる\ruby{春}{はる}あけぼのに\\
        \ruby{昔日}{せき|じつ}の\ruby{影}{かげ}たゆたい\ruby{惑}{まど}う\\
        されど\ruby{緑}{みどり}はまだ\ruby{若}{わか}くして\\
        \ruby{咲}{さ}き\ruby{初}{そ}む\ruby{花}{はな}の\ruby{望}{のぞみ}もて\\
        \ruby{新}{あたら}しき\ruby{日}{ひ}のかげろい\ruby{浮}{う}かぶ\\
        \ruby{憧}{あこが}れ\ruby{恵迪}{けい|てき}と\ruby{共}{とも}に
        
        \vspace{\linespace}
        \item~\\
        % 4.
        うす\ruby{花}{はな}いろの\ruby{夏}{なつ}よい\ruby{闇}{やみ}に\\
        たまゆら\ruby{風}{かぜ}はさわやけし\\
        \ruby{我}{わ}が\ruby{宴}{うたげ}にも\ruby{星}{ほし}\ruby{降}{ふ}る\ruby{幸}{さち}と\\
        \ruby{歌}{うた}う\ruby[g]{寮友}{とも}らの\ruby{嬉}{うれ}しさに\\
        \ruby{憩}{いこ}える\ruby{帆}{ほ}にも\ruby{希}{おも}いありたし\\
        \ruby{夢}{ゆめ}こそ\ruby{恵迪}{けい|てき}と\ruby{共}{とも}に
        
    \end{minipage}
    \begin{minipage}[c]{\blocksize}
        
        \vspace{\linespace}
        \item~\\
        % 5.
        うつろう\ruby[g]{四季}{とき}に\ruby[g]{感慨}{おもい}をこめて\\
        \ruby{朽}{く}ちゆくものを\ruby{見}{み}つめつつ\\
        いまだ\ruby{乾}{かわ}かぬ\ruby{血涙}{けつ|るい}をもて\\
        ただひたすらに\ruby{祈}{いの}り\ruby{捧}{ささ}ぐ\\
        \ruby[g]{唯一}{ただ}\ruby{真実}{しん|じつ}の\ruby{迪}{みち}を\ruby{残}{のこ}さむ\\
        \ruby{想}{おも}いは\ruby{恵迪}{けい|てき}を\ruby[g]{永遠}{とわ}に\\
        \ruby{希}{おも}いは\ruby{恵迪}{けい|てき}よ\ruby[g]{永遠}{とわ}に
    
    \end{minipage}
\end{enumerate} % 番号の箇条書き ここまで
%%%%% 歌詞 ここまで %%%%%
% end body

\end{document}

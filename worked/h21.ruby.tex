\documentclass[10pt,b5j]{tarticle} % B6 縦書き
% \documentclass[10pt,b5j]{tarticle} % B6 縦書き
\AtBeginDvi{\special{papersize=128mm,182mm}} % B6 用用紙サイズ
\usepackage{otf} % Unicode で字を入力するのに必要なパッケージ
\usepackage[size=b6j]{bxpapersize} % B6 用紙サイズを指定
\usepackage[dvipdfmx]{graphicx} % 画像を挿入するためのパッケージ
\usepackage[dvipdfmx]{color} % 色をつけるためのパッケージ
\usepackage{pxrubrica} % ルビを振るためのパッケージ
\usepackage{multicol} % 複数段組を作るためのパッケージ
\setlength{\topmargin}{14mm} % 上下方向のマージン
\addtolength{\topmargin}{-1in} % 
\setlength{\oddsidemargin}{11mm} % 左右方向のマージン
\addtolength{\oddsidemargin}{-1in} % 
\setlength{\textwidth}{154mm} % B6 用
\setlength{\textheight}{108mm} % B6 用
\setlength{\headsep}{0mm} % 
\setlength{\headheight}{0mm} % 
\setlength{\topskip}{0mm} % 
\setlength{\parskip}{0pt} % 
\def\labelenumi{\theenumi、} % 箇条書きのフォーマット
\parindent = 0pt % 段落下げしない

 % B6 用テンプレート読み込み

\begin{document}
% begin header
%%%%% タイトルと作者 ここから %%%%%
\begin{minipage}[c]{0.7\hsize} % タイトルは上から 7 割
    \begin{center}
    % begin title
        {\LARGE
            六華雪解に % タイトルを入れる
        }
        {\small 
            (平成二十一年度寮歌) % 年などを入れる
        }
    % end title
    \end{center}
\end{minipage}
\begin{minipage}[c]{0.3\hsize} % 作歌作曲は上から 3 割
    \begin{flushright} % 下寄せにする
        % begin name
        丸田潤君 作歌・作曲 % 作歌・作曲者
        % end name
    \end{flushright}
\end{minipage}
%%%%% タイトルと作者 ここまで %%%%%
% (1,2,3 了あり)
% end header

% begin length
\vspace{1.5em} % タイトル, 作者と歌詞の間に隙間を設ける
\newcommand{\linespace}{0.5em} % 行間の設定
\newcommand{\blocksize}{0.5\hsize} % 段組間の設定
\newcommand{\itemmargin}{3em} % 曲番の位置調整の長さ
% end length
% begin body
%%%%% 歌詞 ここから %%%%%
\begin{enumerate} % 番号の箇条書き ここから
    \setlength{\itemindent}{\itemmargin} % 曲番の位置調整
    \begin{minipage}[c]{\blocksize}
    
        \vspace{\linespace}
        \item~\\
        % 1.
        \ruby{六華}{りっ|か}\ruby{雪解}{ゆき|げ}に\ruby{佇}{たたず}みて\\
        \ruby{故郷}{こ|きょう}を\ruby{去}{さ}りし\ruby[g]{若人}{わこうど}が\\
        \ruby{清}{きよ}き\ruby{野心}{や|しん}を\ruby{胸}{むね}に\ruby{秘}{ひ}め\\
        しばし\ruby{憩}{いこ}わんこの\ruby[g]{宿舎}{やどり}
        
        \vspace{\linespace}
        \item~\\
        % 2.
        \ruby{酒飲}{さけ|の}み\ruby{宴}{うたげ}し\ruby{夜}{よ}は\ruby{更}{ふ}けて\\
        \ruby{明}{あ}く\ruby{迄}{まで}\ruby{語}{かた}り\ruby{日々}{ひ|び}は\ruby{行}{ゆ}き\\
        \ruby{燈火}{ともし|び}\ruby{闇}{やみ}に\ruby{浮}{う}かび\ruby{出}{い}づ\\
        \ruby{輝}{かがや}き\ruby[g]{永久}{とわ}に\ruby{絶}{た}やさずや
        
    \end{minipage}
    \begin{minipage}[c]{\blocksize}
        
        \vspace{\linespace}
        \item~\\
        % 3.
        \ruby{理想}{り|そう}の\ruby{自治}{じ|ち}を\ruby{手}{て}にするは\\
        \ruby{常}{つね}に\ruby[g]{寮生}{われら}が\ruby{高}{たか}みなり\\
        \ruby{崩}{くず}れゆくこの\ruby{時}{とき}にこそ\\
        \ruby{不断}{ふ|だん}の\ruby[g]{尽力}{つとめ}\ruby{忘}{わす}るまじ
    
    \end{minipage}
\end{enumerate} % 番号の箇条書き ここまで
%%%%% 歌詞 ここまで %%%%%
% end body

\end{document}

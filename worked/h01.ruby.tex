\documentclass[10pt,b5j]{tarticle} % B6 縦書き
% \documentclass[10pt,b5j]{tarticle} % B6 縦書き
\AtBeginDvi{\special{papersize=128mm,182mm}} % B6 用用紙サイズ
\usepackage{otf} % Unicode で字を入力するのに必要なパッケージ
\usepackage[size=b6j]{bxpapersize} % B6 用紙サイズを指定
\usepackage[dvipdfmx]{graphicx} % 画像を挿入するためのパッケージ
\usepackage[dvipdfmx]{color} % 色をつけるためのパッケージ
\usepackage{pxrubrica} % ルビを振るためのパッケージ
\usepackage{multicol} % 複数段組を作るためのパッケージ
\setlength{\topmargin}{14mm} % 上下方向のマージン
\addtolength{\topmargin}{-1in} % 
\setlength{\oddsidemargin}{11mm} % 左右方向のマージン
\addtolength{\oddsidemargin}{-1in} % 
\setlength{\textwidth}{154mm} % B6 用
\setlength{\textheight}{108mm} % B6 用
\setlength{\headsep}{0mm} % 
\setlength{\headheight}{0mm} % 
\setlength{\topskip}{0mm} % 
\setlength{\parskip}{0pt} % 
\def\labelenumi{\theenumi、} % 箇条書きのフォーマット
\parindent = 0pt % 段落下げしない

 % B6 用テンプレート読み込み

\begin{document}
% begin header
%%%%% タイトルと作者 ここから %%%%%
\begin{minipage}[c]{0.7\hsize} % タイトルは上から 7 割
    \begin{center}
    % begin title
        {\LARGE
            惡魔死す瞬間 % タイトルを入れる
        }
        {\small 
            (平成元年度寮歌) % 年などを入れる
        }
    % end title
    \end{center}
\end{minipage}
\begin{minipage}[c]{0.3\hsize} % 作歌作曲は上から 3 割
    \begin{flushright} % 下寄せにする
        % begin name
        宜寿次盛生君 作歌\\田口拓君 作曲 % 作歌・作曲者
        % end name
    \end{flushright}
\end{minipage}
%%%%% タイトルと作者 ここまで %%%%%
% (1,2,3,4 了あり)
% end header

% begin length
\vspace{1.5em} % タイトル, 作者と歌詞の間に隙間を設ける
\newcommand{\linespace}{0.5em} % 行間の設定
\newcommand{\blocksize}{0.5\hsize} % 段組間の設定
\newcommand{\itemmargin}{3em} % 曲番の位置調整の長さ
% end length
% begin body
%%%%% 歌詞 ここから %%%%%
\begin{enumerate} % 番号の箇条書き ここから
    \setlength{\itemindent}{\itemmargin} % 曲番の位置調整
    \begin{minipage}[c]{\blocksize}
    
        \vspace{\linespace}
        \item~\\
        % 1.
        \ruby[g]{惡魔}{サタン}\ruby{死}{し}す\ruby[g]{瞬間}{とき}\ruby{何}{なに}を\ruby[g]{凝視}{み}る\\
        \ruby{解}{と}けざる\ruby{呪}{のろい}\ruby[g]{鬼ヶ島}{おにがしま}\\
        \ruby{北溟}{ほく|めい}の\ruby{国}{くに}この\ruby{城}{しろ}に\\
        \ruby{我}{われ}\ruby{旅立}{たび|だ}ちの\ruby{時}{とき}を\ruby{待}{ま}つ
        
        \vspace{\linespace}
        \item~\\
        % 2.
        \ruby{降}{を}りたる\ruby{魔王}{ま|おう}\ruby{荒}{あ}れ\ruby{狂}{くる}ふ\\
        \ruby{若}{わか}き\ruby[g]{生血}{ち}を\ruby{吸}{す}ひ\ruby{蘇}{よみが}へる\\
        \ruby{西都}{せい|と}の\ruby{異変}{い|へん}\ruby{我知}{われ|し}らず\\
        \ruby{春欄漫}{はる|らん|まん}に\ruby{酔}{よ}ひ\ruby{狂}{くる}ふ
        
    \end{minipage}
    \begin{minipage}[c]{\blocksize}
        
        \vspace{\linespace}
        \item~\\
        % 3.
        \ruby{祭}{まつり}\ruby{終}{おわ}りて\ruby[g]{黄葉}{もみじ}\ruby{散}{ち}り\\
        \ruby{暗雲}{あん|うん}\ruby{広}{ひろ}がる\ruby{秋}{あき}の\ruby{空}{そら}\\
        \ruby{希望}{き|ぼう}の\ruby[g]{東光}{ひかり}\ruby{恨}{うら}みつつ\\
        \ruby{冬将軍}{ふゆ|しょう|ぐん}が\ruby{猛狂}{たけ|くる}ふ
        
        \vspace{\linespace}
        \item~\\
        % 4.
        \ruby[g]{白銀}{しろがね}の\ruby[g]{原野}{の}は\ruby{静}{しず}まりて\\
        \ruby{地獄}{じ|ごく}\ruby{転}{てん}じて\ruby[g]{黄泉}{よみ}の\ruby{国}{くに}\\
        \ruby{野人}{や|じん}\ruby{籠}{こも}りて\ruby[g]{微睡}{まどろ}みて\\
        \ruby{今}{いま}\ruby{旅立}{たび|だ}ちの\ruby{春}{はる}を\ruby{待}{ま}つ
    
    \end{minipage}
\end{enumerate} % 番号の箇条書き ここまで
%%%%% 歌詞 ここまで %%%%%
% end body

\end{document}

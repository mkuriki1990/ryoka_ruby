\documentclass[10pt,b5j]{tarticle} % B6 縦書き
% \documentclass[10pt,b5j]{tarticle} % B6 縦書き
\AtBeginDvi{\special{papersize=128mm,182mm}} % B6 用用紙サイズ
\usepackage{otf} % Unicode で字を入力するのに必要なパッケージ
\usepackage[size=b6j]{bxpapersize} % B6 用紙サイズを指定
\usepackage[dvipdfmx]{graphicx} % 画像を挿入するためのパッケージ
\usepackage[dvipdfmx]{color} % 色をつけるためのパッケージ
\usepackage{pxrubrica} % ルビを振るためのパッケージ
\usepackage{multicol} % 複数段組を作るためのパッケージ
\setlength{\topmargin}{14mm} % 上下方向のマージン
\addtolength{\topmargin}{-1in} % 
\setlength{\oddsidemargin}{11mm} % 左右方向のマージン
\addtolength{\oddsidemargin}{-1in} % 
\setlength{\textwidth}{154mm} % B6 用
\setlength{\textheight}{108mm} % B6 用
\setlength{\headsep}{0mm} % 
\setlength{\headheight}{0mm} % 
\setlength{\topskip}{0mm} % 
\setlength{\parskip}{0pt} % 
\def\labelenumi{\theenumi、} % 箇条書きのフォーマット
\parindent = 0pt % 段落下げしない

 % B6 用テンプレート読み込み

\begin{document}
% begin header
%%%%% タイトルと作者 ここから %%%%%
\begin{minipage}[c]{0.7\hsize} % タイトルは上から 7 割
    \begin{center}
    % begin title
        {\LARGE
            若人よ % タイトルを入れる
        }
        {\small 
            (平成十二年度寮歌) % 年などを入れる
        }
    % end title
    \end{center}
\end{minipage}
\begin{minipage}[c]{0.3\hsize} % 作歌作曲は上から 3 割
    \begin{flushright} % 下寄せにする
        % begin name
        野路直之君 作歌\\村中剛洋君 作曲 % 作歌・作曲者
        % end name
    \end{flushright}
\end{minipage}
%%%%% タイトルと作者 ここまで %%%%%
% (1,2,3 繰り返しなし)
% end header

% begin length
\vspace{1.5em} % タイトル, 作者と歌詞の間に隙間を設ける
\newcommand{\linespace}{0.5em} % 行間の設定
\newcommand{\blocksize}{0.33\hsize} % 段組間の設定
\newcommand{\itemmargin}{3em} % 曲番の位置調整の長さ
% end length
% begin body
%%%%% 歌詞 ここから %%%%%
\begin{enumerate} % 番号の箇条書き ここから
    \setlength{\itemindent}{\itemmargin} % 曲番の位置調整
    \begin{minipage}[c]{\blocksize}
    
        \vspace{\linespace}
        \item~\\
        % 1.
        \ruby{春風}{しゅん|ぷう}\ruby{興}{おこ}せ\ruby{我}{わ}が\ruby[g]{若人}{わこうど}よ\\
        \ruby[g]{大地}{ち}を\ruby{君}{きみ}の\ruby{色}{いろ}に\ruby{染}{そ}めよ\\
        \ruby{理知}{り|ち}\ruby{無}{な}かりしも\ruby{血気}{けっ|き}\ruby{注}{そそ}がば\\
        \ruby[g]{光明}{ひかり}の\ruby{迪}{みち}\ruby{拓}{ひら}かれん\\
        
    \end{minipage}
    \begin{minipage}[c]{\blocksize}
        
        \vspace{\linespace}
        \item~\\
        % 2.
        \ruby{使命}{し|めい}は\ruby{未}{いま}だ\ruby{君等}{きみ|ら}が\ruby{華}{はな}ぞ\\
        \ruby{捧}{ささ}げよ\ruby{汝}{な}が\ruby[g]{情熱}{おもい}\\
        \ruby{留}{とど}まり\ruby{酌}{く}みてただ\ruby{人}{ひと}を\ruby{待}{ま}つ\\
        \ruby{尽}{つ}きる\ruby{事}{こと}なき\ruby{我}{わ}が\ruby[g]{希望}{のぞみ}\\
        (※繰り返し)
        
    \end{minipage}
    \begin{minipage}[c]{\blocksize}
        
        \vspace{\linespace}
        \item~\\
        % 3.
        \ruby{別}{わか}れは\ruby{近}{ちか}く\ruby[g]{再会}{であい}は\ruby{遠}{とお}し\\
        \ruby{巣立}{す|だ}つ\ruby{友等}{とも|ら}の\ruby{愛}{いと}しさよ\\
        \ruby{君}{きみ}の\ruby{比翼}{ひ|よく}の\ruby{鳥}{とり}となりたし\\
        \ruby{翔}{か}けめぐらんかな\ruby{共}{とも}に\\
        (※繰り返し)
    
    \end{minipage}
\end{enumerate} % 番号の箇条書き ここまで
\begin{enumerate} % 番号の箇条書き ここから
    \vspace{\linespace}
    \item[(※)]
    \ruby{涙}{なみだ}たぎりて\ruby[g]{白雲}{くも}を\ruby{流}{なが}せば\\
    \ruby[g]{満月}{つき}も\ruby{我等}{われ|ら}を\ruby{讃}{たた}へんや\\
    \ruby{一矢}{いっ|し}の\ruby{猛}{たけ}りが\ruby{青竜}{せい|りゅう}となりて\\
    \ruby{天}{てん}に\ruby{昇}{のぼ}るは\ruby{今}{いま}この\ruby{時}{とき}ぞ
\end{enumerate} % 番号の箇条書き ここまで
%%%%% 歌詞 ここまで %%%%%
% end body

\end{document}

\documentclass[10pt,b5j]{tarticle} % B6 縦書き
% \documentclass[10pt,b5j]{tarticle} % B6 縦書き
\AtBeginDvi{\special{papersize=128mm,182mm}} % B6 用用紙サイズ
\usepackage{otf} % Unicode で字を入力するのに必要なパッケージ
\usepackage[size=b6j]{bxpapersize} % B6 用紙サイズを指定
\usepackage[dvipdfmx]{graphicx} % 画像を挿入するためのパッケージ
\usepackage[dvipdfmx]{color} % 色をつけるためのパッケージ
\usepackage{pxrubrica} % ルビを振るためのパッケージ
\usepackage{plext} % 漢数字の enumerate を使うためのパッケージ
\usepackage{multicol} % 複数段組を作るためのパッケージ
\setlength{\topmargin}{14mm} % 上下方向のマージン
\addtolength{\topmargin}{-1in} % 
\setlength{\oddsidemargin}{11mm} % 左右方向のマージン
\addtolength{\oddsidemargin}{-1in} % 
\setlength{\textwidth}{154mm} % B6 用
\setlength{\textheight}{108mm} % B6 用
\setlength{\headsep}{0mm} % 
\setlength{\headheight}{0mm} % 
\setlength{\topskip}{0mm} % 
\setlength{\parskip}{0pt} % 
\def\theenumi{\Kanji{enumi}} % 箇条書きのフォーマットを漢数字に変更
\parindent = 0pt % 段落下げしない
\pagestyle{empty} % すべてのページ番号を消去
% \renewcommand{\baselinestretch}{0.9} % 行間の倍率
 % B6 用テンプレート読み込み

\begin{document}
% begin header
%%%%% タイトルと作者 ここから %%%%%
\begin{minipage}[c]{0.7\hsize} % タイトルは上から 7 割
    \begin{center}
    % begin title
        {\LARGE
            嗚呼青春の % タイトルを入れる
        }
        {\small 
            (昭和五年寮歌) % 年などを入れる
        }
    % end title
    \end{center}
\end{minipage}
\begin{minipage}[c]{0.3\hsize} % 作歌作曲は上から 3 割
    \begin{flushright} % 下寄せにする
        % begin name
        児山信蔵君 作歌\\有村徹君 作曲 % 作歌・作曲者
        % end name
    \end{flushright}
\end{minipage}
%%%%% タイトルと作者 ここまで %%%%%
% (1,3,5 了あり)
% end header

% begin length
\vspace{1.5em} % タイトル, 作者と歌詞の間に隙間を設ける
\newcommand{\linespace}{0.5em} % 行間の設定
\newcommand{\blocksize}{0.33\hsize} % 段組間の設定
\newcommand{\itemmargin}{3em} % 曲番の位置調整の長さ
% end length
% begin body
%%%%% 歌詞 ここから %%%%%
\begin{enumerate} % 番号の箇条書き ここから
    \setlength{\itemindent}{\itemmargin} % 曲番の位置調整
    \begin{minipage}[c]{\blocksize}
    
        \vspace{\linespace}
        \item~\\
        % 1.
        \ruby{嗚呼}{あ|あ}\ruby{青春}{せい|しゅん}の\ruby{夢}{ゆめ}\ruby{高}{たか}く\\
        \ruby{理想}{り|そう}のあとに\ruby[g]{憧憬}{あこが}れて\\
        \ruby{楡}{にれ}の\ruby{花}{はな}\ruby{散}{ち}る\ruby[g]{学都}{みやこ}にぞ\\
        \ruby[g]{啓示}{さとし}を\ruby{求}{もと}む\ruby[g]{若人}{わこうど}は\\
        \ruby[g]{綺花}{はな}を\ruby{流}{なが}して\ruby{逝}{ゆ}く\ruby{水}{みず}に\\
        \ruby{十九}{じゅう|く}の\ruby{春}{はる}を\ruby{嘆}{なげ}くなり
        
        \vspace{\linespace}
        \item~\\
        % 2.
        \ruby{牧場}{まき|ば}の\ruby[g]{緑草}{みどり}\ruby{踏}{ふ}みしだき\\
        \ruby{栗毛}{くり|げ}の\ruby{駒}{こま}に\ruby{鞍}{くら}\ruby{置}{お}きて\\
        うち\ruby{振}{ふ}る\ruby{鞭}{むち}の\ruby{音}{ね}も\ruby{高}{たか}く\\
        \ruby[g]{希望}{のぞみ}の\ruby[g]{大空}{そら}を\ruby{朗}{ほが}らかに\\
        \ruby[g]{寮歌}{うた}を\ruby{歌}{うた}ひつ\ruby{眺}{なが}むれば\\
        \ruby[g]{白雲}{くも}\ruby{流}{なが}れゆく\ruby[g]{手稲山}{やま}\ruby{静}{しず}か
        
    \end{minipage}
    \begin{minipage}[c]{\blocksize}
        
        \vspace{\linespace}
        \item~\\
        % 3.
        \ruby[g]{学堂}{みどう}の\ruby[g]{古鐘}{かね}の\ruby{沈}{しず}みゆき\\
        \ruby{楡陵}{ゆ|りょう}の\ruby[g]{蒼空}{そら}に\ruby[g]{銀月}{つき}\ruby{冴}{さ}えて\\
        \ruby{羊}{ひつじ}の\ruby{群}{む}れの\ruby[g]{片影}{かげ}もなし\\
        \ruby[g]{沈黙}{しじま}の\ruby[g]{原始}{もり}に\ruby{散}{ち}りしける\\
        \ruby{落葉}{おち|ば}\ruby{踏}{ふ}みゆく\ruby{雄}{たけ}き\ruby{子}{こ}は\\
        \ruby[g]{三年}{みとせ}の\ruby[g]{絢夢}{ゆめ}に\ruby{涙}{なんだ}する
        
        \vspace{\linespace}
        \item~\\
        % 4.
        \ruby{疎林}{そ|りん}のほとり\ruby[g]{夕陽}{ひ}は\ruby{落}{お}ちて\\
        \ruby{凩}{こがらし}さへも\ruby{絶}{た}えし\ruby[g]{真夜}{よ}に\\
        \ruby{涯}{はて}なく\ruby{白}{しろ}き\ruby{石狩}{いし|かり}の\\
        \ruby[g]{銀雪}{ゆき}に\ruby{連}{つら}なる\ruby[g]{曠野}{の}の\ruby[g]{静寂}{しじま}\\
        \ruby{震}{ふる}はせ\ruby{乍}{なが}ら\ruby{橇}{そり}\ruby{唄}{うた}は\\
        \ruby[g]{神秘}{くしび}の\ruby{闇}{やみ}を\ruby{縫}{ぬ}ひてゆく
        
    \end{minipage}
    \begin{minipage}[c]{\blocksize}
        
        \vspace{\linespace}
        \item~\\
        % 5.
        \ruby{北斗}{ほく|と}は\ruby{遠}{とお}く\ruby[g]{七星}{ほし}\ruby{清}{きよ}し\\
        「\ruby{妄執}{もう|しゅう}」の\ruby[g]{現世}{よ}を\ruby{見下}{み|おろ}して\\
        \ruby{真実一路}{しん|じつ|いち|ろ}の\ruby{迪恵}{みち|たづ}ぬ\\
        「\ruby{意気}{い|き}」と「\ruby{血潮}{ち|しお}」に\ruby{生}{い}くる\ruby{子}{こ}の\\
        \ruby{瞳}{ひとみ}に\ruby{燃}{も}ゆる\ruby[g]{紅{\CID{7644}}}{ほのむら}は\\ % CID 焰
        \ruby[g]{永遠}{とは}なる\ruby[g]{生命}{たま}の\ruby{証}{あかし}なり
    
    \end{minipage}
\end{enumerate} % 番号の箇条書き ここまで
%%%%% 歌詞 ここまで %%%%%
% end body

\end{document}

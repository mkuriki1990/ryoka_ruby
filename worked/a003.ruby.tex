\documentclass[10pt,b5j]{tarticle} % B6 縦書き
% \documentclass[10pt,b5j]{tarticle} % B6 縦書き
\AtBeginDvi{\special{papersize=128mm,182mm}} % B6 用用紙サイズ
\usepackage{otf} % Unicode で字を入力するのに必要なパッケージ
\usepackage[size=b6j]{bxpapersize} % B6 用紙サイズを指定
\usepackage[dvipdfmx]{graphicx} % 画像を挿入するためのパッケージ
\usepackage[dvipdfmx]{color} % 色をつけるためのパッケージ
\usepackage{pxrubrica} % ルビを振るためのパッケージ
\usepackage{multicol} % 複数段組を作るためのパッケージ
\setlength{\topmargin}{14mm} % 上下方向のマージン
\addtolength{\topmargin}{-1in} % 
\setlength{\oddsidemargin}{11mm} % 左右方向のマージン
\addtolength{\oddsidemargin}{-1in} % 
\setlength{\textwidth}{154mm} % B6 用
\setlength{\textheight}{108mm} % B6 用
\setlength{\headsep}{0mm} % 
\setlength{\headheight}{0mm} % 
\setlength{\topskip}{0mm} % 
\setlength{\parskip}{0pt} % 
\def\labelenumi{\theenumi、} % 箇条書きのフォーマット
\parindent = 0pt % 段落下げしない

 % B6 用テンプレート読み込み

\begin{document}
% begin header
%%%%% タイトルと作者 ここから %%%%%
\begin{minipage}[c]{0.7\hsize} % タイトルは上から 7 割
    \begin{center}
    % begin title
        {\LARGE
            島浪かへる % タイトルを入れる
        }
        {\small 
            (大正三年桜星会歌) % 年などを入れる
        }
    % end title
    \end{center}
\end{minipage}
\begin{minipage}[c]{0.3\hsize} % 作歌作曲は上から 3 割
    \begin{flushright} % 下寄せにする
        % begin name
        木原均君 作歌\\岩崎直砥君 作曲 % 作歌・作曲者
        % end name
    \end{flushright}
\end{minipage}
%%%%% タイトルと作者 ここまで %%%%%
% % end header

% begin length
\vspace{1.5em} % タイトル, 作者と歌詞の間に隙間を設ける
\newcommand{\linespace}{0.5em} % 行間の設定
\newcommand{\blocksize}{0.5\hsize} % 段組間の設定
\newcommand{\itemmargin}{3em} % 曲番の位置調整の長さ
% end length
% begin body
%%%%% 歌詞 ここから %%%%%
\begin{enumerate} % 番号の箇条書き ここから
    \setlength{\itemindent}{\itemmargin} % 曲番の位置調整
    \begin{minipage}[c]{\blocksize}
    
        \vspace{\linespace}
        \item~\\
        % 1.
        \ruby{島}{しま}\ruby{浪}{なみ}かへる\ruby{北}{きた}\ruby{溟}{}さして\\
        \ruby{石狩}{いしかり}の\ruby{水}{みず}\ruby{末}{まつ}\ruby{遠}{とお}く\\
        \ruby{霞}{かすみ}のあなた\ruby{流}{ながれ}るヽ\ruby{郷土}{きょうど}に\\
        あけくれなれし\ruby{我}{われ}\ruby{友}{とも}の\\
        \ruby{學}{まな}びに\ruby{集}{しゅう}ふ\ruby{楡}{にれ}\ruby{影}{かげ}の\ruby{庭}{にわ}に\\
        \ruby{絢爛}{けんらん}の\ruby{春}{はる}またおとづれぬ
        
    \end{minipage}
    \begin{minipage}[c]{\blocksize}
        
        \vspace{\linespace}
        \item~\\
        % 2.
        \ruby{春陽}{しゅんよう}のもと\ruby{下}{か}\ruby{萠}{めぐむ}えそめて\\
        \ruby{遙}{はる}かなるかな\ruby{我}{われ}\ruby{思}{おも}ひ\\
        \ruby{無相}{むそう}の\ruby{智慧}{ちえ}を\ruby{追}{つい}ひ\ruby{求}{もと}めつ、\\
        \ruby{無明}{むみょう}の\ruby{闇}{やみ}をわけ\ruby{入}{はい}りて\\
        \ruby{生命}{いのち}の\ruby{流}{なが}れ\ruby{深}{ふか}くも\ruby{進}{すす}む\\
        \ruby{雄々}{おお}しき\ruby{學}{まなぶ}\ruby{徒}{と}こヽ\ruby{北}{きた}にあり
    
    \end{minipage}
\end{enumerate} % 番号の箇条書き ここまで
%%%%% 歌詞 ここまで %%%%%
% end body

\end{document}

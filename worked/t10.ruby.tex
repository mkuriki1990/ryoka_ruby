\documentclass[10pt,b5j]{tarticle} % B6 縦書き
% \documentclass[10pt,b5j]{tarticle} % B6 縦書き
\AtBeginDvi{\special{papersize=128mm,182mm}} % B6 用用紙サイズ
\usepackage{otf} % Unicode で字を入力するのに必要なパッケージ
\usepackage[size=b6j]{bxpapersize} % B6 用紙サイズを指定
\usepackage[dvipdfmx]{graphicx} % 画像を挿入するためのパッケージ
\usepackage[dvipdfmx]{color} % 色をつけるためのパッケージ
\usepackage{pxrubrica} % ルビを振るためのパッケージ
\usepackage{multicol} % 複数段組を作るためのパッケージ
\setlength{\topmargin}{14mm} % 上下方向のマージン
\addtolength{\topmargin}{-1in} % 
\setlength{\oddsidemargin}{11mm} % 左右方向のマージン
\addtolength{\oddsidemargin}{-1in} % 
\setlength{\textwidth}{154mm} % B6 用
\setlength{\textheight}{108mm} % B6 用
\setlength{\headsep}{0mm} % 
\setlength{\headheight}{0mm} % 
\setlength{\topskip}{0mm} % 
\setlength{\parskip}{0pt} % 
\def\labelenumi{\theenumi、} % 箇条書きのフォーマット
\parindent = 0pt % 段落下げしない

 % B6 用テンプレート読み込み

\begin{document}
% begin header
%%%%% タイトルと作者 ここから %%%%%
\begin{minipage}[c]{0.7\hsize} % タイトルは上から 7 割
    \begin{center}
    % begin title
        {\LARGE
            生命の争闘 % タイトルを入れる
        }
        {\small 
            (大正十年寮歌) % 年などを入れる
        }
    % end title
    \end{center}
\end{minipage}
\begin{minipage}[c]{0.3\hsize} % 作歌作曲は上から 3 割
    \begin{flushright} % 下寄せにする
        % begin name
        青野正男君 作歌\\小峰三千男君 作曲 % 作歌・作曲者
        % end name
    \end{flushright}
\end{minipage}
%%%%% タイトルと作者 ここまで %%%%%
% (1,6 了あり)
% end header

% begin length
\vspace{1.5em} % タイトル, 作者と歌詞の間に隙間を設ける
\newcommand{\linespace}{0.5em} % 行間の設定
\newcommand{\blocksize}{0.33\hsize} % 段組間の設定
\newcommand{\itemmargin}{3em} % 曲番の位置調整の長さ
% end length
% begin body
%%%%% 歌詞 ここから %%%%%
\begin{enumerate} % 番号の箇条書き ここから
    \setlength{\itemindent}{\itemmargin} % 曲番の位置調整
    \begin{minipage}[c]{\blocksize}
    
        \vspace{\linespace}
        \item~\\
        % 1.
        \ruby[g]{生命}{いのち}の\ruby[g]{争闘}{いくさ}\ruby{敗}{やぶ}れじと\\
        \ruby{雪解}{ゆき|げ}の\ruby{野辺}{の|べ}に\ruby{萠}{も}え\ruby{出}{い}でし\\
        \ruby{浅緑}{あさ|みどり}なる\ruby{若草}{わか|くさ}の\\
        \ruby[g]{伸展}{のび}ゆく\ruby[g]{生命}{いのち}\ruby{思}{おも}ふとき\\
        \ruby{若}{わか}き\ruby{力}{ちから}のよろこびは\\
        \ruby{我等}{われ|ら}が\ruby{胸}{むね}に\ruby{溢}{あふ}るなり
        
        \vspace{\linespace}
        \item~\\
        % 2.
        \ruby[g]{悲哀}{かなしみ}\ruby{誘}{さそ}ふ\ruby{郭公}{かっ|こう}の\\
        \ruby{声}{こえ}を\ruby{聞}{き}きつつ\ruby[g]{逍遙}{さまよ}へば\\
        \ruby{今}{いま}は\ruby{小暗}{を|ぐら}き\ruby{木下}{こ|した}\ruby{闇}{やみ}\\
        \ruby[g]{黒百合}{くろゆり}\ruby{咲}{さ}けど\ruby{春}{はる}いづこ\\
        うつろひやすき\ruby{若}{わか}き\ruby{日}{ひ}を\\
        \ruby{盧生}{ろ|せい}の\ruby{夢}{ゆめ}となすなかれ
        
    \end{minipage}
    \begin{minipage}[c]{\blocksize}
        
        \vspace{\linespace}
        \item~\\
        % 3.
        \ruby{牧場}{まき|ば}に\ruby{虫}{むし}の\ruby{音}{ね}も\ruby{淡}{あわ}く\\
        \ruby{仰}{あお}げば\ruby{高}{たか}し\ruby{秋}{あき}の\ruby{空}{そら}\\
        \ruby{肥馬}{ひ|ば}\ruby{原頭}{げん|とう}に\ruby{嘶}{いなな}きて\\
        \ruby{雄渾}{ゆう|こん}の\ruby{気}{き}はあふれつつ\\
        \ruby{崇}{たか}き\ruby{理想}{り|そう}を\ruby{胸}{むね}にして\\
        \ruby{生}{い}くる\ruby[g]{喜悦}{よろこび}\ruby{謳}{うた}ふ\ruby{哉}{かな}
        
        \vspace{\linespace}
        \item~\\
        % 4.
        \ruby{眺}{なが}めはてなき\ruby{石狩}{いし|かり}の\\
        \ruby[g]{曠野}{の}に\ruby{凋落}{てう|らく}の\ruby{秋}{あき}\ruby{更}{た}けて\\
        \ruby{寂}{さび}しく\ruby{暮}{く}るる\ruby{手稲山}{て|いね|やま}\\
        \ruby{今}{いま}うすれゆく\ruby{赤陽}{せき|やう}に\\
        \ruby{想}{おも}ひぞ\ruby{馳}{は}する\ruby{北}{きた}\ruby{欧州}{おう|しゅう}\\
        \ruby[g]{戦{\UTF{FA52}}}{いくさ}の\ruby{跡}{あと}の\ruby{夕}{ゆう}まぐれ % UTF 禍
        
    \end{minipage}
    \begin{minipage}[c]{\blocksize}
        
        \vspace{\linespace}
        \item~\\
        % 5.
        \ruby{夕}{ゆふべ}\ruby{吹}{ふ}く\ruby{風}{かぜ}\ruby{膚}{はだ}にしみ\\
        \ruby{音}{おと}も\ruby{淋}{さび}しく\ruby{行}{ゆ}く\ruby{橇}{そり}の\\
        \ruby{大}{だい}\ruby{雪原}{せつ|げん}に\ruby{消}{き}ゆるとき\\
        \ruby{寒月}{かん|げつ}\ruby{高}{たか}く\ruby{冴}{は}ゆる\ruby{夜半}{や|わ}\\
        \ruby[g]{哀愁}{うれひ}をこむる\ruby[g]{若人}{わこうど}の\\
        \ruby[g]{瞑想}{おもひ}ぞ\ruby[g]{如何}{いか}に\ruby{深}{ふか}からん
        
        \vspace{\linespace}
        \item~\\
        % 6.
        \ruby{嗚呼}{あ|あ}\ruby{北州}{ほく|しゅう}の\ruby{春秋}{しゅん|じゅう}に\\
        \ruby{自然}{し|ぜん}の\ruby[g]{教訓}{をしへ}\ruby{学}{まな}びつつ\\
        \ruby{尚}{たか}き\ruby[g]{生命}{いのち}に\ruby{生}{い}きなんと\\
        \ruby[g]{精神}{こころ}を\ruby{磨}{みが}く\ruby{友}{とも}どちよ\\
        \ruby{先人}{せん|じん}\ruby{建}{た}てし\ruby{自治寮}{じ|ち|りょう}の\\
        \ruby{貴}{とうと}き\ruby{歴史}{れき|し}\ruby{伝}{つた}へかし
    
    \end{minipage}
\end{enumerate} % 番号の箇条書き ここまで
%%%%% 歌詞 ここまで %%%%%
% end body

\end{document}

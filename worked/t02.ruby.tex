\documentclass[10pt,b5j]{tarticle} % B6 縦書き
% \documentclass[10pt,b5j]{tarticle} % B6 縦書き
\AtBeginDvi{\special{papersize=128mm,182mm}} % B6 用用紙サイズ
\usepackage{otf} % Unicode で字を入力するのに必要なパッケージ
\usepackage[size=b6j]{bxpapersize} % B6 用紙サイズを指定
\usepackage[dvipdfmx]{graphicx} % 画像を挿入するためのパッケージ
\usepackage[dvipdfmx]{color} % 色をつけるためのパッケージ
\usepackage{pxrubrica} % ルビを振るためのパッケージ
\usepackage{plext} % 漢数字の enumerate を使うためのパッケージ
\usepackage{multicol} % 複数段組を作るためのパッケージ
\setlength{\topmargin}{14mm} % 上下方向のマージン
\addtolength{\topmargin}{-1in} % 
\setlength{\oddsidemargin}{11mm} % 左右方向のマージン
\addtolength{\oddsidemargin}{-1in} % 
\setlength{\textwidth}{154mm} % B6 用
\setlength{\textheight}{108mm} % B6 用
\setlength{\headsep}{0mm} % 
\setlength{\headheight}{0mm} % 
\setlength{\topskip}{0mm} % 
\setlength{\parskip}{0pt} % 
\def\theenumi{\Kanji{enumi}} % 箇条書きのフォーマットを漢数字に変更
\parindent = 0pt % 段落下げしない
\pagestyle{empty} % すべてのページ番号を消去
% \renewcommand{\baselinestretch}{0.9} % 行間の倍率
 % B6 用テンプレート読み込み

\begin{document}
% begin header
%%%%% タイトルと作者 ここから %%%%%
\begin{minipage}[c]{0.7\hsize} % タイトルは上から 7 割
    \begin{center}
    % begin title
        {\LARGE
            幾世幾年 % タイトルを入れる
        }
        {\small 
            (大正二年寮歌) % 年などを入れる
        }
    % end title
    \end{center}
\end{minipage}
\begin{minipage}[c]{0.3\hsize} % 作歌作曲は上から 3 割
    \begin{flushright} % 下寄せにする
        % begin name
        木原均君 作歌\\柳沢秀雄君 作曲 % 作歌・作曲者
        % end name
    \end{flushright}
\end{minipage}
%%%%% タイトルと作者 ここまで %%%%%
% (1,5 了あり)
% end header

% begin length
\vspace{1.5em} % タイトル, 作者と歌詞の間に隙間を設ける
\newcommand{\linespace}{0.5em} % 行間の設定
\newcommand{\blocksize}{0.5\hsize} % 段組間の設定
\newcommand{\itemmargin}{3em} % 曲番の位置調整の長さ
% end length
% begin body
%%%%% 歌詞 ここから %%%%%
\begin{enumerate} % 番号の箇条書き ここから
    \setlength{\itemindent}{\itemmargin} % 曲番の位置調整
    \begin{minipage}[c]{\blocksize}
    
        \vspace{\linespace}
        \item~\\
        % 1.
        \ruby{\ruby{幾}{いく}世}{いくよ}\ruby{幾}{いく}\ruby{年}{ねん}\ruby{流}{なが}れけん\\
        \ruby{永劫}{えいごう}\ruby{隔}{}つ\ruby{後}{のち}までも\\
        \ruby{洋々}{ようよう}\ruby{声}{こえ}なく\ruby{野}{の}をこえて\\
        \ruby{銀河}{ぎんが}に\ruby{似}{に}たる\ruby{石狩}{いしかり}の\\
        \ruby{岸辺}{きしべ}\ruby{静}{しず}けき\ruby{夕}{ゆう}まぐれ\\
        \ruby{導}{みちび}く\ruby{星}{ほし}を\ruby{仰}{あお}がずや
        
    \end{minipage}
    \begin{minipage}[c]{\blocksize}
        
        \vspace{\linespace}
        \item~\\
        % 2.
        \ruby{巷}{ちまた}の\ruby{塵}{ちり}の\ruby{跡}{あと}を\ruby{絶}{た}ち\\
        \ruby{惰眠}{だみん}をさます\ruby{雪嵐}{ゆきあらし}\\
        \ruby{毘嵐}{びらん}\ruby{万里}{ばんり}をかけりては\\
        \ruby{天地}{てんち}もゆらぐすさまじさ\\
        \ruby{万象}{ばんしょう}\ruby{淋}{さび}しく\ruby{装}{そう}ひて\\
        \ruby{蕭々}{しょうしょう}\ruby{寒}{さむ}き\ruby{冬景色}{ふゆげしき}
        
    \end{minipage}
    \begin{minipage}[c]{\blocksize}
        
        \vspace{\linespace}
        \item~\\
        % 3.
        めぐる\ruby{月日}{つきひ}の\ruby{尾車}{びしゃ}や\\
        さざめく\ruby{小河}{おがわ}\ruby{春}{はる}\ruby{告}{つ}げぬ\\
        あはれ\ruby{幸}{こう}ある\ruby{北}{きた}の\ruby{国}{くに}\\
        \ruby{緑}{みどり}が\ruby{丘}{おか}に\ruby{打}{う}ち\ruby{臥}{ふ}して\\
        \ruby{薫}{かお}る\ruby{微風}{びふう}\ruby{身}{み}にうけて\\
        \ruby{常世}{とこよ}の\ruby{春}{はる}を\ruby{偲}{しの}べかし
        
    \end{minipage}
    \begin{minipage}[c]{\blocksize}
        
        \vspace{\linespace}
        \item~\\
        % 4.
        \ruby{清}{きよ}き\ruby{真理}{しんり}の\ruby{渚}{なぎさ}より\\
        \ruby{無窮}{むきゅう}を\ruby{照}{て}らす\ruby{最高}{さいこう}の\\
        \ruby{天}{てん}つ\ruby{光明}{こうみょう}を\ruby{探}{さぐ}り\ruby{得}{え}て\\
        \ruby{迷}{めい}ひの\ruby{羈絆}{きはん}\ruby{解}{と}きほどき\\
        \ruby{闇}{やみ}を\ruby{排}{はい}して\ruby{永遠}{えいえん}の\\
        \ruby{理想}{りそう}の\ruby{郷}{さと}を\ruby{拓}{ひら}く\ruby{可}{か}し
        
        
    \end{minipage}
    \begin{minipage}[c]{\blocksize}
        
        \vspace{\linespace}
        \item~\\
        % 5.
        \ruby{一}{いち}\ruby{百}{ひゃく}\ruby{意気}{いき}みつ\ruby{北蝦夷}{きたえみし}の\\
        \ruby{健児}{けんじ}よいざや\ruby{奪}{だつ}ひ\ruby{起}{た}て\\
        \ruby{白}{しろ}き\ruby{朔風}{さくふう}われにあり\\
        \ruby{曠野}{あらの}に\ruby{練}{ねり}へし\ruby{心身}{しんしん}も\\
        \ruby{歌}{うた}へ\ruby{壮}{たけし}なる\ruby{勝}{かち}\ruby{歌}{うた}を\\
        \ruby{島根}{しまね}に\ruby{高}{たか}く\ruby{勇}{いさ}ましく
    
    \end{minipage}
\end{enumerate} % 番号の箇条書き ここまで
%%%%% 歌詞 ここまで %%%%%
% end body

\end{document}

\documentclass[10pt,b5j]{tarticle} % B6 縦書き
% \documentclass[10pt,b5j]{tarticle} % B6 縦書き
\AtBeginDvi{\special{papersize=128mm,182mm}} % B6 用用紙サイズ
\usepackage{otf} % Unicode で字を入力するのに必要なパッケージ
\usepackage[size=b6j]{bxpapersize} % B6 用紙サイズを指定
\usepackage[dvipdfmx]{graphicx} % 画像を挿入するためのパッケージ
\usepackage[dvipdfmx]{color} % 色をつけるためのパッケージ
\usepackage{pxrubrica} % ルビを振るためのパッケージ
\usepackage{plext} % 漢数字の enumerate を使うためのパッケージ
\usepackage{multicol} % 複数段組を作るためのパッケージ
\setlength{\topmargin}{14mm} % 上下方向のマージン
\addtolength{\topmargin}{-1in} % 
\setlength{\oddsidemargin}{11mm} % 左右方向のマージン
\addtolength{\oddsidemargin}{-1in} % 
\setlength{\textwidth}{154mm} % B6 用
\setlength{\textheight}{108mm} % B6 用
\setlength{\headsep}{0mm} % 
\setlength{\headheight}{0mm} % 
\setlength{\topskip}{0mm} % 
\setlength{\parskip}{0pt} % 
\def\theenumi{\Kanji{enumi}} % 箇条書きのフォーマットを漢数字に変更
\parindent = 0pt % 段落下げしない
\pagestyle{empty} % すべてのページ番号を消去
% \renewcommand{\baselinestretch}{0.9} % 行間の倍率
 % B6 用テンプレート読み込み

\begin{document}
% begin header
%%%%% タイトルと作者 ここから %%%%%
\begin{minipage}[c]{0.7\hsize} % タイトルは上から 7 割
    \begin{center}
    % begin title
        {\LARGE
            心の故郷 % タイトルを入れる
        }
        {\small 
            (大正十一年桜星会歌) % 年などを入れる
        }
    % end title
    \end{center}
\end{minipage}
\begin{minipage}[c]{0.3\hsize} % 作歌作曲は上から 3 割
    \begin{flushright} % 下寄せにする
        % begin name
         % 作歌・作曲者
        % end name
    \end{flushright}
\end{minipage}
%%%%% タイトルと作者 ここまで %%%%%
% % end header

% begin length
\vspace{1.5em} % タイトル, 作者と歌詞の間に隙間を設ける
\newcommand{\linespace}{0.5em} % 行間の設定
\newcommand{\blocksize}{0.5\hsize} % 段組間の設定
\newcommand{\itemmargin}{3em} % 曲番の位置調整の長さ
% end length
% begin body
%%%%% 歌詞 ここから %%%%%
\begin{enumerate} % 番号の箇条書き ここから
    \setlength{\itemindent}{\itemmargin} % 曲番の位置調整
    \begin{minipage}[c]{\blocksize}
    
        \vspace{\linespace}
        \item~\\
        % 1.
        \ruby{心}{こころ}の\ruby[g]{故郷}{さと}よ\ruby{石狩}{いし|かり}の\\
        \ruby{夢}{ゆめ}\ruby{杳}{はる}かなる\ruby{草}{くさ}の\ruby{野邊}{の|べ}\\
        \ruby{花}{はな}は\ruby{煙}{けぶ}りて\ruby{影}{かげ}\ruby{仄}{ほの}に\\
        \ruby[g]{生命}{いのち}の\ruby[g]{光榮}{はえ}と\ruby[g]{喜悦}{よろこび}を\\
        \ruby[g]{恍惚}{うつつ}につゝむ\ruby[g]{憧憬}{あこがれ}の\\
        \ruby{薔薇色}{ば|ら|いろ}の\ruby{露}{つゆ}\ruby{慕}{した}はしや
        
        \vspace{\linespace}
        \item~\\
        % 2.
        \ruby{夏}{なつ}の\ruby{園生}{その|ふ}の\ruby[g]{逍遥}{さすらひ}や\\
        \ruby[g]{野花}{はな}の\ruby[g]{息吹}{いぶき}に\ruby{風}{かぜ}の\ruby{香}{か}に\\
        \ruby{燦}{きら}めく\ruby{光}{ひか}りさゆらぎつ\\
        \ruby{樺}{かば}の\ruby{緑}{みどり}のほの\ruby{薫}{かを}る\\
        \ruby[g]{木梢}{こぬれ}に\ruby{歌}{うた}ふ\ruby{若鳥}{わか|どり}の\\
        \ruby{朗}{ろう}にひゞく\ruby{曙}{あけ}の\ruby{聲}{こゑ}
        
    \end{minipage}
    \begin{minipage}[c]{\blocksize}
        
        \vspace{\linespace}
        \item~\\
        % 3.
        \ruby{楡}{にれ}の\ruby{林}{はやし}の\ruby{星}{ほし}の\ruby{灯}{ひ}よ\\
        あはれ\ruby{高鳴}{たか|な}る\ruby{靈}{れい}と\ruby{智}{ち}の\\
        \ruby[g]{諧調}{しらべ}\ruby{豊}{ゆた}けき\ruby{魂}{たま}の\ruby{琴}{こと}\\
        \ruby[g]{黄金}{こがね}のさやき\ruby{銀}{ぎん}のいろ\\
        \ruby[g]{郷愁}{かなしみ}あはき\ruby{秋}{あき}の\ruby{夜}{よ}の\\
        \ruby[g]{沈黙}{しじま}にふるふ\ruby{星}{ほし}の\ruby{灯}{ひ}よ
        
        \vspace{\linespace}
        \item~\\
        % 4.
        \ruby[g]{白銀}{しろがね}の\ruby{宵闇}{よひ|やみ}\ruby{深}{ふか}く\\
        \ruby[g]{氷柱}{つらら}に\ruby{映}{は}ゆる\ruby{紅}{くれなゐ}の\\
        \ruby[g]{{\UTF{FA19}}秘}{くしび}たゞよふ\ruby{火明}{ほ|あか}りよ\\ % UTF 神
        \ruby{熱}{あつ}き\ruby[g]{情想}{おもひ}の\ruby[g]{律動}{をのの}きて\\
        \ruby{明}{めい}と\ruby{暗}{あん}との\ruby[g]{幻影}{まぼろし}に\\
        \ruby{聖}{きよ}き\ruby[g]{黙{\UTF{79B1}}}{いのり}の\ruby{魂}{たま}ゆるる % UTF 禱
    
    \end{minipage}
\end{enumerate} % 番号の箇条書き ここまで
%%%%% 歌詞 ここまで %%%%%
% end body

\end{document}

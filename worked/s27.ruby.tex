\documentclass[10pt,b5j]{tarticle} % B6 縦書き
% \documentclass[10pt,b5j]{tarticle} % B6 縦書き
\AtBeginDvi{\special{papersize=128mm,182mm}} % B6 用用紙サイズ
\usepackage{otf} % Unicode で字を入力するのに必要なパッケージ
\usepackage[size=b6j]{bxpapersize} % B6 用紙サイズを指定
\usepackage[dvipdfmx]{graphicx} % 画像を挿入するためのパッケージ
\usepackage[dvipdfmx]{color} % 色をつけるためのパッケージ
\usepackage{pxrubrica} % ルビを振るためのパッケージ
\usepackage{multicol} % 複数段組を作るためのパッケージ
\setlength{\topmargin}{14mm} % 上下方向のマージン
\addtolength{\topmargin}{-1in} % 
\setlength{\oddsidemargin}{11mm} % 左右方向のマージン
\addtolength{\oddsidemargin}{-1in} % 
\setlength{\textwidth}{154mm} % B6 用
\setlength{\textheight}{108mm} % B6 用
\setlength{\headsep}{0mm} % 
\setlength{\headheight}{0mm} % 
\setlength{\topskip}{0mm} % 
\setlength{\parskip}{0pt} % 
\def\labelenumi{\theenumi、} % 箇条書きのフォーマット
\parindent = 0pt % 段落下げしない

 % B6 用テンプレート読み込み

\begin{document}
% begin header
%%%%% タイトルと作者 ここから %%%%%
\begin{minipage}[c]{0.7\hsize} % タイトルは上から 7 割
    \begin{center}
    % begin title
        {\LARGE
            永遠の水のひろごり % タイトルを入れる
        }
        {\small 
            (昭和二十七年寮歌) % 年などを入れる
        }
    % end title
    \end{center}
\end{minipage}
\begin{minipage}[c]{0.3\hsize} % 作歌作曲は上から 3 割
    \begin{flushright} % 下寄せにする
        % begin name
        村上啓司君 作歌\\田畑実君 作曲 % 作歌・作曲者
        % end name
    \end{flushright}
\end{minipage}
%%%%% タイトルと作者 ここまで %%%%%
% (1 繰り返しなし)
% end header

% begin length
\vspace{0.5em} % タイトル, 作者と歌詞の間に隙間を設ける
\newcommand{\linespace}{0.3em} % 行間の設定
\newcommand{\blocksize}{0.33\hsize} % 段組間の設定
\newcommand{\itemmargin}{3em} % 曲番の位置調整の長さ
% end length
% begin body
%%%%% 歌詞 ここから %%%%%
\begin{enumerate} % 番号の箇条書き ここから
    \setlength{\itemindent}{\itemmargin} % 曲番の位置調整
    \begin{minipage}[c]{\blocksize}
    
        \vspace{\linespace}
        \item~\\
        % 1.
        \ruby{永遠}{えい|えん}の\ruby{水}{みず}の\ruby{広}{ひろ}ごり\\
        \ruby{去}{い}にし\ruby{全}{すべ}ての\ruby[g]{名残}{なご}りをしるす\\
        \ruby{陽}{ひ}の\ruby{光}{ひかり}\ruby{水}{み}の\ruby{面}{も}にわたらず\\
        \ruby{厚}{あつ}き\ruby{雲}{くも}の\ruby{低}{ひく}くたれたり\\
        \ruby{大}{おお}いなる\ruby{水}{みず}と\ruby{強}{つよ}き\ruby{風}{かぜ}との\\
        \ruby[g]{須臾}{しゅゆ}なる\ruby{静}{しず}けさ\ruby{今}{いま}ぞ\ruby{破}{やぶ}れん\\
        \ruby{無限}{む|げん}の\ruby{過去}{か|こ}の\ruby[g]{名残}{なご}りを\ruby{無}{な}みと\\
        \ruby{今}{いま}こそ\ruby{吾等}{われ|ら}\ruby{雄々}{お|お}しく\ruby{立}{た}たん
        
    \end{minipage}
    \begin{minipage}[c]{\blocksize}
        
        \vspace{\linespace}
        \item~\\
        % 2.
        \ruby{再}{ふたた}びす\ruby{宣臂}{せん|ぴ}の\ruby{叫}{さけ}び\\
        \ruby{血}{ち}をもて\ruby{験}{ふ}りし\ruby{訓}{おし}えを\ruby{忘}{わす}る\\
        \ruby{屈辱}{くつ|じょく}の\ruby[g]{条文}{ふみ}は\ruby{結}{むす}ばれ\\
        \ruby{時}{とき}の\ruby{声}{こえ}の\ruby{高}{たか}く\ruby{顕}{あらわ}る\\
        \ruby{核}{かく}\ruby{崩壊}{ほう|かい}なる\ruby{強}{つよ}き\ruby{力}{ちから}は\\
        \ruby[g]{生命}{いのち}と\ruby{愛}{あい}とを\ruby{毀}{こぼ}ち\ruby{捨}{す}てなん\\
        \ruby{再}{ふたた}び\ruby{過去}{か|こ}の\ruby{犯}{あやま}ちせじと\\
        \ruby{今}{いま}こそ\ruby{吾等}{われ|ら}\ruby{凛乎}{りん|こ}と\ruby{起}{た}たん
        
    \end{minipage}
    \begin{minipage}[c]{\blocksize}
        
        \vspace{\linespace}
        \item~\\
        % 3.
        \ruby{北国}{きた|ぐに}の\ruby{樹々}{き|き}の\ruby{直}{なお}さよ\\
        \ruby{牧場}{まき|ば}の\ruby{草}{くさ}の\ruby{色}{いろ}の\ruby[g]{濃緑}{ふか}さよ\\
        \ruby{永}{なが}き\ruby{冬}{ふゆ}\ruby{厳}{きび}しき\ruby{試練}{し|れん}に\\
        \ruby{打}{う}ち\ruby{耐}{た}えたる\ruby{姿}{すがた}\ruby{美}{うる}わし\\
        \ruby{潮風}{しお|かぜ}\ruby{荒}{すさ}べる\ruby{荒磯}{あら|いそ}にさえ\\
        \ruby{名}{な}もなき\ruby{草木}{くさ|き}の\ruby{生}{せい}をば\ruby[g]{享受}{うけ}ぬ\\
        \ruby{自然}{し|ぜん}の\ruby{真理}{しん|り}の\ruby{頌歌}{しょう|か}を\ruby{唱}{うた}い\\
        \ruby{今}{いま}こそ\ruby{吾等}{われ|ら}\ruby{深}{ふか}く\ruby{究}{きわ}めん
    
    \end{minipage}
\end{enumerate} % 番号の箇条書き ここまで
%%%%% 歌詞 ここまで %%%%%
% end body

\end{document}

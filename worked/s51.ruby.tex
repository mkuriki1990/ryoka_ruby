\documentclass[10pt,b5j]{tarticle} % B6 縦書き
% \documentclass[10pt,b5j]{tarticle} % B6 縦書き
\AtBeginDvi{\special{papersize=128mm,182mm}} % B6 用用紙サイズ
\usepackage{otf} % Unicode で字を入力するのに必要なパッケージ
\usepackage[size=b6j]{bxpapersize} % B6 用紙サイズを指定
\usepackage[dvipdfmx]{graphicx} % 画像を挿入するためのパッケージ
\usepackage[dvipdfmx]{color} % 色をつけるためのパッケージ
\usepackage{pxrubrica} % ルビを振るためのパッケージ
\usepackage{multicol} % 複数段組を作るためのパッケージ
\setlength{\topmargin}{14mm} % 上下方向のマージン
\addtolength{\topmargin}{-1in} % 
\setlength{\oddsidemargin}{11mm} % 左右方向のマージン
\addtolength{\oddsidemargin}{-1in} % 
\setlength{\textwidth}{154mm} % B6 用
\setlength{\textheight}{108mm} % B6 用
\setlength{\headsep}{0mm} % 
\setlength{\headheight}{0mm} % 
\setlength{\topskip}{0mm} % 
\setlength{\parskip}{0pt} % 
\def\labelenumi{\theenumi、} % 箇条書きのフォーマット
\parindent = 0pt % 段落下げしない

 % B6 用テンプレート読み込み

\begin{document}
% begin header
%%%%% タイトルと作者 ここから %%%%%
\begin{minipage}[c]{0.7\hsize} % タイトルは上から 7 割
    \begin{center}
    % begin title
        {\LARGE
            いつの日にか % タイトルを入れる
        }
        {\small 
            (昭和五十一年寮歌) % 年などを入れる
        }
    % end title
    \end{center}
\end{minipage}
\begin{minipage}[c]{0.3\hsize} % 作歌作曲は上から 3 割
    \begin{flushright} % 下寄せにする
        % begin name
        小嶋茂君 作歌\\真鍋利徳君 作曲 % 作歌・作曲者
        % end name
    \end{flushright}
\end{minipage}
%%%%% タイトルと作者 ここまで %%%%%
% (1,4,5 繰り返しなし)
% end header

% begin length
\vspace{1.5em} % タイトル, 作者と歌詞の間に隙間を設ける
\newcommand{\linespace}{0.5em} % 行間の設定
\newcommand{\blocksize}{0.33\hsize} % 段組間の設定
\newcommand{\itemmargin}{3em} % 曲番の位置調整の長さ
% end length
% begin body
%%%%% 歌詞 ここから %%%%%
\begin{enumerate} % 番号の箇条書き ここから
    \setlength{\itemindent}{\itemmargin} % 曲番の位置調整
    \begin{minipage}[c]{\blocksize}
    
        \vspace{\linespace}
        \item~\\
        % 1.
        \ruby{夜}{よる}は\ruby{巡}{めぐ}り\\
        \ruby{限}{かぎ}りなき\ruby{光}{ひかり}の\ruby{束}{たば}は\\
        \ruby{樹林}{じゅ|りん}をつらぬきぬ\\
        \ruby{朝}{あさ}の\ruby{静寂}{せい|じゃく}の\ruby{中}{なか}\ruby[g]{一人}{ひとり}にて\\
        \ruby{無為}{む|い}の\ruby{思}{おも}いもち\ruby{嘆}{なげ}き\ruby{憂}{うれ}える\\
        もう\ruby[g]{情熱}{ねつ}もなく\ruby{涙}{なみだ}ながる
        
        \vspace{\linespace}
        \item~\\
        % 2.
        \ruby{何}{なに}を\ruby{求}{もと}め\\
        ほの\ruby{暗}{ぐら}き\ruby{大気}{たい|き}の\ruby{底}{そこ}に\\
        \ruby{真摯}{しん|し}な\ruby{魂}{たましい}は\\
        \ruby{一}{ひと}つの\ruby{心}{こころ}を\ruby{持}{も}ちさまよいぬ\\
        もはや\ruby{言葉}{こと|ば}なく\ruby{凍}{い}てつきて\ruby{立}{た}つ\\
        ポプラを\ruby{見}{み}つめ\ruby{祈}{いの}りささぐ
        
    \end{minipage}
    \begin{minipage}[c]{\blocksize}
        
        \vspace{\linespace}
        \item~\\
        % 3.
        \ruby{大}{おお}き\ruby[g]{精神}{こころ}\\
        \ruby{物思}{もの|おも}う\ruby{我}{われ}らに\\
        いまだあれどかすかなり\\
        \ruby{不毛}{ふ|もう}の\ruby{日々}{ひ|び}はかわき\ruby{過}{す}ぎ\ruby{去}{さ}りぬ\\
        なれどいつの\ruby{日}{ひ}か\ruby{結}{むす}びつけなん\\
        \ruby{我等}{われ|ら}が\ruby{命}{いのち}\ruby{大}{おお}き\ruby{魂}{たま}へ
        
        \vspace{\linespace}
        \item~\\
        % 4.
        \ruby[g]{女性}{ひと}の\ruby{清}{きよ}き\ruby{美}{うつく}しさ\\
        \ruby{真摯}{しん|し}な\ruby[g]{理性}{こころ}の\ruby{輝}{かがや}きにさそわれて\\
        ほのかな\ruby{恋}{こい}の\ruby{想}{おも}い\ruby{胸}{むね}に\\
        なれど\ruby{結}{むす}びえず\\
        あまりに\ruby{深}{ふか}き\ruby{心}{こころ}のあがき\\
        この\ruby{暗}{くら}さに
        
    \end{minipage}
    \begin{minipage}[c]{\blocksize}
        
        \vspace{\linespace}
        \item~\\
        % 5.
        \ruby{深}{ふか}き\ruby{森}{もり}のささやき\\
        \ruby{清冷}{せい|れい}な\ruby{川}{かわ}の\ruby{流}{なが}れに\ruby{聞}{き}きいりて\\
        \ruby{清}{きよ}らかさの\ruby{中}{なか}\ruby{我}{われ}\ruby{息}{いき}しなん\\
        \ruby{物}{もの}を\ruby{思}{おも}わなん\\
        \ruby{静}{しず}けさの\ruby{中}{なか}とけこみいりて\\
        いつの\ruby{日}{ひ}にか
    
    \end{minipage}
\end{enumerate} % 番号の箇条書き ここまで
%%%%% 歌詞 ここまで %%%%%
% end body

\end{document}

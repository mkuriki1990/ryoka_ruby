\documentclass[10pt,b5j]{tarticle} % B6 縦書き
% \documentclass[10pt,b5j]{tarticle} % B6 縦書き
\AtBeginDvi{\special{papersize=128mm,182mm}} % B6 用用紙サイズ
\usepackage{otf} % Unicode で字を入力するのに必要なパッケージ
\usepackage[size=b6j]{bxpapersize} % B6 用紙サイズを指定
\usepackage[dvipdfmx]{graphicx} % 画像を挿入するためのパッケージ
\usepackage[dvipdfmx]{color} % 色をつけるためのパッケージ
\usepackage{pxrubrica} % ルビを振るためのパッケージ
\usepackage{multicol} % 複数段組を作るためのパッケージ
\setlength{\topmargin}{14mm} % 上下方向のマージン
\addtolength{\topmargin}{-1in} % 
\setlength{\oddsidemargin}{11mm} % 左右方向のマージン
\addtolength{\oddsidemargin}{-1in} % 
\setlength{\textwidth}{154mm} % B6 用
\setlength{\textheight}{108mm} % B6 用
\setlength{\headsep}{0mm} % 
\setlength{\headheight}{0mm} % 
\setlength{\topskip}{0mm} % 
\setlength{\parskip}{0pt} % 
\def\labelenumi{\theenumi、} % 箇条書きのフォーマット
\parindent = 0pt % 段落下げしない

 % B6 用テンプレート読み込み

\begin{document}
% begin header
%%%%% タイトルと作者 ここから %%%%%
\begin{minipage}[c]{0.7\hsize} % タイトルは上から 7 割
    \begin{center}
    % begin title
        {\LARGE
            噫妖雲は % タイトルを入れる
        }
        {\small 
            (昭和十年寮歌) % 年などを入れる
        }
    % end title
    \end{center}
\end{minipage}
\begin{minipage}[c]{0.3\hsize} % 作歌作曲は上から 3 割
    \begin{flushright} % 下寄せにする
        % begin name
        川村真君 作歌\\荻野辰夫君 作曲 % 作歌・作曲者
        % end name
    \end{flushright}
\end{minipage}
%%%%% タイトルと作者 ここまで %%%%%
% (1,2,3,6 了あり)
% end header

% begin length
\vspace{1.5em} % タイトル, 作者と歌詞の間に隙間を設ける
\newcommand{\linespace}{0.5em} % 行間の設定
\newcommand{\blocksize}{0.5\hsize} % 段組間の設定
\newcommand{\itemmargin}{3em} % 曲番の位置調整の長さ
% end length
% begin body
%%%%% 歌詞 ここから %%%%%
\begin{enumerate} % 番号の箇条書き ここから
    \setlength{\itemindent}{\itemmargin} % 曲番の位置調整
    \begin{minipage}[c]{\blocksize}
    
        \vspace{\linespace}
        \item~\\
        % 1.
        \ruby{噫}{ああ}\ruby{妖雲}{よう|うん}は\ruby{狂}{くる}へども\\
        \ruby{迪}{みち}を\ruby{恵}{もと}めし\ruby[g]{若人}{わこうど}\ruby{等}{ら}\\
        \ruby{巍然}{ぎ|ぜん}\ruby{四寮}{し|りょう}に\ruby{立}{たて}\ruby{籠}{こ}もり\\
        \ruby{覚醒}{かく|せい}の\ruby{歌}{うた}\ruby[g]{高誦}{うた}ふかな
        
        \vspace{\linespace}
        \item~\\
        % 2.
        \ruby[g]{三年}{みとせ}の\ruby{契}{ちぎり}\ruby{浅}{あさ}からず\\
        \ruby{爛漫}{らん|まん}\ruby{春}{はる}を\ruby{欺}{あざむ}けど\\
        \ruby{銀觴}{ぎん|しょう}\ruby[g]{口辺}{くち}にうつろへば\\
        \ruby[g]{名残}{なごり}の\ruby{春}{はる}を\ruby{惜}{おし}むべし
        
        \vspace{\linespace}
        \item~\\
        % 3.
        \ruby{羊}{ひつじ}の\ruby{群}{むれ}は\ruby{去}{さ}り\ruby{行}{ゆ}きて\\
        \ruby{角笛}{つの|ぶえ}\ruby{遠}{とほ}くこだましぬ\\
        \ruby{夏草}{なつ|くさ}\ruby{深}{ふか}き\ruby[g]{丘上}{をかのへ}に\\
        \ruby{月}{つき}\ruby{三更}{さん|こう}の\ruby{影}{かげ}\ruby{冴}{は}ゆる
        
    \end{minipage}
    \begin{minipage}[c]{\blocksize}
        
        \vspace{\linespace}
        \item~\\
        % 4.
        \ruby{不壊}{ふ|ゑ}の\ruby[g]{生命}{いのち}と\ruby{輝}{かがや}きし\\
        \ruby[g]{緑葉}{みどり}\ruby{漸}{ようや}く\ruby[g]{紅葉}{もみぢ}して\\
        \ruby{今}{いま}\ruby{玲瓏}{れい|ろう}の\ruby{谿谷}{けい|こく}に\\
        \ruby{若}{わか}き\ruby{男}{お}の\ruby{子}{こ}の\ruby[g]{寮歌}{うた}\ruby{消}{き}ゆる
        
        \vspace{\linespace}
        \item~\\
        % 5.
        \ruby{颯々}{さっ|さつ}の\ruby{風}{かぜ}\ruby{音}{おと}\ruby{寒}{さむ}く\\
        \ruby{橇}{そり}の\ruby{音}{ね}\ruby{孤弦}{こ|げん}の\ruby{月}{つき}を\ruby{呼}{よ}ぶ\\
        \ruby{窓}{まど}に\ruby{佇}{たたず}む\ruby{多感}{た|かん}の\ruby[g]{遊子}{こ}\\
        \ruby[g]{今{\CID{13831}}}{こよい}\ruby{何}{なに}をか\ruby{思}{おも}ふらん % CID 宵󠄁
        
        \vspace{\linespace}
        \item~\\
        % 6.
        \ruby{月影}{つき|かげ}\ruby{淡}{あわ}き\ruby{楡}{にれ}の\ruby{陵}{をか}\\
        \ruby{記念}{き|ねん}の\ruby{祭}{まつり}\ruby{終}{おわ}るなり\\
        \ruby{篝火}{かがり|び}\ruby{焚}{た}きて\ruby{我}{われ}は\ruby{今}{いま}\\
        \ruby{静}{しず}かに\ruby[g]{{\CID{13831}}}{よい}を\ruby{誦}{うた}はなん % CID 宵󠄁
    
    \end{minipage}
\end{enumerate} % 番号の箇条書き ここまで
%%%%% 歌詞 ここまで %%%%%
% end body

\end{document}

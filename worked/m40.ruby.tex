\documentclass[10pt,b5j]{tarticle} % B6 縦書き
% \documentclass[10pt,b5j]{tarticle} % B6 縦書き
\AtBeginDvi{\special{papersize=128mm,182mm}} % B6 用用紙サイズ
\usepackage{otf} % Unicode で字を入力するのに必要なパッケージ
\usepackage[size=b6j]{bxpapersize} % B6 用紙サイズを指定
\usepackage[dvipdfmx]{graphicx} % 画像を挿入するためのパッケージ
\usepackage[dvipdfmx]{color} % 色をつけるためのパッケージ
\usepackage{pxrubrica} % ルビを振るためのパッケージ
\usepackage{multicol} % 複数段組を作るためのパッケージ
\setlength{\topmargin}{14mm} % 上下方向のマージン
\addtolength{\topmargin}{-1in} % 
\setlength{\oddsidemargin}{11mm} % 左右方向のマージン
\addtolength{\oddsidemargin}{-1in} % 
\setlength{\textwidth}{154mm} % B6 用
\setlength{\textheight}{108mm} % B6 用
\setlength{\headsep}{0mm} % 
\setlength{\headheight}{0mm} % 
\setlength{\topskip}{0mm} % 
\setlength{\parskip}{0pt} % 
\def\labelenumi{\theenumi、} % 箇条書きのフォーマット
\parindent = 0pt % 段落下げしない

 % B6 用テンプレート読み込み

\begin{document}
% begin header
%%%%% タイトルと作者 ここから %%%%%
\begin{minipage}[c]{0.7\hsize} % タイトルは上から 7 割
    \begin{center}
    % begin title
        {\LARGE
            一帯ゆるき % タイトルを入れる
        }
        {\small 
            (明治四十年寮歌) % 年などを入れる
        }
    % end title
    \end{center}
\end{minipage}
\begin{minipage}[c]{0.3\hsize} % 作歌作曲は上から 3 割
    \begin{flushright} % 下寄せにする
        % begin name
        田中義麿君 作歌\\高松正信君 作曲 % 作歌・作曲者
        % end name
    \end{flushright}
\end{minipage}
%%%%% タイトルと作者 ここまで %%%%%
% (1,2,3 了あり)
% end header

% begin body
\vspace{1.5em} % タイトル, 作者と歌詞の間に隙間を設ける
\newcommand{\linespace}{0.5em} % 行間の設定
\newcommand{\blocksize}{0.33\hsize} % 段組間の設定
%%%%% 歌詞 ここから %%%%%
% begin lilycs
\begin{enumerate} % 番号の箇条書き ここから
    \begin{minipage}[c]{\blocksize}
    
        \vspace{\linespace}
        \item
        % 1.
        \ruby{一帯}{いっ|たい}ゆるき\ruby{石狩}{いし|かり}の\\
        \ruby{源}{みなもと}\ruby{遠}{とお}く\ruby{霞}{かすみ}\ruby{罩}{こ}め\\
        \ruby{五彩}{ご|さい}を\ruby{染}{そ}むる\ruby{夕照}{ゆう|やけ}は\\
        \ruby{手稲}{て|いね}の\ruby{夏}{なつ}の\ruby{栄}{はえ}にして\\
        そこに\ruby{無限}{む|げん}の\ruby[g]{恩寵}{めぐみ}あり\\
        \ruby{是}{これ}\ruby{吾}{わが}\ruby{校}{こう}の\ruby{在}{あ}る\ruby{処}{ところ}
        
        \vspace{\linespace}
        \item
        % 2.
        \ruby{胡沙}{こ|さ}\ruby{吹}{ふ}く\ruby{風}{かぜ}に\ruby{秋}{あき}\ruby{闌}{た}けて\\
        \ruby[g]{黄葉}{もみぢ}\ruby{散}{ち}りしく\ruby[g]{牧場}{まき}\ruby{千里}{せん|り}\\
        \ruby{満野}{まん|や}の\ruby[g]{吹雪}{ふぶき}\ruby{叱咤}{しっ|た}する\\
        エルムの\ruby{姿}{すがた}\ruby{壮}{そう}なれや\\
        そこに\ruby{無限}{む|げん}の\ruby[g]{偉力}{ちから}あり\\
        \ruby{是}{これ}\ruby{吾}{わが}\ruby{寮}{りょう}の\ruby{在}{あ}る\ruby{処}{ところ}

    \end{minipage}
    \begin{minipage}[c]{\blocksize}
        
        \vspace{\linespace}
        \item
        % 3.
        \ruby{偲}{おも}へば\ruby{遠}{とお}き\ruby[g]{三十年}{みそとせ}の\\
        \ruby{榛莽}{しん|もう}あしたの\ruby{日}{ひ}を\ruby{蔽}{おほ}ひ\\
        ゆふべの\ruby{月}{つき}に\ruby[g]{羆熊}{くま}\ruby{吼}{ほ}ゆる\\
        \ruby{北海}{ほっ|かい}の\ruby{野}{の}に\ruby{鋤入}{すき|い}れて\\
        \ruby{偉人}{い|じん}が\ruby{植}{う}ゑし\ruby{桜花}{さくら|ばな}\\
        \ruby{薫}{かほり}は\ruby{高}{たか}し\ruby{千万古}{せん|ばん|こ}
        
        \vspace{\linespace}
        \item
        % 4.
        \ruby{海}{うみ}を\ruby{距}{へだ}てて\ruby{南}{みんなみ}の\\
        \ruby{空}{そら}の\ruby[g]{彼方}{かなた}を\ruby{眺}{なが}むれば\\
        \ruby{古人}{こ|じん}の\ruby{道}{みち}は\ruby{跡}{あと}もなく\\
        \ruby{文明}{ぶん|めい}の\ruby{徳}{とく}は\ruby{尚}{なお}\ruby{成}{な}らず\\
        \ruby{溟濛天}{めい|もう|てん}に\ruby{漲}{みなぎ}りて\\
        \ruby{帰鳥}{き|ちょう}\ruby{夕}{ゆうひ}に\ruby[g]{彷徨}{さまよ}いぬ
        
    \end{minipage}
    \begin{minipage}[c]{\blocksize}

        \vspace{\linespace}
        \item
        % 5.
        \ruby{{\UTF{98B7}}々}{へう|へう}として % UTF 颷
        \ruby{風狂}{かぜ|くる}ひ\\
        \ruby{北海}{ほっ|かい}の\ruby{潮黒}{しお|くろ}むとき\\
        \ruby{電光凄}{でん|こう|すご}く\ruby{駛}{はや}りては\\
        \ruby{鬼}{おに}\ruby{啾々}{しゅう|しゅう}の\ruby{声}{こえ}すなり\\
        \ruby{破邪}{は|じゃ}の\ruby{剣}{つるぎ}を\ruby{右手}{め|て}にして\\
        
        \vspace{\linespace}
        \item
        % 6.
        \ruby{岩間}{いわ|ま}に\ruby{咽}{たけ}ぶ\ruby{渓流}{けい|りゅう}も\\
        \ruby{明日}{あ|す}は\ruby{黄河}{こう|が}に\ruby{波}{なみ}うたむ\\
        \ruby{蟄竜}{ちつ|りょう}\ruby{遂}{つひ}に\ruby{雲}{くも}を\ruby{呼}{よ}び\\
        \ruby{鳳雛}{ほう|すう}やがて\ruby{時}{とき}を\ruby{得}{え}て\\
        \ruby{扶揺}{ふ|よう}に\ruby{搏}{う}って\ruby{騰}{うつ}りなば\\
        \ruby{魍魎}{まう|りゃう}\ruby{遂}{つひ}に\ruby{影}{かげ}もなし
    
    \end{minipage}
\end{enumerate} % 番号の箇条書き ここまで
% end lilycs
%%%%% 歌詞 ここまで %%%%%
% end body

\end{document}

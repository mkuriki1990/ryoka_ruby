\documentclass[10pt,b5j]{tarticle} % B6 縦書き
% \documentclass[10pt,b5j]{tarticle} % B6 縦書き
\AtBeginDvi{\special{papersize=128mm,182mm}} % B6 用用紙サイズ
\usepackage{otf} % Unicode で字を入力するのに必要なパッケージ
\usepackage[size=b6j]{bxpapersize} % B6 用紙サイズを指定
\usepackage[dvipdfmx]{graphicx} % 画像を挿入するためのパッケージ
\usepackage[dvipdfmx]{color} % 色をつけるためのパッケージ
\usepackage{pxrubrica} % ルビを振るためのパッケージ
\usepackage{multicol} % 複数段組を作るためのパッケージ
\setlength{\topmargin}{14mm} % 上下方向のマージン
\addtolength{\topmargin}{-1in} % 
\setlength{\oddsidemargin}{11mm} % 左右方向のマージン
\addtolength{\oddsidemargin}{-1in} % 
\setlength{\textwidth}{154mm} % B6 用
\setlength{\textheight}{108mm} % B6 用
\setlength{\headsep}{0mm} % 
\setlength{\headheight}{0mm} % 
\setlength{\topskip}{0mm} % 
\setlength{\parskip}{0pt} % 
\def\labelenumi{\theenumi、} % 箇条書きのフォーマット
\parindent = 0pt % 段落下げしない

 % B6 用テンプレート読み込み

\begin{document}
% begin header
%%%%% タイトルと作者 ここから %%%%%
\begin{minipage}[c]{0.7\hsize} % タイトルは上から 7 割
    \begin{center}
    % begin title
        {\LARGE
            浅緑燃ゆる % タイトルを入れる
        }
        {\small 
            (昭和二十二年第四十回記念祭歌) % 年などを入れる
        }
    % end title
    \end{center}
\end{minipage}
\begin{minipage}[c]{0.3\hsize} % 作歌作曲は上から 3 割
    \begin{flushright} % 下寄せにする
        % begin name
        山家貫之君 作歌\\堀井洵君 作曲 % 作歌・作曲者
        % end name
    \end{flushright}
\end{minipage}
%%%%% タイトルと作者 ここまで %%%%%
% (1,2,4,5 繰り返しなし)
% end header

% begin length
\vspace{1.5em} % タイトル, 作者と歌詞の間に隙間を設ける
\newcommand{\linespace}{0.5em} % 行間の設定
\newcommand{\blocksize}{0.33\hsize} % 段組間の設定
\newcommand{\itemmargin}{3em} % 曲番の位置調整の長さ
% end length
% begin body
%%%%% 歌詞 ここから %%%%%
\begin{enumerate} % 番号の箇条書き ここから
    \setlength{\itemindent}{\itemmargin} % 曲番の位置調整
    \begin{minipage}[c]{\blocksize}
    
        \vspace{\linespace}
        \item~\\
        % 1.
        \ruby[g]{浅緑}{さみどり}\ruby{燃}{も}ゆる\ruby{北}{きた}の\ruby[g]{曠里}{さと}\\
        \ruby{荒}{すさ}ぶ\ruby{嵐}{あらし}を\ruby{身}{み}に\ruby{受}{う}けて\\
        \ruby[g]{神秘}{くしび}の\ruby{扉}{とびら}\ruby{開}{あ}け\ruby{放}{はな}ち\\
        \ruby{雄叫}{を|たけ}び\ruby{高}{たか}く\ruby{濁世}{ぢょく|せい}に\\
        \ruby{叱{\CID{8411}}}{しっ|た}の\ruby{剣}{けん}を\ruby{振}{ふ}るふかな % CID 咜
        
        \vspace{\linespace}
        \item~\\
        % 2.
        \ruby[g]{沈黙}{しじま}の\ruby[g]{楡林}{もり}のほの\ruby{暗}{ぐら}く\\
        \ruby{友}{とも}と\ruby[g]{高望}{のぞみ}を\ruby{語}{かた}りてし\\
        \ruby[g]{三年}{みとせ}の\ruby{夢}{ゆめ}は\ruby{淡}{あは}くとも\\
        \ruby[g]{羽搏}{はばた}かんかな\ruby{大鳳}{たい|ほう}は\\
        アンデスの\ruby{嶺}{みね}\ruby{越}{こ}えゆかん
        
    \end{minipage}
    \begin{minipage}[c]{\blocksize}
        
        \vspace{\linespace}
        \item~\\
        % 3.
        ソロモンの\ruby[g]{栄華}{はえ}すでになし\\
        \ruby[g]{血涙}{ち}もて\ruby{築}{きづ}きし\ruby{幾春秋}{いく|しゅん|じゅう}\\
        \ruby{花}{はな}を\ruby{褥}{しとね}に\ruby[g]{仮睡}{まどろ}めば\\
        \ruby{春}{はる}\ruby{駘蕩}{たい|たう}の\ruby[g]{微風}{かぜ}の\ruby{香}{か}に\\
        \ruby[g]{私語}{ささめ}く\ruby[g]{永遠}{とは}の\ruby{理想}{り|さう}かな
        
        \vspace{\linespace}
        \item~\\
        % 4.
        \ruby{北斗}{ほく|と}の\ruby[g]{啓示}{さとし}なほ\ruby{清}{きよ}く\\
        \ruby[g]{今{\CID{13831}}}{こよひ}\ruby{四寮}{し|れう}に\ruby{輝}{かがや}けば\\ % CID 宵󠄁
        \ruby{猛}{たけ}き\ruby{遊児}{ゆう|じ}の\ruby{熱血}{ねっ|けつ}は\\
        ナイルの\ruby{河}{かは}のなほ\ruby{浩}{ひろ}く\\
        \ruby{乱}{みだ}れし\ruby{世}{よ}をば\ruby{呑}{の}みほさん
        
    \end{minipage}
    \begin{minipage}[c]{\blocksize}
        
        \vspace{\linespace}
        \item~\\
        % 5.
        \ruby{青史}{せい|し}は\ruby{薫}{かを}る\ruby[g]{七十星霜}{なそとせ}の\\
        \ruby[g]{崇高}{たか}き\ruby{歴史}{れき|し}を\ruby[g]{承継}{うけつ}ぎて\\
        \ruby[g]{明日}{あす}\ruby{創造}{さう|ざう}の\ruby[g]{首途}{かどいで}に\\
        \ruby[g]{今日}{きょう}\ruby[g]{四十回}{よそたび}の\ruby{記念祭}{き|ねん|さい}\\
        \ruby{浩歌}{たか|うた}はんかな\ruby{吾}{わ}が\ruby{友}{とも}よ
    
    \end{minipage}
\end{enumerate} % 番号の箇条書き ここまで
%%%%% 歌詞 ここまで %%%%%
% end body

\end{document}

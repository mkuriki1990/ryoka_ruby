\documentclass[10pt,b5j]{tarticle} % B6 縦書き
% \documentclass[10pt,b5j]{tarticle} % B6 縦書き
\AtBeginDvi{\special{papersize=128mm,182mm}} % B6 用用紙サイズ
\usepackage{otf} % Unicode で字を入力するのに必要なパッケージ
\usepackage[size=b6j]{bxpapersize} % B6 用紙サイズを指定
\usepackage[dvipdfmx]{graphicx} % 画像を挿入するためのパッケージ
\usepackage[dvipdfmx]{color} % 色をつけるためのパッケージ
\usepackage{pxrubrica} % ルビを振るためのパッケージ
\usepackage{plext} % 漢数字の enumerate を使うためのパッケージ
\usepackage{multicol} % 複数段組を作るためのパッケージ
\setlength{\topmargin}{14mm} % 上下方向のマージン
\addtolength{\topmargin}{-1in} % 
\setlength{\oddsidemargin}{11mm} % 左右方向のマージン
\addtolength{\oddsidemargin}{-1in} % 
\setlength{\textwidth}{154mm} % B6 用
\setlength{\textheight}{108mm} % B6 用
\setlength{\headsep}{0mm} % 
\setlength{\headheight}{0mm} % 
\setlength{\topskip}{0mm} % 
\setlength{\parskip}{0pt} % 
\def\theenumi{\Kanji{enumi}} % 箇条書きのフォーマットを漢数字に変更
\parindent = 0pt % 段落下げしない
\pagestyle{empty} % すべてのページ番号を消去
% \renewcommand{\baselinestretch}{0.9} % 行間の倍率
 % B6 用テンプレート読み込み

\begin{document}
% begin header
%%%%% タイトルと作者 ここから %%%%%
\begin{minipage}[c]{0.7\hsize} % タイトルは上から 7 割
    \begin{center}
    % begin title
        {\LARGE
            平和の光輝ける % タイトルを入れる
        }
        {\small 
            (昭和六年寮歌) % 年などを入れる
        }
    % end title
    \end{center}
\end{minipage}
\begin{minipage}[c]{0.3\hsize} % 作歌作曲は上から 3 割
    \begin{flushright} % 下寄せにする
        % begin name
        広瀬英三君 作歌\\金景洙君 作曲 % 作歌・作曲者
        % end name
    \end{flushright}
\end{minipage}
%%%%% タイトルと作者 ここまで %%%%%
% (1,5 了あり)
% end header

% begin length
\vspace{1.5em} % タイトル, 作者と歌詞の間に隙間を設ける
\newcommand{\linespace}{0.5em} % 行間の設定
\newcommand{\blocksize}{0.33\hsize} % 段組間の設定
\newcommand{\itemmargin}{3em} % 曲番の位置調整の長さ
% end length
% begin body
%%%%% 歌詞 ここから %%%%%
\begin{enumerate} % 番号の箇条書き ここから
    \setlength{\itemindent}{\itemmargin} % 曲番の位置調整
    \begin{minipage}[c]{\blocksize}
    
        \vspace{\linespace}
        \item~\\
        % 1.
        \ruby{平和}{へい|わ}の\ruby{光}{ひかり}\ruby{輝}{かがや}ける\\
        \ruby{春}{はる}\ruby{未}{ま}だ\ruby{浅}{あさ}き\ruby{曙}{あけぼの}に\\
        \ruby{綾}{あや}なす\ruby[g]{紫雲}{くも}を\ruby{分}{わ}け\ruby{出}{い}でて\\
        \ruby[g]{彩色}{いろど}られ\ruby{行}{ゆ}く\ruby{青春}{せい|しゅん}の\\
        \ruby[g]{久遠}{くをん}の\ruby[g]{迷夢}{ゆめ}を\ruby{求}{もと}めつつ\\
        \ruby{声}{こゑ}\ruby{高}{たか}らかに\ruby{歌}{うた}はなん
        
        \vspace{\linespace}
        \item~\\
        % 2.
        \ruby{陽光}{やう|くわう}\ruby{燦然}{さん|ぜん}\ruby{乱}{みだ}れ\ruby{入}{い}る\\
        \ruby{夏}{なつ}の\ruby{窓辺}{まど|べ}に\ruby{書}{ふみ}よめば\\
        \ruby[g]{寮庭}{には}に\ruby{年}{とし}\ruby{経}{ふ}るアカシヤの\\
        \ruby{床}{ゆか}しき\ruby[g]{薫香}{かをり}\ruby{漂}{ただよ}ひて\\
        いつか\ruby[g]{心懐}{こころ}の\ruby{極}{きわ}みなく\\
        \ruby{蝦夷}{え|ぞ}の\ruby{昔}{むかし}にいたる\ruby{哉}{かな}
        
    \end{minipage}
    \begin{minipage}[c]{\blocksize}
        
        \vspace{\linespace}
        \item~\\
        % 3.
        \ruby{秋}{あき}も\ruby{闌}{た}け\ruby{行}{ゆ}く\ruby[g]{北溟}{きた}の\ruby{州}{くに}\\
        \ruby{白楊}{はく|やう}の\ruby{華}{はな}\ruby{乱}{みだ}れとぶ\\
        \ruby{聖}{きよ}き\ruby{都}{みやこ}に\ruby[g]{寂寥}{さみしさ}の\\
        \ruby{静}{しづ}かに\ruby{迫}{せま}る\ruby{此}{こ}の\ruby{夕}{ゆふ}べ\\
        \ruby{思索}{し|さく}の\ruby{迪}{みち}を\ruby{恵}{たづ}ぬれば\\
        \ruby{楡林}{ゆ|りん}に\ruby{鐘}{かね}はなり\ruby{響}{ひび}く
        
        \vspace{\linespace}
        \item~\\
        % 4.
        \ruby{馬橇}{ば|そり}の\ruby{鈴}{すず}の\ruby{音}{ね}も\ruby{絶}{た}えし\\
        \ruby{雪}{ゆき}の\ruby{大路}{おほ|ぢ}を\ruby{歩}{あゆ}みつつ\\
        \ruby{声}{こゑ}をかぎりに\ruby[g]{寮歌}{うた}うたふ\\
        \ruby{凍}{こほ}れるものみな\ruby{揺}{うご}かして\\
        \ruby{星斗}{せい|と}は\ruby{高}{たか}く\ruby{冴}{は}ゆる\ruby{夜}{よ}の\\
        \ruby[g]{大空}{そら}のかなたへ\ruby{消}{き}えて\ruby{行}{ゆ}く
        
    \end{minipage}
    \begin{minipage}[c]{\blocksize}
        
        \vspace{\linespace}
        \item~\\
        % 5.
        \ruby{高}{たか}き「\ruby{理想}{り|さう}」と「\ruby{純情}{じゅん|じゃう}」に\\
        たぎる\ruby[g]{生命}{いのち}を\ruby{託}{たく}しつつ\\
        \ruby{憧}{あこが}れ\ruby{集}{つど}ふ\ruby[g]{若人}{わこうど}の\\
        \ruby[g]{情熱}{おもひ}のかがり\ruby{火}{び}\ruby{打}{う}ち\ruby{囲}{かこ}み\\
        \ruby{月下}{げっ|か}に\ruby{酌}{く}むや\ruby{楡}{にれ}の\ruby{宴}{えん}\\
        いざや\ruby[g]{謳歌}{たた}へん\ruby{記念祭}{き|ねん|さい}
    
    \end{minipage}
\end{enumerate} % 番号の箇条書き ここまで
%%%%% 歌詞 ここまで %%%%%
% end body

\end{document}

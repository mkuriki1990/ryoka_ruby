\documentclass[10pt,b5j]{tarticle} % B6 縦書き
% \documentclass[10pt,b5j]{tarticle} % B6 縦書き
\AtBeginDvi{\special{papersize=128mm,182mm}} % B6 用用紙サイズ
\usepackage{otf} % Unicode で字を入力するのに必要なパッケージ
\usepackage[size=b6j]{bxpapersize} % B6 用紙サイズを指定
\usepackage[dvipdfmx]{graphicx} % 画像を挿入するためのパッケージ
\usepackage[dvipdfmx]{color} % 色をつけるためのパッケージ
\usepackage{pxrubrica} % ルビを振るためのパッケージ
\usepackage{multicol} % 複数段組を作るためのパッケージ
\setlength{\topmargin}{14mm} % 上下方向のマージン
\addtolength{\topmargin}{-1in} % 
\setlength{\oddsidemargin}{11mm} % 左右方向のマージン
\addtolength{\oddsidemargin}{-1in} % 
\setlength{\textwidth}{154mm} % B6 用
\setlength{\textheight}{108mm} % B6 用
\setlength{\headsep}{0mm} % 
\setlength{\headheight}{0mm} % 
\setlength{\topskip}{0mm} % 
\setlength{\parskip}{0pt} % 
\def\labelenumi{\theenumi、} % 箇条書きのフォーマット
\parindent = 0pt % 段落下げしない

 % B6 用テンプレート読み込み

\begin{document}
% begin header
%%%%% タイトルと作者 ここから %%%%%
\begin{minipage}[c]{0.7\hsize} % タイトルは上から 7 割
    \begin{center}
    % begin title
        {\LARGE
            藻岩の緑 % タイトルを入れる
        }
        {\small 
            (明治四十四年寮歌) % 年などを入れる
        }
    % end title
    \end{center}
\end{minipage}
\begin{minipage}[c]{0.3\hsize} % 作歌作曲は上から 3 割
    \begin{flushright} % 下寄せにする
        % begin name
        松山茂助君 作歌\\柳沢秀雄君 作曲 % 作歌・作曲者
        % end name
    \end{flushright}
\end{minipage}
%%%%% タイトルと作者 ここまで %%%%%
% (1,2,3,4 了あり)
% end header

% begin length
\vspace{1.5em} % タイトル, 作者と歌詞の間に隙間を設ける
\newcommand{\linespace}{0.5em} % 行間の設定
\newcommand{\blocksize}{0.33\hsize} % 段組間の設定
\newcommand{\itemmargin}{3em} % 曲番の位置調整の長さ
% end length
% begin body
%%%%% 歌詞 ここから %%%%%
\begin{enumerate} % 番号の箇条書き ここから
    \setlength{\itemindent}{\itemmargin} % 曲番の位置調整
    \begin{minipage}[c]{\blocksize}
    
        \vspace{\linespace}
        \item~\\
        % 1.
        \ruby{藻岩}{も|いわ}の\ruby{緑}{みどり}\ruby{春}{はる}\ruby{闌}{た}けて\\
        \ruby{万朶}{ばん|だ}\ruby{一朶}{いち|だ}の\ruby{朝霞}{あさ|がすみ}\\
        \ruby[g]{憧憬}{あこがれ}\ruby{彩}{あや}と\ruby{流}{なが}れては\\
        \ruby{花}{はな}\ruby{皆}{みな}\ruby{奇}{く}しき\ruby{香}{か}ならずや\\
        \ruby{若}{わか}き\ruby{血潮}{ち|しほ}の\ruby{踊}{おど}る\ruby{時}{とき}\\
        \ruby{希望}{き|ぼう}の\ruby{前途}{ぜん|と}\ruby{光}{ひかり}あり
        
        \vspace{\linespace}
        \item~\\
        % 2.
        \ruby{青葉}{あお|ば}\ruby{波}{なみ}よるアカシヤの\\
        \ruby{薫}{かを}る\ruby{木影}{こ|かげ}に\ruby{立}{た}ちよれば\\
        \ruby{長風}{ちゃう|ふう}\ruby{夏}{なつ}の\ruby{雲}{くも}ゆらぎ\\
        \ruby{秋}{あき}は\ruby{牧場}{まき|ば}の\ruby{夕}{ゆふ}まぐれ\\
        \ruby{鐘声}{しょう|せい}\ruby{止}{や}みて\ruby{今}{いま}\ruby{暫}{しば}し\\
        \ruby{牛}{うし}の\ruby{背}{せ}に\ruby{散}{ち}る\ruby{蔦}{つた}\ruby[g]{紅葉}{もみぢ}
        
    \end{minipage}
    \begin{minipage}[c]{\blocksize}
        
        \vspace{\linespace}
        \item~\\
        % 3.
        あはれ「\ruby{美}{み}の\ruby{国}{くに}」\ruby{石狩}{いし|かり}の\\
        \ruby{自然}{し|ぜん}を\ruby{己}{おの}が\ruby{揺籃}{えう|らん}に\\
        おほし\ruby{立}{た}つ\ruby{可}{べ}き\ruby{人}{ひと}\ruby{皆}{みな}の\\
        \ruby{意気}{い|き}\ruby{紅霓}{こう|げい}に\ruby{似}{に}たるかな\\
        \ruby{一撃}{いち|げき}\ruby{万里}{ばん|り}す\ruby{大鵬}{おお|とり}の\\
        \ruby{翼}{つばさ}\ruby[g]{整装}{つくろ}ふ\ruby{思}{おもひ}あり
        
        \vspace{\linespace}
        \item~\\
        % 4.
        \ruby{斗南}{と|なん}の\ruby{翼}{つばさ}\ruby{拡}{ひろ}げては\\
        \ruby{天地}{てん|ち}\ruby{広}{ひろ}しと\ruby{誰}{だれ}か\ruby{云}{い}ふ\\
        \ruby{雲}{くも}より\ruby{高}{たか}きアンデスの\\
        \ruby{裾野}{すそ|の}に\ruby{友}{とも}よ\ruby{羊}{ひつじ}\ruby{逐}{お}へ\\
        \ruby{天}{てん}に\ruby{漲}{みなぎ}るアマゾンの\\
        \ruby{岸辺}{きし|べ}の\ruby{森}{もり}に\ruby{斧}{おの}を\ruby{振}{ふ}れ
        
    \end{minipage}
    \begin{minipage}[c]{\blocksize}
        
        \vspace{\linespace}
        \item~\\
        % 5.
        \ruby{弦月}{げん|げつ}\ruby{落}{お}ちて\ruby{白楊}{はく|よう}の\\
        \ruby{樹林}{じゅ|りん}の\ruby{暗}{やみ}の\ruby{深}{ふか}き\ruby{時}{とき}\\
        \ruby{八荒}{はっ|くわう}\ruby{裂}{さ}けて\ruby{万籟}{ばん|らい}の\\
        \ruby{声}{こえ}すさまじく\ruby[g]{吹雪}{ふぶ}く\ruby{時}{とき}\\
        \ruby{世}{よ}の\ruby{濁流}{だく|りゅう}を\ruby{叱{\UTF{549C}}}{しっ|た}して\\ % UTF 咜
        \ruby{巨人}{きょ|じん}の\ruby{叫}{さけ}び\ruby{茲}{ここ}にあり
        
        \vspace{\linespace}
        \item~\\
        % 6.
        \ruby{浮華}{ふ|か}\ruby{軽佻}{けい|てう}の\ruby{風}{かぜ}あれて\\
        \ruby{驕奢}{けう|しゃ}の\ruby{波}{なみ}は\ruby{狂}{くる}ふとも\\
        \ruby{北斗}{ほく|と}の\ruby{光}{ひかり}\ruby{清}{きよ}ければ\\
        \ruby{世}{よ}は\ruby[g]{永久}{とこしへ}に\ruby{我}{わが}\ruby{世}{よ}なり\\
        \ruby{聞}{き}けや\ruby{人々}{ひと|びと}\ruby{北州}{ほく|しう}に\\
        \ruby{正気}{せい|き}\ruby{溢}{あふ}るる\ruby{意気}{い|き}の\ruby{歌}{うた}
    
    \end{minipage}
\end{enumerate} % 番号の箇条書き ここまで
%%%%% 歌詞 ここまで %%%%%
% end body

\end{document}

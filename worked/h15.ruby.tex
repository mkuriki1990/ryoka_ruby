\documentclass[10pt,b5j]{tarticle} % B6 縦書き
% \documentclass[10pt,b5j]{tarticle} % B6 縦書き
\AtBeginDvi{\special{papersize=128mm,182mm}} % B6 用用紙サイズ
\usepackage{otf} % Unicode で字を入力するのに必要なパッケージ
\usepackage[size=b6j]{bxpapersize} % B6 用紙サイズを指定
\usepackage[dvipdfmx]{graphicx} % 画像を挿入するためのパッケージ
\usepackage[dvipdfmx]{color} % 色をつけるためのパッケージ
\usepackage{pxrubrica} % ルビを振るためのパッケージ
\usepackage{multicol} % 複数段組を作るためのパッケージ
\setlength{\topmargin}{14mm} % 上下方向のマージン
\addtolength{\topmargin}{-1in} % 
\setlength{\oddsidemargin}{11mm} % 左右方向のマージン
\addtolength{\oddsidemargin}{-1in} % 
\setlength{\textwidth}{154mm} % B6 用
\setlength{\textheight}{108mm} % B6 用
\setlength{\headsep}{0mm} % 
\setlength{\headheight}{0mm} % 
\setlength{\topskip}{0mm} % 
\setlength{\parskip}{0pt} % 
\def\labelenumi{\theenumi、} % 箇条書きのフォーマット
\parindent = 0pt % 段落下げしない

 % B6 用テンプレート読み込み

\renewcommand{\baselinestretch}{0.85} % 行間の倍率
\begin{document}
% begin header
%%%%% タイトルと作者 ここから %%%%%
\begin{minipage}[c]{0.7\hsize} % タイトルは上から 7 割
    \begin{center}
    % begin title
        {\LARGE
            ああグッと % タイトルを入れる
        }
        {\small 
            (平成十五年度寮歌) % 年などを入れる
        }
    % end title
    \end{center}
\end{minipage}
\begin{minipage}[c]{0.3\hsize} % 作歌作曲は上から 3 割
    \begin{flushright} % 下寄せにする
        % begin name
        井口拓君 作歌\\持田翼君 作曲 % 作歌・作曲者
        % end name
    \end{flushright}
\end{minipage}
%%%%% タイトルと作者 ここまで %%%%%
% (1,2,3,9,10 了なし繰り返しあり)
% end header

% begin length
\vspace{0.2em} % タイトル, 作者と歌詞の間に隙間を設ける
\newcommand{\linespace}{-0.1em} % 行間の設定
\newcommand{\blocksize}{0.33\hsize} % 段組間の設定
\newcommand{\itemmargin}{3em} % 曲番の位置調整の長さ
% end length
% begin body
%%%%% 歌詞 ここから %%%%%
\begin{enumerate} % 番号の箇条書き ここから
    \setlength{\itemindent}{\itemmargin}
    \begin{minipage}[c]{\blocksize}
    
        \vspace{\linespace}
        \item~\\
        % 1.
        もしも \ruby{海}{うみ}が \ruby{酒}{さけ}ならば\\
        お\ruby{前}{まえ}は\ruby{魚}{さかな}に なるという\\
        \ruby{俺}{おれ}は\ruby[g]{{\UTF{fa46}}}{なぎさ}の \ruby{貝}{かい}になる\\ % CID 渚
        \ruby{波}{なみ}が\ruby{来}{く}るたび \ruby{酒}{さけ}を\ruby{飲}{の}む
        
        \vspace{\linespace}
        \item~\\
        % 2.
        つまみはそうさ \ruby{俺}{おれ}の\ruby{脳}{のう}\\
        \ruby{酒}{さけ}にとろけた \ruby{脳}{のう}みそさ\\
        \ruby{代}{か}わりにお\ruby{前}{まえ}を \ruby{盃}{さかづき}に\\
        \ruby{空}{から}の\ruby{頭蓋}{ず|がい}に \ruby{酒}{さけ}を\ruby{注}{つ}ぐ
        
        \vspace{\linespace}
        \item~\\
        % 3.
        \ruby{明日}{あ|す}は\ruby{泥土}{でい|ど}に \ruby{墜}{お}ちるとも\\
        \ruby{今}{いま}は\ruby{昇}{のぼ}らん はしご\ruby{酒}{ざけ}\\
        \ruby{美}{うま}しの\ruby{盃}{つき}を \ruby{重}{かさ}ねては\\
        その\ruby{身}{み}\ruby{月}{つき}にも \ruby{届}{とど}くべし
        
        \vspace{\linespace}
        \item~\\
        % 4.
        \ruby{盃}{つき}もめぐりて \ruby{今}{いま}や\ruby{今}{いま}\\
        \ruby{魑魅魍魎}{ち|み|もう|りょう}が \ruby{顔}{かお}を\ruby{出}{だ}す\\
        ヤマタノオロチ \ruby{現}{あらわ}れる\\
        \ruby{大}{おお}トラ \ruby{小}{こ}トラ\ruby{管}{くだ}を\ruby{巻}{ま}く
        
    \end{minipage}
    \begin{minipage}[c]{\blocksize}

        \vspace{\linespace}
        \item~\\
        % 5.
        \ruby{更}{ふ}け\ruby{行}{ゆ}く\ruby{夜}{よる}に \ruby{浮}{う}かぶ\ruby{月}{つき}\\
        \ruby{窓辺}{まど|べ}にうつる \ruby{影}{かげ}は\ruby{今}{いま}\\
        \ruby{何}{なに}をし\ruby{何}{なに}を されるのか\\
        \ruby{月}{つき}は\ruby{黙}{だま}って \ruby{見}{み}るばかり
        
        \vspace{\linespace}
        \item~\\
        % 6.
        \ruby{中天}{ちゅう|てん}\ruby{高}{たか}く \ruby{日}{ひ}は\ruby{昇}{のぼ}り\\
        \ruby[g]{今日}{きょう}もマグロの \ruby{大漁旗}{たい|りょう|き}\\
        \ruby{死屍累々}{し|し|るい|るい}の \ruby{戦場}{せん|じょう}に\\
        \ruby{兵}{つわもの}どもが \ruby{夢}{ゆめ}の\ruby{跡}{あと}
        
        \vspace{\linespace}
        \item~\\
        % 7.
        \ruby{天}{てん}の\ruby{夢}{ゆめ}から \ruby{落}{お}っこちて\\
        \ruby[g]{今日}{きょう}は\ruby{地}{ち}を\ruby{這}{は}う \ruby[g]{宿酔}{ふつかよい}\\
        「なぜ\ruby{繰}{く}り\ruby{返}{かえ}す \ruby{過}{あやま}ちを」\\
        \ruby{空}{むな}しく\ruby{響}{ひび}く いつもの\ruby{問}{と}い
        
        \vspace{\linespace}
        \item~\\
        % 8.
        \ruby{積}{つ}んでは\ruby{崩}{くず}す \ruby{盃}{さかづき}は\\
        \ruby{賽}{さい}の\ruby[g]{河原}{かわら}の \ruby{石積}{いし|つ}みか\\
        それでもいつか \ruby{天}{てん}に\ruby{着}{つ}く\\
        その\ruby{日}{ひ}を\ruby{信}{しん}じ \ruby{盃}{つき}を\ruby{酌}{く}む
        
    \end{minipage}
    \begin{minipage}[c]{\blocksize}
        
        \vspace{\linespace}
        \item~\\
        % 9.
        とかく\ruby{憂}{うれい}の \ruby{多}{おお}い\ruby{世}{よ}を\\
        されば\ruby{払}{はら}えよ \ruby{玉帚}{たま|ははき}\\
        \ruby{積}{つ}もる\ruby{芥}{あくた}の \ruby{流}{なが}れては\\
        \ruby{自}{おの}ずと\ruby{心}{こころ} \ruby{開}{ひら}くべし
        
        \vspace{\linespace}
        \item[十]~\\
        % 10.
        たとえ\ruby{百年}{ひゃく|ねん} \ruby{生}{い}きたとて\\
        わずかに\ruby{三万}{さん|まん}\ruby{六千日}{ろく|せん|にち}\\
        されば\ruby{尽}{つ}くさん この\ruby{盃}{はい}を\\
        \ruby{一日}{いち|にち}\ruby{必}{かなら}ず \ruby{三百杯}{さん|びゃく|ぱい}
    
    \end{minipage}
\end{enumerate} % 番号の箇条書き ここまで
%%%%% 歌詞 ここまで %%%%%
% end body

\end{document}

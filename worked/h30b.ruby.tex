\documentclass[10pt,b5j]{tarticle} % B6 縦書き
% \documentclass[10pt,b5j]{tarticle} % B6 縦書き
\AtBeginDvi{\special{papersize=128mm,182mm}} % B6 用用紙サイズ
\usepackage{otf} % Unicode で字を入力するのに必要なパッケージ
\usepackage[size=b6j]{bxpapersize} % B6 用紙サイズを指定
\usepackage[dvipdfmx]{graphicx} % 画像を挿入するためのパッケージ
\usepackage[dvipdfmx]{color} % 色をつけるためのパッケージ
\usepackage{pxrubrica} % ルビを振るためのパッケージ
\usepackage{plext} % 漢数字の enumerate を使うためのパッケージ
\usepackage{multicol} % 複数段組を作るためのパッケージ
\setlength{\topmargin}{14mm} % 上下方向のマージン
\addtolength{\topmargin}{-1in} % 
\setlength{\oddsidemargin}{11mm} % 左右方向のマージン
\addtolength{\oddsidemargin}{-1in} % 
\setlength{\textwidth}{154mm} % B6 用
\setlength{\textheight}{108mm} % B6 用
\setlength{\headsep}{0mm} % 
\setlength{\headheight}{0mm} % 
\setlength{\topskip}{0mm} % 
\setlength{\parskip}{0pt} % 
\def\theenumi{\Kanji{enumi}} % 箇条書きのフォーマットを漢数字に変更
\parindent = 0pt % 段落下げしない
\pagestyle{empty} % すべてのページ番号を消去
% \renewcommand{\baselinestretch}{0.9} % 行間の倍率
 % B6 用テンプレート読み込み

\begin{document}
% begin header
%%%%% タイトルと作者 ここから %%%%%
\begin{minipage}[c]{0.7\hsize} % タイトルは上から 7 割
    \begin{center}
    % begin title
        {\LARGE
            広がりし海原に % タイトルを入れる
        }
        {\small 
            (平成三十年度寮歌) % 年などを入れる
        }
    % end title
    \end{center}
\end{minipage}
\begin{minipage}[c]{0.3\hsize} % 作歌作曲は上から 3 割
    \begin{flushright} % 下寄せにする
        % begin name
        樋浦一希君 作歌・作曲 % 作歌・作曲者
        % end name
    \end{flushright}
\end{minipage}
%%%%% タイトルと作者 ここまで %%%%%
% (2,5 繰り返しなし)
% end header

% begin length
\vspace{1.5em} % タイトル, 作者と歌詞の間に隙間を設ける
\newcommand{\linespace}{0.5em} % 行間の設定
\newcommand{\blocksize}{0.33\hsize} % 段組間の設定
\newcommand{\itemmargin}{3em} % 曲番の位置調整の長さ
% end length
% begin body
%%%%% 歌詞 ここから %%%%%
\begin{enumerate} % 番号の箇条書き ここから
    \setlength{\itemindent}{\itemmargin} % 曲番の位置調整
    \begin{minipage}[c]{\blocksize}
    
        \vspace{\linespace}
        \item~\\
        % 1.
        \ruby{春}{はる}あけぼのの\ruby{夢}{ゆめ}に\ruby{見}{み}て\\
        カムイの\ruby{声}{こえ}に\ruby{導}{みちび}かれ\\
        \ruby{舟}{ふね}をこぎいで\ruby{流}{なが}れ\ruby{来}{こ}ぬ\\
        \ruby{北都}{ほく|と}\ruby{夜明}{よ|あ}けの\ruby{金字塔}{きん|じ|とう}\\
        \ruby{広}{ひろ}がりし\ruby[g]{草原}{の}に ひとりたち\\
        はるかなる\ruby[g]{大雪}{ゆき}の\ruby{山}{やま}\\
        のぞみみん
        
        \vspace{\linespace}
        \item~\\
        % 2.
        \ruby{夏}{なつ}\ruby{宵闇}{よい|やみ}の\ruby{緑風}{りょく|ふう}に\\
        \ruby{森}{もり}が\ruby{葉音}{は|おと}を\ruby{雨}{あめ}ときき\\
        \ruby{楡}{にれ}の\ruby{木立}{こ|だち}をさまよえば\\
        \ruby{紅}{くれない}はゆる\ruby[g]{山小屋}{こや}ひとつ\\
        \ruby{広}{ひろ}がりし\ruby[g]{高原}{の}に ひとりたち\\
        はるかなる\ruby[g]{天空}{そら}の\ruby{星}{ほし}\\
        \ruby{身}{み}に\ruby{浴}{あ}びん
        
    \end{minipage}
    \begin{minipage}[c]{\blocksize}
        
        \vspace{\linespace}
        \item~\\
        % 3.
        \ruby{秋}{あき}\ruby{夕暮}{ゆう|ぐ}れの\ruby{鹿}{しか}の\ruby{声}{ね}に\\
        \ruby{恵}{めぐ}みの\ruby[g]{季節}{とき}は\ruby{過}{す}ぎゆきて\\
        \ruby{入日}{いり|ひ}の\ruby{茜}{あかね}に\ruby{涙}{なんだ}する\\
        \ruby{冬}{ふゆ}\ruby{音}{おと}せまりき\ruby{危機}{き|き}\ruby{焦燥}{しょう|そう}\\
        \ruby{広}{ひろ}がりし\ruby[g]{牧野}{の}に ひとりたち\\
        はるかなる\ruby[g]{シベリア}{きた}の\ruby{風}{かぜ}\\
        \ruby{気}{き}も\ruby[g]{霧散}{ちら}す
        
        \vspace{\linespace}
        \item~\\
        % 4.
        \ruby{冬}{ふゆ}つとめてのゆめうつつ\\
        かそかに\ruby{遠}{とお}く\ruby[g]{銀狼}{ぎん}の\ruby[g]{咆哮}{こえ}\\
        \ruby{凍}{い}てつく\ruby{寒}{さむ}さに\ruby{身}{み}を\ruby{起}{お}こし\\
        \ruby{胸}{むね}に\ruby{秘}{ひ}めたる\ruby{青写真}{あお|じゃ|しん}\\
        \ruby{広}{ひろ}がりし\ruby[g]{雪原}{の}に ひとりたち\\
        はるかなる\ruby[g]{白雲}{くも}の\ruby{頂}{さき}\\
        \ruby{旅}{たび}に\ruby{追}{お}ふ
        
    \end{minipage}
    \begin{minipage}[c]{\blocksize}
        
        \vspace{\linespace}
        \item~\\
        % 5.
        \ruby{今}{いま}\ruby{祭日}{まつり|び}の\ruby{猛}{たけ}き\ruby{火}{ひ}よ\\
        \ruby{寒風}{かん|ぷう}\ruby[g]{蒼碧}{そら}を\ruby{貫}{つらぬ}かん\\
        \ruby[g]{大地}{ち}を\ruby{揺}{ゆ}るがして\ruby{嵐}{かぜ}おこる\\
        \ruby{新風}{しん|ぷう}\ruby{破天}{は|てん}の\ruby{新時代}{しん|じ|だい}\\
        \ruby{広}{ひろ}がりし\ruby{蝦夷}{え|ぞ}に \ruby[g]{寮友}{とも}は\ruby{和}{わ}し\\
        はるかなる\ruby[g]{先代}{とも}の\ruby{魂}{たま}\\
        \ruby{解}{と}き\ruby{放}{はな}つ
    
    \end{minipage}
\end{enumerate} % 番号の箇条書き ここまで
%%%%% 歌詞 ここまで %%%%%
% end body

\end{document}

\documentclass[10pt,b5j]{tarticle} % B6 縦書き
% \documentclass[10pt,b5j]{tarticle} % B6 縦書き
\AtBeginDvi{\special{papersize=128mm,182mm}} % B6 用用紙サイズ
\usepackage{otf} % Unicode で字を入力するのに必要なパッケージ
\usepackage[size=b6j]{bxpapersize} % B6 用紙サイズを指定
\usepackage[dvipdfmx]{graphicx} % 画像を挿入するためのパッケージ
\usepackage[dvipdfmx]{color} % 色をつけるためのパッケージ
\usepackage{pxrubrica} % ルビを振るためのパッケージ
\usepackage{multicol} % 複数段組を作るためのパッケージ
\setlength{\topmargin}{14mm} % 上下方向のマージン
\addtolength{\topmargin}{-1in} % 
\setlength{\oddsidemargin}{11mm} % 左右方向のマージン
\addtolength{\oddsidemargin}{-1in} % 
\setlength{\textwidth}{154mm} % B6 用
\setlength{\textheight}{108mm} % B6 用
\setlength{\headsep}{0mm} % 
\setlength{\headheight}{0mm} % 
\setlength{\topskip}{0mm} % 
\setlength{\parskip}{0pt} % 
\def\labelenumi{\theenumi、} % 箇条書きのフォーマット
\parindent = 0pt % 段落下げしない

 % B6 用テンプレート読み込み

\begin{document}
% begin header
%%%%% タイトルと作者 ここから %%%%%
\begin{minipage}[c]{0.7\hsize} % タイトルは上から 7 割
    \begin{center}
    % begin title
        {\LARGE
            広がりし海原に % タイトルを入れる
        }
        {\small 
            (平成三十年度寮歌) % 年などを入れる
        }
    % end title
    \end{center}
\end{minipage}
\begin{minipage}[c]{0.3\hsize} % 作歌作曲は上から 3 割
    \begin{flushright} % 下寄せにする
        % begin name
        樋浦一希君 作歌・作曲 % 作歌・作曲者
        % end name
    \end{flushright}
\end{minipage}
%%%%% タイトルと作者 ここまで %%%%%
% (2,5 繰り返しなし)
% end header

% begin length
\vspace{1.5em} % タイトル, 作者と歌詞の間に隙間を設ける
\newcommand{\linespace}{0.5em} % 行間の設定
\newcommand{\blocksize}{0.5\hsize} % 段組間の設定
\newcommand{\itemmargin}{3em} % 曲番の位置調整の長さ
% end length
% begin body
%%%%% 歌詞 ここから %%%%%
\begin{enumerate} % 番号の箇条書き ここから
    \setlength{\itemindent}{\itemmargin} % 曲番の位置調整
    \begin{minipage}[c]{\blocksize}
    
        \vspace{\linespace}
        \item~\\
        % 1.
        \ruby{春}{はる}あけぼのの\ruby{夢}{ゆめ}に\ruby{見}{み}て\\
        カムイの\ruby{声}{こえ}に\ruby{導}{みちび}かれ\\
        \ruby{舟}{ふね}をこぎいで\ruby{流}{なが}れ\ruby{来}{こ}ぬ\\
        \ruby{北都}{ほくと}\ruby{夜明}{よあ}けの\ruby{金字塔}{きんじとう}\\
        \ruby{広}{ひろ}がりし\ruby{草原}{そうげん}に ひとりたち\\
        はるかなる\ruby{大雪}{おおゆき}の\ruby{山}{やま} のぞみみん
        
    \end{minipage}
    \begin{minipage}[c]{\blocksize}
        
        \vspace{\linespace}
        \item~\\
        % 2.
        \ruby{夏}{なつ}\ruby{宵闇}{よいやみ}の\ruby{緑風}{りょくふう}に\\
        \ruby{森}{もり}が\ruby{葉音}{はおと}を\ruby{雨}{あめ}ときき\\
        \ruby{楡}{にれ}の\ruby{木立}{こだち}をさまよえば\\
        \ruby{紅}{べに}はゆる\ruby{山小屋}{やまごや}ひとつ\\
        \ruby{広}{ひろ}がりし\ruby{高原}{こうげん}に ひとりたち\\
        はるかなる\ruby{天空}{てんくう}の\ruby{星}{ほし}を \ruby{身}{み}に\ruby{浴}{あ}びん
        
    \end{minipage}
    \begin{minipage}[c]{\blocksize}
        
        \vspace{\linespace}
        \item~\\
        % 3.
        \ruby{秋夕}{ちゅそく}\ruby{暮}{く}れの\ruby{鹿}{しか}の\ruby{声}{こえ}に\\
        \ruby{恵}{めぐ}みの\ruby{季節}{きせつ}は\ruby{過}{す}ぎゆきて\\
        \ruby{入日}{いりひ}の\ruby{茜}{あかね}に\ruby{涙}{なみだ}する\\
        \ruby{冬}{ふゆ}\ruby{音}{おと}せまりき\ruby{危機}{きき}\ruby{焦燥}{しょうそう}\\
        \ruby{広}{ひろ}がりし\ruby{牧野}{ぼくや}に ひとりたち\\
        はるかなるシベリアの\ruby{風}{かぜ} \ruby{気}{き}も\ruby{霧散}{むさん}す
        
    \end{minipage}
    \begin{minipage}[c]{\blocksize}
        
        \vspace{\linespace}
        \item~\\
        % 4.
        \ruby{冬}{ふゆ}つとめてのゆめうつつ\\
        かそかに\ruby{遠}{とお}く\ruby{銀}{ぎん}\ruby{狼}{おおかみ}の\ruby{咆哮}{ほうこう}\\
        \ruby{凍}{い}てつく\ruby{寒}{さむ}さに\ruby{身}{み}を\ruby{起}{お}こし\\
        \ruby{胸}{むね}に\ruby{秘}{ひ}めたる\ruby{青写真}{あおじゃしん}\\
        \ruby{広}{ひろ}がりし\ruby{雪原}{せつげん}に ひとりたち\\
        はるかなる\ruby{白雲}{しらくも}の\ruby{頂}{いただき} \ruby{旅}{たび}に\ruby{追}{つい}ふ
        
    \end{minipage}
    \begin{minipage}[c]{\blocksize}
        
        \vspace{\linespace}
        \item~\\
        % 5.
        \ruby{今}{こん}\ruby{祭日}{さいじつ}の\ruby{猛}{もう}き\ruby{火}{ひ}よ\\
        \ruby{寒風}{かんぷう}\ruby{蒼}{あお}\ruby{碧}{へき}を\ruby{貫}{つらぬ}かん\\
        \ruby{大地}{だいち}を\ruby{揺}{ゆ}るがして\ruby{嵐}{あらし}おこる\\
        \ruby{新風}{しんぷう}\ruby{破}{やぶ}\ruby{天}{てん}の\ruby{新時代}{しんじだい}\\
        \ruby{広}{ひろ}がりし\ruby{蝦夷}{えぞ}に \ruby{寮友}{とも}は\ruby{和}{わ}し\\
        はるかなる\ruby{先代}{せんだい}の\ruby{魂}{たましい} \ruby{解}{と}き\ruby{放}{}つ

    
    \end{minipage}
\end{enumerate} % 番号の箇条書き ここまで
%%%%% 歌詞 ここまで %%%%%
% end body

\end{document}

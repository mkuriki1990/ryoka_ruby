\documentclass[10pt,b5j]{tarticle} % B6 縦書き
% \documentclass[10pt,b5j]{tarticle} % B6 縦書き
\AtBeginDvi{\special{papersize=128mm,182mm}} % B6 用用紙サイズ
\usepackage{otf} % Unicode で字を入力するのに必要なパッケージ
\usepackage[size=b6j]{bxpapersize} % B6 用紙サイズを指定
\usepackage[dvipdfmx]{graphicx} % 画像を挿入するためのパッケージ
\usepackage[dvipdfmx]{color} % 色をつけるためのパッケージ
\usepackage{pxrubrica} % ルビを振るためのパッケージ
\usepackage{multicol} % 複数段組を作るためのパッケージ
\setlength{\topmargin}{14mm} % 上下方向のマージン
\addtolength{\topmargin}{-1in} % 
\setlength{\oddsidemargin}{11mm} % 左右方向のマージン
\addtolength{\oddsidemargin}{-1in} % 
\setlength{\textwidth}{154mm} % B6 用
\setlength{\textheight}{108mm} % B6 用
\setlength{\headsep}{0mm} % 
\setlength{\headheight}{0mm} % 
\setlength{\topskip}{0mm} % 
\setlength{\parskip}{0pt} % 
\def\labelenumi{\theenumi、} % 箇条書きのフォーマット
\parindent = 0pt % 段落下げしない

 % B6 用テンプレート読み込み

\begin{document}
% begin header
%%%%% タイトルと作者 ここから %%%%%
\begin{minipage}[c]{0.7\hsize} % タイトルは上から 7 割
    \begin{center}
    % begin title
        {\LARGE
            陽春新しき % タイトルを入れる
        }
        {\small 
            (昭和六十一年度寮歌) % 年などを入れる
        }
    % end title
    \end{center}
\end{minipage}
\begin{minipage}[c]{0.3\hsize} % 作歌作曲は上から 3 割
    \begin{flushright} % 下寄せにする
        % begin name
        原沢辰明君 作歌\\山森聡君 作曲 % 作歌・作曲者
        % end name
    \end{flushright}
\end{minipage}
%%%%% タイトルと作者 ここまで %%%%%
% (1,2,3,4 了なし繰り返しあり)
% end header

% begin length
\vspace{1.5em} % タイトル, 作者と歌詞の間に隙間を設ける
\newcommand{\linespace}{0.5em} % 行間の設定
\newcommand{\blocksize}{0.33\hsize} % 段組間の設定
\newcommand{\itemmargin}{3em} % 曲番の位置調整の長さ
% end length
% begin body
%%%%% 歌詞 ここから %%%%%
\begin{enumerate} % 番号の箇条書き ここから
    \setlength{\itemindent}{\itemmargin} % 曲番の位置調整
    \begin{minipage}[c]{\blocksize}
    
        \vspace{\linespace}
        \item~\\
        % 1.
        \ruby[g]{陽春}{はる}\ruby{新}{あたら}しき\ruby[g]{希望}{のぞみ}\ruby{満}{み}つ\\
        \ruby[g]{恵迪寮}{やどり}に\ruby{若}{わか}き\ruby[g]{男子}{おのこ}\ruby{等}{ら}が\\
        \ruby[g]{野心}{こころ}も\ruby{赤}{あか}き\ruby{夕}{ゆう}\ruby{手稲}{てい|ね}\\
        \ruby{鳴呼}{あ|あ}\ruby{力}{ちから}もて\ruby{進}{すす}まんか
        
        \vspace{\linespace}
        \item~\\
        % 2.
        \ruby[g]{盛夏}{なつ}\ruby{短}{みじ}かくてストームに\\
        \ruby{太鼓}{たい|こ}\ruby{音}{ね}\ruby{闇}{やみ}に\ruby{消}{き}えるかな\\
        \ruby{朝}{あさ}の\ruby[g]{日露}{つゆ}に\ruby[g]{寮歌}{うた}の\ruby{声}{こえ}\\
        \ruby{鳴呼}{あ|あ}\ruby{轟}{とどろ}くかこの\ruby[g]{石狩平野}{だいち}
        
        \vspace{\linespace}
        \item~\\
        % 3.
        \ruby{夕暮}{ゆう|ぐれ}\ruby{風}{かぜ}の\ruby{涼}{すず}しさに\\
        \ruby{楡}{にれ}の\ruby{悲}{かな}しみ\ruby{知}{し}れるかな\\
        \ruby{雁}{かり}より\ruby{暮}{く}れる\ruby{原始林}{げん|し|りん}\\
        \ruby{鳴呼}{あ|あ}\ruby{我}{わ}が\ruby{憂}{うれ}ひすずろかな
        
    \end{minipage}
    \begin{minipage}[c]{\blocksize}
        
        \vspace{\linespace}
        \item~\\
        % 4.
        \ruby{北溟}{ほく|めい}\ruby[g]{粉雪}{ゆき}に\ruby{荒}{すさ}ぶれど\\
        \ruby{詩}{うた}を\ruby[g]{忘却}{わす}れぬ\ruby[g]{若人}{わこうど}が\\
        \ruby[g]{理想}{ロマン}の\ruby[g]{存在}{ありか}\ruby{求}{もと}めつつ\\
        \ruby{鳴呼}{あ|あ}その\ruby[g]{自治寮}{とりで}\ruby[g]{創造}{きづ}くかな
        
        \vspace{\linespace}
        \item~\\
        % 5.
        \ruby{淡}{あわ}き\ruby[g]{憧憬}{おもい}に\ruby{焦}{こが}れ\ruby{来}{く}る\\
        \ruby{拙}{つたな}き\ruby{言葉}{こと|ば}\ruby{操}{あやつ}りて\\
        \ruby{胸}{こころ}の\ruby{内}{なか}を\ruby{打}{う}ち\ruby{明}{あ}けし\\
        \ruby{鳴呼}{あ|あ}この\ruby[g]{青春}{はる}も\ruby{早}{は}や\ruby{行}{ゆ}かん
        
        \vspace{\linespace}
        \item~\\
        % 6.
        \ruby{宴}{うたげ}の\ruby[g]{酔狂}{よい}も\ruby[g]{静寂}{しず}まりて\\
        \ruby[g]{沈黙}{しじま}の\ruby[g]{彼方}{かなた}\ruby{微}{かす}かなる\\
        \ruby{郭公}{かっ|こう}の\ruby[g]{啼声}{こえ}の\ruby{清}{きよ}らかさ\\
        \ruby{鳴呼}{あ|あ}この\ruby[g]{初夏}{なつ}も\ruby{過}{す}ぐるかな
        
    \end{minipage}
    \begin{minipage}[c]{\blocksize}
        
        \vspace{\linespace}
        \item~\\
        % 7.
        \ruby{北斗}{ほく|と}\ruby{煌}{きらめ}く\ruby{晩秋}{ばん|しゅう}\ruby{夜}{や}\\
        \ruby{望月}{もち|づき}\ruby{写}{うつ}す\ruby[g]{支笏湖}{うみ}の\ruby{波}{なみ}\\
        \ruby[g]{明日}{あす}の\ruby{旅路}{たび|じ}を\ruby{思}{おも}いつつ\\
        \ruby{鳴呼}{あ|あ}\ruby{涙}{なみだ}して\ruby{更}{ふ}くる\ruby{夜}{よる}
        
        \vspace{\linespace}
        \item~\\
        % 8.
        \ruby{疎々}{そ|そ}たる\ruby[g]{原始林}{もり}に\ruby{我}{われ}\ruby[g]{一人}{ひとり}\\
        \ruby[g]{白雪}{ゆき}\ruby{舞}{ま}う\ruby{木立}{こ|だち}\ruby[g]{烈風}{かぜ}\ruby{強}{つよ}く\\
        \ruby[g]{冷徹}{つめ}たき\ruby{真理}{しん|り}\ruby{索}{もと}めんと\\
        \ruby{鳴呼}{あ|あ}\ruby{声}{こえ}もなく\ruby{迪}{みち}を\ruby{行}{ゆ}く
        
        \vspace{\linespace}
        \item~\\
        % 9.
        \ruby{春}{はる}も\ruby{巡}{めぐ}れる\ruby{四}{よん}\ruby{度}{たび}に\\
        \ruby{若}{わか}き\ruby[g]{明日}{あした}の\ruby[g]{祝極}{よろこび}と\\
        \ruby{南風}{なん|ぷう}\ruby{頻}{しき}りに\ruby{頬}{ほお}を\ruby{打}{う}つ\\
        \ruby{鳴呼}{あ|あ}この\ruby[g]{別離}{わかれ}\ruby[g]{永却}{なが}からず
    
    \end{minipage}
\end{enumerate} % 番号の箇条書き ここまで
%%%%% 歌詞 ここまで %%%%%
% end body

\end{document}

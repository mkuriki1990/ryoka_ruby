\documentclass[10pt,b5j]{tarticle} % B6 縦書き
% \documentclass[10pt,b5j]{tarticle} % B6 縦書き
\AtBeginDvi{\special{papersize=128mm,182mm}} % B6 用用紙サイズ
\usepackage{otf} % Unicode で字を入力するのに必要なパッケージ
\usepackage[size=b6j]{bxpapersize} % B6 用紙サイズを指定
\usepackage[dvipdfmx]{graphicx} % 画像を挿入するためのパッケージ
\usepackage[dvipdfmx]{color} % 色をつけるためのパッケージ
\usepackage{pxrubrica} % ルビを振るためのパッケージ
\usepackage{multicol} % 複数段組を作るためのパッケージ
\setlength{\topmargin}{14mm} % 上下方向のマージン
\addtolength{\topmargin}{-1in} % 
\setlength{\oddsidemargin}{11mm} % 左右方向のマージン
\addtolength{\oddsidemargin}{-1in} % 
\setlength{\textwidth}{154mm} % B6 用
\setlength{\textheight}{108mm} % B6 用
\setlength{\headsep}{0mm} % 
\setlength{\headheight}{0mm} % 
\setlength{\topskip}{0mm} % 
\setlength{\parskip}{0pt} % 
\def\labelenumi{\theenumi、} % 箇条書きのフォーマット
\parindent = 0pt % 段落下げしない

 % B6 用テンプレート読み込み

\begin{document}
% begin header
%%%%% タイトルと作者 ここから %%%%%
\begin{minipage}[c]{0.7\hsize} % タイトルは上から 7 割
    \begin{center}
    % begin title
        {\LARGE
            茫洋の海 % タイトルを入れる
        }
        {\small 
            (昭和三十五年寮歌) % 年などを入れる
        }
    % end title
    \end{center}
\end{minipage}
\begin{minipage}[c]{0.3\hsize} % 作歌作曲は上から 3 割
    \begin{flushright} % 下寄せにする
        % begin name
        三浦清一郎君 作歌\\前野紀一君 作曲 % 作歌・作曲者
        % end name
    \end{flushright}
\end{minipage}
%%%%% タイトルと作者 ここまで %%%%%
% (1,2,3 了あり)
% end header

% begin length
\vspace{1.5em} % タイトル, 作者と歌詞の間に隙間を設ける
\newcommand{\linespace}{0.5em} % 行間の設定
\newcommand{\blocksize}{0.5\hsize} % 段組間の設定
\newcommand{\itemmargin}{3em} % 曲番の位置調整の長さ
% end length
% begin body
%%%%% 歌詞 ここから %%%%%
\begin{enumerate} % 番号の箇条書き ここから
    \setlength{\itemindent}{\itemmargin} % 曲番の位置調整
    \begin{minipage}[c]{\blocksize}
    
        \vspace{\linespace}
        \item~\\
        % 1.
        \ruby{茫洋}{ぼうよう}の\ruby{海}{うみ}に\ruby{憧}{あこが}れ\\
        \ruby{峻険}{しゅんけん}の\ruby{峰}{みね}を\ruby{慕}{した}いて\\
        \ruby{北国}{きたぐに}の\ruby{大地}{だいち}に\ruby{旅}{たび}\ruby{行}{い}けば\\
        \ruby{溢}{あふ}れ\ruby{満}{み}つ\ruby{夢}{ゆめ}\ruby{若}{わか}さ\\
        \ruby{果}{は}てしなく\ruby{広}{ひろ}ごれる\ruby{地平線}{ちへいせん}
        
    \end{minipage}
    \begin{minipage}[c]{\blocksize}
        
        \vspace{\linespace}
        \item~\\
        % 2.
        \ruby{曇}{くも}りなき\ruby{心}{こころ}\ruby{求}{もと}め\\
        \ruby{厳}{きび}しかる\ruby{努}{つと}めの\ruby{道}{みち}に\\
        \ruby{真}{しん}なる\ruby{美}{び}を\ruby{探}{さぐ}らんと\\
        \ruby{人}{ひと}の\ruby{世}{よ}の\ruby{旅}{たび}にして\\
        \ruby{結}{むす}ばれし\ruby{二年}{にねん,にねん}の\ruby{宿}{やど}なれや
        
    \end{minipage}
    \begin{minipage}[c]{\blocksize}
        
        \vspace{\linespace}
        \item~\\
        % 3.
        \ruby{移}{うつ}り\ruby{行}{い}く\ruby{時}{とき}にはあれど\\
        \ruby{涙}{なみだ}して\ruby{誓}{ちか}いし\ruby{言葉}{ことば}\\
        \ruby{尊}{とうと}しや\ruby{若}{わか}き\ruby{日}{ひ}の\ruby{夢}{ゆめ}\\
        \ruby{春秋}{しゅんじゅう}の\ruby{十年}{じゅうねん,じゅーねん}の\ruby{後}{のち}に\\
        \ruby{思}{おも}い\ruby{出}{で}\ruby{声}{ごえ}もなく\ruby{偲}{しの}ばんや
    
    \end{minipage}
\end{enumerate} % 番号の箇条書き ここまで
%%%%% 歌詞 ここまで %%%%%
% end body

\end{document}

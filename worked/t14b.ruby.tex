\documentclass[10pt,b5j]{tarticle} % B6 縦書き
% \documentclass[10pt,b5j]{tarticle} % B6 縦書き
\AtBeginDvi{\special{papersize=128mm,182mm}} % B6 用用紙サイズ
\usepackage{otf} % Unicode で字を入力するのに必要なパッケージ
\usepackage[size=b6j]{bxpapersize} % B6 用紙サイズを指定
\usepackage[dvipdfmx]{graphicx} % 画像を挿入するためのパッケージ
\usepackage[dvipdfmx]{color} % 色をつけるためのパッケージ
\usepackage{pxrubrica} % ルビを振るためのパッケージ
\usepackage{multicol} % 複数段組を作るためのパッケージ
\setlength{\topmargin}{14mm} % 上下方向のマージン
\addtolength{\topmargin}{-1in} % 
\setlength{\oddsidemargin}{11mm} % 左右方向のマージン
\addtolength{\oddsidemargin}{-1in} % 
\setlength{\textwidth}{154mm} % B6 用
\setlength{\textheight}{108mm} % B6 用
\setlength{\headsep}{0mm} % 
\setlength{\headheight}{0mm} % 
\setlength{\topskip}{0mm} % 
\setlength{\parskip}{0pt} % 
\def\labelenumi{\theenumi、} % 箇条書きのフォーマット
\parindent = 0pt % 段落下げしない

 % B6 用テンプレート読み込み

\begin{document}
% begin header
%%%%% タイトルと作者 ここから %%%%%
\begin{minipage}[c]{0.7\hsize} % タイトルは上から 7 割
    \begin{center}
    % begin title
        {\LARGE
            大地はなごやかに % タイトルを入れる
        }
        {\small 
            (大正十四年開舎二十周年記念寮歌) % 年などを入れる
        }
    % end title
    \end{center}
\end{minipage}
\begin{minipage}[c]{0.3\hsize} % 作歌作曲は上から 3 割
    \begin{flushright} % 下寄せにする
        % begin name
        黒沢徹君 作歌\\三溝清美君 作曲 % 作歌・作曲者
        % end name
    \end{flushright}
\end{minipage}
%%%%% タイトルと作者 ここまで %%%%%
% (1,2,3 了あり)
% end header

% begin length
\vspace{1.5em} % タイトル, 作者と歌詞の間に隙間を設ける
\newcommand{\linespace}{0.5em} % 行間の設定
\newcommand{\blocksize}{0.33\hsize} % 段組間の設定
\newcommand{\itemmargin}{3em} % 曲番の位置調整の長さ
% end length
% begin body
%%%%% 歌詞 ここから %%%%%
\begin{enumerate} % 番号の箇条書き ここから
    \setlength{\itemindent}{\itemmargin} % 曲番の位置調整
    \begin{minipage}[c]{\blocksize}
    
        \vspace{\linespace}
        \item~\\
        % 1.
        \ruby[g]{大地}{ち}はなごやかにうるほひて\\
        \ruby[g]{丘陵}{をか}の\ruby[g]{傾斜}{なぞへ}の\ruby{若草}{わか|くさ}や\\
        さゆらぐ\ruby{楡}{にれ}の\ruby{嫩葉}{わか|ば}にも\\
        \ruby{春}{はる}\ruby{新生}{しん|せい}の\ruby[g]{精気}{き}は\ruby[g]{{\CID{7635}}}{あふ}る\\ % CID 溢
        \ruby[g]{原始林}{もり}の\ruby{緑}{みどり}に\ruby{流}{なが}れ\ruby{来}{く}る\\
        \ruby{嗚呼}{あ|あ}\ruby{青春}{せい|しゅん}の\ruby{讃歌}{たたへ|うた}
        
        \vspace{\linespace}
        \item~\\
        % 2.
        \ruby{色}{いろ}\ruby{紫}{むらさき}の\ruby{彩絹}{あや|ぎぬ}に\\
        \ruby{染}{そ}めて\ruby{溶}{と}けたる\ruby{朝霧}{あさ|ぎり}の\\
        \ruby{悠久}{いう|きう}の\ruby[g]{蒼穹}{そら}はるかにも\\
        \ruby{濃}{こ}き\ruby{水色}{みづ|いろ}にうつろへば\\
        \ruby{白鳥}{はく|てう}\ruby{高}{たか}く\ruby{海}{うみ}に\ruby{飛}{と}び\\
        \ruby{入江}{いり|え}の\ruby{波}{なみ}に\ruby[g]{夏陽}{ひ}は\ruby{映}{は}ゆる
        
    \end{minipage}
    \begin{minipage}[c]{\blocksize}
        
        \vspace{\linespace}
        \item~\\
        % 3.
        \ruby[g]{連嶺}{やま}\ruby{紅}{くれなゐ}に\ruby[g]{黄昏}{たそが}れて\\
        \ruby{夕靄}{ゆふ|もや}\ruby{流}{なが}る\ruby{水沼}{すい|せふ}の\\
        \ruby{白}{しろ}き\ruby[g]{葦穂波}{ほなみ}に\ruby{顫}{ふる}ふ\ruby{月}{つき}\\
        \ruby{幽暗}{いう|あん}の\ruby[g]{草野}{の}に\ruby{訪}{おと}づれば\\
        \ruby{仄}{ほの}かに\ruby{響}{ひび}く\ruby{胸}{むね}うちの\\
        \ruby[g]{高遠}{たか}き\ruby[g]{感激}{おもひ}に\ruby[g]{逍遙}{さまよ}ふ\ruby{哉}{や}
        
        \vspace{\linespace}
        \item~\\
        % 4.
        \ruby[g]{神秘}{くしび}の\ruby[g]{森林}{もり}に\ruby[g]{群星}{ほし}さえて\\
        \ruby{雪}{ゆき}の\ruby[g]{曠野}{の}\ruby{遠}{とほ}く\ruby[g]{静謐}{しづか}なり\\
        \ruby{銀壺}{ぎん|こ}にゆるる\ruby{灯}{ともしび}に\\
        \ruby{崇}{たか}き\ruby[g]{教訓}{をしへ}を\ruby{胸}{むね}にして\\
        \ruby{心}{こころ}の\ruby[g]{憧憬郷}{さと}にまどゐする\\
        \ruby{若}{わか}き\ruby{人等}{ひと|ら}の\ruby{哀歓}{あい|くわん}よ
        
    \end{minipage}
    \begin{minipage}[c]{\blocksize}
        
        \vspace{\linespace}
        \item~\\
        % 5.
        \ruby{陽炎}{やう|えん}ゆらぐ\ruby{春}{はる}の\ruby{日}{ひ}に\\
        \ruby{落葉}{らく|えふ}しぐる\ruby{秋}{あき}の\ruby{夜}{よ}に\\
        \ruby{胸}{むね}に\ruby{高鳴}{たか|な}る\ruby{青春}{せい|しゅん}の\\
        \ruby{若}{わか}き\ruby{誇}{ほこ}りを\ruby{歌}{うた}ひつつ\\
        \ruby{限}{かぎ}れる\ruby{生}{せい}の\ruby[g]{瞬時}{ひととき}を\\
        \ruby{深}{ふか}き\ruby[g]{瞑想}{おもひ}に\ruby{過}{すご}さずや
    
    \end{minipage}
\end{enumerate} % 番号の箇条書き ここまで
%%%%% 歌詞 ここまで %%%%%
% end body

\end{document}

\documentclass[10pt,b5j]{tarticle} % B6 縦書き
% \documentclass[10pt,b5j]{tarticle} % B6 縦書き
\AtBeginDvi{\special{papersize=128mm,182mm}} % B6 用用紙サイズ
\usepackage{otf} % Unicode で字を入力するのに必要なパッケージ
\usepackage[size=b6j]{bxpapersize} % B6 用紙サイズを指定
\usepackage[dvipdfmx]{graphicx} % 画像を挿入するためのパッケージ
\usepackage[dvipdfmx]{color} % 色をつけるためのパッケージ
\usepackage{pxrubrica} % ルビを振るためのパッケージ
\usepackage{multicol} % 複数段組を作るためのパッケージ
\setlength{\topmargin}{14mm} % 上下方向のマージン
\addtolength{\topmargin}{-1in} % 
\setlength{\oddsidemargin}{11mm} % 左右方向のマージン
\addtolength{\oddsidemargin}{-1in} % 
\setlength{\textwidth}{154mm} % B6 用
\setlength{\textheight}{108mm} % B6 用
\setlength{\headsep}{0mm} % 
\setlength{\headheight}{0mm} % 
\setlength{\topskip}{0mm} % 
\setlength{\parskip}{0pt} % 
\def\labelenumi{\theenumi、} % 箇条書きのフォーマット
\parindent = 0pt % 段落下げしない

 % B6 用テンプレート読み込み

\begin{document}
% begin header
%%%%% タイトルと作者 ここから %%%%%
\begin{minipage}[c]{0.7\hsize} % タイトルは上から 7 割
    \begin{center}
    % begin title
        {\LARGE
            朝葉末の % タイトルを入れる
        }
        {\small 
            (第三期卒業生贈桜星会歌) % 年などを入れる
        }
    % end title
    \end{center}
\end{minipage}
\begin{minipage}[c]{0.3\hsize} % 作歌作曲は上から 3 割
    \begin{flushright} % 下寄せにする
        % begin name
        加藤義夫君 作歌\\角倉邦彦君 作曲 % 作歌・作曲者
        % end name
    \end{flushright}
\end{minipage}
%%%%% タイトルと作者 ここまで %%%%%
% % end header

% begin length
\vspace{1.5em} % タイトル, 作者と歌詞の間に隙間を設ける
\newcommand{\linespace}{0.5em} % 行間の設定
\newcommand{\blocksize}{0.33\hsize} % 段組間の設定
\newcommand{\itemmargin}{3em} % 曲番の位置調整の長さ
% end length
% begin body
%%%%% 歌詞 ここから %%%%%
\begin{enumerate} % 番号の箇条書き ここから
    \setlength{\itemindent}{\itemmargin} % 曲番の位置調整
    \begin{minipage}[c]{\blocksize}
    
        \vspace{\linespace}
        \item~\\
        % 1.
        \ruby{朝}{あした}\ruby{葉末}{は|ずゑ}の\ruby{露}{つゆ}を\ruby{受}{う}け\\
        \ruby{夕}{ゆふべ}\ruby{歸鳥}{き|てう}の\ruby{影}{かげ}\ruby{宿}{やど}し\\
        \ruby{曙}{あけぼの}\ruby{匂}{にほ}ふ\ruby{石狩}{いし|かり}に\\
        \ruby{玉}{たま}の\ruby{泉}{いずみ}と\ruby{湧}{わ}きしより\\
        \ruby{思}{おも}へば\ruby{茲}{ここ}に\ruby[g]{三歳}{みとせ}の\\
        \ruby{過}{す}ぎにし\ruby[g]{水路}{あと}を\ruby{偲}{しの}ぶ\ruby{哉}{かな}
        
        \vspace{\linespace}
        \item~\\
        % 2.
        \ruby{大気}{たい|き}は\ruby{凍}{こお}り\ruby{雪}{ゆき}もやの\\
        \ruby{荒}{あ}れし\ruby{廣野}{ひろ|の}の\ruby{面}{おも}をこむ\\
        \ruby{時}{とき}しも\ruby{高}{たか}く\ruby{天界}{てん|かい}に\\
        \ruby{光芒}{くわう|ばう}\ruby{強}{つよ}き\ruby{北極星}{ほく|きょく|せい}\\
        いさごと\ruby{光}{ひか}る\ruby{星}{ほし}くづは\\
        \ruby{我}{われ}をばめぐり\ruby{走}{はし}るなり
        
    \end{minipage}
    \begin{minipage}[c]{\blocksize}
        
        \vspace{\linespace}
        \item~\\
        % 3.
            かつらの\ruby{若芽}{わか|が}\ruby{色}{いろ}も\ruby{濃}{こ}く\\
        \ruby{森}{もり}に\ruby{生氣}{せい|き}の\ruby{溢}{あふ}る\ruby{時}{とき}\\
        \ruby{奇}{く}しき\ruby{天地}{てん|ち}の\ruby{靈}{たま}\ruby{受}{う}けて\\
        \ruby[g]{大和}{やまと}\ruby{心}{ごころ}と\ruby{咲}{さ}き\ruby{出}{い}でし\\
        \ruby{蝦夷}{え|ぞ}の\ruby{深山}{み|やま}の\ruby{山櫻}{やま|ざくら}\\
        \ruby{我等}{われ|ら}が\ruby[g]{理想}{ロマン}\ruby[g]{此處}{ここ}にあり
        
        \vspace{\linespace}
        \item~\\
        % 4.
        \ruby{雲}{くも}\ruby{漠々}{ばく|ばく}に\ruby{水}{みず}ゆるぎ\\
        \ruby{大野}{おほ|の}の\ruby{心}{こころ}\ruby{我}{われ}にあり\\
        \ruby{眞理}{しん|り}\ruby{求}{もと}めて\ruby{息}{やす}まざる\\
        \ruby{久遠}{く|をん}の\ruby{望}{のぞみ}\ruby{我}{われ}にあり\\
        \ruby{衆愚}{しゅう|ぐ}の\ruby{聲}{こゑ}にまどはざる\\
        \ruby{我}{われ}に\ruby{男}{を}の\ruby{子}{こ}の\ruby{覺悟}{かく|ご}あり
        
    \end{minipage}
    \begin{minipage}[c]{\blocksize}
        
        \vspace{\linespace}
        \item~\\
        % 5.
        \ruby{消}{き}ゆる\ruby{榮華}{えい|が}を\ruby{夢}{ゆめ}に\ruby{見}{み}て\\
        \ruby{虚}{むな}しき\ruby{名}{な}をば\ruby{人}{ひと}よ\ruby{追}{お}へ\\
        \ruby{北}{きた}の\ruby{荒野}{あら|の}に\ruby{三百}{さん|びゃく}の\\
        \ruby{健兒}{けん|じ}\ruby{浮雲}{ふ|うん}を\ruby{嘲}{あざけ}りつ\\
        \ruby[g]{永遠}{とは}に\ruby{變}{かは}らぬ\ruby[g]{美土}{うまつち}に\\
        \ruby{注}{つ}ぎし\ruby{汗}{あせ}の\ruby{寶}{たから}を\ruby{求}{もと}む
        
        \vspace{\linespace}
        \item~\\
        % 6.
        \ruby{黄花}{おう|か}の\ruby{牧}{まき}に\ruby{新緑}{しん|りょく}の\\
        \ruby{森}{もり}に\ruby{鍛}{きた}へよ\ruby{鐵}{てつ}の\ruby{腕}{うで}\\
        \ruby[g]{紅葉}{もみぢ}\ruby{彩}{あや}どる\ruby{野}{の}に\ruby{山}{やま}に\\
        \ruby[g]{吹雪}{ふぶき}の\ruby{里}{さと}に\ruby[g]{思想}{おもひ}\ruby{錬}{ね}れ\\
        \ruby{勉}{つと}めよ\ruby{奮}{ふる}へ\ruby{我}{わが}\ruby{友}{とも}よ\\
        やがてぞ\ruby{起}{た}たん\ruby{時}{とき}は\ruby{來}{こ}ん
    
    \end{minipage}
\end{enumerate} % 番号の箇条書き ここまで
%%%%% 歌詞 ここまで %%%%%
% end body

\end{document}

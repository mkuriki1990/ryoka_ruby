\documentclass[10pt,b5j]{tarticle} % B6 縦書き
% \documentclass[10pt,b5j]{tarticle} % B6 縦書き
\AtBeginDvi{\special{papersize=128mm,182mm}} % B6 用用紙サイズ
\usepackage{otf} % Unicode で字を入力するのに必要なパッケージ
\usepackage[size=b6j]{bxpapersize} % B6 用紙サイズを指定
\usepackage[dvipdfmx]{graphicx} % 画像を挿入するためのパッケージ
\usepackage[dvipdfmx]{color} % 色をつけるためのパッケージ
\usepackage{pxrubrica} % ルビを振るためのパッケージ
\usepackage{multicol} % 複数段組を作るためのパッケージ
\setlength{\topmargin}{14mm} % 上下方向のマージン
\addtolength{\topmargin}{-1in} % 
\setlength{\oddsidemargin}{11mm} % 左右方向のマージン
\addtolength{\oddsidemargin}{-1in} % 
\setlength{\textwidth}{154mm} % B6 用
\setlength{\textheight}{108mm} % B6 用
\setlength{\headsep}{0mm} % 
\setlength{\headheight}{0mm} % 
\setlength{\topskip}{0mm} % 
\setlength{\parskip}{0pt} % 
\def\labelenumi{\theenumi、} % 箇条書きのフォーマット
\parindent = 0pt % 段落下げしない

 % B6 用テンプレート読み込み

\begin{document}
% begin header
%%%%% タイトルと作者 ここから %%%%%
\begin{minipage}[c]{0.7\hsize} % タイトルは上から 7 割
    \begin{center}
    % begin title
        {\LARGE
            天地の奥に % タイトルを入れる
        }
        {\small 
            (昭和十八年寮歌) % 年などを入れる
        }
    % end title
    \end{center}
\end{minipage}
\begin{minipage}[c]{0.3\hsize} % 作歌作曲は上から 3 割
    \begin{flushright} % 下寄せにする
        % begin name
        橋爪秀雄君 作歌\\池田政晴君 作曲 % 作歌・作曲者
        % end name
    \end{flushright}
\end{minipage}
%%%%% タイトルと作者 ここまで %%%%%
% (1,2,6 了あり)
% end header

% begin length
\vspace{1.5em} % タイトル, 作者と歌詞の間に隙間を設ける
\newcommand{\linespace}{0.5em} % 行間の設定
\newcommand{\blocksize}{0.33\hsize} % 段組間の設定
\newcommand{\itemmargin}{3em} % 曲番の位置調整の長さ
% end length
% begin body
%%%%% 歌詞 ここから %%%%%
\begin{enumerate} % 番号の箇条書き ここから
    \setlength{\itemindent}{\itemmargin} % 曲番の位置調整
    \begin{minipage}[c]{\blocksize}
    
        \vspace{\linespace}
        \item~\\
        % 1.
        \ruby{天地}{てん|ち}の\ruby{奥}{おく}に\ruby{征}{ゆ}く\ruby{吾}{われ}や\\
        \ruby{弧杖}{こ|じょう}\ruby{無限}{む|げん}に\ruby{旅立}{たび|だ}ちて\\
        \ruby{渓巒}{けい|らん}はるか\ruby{訪}{たず}ね\ruby{来}{こ}し\\
        \ruby{楡陵}{ゆ|りょう}の\ruby{宿}{やど}や\ruby{三春}{さん|しゅん}の\\
        \ruby{旅}{たび}にしあれどそは\ruby{深}{ふか}き\\
        \ruby{噫}{ああ}\ruby{魂}{たましい}のふるさとか
        
        \vspace{\linespace}
        \item~\\
        % 2.
        \ruby{四大}{し|だい}も\ruby{夢}{ゆめ}む\ruby{幌}{ほろ}のさと\\
        \ruby{歌}{うた}の\ruby{心}{こころ}を\ruby{温}{たづ}ぬれば\\
        \ruby{馥}{かを}り\ruby{床}{ゆか}しきアカシヤの\\
        \ruby{花}{はな}\ruby{仄白}{ほの|じろ}き\ruby{憂}{うれひ}あり\\
        \ruby[g]{夏宵󠄁}{かしょう}の\ruby{霞}{かすみ}\ruby{靉}{たな}びきて\\
        \ruby{月}{つき}\ruby{皎々}{かう|かう}の\ruby[g]{滄海}{うみ}をゆく
        
    \end{minipage}
    \begin{minipage}[c]{\blocksize}
        
        \vspace{\linespace}
        \item~\\
        % 3.
        \ruby{大空}{たい|くう}\ruby{風}{かぜ}に\ruby{咽}{むせ}ぶよひ\\
        \ruby{暮鐘}{ぼ|しょう}は\ruby{低}{ひく}く\ruby{漂}{ただよ}ひて\\
        \ruby[g]{荒野}{の}は\ruby{凋落}{ちょう|らく}の\ruby{悲歌}{ひ|か}に\ruby{泣}{な}く\\
        \ruby{栄枯}{えい|こ}は\ruby{移}{うつ}る\ruby{秋}{あき}の\ruby{日}{ひ}の\\
        \ruby{秋思}{しゅう|し}の\ruby{歩}{あゆ}み\ruby{運}{はこ}ぶ\ruby{夜半}{よ|は}\\
        \ruby{久遠}{く|をん}の\ruby{星}{ほし}を\ruby{仰}{あお}がずや
        
        \vspace{\linespace}
        \item~\\
        % 4.
        \ruby{高}{たか}き\ruby[g]{理想}{こころ}は\ruby{人}{ひと}の\ruby{世}{よ}を\\
        \ruby{人}{ひと}の\ruby{世}{よ}と\ruby{生}{い}く\ruby{佗}{わび}しさに\\
        \ruby{坤球}{こん|きゅう}\ruby{鳴}{な}りて\ruby[g]{吹雪}{ふ}き\ruby{狂}{くる}ふ\\
        \ruby{孤高}{こ|こう}の\ruby{峯}{みね}に\ruby{伏}{ふ}する\ruby{今}{いま}\\
        \ruby{浮生}{ふ|せい}の\ruby{夢}{ゆめ}は\ruby{消}{き}え\ruby{果}{は}てて\\
        \ruby{心}{こころ}\ruby{虚}{むな}しき\ruby[g]{歓喜}{よろこび}よ
        
    \end{minipage}
    \begin{minipage}[c]{\blocksize}
        
        \vspace{\linespace}
        \item~\\
        % 5.
        \ruby{北溟}{ほく|めい}\ruby{春}{はる}は\ruby{浅}{あさ}けれど\\
        \ruby{森}{もり}かげ\ruby{清}{きよ}く\ruby{黄花}{き|ばな}\ruby{咲}{さ}き\\
        \ruby[g]{雲雀}{ひばり}は\ruby{高}{たか}く\ruby{空}{そら}に\ruby{入}{い}り\\
        \ruby{新生}{しん|せい}の\ruby[g]{合唱}{うた}\ruby{野}{の}に\ruby{満}{み}てり\\
        \ruby{古}{こ}\ruby{衣}{ぎぬ}を\ruby{重}{かさ}ぬる\ruby{日}{ひ}は\ruby{逝}{ゆ}いて\\
        \ruby{時}{とき}\ruby{乾坤}{けん|こん}に\ruby{春}{はる}よ\ruby{立}{た}つ
        
        \vspace{\linespace}
        \item~\\
        % 6.
        いざ\ruby[g]{浩歌}{うた}はなん\ruby{天壤}{あめ|つち}の\\
        \ruby{栄}{さか}ゆる\ruby{時}{とき}ぞ\ruby[g]{益荒男}{ますらお}の\\
        \ruby{事}{つか}ふる\ruby{道}{みち}は\ruby{烈}{きび}しかる\\
        \ruby[g]{今宵󠄁}{こよい}\ruby{祭}{まつり}の\ruby{聖}{きよ}き\ruby{火}{ひ}に\\
        \ruby{尊}{とうと}き\ruby{誓}{ちか}ひ\ruby{立}{た}てよかし\\
        \ruby{興亡}{こう|ぼう}\ruby{分}{わか}るる\ruby{秋}{とき}なれば
    
    \end{minipage}
\end{enumerate} % 番号の箇条書き ここまで
%%%%% 歌詞 ここまで %%%%%
% end body

\end{document}

\documentclass[10pt,b5j]{tarticle} % B6 縦書き
% \documentclass[10pt,b5j]{tarticle} % B6 縦書き
\AtBeginDvi{\special{papersize=128mm,182mm}} % B6 用用紙サイズ
\usepackage{otf} % Unicode で字を入力するのに必要なパッケージ
\usepackage[size=b6j]{bxpapersize} % B6 用紙サイズを指定
\usepackage[dvipdfmx]{graphicx} % 画像を挿入するためのパッケージ
\usepackage[dvipdfmx]{color} % 色をつけるためのパッケージ
\usepackage{pxrubrica} % ルビを振るためのパッケージ
\usepackage{plext} % 漢数字の enumerate を使うためのパッケージ
\usepackage{multicol} % 複数段組を作るためのパッケージ
\setlength{\topmargin}{14mm} % 上下方向のマージン
\addtolength{\topmargin}{-1in} % 
\setlength{\oddsidemargin}{11mm} % 左右方向のマージン
\addtolength{\oddsidemargin}{-1in} % 
\setlength{\textwidth}{154mm} % B6 用
\setlength{\textheight}{108mm} % B6 用
\setlength{\headsep}{0mm} % 
\setlength{\headheight}{0mm} % 
\setlength{\topskip}{0mm} % 
\setlength{\parskip}{0pt} % 
\def\theenumi{\Kanji{enumi}} % 箇条書きのフォーマットを漢数字に変更
\parindent = 0pt % 段落下げしない
\pagestyle{empty} % すべてのページ番号を消去
% \renewcommand{\baselinestretch}{0.9} % 行間の倍率
 % B6 用テンプレート読み込み

\begin{document}
% begin header
%%%%% タイトルと作者 ここから %%%%%
\begin{minipage}[c]{0.7\hsize} % タイトルは上から 7 割
    \begin{center}
    % begin title
        {\LARGE
            希望の光 % タイトルを入れる
        }
        {\small 
            (明治四十二年寮歌) % 年などを入れる
        }
    % end title
    \end{center}
\end{minipage}
\begin{minipage}[c]{0.3\hsize} % 作歌作曲は上から 3 割
    \begin{flushright} % 下寄せにする
        % begin name
        加藤茂雄君 作歌\\金原善知君 作曲 % 作歌・作曲者
        % end name
    \end{flushright}
\end{minipage}
%%%%% タイトルと作者 ここまで %%%%%
% (1,6 了あり)
% end header

% begin length
\vspace{1.5em} % タイトル, 作者と歌詞の間に隙間を設ける
\newcommand{\linespace}{0.5em} % 行間の設定
\newcommand{\blocksize}{0.33\hsize} % 段組間の設定
\newcommand{\itemmargin}{3em} % 曲番の位置調整の長さ
% end length
% begin body
%%%%% 歌詞 ここから %%%%%
\begin{enumerate} % 番号の箇条書き ここから
    \setlength{\itemindent}{\itemmargin} % 曲番の位置調整
    \begin{minipage}[c]{\blocksize}
    
        \vspace{\linespace}
        \item~\\
        % 1.
        \ruby{希望}{き|ぼう}の\ruby{光}{ひかり}\ruby{仰}{あお}ぎつつ\\
        \ruby{思}{おも}へば\ruby{友}{とも}と\ruby{尋}{たず}ね\ruby{来}{こ}し\\
        \ruby{山}{やま}は\ruby{紅}{くれない}\ruby{朝日子}{あさ|ひ|こ}の\\
        \ruby{燃}{も}ゆる\ruby{姿}{すがた}に\ruby{似}{に}たる\ruby{哉}{かな}\\
        \ruby{嘶}{いなな}く\ruby{駒}{こま}は\ruby{秋}{あき}に\ruby{肥}{こ}え\\
        \ruby{我等}{われ|ら}が\ruby{門出}{かど|で}\ruby{栄}{はえ}ありき
        
        \vspace{\linespace}
        \item~\\
        % 2.
        ああ\ruby{冬}{ふゆ}\ruby{寒}{さむ}し\ruby{北国}{きた|ぐに}の\\
        \ruby{大野}{おほ|の}の\ruby{果}{はて}を\ruby{眺}{なが}むれば\\
        \ruby{雪}{ゆき}かあられか\ruby{空}{そら}たえて\\
        \ruby{限}{かぎ}りは\ruby{知}{し}らず\ruby{暮}{くる}るとも\\
        \ruby{我等}{われ|ら}が\ruby{胸}{むね}に\ruby[g]{黙想}{おもひ}あり\\
        \ruby{星}{ほし}の\ruby{光}{ひかり}に\ruby[g]{啓示}{しめし}あり
        
    \end{minipage}
    \begin{minipage}[c]{\blocksize}
        
        \vspace{\linespace}
        \item~\\
        % 3.
        \ruby[g]{黙想}{おもひ}を\ruby{胸}{むね}に\ruby{結}{むす}ぶ\ruby{時}{とき}\\
        \ruby[g]{啓示}{しめし}を\ruby{空}{そら}に\ruby{望}{のぞ}む\ruby{時}{とき}\\
        \ruby{見}{み}よ\ruby{下}{した}\ruby{萠}{も}ゆる\ruby{若草}{わか|くさ}の\\
        \ruby[g]{息吹}{いぶき}さやかに\ruby{風}{かぜ}\ruby{薫}{かを}る\\
        \ruby{春}{はる}は\ruby{来}{きた}れり\ruby{春}{はる}は\ruby{来}{こ}ぬ\\
        \ruby{物}{もの}\ruby{皆}{みな}\ruby[g]{此処}{ここ}に\ruby{力}{ちから}あり
        
        \vspace{\linespace}
        \item~\\
        % 4.
        \ruby{春}{はる}の\ruby{光}{ひかり}の\ruby{照}{て}る\ruby{所}{ところ}\\
        \ruby{色}{いろ}を\ruby{交}{まじ}へて\ruby{咲}{さ}く\ruby{花}{はな}に\\
        \ruby{蝶}{てふ}\ruby{舞}{ま}ひ\ruby{鳥}{とり}は\ruby{囀}{さえず}りて\\
        \ruby{我等}{われ|ら}が\ruby{血潮}{ち|しほ}\ruby{躍}{おど}るなり\\
        \ruby{斯}{か}くて\ruby{見渡}{み|わた}す\ruby{行手}{ゆく|て}には\\
        \ruby{光}{ひかり}\ruby{蔽}{おほ}はん\ruby{影}{かげ}もなし
        
    \end{minipage}
    \begin{minipage}[c]{\blocksize}
        
        \vspace{\linespace}
        \item~\\
        % 5.
        \ruby{深}{ふか}く\ruby{霞}{かすみ}に\ruby{鎖}{とざ}されて\\
        \ruby{都}{みやこ}の\ruby{様}{さま}は\ruby{知}{し}らねども\\
        \ruby{夕}{ゆふべ}\ruby{孤雁}{こ|がん}の\ruby{声}{こえ}\ruby{聞}{き}けば\\
        \ruby{人}{ひと}\ruby{太平}{たい|へい}に\ruby{眠}{ねむ}るとや\\
        \ruby[g]{吹雪}{ふぶき}に\ruby{練}{ね}りし\ruby{双}{さう}の\ruby{腕}{うで}\\
        \ruby{鳴}{な}るよ\ruby[g]{常盤}{ときは}の\ruby{夢}{ゆめ}\ruby{醒}{さ}ませ
        
        \vspace{\linespace}
        \item~\\
        % 6.
        \ruby[g]{四年}{よとせ}の\ruby{昔}{むかし}\ruby{人々}{ひと|びと}の\\
        \ruby{耘}{くさぎ}り\ruby{建}{た}てし\ruby{我}{わ}が\ruby{寮}{れう}に\\
        \ruby{春}{はる}\ruby{立}{た}ち\ruby{還}{かへ}る\ruby{時}{とき}よ\ruby{今}{いま}\\
        \ruby{希望}{き|ぼう}の\ruby{光}{ひかり}\ruby{新}{あらた}なり\\
        さらば\ruby{起}{た}て\ruby{友}{とも}\ruby{諸共}{もろ|とも}に\\
        \ruby{我等}{われ|ら}\ruby{起}{た}つべき\ruby{時}{とき}なれば
    
    \end{minipage}
\end{enumerate} % 番号の箇条書き ここまで
%%%%% 歌詞 ここまで %%%%%
% end body

\end{document}

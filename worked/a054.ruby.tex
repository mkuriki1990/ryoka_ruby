\documentclass[10pt,b5j]{tarticle} % B6 縦書き
% \documentclass[10pt,b5j]{tarticle} % B6 縦書き
\AtBeginDvi{\special{papersize=128mm,182mm}} % B6 用用紙サイズ
\usepackage{otf} % Unicode で字を入力するのに必要なパッケージ
\usepackage[size=b6j]{bxpapersize} % B6 用紙サイズを指定
\usepackage[dvipdfmx]{graphicx} % 画像を挿入するためのパッケージ
\usepackage[dvipdfmx]{color} % 色をつけるためのパッケージ
\usepackage{pxrubrica} % ルビを振るためのパッケージ
\usepackage{multicol} % 複数段組を作るためのパッケージ
\setlength{\topmargin}{14mm} % 上下方向のマージン
\addtolength{\topmargin}{-1in} % 
\setlength{\oddsidemargin}{11mm} % 左右方向のマージン
\addtolength{\oddsidemargin}{-1in} % 
\setlength{\textwidth}{154mm} % B6 用
\setlength{\textheight}{108mm} % B6 用
\setlength{\headsep}{0mm} % 
\setlength{\headheight}{0mm} % 
\setlength{\topskip}{0mm} % 
\setlength{\parskip}{0pt} % 
\def\labelenumi{\theenumi、} % 箇条書きのフォーマット
\parindent = 0pt % 段落下げしない

 % B6 用テンプレート読み込み

\begin{document}
% begin header
%%%%% タイトルと作者 ここから %%%%%
\begin{minipage}[c]{0.7\hsize} % タイトルは上から 7 割
    \begin{center}
    % begin title
        {\LARGE
            水産放浪歌 % タイトルを入れる
        }
        {\small 
             % 年などを入れる
        }
    % end title
    \end{center}
\end{minipage}
\begin{minipage}[c]{0.3\hsize} % 作歌作曲は上から 3 割
    \begin{flushright} % 下寄せにする
        % begin name
         % 作歌・作曲者
        % end name
    \end{flushright}
\end{minipage}
%%%%% タイトルと作者 ここまで %%%%%
% (1,2,3 了あり)
% end header

% begin length
\vspace{1.0em} % タイトル, 作者と歌詞の間に隙間を設ける
\newcommand{\linespace}{0.5em} % 行間の設定
\newcommand{\blocksize}{0.33\hsize} % 段組間の設定
\newcommand{\itemmargin}{3em} % 曲番の位置調整の長さ
% end length
% begin body
%%%%% 歌詞 ここから %%%%%

\ruby{富貴名門}{ふう|き|めい|もん}の\ruby{女性}{じょ|せい}に\ruby{恋}{こい}するを
\ruby{純情}{じゅん|じょう}の\ruby{恋}{こい}と\ruby{誰}{だれ}が\ruby{言}{い}うぞ。\\
\ruby{暗鬼紅灯}{あん|き|こう|とう}の\ruby{巷}{ちまた}に\ruby[g]{彷徨}{さまよ}う\ruby{女性}{じょ|せい}に
\ruby{恋}{こい}するを\ruby{不情}{ふ|じょう}の\ruby{恋}{こい}と\ruby{誰}{だれ}が\ruby{言}{い}うぞ。\\
\ruby{雨降}{あめ|ふ}らば\ruby{雨降}{あめ|ふ}るもよし 
\ruby{風吹}{かぜ|ふ}かば\ruby{風吹}{かぜ|ふ}くもよし\\
\ruby[g]{月下}{げっか}の\ruby{酒場}{さか|ば}にて\ruby{媚}{こび}を\ruby{売}{う}る\ruby{女性}{じょ|せい}にも
\ruby{純情可憐}{じゅん|じょう|か|れん}なる\ruby{者}{もの}あれ。\\
\ruby{女}{おんな}の\ruby{膝枕}{ひざ|まくら}にて\ruby{一夜}{いち|や}の\ruby{快楽}{け|らく}を
\ruby{共}{とも}に\ruby{過}{すご}さずんば
\ruby{人生夢}{じん|せい|ゆめ}もなければ\ruby{恋}{こい}もなし。\\
\ruby{響}{とどろ}く\ruby{雷鳴}{らい|めい} \ruby{握}{にぎ}る\ruby{舵輪}{だ|りん} 
\ruby{睨}{にら}むコンパス\ruby{六分儀}{りく|ぶん|ぎ}\\
\ruby{吾}{われ}ら\ruby{海行}{うみ|ゆ}く\ruby{鴎鳥}{かもめ|どり} さらば\ruby{歌}{うた}わん\ruby{哉}{かな}\\
\ruby{吾}{われ}らが\ruby{水産放浪歌}{すい|さん|ほう|ろう|か}\\

\begin{enumerate} % 番号の箇条書き ここから
    \setlength{\itemindent}{\itemmargin}
    \begin{minipage}[c]{\blocksize}
    
        \vspace{\linespace}
        \item~\\
        % 1.
        \ruby{心猛}{こころ|たけ}くも\ruby{鬼神}{おに|がみ}ならず\\
        \ruby{男}{おとこ}と\ruby{生}{うま}れて\ruby{情}{なさけ}はあれど\\
        \ruby{母}{はは}を\ruby{見捨}{み|す}てて\ruby{浪越}{なみ|こ}えてゆく\\
        \ruby{友}{とも}よ\ruby{兄等}{けい|ら}よ\ruby[g]{何時}{いつ}また\ruby{会}{あ}わん

	\end{minipage}
	\begin{minipage}[c]{\blocksize}

        \vspace{\linespace}
        \item~\\
        % 2.
        \ruby{朝日夕日}{あさ|ひ|ゆう|ひ}をデッキに\ruby{浴}{あ}びて\\
        \ruby{続}{つづ}く\ruby{海原一筋道}{うな|ばら|ひと|すじ|みち}を\\
        \ruby[g]{大和}{やまと}\ruby[g]{男子}{おのこ}が\ruby{心}{こころ}に\ruby{秘}{ひ}めて\\
        \ruby{行}{ゆ}くや\ruby{万里}{ばん|り}の\ruby{荒波越}{あら|なみ|こ}えて
        
	\end{minipage}
	\begin{minipage}[c]{\blocksize}

        \vspace{\linespace}
        \item~\\
        % 3.
        \ruby{波}{なみ}の\ruby[g]{彼方}{かなた}の\ruby{南氷洋}{なん|ぴょう|よう}は\\
        \ruby{男}{おとこ}\ruby{多恨}{た|こん}の\ruby{身}{み}の\ruby{捨}{す}てどころ\\
        \ruby{胸}{むね}に\ruby{秘}{ひ}めたる\ruby{大願}{たい|がん}あれど\\
        \ruby{行}{ゆ}きて\ruby{帰}{かえ}らじ\ruby{望}{のぞ}みは\ruby{待}{も}たじ
    
    \end{minipage}
\end{enumerate} % 番号の箇条書き ここまで
\begin{flushright}
        注 成立事情不明なるも蒙古放浪歌\\
        (仲田三孝作詞、川上義彦作曲)の\\
        換え歌と推定される。
\end{flushright}
%%%%% 歌詞 ここまで %%%%%
% end body

\end{document}

\documentclass[10pt,b5j]{tarticle} % B6 縦書き
% \documentclass[10pt,b5j]{tarticle} % B6 縦書き
\AtBeginDvi{\special{papersize=128mm,182mm}} % B6 用用紙サイズ
\usepackage{otf} % Unicode で字を入力するのに必要なパッケージ
\usepackage[size=b6j]{bxpapersize} % B6 用紙サイズを指定
\usepackage[dvipdfmx]{graphicx} % 画像を挿入するためのパッケージ
\usepackage[dvipdfmx]{color} % 色をつけるためのパッケージ
\usepackage{pxrubrica} % ルビを振るためのパッケージ
\usepackage{multicol} % 複数段組を作るためのパッケージ
\setlength{\topmargin}{14mm} % 上下方向のマージン
\addtolength{\topmargin}{-1in} % 
\setlength{\oddsidemargin}{11mm} % 左右方向のマージン
\addtolength{\oddsidemargin}{-1in} % 
\setlength{\textwidth}{154mm} % B6 用
\setlength{\textheight}{108mm} % B6 用
\setlength{\headsep}{0mm} % 
\setlength{\headheight}{0mm} % 
\setlength{\topskip}{0mm} % 
\setlength{\parskip}{0pt} % 
\def\labelenumi{\theenumi、} % 箇条書きのフォーマット
\parindent = 0pt % 段落下げしない

 % B6 用テンプレート読み込み

\renewcommand{\baselinestretch}{0.90} % 行間の倍率

\begin{document}
% begin header
%%%%% タイトルと作者 ここから %%%%%
\begin{minipage}[c]{0.7\hsize} % タイトルは上から 7 割
    \begin{center}
    % begin title
        {\LARGE
            ヨットマンの歌 % タイトルを入れる
        }
        {\small 
             % 年などを入れる
        }
    % end title
    \end{center}
\end{minipage}
\begin{minipage}[c]{0.3\hsize} % 作歌作曲は上から 3 割
    \begin{flushright} % 下寄せにする
        % begin name
         % 作歌・作曲者
        % end name
    \end{flushright}
\end{minipage}
%%%%% タイトルと作者 ここまで %%%%%
% % end header

% begin length
\vspace{1.0em} % タイトル, 作者と歌詞の間に隙間を設ける
\newcommand{\linespace}{0.4em} % 行間の設定
\newcommand{\blocksize}{0.33\hsize} % 段組間の設定
\newcommand{\itemmargin}{3em} % 曲番の位置調整の長さ
% end length
% begin body
%%%%% 歌詞 ここから %%%%%
\begin{enumerate} % 番号の箇条書き ここから
    \setlength{\itemindent}{\itemmargin} % 曲番の位置調整
    \begin{minipage}[c]{\blocksize}
    
        \vspace{\linespace}
        \item~\\
        % 1.
        \ruby{腰}{こし}のシーナイフにすがりつき\\
        ついて\ruby{行}{い}きます\ruby[g]{何処}{どこ}までも\\
        ついて\ruby{行}{い}くのは\ruby{易}{やす}けれど\\
        \ruby{女}{おんな}\ruby{乗}{の}せない\ruby{三号艇}{さん|ごう|てい}
        
        \vspace{\linespace}
        \item~\\
        % 2.
        \ruby{女}{おんな}\ruby{乗}{の}せない\ruby{三号艇}{さん|ごう|てい}なら\\
        \ruby{長}{なが}い\ruby{黒髪}{くろ|かみ}\ruby{断}{た}ち\ruby{切}{き}って\\
        \ruby[g]{可愛}{かわ}いいクルーになりすまし\\
        ついて\ruby{行}{い}きます\ruby[g]{何処}{どこ}までも
        
        \vspace{\linespace}
        \item~\\
        % 3.
        \ruby{四年}{よ|ねん}\ruby{三年}{さん|ねん}ジジクサイ\\
        \ruby{二年}{に|ねん}それぞれ\ruby{彼女}{かの|じょ}あり\\
        \ruby[g]{可愛}{かわ}いい\ruby{一年}{いち|ねん}にゃ\ruby{金}{かね}が\ruby{無}{な}い\\
        \ruby{女}{おんな}\ruby{泣}{な}かせのヨットマン
        
    \end{minipage}
    \begin{minipage}[c]{\blocksize}
        
        \vspace{\linespace}
        \item~\\
        % 4.
        いいじゃありませんか\\
        ヨットマンは\\
        ボロのズボンにボロのシャツ\\
        ボロのヨットに\ruby{乗}{の}ってても\\
        \ruby{腕}{うで}は\ruby{確}{たし}かなヨットマン
        
        \vspace{\linespace}
        \item~\\
        % 5.
        いいじゃありませんか\\
        ヨットマンは\\
        \ruby{欠}{か}けた\ruby{茶碗}{ちゃ|わん}に\ruby{折}{お}れた\ruby{箸}{はし}\\
        クサレズッペにクサレ\ruby{飯}{めし}\\
        それでも\ruby{生}{い}きてるヨットマン
        
    \end{minipage}
    \begin{minipage}[c]{\blocksize}
        
        \vspace{\linespace}
        \item~\\
        % 6.
        いいじゃありませんか\\
        ヨットマンは\\
        \ruby{太}{ふと}い\ruby{腕}{かいな}に\ruby{黒}{くろ}い\ruby{顔}{かお}\\
        キリリと\ruby{締}{し}まった\ruby{口許}{くち|もと}が\\
        グーッといかすぜヨットマン
        
        \vspace{\linespace}
        \item~\\
        % 7.
        \ruby{浜}{はま}の\ruby{娘}{むすめ}が\ruby{噂}{うわさ}する\\
        \ruby{愛}{いと}しあの\ruby{人}{ひと}ヨットマン\\
        お\ruby{嫁}{よめ}に\ruby{行}{い}くならヨットマン\\
        \ruby{今夜}{こん|や}も\ruby{楽}{たの}しい\ruby{夢}{ゆめ}を\ruby{見}{み}る\\
        ヨットマン~~ヨットマン\\
        ヨットヨットマン
    
    \end{minipage}
\end{enumerate} % 番号の箇条書き ここまで
%%%%% 歌詞 ここまで %%%%%
% end body

\end{document}

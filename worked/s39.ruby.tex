\documentclass[10pt,b5j]{tarticle} % B6 縦書き
% \documentclass[10pt,b5j]{tarticle} % B6 縦書き
\AtBeginDvi{\special{papersize=128mm,182mm}} % B6 用用紙サイズ
\usepackage{otf} % Unicode で字を入力するのに必要なパッケージ
\usepackage[size=b6j]{bxpapersize} % B6 用紙サイズを指定
\usepackage[dvipdfmx]{graphicx} % 画像を挿入するためのパッケージ
\usepackage[dvipdfmx]{color} % 色をつけるためのパッケージ
\usepackage{pxrubrica} % ルビを振るためのパッケージ
\usepackage{plext} % 漢数字の enumerate を使うためのパッケージ
\usepackage{multicol} % 複数段組を作るためのパッケージ
\setlength{\topmargin}{14mm} % 上下方向のマージン
\addtolength{\topmargin}{-1in} % 
\setlength{\oddsidemargin}{11mm} % 左右方向のマージン
\addtolength{\oddsidemargin}{-1in} % 
\setlength{\textwidth}{154mm} % B6 用
\setlength{\textheight}{108mm} % B6 用
\setlength{\headsep}{0mm} % 
\setlength{\headheight}{0mm} % 
\setlength{\topskip}{0mm} % 
\setlength{\parskip}{0pt} % 
\def\theenumi{\Kanji{enumi}} % 箇条書きのフォーマットを漢数字に変更
\parindent = 0pt % 段落下げしない
\pagestyle{empty} % すべてのページ番号を消去
% \renewcommand{\baselinestretch}{0.9} % 行間の倍率
 % B6 用テンプレート読み込み

\begin{document}
% begin header
%%%%% タイトルと作者 ここから %%%%%
\begin{minipage}[c]{0.7\hsize} % タイトルは上から 7 割
    \begin{center}
    % begin title
        {\LARGE
            偉大なる北溟の自然 % タイトルを入れる
        }
        {\small 
            (昭和三十九年寮歌) % 年などを入れる
        }
    % end title
    \end{center}
\end{minipage}
\begin{minipage}[c]{0.3\hsize} % 作歌作曲は上から 3 割
    \begin{flushright} % 下寄せにする
        % begin name
        司馬威彦君 作歌・作曲 % 作歌・作曲者
        % end name
    \end{flushright}
\end{minipage}
%%%%% タイトルと作者 ここまで %%%%%
% (序,1,結 了なし繰り返しあり)
% end header

% begin length
\vspace{1.5em} % タイトル, 作者と歌詞の間に隙間を設ける
\newcommand{\linespace}{0.5em} % 行間の設定
\newcommand{\blocksize}{0.33\hsize} % 段組間の設定
\newcommand{\itemmargin}{3em} % 曲番の位置調整の長さ
% end length
% begin body
%%%%% 歌詞 ここから %%%%%
\begin{enumerate} % 番号の箇条書き ここから
    \setlength{\itemindent}{\itemmargin} % 曲番の位置調整
    \begin{minipage}[c]{\blocksize}
    
        \vspace{\linespace}
        \item[序]~\\
        % \ruby{序}{ついで}.
        \ruby[g]{偉大}{おおい}なる\ruby[g]{北溟}{きた}の\ruby{自然}{し|ぜん}は\\
        \ruby{我}{わ}が\ruby[g]{眼前}{まえ}に\ruby{限}{かぎ}りなく\ruby{広}{ひろ}ごりて\\
        \ruby{野}{の}に\ruby{満}{み}てる\ruby{清冽}{せい|れつ}の\ruby{気}{き}は\\
        \ruby{雄々}{お|お}しくも\ruby{気高}{け|だか}き\ruby[g]{情懐}{こころ}もて\\
        \ruby{嶮路}{けん|ろ}\ruby{遙}{はろ}かに\ruby{辿}{たど}り\ruby{来}{こ}し\\
        \ruby{遊子}{ゆう|し}が\ruby{胸}{むね}を\ruby{今}{いま}や\ruby{満}{みた}しぬ
        
        \vspace{\linespace}
        \item~\\
        % 1.
        \ruby{{\UTF{98B7}}々}{ひょう|ひょう}の\ruby{北風}{きた|かぜ}は\ruby{荒}{すさ}び\\ % UTF 颷
        \ruby[g]{白銀}{しろがね}の\ruby{華}{はな}\ruby{大地}{だい|ち}\ruby{覆}{おお}えど\\
        そははろかなる\ruby{古}{いにしえ}より\\
        \ruby{汚}{けが}れなき\ruby{美}{び}の\ruby{世界}{せ|かい}なれば\\
        \ruby{若人}{わこ|うど}はひたぶるの\\
        \ruby{愁}{おも}いを\ruby{秘}{ひ}めて\\
        \ruby[g]{異邦}{とつくに}ゆ\ruby[g]{憧憬}{あこが}れ\ruby{集}{つど}いぬ
        
    \end{minipage}
    \begin{minipage}[c]{\blocksize}
        
        \vspace{\linespace}
        \item~\\
        % 2.
        いよよ\ruby{増}{ま}す\ruby[g]{静寂}{しじま}のなかに\\
        \ruby{永劫}{えい|ごう}の\ruby{影}{かげ}\ruby{宿}{やど}す\ruby{原始}{げん|し}の\ruby[g]{深森}{もり}よ\\
        \ruby{先哲}{せん|てつ}の\ruby[g]{行路}{あと}を\ruby{慕}{した}いて\\
        \ruby[g]{思索}{おもい}\ruby{胸}{むね}に\ruby[g]{楡陵}{おか}を\ruby{歩}{あゆ}めば\\
        \ruby{仰}{あお}ぎみるエルムの\ruby{梢}{こずえ}に\\
        \ruby{萠}{も}え\ruby{出}{いで}ん\ruby{若}{わか}き\ruby[g]{情熱}{ちから}は
        
        \vspace{\linespace}
        \item~\\
        % 3.
        かりそめの\ruby{宿}{やど}にはあれど\\
        \ruby{忘}{わす}れ\ruby{得}{え}じ\ruby{若}{わか}き\ruby{日}{ひ}の\ruby[g]{遍歴}{たび}\\
        \ruby[g]{彷徨}{さまよ}えば\ruby{夕陽}{ゆう|ひ}の\ruby[g]{楡陵}{おか}に\\
        \ruby{宵闇}{よい|やみ}はかそけくも\ruby{訪}{おとず}れ\\
        \ruby{睦}{むつ}みてし\ruby[g]{真心}{こころ}と\ruby[g]{友情}{こころ}に\\
        \ruby{篝火}{かがり|び}は\ruby{赤}{あか}く\ruby{燃}{も}えたり
        
    \end{minipage}
    \begin{minipage}[c]{\blocksize}
        
        \vspace{\linespace}
        \item~\\
        % 4.
        \ruby{輝}{かがや}ける\ruby{北国}{きた|ぐに}のたくみよ\\
        されど\ruby{優}{まさ}りて\ruby{美}{うつく}しき\ruby{自治}{じ|ち}の\ruby[g]{伝統}{つたえ}よ\\
        \ruby{斗}{たたか}い\ruby[g]{苦悩}{なや}み\ruby[g]{寮友}{とも}と\ruby{語}{かた}れば\\
        などて\ruby{疾}{と}く\ruby{過}{す}ぎ\ruby{行}{ゆ}く\ruby[g]{二年}{ふたとせ}の\ruby{春}{はる}\\
        \ruby{願}{ねが}わなん\ruby[g]{永久}{とわ}の\ruby{栄}{は}えを\\
        \ruby{恵迪}{けい|てき}の\ruby[g]{寮故郷}{ふるさと}の\ruby{上}{え}に
        
        \vspace{\linespace}
        \item[結]~\\
        % \ruby{結}{ゆい}.
        されど\ruby{視}{み}よ\ruby{我等}{われ|ら}が\ruby[g]{周囲}{あたり}を\\
        \ruby[g]{邪悪}{よこしま}なる\ruby[g]{権力}{ちから}は\ruby{四方}{よ|も}に\ruby{荒}{すさ}び\\
        \ruby{我等}{われ|ら}が\ruby{愛}{あい}し\ruby{誇}{ほこ}らん\ruby{自治}{じ|ち}の\ruby{砦}{とりで}に\\
        \ruby{暴逆}{ぼう|ぎゃく}の\ruby{誠}{いましめ}は\ruby{課}{か}されんとす\\
        されば\ruby{我}{わ}が\ruby[g]{寮友}{ともどち}よ\ruby{腕}{かいな}むすびて\\
        \ruby{今}{いま}ぞ\ruby{正義}{せい|ぎ}の\ruby{旗}{はた}を\ruby{高}{たか}くかかげん
    
    \end{minipage}
\end{enumerate} % 番号の箇条書き ここまで
%%%%% 歌詞 ここまで %%%%%
% end body

\end{document}

\documentclass[10pt,b5j]{tarticle} % B6 縦書き
% \documentclass[10pt,b5j]{tarticle} % B6 縦書き
\AtBeginDvi{\special{papersize=128mm,182mm}} % B6 用用紙サイズ
\usepackage{otf} % Unicode で字を入力するのに必要なパッケージ
\usepackage[size=b6j]{bxpapersize} % B6 用紙サイズを指定
\usepackage[dvipdfmx]{graphicx} % 画像を挿入するためのパッケージ
\usepackage[dvipdfmx]{color} % 色をつけるためのパッケージ
\usepackage{pxrubrica} % ルビを振るためのパッケージ
\usepackage{plext} % 漢数字の enumerate を使うためのパッケージ
\usepackage{multicol} % 複数段組を作るためのパッケージ
\setlength{\topmargin}{14mm} % 上下方向のマージン
\addtolength{\topmargin}{-1in} % 
\setlength{\oddsidemargin}{11mm} % 左右方向のマージン
\addtolength{\oddsidemargin}{-1in} % 
\setlength{\textwidth}{154mm} % B6 用
\setlength{\textheight}{108mm} % B6 用
\setlength{\headsep}{0mm} % 
\setlength{\headheight}{0mm} % 
\setlength{\topskip}{0mm} % 
\setlength{\parskip}{0pt} % 
\def\theenumi{\Kanji{enumi}} % 箇条書きのフォーマットを漢数字に変更
\parindent = 0pt % 段落下げしない
\pagestyle{empty} % すべてのページ番号を消去
% \renewcommand{\baselinestretch}{0.9} % 行間の倍率
 % B6 用テンプレート読み込み

\begin{document}
% begin header
%%%%% タイトルと作者 ここから %%%%%
\begin{minipage}[c]{0.7\hsize} % タイトルは上から 7 割
    \begin{center}
    % begin title
        {\LARGE
            医学部学友会歌 % タイトルを入れる
        }
        {\small 
            (昭和十四年) % 年などを入れる
        }
    % end title
    \end{center}
\end{minipage}
\begin{minipage}[c]{0.3\hsize} % 作歌作曲は上から 3 割
    \begin{flushright} % 下寄せにする
        % begin name
        大槻均君 作歌\\森鼻正美君 作曲 % 作歌・作曲者
        % end name
    \end{flushright}
\end{minipage}
%%%%% タイトルと作者 ここまで %%%%%
% % end header

% begin length
\vspace{1.5em} % タイトル, 作者と歌詞の間に隙間を設ける
\newcommand{\linespace}{0.5em} % 行間の設定
\newcommand{\blocksize}{0.5\hsize} % 段組間の設定
\newcommand{\itemmargin}{6em} % 曲番の位置調整の長さ
% end length
% begin body
%%%%% 歌詞 ここから %%%%%
\begin{enumerate} % 番号の箇条書き ここから
    \setlength{\itemindent}{\itemmargin} % 曲番の位置調整
    \begin{minipage}[c]{\blocksize}
    
        \vspace{\linespace}
        \item~\\
        % 1.
        \ruby{北国}{}の\ruby{聖}{}き\ruby{都}{}に\\
        \ruby{伝}{}へたる\ruby{崇}{}き\ruby{教}{}を\\
        \ruby{慕}{}ひ\ruby{來}{}りぬ\\
        \ruby{仰}{}がずや\\
        \ruby{先人}{}の\ruby{跡}{}\\
        \ruby{拓}{}きてむ\\
        \ruby{未踏}{}の\ruby{境}{}\\
        その\ruby{榮光}{}に\ruby{輝}{}く\ruby{我等}{}!
        
        \vspace{\linespace}
        \item~\\
        % 2.
        \ruby{渦巻}{}ける\ruby{世紀}{}の\ruby{嵐}{}\\
        \ruby{日}{}の\ruby{本}{}の\ruby{道}{}は\ruby{尊}{}し\\
        いざ\ruby{起}{}たんかな\\
        \ruby{究}{}めずや\\
        \ruby{生命}{}の\ruby{神秘}{}\\
        \ruby{讃}{}へなむ\\
        \ruby{医学}{}の\ruby{勝利}{}\\
        その\ruby{榮光}{}に\ruby{輝}{}く\ruby{我等}{}!
        
        \vspace{\linespace}
        \item~\\
        % 3.
        「フラーテ」は\ruby{平和}{}の\ruby{団欒}{}\\
        \ruby{美}{}はしの\ruby{山河}{}に\ruby{充}{}つる\\
        \ruby{活氣亨}{}けつつ\\
        \ruby{誇}{}らずや\\
        \ruby{熱}{}き\ruby{血潮}{}を\\
        \ruby{謳}{}ふべし\\
        \ruby{躍}{}る\ruby{力}{}を\\
        その\ruby{榮光}{}に\ruby{輝}{}く\ruby{我等}{}!
    
    \end{minipage}
\end{enumerate} % 番号の箇条書き ここまで
%%%%% 歌詞 ここまで %%%%%
% end body

\end{document}

\documentclass[10pt,b5j]{tarticle} % B6 縦書き
% \documentclass[10pt,b5j]{tarticle} % B6 縦書き
\AtBeginDvi{\special{papersize=128mm,182mm}} % B6 用用紙サイズ
\usepackage{otf} % Unicode で字を入力するのに必要なパッケージ
\usepackage[size=b6j]{bxpapersize} % B6 用紙サイズを指定
\usepackage[dvipdfmx]{graphicx} % 画像を挿入するためのパッケージ
\usepackage[dvipdfmx]{color} % 色をつけるためのパッケージ
\usepackage{pxrubrica} % ルビを振るためのパッケージ
\usepackage{multicol} % 複数段組を作るためのパッケージ
\setlength{\topmargin}{14mm} % 上下方向のマージン
\addtolength{\topmargin}{-1in} % 
\setlength{\oddsidemargin}{11mm} % 左右方向のマージン
\addtolength{\oddsidemargin}{-1in} % 
\setlength{\textwidth}{154mm} % B6 用
\setlength{\textheight}{108mm} % B6 用
\setlength{\headsep}{0mm} % 
\setlength{\headheight}{0mm} % 
\setlength{\topskip}{0mm} % 
\setlength{\parskip}{0pt} % 
\def\labelenumi{\theenumi、} % 箇条書きのフォーマット
\parindent = 0pt % 段落下げしない

 % B6 用テンプレート読み込み

\begin{document}
% begin header
%%%%% タイトルと作者 ここから %%%%%
\begin{minipage}[c]{0.7\hsize} % タイトルは上から 7 割
    \begin{center}
    % begin title
        {\LARGE
            郭公の声に % タイトルを入れる
        }
        {\small 
            (昭和三年寮歌) % 年などを入れる
        }
    % end title
    \end{center}
\end{minipage}
\begin{minipage}[c]{0.3\hsize} % 作歌作曲は上から 3 割
    \begin{flushright} % 下寄せにする
        % begin name
        古河勝夫君 作歌\\宮本正治君 作曲 % 作歌・作曲者
        % end name
    \end{flushright}
\end{minipage}
%%%%% タイトルと作者 ここまで %%%%%
% (1,6 了なし)
% end header

% begin length
\vspace{1.5em} % タイトル, 作者と歌詞の間に隙間を設ける
\newcommand{\linespace}{0.5em} % 行間の設定
\newcommand{\blocksize}{0.5\hsize} % 段組間の設定
\newcommand{\itemmargin}{3em} % 曲番の位置調整の長さ
% end length
% begin body
%%%%% 歌詞 ここから %%%%%
\begin{enumerate} % 番号の箇条書き ここから
    \setlength{\itemindent}{\itemmargin} % 曲番の位置調整
    \begin{minipage}[c]{\blocksize}
    
        \vspace{\linespace}
        \item~\\
        % 1.
        \ruby{郭公}{}の\ruby{声}{}に\ruby{迷夢}{}の\ruby{夜}{}は\ruby{明}{}けて\\
        \ruby{紫紺}{}の\ruby{雲}{}の\ruby{色}{}も\ruby{褪}{}めゆき\\
        \ruby{春芝草}{}に\ruby{風}{}のそよげば\\
        \ruby{旭光}{}は\ruby{見}{}よ\ruby{東雲}{}の\ruby{沈黙}{}を\ruby{破}{}り\\
        \ruby{自然}{}の\ruby{精姿紅}{}に\ruby{揺}{}らぎぬ\\
        \ruby{讃}{}へなんうら\ruby{若}{}き\ruby{日}{}の\\
        \ruby{朝}{}の\ruby{神秘}{}よ
        
    \end{minipage}
    \begin{minipage}[c]{\blocksize}
        
        \vspace{\linespace}
        \item~\\
        % 2.
        \ruby{濃緑}{}に\ruby{原始}{}の\ruby{森}{}の\ruby{茂}{}る\ruby{候}{}\\
        \ruby{君影草}{}の\ruby{花}{}も\ruby{散}{}り\ruby{果}{}て\\
        クローバの\ruby{上}{}に\ruby{胡蝶舞}{}ひ\ruby{舞}{}ふ\\
        \ruby{蒼空}{}の\ruby{小鳥}{}を\ruby{追}{}ふか\ruby{陽炎立}{}ちて\\
        \ruby{牧場}{}に\ruby{悠}{}き\ruby{牛}{}の\ruby{声聞}{}く\\
        \ruby{仰臥}{}せる\ruby{牧童}{}の\ruby{上}{}に\ruby{雲}{}は\ruby{動}{}かず
        
    \end{minipage}
    \begin{minipage}[c]{\blocksize}
        
        \vspace{\linespace}
        \item~\\
        % 3.
        \ruby{俊厳}{}の\ruby{秋気何時}{}しか\ruby{野}{}に\ruby{充}{}ちて\\
        \ruby{可憐}{}し\ruby{虫}{}の\ruby{音}{}ものを\ruby{思}{}はす\\
        \ruby{移}{}ろふ\ruby{自然}{}の\ruby{色彩賑}{}はへど\\
        \ruby{沁々}{}と\ruby{人}{}の\ruby{運命}{}の\ruby{秋}{}も\ruby{偲}{}ばれ\\
        \ruby{淋}{}しき\ruby{哀愁}{}に\ruby{涙}{}にじみて\\
        \ruby{蕭々夕風}{}いとど\ruby{身}{}には\ruby{悩}{}し
        
    \end{minipage}
    \begin{minipage}[c]{\blocksize}
        
        \vspace{\linespace}
        \item~\\
        % 4.
        \ruby{銀月}{}は\ruby{今雪原}{}の\ruby{上}{}に\ruby{照}{}り\\
        エルムの\ruby{梢淡青}{}く\ruby{映}{}りて\\
        \ruby{野末}{}に\ruby{籠}{}むる\ruby{夢}{}の\ruby{狭霧}{}の\\
        \ruby{奥深}{}く\ruby{幻想}{}の\ruby{燈火}{}の\ruby{明滅}{}を\ruby{見}{}る\\
        \ruby{凍}{}らんとする\ruby{霊気}{}かすかに\\
        \ruby{一条}{}の\ruby{橇路}{}に\ruby{残}{}る\ruby{鈴}{}に\ruby{震}{}へり
        
    \end{minipage}
    \begin{minipage}[c]{\blocksize}
        
        \vspace{\linespace}
        \item~\\
        % 5.
        \ruby{丈}{}なる\ruby{草踏}{}み\ruby{分}{}けて\ruby{蝦夷}{}ヶ\ruby{野}{}に\\
        \ruby{迪}{}を\ruby{恵}{}ねし\ruby{人}{}の\ruby{姿}{}よ\\
        さ\ruby{迷}{}ひ\ruby{暮}{}れて\ruby{星仰}{}ぎけん\\
        ああそこに\ruby{原始}{}の\ruby{影}{}は\ruby{更}{}に\ruby{薄}{}れて\\
        \ruby{老}{}いし\ruby{楡}{}に\ruby{嵐荒涼}{}びつ\\
        \ruby{夕陽}{}は\ruby{手稲}{}の\ruby{背淡紅}{}く\ruby{映}{}せり
        
    \end{minipage}
    \begin{minipage}[c]{\blocksize}
        
        \vspace{\linespace}
        \item~\\
        % 6.
        \ruby{白樺}{}よポプラ\ruby{並木}{}よアカシヤよ\\
        \ruby{春秋三度廻}{}り\ruby{去}{}りなば\\
        \ruby{若}{}き\ruby{生命}{}は\ruby{疾}{}くに\ruby{萎}{}え\ruby{果}{}て\\
        \ruby{逝}{}にし\ruby{日}{}の\ruby{宴遊}{}の\ruby{宵}{}の\\
        \ruby{夢}{}も\ruby{追}{}ひ\ruby{得}{}じ\\
        \ruby{此}{}の\ruby{経営}{}に\ruby{思想分}{}ちし\\
        \ruby{寮友}{}よ\ruby{心}{}の\ruby{記念永久}{}に\ruby{謳}{}はん
    
    \end{minipage}
\end{enumerate} % 番号の箇条書き ここまで
%%%%% 歌詞 ここまで %%%%%
% end body

\end{document}

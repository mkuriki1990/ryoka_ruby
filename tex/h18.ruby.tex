\documentclass[10pt,b5j]{tarticle} % B6 縦書き
% \documentclass[10pt,b5j]{tarticle} % B6 縦書き
\AtBeginDvi{\special{papersize=128mm,182mm}} % B6 用用紙サイズ
\usepackage{otf} % Unicode で字を入力するのに必要なパッケージ
\usepackage[size=b6j]{bxpapersize} % B6 用紙サイズを指定
\usepackage[dvipdfmx]{graphicx} % 画像を挿入するためのパッケージ
\usepackage[dvipdfmx]{color} % 色をつけるためのパッケージ
\usepackage{pxrubrica} % ルビを振るためのパッケージ
\usepackage{multicol} % 複数段組を作るためのパッケージ
\setlength{\topmargin}{14mm} % 上下方向のマージン
\addtolength{\topmargin}{-1in} % 
\setlength{\oddsidemargin}{11mm} % 左右方向のマージン
\addtolength{\oddsidemargin}{-1in} % 
\setlength{\textwidth}{154mm} % B6 用
\setlength{\textheight}{108mm} % B6 用
\setlength{\headsep}{0mm} % 
\setlength{\headheight}{0mm} % 
\setlength{\topskip}{0mm} % 
\setlength{\parskip}{0pt} % 
\def\labelenumi{\theenumi、} % 箇条書きのフォーマット
\parindent = 0pt % 段落下げしない

 % B6 用テンプレート読み込み

\begin{document}
% begin header
%%%%% タイトルと作者 ここから %%%%%
\begin{minipage}[c]{0.7\hsize} % タイトルは上から 7 割
    \begin{center}
    % begin title
        {\LARGE
            ただ一心に % タイトルを入れる
        }
        {\small 
            (平成十八年度寮歌) % 年などを入れる
        }
    % end title
    \end{center}
\end{minipage}
\begin{minipage}[c]{0.3\hsize} % 作歌作曲は上から 3 割
    \begin{flushright} % 下寄せにする
        % begin name
        岩崎良平君 作歌\\吉田和史君 作曲 % 作歌・作曲者
        % end name
    \end{flushright}
\end{minipage}
%%%%% タイトルと作者 ここまで %%%%%
% (1,2,3,4 了なし繰り返しあり)
% end header

% begin length
\vspace{1.5em} % タイトル, 作者と歌詞の間に隙間を設ける
\newcommand{\linespace}{0.5em} % 行間の設定
\newcommand{\blocksize}{0.5\hsize} % 段組間の設定
\newcommand{\itemmargin}{6em} % 曲番の位置調整の長さ
% end length
% begin body
%%%%% 歌詞 ここから %%%%%
\begin{enumerate} % 番号の箇条書き ここから
    \setlength{\itemindent}{\itemmargin} % 曲番の位置調整
    \begin{minipage}[c]{\blocksize}
    
        \vspace{\linespace}
        \item~\\
        % 1.
        \ruby{紺碧}{}の\ruby{空}{}を\ruby{貫}{}く\\
        \ruby{一筋}{}の\ruby{白雲無限}{}の\ruby{可能性}{}\\
        \ruby{我}{}が\ruby{胸}{}に\ruby{秘}{}め\\
        \ruby{広}{}がれる\ruby{迪}{}\\
        ただ\ruby{一心}{}に\ruby{信}{}じ\ruby{歩}{}もう
        
        \vspace{\linespace}
        \item~\\
        % 2.
        \ruby{漆黒}{}の\ruby{闇}{}を\ruby{貫}{}く\\
        \ruby{静}{}かな\ruby{月明}{}り\\
        \ruby{内}{}なる\ruby{大志息}{}を\ruby{潜}{}めて\\
        \ruby{開}{}かれる\ruby{朝}{}\\
        ただ\ruby{一心}{}に\ruby{信}{}じ\ruby{臨}{}まん
        
        \vspace{\linespace}
        \item~\\
        % 3.
        \ruby{朧灰}{}の\ruby{雲}{}を\ruby{貫}{}く\\
        \ruby{揺}{}るがぬ\ruby{意思}{}\\
        \ruby{昔年}{}の\ruby{想}{}い\ruby{翼}{}と\ruby{共}{}に\\
        \ruby{光差}{}す\ruby{先}{}\\
        ただ\ruby{一心}{}に\ruby{信}{}じ\ruby{飛}{}びたつ
        
        \vspace{\linespace}
        \item~\\
        % 4.
        \ruby{真紅}{}の\ruby{心}{}を\ruby{貫}{}く\\
        \ruby{熱}{}き\ruby{眼差}{}し\\
        \ruby{叶}{}わぬ\ruby{夢友}{}らに\ruby{託}{}し\\
        \ruby{新}{}たなる\ruby{迪}{}\\
        ただ\ruby{一心}{}に\ruby{信}{}じ\ruby{進}{}もう
    
    \end{minipage}
\end{enumerate} % 番号の箇条書き ここまで
%%%%% 歌詞 ここまで %%%%%
% end body

\end{document}

\documentclass[10pt,b5j]{tarticle} % B6 縦書き
% \documentclass[10pt,b5j]{tarticle} % B6 縦書き
\AtBeginDvi{\special{papersize=128mm,182mm}} % B6 用用紙サイズ
\usepackage{otf} % Unicode で字を入力するのに必要なパッケージ
\usepackage[size=b6j]{bxpapersize} % B6 用紙サイズを指定
\usepackage[dvipdfmx]{graphicx} % 画像を挿入するためのパッケージ
\usepackage[dvipdfmx]{color} % 色をつけるためのパッケージ
\usepackage{pxrubrica} % ルビを振るためのパッケージ
\usepackage{multicol} % 複数段組を作るためのパッケージ
\setlength{\topmargin}{14mm} % 上下方向のマージン
\addtolength{\topmargin}{-1in} % 
\setlength{\oddsidemargin}{11mm} % 左右方向のマージン
\addtolength{\oddsidemargin}{-1in} % 
\setlength{\textwidth}{154mm} % B6 用
\setlength{\textheight}{108mm} % B6 用
\setlength{\headsep}{0mm} % 
\setlength{\headheight}{0mm} % 
\setlength{\topskip}{0mm} % 
\setlength{\parskip}{0pt} % 
\def\labelenumi{\theenumi、} % 箇条書きのフォーマット
\parindent = 0pt % 段落下げしない

 % B6 用テンプレート読み込み

\begin{document}
% begin header
%%%%% タイトルと作者 ここから %%%%%
\begin{minipage}[c]{0.7\hsize} % タイトルは上から 7 割
    \begin{center}
    % begin title
        {\LARGE
            瓔珞みがく % タイトルを入れる
        }
        {\small 
            (大正九年桜星会歌) % 年などを入れる
        }
    % end title
    \end{center}
\end{minipage}
\begin{minipage}[c]{0.3\hsize} % 作歌作曲は上から 3 割
    \begin{flushright} % 下寄せにする
        % begin name
        佐藤一雄君 作歌\\置塩寄君 作曲 % 作歌・作曲者
        % end name
    \end{flushright}
\end{minipage}
%%%%% タイトルと作者 ここまで %%%%%
% (1,2,3,4,8 了あり)
% end header

% begin body
\vspace{1.5em} % タイトル, 作者と歌詞の間に隙間を設ける
\newcommand{\linespace}{0.5em} % 行間の設定
\newcommand{\blocksize}{0.5\hsize} % 段組間の設定
%%%%% 歌詞 ここから %%%%%
% begin lilycs
\begin{enumerate} % 番号の箇条書き ここから
    \begin{minipage}[c]{\blocksize}
    
        \vspace{\linespace}
        \item
        % 1.
        \ruby{瓔珞}{}みがく\ruby{石狩}{}の\\
        \ruby{源遠}{}く\ruby{訪}{}ひくれば\\
        \ruby{原始}{}の\ruby{森}{}は\ruby{闇}{}くして\\
        \ruby{雪解}{}の\ruby{泉玉}{}と\ruby{湧}{}く
        
        \vspace{\linespace}
        \item
        % 2.
        \ruby{浜茄子紅}{}き\ruby{磯辺}{}にも\\
        \ruby{鈴蘭薫}{}る\ruby{谷間}{}にも\\
        \ruby{愛奴}{}の\ruby{姿薄}{}れゆく\\
        \ruby{蝦夷}{}の\ruby{昔}{}を\ruby{懐}{}ふかな
        
        \vspace{\linespace}
        \item
        % 3.
        \ruby{今円山}{}の\ruby{桜花}{}\\
        \ruby{歴史}{}は\ruby{旧}{}りて\ruby{四十年}{}\\
        \ruby{我}{}が\ruby{学}{}び\ruby{舎}{}の\ruby{先人}{}が\\
        \ruby{建}{}てし\ruby{功}{}はいや\ruby{栄}{}ゆ
        
        \vspace{\linespace}
        \item
        % 4.
        その\ruby{絢爛}{}の\ruby{花霞}{}\\
        \ruby{憧憬集}{}ふ\ruby{四百}{}の\\
        \ruby{健児}{}が\ruby{希望深}{}ければ\\
        \ruby{北斗}{}に\ruby{強}{}き\ruby{黙示}{}あり
        
        \vspace{\linespace}
        \item
        % 5.
        \ruby{醜雲消}{}えて\ruby{人}{}の\ruby{世}{}に\\
        \ruby{陽光}{}はうららかに\ruby{輝}{}けど\\
        \ruby{風}{}の\ruby{名残}{}のつきやらで\\
        \ruby{狂瀾}{}さわぐ\ruby{今}{}し\ruby{今}{}
        
        \vspace{\linespace}
        \item
        % 6.
        \ruby{潮}{}に\ruby{暮}{}るる\ruby{西}{}の\ruby{空}{}\\
        \ruby{月}{}も\ruby{凍}{}らむシベリアの\\
        \ruby{吾}{}が\ruby{皇軍}{}を\ruby{思}{}ひては\\
        \ruby{猛}{}けき\ruby{心}{}の\ruby{踊}{}らずや
        
        \vspace{\linespace}
        \item
        % 7.
        \ruby{白銀狂}{}ふ\ruby{埋}{}れ\ruby{路}{}も\\
        \ruby{踏}{}みて\ruby{拓}{}かむわが\ruby{前途}{}\\
        はろけき\ruby{牧場}{}に\ruby{嘯}{}けば\\
        \ruby{雲影}{}はやし\ruby{草}{}の\ruby{波}{}
        
        \vspace{\linespace}
        \item
        % 8.
        \ruby{想}{}を\ruby{秘}{}めし\ruby{若人}{}が\\
        \ruby{唇}{}かたくほほゑみつ\\
        \ruby{仰}{}げば\ruby{高}{}く\ruby{聳}{}え\ruby{立}{}つ\\
        \ruby{羊蹄山}{}に\ruby{雪潔}{}し
    
    \end{minipage}
\end{enumerate} % 番号の箇条書き ここまで
% end lilycs
%%%%% 歌詞 ここまで %%%%%
% end body

\end{document}

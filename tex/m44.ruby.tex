\documentclass[10pt,b5j]{tarticle} % B6 縦書き
% \documentclass[10pt,b5j]{tarticle} % B6 縦書き
\AtBeginDvi{\special{papersize=128mm,182mm}} % B6 用用紙サイズ
\usepackage{otf} % Unicode で字を入力するのに必要なパッケージ
\usepackage[size=b6j]{bxpapersize} % B6 用紙サイズを指定
\usepackage[dvipdfmx]{graphicx} % 画像を挿入するためのパッケージ
\usepackage[dvipdfmx]{color} % 色をつけるためのパッケージ
\usepackage{pxrubrica} % ルビを振るためのパッケージ
\usepackage{multicol} % 複数段組を作るためのパッケージ
\setlength{\topmargin}{14mm} % 上下方向のマージン
\addtolength{\topmargin}{-1in} % 
\setlength{\oddsidemargin}{11mm} % 左右方向のマージン
\addtolength{\oddsidemargin}{-1in} % 
\setlength{\textwidth}{154mm} % B6 用
\setlength{\textheight}{108mm} % B6 用
\setlength{\headsep}{0mm} % 
\setlength{\headheight}{0mm} % 
\setlength{\topskip}{0mm} % 
\setlength{\parskip}{0pt} % 
\def\labelenumi{\theenumi、} % 箇条書きのフォーマット
\parindent = 0pt % 段落下げしない

 % B6 用テンプレート読み込み

\begin{document}
% begin header
%%%%% タイトルと作者 ここから %%%%%
\begin{minipage}[c]{0.7\hsize} % タイトルは上から 7 割
    \begin{center}
    % begin title
        {\LARGE
            藻岩の緑 % タイトルを入れる
        }
        {\small 
            (明治四十四年寮歌) % 年などを入れる
        }
    % end title
    \end{center}
\end{minipage}
\begin{minipage}[c]{0.3\hsize} % 作歌作曲は上から 3 割
    \begin{flushright} % 下寄せにする
        % begin name
        松山茂助君 作歌\\柳沢秀雄君 作曲 % 作歌・作曲者
        % end name
    \end{flushright}
\end{minipage}
%%%%% タイトルと作者 ここまで %%%%%
% (1,2,3,4 了あり)
% end header

% begin body
\vspace{1.5em} % タイトル, 作者と歌詞の間に隙間を設ける
\newcommand{\linespace}{0.5em} % 行間の設定
\newcommand{\blocksize}{0.5\hsize} % 段組間の設定
%%%%% 歌詞 ここから %%%%%
% begin lilycs
\begin{enumerate} % 番号の箇条書き ここから
    \begin{minipage}[c]{\blocksize}
    
        \vspace{\linespace}
        \item
        % 1.
        \ruby{藻岩}{}の\ruby{緑春闌}{}けて\\
        \ruby{万朶一朶}{}の\ruby{朝霞}{}\\
        \ruby{憧憬彩}{}と\ruby{流}{}れては\\
        \ruby{花皆奇}{}しき\ruby{香}{}ならずや\\
        \ruby{若}{}き\ruby{血潮}{}の\ruby{踊}{}る\ruby{時}{}\\
        \ruby{希望}{}の\ruby{前途光}{}あり
        
        \vspace{\linespace}
        \item
        % 2.
        \ruby{青葉波}{}よるアカシヤの\\
        \ruby{薫}{}る\ruby{木影}{}に\ruby{立}{}ちよれば\\
        \ruby{長風夏}{}の\ruby{雲}{}ゆらぎ\\
        \ruby{秋}{}は\ruby{牧場}{}の\ruby{夕}{}まぐれ\\
        \ruby{鐘声止}{}みて\ruby{今暫}{}し\\
        \ruby{牛}{}の\ruby{背}{}に\ruby{散}{}る\ruby{蔦紅葉}{}
        
        \vspace{\linespace}
        \item
        % 3.
        あはれ「\ruby{美}{}の\ruby{国}{}」\ruby{石狩}{}の\\
        \ruby{自然}{}を\ruby{己}{}が\ruby{揺籃}{}に\\
        おほし\ruby{立}{}つ\ruby{可}{}き\ruby{人皆}{}の\\
        \ruby{意気紅霓}{}に\ruby{似}{}たるかな\\
        \ruby{一撃万里}{}す\ruby{大鵬}{}の\\
        \ruby{翼整装}{}ふ\ruby{思}{}あり
        
        \vspace{\linespace}
        \item
        % 4.
        \ruby{斗南}{}の\ruby{翼拡}{}げては\\
        \ruby{天地広}{}しと\ruby{誰}{}か\ruby{云}{}ふ\\
        \ruby{雲}{}より\ruby{高}{}きアンデスの\\
        \ruby{裾野}{}に\ruby{友}{}よ\ruby{羊逐}{}へ\\
        \ruby{天}{}に\ruby{漲}{}るアマゾンの\\
        \ruby{岸辺}{}の\ruby{森}{}に\ruby{斧}{}を\ruby{振}{}れ
        
        \vspace{\linespace}
        \item
        % 5.
        \ruby{弦月落}{}ちて\ruby{白楊}{}の\\
        \ruby{樹林}{}の\ruby{暗}{}の\ruby{深}{}き\ruby{時}{}\\
        \ruby{八荒裂}{}けて\ruby{万籟}{}の\\
        \ruby{声}{}すさまじく\ruby{吹雪}{}く\ruby{時}{}\\
        \ruby{世}{}の\ruby{濁流}{}を\ruby{叱咤}{}して\\
        \ruby{巨人}{}の\ruby{叫}{}び\ruby{茲}{}にあり
        
        \vspace{\linespace}
        \item
        % 6.
        \ruby{浮華軽佻}{}の\ruby{風}{}あれて\\
        \ruby{驕奢}{}の\ruby{波}{}は\ruby{狂}{}ふとも\\
        \ruby{北斗}{}の\ruby{光清}{}ければ\\
        \ruby{世}{}は\ruby{永久}{}に\ruby{我世}{}なり\\
        \ruby{聞}{}けや\ruby{人々北州}{}に\\
        \ruby{正気溢}{}るる\ruby{意気}{}の\ruby{歌}{}
    
    \end{minipage}
\end{enumerate} % 番号の箇条書き ここまで
% end lilycs
%%%%% 歌詞 ここまで %%%%%
% end body

\end{document}

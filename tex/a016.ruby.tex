\documentclass[10pt,b5j]{tarticle} % B6 縦書き
% \documentclass[10pt,b5j]{tarticle} % B6 縦書き
\AtBeginDvi{\special{papersize=128mm,182mm}} % B6 用用紙サイズ
\usepackage{otf} % Unicode で字を入力するのに必要なパッケージ
\usepackage[size=b6j]{bxpapersize} % B6 用紙サイズを指定
\usepackage[dvipdfmx]{graphicx} % 画像を挿入するためのパッケージ
\usepackage[dvipdfmx]{color} % 色をつけるためのパッケージ
\usepackage{pxrubrica} % ルビを振るためのパッケージ
\usepackage{multicol} % 複数段組を作るためのパッケージ
\setlength{\topmargin}{14mm} % 上下方向のマージン
\addtolength{\topmargin}{-1in} % 
\setlength{\oddsidemargin}{11mm} % 左右方向のマージン
\addtolength{\oddsidemargin}{-1in} % 
\setlength{\textwidth}{154mm} % B6 用
\setlength{\textheight}{108mm} % B6 用
\setlength{\headsep}{0mm} % 
\setlength{\headheight}{0mm} % 
\setlength{\topskip}{0mm} % 
\setlength{\parskip}{0pt} % 
\def\labelenumi{\theenumi、} % 箇条書きのフォーマット
\parindent = 0pt % 段落下げしない

 % B6 用テンプレート読み込み

\begin{document}
% begin header
%%%%% タイトルと作者 ここから %%%%%
\begin{minipage}[c]{0.7\hsize} % タイトルは上から 7 割
    \begin{center}
    % begin title
        {\LARGE
            STUDENTENLIED % タイトルを入れる
        }
        {\small 
             % 年などを入れる
        }
    % end title
    \end{center}
\end{minipage}
\begin{minipage}[c]{0.3\hsize} % 作歌作曲は上から 3 割
    \begin{flushright} % 下寄せにする
        % begin name
        Prof.Hans Koller君 作歌・作曲 % 作歌・作曲者
        % end name
    \end{flushright}
\end{minipage}
%%%%% タイトルと作者 ここまで %%%%%
% % end header

% begin length
\vspace{1.5em} % タイトル, 作者と歌詞の間に隙間を設ける
\newcommand{\linespace}{0.5em} % 行間の設定
\newcommand{\blocksize}{0.5\hsize} % 段組間の設定
\newcommand{\itemmargin}{3em} % 曲番の位置調整の長さ
% end length
% begin body
%%%%% 歌詞 ここから %%%%%
\begin{enumerate} % 番号の箇条書き ここから
    \setlength{\itemindent}{\itemmargin} % 曲番の位置調整
    \begin{minipage}[c]{\blocksize}
    
        \vspace{\linespace}
        \item~\\
        % 1.
        Erst kommt der junge Studio, tralala, tralala,\\
        % Ganz still und scheu nach Sapporo, tralala, tralala.
        Die erste Zeit wirt ihm nicht leicht,\\
        % Denn oft das Heimweh ihn beschleicht.
            Jupeidi, peidi, peida, jupeidi, jupeida,\\
            jupeidi, peidi, peida, jupeidi, peida !
        
    \end{minipage}
    \begin{minipage}[c]{\blocksize}
        
        \vspace{\linespace}
        \item~\\
        % 2.
        Im Herbst lebt unser Studio, tralala, tralala,\\
        Vergnugt in delce jubilo, tralala, tralala,\\
        Erst Hasenjagd in Kotoni,\\
        Und dann zu Nacht Jabiraki.
        
    \end{minipage}
    \begin{minipage}[c]{\blocksize}
        
        \vspace{\linespace}
        \item~\\
        % 3.
        Im Mai vom schmucken T??nspielring, tralala, tralala,\\
        Schallt lusting es, bum, bum, tsching, tsching, tsching,\\
        Das ist das Fest vom Undokwai,\\
        Drum schmettert all: Hurrah, Banzai !
        
    \end{minipage}
    \begin{minipage}[c]{\blocksize}
        
        \vspace{\linespace}
        \item~\\
        % 4.
        Oft hat der Studio nicht viel Geld, tralala, tralala,\\
        % Deshalb ist er recht schlecht bestellt, tralala, tralala.
        O Kashi, Mochi und Manju,\\
        Verchafft er sich durch Komparu.
        
    \end{minipage}
    \begin{minipage}[c]{\blocksize}
        
        \vspace{\linespace}
        \item~\\
        % 5.
        Doch kommen die Examina, tralala, tralala,\\
        % Da singt er nicht mehr Jupeidi, tralala, tralala.
        Er sieht so blass und traurig drein;\\
        Das wird wohl vom Studieren sein.
        
    \end{minipage}
    \begin{minipage}[c]{\blocksize}
        
        \vspace{\linespace}
        \item~\\
        % 6.
        Ihm wird von all dem Zeug so dumm, tralala, tralala,\\
        % Als ging im Kopf ein Muhlrad um, tralala, tralala.
        Er ochst und schanzt so mancherlei,\\
        Und traumt sogar von Rakudai.
        
    \end{minipage}
    \begin{minipage}[c]{\blocksize}
        
        \vspace{\linespace}
        \item~\\
        % 7.
        Dann in dem Heft steht schwarz auf weiss, tralala, tralala,\\
        % Wovon der Studio nicht viel weiss, tralala, tralala.
        Doch kommt zuletzt der Tag herbei,\\
        So spricht er leis: Shikata-ga-nai.
        
        
    \end{minipage}
    \begin{minipage}[c]{\blocksize}
        
        \vspace{\linespace}
        \item~\\
        コラー\ruby{先生}{せんせい}(スイス\ruby{人}{すいすじん})は\ruby{明治}{}41\ruby{年}{めいじよんじゅういちねん}から\ruby{予科}{よか}でドイツ\ruby{語}{どいつご}を\ruby{教}{おし}えられ、\ruby{大正}{}14\ruby{年}{めいじよんじゅういちねん}、\\
        \ruby{在職}{ざいしょく}\ruby{中}{ちゅう}に\ruby{死去}{しきょ}された。この\ruby{歌}{うた}は\ruby{明治}{めいじ}\ruby{末期}{まっき}から\ruby{大正}{たいしょう}\ruby{初期}{しょき}に\ruby{作}{つく}られたものである。\\
        \ruby{曲}{きょく}は「Jupeidiの\ruby{譜}{ふ}」とのみ\ruby{記}{しる}されている。1845\ruby{年}{せんはっぴゃくよんじゅうごねん}のUrbummellied(G.Weber\\
        \ruby{作詞}{さくし}、R.Schaffer \ruby{作曲}{さっきょく})が\ruby{元}{もと}\ruby{歌}{うた}と\ruby{推定}{すいてい}される。
    
    \end{minipage}
\end{enumerate} % 番号の箇条書き ここまで
%%%%% 歌詞 ここまで %%%%%
% end body

\end{document}

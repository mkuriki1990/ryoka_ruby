\documentclass[10pt,b5j]{tarticle} % B6 縦書き
% \documentclass[10pt,b5j]{tarticle} % B6 縦書き
\AtBeginDvi{\special{papersize=128mm,182mm}} % B6 用用紙サイズ
\usepackage{otf} % Unicode で字を入力するのに必要なパッケージ
\usepackage[size=b6j]{bxpapersize} % B6 用紙サイズを指定
\usepackage[dvipdfmx]{graphicx} % 画像を挿入するためのパッケージ
\usepackage[dvipdfmx]{color} % 色をつけるためのパッケージ
\usepackage{pxrubrica} % ルビを振るためのパッケージ
\usepackage{multicol} % 複数段組を作るためのパッケージ
\setlength{\topmargin}{14mm} % 上下方向のマージン
\addtolength{\topmargin}{-1in} % 
\setlength{\oddsidemargin}{11mm} % 左右方向のマージン
\addtolength{\oddsidemargin}{-1in} % 
\setlength{\textwidth}{154mm} % B6 用
\setlength{\textheight}{108mm} % B6 用
\setlength{\headsep}{0mm} % 
\setlength{\headheight}{0mm} % 
\setlength{\topskip}{0mm} % 
\setlength{\parskip}{0pt} % 
\def\labelenumi{\theenumi、} % 箇条書きのフォーマット
\parindent = 0pt % 段落下げしない

 % B6 用テンプレート読み込み

\begin{document}
% begin header
%%%%% タイトルと作者 ここから %%%%%
\begin{minipage}[c]{0.7\hsize} % タイトルは上から 7 割
    \begin{center}
    % begin title
        {\LARGE
            茫洋の海 % タイトルを入れる
        }
        {\small 
            (昭和三十五年寮歌) % 年などを入れる
        }
    % end title
    \end{center}
\end{minipage}
\begin{minipage}[c]{0.3\hsize} % 作歌作曲は上から 3 割
    \begin{flushright} % 下寄せにする
        % begin name
        三浦清一郎君 作歌\\前野紀一君 作曲 % 作歌・作曲者
        % end name
    \end{flushright}
\end{minipage}
%%%%% タイトルと作者 ここまで %%%%%
% (1,2,3 了あり)
% end header

% begin body
\vspace{1.5em} % タイトル, 作者と歌詞の間に隙間を設ける
\newcommand{\linespace}{0.5em} % 行間の設定
\newcommand{\blocksize}{0.5\hsize} % 段組間の設定
%%%%% 歌詞 ここから %%%%%
% begin lilycs
\begin{enumerate} % 番号の箇条書き ここから
    \begin{minipage}[c]{\blocksize}
    
        \vspace{\linespace}
        \item
        % 1.
        \ruby{茫洋}{}の\ruby{海}{}に\ruby{憧}{}れ\\
        \ruby{峻険}{}の\ruby{峰}{}を\ruby{慕}{}いて\\
        \ruby{北国}{}の\ruby{大地}{}に\ruby{旅行}{}けば\\
        \ruby{溢}{}れ\ruby{満}{}つ\ruby{夢若}{}さ\\
        \ruby{果}{}てしなく\ruby{広}{}ごれる\ruby{地平線}{}
        
        \vspace{\linespace}
        \item
        % 2.
        \ruby{曇}{}りなき\ruby{心求}{}め\\
        \ruby{厳}{}しかる\ruby{努}{}めの\ruby{道}{}に\\
        \ruby{真}{}なる\ruby{美}{}を\ruby{探}{}らんと\\
        \ruby{人}{}の\ruby{世}{}の\ruby{旅}{}にして\\
        \ruby{結}{}ばれし\ruby{二年}{}の\ruby{宿}{}なれや
        
        \vspace{\linespace}
        \item
        % 3.
        \ruby{移}{}り\ruby{行}{}く\ruby{時}{}にはあれど\\
        \ruby{涙}{}して\ruby{誓}{}いし\ruby{言葉}{}\\
        \ruby{尊}{}しや\ruby{若}{}き\ruby{日}{}の\ruby{夢}{}\\
        \ruby{春秋}{}の\ruby{十年}{}の\ruby{後}{}に\\
        \ruby{思}{}い\ruby{出声}{}もなく\ruby{偲}{}ばんや
    
    \end{minipage}
\end{enumerate} % 番号の箇条書き ここまで
% end lilycs
%%%%% 歌詞 ここまで %%%%%
% end body

\end{document}

\documentclass[10pt,b5j]{tarticle} % B6 縦書き
% \documentclass[10pt,b5j]{tarticle} % B6 縦書き
\AtBeginDvi{\special{papersize=128mm,182mm}} % B6 用用紙サイズ
\usepackage{otf} % Unicode で字を入力するのに必要なパッケージ
\usepackage[size=b6j]{bxpapersize} % B6 用紙サイズを指定
\usepackage[dvipdfmx]{graphicx} % 画像を挿入するためのパッケージ
\usepackage[dvipdfmx]{color} % 色をつけるためのパッケージ
\usepackage{pxrubrica} % ルビを振るためのパッケージ
\usepackage{multicol} % 複数段組を作るためのパッケージ
\setlength{\topmargin}{14mm} % 上下方向のマージン
\addtolength{\topmargin}{-1in} % 
\setlength{\oddsidemargin}{11mm} % 左右方向のマージン
\addtolength{\oddsidemargin}{-1in} % 
\setlength{\textwidth}{154mm} % B6 用
\setlength{\textheight}{108mm} % B6 用
\setlength{\headsep}{0mm} % 
\setlength{\headheight}{0mm} % 
\setlength{\topskip}{0mm} % 
\setlength{\parskip}{0pt} % 
\def\labelenumi{\theenumi、} % 箇条書きのフォーマット
\parindent = 0pt % 段落下げしない

 % B6 用テンプレート読み込み

\begin{document}
% begin header
%%%%% タイトルと作者 ここから %%%%%
\begin{minipage}[c]{0.7\hsize} % タイトルは上から 7 割
    \begin{center}
    % begin title
        {\LARGE
            沈黙の杜に % タイトルを入れる
        }
        {\small 
            (昭和六十年寮歌) % 年などを入れる
        }
    % end title
    \end{center}
\end{minipage}
\begin{minipage}[c]{0.3\hsize} % 作歌作曲は上から 3 割
    \begin{flushright} % 下寄せにする
        % begin name
        角田勤君 作歌\\佐々木徹也君 作曲 % 作歌・作曲者
        % end name
    \end{flushright}
\end{minipage}
%%%%% タイトルと作者 ここまで %%%%%
% (1,2,3,4,5 了あり)
% end header

% begin length
\vspace{1.5em} % タイトル, 作者と歌詞の間に隙間を設ける
\newcommand{\linespace}{0.5em} % 行間の設定
\newcommand{\blocksize}{0.5\hsize} % 段組間の設定
\newcommand{\itemmargin}{6em} % 曲番の位置調整の長さ
% end length
% begin body
%%%%% 歌詞 ここから %%%%%
\begin{enumerate} % 番号の箇条書き ここから
    \setlength{\itemindent}{\itemmargin} % 曲番の位置調整
    \begin{minipage}[c]{\blocksize}
    
        \vspace{\linespace}
        \item~\\
        % 1.
        \ruby{沈黙}{}の\ruby{杜}{}に\ruby{春来告}{}げる\\
        \ruby{芳香馨}{}し\ruby{辛夷}{}の\ruby{花}{}よ\\
        \ruby{純白}{}き\ruby{残雪未}{}だ\ruby{消}{}えやらず\\
        \ruby{永}{}き\ruby{寒冬偲}{}ばるる\ruby{哉}{}\\
        \ruby{郷愁胸}{}に\ruby{充満}{}つるとも\\
        されど\ruby{恵迪此処}{}に\ruby{在}{}り
        
        \vspace{\linespace}
        \item~\\
        % 2.
        \ruby{水恋鳥}{}の\ruby{哀}{}しき\ruby{聲}{}に\\
        \ruby{我故知}{}らず\ruby{涙流}{}しぬ\\
        \ruby{短}{}き\ruby{夏}{}と\ruby{認識}{}りはすれども\\
        \ruby{樹樹色}{}づきてはや\ruby{盛夏逝}{}きぬ\\
        \ruby{哀愁胸}{}に\ruby{充満}{}つるとも\\
        されど\ruby{憧憬恵迪}{}に\ruby{在}{}り
        
        \vspace{\linespace}
        \item~\\
        % 3.
        \ruby{紅雲流}{}るる\ruby{黄昏}{}どきに\\
        \ruby{夕細道}{}は\ruby{幽}{}か\ruby{続}{}きて\\
        \ruby{何望}{}むなく\ruby{彷徨}{}ひゆける\\
        この\ruby{現身}{}を\ruby{悲哀}{}しみにけり\\
        \ruby{愁心胸}{}に\ruby{充満}{}つるとも\\
        されど\ruby{青春恵迪}{}に\ruby{在}{}り
        
        \vspace{\linespace}
        \item~\\
        % 4.
        \ruby{雪舞}{}ひ\ruby{踊}{}る\ruby{白銀}{}の\ruby{世}{}よ\\
        \ruby{天指}{}す\ruby{枝柯}{}に\ruby{樹氷咲}{}く\\
        \ruby{数多群}{}なす\ruby{星座}{}の\ruby{中}{}に\\
        \ruby{我}{}に\ruby{向}{}かいて\ruby{天狼星光}{}る\\
        \ruby{寂寥胸}{}に\ruby{充満}{}つるとも\\
        されど\ruby{経営恵迪}{}に\ruby{在}{}り
        
        \vspace{\linespace}
        \item~\\
        % 5.
        \ruby{弛}{}むことなく\ruby{唯時}{}は\ruby{逝}{}き\\
        \ruby{生}{}きとし\ruby{生}{}けるものは\ruby{去}{}りゆく\\
        \ruby{其}{}は\ruby{人}{}の\ruby{世}{}の\ruby{眞理}{}なれども\\
        \ruby{限}{}れる\ruby{生}{}を\ruby{燃}{}やし\ruby{尽}{}くさむ\\
        \ruby{追憶胸}{}に\ruby{充満}{}つるとも\\
        されど\ruby{恵迪永遠}{}に\ruby{在}{}れ
    
    \end{minipage}
\end{enumerate} % 番号の箇条書き ここまで
%%%%% 歌詞 ここまで %%%%%
% end body

\end{document}

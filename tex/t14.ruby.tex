\documentclass[10pt,b5j]{tarticle} % B6 縦書き
% \documentclass[10pt,b5j]{tarticle} % B6 縦書き
\AtBeginDvi{\special{papersize=128mm,182mm}} % B6 用用紙サイズ
\usepackage{otf} % Unicode で字を入力するのに必要なパッケージ
\usepackage[size=b6j]{bxpapersize} % B6 用紙サイズを指定
\usepackage[dvipdfmx]{graphicx} % 画像を挿入するためのパッケージ
\usepackage[dvipdfmx]{color} % 色をつけるためのパッケージ
\usepackage{pxrubrica} % ルビを振るためのパッケージ
\usepackage{multicol} % 複数段組を作るためのパッケージ
\setlength{\topmargin}{14mm} % 上下方向のマージン
\addtolength{\topmargin}{-1in} % 
\setlength{\oddsidemargin}{11mm} % 左右方向のマージン
\addtolength{\oddsidemargin}{-1in} % 
\setlength{\textwidth}{154mm} % B6 用
\setlength{\textheight}{108mm} % B6 用
\setlength{\headsep}{0mm} % 
\setlength{\headheight}{0mm} % 
\setlength{\topskip}{0mm} % 
\setlength{\parskip}{0pt} % 
\def\labelenumi{\theenumi、} % 箇条書きのフォーマット
\parindent = 0pt % 段落下げしない

 % B6 用テンプレート読み込み

\begin{document}
% begin header
%%%%% タイトルと作者 ここから %%%%%
\begin{minipage}[c]{0.7\hsize} % タイトルは上から 7 割
    \begin{center}
    % begin title
        {\LARGE
            敝れし衣 % タイトルを入れる
        }
        {\small 
            (大正十四年寮歌) % 年などを入れる
        }
    % end title
    \end{center}
\end{minipage}
\begin{minipage}[c]{0.3\hsize} % 作歌作曲は上から 3 割
    \begin{flushright} % 下寄せにする
        % begin name
        外山徳次郎君 作歌\\三溝清美君 作曲 % 作歌・作曲者
        % end name
    \end{flushright}
\end{minipage}
%%%%% タイトルと作者 ここまで %%%%%
% (1,2,3,4,5 了あり)
% end header

% begin length
\vspace{1.5em} % タイトル, 作者と歌詞の間に隙間を設ける
\newcommand{\linespace}{0.5em} % 行間の設定
\newcommand{\blocksize}{0.5\hsize} % 段組間の設定
\newcommand{\itemmargin}{3em} % 曲番の位置調整の長さ
% end length
% begin body
%%%%% 歌詞 ここから %%%%%
\begin{enumerate} % 番号の箇条書き ここから
    \setlength{\itemindent}{\itemmargin} % 曲番の位置調整
    \begin{minipage}[c]{\blocksize}
    
        \vspace{\linespace}
        \item~\\
        % 1.
        \ruby{敝}{}れし\ruby{衣}{}の\ruby{袖}{}に\ruby{散}{}る\\
        \ruby{不香}{}の\ruby{花}{}の\ruby{小夜嵐}{}\\
        \ruby{淋}{}しく\ruby{強}{}く\ruby{生}{}きぬ\ruby{可}{}く\\
        \ruby{手稲}{}の\ruby{峯}{}に\ruby{響}{}くかな
        
    \end{minipage}
    \begin{minipage}[c]{\blocksize}
        
        \vspace{\linespace}
        \item~\\
        % 2.
        \ruby{送}{}る\ruby{梅花}{}の\ruby{芳}{}せに\\
        \ruby{熱腸}{}しぼる\ruby{杜鵑}{}\\
        \ruby{誘}{}ふ\ruby{春風恨}{}みては\\
        \ruby{散}{}るも\ruby{惜}{}しまぬ\ruby{山桜}{}
        
    \end{minipage}
    \begin{minipage}[c]{\blocksize}
        
        \vspace{\linespace}
        \item~\\
        % 3.
        きのふぞ\ruby{移}{}る\ruby{秋風}{}に\\
        \ruby{草木悲歌}{}を\ruby{奏}{}ひつつ\\
        \ruby{月}{}の\ruby{面}{}ゆく\ruby{鳥}{}の\ruby{影}{}\\
        \ruby{故山}{}の\ruby{空}{}に\ruby{微}{}み\ruby{行}{}く
        
    \end{minipage}
    \begin{minipage}[c]{\blocksize}
        
        \vspace{\linespace}
        \item~\\
        % 4.
        \ruby{駄鞭荒野}{}に\ruby{打}{}ふりて\\
        \ruby{赴}{}くや\ruby{皇土}{}の\ruby{城}{}の\ruby{外}{}\\
        \ruby{青山我}{}が\ruby{有}{}に\ruby{帰}{}し\\
        \ruby{緑水我}{}を\ruby{弔}{}はん
        
    \end{minipage}
    \begin{minipage}[c]{\blocksize}
        
        \vspace{\linespace}
        \item~\\
        % 5.
        \ruby{国}{}に\ruby{誓}{}ひし\ruby{丈夫}{}の\\
        \ruby{夢中原}{}にさまよひて\\
        \ruby{仰}{}ぐみ\ruby{空}{}にまたたける\\
        \ruby{北極星}{}のかげ\ruby{清}{}し
    
    \end{minipage}
\end{enumerate} % 番号の箇条書き ここまで
%%%%% 歌詞 ここまで %%%%%
% end body

\end{document}

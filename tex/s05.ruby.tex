\documentclass[10pt,b5j]{tarticle} % B6 縦書き
% \documentclass[10pt,b5j]{tarticle} % B6 縦書き
\AtBeginDvi{\special{papersize=128mm,182mm}} % B6 用用紙サイズ
\usepackage{otf} % Unicode で字を入力するのに必要なパッケージ
\usepackage[size=b6j]{bxpapersize} % B6 用紙サイズを指定
\usepackage[dvipdfmx]{graphicx} % 画像を挿入するためのパッケージ
\usepackage[dvipdfmx]{color} % 色をつけるためのパッケージ
\usepackage{pxrubrica} % ルビを振るためのパッケージ
\usepackage{multicol} % 複数段組を作るためのパッケージ
\setlength{\topmargin}{14mm} % 上下方向のマージン
\addtolength{\topmargin}{-1in} % 
\setlength{\oddsidemargin}{11mm} % 左右方向のマージン
\addtolength{\oddsidemargin}{-1in} % 
\setlength{\textwidth}{154mm} % B6 用
\setlength{\textheight}{108mm} % B6 用
\setlength{\headsep}{0mm} % 
\setlength{\headheight}{0mm} % 
\setlength{\topskip}{0mm} % 
\setlength{\parskip}{0pt} % 
\def\labelenumi{\theenumi、} % 箇条書きのフォーマット
\parindent = 0pt % 段落下げしない

 % B6 用テンプレート読み込み

\begin{document}
% begin header
%%%%% タイトルと作者 ここから %%%%%
\begin{minipage}[c]{0.7\hsize} % タイトルは上から 7 割
    \begin{center}
    % begin title
        {\LARGE
            嗚呼青春の % タイトルを入れる
        }
        {\small 
            (昭和五年寮歌) % 年などを入れる
        }
    % end title
    \end{center}
\end{minipage}
\begin{minipage}[c]{0.3\hsize} % 作歌作曲は上から 3 割
    \begin{flushright} % 下寄せにする
        % begin name
        児山信蔵君 作歌\\有村徹君 作曲 % 作歌・作曲者
        % end name
    \end{flushright}
\end{minipage}
%%%%% タイトルと作者 ここまで %%%%%
% (1,3,5 了あり)
% end header

% begin length
\vspace{1.5em} % タイトル, 作者と歌詞の間に隙間を設ける
\newcommand{\linespace}{0.5em} % 行間の設定
\newcommand{\blocksize}{0.5\hsize} % 段組間の設定
\newcommand{\itemmargin}{3em} % 曲番の位置調整の長さ
% end length
% begin body
%%%%% 歌詞 ここから %%%%%
\begin{enumerate} % 番号の箇条書き ここから
    \setlength{\itemindent}{\itemmargin} % 曲番の位置調整
    \begin{minipage}[c]{\blocksize}
    
        \vspace{\linespace}
        \item~\\
        % 1.
        \ruby{嗚呼青春}{}の\ruby{夢高}{}く\\
        \ruby{理想}{}のあとに\ruby{憧憬}{}れて\\
        \ruby{楡}{}の\ruby{花散}{}る\ruby{学都}{}にぞ\\
        \ruby{啓示}{}を\ruby{求}{}む\ruby{若人}{}は\\
        \ruby{綺花}{}を\ruby{流}{}して\ruby{逝}{}く\ruby{水}{}に\\
        \ruby{十九}{}の\ruby{春}{}を\ruby{嘆}{}くなり
        
    \end{minipage}
    \begin{minipage}[c]{\blocksize}
        
        \vspace{\linespace}
        \item~\\
        % 2.
        \ruby{牧場}{}の\ruby{緑草踏}{}みしだき\\
        \ruby{栗毛}{}の\ruby{駒}{}に\ruby{鞍置}{}きて\\
        うち\ruby{振}{}る\ruby{鞭}{}の\ruby{音}{}も\ruby{高}{}く\\
        \ruby{希望}{}の\ruby{大空}{}を\ruby{朗}{}らかに\\
        \ruby{寮歌}{}を\ruby{歌}{}ひつ\ruby{眺}{}むれば\\
        \ruby{白雲流}{}れゆく\ruby{手稲山静}{}か
        
    \end{minipage}
    \begin{minipage}[c]{\blocksize}
        
        \vspace{\linespace}
        \item~\\
        % 3.
        \ruby{学堂}{}の\ruby{古鐘}{}の\ruby{沈}{}みゆき\\
        \ruby{楡陵}{}の\ruby{蒼空}{}に\ruby{銀月冴}{}えて\\
        \ruby{羊}{}の\ruby{群}{}れの\ruby{片影}{}もなし\\
        \ruby{沈黙}{}の\ruby{原始林}{}に\ruby{散}{}りしける\\
        \ruby{落葉踏}{}みゆく\ruby{雄}{}き\ruby{子}{}は\\
        \ruby{三年}{}の\ruby{絢夢}{}に\ruby{涙}{}する
        
    \end{minipage}
    \begin{minipage}[c]{\blocksize}
        
        \vspace{\linespace}
        \item~\\
        % 4.
        \ruby{疎林}{}のほとり\ruby{夕陽}{}は\ruby{落}{}ちて\\
        \ruby{凩}{}さへも\ruby{絶}{}えし\ruby{真夜}{}に\\
        \ruby{涯}{}なく\ruby{白}{}き\ruby{石狩}{}の\\
        \ruby{銀雪}{}に\ruby{連}{}なる\ruby{曠野}{}の\ruby{静寂}{}\\
        \ruby{震}{}はせ\ruby{乍}{}ら\ruby{橇唄}{}は\\
        \ruby{神秘}{}の\ruby{闇}{}を\ruby{縫}{}ひてゆく
        
    \end{minipage}
    \begin{minipage}[c]{\blocksize}
        
        \vspace{\linespace}
        \item~\\
        % 5.
        \ruby{北斗}{}は\ruby{遠}{}く\ruby{七星清}{}し\\
        「\ruby{妄執}{}」の\ruby{現世}{}を\ruby{見下}{}して\\
        \ruby{真実一路}{}の\ruby{迪恵}{}ぬ\\
        「\ruby{意気}{}」と「\ruby{血潮}{}」に\ruby{生}{}くる\ruby{子}{}の\\
        \ruby{瞳}{}に\ruby{燃}{}ゆる\ruby{紅焰}{}は\\
        \ruby{永遠}{}なる\ruby{生命}{}の\ruby{証}{}なり
    
    \end{minipage}
\end{enumerate} % 番号の箇条書き ここまで
%%%%% 歌詞 ここまで %%%%%
% end body

\end{document}

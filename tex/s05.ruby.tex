\documentclass[10pt,b5j]{tarticle} % B6 縦書き
% \documentclass[10pt,b5j]{tarticle} % B6 縦書き
\AtBeginDvi{\special{papersize=128mm,182mm}} % B6 用用紙サイズ
\usepackage{otf} % Unicode で字を入力するのに必要なパッケージ
\usepackage[size=b6j]{bxpapersize} % B6 用紙サイズを指定
\usepackage[dvipdfmx]{graphicx} % 画像を挿入するためのパッケージ
\usepackage[dvipdfmx]{color} % 色をつけるためのパッケージ
\usepackage{pxrubrica} % ルビを振るためのパッケージ
\usepackage{multicol} % 複数段組を作るためのパッケージ
\setlength{\topmargin}{14mm} % 上下方向のマージン
\addtolength{\topmargin}{-1in} % 
\setlength{\oddsidemargin}{11mm} % 左右方向のマージン
\addtolength{\oddsidemargin}{-1in} % 
\setlength{\textwidth}{154mm} % B6 用
\setlength{\textheight}{108mm} % B6 用
\setlength{\headsep}{0mm} % 
\setlength{\headheight}{0mm} % 
\setlength{\topskip}{0mm} % 
\setlength{\parskip}{0pt} % 
\def\labelenumi{\theenumi、} % 箇条書きのフォーマット
\parindent = 0pt % 段落下げしない

 % B6 用テンプレート読み込み

\begin{document}
% begin header
%%%%% タイトルと作者 ここから %%%%%
\begin{minipage}[c]{0.7\hsize} % タイトルは上から 7 割
    \begin{center}
    % begin title
        {\LARGE
            嗚呼青春の % タイトルを入れる
        }
        {\small 
            (昭和五年寮歌) % 年などを入れる
        }
    % end title
    \end{center}
\end{minipage}
\begin{minipage}[c]{0.3\hsize} % 作歌作曲は上から 3 割
    \begin{flushright} % 下寄せにする
        % begin name
        児山信蔵君 作歌\\有村徹君 作曲 % 作歌・作曲者
        % end name
    \end{flushright}
\end{minipage}
%%%%% タイトルと作者 ここまで %%%%%
% (1,3,5 了あり)
% end header

% begin length
\vspace{1.5em} % タイトル, 作者と歌詞の間に隙間を設ける
\newcommand{\linespace}{0.5em} % 行間の設定
\newcommand{\blocksize}{0.5\hsize} % 段組間の設定
\newcommand{\itemmargin}{3em} % 曲番の位置調整の長さ
% end length
% begin body
%%%%% 歌詞 ここから %%%%%
\begin{enumerate} % 番号の箇条書き ここから
    \setlength{\itemindent}{\itemmargin} % 曲番の位置調整
    \begin{minipage}[c]{\blocksize}
    
        \vspace{\linespace}
        \item~\\
        % 1.
        \ruby{嗚呼}{ああ}\ruby{青春}{せいしゅん}の\ruby{夢}{ゆめ}\ruby{高}{たか}く\\
        \ruby{理想}{りそう}のあとに\ruby{憧憬}{どうけい}れて\\
        \ruby{楡}{にれ}の\ruby{花}{はな}\ruby{散}{ち}る\ruby{学都}{がくと}にぞ\\
        \ruby{啓示}{けいじ}を\ruby{求}{もと}む\ruby{若人}{わこうど}は\\
        \ruby{綺}{あやぎぬ}\ruby{花}{はな}を\ruby{流}{なが}して\ruby{逝}{ゆ}く\ruby{水}{みず}に\\
        \ruby{十}{じゅう}\ruby{九}{きゅう}の\ruby{春}{はる}を\ruby{嘆}{なげ}くなり
        
    \end{minipage}
    \begin{minipage}[c]{\blocksize}
        
        \vspace{\linespace}
        \item~\\
        % 2.
        \ruby{牧場}{ぼくじょう}の\ruby{緑}{みどり}\ruby{草}{そう}\ruby{踏}{ふ}みしだき\\
        \ruby{栗毛}{くりげ}の\ruby{駒}{こま}に\ruby{鞍}{くら}\ruby{置}{お}きて\\
        うち\ruby{振}{ふ}る\ruby{鞭}{むち}の\ruby{音}{おと}も\ruby{高}{たか}く\\
        \ruby{希望}{きぼう}の\ruby{大空}{おおぞら}を\ruby{朗}{ほが}らかに\\
        \ruby{寮歌}{りょうか}を\ruby{歌}{うた}ひつ\ruby{眺}{}むれば\\
        \ruby{白雲}{しらくも}\ruby{流}{なが}れゆく\ruby{手稲山}{ていねやま}\ruby{静}{しず}か
        
    \end{minipage}
    \begin{minipage}[c]{\blocksize}
        
        \vspace{\linespace}
        \item~\\
        % 3.
        \ruby{学堂}{がくどう}の\ruby{古}{こ}\ruby{鐘}{かね}の\ruby{沈}{しず}みゆき\\
        \ruby{楡}{にれ}\ruby{陵}{りょう}の\ruby{蒼空}{そうくう}に\ruby{銀月}{ぎんげつ}\ruby{冴}{さ}えて\\
        \ruby{羊}{ひつじ}の\ruby{群}{む}れの\ruby{片影}{へんえい}もなし\\
        \ruby{沈黙}{ちんもく}の\ruby{原始}{げんし}\ruby{林}{りん}に\ruby{散}{ち}りしける\\
        \ruby{落葉}{らくよう}\ruby{踏}{ふ}みゆく\ruby{雄}{ゆう}き\ruby{子}{こ}は\\
        \ruby{三}{さん}\ruby{年}{ねん}の\ruby{絢}{あや}\ruby{夢}{ゆめ}に\ruby{涙}{なみだ}する
        
    \end{minipage}
    \begin{minipage}[c]{\blocksize}
        
        \vspace{\linespace}
        \item~\\
        % 4.
        \ruby{疎林}{そりん}のほとり\ruby{夕陽}{ゆうひ}は\ruby{落}{お}ちて\\
        \ruby{凩}{こがらし}さへも\ruby{絶}{た}えし\ruby{真夜}{まよ}に\\
        \ruby{涯}{}なく\ruby{白}{しろ}き\ruby{石狩}{いしかり}の\\
        \ruby{銀}{ぎん}\ruby{雪}{ゆき}に\ruby{連}{つら}なる\ruby{曠野}{あらの}の\ruby{静寂}{せいじゃく}\\
        \ruby{震}{しん}はせ\ruby{乍}{}ら\ruby{橇}{そり}\ruby{唄}{うた}は\\
        \ruby{神秘}{しんぴ}の\ruby{闇}{やみ}を\ruby{縫}{ぬい}ひてゆく
        
    \end{minipage}
    \begin{minipage}[c]{\blocksize}
        
        \vspace{\linespace}
        \item~\\
        % 5.
        \ruby{北斗}{ほくと}は\ruby{遠}{とお}く\ruby{七}{なな}\ruby{星}{ほし}\ruby{清}{きよ}し\\
        「\ruby{妄執}{もうしゅう}」の\ruby{現世}{げんせい}を\ruby{見下}{みくだ}して\\
        \ruby{真実}{しんじつ}\ruby{一路}{いちろ}の\ruby{迪}{すすむ}\ruby{恵}{めぐみ}ぬ\\
        「\ruby{意気}{いき}」と「\ruby{血潮}{ちしお}」に\ruby{生}{なま}くる\ruby{子}{こ}の\\
        \ruby{瞳}{ひとみ}に\ruby{燃}{}ゆる\ruby{紅}{べに}\ruby{焰}{}は\\
        \ruby{永遠}{えいえん}なる\ruby{生命}{せいめい}の\ruby{証}{あかし}なり
    
    \end{minipage}
\end{enumerate} % 番号の箇条書き ここまで
%%%%% 歌詞 ここまで %%%%%
% end body

\end{document}

\documentclass[10pt,b5j]{tarticle} % B6 縦書き
% \documentclass[10pt,b5j]{tarticle} % B6 縦書き
\AtBeginDvi{\special{papersize=128mm,182mm}} % B6 用用紙サイズ
\usepackage{otf} % Unicode で字を入力するのに必要なパッケージ
\usepackage[size=b6j]{bxpapersize} % B6 用紙サイズを指定
\usepackage[dvipdfmx]{graphicx} % 画像を挿入するためのパッケージ
\usepackage[dvipdfmx]{color} % 色をつけるためのパッケージ
\usepackage{pxrubrica} % ルビを振るためのパッケージ
\usepackage{multicol} % 複数段組を作るためのパッケージ
\setlength{\topmargin}{14mm} % 上下方向のマージン
\addtolength{\topmargin}{-1in} % 
\setlength{\oddsidemargin}{11mm} % 左右方向のマージン
\addtolength{\oddsidemargin}{-1in} % 
\setlength{\textwidth}{154mm} % B6 用
\setlength{\textheight}{108mm} % B6 用
\setlength{\headsep}{0mm} % 
\setlength{\headheight}{0mm} % 
\setlength{\topskip}{0mm} % 
\setlength{\parskip}{0pt} % 
\def\labelenumi{\theenumi、} % 箇条書きのフォーマット
\parindent = 0pt % 段落下げしない

 % B6 用テンプレート読み込み

\begin{document}
% begin header
%%%%% タイトルと作者 ここから %%%%%
\begin{minipage}[c]{0.7\hsize} % タイトルは上から 7 割
    \begin{center}
    % begin title
        {\LARGE
            弥生の空に % タイトルを入れる
        }
        {\small 
            (昭和十五年寮歌) % 年などを入れる
        }
    % end title
    \end{center}
\end{minipage}
\begin{minipage}[c]{0.3\hsize} % 作歌作曲は上から 3 割
    \begin{flushright} % 下寄せにする
        % begin name
        大井徹夫君 作歌・作曲 % 作歌・作曲者
        % end name
    \end{flushright}
\end{minipage}
%%%%% タイトルと作者 ここまで %%%%%
% (1,2,3 了あり)
% end header

% begin length
\vspace{1.5em} % タイトル, 作者と歌詞の間に隙間を設ける
\newcommand{\linespace}{0.5em} % 行間の設定
\newcommand{\blocksize}{0.5\hsize} % 段組間の設定
\newcommand{\itemmargin}{6em} % 曲番の位置調整の長さ
% end length
% begin body
%%%%% 歌詞 ここから %%%%%
\begin{enumerate} % 番号の箇条書き ここから
    \setlength{\itemindent}{\itemmargin} % 曲番の位置調整
    \begin{minipage}[c]{\blocksize}
    
        \vspace{\linespace}
        \item~\\
        % 1.
        \ruby{弥生}{}の\ruby{空}{}に\ruby{消}{}え\ruby{残}{}る\\
        \ruby{霞}{}に\ruby{春}{}の\ruby{絢夢闌}{}けて\\
        \ruby{首途}{}を\ruby{祝}{}ふ\ruby{花吹雪}{}\\
        \ruby{友情}{}の\ruby{盃}{}を\ruby{交}{}しつつ\\
        \ruby{北斗}{}の\ruby{光身}{}に\ruby{享}{}けて\\
        \ruby{仰}{}ぐ\ruby{健児}{}の\ruby{影清}{}し
        
        \vspace{\linespace}
        \item~\\
        % 2.
        \ruby{手稲}{}の\ruby{山}{}に\ruby{陽}{}は\ruby{落}{}ちて\\
        \ruby{広}{}き\ruby{蒼空}{}の\ruby{茜雲}{}\\
        「\ruby{我立}{}たずんば」の\ruby{意気}{}あれど\\
        \ruby{昇天}{}の\ruby{機}{}を\ruby{小百合咲}{}く\\
        \ruby{静}{}けき\ruby{故郷}{}に\ruby{憩}{}して\\
        \ruby{暫}{}し\ruby{臥竜}{}の\ruby{夢}{}に\ruby{見}{}む
        
        \vspace{\linespace}
        \item~\\
        % 3.
        \ruby{春雨煙}{}る\ruby{並木路}{}に\\
        \ruby{輪廻}{}の\ruby{相偲}{}びては\\
        \ruby{露置}{}く\ruby{花}{}を\ruby{愛}{}しみて\\
        \ruby{遠}{}き\ruby{思索}{}に\ruby{逍遙}{}へば\\
        \ruby{緑}{}の\ruby{牧場眼}{}に\ruby{著}{}き\\
        \ruby{野路}{}は\ruby{果}{}てなく\ruby{黄昏}{}れぬ
        
        \vspace{\linespace}
        \item~\\
        % 4.
        \ruby{究理}{}の\ruby{道}{}は\ruby{遠}{}くとも\\
        \ruby{研磨}{}の\ruby{窓}{}に\ruby{月匂}{}ふ\\
        \ruby{白魔曠野}{}に\ruby{狂}{}ふとも\\
        \ruby{明日}{}は\ruby{希望}{}の\ruby{太陽笑}{}まずや\\
        \ruby{正義}{}の\ruby{大道濶歩}{}する\\
        \ruby{熱血男児}{}ここにあり
        
        \vspace{\linespace}
        \item~\\
        % 5.
        \ruby{光}{}かそけき\ruby{原始林蔭}{}の\\
        \ruby{月}{}に\ruby{散}{}り\ruby{布}{}く\ruby{花蓆}{}\\
        エルムの\ruby{精}{}も\ruby{踊}{}るてふ\\
        \ruby{記念祭}{}の\ruby{歌}{}は\ruby{谺}{}して\\
        \ruby{永世}{}を\ruby{寿}{}ぐ\ruby{篝火}{}に\\
        \ruby{歓喜}{}の\ruby{夜}{}は\ruby{更}{}けゆきぬ
        
        \vspace{\linespace}
        \item~\\
        % 6.
        \ruby{不壊}{}の\ruby{智玉}{}を\ruby{育}{}みて\\
        \ruby{恵迪}{}ここに\ruby{早三年}{}\\
        \ruby{静寂}{}の\ruby{楡鐘}{}に\ruby{眼}{}をやれば\\
        \ruby{見}{}よ\ruby{東雲}{}は\ruby{輝}{}けり\\
        いざ\ruby{船出}{}せむ\ruby{波濤越}{}えて\\
        \ruby{嗚呼人生}{}の\ruby{朝}{}ぼらけ
    
    \end{minipage}
\end{enumerate} % 番号の箇条書き ここまで
%%%%% 歌詞 ここまで %%%%%
% end body

\end{document}

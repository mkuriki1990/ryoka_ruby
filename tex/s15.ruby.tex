\documentclass[10pt,b5j]{tarticle} % B6 縦書き
% \documentclass[10pt,b5j]{tarticle} % B6 縦書き
\AtBeginDvi{\special{papersize=128mm,182mm}} % B6 用用紙サイズ
\usepackage{otf} % Unicode で字を入力するのに必要なパッケージ
\usepackage[size=b6j]{bxpapersize} % B6 用紙サイズを指定
\usepackage[dvipdfmx]{graphicx} % 画像を挿入するためのパッケージ
\usepackage[dvipdfmx]{color} % 色をつけるためのパッケージ
\usepackage{pxrubrica} % ルビを振るためのパッケージ
\usepackage{multicol} % 複数段組を作るためのパッケージ
\setlength{\topmargin}{14mm} % 上下方向のマージン
\addtolength{\topmargin}{-1in} % 
\setlength{\oddsidemargin}{11mm} % 左右方向のマージン
\addtolength{\oddsidemargin}{-1in} % 
\setlength{\textwidth}{154mm} % B6 用
\setlength{\textheight}{108mm} % B6 用
\setlength{\headsep}{0mm} % 
\setlength{\headheight}{0mm} % 
\setlength{\topskip}{0mm} % 
\setlength{\parskip}{0pt} % 
\def\labelenumi{\theenumi、} % 箇条書きのフォーマット
\parindent = 0pt % 段落下げしない

 % B6 用テンプレート読み込み

\begin{document}
% begin header
%%%%% タイトルと作者 ここから %%%%%
\begin{minipage}[c]{0.7\hsize} % タイトルは上から 7 割
    \begin{center}
    % begin title
        {\LARGE
            弥生の空に % タイトルを入れる
        }
        {\small 
            (昭和十五年寮歌) % 年などを入れる
        }
    % end title
    \end{center}
\end{minipage}
\begin{minipage}[c]{0.3\hsize} % 作歌作曲は上から 3 割
    \begin{flushright} % 下寄せにする
        % begin name
        大井徹夫君 作歌・作曲 % 作歌・作曲者
        % end name
    \end{flushright}
\end{minipage}
%%%%% タイトルと作者 ここまで %%%%%
% (1,2,3 了あり)
% end header

% begin length
\vspace{1.5em} % タイトル, 作者と歌詞の間に隙間を設ける
\newcommand{\linespace}{0.5em} % 行間の設定
\newcommand{\blocksize}{0.5\hsize} % 段組間の設定
\newcommand{\itemmargin}{3em} % 曲番の位置調整の長さ
% end length
% begin body
%%%%% 歌詞 ここから %%%%%
\begin{enumerate} % 番号の箇条書き ここから
    \setlength{\itemindent}{\itemmargin} % 曲番の位置調整
    \begin{minipage}[c]{\blocksize}
    
        \vspace{\linespace}
        \item~\\
        % 1.
        \ruby{弥生}{やよい}の\ruby{空}{そら}に\ruby{消}{き}え\ruby{残}{のこ}る\\
        \ruby{霞}{かすみ}に\ruby{春}{はる}の\ruby{絢}{あや}\ruby{夢}{ゆめ}\ruby{闌}{た}けて\\
        \ruby{首途}{しゅと}を\ruby{祝}{しゅく}ふ\ruby{花吹雪}{はなふぶき}\\
        \ruby{友情}{ゆうじょう}の\ruby{盃}{さかずき}を\ruby{交}{かわ}しつつ\\
        \ruby{北斗}{ほくと}の\ruby{光}{ひかり}\ruby{身}{み}に\ruby{享}{とおる}けて\\
        \ruby{仰}{あお}ぐ\ruby{健児}{けんじ}の\ruby{影}{かげ}\ruby{清}{きよ}し
        
    \end{minipage}
    \begin{minipage}[c]{\blocksize}
        
        \vspace{\linespace}
        \item~\\
        % 2.
        \ruby{手稲}{ていね}の\ruby{山}{やま}に\ruby{陽}{ひ}は\ruby{落}{お}ちて\\
        \ruby{広}{ひろ}き\ruby{蒼空}{そうくう}の\ruby{茜}{あかね}\ruby{雲}{くも}\\
        「\ruby{我}{わが}\ruby{立}{た}たずんば」の\ruby{意気}{いき}あれど\\
        \ruby{昇天}{しょうてん}の\ruby{機}{き}を\ruby{小}{しょう}\ruby{百}{ひゃく}\ruby{合}{ごう}\ruby{咲}{さ}く\\
        \ruby{静}{せい}けき\ruby{故郷}{こきょう}に\ruby{憩}{いこい}して\\
        \ruby{暫}{しば}し\ruby{臥竜}{がりょう}の\ruby{夢}{ゆめ}に\ruby{見}{み}む
        
    \end{minipage}
    \begin{minipage}[c]{\blocksize}
        
        \vspace{\linespace}
        \item~\\
        % 3.
        \ruby{春雨}{はるさめ}\ruby{煙}{けむ}る\ruby{並木}{なみき}\ruby{路}{ろ}に\\
        \ruby{輪廻}{りんね}の\ruby{相}{あい}\ruby{偲}{しの}びては\\
        \ruby{露}{つゆ}\ruby{置}{お}く\ruby{花}{はな}を\ruby{愛}{あい}しみて\\
        \ruby{遠}{とお}き\ruby{思索}{しさく}に\ruby{逍遙}{}へば\\
        \ruby{緑}{みどり}の\ruby{牧場}{ぼくじょう}\ruby{眼}{め}に\ruby{著}{しる}き\\
        \ruby{野路}{のじ}は\ruby{果}{は}てなく\ruby{黄}{き}\ruby{昏}{く}れぬ
        
    \end{minipage}
    \begin{minipage}[c]{\blocksize}
        
        \vspace{\linespace}
        \item~\\
        % 4.
        \ruby{究理}{きゅうり}の\ruby{道}{みち}は\ruby{遠}{とお}くとも\\
        \ruby{研磨}{けんま}の\ruby{窓}{まど}に\ruby{月}{つき}\ruby{匂}{におい}ふ\\
        \ruby{白魔}{はくま}\ruby{曠野}{あらの}に\ruby{狂}{きょう}ふとも\\
        \ruby{明日}{あした}は\ruby{希望}{きぼう}の\ruby{太陽}{たいよう}\ruby{笑}{え}まずや\\
        \ruby{正義}{せいぎ}の\ruby{大道}{だいどう}\ruby{濶歩}{}する\\
        \ruby{熱血}{ねっけつ}\ruby{男児}{だんじ}ここにあり
        
    \end{minipage}
    \begin{minipage}[c]{\blocksize}
        
        \vspace{\linespace}
        \item~\\
        % 5.
        \ruby{光}{ひかり}かそけき\ruby{原始}{げんし}\ruby{林}{りん}\ruby{蔭}{かげ}の\\
        \ruby{月}{つき}に\ruby{散}{ち}り\ruby{布}{し}く\ruby{花蓆}{はなむしろ}\\
        エルムの\ruby{精}{せい}も\ruby{踊}{おど}るてふ\\
        \ruby{記念}{きねん}\ruby{祭}{さい}の\ruby{歌}{うた}は\ruby{谺}{こだま}して\\
        \ruby{永世}{えいせい}を\ruby{寿}{ことほ}ぐ\ruby{篝火}{かがりび}に\\
        \ruby{歓喜}{かんき}の\ruby{夜}{よる}は\ruby{更}{ふ}けゆきぬ
        
    \end{minipage}
    \begin{minipage}[c]{\blocksize}
        
        \vspace{\linespace}
        \item~\\
        % 6.
        \ruby{不壊}{ふえ}の\ruby{智}{さとし}\ruby{玉}{だま}を\ruby{育}{はぐく}みて\\
        \ruby{恵}{めぐみ}\ruby{迪}{すすむ}ここに\ruby{早}{はや}\ruby{三}{さん}\ruby{年}{ねん}\\
        \ruby{静寂}{せいじゃく}の\ruby{楡}{にれ}\ruby{鐘}{かね}に\ruby{眼}{め}をやれば\\
        \ruby{見}{み}よ\ruby{東雲}{しののめ}は\ruby{輝}{てる}けり\\
        いざ\ruby{船出}{ふなで}せむ\ruby{波}{は}\ruby{濤}{}\ruby{越}{こ}えて\\
        \ruby{嗚呼}{ああ}\ruby{人生}{じんせい}の\ruby{朝}{あさ}ぼらけ
    
    \end{minipage}
\end{enumerate} % 番号の箇条書き ここまで
%%%%% 歌詞 ここまで %%%%%
% end body

\end{document}

\documentclass[10pt,b5j]{tarticle} % B6 縦書き
% \documentclass[10pt,b5j]{tarticle} % B6 縦書き
\AtBeginDvi{\special{papersize=128mm,182mm}} % B6 用用紙サイズ
\usepackage{otf} % Unicode で字を入力するのに必要なパッケージ
\usepackage[size=b6j]{bxpapersize} % B6 用紙サイズを指定
\usepackage[dvipdfmx]{graphicx} % 画像を挿入するためのパッケージ
\usepackage[dvipdfmx]{color} % 色をつけるためのパッケージ
\usepackage{pxrubrica} % ルビを振るためのパッケージ
\usepackage{multicol} % 複数段組を作るためのパッケージ
\setlength{\topmargin}{14mm} % 上下方向のマージン
\addtolength{\topmargin}{-1in} % 
\setlength{\oddsidemargin}{11mm} % 左右方向のマージン
\addtolength{\oddsidemargin}{-1in} % 
\setlength{\textwidth}{154mm} % B6 用
\setlength{\textheight}{108mm} % B6 用
\setlength{\headsep}{0mm} % 
\setlength{\headheight}{0mm} % 
\setlength{\topskip}{0mm} % 
\setlength{\parskip}{0pt} % 
\def\labelenumi{\theenumi、} % 箇条書きのフォーマット
\parindent = 0pt % 段落下げしない

 % B6 用テンプレート読み込み

\begin{document}
% begin header
%%%%% タイトルと作者 ここから %%%%%
\begin{minipage}[c]{0.7\hsize} % タイトルは上から 7 割
    \begin{center}
    % begin title
        {\LARGE
            寮生の道 % タイトルを入れる
        }
        {\small 
            (昭和58年寮歌) % 年などを入れる
        }
    % end title
    \end{center}
\end{minipage}
\begin{minipage}[c]{0.3\hsize} % 作歌作曲は上から 3 割
    \begin{flushright} % 下寄せにする
        % begin name
        泉進介君 作歌\\島倉朝雄君 作曲 % 作歌・作曲者
        % end name
    \end{flushright}
\end{minipage}
%%%%% タイトルと作者 ここまで %%%%%
% (春,夏,秋,冬,まとめ
% end header

% begin body
\vspace{1.5em} % タイトル, 作者と歌詞の間に隙間を設ける
\newcommand{\linespace}{0.5em} % 行間の設定
\newcommand{\blocksize}{0.5\hsize} % 段組間の設定
%%%%% 歌詞 ここから %%%%%
% begin lilycs
\begin{enumerate} % 番号の箇条書き ここから
    \begin{minipage}[c]{\blocksize}
    
        \vspace{\linespace}
        \item
        \ruby{凍}{}てつきし\ruby{氷}{}の\ruby{路}{}も\ruby{溶}{}け\ruby{始}{}め、\\
        \ruby{見}{}はるかす\ruby{山}{}に\ruby{白雪消}{}ゆる\ruby{頃}{}\\
        \ruby{集}{}い\ruby{来}{}し\ruby{百}{}と\ruby{四十}{}の\ruby{若人}{}は\\
        \ruby{故郷}{}も\ruby{親}{}も\ruby{銭}{}もなく\\
        \ruby{恃}{}むは\ruby{己}{}の\ruby{仁侠}{}ばかり\\
        \ruby{然}{}れども\ruby{新}{}たな\ruby{舎}{}りの\ruby{恵迪}{}は\\
        \ruby{五層六刃}{}の\ruby{白亜城}{}\\
        \ruby{夜}{}も\ruby{希望}{}の\ruby{灯}{}は\ruby{消}{}さず、\\
        \ruby{棲}{}むは\ruby{豪傑酒乱}{}の\ruby{徒}{}\\
        さあ\ruby{来}{}いさあ\ruby{来}{}い\ruby{恵迪}{}へ\\
        \ruby{北都}{}に\ruby{築}{}かん\ruby{我等}{}が\ruby{自治寮}{}
        
        \vspace{\linespace}
        \item
        % \ruby{春}{}(\ruby{四月}{}).
        ちょいとそこ\ruby{行}{}く\ruby{新入寮生}{}さん\\
        \ruby{明日}{}は\ruby{我身}{}か\ruby{知}{}らねども\\
        \ruby{大酒}{}くらって\ruby{逆噴射}{}\\
        これぞ\ruby{寮生}{}の\ruby{生}{}きる\ruby{道}{}
        
        \vspace{\linespace}
        \item
        % \ruby{夏}{}(\ruby{八月}{}).
        ちょいとそこ\ruby{行}{}く\ruby{寮生}{}さん\\
        \ruby{弊衣破帽}{}に\ruby{食糧難}{}\\
        \ruby{両親}{}の\ruby{顔}{}が\ruby{眼}{}に\ruby{浮}{}かぶ\\
        これぞ\ruby{寮生}{}の\ruby{生}{}きる\ruby{道}{}
        
        \vspace{\linespace}
        \item
        % \ruby{秋}{}(\ruby{十月}{}).
        ちょいとそこ\ruby{行}{}く\ruby{寮生}{}さん\\
        \ruby{尻}{}に\ruby{赤}{}フン\ruby{巻}{}きつけて\\
        \ruby{狂喜乱舞}{}す\ruby{交差点}{}\\
        これぞ\ruby{寮生}{}の\ruby{生}{}きる\ruby{道}{}
        
        \vspace{\linespace}
        \item
        % \ruby{冬}{}(\ruby{二月}{}).
        ちょいとそこ\ruby{行}{}く\ruby{寮生}{}さん\\
        ジャンプ\ruby{大会変態}{}か\\
        \ruby{花}{}の\ruby{女子大赤面}{}す\\
        これぞ\ruby{寮生}{}の\ruby{生}{}きる\ruby{道}{}
        
        \vspace{\linespace}
        \item
        % まとめ.
        ちょいとそこ\ruby{行}{}く\ruby{寮生}{}さん\\
        クラーク\ruby{精神胸}{}に\ruby{秘}{}め\\
        \ruby{天下}{}の\ruby{北大恵迪}{}でもつ\\
        これぞ\ruby{寮生}{}の\ruby{生}{}きる\ruby{道}{}
        (※
        前口上は島倉朝雄君の作による)

    
    \end{minipage}
\end{enumerate} % 番号の箇条書き ここまで
% end lilycs
%%%%% 歌詞 ここまで %%%%%
% end body

\end{document}

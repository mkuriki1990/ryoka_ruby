\documentclass[10pt,b5j]{tarticle} % B6 縦書き
% \documentclass[10pt,b5j]{tarticle} % B6 縦書き
\AtBeginDvi{\special{papersize=128mm,182mm}} % B6 用用紙サイズ
\usepackage{otf} % Unicode で字を入力するのに必要なパッケージ
\usepackage[size=b6j]{bxpapersize} % B6 用紙サイズを指定
\usepackage[dvipdfmx]{graphicx} % 画像を挿入するためのパッケージ
\usepackage[dvipdfmx]{color} % 色をつけるためのパッケージ
\usepackage{pxrubrica} % ルビを振るためのパッケージ
\usepackage{multicol} % 複数段組を作るためのパッケージ
\setlength{\topmargin}{14mm} % 上下方向のマージン
\addtolength{\topmargin}{-1in} % 
\setlength{\oddsidemargin}{11mm} % 左右方向のマージン
\addtolength{\oddsidemargin}{-1in} % 
\setlength{\textwidth}{154mm} % B6 用
\setlength{\textheight}{108mm} % B6 用
\setlength{\headsep}{0mm} % 
\setlength{\headheight}{0mm} % 
\setlength{\topskip}{0mm} % 
\setlength{\parskip}{0pt} % 
\def\labelenumi{\theenumi、} % 箇条書きのフォーマット
\parindent = 0pt % 段落下げしない

 % B6 用テンプレート読み込み

\begin{document}
% begin header
%%%%% タイトルと作者 ここから %%%%%
\begin{minipage}[c]{0.7\hsize} % タイトルは上から 7 割
    \begin{center}
    % begin title
        {\LARGE
            うす紅の % タイトルを入れる
        }
        {\small 
            (昭和五十四年寮歌) % 年などを入れる
        }
    % end title
    \end{center}
\end{minipage}
\begin{minipage}[c]{0.3\hsize} % 作歌作曲は上から 3 割
    \begin{flushright} % 下寄せにする
        % begin name
        鶴原文孝君 作歌\\高田和重君 作曲 % 作歌・作曲者
        % end name
    \end{flushright}
\end{minipage}
%%%%% タイトルと作者 ここまで %%%%%
% (1,2,3,4,5 繰り返しなし)
% end header

% begin length
\vspace{1.5em} % タイトル, 作者と歌詞の間に隙間を設ける
\newcommand{\linespace}{0.5em} % 行間の設定
\newcommand{\blocksize}{0.5\hsize} % 段組間の設定
\newcommand{\itemmargin}{3em} % 曲番の位置調整の長さ
% end length
% begin body
%%%%% 歌詞 ここから %%%%%
\begin{enumerate} % 番号の箇条書き ここから
    \setlength{\itemindent}{\itemmargin} % 曲番の位置調整
    \begin{minipage}[c]{\blocksize}
    
        \vspace{\linespace}
        \item~\\
        % 1.
        うす\ruby{紅}{}の\ruby{秋}{}ゆうぐれに\\
        \ruby{滅}{}びの\ruby{風}{}は\ruby{吹}{}き\ruby{荒}{}ぶ\\
        \ruby{斜陽}{}かげ\ruby{射}{}す\ruby{日}{}に\ruby{移}{}ろいて\\
        \ruby{傾}{}く\ruby{姿痛}{}ましく\\
        \ruby{我}{}が\ruby{胸}{}に\ruby{満}{}つ\ruby{過}{}にし\ruby{日}{}の\ruby{映}{}え\\
        \ruby{懐}{}いは\ruby{恵迪}{}と\ruby{共}{}に
        
    \end{minipage}
    \begin{minipage}[c]{\blocksize}
        
        \vspace{\linespace}
        \item~\\
        % 2.
        うす\ruby{紫}{}の\ruby{冬}{}あけどきに\\
        \ruby{透}{}みわたる\ruby{風底凍}{}る\\
        もの\ruby{音絶}{}えて\ruby{冷}{}たく\ruby{寒}{}く\\
        \ruby{暗}{}くも\ruby{映}{}る\ruby{空}{}しさに\\
        \ruby{倒}{}れゆくもの\ruby{今}{}この\ruby{時}{}に\\
        \ruby{想}{}いは\ruby{恵迪}{}と\ruby{共}{}に
        
    \end{minipage}
    \begin{minipage}[c]{\blocksize}
        
        \vspace{\linespace}
        \item~\\
        % 3.
        うす\ruby{靄}{}けぶる\ruby{春}{}あけぼのに\\
        \ruby{昔日}{}の\ruby{影}{}たゆたい\ruby{惑}{}う\\
        されど\ruby{緑}{}はまだ\ruby{若}{}くして\\
        \ruby{咲}{}き\ruby{初}{}む\ruby{花}{}の\ruby{望}{}もて\\
        \ruby{新}{}しき\ruby{日}{}のかげろい\ruby{浮}{}かぶ\\
        \ruby{憧}{}れ\ruby{恵迪}{}と\ruby{共}{}に
        
    \end{minipage}
    \begin{minipage}[c]{\blocksize}
        
        \vspace{\linespace}
        \item~\\
        % 4.
        うす\ruby{花}{}いろの\ruby{夏}{}よい\ruby{闇}{}に\\
        たまゆら\ruby{風}{}はさわやけし\\
        \ruby{我}{}が\ruby{宴}{}にも\ruby{星降}{}る\ruby{幸}{}と\\
        \ruby{歌}{}う\ruby{寮友}{}らの\ruby{嬉}{}しさに\\
        \ruby{憩}{}える\ruby{帆}{}にも\ruby{希}{}いありたし\\
        \ruby{夢}{}こそ\ruby{恵迪}{}と\ruby{共}{}に
        
    \end{minipage}
    \begin{minipage}[c]{\blocksize}
        
        \vspace{\linespace}
        \item~\\
        % 5.
        うつろう\ruby{四季}{}に\ruby{感慨}{}をこめて\\
        \ruby{朽}{}ちゆくものを\ruby{見}{}つめつつ\\
        いまだ\ruby{乾}{}かぬ\ruby{血涙}{}をもて\\
        ただひたすらに\ruby{祈}{}り\ruby{捧}{}ぐ\\
        \ruby{唯一真実}{}の\ruby{迪}{}を\ruby{残}{}さむ\\
        \ruby{想}{}いは\ruby{恵迪}{}を\ruby{永遠}{}に\\
        \ruby{希}{}いは\ruby{恵迪}{}よ\ruby{永遠}{}に
    
    \end{minipage}
\end{enumerate} % 番号の箇条書き ここまで
%%%%% 歌詞 ここまで %%%%%
% end body

\end{document}

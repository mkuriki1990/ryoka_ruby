\documentclass[10pt,b5j]{tarticle} % B6 縦書き
% \documentclass[10pt,b5j]{tarticle} % B6 縦書き
\AtBeginDvi{\special{papersize=128mm,182mm}} % B6 用用紙サイズ
\usepackage{otf} % Unicode で字を入力するのに必要なパッケージ
\usepackage[size=b6j]{bxpapersize} % B6 用紙サイズを指定
\usepackage[dvipdfmx]{graphicx} % 画像を挿入するためのパッケージ
\usepackage[dvipdfmx]{color} % 色をつけるためのパッケージ
\usepackage{pxrubrica} % ルビを振るためのパッケージ
\usepackage{multicol} % 複数段組を作るためのパッケージ
\setlength{\topmargin}{14mm} % 上下方向のマージン
\addtolength{\topmargin}{-1in} % 
\setlength{\oddsidemargin}{11mm} % 左右方向のマージン
\addtolength{\oddsidemargin}{-1in} % 
\setlength{\textwidth}{154mm} % B6 用
\setlength{\textheight}{108mm} % B6 用
\setlength{\headsep}{0mm} % 
\setlength{\headheight}{0mm} % 
\setlength{\topskip}{0mm} % 
\setlength{\parskip}{0pt} % 
\def\labelenumi{\theenumi、} % 箇条書きのフォーマット
\parindent = 0pt % 段落下げしない

 % B6 用テンプレート読み込み

\begin{document}
% begin header
%%%%% タイトルと作者 ここから %%%%%
\begin{minipage}[c]{0.7\hsize} % タイトルは上から 7 割
    \begin{center}
    % begin title
        {\LARGE
            水産逍遥歌 % タイトルを入れる
        }
        {\small 
            (昭和十二年) % 年などを入れる
        }
    % end title
    \end{center}
\end{minipage}
\begin{minipage}[c]{0.3\hsize} % 作歌作曲は上から 3 割
    \begin{flushright} % 下寄せにする
        % begin name
        斉藤光治君 作歌\\塩見潔君 作曲 % 作歌・作曲者
        % end name
    \end{flushright}
\end{minipage}
%%%%% タイトルと作者 ここまで %%%%%
% % end header

% begin length
\vspace{1.5em} % タイトル, 作者と歌詞の間に隙間を設ける
\newcommand{\linespace}{0.5em} % 行間の設定
\newcommand{\blocksize}{0.5\hsize} % 段組間の設定
\newcommand{\itemmargin}{3em} % 曲番の位置調整の長さ
% end length
% begin body
%%%%% 歌詞 ここから %%%%%
\begin{enumerate} % 番号の箇条書き ここから
    \setlength{\itemindent}{\itemmargin} % 曲番の位置調整
    \begin{minipage}[c]{\blocksize}
    
        \vspace{\linespace}
        \item~\\
        % 1.
        \ruby{星}{}の\ruby{姿}{}に\ruby{憧憬}{}れて\\
        \ruby{横津}{}の\ruby{丘陵}{}に\ruby{逍遥}{}へば\\
        \ruby{黄昏}{}てゆく\ruby{大空}{}に\\
        \ruby{松前城}{}の\ruby{夢}{}の\ruby{跡}{}\\
        \ruby{栄華}{}は\ruby{地上}{}の\ruby{瞬間}{}の\\
        \ruby{一枝}{}の\ruby{花}{}に\ruby{命}{}あり
        
    \end{minipage}
    \begin{minipage}[c]{\blocksize}
        
        \vspace{\linespace}
        \item~\\
        % 2.
        \ruby{白鷗海峡}{}に\ruby{舞}{}う\ruby{晨}{}\\
        \ruby{軽舸船首}{}に\ruby{珠}{}と\ruby{散}{}る\\
        \ruby{碧落}{}の\ruby{水浄}{}ければ\\
        \ruby{自由}{}と\ruby{希望}{}の\ruby{水郷}{}に\\
        \ruby{図南}{}の\ruby{夢}{}を\ruby{守}{}る\ruby{子等}{}の\\
        \ruby{高}{}き\ruby{矜持}{}を\ruby{君知}{}るや
        
    \end{minipage}
    \begin{minipage}[c]{\blocksize}
        
        \vspace{\linespace}
        \item~\\
        % 3.
        \ruby{戦}{}の\ruby{場}{}に\ruby{咲}{}く\ruby{花}{}の\\
        \ruby{名}{}は\ruby{知}{}らねども\ruby{血}{}の\ruby{花}{}か\\
        \ruby{時世}{}を\ruby{怒}{}る\ruby{烈士}{}らの\\
        その\ruby{熱血}{}を\ruby{呼}{}ぶありて\\
        まなざし\ruby{上}{}げて\ruby{眺}{}むれば\\
        \ruby{五稜原頭雨}{}ぞふる
    
    \end{minipage}
\end{enumerate} % 番号の箇条書き ここまで
%%%%% 歌詞 ここまで %%%%%
% end body

\end{document}

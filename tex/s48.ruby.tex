\documentclass[10pt,b5j]{tarticle} % B6 縦書き
% \documentclass[10pt,b5j]{tarticle} % B6 縦書き
\AtBeginDvi{\special{papersize=128mm,182mm}} % B6 用用紙サイズ
\usepackage{otf} % Unicode で字を入力するのに必要なパッケージ
\usepackage[size=b6j]{bxpapersize} % B6 用紙サイズを指定
\usepackage[dvipdfmx]{graphicx} % 画像を挿入するためのパッケージ
\usepackage[dvipdfmx]{color} % 色をつけるためのパッケージ
\usepackage{pxrubrica} % ルビを振るためのパッケージ
\usepackage{plext} % 漢数字の enumerate を使うためのパッケージ
\usepackage{multicol} % 複数段組を作るためのパッケージ
\setlength{\topmargin}{14mm} % 上下方向のマージン
\addtolength{\topmargin}{-1in} % 
\setlength{\oddsidemargin}{11mm} % 左右方向のマージン
\addtolength{\oddsidemargin}{-1in} % 
\setlength{\textwidth}{154mm} % B6 用
\setlength{\textheight}{108mm} % B6 用
\setlength{\headsep}{0mm} % 
\setlength{\headheight}{0mm} % 
\setlength{\topskip}{0mm} % 
\setlength{\parskip}{0pt} % 
\def\theenumi{\Kanji{enumi}} % 箇条書きのフォーマットを漢数字に変更
\parindent = 0pt % 段落下げしない
\pagestyle{empty} % すべてのページ番号を消去
% \renewcommand{\baselinestretch}{0.9} % 行間の倍率
 % B6 用テンプレート読み込み

\begin{document}
% begin header
%%%%% タイトルと作者 ここから %%%%%
\begin{minipage}[c]{0.7\hsize} % タイトルは上から 7 割
    \begin{center}
    % begin title
        {\LARGE
            冬の大地に % タイトルを入れる
        }
        {\small 
            (昭和四十八年寮歌) % 年などを入れる
        }
    % end title
    \end{center}
\end{minipage}
\begin{minipage}[c]{0.3\hsize} % 作歌作曲は上から 3 割
    \begin{flushright} % 下寄せにする
        % begin name
        伊藤潤平君 作歌\\矢野哲憲君 作曲 % 作歌・作曲者
        % end name
    \end{flushright}
\end{minipage}
%%%%% タイトルと作者 ここまで %%%%%
% (1,3,5 了あり)
% end header

% begin length
\vspace{1.5em} % タイトル, 作者と歌詞の間に隙間を設ける
\newcommand{\linespace}{0.5em} % 行間の設定
\newcommand{\blocksize}{0.5\hsize} % 段組間の設定
\newcommand{\itemmargin}{3em} % 曲番の位置調整の長さ
% end length
% begin body
%%%%% 歌詞 ここから %%%%%
\begin{enumerate} % 番号の箇条書き ここから
    \setlength{\itemindent}{\itemmargin} % 曲番の位置調整
    \begin{minipage}[c]{\blocksize}
    
        \vspace{\linespace}
        \item~\\
        % 1.
        \ruby{冬}{ふゆ}の\ruby{大地}{だいち}に\ruby{夢}{ゆめ}は\ruby{醒}{さ}め\\
        \ruby{旭日}{きょくじつ}に\ruby{浮}{うか}ぶ\ruby{白亜}{はくあ}\ruby{城}{じょう}\\
        \ruby{原始}{げんし}の\ruby{森}{もり}は\ruby{樹氷}{じゅひょう}\ruby{咲}{さ}き\\
        \ruby{西方}{せいほう}\ruby{空}{そら}を\ruby{眺}{ながむ}むれば\\
        \ruby{新雪}{しんせつ}\ruby{淡}{あわ}き\ruby{手稲山}{ていねやま}
        
    \end{minipage}
    \begin{minipage}[c]{\blocksize}
        
        \vspace{\linespace}
        \item~\\
        % 2.
        ポプラ\ruby{並木}{なみき}の\ruby{葉}{は}も\ruby{落}{お}ちて\\
        \ruby{秋}{あき}の\ruby{香}{こう}\ruby{深}{ふか}き\ruby{夕}{ゆう}\ruby{間}{かん}\ruby{暮}{く}れ\\
        \ruby{白日}{はくじつ}\ruby{西}{にし}に\ruby{沈}{しず}み\ruby{行}{い}き\\
        \ruby{素}{もと}\ruby{月東}{がっとう}の\ruby{森}{もり}に\ruby{出}{で}ず\\
        \ruby{乾坤}{けんこん}\ruby{環}{たまき}り\ruby{復}{ふく}た\ruby{周}{しゅう}る
        
    \end{minipage}
    \begin{minipage}[c]{\blocksize}
        
        \vspace{\linespace}
        \item~\\
        % 3.
        \ruby{浜茄子}{はなます}の\ruby{砂丘}{さきゅう}たたずみて\\
        はるかに\ruby{眺}{ながむ}むオホーツク\\
        \ruby{知床}{しれとこ}の\ruby{嶺}{みね}\ruby{雪}{ゆき}かぶり\\
        \ruby{沈}{しず}む\ruby{入日}{いりひ}に\ruby{白鳥}{はくちょう}の\\
        \ruby{飛影}{ひえい}ぞ\ruby{哀}{かな}しく\ruby{消}{き}え\ruby{去}{さ}りぬ
        
    \end{minipage}
    \begin{minipage}[c]{\blocksize}
        
        \vspace{\linespace}
        \item~\\
        % 4.
        \ruby{旅}{たび}のロマンに\ruby{誘}{さそ}われて\\
        \ruby{支}{さそお}\ruby{笏}{しゃく}の\ruby{岸}{きし}にさまよえば\\
        \ruby{静寂}{せいじゃく}の\ruby{嶺}{みね}は\ruby{荘厳}{そうごん}に\\
        \ruby{仰}{あお}ぐ\ruby{星座}{せいざ}は\ruby{闇}{やみ}に\ruby{浮}{う}き\\
        \ruby{静}{せい}に\ruby{光}{ひか}る\ruby{北極星}{ほっきょくせい}
        
    \end{minipage}
    \begin{minipage}[c]{\blocksize}
        
        \vspace{\linespace}
        \item~\\
        % 5.
        \ruby{荒}{すさ}ぶ\ruby{吹雪}{ふぶき}ぞ\ruby{旅}{たび}の\ruby{魂}{たましい}\\
        \ruby{一年}{いちねん}\ruby{涙}{なみだ}\ruby{胸}{むね}に\ruby{秘}{ひ}め\\
        \ruby{我}{わ}が\ruby{夢}{ゆめ}かけるオリオンに\\
        \ruby{我}{わ}が\ruby{春}{はる}\ruby{永久}{えいきゅう}に\ruby{朽}{く}ちざらん\\
        \ruby{蝦夷}{えぞ}が\ruby{大地}{だいち}ぞ\ruby{忘}{}るまじ
    
    \end{minipage}
\end{enumerate} % 番号の箇条書き ここまで
%%%%% 歌詞 ここまで %%%%%
% end body

\end{document}

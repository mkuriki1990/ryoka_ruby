\documentclass[10pt,b5j]{tarticle} % B6 縦書き
% \documentclass[10pt,b5j]{tarticle} % B6 縦書き
\AtBeginDvi{\special{papersize=128mm,182mm}} % B6 用用紙サイズ
\usepackage{otf} % Unicode で字を入力するのに必要なパッケージ
\usepackage[size=b6j]{bxpapersize} % B6 用紙サイズを指定
\usepackage[dvipdfmx]{graphicx} % 画像を挿入するためのパッケージ
\usepackage[dvipdfmx]{color} % 色をつけるためのパッケージ
\usepackage{pxrubrica} % ルビを振るためのパッケージ
\usepackage{multicol} % 複数段組を作るためのパッケージ
\setlength{\topmargin}{14mm} % 上下方向のマージン
\addtolength{\topmargin}{-1in} % 
\setlength{\oddsidemargin}{11mm} % 左右方向のマージン
\addtolength{\oddsidemargin}{-1in} % 
\setlength{\textwidth}{154mm} % B6 用
\setlength{\textheight}{108mm} % B6 用
\setlength{\headsep}{0mm} % 
\setlength{\headheight}{0mm} % 
\setlength{\topskip}{0mm} % 
\setlength{\parskip}{0pt} % 
\def\labelenumi{\theenumi、} % 箇条書きのフォーマット
\parindent = 0pt % 段落下げしない

 % B6 用テンプレート読み込み

\begin{document}
% begin header
%%%%% タイトルと作者 ここから %%%%%
\begin{minipage}[c]{0.7\hsize} % タイトルは上から 7 割
    \begin{center}
    % begin title
        {\LARGE
            冬の大地に % タイトルを入れる
        }
        {\small 
            (昭和四十八年寮歌) % 年などを入れる
        }
    % end title
    \end{center}
\end{minipage}
\begin{minipage}[c]{0.3\hsize} % 作歌作曲は上から 3 割
    \begin{flushright} % 下寄せにする
        % begin name
        伊藤潤平君 作歌\\矢野哲憲君 作曲 % 作歌・作曲者
        % end name
    \end{flushright}
\end{minipage}
%%%%% タイトルと作者 ここまで %%%%%
% (1,3,5 了あり)
% end header

% begin length
\vspace{1.5em} % タイトル, 作者と歌詞の間に隙間を設ける
\newcommand{\linespace}{0.5em} % 行間の設定
\newcommand{\blocksize}{0.5\hsize} % 段組間の設定
\newcommand{\itemmargin}{6em} % 曲番の位置調整の長さ
% end length
% begin body
%%%%% 歌詞 ここから %%%%%
\begin{enumerate} % 番号の箇条書き ここから
    \setlength{\itemindent}{\itemmargin} % 曲番の位置調整
    \begin{minipage}[c]{\blocksize}
    
        \vspace{\linespace}
        \item~\\
        % 1.
        \ruby{冬}{}の\ruby{大地}{}に\ruby{夢}{}は\ruby{醒}{}め\\
        \ruby{旭日}{}に\ruby{浮}{}ぶ\ruby{白亜城}{}\\
        \ruby{原始}{}の\ruby{森}{}は\ruby{樹氷咲}{}き\\
        \ruby{西方空}{}を\ruby{眺}{}むれば\\
        \ruby{新雪淡}{}き\ruby{手稲山}{}
        
        \vspace{\linespace}
        \item~\\
        % 2.
        ポプラ\ruby{並木}{}の\ruby{葉}{}も\ruby{落}{}ちて\\
        \ruby{秋}{}の\ruby{香深}{}き\ruby{夕間暮}{}れ\\
        \ruby{白日西}{}に\ruby{沈}{}み\ruby{行}{}き\\
        \ruby{素月東}{}の\ruby{森}{}に\ruby{出}{}ず\\
        \ruby{乾坤環}{}り\ruby{復}{}た\ruby{周}{}る
        
        \vspace{\linespace}
        \item~\\
        % 3.
        \ruby{浜茄子}{}の\ruby{砂丘}{}たたずみて\\
        はるかに\ruby{眺}{}むオホーツク\\
        \ruby{知床}{}の\ruby{嶺雪}{}かぶり\\
        \ruby{沈}{}む\ruby{入日}{}に\ruby{白鳥}{}の\\
        \ruby{飛影}{}ぞ\ruby{哀}{}しく\ruby{消}{}え\ruby{去}{}りぬ
        
        \vspace{\linespace}
        \item~\\
        % 4.
        \ruby{旅}{}のロマンに\ruby{誘}{}われて\\
        \ruby{支笏}{}の\ruby{岸}{}にさまよえば\\
        \ruby{静寂}{}の\ruby{嶺}{}は\ruby{荘厳}{}に\\
        \ruby{仰}{}ぐ\ruby{星座}{}は\ruby{闇}{}に\ruby{浮}{}き\\
        \ruby{静}{}に\ruby{光}{}る\ruby{北極星}{}
        
        \vspace{\linespace}
        \item~\\
        % 5.
        \ruby{荒}{}ぶ\ruby{吹雪}{}ぞ\ruby{旅}{}の\ruby{魂}{}\\
        \ruby{一年涙胸}{}に\ruby{秘}{}め\\
        \ruby{我}{}が\ruby{夢}{}かけるオリオンに\\
        \ruby{我}{}が\ruby{春永久}{}に\ruby{朽}{}ちざらん\\
        \ruby{蝦夷}{}が\ruby{大地}{}ぞ\ruby{忘}{}るまじ
    
    \end{minipage}
\end{enumerate} % 番号の箇条書き ここまで
%%%%% 歌詞 ここまで %%%%%
% end body

\end{document}

\documentclass[10pt,b5j]{tarticle} % B6 縦書き
% \documentclass[10pt,b5j]{tarticle} % B6 縦書き
\AtBeginDvi{\special{papersize=128mm,182mm}} % B6 用用紙サイズ
\usepackage{otf} % Unicode で字を入力するのに必要なパッケージ
\usepackage[size=b6j]{bxpapersize} % B6 用紙サイズを指定
\usepackage[dvipdfmx]{graphicx} % 画像を挿入するためのパッケージ
\usepackage[dvipdfmx]{color} % 色をつけるためのパッケージ
\usepackage{pxrubrica} % ルビを振るためのパッケージ
\usepackage{multicol} % 複数段組を作るためのパッケージ
\setlength{\topmargin}{14mm} % 上下方向のマージン
\addtolength{\topmargin}{-1in} % 
\setlength{\oddsidemargin}{11mm} % 左右方向のマージン
\addtolength{\oddsidemargin}{-1in} % 
\setlength{\textwidth}{154mm} % B6 用
\setlength{\textheight}{108mm} % B6 用
\setlength{\headsep}{0mm} % 
\setlength{\headheight}{0mm} % 
\setlength{\topskip}{0mm} % 
\setlength{\parskip}{0pt} % 
\def\labelenumi{\theenumi、} % 箇条書きのフォーマット
\parindent = 0pt % 段落下げしない

 % B6 用テンプレート読み込み

\begin{document}
% begin header
%%%%% タイトルと作者 ここから %%%%%
\begin{minipage}[c]{0.7\hsize} % タイトルは上から 7 割
    \begin{center}
    % begin title
        {\LARGE
            開校祝賀の歌 % タイトルを入れる
        }
        {\small 
             % 年などを入れる
        }
    % end title
    \end{center}
\end{minipage}
\begin{minipage}[c]{0.3\hsize} % 作歌作曲は上から 3 割
    \begin{flushright} % 下寄せにする
        % begin name
         % 作歌・作曲者
        % end name
    \end{flushright}
\end{minipage}
%%%%% タイトルと作者 ここまで %%%%%
% % end header

% begin length
\vspace{1.5em} % タイトル, 作者と歌詞の間に隙間を設ける
\newcommand{\linespace}{0.5em} % 行間の設定
\newcommand{\blocksize}{0.5\hsize} % 段組間の設定
\newcommand{\itemmargin}{3em} % 曲番の位置調整の長さ
% end length
% begin body
%%%%% 歌詞 ここから %%%%%
\begin{enumerate} % 番号の箇条書き ここから
    \setlength{\itemindent}{\itemmargin} % 曲番の位置調整
    \begin{minipage}[c]{\blocksize}
    
        \vspace{\linespace}
        \item~\\
        % 1.
        \ruby{神統}{}\ruby{二}{に}\ruby{千}{せん}\ruby{五}{ご}\ruby{百}{ひゃく}\ruby{年}{ねん}\\
        \ruby{東海}{とうかい}の\ruby{果}{はて}に\ruby{眠}{ねむ}りたる\\
        \ruby{大和島根}{やまとしまね}の\ruby{民衆}{みんしゅう}は\\
        \ruby{見}{み}よや\ruby{目覚}{めざ}めて\ruby{明治}{めいじ}の\ruby{世}{よ}\\
        \ruby{天}{てん}の\ruby{使命}{しめい}を\ruby{果}{はた}すべく\\
        \ruby{進取}{しんしゅ}の\ruby{旗}{はた}を\ruby{振}{ふ}り\ruby{立}{た}てぬ
        
    \end{minipage}
    \begin{minipage}[c]{\blocksize}
        
        \vspace{\linespace}
        \item~\\
        % 2.
        \ruby{天}{てん}に\ruby{二}{ふた}つの\ruby{日}{ひ}なければ\\
        \ruby{地上}{ちじょう}を\ruby{西}{にし}し\ruby{東}{ひがし}せる\\
        \ruby{文化}{ぶんか}のの\ruby{潮}{しお}\ruby{渦巻}{うずま}きて\\
        \ruby{日出}{ひので}づる\ruby{国}{こく}に\ruby{相}{あい}\ruby{会}{かい}し\\
        \ruby{炳焉}{}として\ruby{虹}{にじ}の\ruby{如}{}\\
        \ruby{乾坤}{けんこん}\ruby{茲光}{}あり
        
    \end{minipage}
    \begin{minipage}[c]{\blocksize}
        
        \vspace{\linespace}
        \item~\\
        % 3.
        \ruby{此}{}の\ruby{国運}{こくうん}に\ruby{魁}{さきがけ}し\\
        \ruby{先}{せん}づ\ruby{北辺}{ほくへん}の\ruby{島}{しま}の\ruby{上}{うえ}\\
        \ruby{荒蕪}{こうぶ}を\ruby{拓}{ひら}き\ruby{民}{みん}を\ruby{植}{うえ}ゑ\\
        \ruby{不明}{ふめい}を\ruby{教}{きょう}へ\ruby{道}{みち}を\ruby{樹}{き}て\\
        \ruby{進取}{しんしゅ}の\ruby{民}{みん}\ruby{範}{はん}たりし\\
        \ruby{百}{ひゃく}\ruby{万}{まん}の\ruby{民}{みん}\ruby{若}{わか}かりき
        
    \end{minipage}
    \begin{minipage}[c]{\blocksize}
        
        \vspace{\linespace}
        \item~\\
        % 4.
        \ruby{此}{}の\ruby{民衆}{みんしゅう}を\ruby{導}{みちび}きて\\
        \ruby{重}{おも}き\ruby{使命}{しめい}に\ruby{負}{ふ}かじと\\
        \ruby{我}{わ}が\ruby{札幌}{さっぽろ}に\ruby{建}{た}てられし\\
        \ruby{祖}{そ}\ruby{校}{こう}よく\ruby{其}{そ}の\ruby{任}{にん}に\ruby{耐}{}へ\\
        \ruby{北辰}{ほくしん}\ruby{高}{たか}く\ruby{輝}{かがや}きし\\
        \ruby{其}{そ}の\ruby{名声}{めいせい}や\ruby{將}{}た\ruby{説}{と}かじ
        
    \end{minipage}
    \begin{minipage}[c]{\blocksize}
        
        \vspace{\linespace}
        \item~\\
        % 5.
        \ruby{今}{いま}や\ruby{羽翼}{うよく}を\ruby{整}{せい}へて\\
        \ruby{徳}{とく}\ruby{乾坤}{けんこん}を\ruby{被}{ひ}ふ\ruby{可}{か}き\\
        \ruby{国}{くに}の\ruby{使命}{しめい}を\ruby{提}{さ}げて\\
        \ruby{千}{せん}\ruby{余}{よ}の\ruby{学徒}{がくと}\ruby{麾}{さしまね}き\\
        \ruby{坤}{ひつじさる}\ruby{輿}{こし}の\ruby{民}{みん}\ruby{師}{し}たる\ruby{可}{か}き\\
        \ruby{新}{しん}\ruby{職分}{しょくぶん}は\ruby{下}{お}りたり
        
    \end{minipage}
    \begin{minipage}[c]{\blocksize}
        
        \vspace{\linespace}
        \item~\\
        % 6.
        \ruby{思}{おも}へ\ruby{嘗}{かつ}ては\ruby{北辰}{ほくしん}と\\
        \ruby{光}{ひかり}を\ruby{競}{けい}ひ\ruby{白}{しろ}\ruby{雪}{ゆき}と\\
        \ruby{意}{い}\ruby{氣爭}{}ひし\ruby{校風}{こうふう}を\\
        \ruby{享}{とおる}けし\ruby{我}{われ}らの\ruby{前}{ぜん}\ruby{程}{ほど}は\\
        \ruby{高}{たか}く\ruby{大}{おお}きく\ruby{清}{きよ}らなる\\
        \ruby{希望}{きぼう}の\ruby{色}{いろ}に\ruby{溢}{あふ}れずや
        
    \end{minipage}
    \begin{minipage}[c]{\blocksize}
        
        \vspace{\linespace}
        \item~\\
        % 7.
        \ruby{功利}{こうり}\ruby{若}{わか}し\ruby{世}{よ}の\ruby{風}{かぜ}たらば\\
        \ruby{其所}{}に\ruby{我}{わが}\ruby{等}{とう}の\ruby{戦}{せん}あり\\
        \ruby{遊情}{}\ruby{若}{わか}し\ruby{世}{よ}の\ruby{俗}{ぞく}たらば\\
        \ruby{其所}{}に\ruby{我}{わが}\ruby{等}{とう}の\ruby{戦}{せん}あり\\
        \ruby{邪曲}{じゃきょく}\ruby{若}{わか}し\ruby{世}{よ}の\ruby{弊}{へい}たらば\\
        \ruby{其所}{}に\ruby{我}{わが}\ruby{等}{とう}の\ruby{戦}{せん}あり
        
    \end{minipage}
    \begin{minipage}[c]{\blocksize}
        
        \vspace{\linespace}
        \item~\\
        % 8.
        \ruby{楡}{にれ}の\ruby{梢風}{しょうふう}\ruby{鳴}{な}りて\\
        \ruby{平和}{へいわ}の\ruby{歌}{うた}をなすが\ruby{如}{}\\
        \ruby{藻}{も}\ruby{岩}{いわ}の\ruby{雲}{くも}の\ruby{峯}{みね}そひて\\
        \ruby{莊厳}{}の\ruby{色}{いろ}\ruby{動}{うご}く\ruby{如}{}\\
        \ruby{我}{わが}\ruby{等}{とう}の\ruby{歌}{うた}に\ruby{歓喜}{かんき}と\\
        \ruby{自信}{じしん}の\ruby{響}{ひびき}こもれかし
    
    \end{minipage}
\end{enumerate} % 番号の箇条書き ここまで
%%%%% 歌詞 ここまで %%%%%
% end body

\end{document}

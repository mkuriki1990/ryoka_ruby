\documentclass[10pt,b5j]{tarticle} % B6 縦書き
% \documentclass[10pt,b5j]{tarticle} % B6 縦書き
\AtBeginDvi{\special{papersize=128mm,182mm}} % B6 用用紙サイズ
\usepackage{otf} % Unicode で字を入力するのに必要なパッケージ
\usepackage[size=b6j]{bxpapersize} % B6 用紙サイズを指定
\usepackage[dvipdfmx]{graphicx} % 画像を挿入するためのパッケージ
\usepackage[dvipdfmx]{color} % 色をつけるためのパッケージ
\usepackage{pxrubrica} % ルビを振るためのパッケージ
\usepackage{plext} % 漢数字の enumerate を使うためのパッケージ
\usepackage{multicol} % 複数段組を作るためのパッケージ
\setlength{\topmargin}{14mm} % 上下方向のマージン
\addtolength{\topmargin}{-1in} % 
\setlength{\oddsidemargin}{11mm} % 左右方向のマージン
\addtolength{\oddsidemargin}{-1in} % 
\setlength{\textwidth}{154mm} % B6 用
\setlength{\textheight}{108mm} % B6 用
\setlength{\headsep}{0mm} % 
\setlength{\headheight}{0mm} % 
\setlength{\topskip}{0mm} % 
\setlength{\parskip}{0pt} % 
\def\theenumi{\Kanji{enumi}} % 箇条書きのフォーマットを漢数字に変更
\parindent = 0pt % 段落下げしない
\pagestyle{empty} % すべてのページ番号を消去
% \renewcommand{\baselinestretch}{0.9} % 行間の倍率
 % B6 用テンプレート読み込み

\begin{document}
% begin header
%%%%% タイトルと作者 ここから %%%%%
\begin{minipage}[c]{0.7\hsize} % タイトルは上から 7 割
    \begin{center}
    % begin title
        {\LARGE
            開校祝賀の歌 % タイトルを入れる
        }
        {\small 
             % 年などを入れる
        }
    % end title
    \end{center}
\end{minipage}
\begin{minipage}[c]{0.3\hsize} % 作歌作曲は上から 3 割
    \begin{flushright} % 下寄せにする
        % begin name
         % 作歌・作曲者
        % end name
    \end{flushright}
\end{minipage}
%%%%% タイトルと作者 ここまで %%%%%
% % end header

% begin body
\vspace{1.5em} % タイトル, 作者と歌詞の間に隙間を設ける
\newcommand{\linespace}{0.5em} % 行間の設定
\newcommand{\blocksize}{0.5\hsize} % 段組間の設定
%%%%% 歌詞 ここから %%%%%
% begin lilycs
\begin{enumerate} % 番号の箇条書き ここから
    \begin{minipage}[c]{\blocksize}
    
        \vspace{\linespace}
        \item
        % 1.
        神\ruby{統二千五百年}{}\\
        \ruby{東海}{}の\ruby{果}{}に\ruby{眠}{}りたる\\
        \ruby{大和島根}{}の\ruby{民衆}{}は\\
        \ruby{見}{}よや\ruby{目覚}{}めて\ruby{明治}{}の\ruby{世}{}\\
        \ruby{天}{}の\ruby{使命}{}を\ruby{果}{}すべく\\
        \ruby{進取}{}の\ruby{旗}{}を\ruby{振}{}り\ruby{立}{}てぬ
        
        \vspace{\linespace}
        \item
        % 2.
        \ruby{天}{}に\ruby{二}{}つの\ruby{日}{}なければ\\
        \ruby{地上}{}を\ruby{西}{}し\ruby{東}{}せる\\
        \ruby{文化}{}のの\ruby{潮渦巻}{}きて\\
        \ruby{日出}{}づる\ruby{国}{}に\ruby{相会}{}し\\
        \ruby{炳焉}{}として\ruby{虹}{}の\ruby{如}{}\\
        \ruby{乾坤茲光}{}あり
        
        \vspace{\linespace}
        \item
        % 3.
        \ruby{此}{}の\ruby{国運}{}に\ruby{魁}{}し\\
        \ruby{先}{}づ\ruby{北辺}{}の\ruby{島}{}の\ruby{上}{}\\
        \ruby{荒蕪}{}を\ruby{拓}{}き\ruby{民}{}を\ruby{植}{}ゑ\\
        \ruby{不明}{}を\ruby{教}{}へ\ruby{道}{}を\ruby{樹}{}て\\
        \ruby{進取}{}の\ruby{民範}{}たりし\\
        \ruby{百万}{}の\ruby{民若}{}かりき
        
        \vspace{\linespace}
        \item
        % 4.
        \ruby{此}{}の\ruby{民衆}{}を\ruby{導}{}きて\\
        \ruby{重}{}き\ruby{使命}{}に\ruby{負}{}かじと\\
        \ruby{我}{}が\ruby{札幌}{}に\ruby{建}{}てられし\\
        \ruby{祖校}{}よく\ruby{其}{}の\ruby{任}{}に\ruby{耐}{}へ\\
        \ruby{北辰高}{}く\ruby{輝}{}きし\\
        \ruby{其}{}の\ruby{名声}{}や\ruby{將}{}た\ruby{説}{}かじ
        
        \vspace{\linespace}
        \item
        % 5.
        \ruby{今}{}や\ruby{羽翼}{}を\ruby{整}{}へて\\
        \ruby{徳乾坤}{}を\ruby{被}{}ふ\ruby{可}{}き\\
        \ruby{国}{}の\ruby{使命}{}を\ruby{提}{}げて\\
        \ruby{千余}{}の\ruby{学徒麾}{}き\\
        \ruby{坤輿}{}の\ruby{民師}{}たる\ruby{可}{}き\\
        \ruby{新職分}{}は\ruby{下}{}りたり
        
        \vspace{\linespace}
        \item
        % 6.
        \ruby{思}{}へ\ruby{嘗}{}ては\ruby{北辰}{}と\\
        \ruby{光}{}を\ruby{競}{}ひ\ruby{白雪}{}と\\
        \ruby{意氣爭}{}ひし\ruby{校風}{}を\\
        \ruby{享}{}けし\ruby{我}{}らの\ruby{前程}{}は\\
        \ruby{高}{}く\ruby{大}{}きく\ruby{清}{}らなる\\
        \ruby{希望}{}の\ruby{色}{}に\ruby{溢}{}れずや
        
        \vspace{\linespace}
        \item
        % 7.
        \ruby{功利若}{}し\ruby{世}{}の\ruby{風}{}たらば\\
        \ruby{其所}{}に\ruby{我等}{}の\ruby{戦}{}あり\\
        \ruby{遊情若}{}し\ruby{世}{}の\ruby{俗}{}たらば\\
        \ruby{其所}{}に\ruby{我等}{}の\ruby{戦}{}あり\\
        \ruby{邪曲若}{}し\ruby{世}{}の\ruby{弊}{}たらば\\
        \ruby{其所}{}に\ruby{我等}{}の\ruby{戦}{}あり
        
        \vspace{\linespace}
        \item
        % 8.
        \ruby{楡}{}の\ruby{梢風鳴}{}りて\\
        \ruby{平和}{}の\ruby{歌}{}をなすが\ruby{如}{}\\
        \ruby{藻岩}{}の\ruby{雲}{}の\ruby{峯}{}そひて\\
        \ruby{莊厳}{}の\ruby{色動}{}く\ruby{如}{}\\
        \ruby{我等}{}の\ruby{歌}{}に\ruby{歓喜}{}と\\
        \ruby{自信}{}の\ruby{響}{}こもれかし
    
    \end{minipage}
\end{enumerate} % 番号の箇条書き ここまで
% end lilycs
%%%%% 歌詞 ここまで %%%%%
% end body

\end{document}

\documentclass[10pt,b5j]{tarticle} % B6 縦書き
% \documentclass[10pt,b5j]{tarticle} % B6 縦書き
\AtBeginDvi{\special{papersize=128mm,182mm}} % B6 用用紙サイズ
\usepackage{otf} % Unicode で字を入力するのに必要なパッケージ
\usepackage[size=b6j]{bxpapersize} % B6 用紙サイズを指定
\usepackage[dvipdfmx]{graphicx} % 画像を挿入するためのパッケージ
\usepackage[dvipdfmx]{color} % 色をつけるためのパッケージ
\usepackage{pxrubrica} % ルビを振るためのパッケージ
\usepackage{multicol} % 複数段組を作るためのパッケージ
\setlength{\topmargin}{14mm} % 上下方向のマージン
\addtolength{\topmargin}{-1in} % 
\setlength{\oddsidemargin}{11mm} % 左右方向のマージン
\addtolength{\oddsidemargin}{-1in} % 
\setlength{\textwidth}{154mm} % B6 用
\setlength{\textheight}{108mm} % B6 用
\setlength{\headsep}{0mm} % 
\setlength{\headheight}{0mm} % 
\setlength{\topskip}{0mm} % 
\setlength{\parskip}{0pt} % 
\def\labelenumi{\theenumi、} % 箇条書きのフォーマット
\parindent = 0pt % 段落下げしない

 % B6 用テンプレート読み込み

\begin{document}
% begin header
%%%%% タイトルと作者 ここから %%%%%
\begin{minipage}[c]{0.7\hsize} % タイトルは上から 7 割
    \begin{center}
    % begin title
        {\LARGE
            春静寂なる % タイトルを入れる
        }
        {\small 
            (昭和二十三年逍遙歌) % 年などを入れる
        }
    % end title
    \end{center}
\end{minipage}
\begin{minipage}[c]{0.3\hsize} % 作歌作曲は上から 3 割
    \begin{flushright} % 下寄せにする
        % begin name
        中島通雄君 作歌\\佐々木淳君 作曲 % 作歌・作曲者
        % end name
    \end{flushright}
\end{minipage}
%%%%% タイトルと作者 ここまで %%%%%
% (1 了あり)
% end header

% begin body
\vspace{1.5em} % タイトル, 作者と歌詞の間に隙間を設ける
\newcommand{\linespace}{0.5em} % 行間の設定
\newcommand{\blocksize}{0.5\hsize} % 段組間の設定
%%%%% 歌詞 ここから %%%%%
% begin lilycs
\begin{enumerate} % 番号の箇条書き ここから
    \begin{minipage}[c]{\blocksize}
    
        \vspace{\linespace}
        \item
        % 1.
        \ruby{春静寂}{}なる\ruby{石狩}{}の\\
        \ruby{曠野}{}に\ruby{漂泊}{}ひて\ruby{人}{}を\ruby{哭}{}き\\
        \ruby{秋蕭々}{}の\ruby{寮窓}{}に\ruby{倚}{}り\\
        \ruby{夕雲遠}{}く\ruby{友}{}を\ruby{呼}{}ぶ\\
        \ruby{北斗}{}の\ruby{啓光}{}さしそえど\\
        \ruby{哀}{}れ\ruby{悲}{}しき\ruby{旅}{}ならむ
        
        \vspace{\linespace}
        \item
        % 2.
        \ruby{北溟}{}ゆく\ruby{雁}{}は\ruby{名}{}のみにして\\
        \ruby{暮}{}る\ruby{秋風}{}に\ruby{啼}{}く\ruby{虫}{}か\\
        \ruby{楡梢}{}に\ruby{喘}{}ぐ\ruby{郭公}{}か\\
        はた\ruby{又魂}{}の\ruby{語}{}らひか\\
        \ruby{現}{}の\ruby{波濤}{}は\ruby{荒}{}くとも\\
        \ruby{知}{}るや\ruby{無象}{}の\ruby{天}{}の\ruby{外}{}
        
        \vspace{\linespace}
        \item
        % 3.
        \ruby{十勝}{}の\ruby{峰}{}に\ruby{断雲怒}{}り\\
        \ruby{白銀吼}{}ゆる\ruby{朝風}{}も\\
        \ruby{奇}{}しき\ruby{調}{}の\ruby{琴}{}と\ruby{聴}{}き\\
        \ruby{燃}{}ゆる\ruby{理想}{}に\ruby{悶}{}えつつ\\
        ただひたぶるに\ruby{辿}{}りゆく\\
        \ruby{長}{}き\ruby{生命}{}の\ruby{斗争}{}に
        
        \vspace{\linespace}
        \item
        % 4.
        \ruby{自然}{}の\ruby{芸術変}{}らねど\\
        \ruby{何処}{}に\ruby{祓所}{}を\ruby{尋}{}めゆかむ\\
        ああ\ruby{孤独}{}の\ruby{寂寥}{}を\\
        \ruby{味}{}はひ\ruby{知}{}れる\ruby{人}{}ならで\\
        \ruby{誰}{}に\ruby{語}{}らん\ruby{入相}{}の\\
        \ruby{鐘鳴}{}りひびく\ruby{楡陵}{}の\ruby{上}{}
        
        \vspace{\linespace}
        \item
        % 5.
        \ruby{花咲}{}き\ruby{散}{}りて\ruby{春秋}{}の\\
        \ruby{遷}{}りてここに\ruby{三星霜}{}\\
        \ruby{逝}{}にし\ruby{遊宴}{}の\ruby{宵}{}の\ruby{夢}{}\\
        たぎる\ruby{情熱}{}を\ruby{篝火}{}に\\
        \ruby{残恨}{}の\ruby{杯}{}を\ruby{汲}{}み\ruby{交}{}はし\\
        \ruby{高唱}{}はなんかな\ruby{自治}{}の\ruby{歌}{}
        
        \vspace{\linespace}
        \item
        % 6.
        \ruby{今逍遥}{}の\ruby{原野}{}に\ruby{萠}{}ゆる\\
        \ruby{森}{}の\ruby{翠}{}の\ruby{色深}{}く\\
        \ruby{行手遙}{}けき\ruby{豊平}{}の\\
        \ruby{清流}{}に\ruby{泛}{}ぶ\ruby{綺花}{}の\ruby{影}{}\\
        \ruby{哀}{}れ\ruby{愛}{}しき\ruby{絢夢}{}なれど\\
        \ruby{我}{}が\ruby{生命}{}こそ\ruby{真}{}なれ
    
    \end{minipage}
\end{enumerate} % 番号の箇条書き ここまで
% end lilycs
%%%%% 歌詞 ここまで %%%%%
% end body

\end{document}

\documentclass[10pt,b5j]{tarticle} % B6 縦書き
% \documentclass[10pt,b5j]{tarticle} % B6 縦書き
\AtBeginDvi{\special{papersize=128mm,182mm}} % B6 用用紙サイズ
\usepackage{otf} % Unicode で字を入力するのに必要なパッケージ
\usepackage[size=b6j]{bxpapersize} % B6 用紙サイズを指定
\usepackage[dvipdfmx]{graphicx} % 画像を挿入するためのパッケージ
\usepackage[dvipdfmx]{color} % 色をつけるためのパッケージ
\usepackage{pxrubrica} % ルビを振るためのパッケージ
\usepackage{multicol} % 複数段組を作るためのパッケージ
\setlength{\topmargin}{14mm} % 上下方向のマージン
\addtolength{\topmargin}{-1in} % 
\setlength{\oddsidemargin}{11mm} % 左右方向のマージン
\addtolength{\oddsidemargin}{-1in} % 
\setlength{\textwidth}{154mm} % B6 用
\setlength{\textheight}{108mm} % B6 用
\setlength{\headsep}{0mm} % 
\setlength{\headheight}{0mm} % 
\setlength{\topskip}{0mm} % 
\setlength{\parskip}{0pt} % 
\def\labelenumi{\theenumi、} % 箇条書きのフォーマット
\parindent = 0pt % 段落下げしない

 % B6 用テンプレート読み込み

\begin{document}
% begin header
%%%%% タイトルと作者 ここから %%%%%
\begin{minipage}[c]{0.7\hsize} % タイトルは上から 7 割
    \begin{center}
    % begin title
        {\LARGE
            惡魔死す瞬間 % タイトルを入れる
        }
        {\small 
            (平成元年度寮歌) % 年などを入れる
        }
    % end title
    \end{center}
\end{minipage}
\begin{minipage}[c]{0.3\hsize} % 作歌作曲は上から 3 割
    \begin{flushright} % 下寄せにする
        % begin name
        宜寿次盛生君 作歌\\田口拓君 作曲 % 作歌・作曲者
        % end name
    \end{flushright}
\end{minipage}
%%%%% タイトルと作者 ここまで %%%%%
% (1,2,3,4 了あり)
% end header

% begin body
\vspace{1.5em} % タイトル, 作者と歌詞の間に隙間を設ける
\newcommand{\linespace}{0.5em} % 行間の設定
\newcommand{\blocksize}{0.5\hsize} % 段組間の設定
%%%%% 歌詞 ここから %%%%%
% begin lilycs
\begin{enumerate} % 番号の箇条書き ここから
    \begin{minipage}[c]{\blocksize}
    
        \vspace{\linespace}
        \item
        % 1.
        \ruby{惡魔死}{}す\ruby{瞬間何}{}を\ruby{凝視}{}る\\
        \ruby{解}{}けざる\ruby{呪鬼}{}ヶ\ruby{島}{}\\
        \ruby{北溟}{}の\ruby{国}{}この\ruby{城}{}に\\
        \ruby{我旅立}{}ちの\ruby{時}{}を\ruby{待}{}つ
        
        \vspace{\linespace}
        \item
        % 2.
        \ruby{降}{}りたる\ruby{魔王荒}{}れ\ruby{狂}{}ふ\\
        \ruby{若}{}き\ruby{生血}{}を\ruby{吸}{}ひ\ruby{蘇}{}へる\\
        \ruby{西都}{}の\ruby{異変我知}{}らず\\
        \ruby{春欄漫}{}に\ruby{酔}{}ひ\ruby{狂}{}ふ
        
        \vspace{\linespace}
        \item
        % 3.
        \ruby{祭終}{}りて\ruby{黄葉散}{}り\\
        \ruby{暗雲広}{}がる\ruby{秋}{}の\ruby{空}{}\\
        \ruby{希望}{}の\ruby{東光恨}{}みつつ\\
        \ruby{冬将軍}{}が\ruby{猛狂}{}ふ
        
        \vspace{\linespace}
        \item
        % 4.
        \ruby{白銀}{}の\ruby{原野}{}は\ruby{静}{}まりて\\
        \ruby{地獄転}{}じて\ruby{黄泉}{}の\ruby{国}{}\\
        \ruby{野人籠}{}りて\ruby{微睡}{}みて\\
        \ruby{今旅立}{}ちの\ruby{春}{}を\ruby{待}{}つ
    
    \end{minipage}
\end{enumerate} % 番号の箇条書き ここまで
% end lilycs
%%%%% 歌詞 ここまで %%%%%
% end body

\end{document}

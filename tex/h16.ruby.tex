\documentclass[10pt,b5j]{tarticle} % B6 縦書き
% \documentclass[10pt,b5j]{tarticle} % B6 縦書き
\AtBeginDvi{\special{papersize=128mm,182mm}} % B6 用用紙サイズ
\usepackage{otf} % Unicode で字を入力するのに必要なパッケージ
\usepackage[size=b6j]{bxpapersize} % B6 用紙サイズを指定
\usepackage[dvipdfmx]{graphicx} % 画像を挿入するためのパッケージ
\usepackage[dvipdfmx]{color} % 色をつけるためのパッケージ
\usepackage{pxrubrica} % ルビを振るためのパッケージ
\usepackage{multicol} % 複数段組を作るためのパッケージ
\setlength{\topmargin}{14mm} % 上下方向のマージン
\addtolength{\topmargin}{-1in} % 
\setlength{\oddsidemargin}{11mm} % 左右方向のマージン
\addtolength{\oddsidemargin}{-1in} % 
\setlength{\textwidth}{154mm} % B6 用
\setlength{\textheight}{108mm} % B6 用
\setlength{\headsep}{0mm} % 
\setlength{\headheight}{0mm} % 
\setlength{\topskip}{0mm} % 
\setlength{\parskip}{0pt} % 
\def\labelenumi{\theenumi、} % 箇条書きのフォーマット
\parindent = 0pt % 段落下げしない

 % B6 用テンプレート読み込み

\begin{document}
% begin header
%%%%% タイトルと作者 ここから %%%%%
\begin{minipage}[c]{0.7\hsize} % タイトルは上から 7 割
    \begin{center}
    % begin title
        {\LARGE
            折れたポプラよ % タイトルを入れる
        }
        {\small 
            (平成十六年度寮歌) % 年などを入れる
        }
    % end title
    \end{center}
\end{minipage}
\begin{minipage}[c]{0.3\hsize} % 作歌作曲は上から 3 割
    \begin{flushright} % 下寄せにする
        % begin name
        高橋直樹君 作歌\\山口駿君 作曲 % 作歌・作曲者
        % end name
    \end{flushright}
\end{minipage}
%%%%% タイトルと作者 ここまで %%%%%
% (1,2,3 了なし繰り返しあり)
% end header

% begin length
\vspace{1.5em} % タイトル, 作者と歌詞の間に隙間を設ける
\newcommand{\linespace}{0.5em} % 行間の設定
\newcommand{\blocksize}{0.5\hsize} % 段組間の設定
\newcommand{\itemmargin}{6em} % 曲番の位置調整の長さ
% end length
% begin body
%%%%% 歌詞 ここから %%%%%
\begin{enumerate} % 番号の箇条書き ここから
    \setlength{\itemindent}{\itemmargin} % 曲番の位置調整
    \begin{minipage}[c]{\blocksize}
    
        \vspace{\linespace}
        \item~\\
        % 1.
        \ruby{折}{}れたポプラよ\\
        おまえは\ruby{何}{}を\ruby{言}{}わんとす\\
        \ruby{酒注}{}ぎ\ruby{交}{}わし\ruby{乾}{}した\ruby{夜}{}の\\
        \ruby{見上}{}げた\ruby{月}{}の\ruby{傍}{}らで\\
        おまえの\ruby{匂}{}いが\ruby{映}{}らない\\
        \ruby{心配}{}せなや\ruby{友達}{}よ\\
        \ruby{永久}{}に\ruby{変}{}わらず\ruby{継}{}いでやる\\
        たとえこの\ruby{世}{}が\ruby{変}{}われども\\
        \ruby{俺}{}や\ruby{寮友}{}らが\ruby{歌}{}うだろう\\
        \ruby{生命}{}の\ruby{継}{}ぎ\ruby{目}{}が\ruby{終}{}われども\\
        \ruby{心配}{}せなや\ruby{友達}{}よ\\
        お\ruby{前}{}は\ruby{此処}{}に\ruby{生}{}きている
        
        \vspace{\linespace}
        \item~\\
        % 2.
        \ruby{折}{}れたポプラよ\\
        おまえは\ruby{何}{}を\ruby{言}{}わんとす\\
        \ruby{緑}{}が\ruby{踊}{}る\ruby{夏}{}の\ruby{日}{}も\\
        \ruby{茜}{}に\ruby{溶}{}ける\ruby{秋}{}の\ruby{日}{}も\\
        \ruby{同}{}じ\ruby{生命}{}を\ruby{共}{}にした\\
        \ruby{肩}{}を\ruby{組}{}もうぞ\ruby{友達}{}よ\\
        \ruby{俺}{}とお\ruby{前}{}は\ruby{同}{}じ\ruby{土}{}\\
        \ruby{側}{}になくともその\ruby{根}{}が\\
        \ruby{歌声}{}や\ruby{思}{}いを\ruby{繋}{}ぐだろう\\
        その\ruby{身朽}{}ちゆく\ruby{運命}{}ども\\
        \ruby{肩}{}を\ruby{組}{}もうぞ\ruby{友達}{}よ\\
        \ruby{時代}{}がお\ruby{前}{}を\ruby{芽吹}{}くだろう
        
        \vspace{\linespace}
        \item~\\
        % 3.
        \ruby{折}{}れたポプラよ\\
        おまえは\ruby{何}{}を\ruby{言}{}わんとす\\
        \ruby{別}{}れの\ruby{雪}{}を\ruby{踏}{}みしめて\\
        \ruby{固}{}め\ruby{歩}{}んだ\ruby{迪}{}の\ruby{未来}{}\\
        \ruby{春}{}の\ruby{色}{}する\ruby{夢}{}なれや\\
        \ruby{共}{}に\ruby{称}{}えん\ruby{友達}{}よ\\
        \ruby{思}{}うは\ruby{日々}{}のいたずらか\\
        \ruby{過}{}ごせる\ruby{時間}{}の\ruby{限}{}れるに\\
        \ruby{尽}{}きぬ\ruby{涙}{}は\ruby{言足}{}りず\\
        \ruby{見}{}つめる\ruby{春}{}は\ruby{違}{}えども\\
        \ruby{共}{}に\ruby{称}{}えん\ruby{友達}{}よ\\
        \ruby{六華}{}が\ruby{我等照}{}らすかな
    
    \end{minipage}
\end{enumerate} % 番号の箇条書き ここまで
%%%%% 歌詞 ここまで %%%%%
% end body

\end{document}

\documentclass[10pt,b5j]{tarticle} % B6 縦書き
% \documentclass[10pt,b5j]{tarticle} % B6 縦書き
\AtBeginDvi{\special{papersize=128mm,182mm}} % B6 用用紙サイズ
\usepackage{otf} % Unicode で字を入力するのに必要なパッケージ
\usepackage[size=b6j]{bxpapersize} % B6 用紙サイズを指定
\usepackage[dvipdfmx]{graphicx} % 画像を挿入するためのパッケージ
\usepackage[dvipdfmx]{color} % 色をつけるためのパッケージ
\usepackage{pxrubrica} % ルビを振るためのパッケージ
\usepackage{multicol} % 複数段組を作るためのパッケージ
\setlength{\topmargin}{14mm} % 上下方向のマージン
\addtolength{\topmargin}{-1in} % 
\setlength{\oddsidemargin}{11mm} % 左右方向のマージン
\addtolength{\oddsidemargin}{-1in} % 
\setlength{\textwidth}{154mm} % B6 用
\setlength{\textheight}{108mm} % B6 用
\setlength{\headsep}{0mm} % 
\setlength{\headheight}{0mm} % 
\setlength{\topskip}{0mm} % 
\setlength{\parskip}{0pt} % 
\def\labelenumi{\theenumi、} % 箇条書きのフォーマット
\parindent = 0pt % 段落下げしない

 % B6 用テンプレート読み込み

\begin{document}
% begin header
%%%%% タイトルと作者 ここから %%%%%
\begin{minipage}[c]{0.7\hsize} % タイトルは上から 7 割
    \begin{center}
    % begin title
        {\LARGE
            北工会の歌 % タイトルを入れる
        }
        {\small 
            (昭和十七年) % 年などを入れる
        }
    % end title
    \end{center}
\end{minipage}
\begin{minipage}[c]{0.3\hsize} % 作歌作曲は上から 3 割
    \begin{flushright} % 下寄せにする
        % begin name
        境隆雄君 作歌・作曲 % 作歌・作曲者
        % end name
    \end{flushright}
\end{minipage}
%%%%% タイトルと作者 ここまで %%%%%
% % end header

% begin length
\vspace{1.5em} % タイトル, 作者と歌詞の間に隙間を設ける
\newcommand{\linespace}{0.5em} % 行間の設定
\newcommand{\blocksize}{0.5\hsize} % 段組間の設定
\newcommand{\itemmargin}{3em} % 曲番の位置調整の長さ
% end length
% begin body
%%%%% 歌詞 ここから %%%%%
\begin{enumerate} % 番号の箇条書き ここから
    \setlength{\itemindent}{\itemmargin} % 曲番の位置調整
    \begin{minipage}[c]{\blocksize}
    
        \vspace{\linespace}
        \item~\\
        1.\\
        あゝ\ruby{石狩}{いしかり}の\ruby{平原}{へいげん}の\\
        \ruby{郭公}{かっこう}\ruby{來}{らい}り\ruby{啼}{な}く\ruby{楡}{にれ}の\ruby{樹}{き}に\\
        \ruby{白}{しろ}く\ruby{照}{て}り\ruby{映}{}え\ruby{空高}{そらたか}く\\
        わが\ruby{学}{まな}び\ruby{舎}{や}ぞ\ruby{聳}{そび}え\ruby{立}{た}つ
        
    \end{minipage}
    \begin{minipage}[c]{\blocksize}
        
        \vspace{\linespace}
        \item~\\
        % 2.
        \ruby{新}{しん}らしき\ruby{世}{よ}の\ruby{朝}{あさ}ぼらけ\\
        \ruby{學}{まなぶ}と\ruby{技}{わざ}とを\ruby{積}{つ}みなして\\
        \ruby{文化}{ぶんか}の\ruby{塔}{とう}を\ruby{築}{きず}くこそ\\
        われ\ruby{等}{われら}が\ruby{重}{おも}き\ruby{使命}{しめい}なれ
        
    \end{minipage}
    \begin{minipage}[c]{\blocksize}
        
        \vspace{\linespace}
        \item~\\
        % 3.
        \ruby{春}{はる}\ruby{萠}{めぐむ}え\ruby{出}{しゅつ}づる\ruby{若草}{わかくさ}に\\
        \ruby{伸}{の}び\ruby{行}{い}く\ruby{生命}{いのち}を\ruby{思}{おも}い\ruby{見}{み}つ\\
        \ruby{北極星}{ほっきょくせい}の\ruby{凍}{こお}る\ruby{夜}{よる}は\\
        \ruby{遠}{とお}き\ruby{天地}{てんち}を\ruby{想}{そう}ふかな
        
    \end{minipage}
    \begin{minipage}[c]{\blocksize}
        
        \vspace{\linespace}
        \item~\\
        % 4.
        \ruby{見}{み}よや\ruby{無限}{むげん}の\ruby{大自然}{だいしぜん}\\
        \ruby{富源}{ふげん}あまねく\ruby{滿}{みつる}つるとも\\
        いかで\ruby{開}{かい}かむ\ruby{思慮}{むしりょ}\ruby{深}{ふか}き\\
        \ruby{工匠}{こうしょう}の\ruby{技}{わざ}を\ruby{用}{よう}ゐずば
        
    \end{minipage}
    \begin{minipage}[c]{\blocksize}
        
        \vspace{\linespace}
        \item~\\
        % 5.
        \ruby{日毎}{ひごと}に\ruby{進}{すす}む\ruby{文明}{ぶんめい}の\\
        \ruby{先駆者}{せんくしゃ}たるはそも\ruby{誰}{だれ}ぞ\\
        \ruby{科}{か}\ruby{學}{まなぶ}\ruby{真理}{しんり}の\ruby{行者}{ぎょうじゃ}たる\\
        わが\ruby{技術者}{ぎじゅつしゃ}にあらざるや
        
    \end{minipage}
    \begin{minipage}[c]{\blocksize}
        
        \vspace{\linespace}
        \item~\\
        % 6.
        \ruby{皇国}{こうこく}\ruby{二}{に}\ruby{千}{せん}\ruby{六}{ろく}\ruby{百}{ひゃく}\ruby{年}{ねん}\\
        \ruby{東亜}{とうあ}の\ruby{空}{そら}は\ruby{今}{いま}\ruby{明}{あ}けて\\
        \ruby{大陸}{たいりく}もまたわが\ruby{家}{わがや}ぞ\\
        \ruby{若}{わか}き\ruby{戦士}{せんし}の\ruby{血}{ち}は\ruby{躍}{おど}る
        
    \end{minipage}
    \begin{minipage}[c]{\blocksize}
        
        \vspace{\linespace}
        \item~\\
        % 7.
        あゝ\ruby{北海}{ほっかい}の\ruby{霜雪}{そうせつ}に\\
        \ruby{鍛}{かじ}へられたる\ruby{魂}{たましい}もちて\\
        \ruby{興亜}{こうあ}\ruby{日本}{にっぽん}の\ruby{建設}{けんせつ}の\\
        \ruby{太}{ふと}き\ruby{柱}{はしら}とならむかな
        
    \end{minipage}
    \begin{minipage}[c]{\blocksize}
        
        \vspace{\linespace}
        \item~\\
        \ruby{付記}{ふき}\\
        \ruby{此}{この}の\ruby{歌}{うた}は\ruby{北}{きた}\ruby{工}{こう}\ruby{会}{かい}\ruby{募集}{ぼしゅう}に\ruby{応募}{おうぼ}されたものである。
    
    \end{minipage}
\end{enumerate} % 番号の箇条書き ここまで
%%%%% 歌詞 ここまで %%%%%
% end body

\end{document}

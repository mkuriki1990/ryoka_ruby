\documentclass[10pt,b5j]{tarticle} % B6 縦書き
% \documentclass[10pt,b5j]{tarticle} % B6 縦書き
\AtBeginDvi{\special{papersize=128mm,182mm}} % B6 用用紙サイズ
\usepackage{otf} % Unicode で字を入力するのに必要なパッケージ
\usepackage[size=b6j]{bxpapersize} % B6 用紙サイズを指定
\usepackage[dvipdfmx]{graphicx} % 画像を挿入するためのパッケージ
\usepackage[dvipdfmx]{color} % 色をつけるためのパッケージ
\usepackage{pxrubrica} % ルビを振るためのパッケージ
\usepackage{plext} % 漢数字の enumerate を使うためのパッケージ
\usepackage{multicol} % 複数段組を作るためのパッケージ
\setlength{\topmargin}{14mm} % 上下方向のマージン
\addtolength{\topmargin}{-1in} % 
\setlength{\oddsidemargin}{11mm} % 左右方向のマージン
\addtolength{\oddsidemargin}{-1in} % 
\setlength{\textwidth}{154mm} % B6 用
\setlength{\textheight}{108mm} % B6 用
\setlength{\headsep}{0mm} % 
\setlength{\headheight}{0mm} % 
\setlength{\topskip}{0mm} % 
\setlength{\parskip}{0pt} % 
\def\theenumi{\Kanji{enumi}} % 箇条書きのフォーマットを漢数字に変更
\parindent = 0pt % 段落下げしない
\pagestyle{empty} % すべてのページ番号を消去
% \renewcommand{\baselinestretch}{0.9} % 行間の倍率
 % B6 用テンプレート読み込み

\begin{document}
% begin header
%%%%% タイトルと作者 ここから %%%%%
\begin{minipage}[c]{0.7\hsize} % タイトルは上から 7 割
    \begin{center}
    % begin title
        {\LARGE
            北工会の歌 % タイトルを入れる
        }
        {\small 
            (昭和十七年) % 年などを入れる
        }
    % end title
    \end{center}
\end{minipage}
\begin{minipage}[c]{0.3\hsize} % 作歌作曲は上から 3 割
    \begin{flushright} % 下寄せにする
        % begin name
        境隆雄君 作歌・作曲 % 作歌・作曲者
        % end name
    \end{flushright}
\end{minipage}
%%%%% タイトルと作者 ここまで %%%%%
% % end header

% begin body
\vspace{1.5em} % タイトル, 作者と歌詞の間に隙間を設ける
\newcommand{\linespace}{0.5em} % 行間の設定
\newcommand{\blocksize}{0.5\hsize} % 段組間の設定
%%%%% 歌詞 ここから %%%%%
% begin lilycs
\begin{enumerate} % 番号の箇条書き ここから
    \begin{minipage}[c]{\blocksize}
    
        \vspace{\linespace}
        \item
        1.\\
        あゝ\ruby{石狩}{}の\ruby{平原}{}の\\
        \ruby{郭公來}{}り\ruby{啼}{}く\ruby{楡}{}の\ruby{樹}{}に\\
        \ruby{白}{}く\ruby{照}{}り\ruby{映}{}え\ruby{空高}{}く\\
        わが\ruby{学}{}び\ruby{舎}{}ぞ\ruby{聳}{}え\ruby{立}{}つ
        
        \vspace{\linespace}
        \item
        % 2.
        \ruby{新}{}らしき\ruby{世}{}の\ruby{朝}{}ぼらけ\\
        \ruby{學}{}と\ruby{技}{}とを\ruby{積}{}みなして\\
        \ruby{文化}{}の\ruby{塔}{}を\ruby{築}{}くこそ\\
        われ\ruby{等}{}が\ruby{重}{}き\ruby{使命}{}なれ
        
        \vspace{\linespace}
        \item
        % 3.
        \ruby{春萠}{}え\ruby{出}{}づる\ruby{若草}{}に\\
        \ruby{伸}{}び\ruby{行}{}く\ruby{生命}{}を\ruby{思}{}い\ruby{見}{}つ\\
        \ruby{北極星}{}の\ruby{凍}{}る\ruby{夜}{}は\\
        \ruby{遠}{}き\ruby{天地}{}を\ruby{想}{}ふかな
        
        \vspace{\linespace}
        \item
        % 4.
        \ruby{見}{}よや\ruby{無限}{}の\ruby{大自然}{}\\
        \ruby{富源}{}あまねく\ruby{滿}{}つるとも\\
        いかで\ruby{開}{}かむ\ruby{思慮深}{}き\\
        \ruby{工匠}{}の\ruby{技}{}を\ruby{用}{}ゐずば
        
        \vspace{\linespace}
        \item
        % 5.
        \ruby{日毎}{}に\ruby{進}{}む\ruby{文明}{}の\\
        \ruby{先駆者}{}たるはそも\ruby{誰}{}ぞ\\
        \ruby{科學真理}{}の\ruby{行者}{}たる\\
        わが\ruby{技術者}{}にあらざるや
        
        \vspace{\linespace}
        \item
        % 6.
        \ruby{皇国二千六百年}{}\\
        \ruby{東亜}{}の\ruby{空}{}は\ruby{今明}{}けて\\
        \ruby{大陸}{}もまたわが\ruby{家}{}ぞ\\
        \ruby{若}{}き\ruby{戦士}{}の\ruby{血}{}は\ruby{躍}{}る
        
        \vspace{\linespace}
        \item
        % 7.
        あゝ\ruby{北海}{}の\ruby{霜雪}{}に\\
        \ruby{鍛}{}へられたる\ruby{魂}{}もちて\\
        \ruby{興亜日本}{}の\ruby{建設}{}の\\
        \ruby{太}{}き\ruby{柱}{}とならむかな
        
        \vspace{\linespace}
        \item
        \ruby{付記}{}\\
        \ruby{此}{}の\ruby{歌}{}は\ruby{北工会募集}{}に\ruby{応募}{}されたものである。
    
    \end{minipage}
\end{enumerate} % 番号の箇条書き ここまで
% end lilycs
%%%%% 歌詞 ここまで %%%%%
% end body

\end{document}

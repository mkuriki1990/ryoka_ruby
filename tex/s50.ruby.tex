\documentclass[10pt,b5j]{tarticle} % B6 縦書き
% \documentclass[10pt,b5j]{tarticle} % B6 縦書き
\AtBeginDvi{\special{papersize=128mm,182mm}} % B6 用用紙サイズ
\usepackage{otf} % Unicode で字を入力するのに必要なパッケージ
\usepackage[size=b6j]{bxpapersize} % B6 用紙サイズを指定
\usepackage[dvipdfmx]{graphicx} % 画像を挿入するためのパッケージ
\usepackage[dvipdfmx]{color} % 色をつけるためのパッケージ
\usepackage{pxrubrica} % ルビを振るためのパッケージ
\usepackage{multicol} % 複数段組を作るためのパッケージ
\setlength{\topmargin}{14mm} % 上下方向のマージン
\addtolength{\topmargin}{-1in} % 
\setlength{\oddsidemargin}{11mm} % 左右方向のマージン
\addtolength{\oddsidemargin}{-1in} % 
\setlength{\textwidth}{154mm} % B6 用
\setlength{\textheight}{108mm} % B6 用
\setlength{\headsep}{0mm} % 
\setlength{\headheight}{0mm} % 
\setlength{\topskip}{0mm} % 
\setlength{\parskip}{0pt} % 
\def\labelenumi{\theenumi、} % 箇条書きのフォーマット
\parindent = 0pt % 段落下げしない

 % B6 用テンプレート読み込み

\begin{document}
% begin header
%%%%% タイトルと作者 ここから %%%%%
\begin{minipage}[c]{0.7\hsize} % タイトルは上から 7 割
    \begin{center}
    % begin title
        {\LARGE
            憧憬の故郷 % タイトルを入れる
        }
        {\small 
            (昭和五十年寮歌) % 年などを入れる
        }
    % end title
    \end{center}
\end{minipage}
\begin{minipage}[c]{0.3\hsize} % 作歌作曲は上から 3 割
    \begin{flushright} % 下寄せにする
        % begin name
        佐藤守君 作歌\\関川哲夫君 作曲 % 作歌・作曲者
        % end name
    \end{flushright}
\end{minipage}
%%%%% タイトルと作者 ここまで %%%%%
% (1,2,3,4,5 繰り返しなし)
% end header

% begin length
\vspace{1.5em} % タイトル, 作者と歌詞の間に隙間を設ける
\newcommand{\linespace}{0.5em} % 行間の設定
\newcommand{\blocksize}{0.5\hsize} % 段組間の設定
\newcommand{\itemmargin}{3em} % 曲番の位置調整の長さ
% end length
% begin body
%%%%% 歌詞 ここから %%%%%
\begin{enumerate} % 番号の箇条書き ここから
    \setlength{\itemindent}{\itemmargin} % 曲番の位置調整
    \begin{minipage}[c]{\blocksize}
    
        \vspace{\linespace}
        \item~\\
        % 1.
        「\ruby{汝}{なんじ}が\ruby{故郷}{こきょう}は\ruby{何処}{どこ}にありや」\\
        \ruby{熱}{あつ}き\ruby{血潮}{ちしお}に\ruby{身}{み}は\ruby{溢}{あふ}れども\\
        \ruby{希望}{きぼう}を\ruby{胸}{むね}に\ruby{行方}{ゆくえ}も\ruby{知}{し}れず\\
        \ruby{朔風}{さくふう}に\ruby{身}{み}を\ruby{寄}{よ}せ\ruby{漂泊}{ひょうはく}い\ruby{出}{で}でん
        
    \end{minipage}
    \begin{minipage}[c]{\blocksize}
        
        \vspace{\linespace}
        \item~\\
        % 2.
        \ruby{聳}{}ゆるポプラは\ruby{何}{なに}をか\ruby{象徴}{しょうちょう}し\\
        \ruby{遙}{はる}かな\ruby{大地}{だいち}は\ruby{何}{なに}\ruby{語}{かた}るらん\\
        \ruby{渺茫}{びょうぼう}の\ruby{地}{ち}に\ruby{理想}{りそう}を\ruby{秘}{ひ}めて\\
        \ruby{真摯}{しんし}の\ruby{道}{みち}を\ruby{歩}{あゆ}みゆかん
        
    \end{minipage}
    \begin{minipage}[c]{\blocksize}
        
        \vspace{\linespace}
        \item~\\
        % 3.
        \ruby{逍遙}{}の\ruby{詩}{し}\ruby{静寂}{せいじゃく}に\ruby{透}{とお}り\\
        \ruby{曠野}{あらの}を\ruby{一人}{ひとり}ゆく\ruby{吾}{われ}\ruby{佇}{たたず}めば\\
        \ruby{日輪}{にちりん}\ruby{幽寂}{ゆうじゃく}に\ruby{手稲}{ていね}の\ruby{端}{はじ}にて\\
        \ruby{朱}{しゅ}に\ruby{染}{そ}まらん\ruby{哉原}{さいはら}\ruby{始}{はじめ}の\ruby{森}{もり}は
        
    \end{minipage}
    \begin{minipage}[c]{\blocksize}
        
        \vspace{\linespace}
        \item~\\
        % 4.
        \ruby{嗚呼}{ああ}\ruby{寮}{りょう}\ruby{友}{とも}よ\ruby{夕}{ゆう}の\ruby{瞑想}{めいそう}\\
        \ruby{己}{おのれ}\ruby{身}{み}に\ruby{嘆}{なげ}けども\ruby{憂愁}{ゆうしゅう}はやまず\\
        \ruby{白銀}{はくぎん}の\ruby{季節}{きせつ}\ruby{寮舎}{りょうしゃ}に\ruby{在}{あ}りて\\
        \ruby{熱}{ねっ}き\ruby{心}{しん}を\ruby{語}{かた}り\ruby{明}{あ}かせよ
        
    \end{minipage}
    \begin{minipage}[c]{\blocksize}
        
        \vspace{\linespace}
        \item~\\
        % 5.
        \ruby{光}{ひかり}\ruby{幽}{かそけ}けき\ruby{憧憬}{どうけい}の\ruby{故郷}{こきょう}\\
        \ruby{霞}{かすみ}\ruby{静}{しず}かに\ruby{流}{なが}れ\ruby{渡}{わた}りて\\
        \ruby{新緑}{しんりょく}にみる\ruby{自然}{しぜん}の\ruby{黙示}{もくし}\\
        \ruby{北}{きた}\ruby{溟}{}の\ruby{大地}{だいち}は\ruby{我}{わ}が\ruby{故郷}{こきょう}か
    
    \end{minipage}
\end{enumerate} % 番号の箇条書き ここまで
%%%%% 歌詞 ここまで %%%%%
% end body

\end{document}

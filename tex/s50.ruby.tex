\documentclass[10pt,b5j]{tarticle} % B6 縦書き
% \documentclass[10pt,b5j]{tarticle} % B6 縦書き
\AtBeginDvi{\special{papersize=128mm,182mm}} % B6 用用紙サイズ
\usepackage{otf} % Unicode で字を入力するのに必要なパッケージ
\usepackage[size=b6j]{bxpapersize} % B6 用紙サイズを指定
\usepackage[dvipdfmx]{graphicx} % 画像を挿入するためのパッケージ
\usepackage[dvipdfmx]{color} % 色をつけるためのパッケージ
\usepackage{pxrubrica} % ルビを振るためのパッケージ
\usepackage{plext} % 漢数字の enumerate を使うためのパッケージ
\usepackage{multicol} % 複数段組を作るためのパッケージ
\setlength{\topmargin}{14mm} % 上下方向のマージン
\addtolength{\topmargin}{-1in} % 
\setlength{\oddsidemargin}{11mm} % 左右方向のマージン
\addtolength{\oddsidemargin}{-1in} % 
\setlength{\textwidth}{154mm} % B6 用
\setlength{\textheight}{108mm} % B6 用
\setlength{\headsep}{0mm} % 
\setlength{\headheight}{0mm} % 
\setlength{\topskip}{0mm} % 
\setlength{\parskip}{0pt} % 
\def\theenumi{\Kanji{enumi}} % 箇条書きのフォーマットを漢数字に変更
\parindent = 0pt % 段落下げしない
\pagestyle{empty} % すべてのページ番号を消去
% \renewcommand{\baselinestretch}{0.9} % 行間の倍率
 % B6 用テンプレート読み込み

\begin{document}
% begin header
%%%%% タイトルと作者 ここから %%%%%
\begin{minipage}[c]{0.7\hsize} % タイトルは上から 7 割
    \begin{center}
    % begin title
        {\LARGE
            憧憬の故郷 % タイトルを入れる
        }
        {\small 
            (昭和五十年寮歌) % 年などを入れる
        }
    % end title
    \end{center}
\end{minipage}
\begin{minipage}[c]{0.3\hsize} % 作歌作曲は上から 3 割
    \begin{flushright} % 下寄せにする
        % begin name
        佐藤守君 作歌\\関川哲夫君 作曲 % 作歌・作曲者
        % end name
    \end{flushright}
\end{minipage}
%%%%% タイトルと作者 ここまで %%%%%
% (1,2,3,4,5 繰り返しなし)
% end header

% begin length
\vspace{1.5em} % タイトル, 作者と歌詞の間に隙間を設ける
\newcommand{\linespace}{0.5em} % 行間の設定
\newcommand{\blocksize}{0.5\hsize} % 段組間の設定
\newcommand{\itemmargin}{3em} % 曲番の位置調整の長さ
% end length
% begin body
%%%%% 歌詞 ここから %%%%%
\begin{enumerate} % 番号の箇条書き ここから
    \setlength{\itemindent}{\itemmargin} % 曲番の位置調整
    \begin{minipage}[c]{\blocksize}
    
        \vspace{\linespace}
        \item~\\
        % 1.
        「\ruby{汝}{}が\ruby{故郷}{}は\ruby{何処}{}にありや」\\
        \ruby{熱}{}き\ruby{血潮}{}に\ruby{身}{}は\ruby{溢}{}れども\\
        \ruby{希望}{}を\ruby{胸}{}に\ruby{行方}{}も\ruby{知}{}れず\\
        \ruby{朔風}{}に\ruby{身}{}を\ruby{寄}{}せ\ruby{漂泊}{}い\ruby{出}{}でん
        
    \end{minipage}
    \begin{minipage}[c]{\blocksize}
        
        \vspace{\linespace}
        \item~\\
        % 2.
        \ruby{聳}{}ゆるポプラは\ruby{何}{}をか\ruby{象徴}{}し\\
        \ruby{遙}{}かな\ruby{大地}{}は\ruby{何語}{}るらん\\
        \ruby{渺茫}{}の\ruby{地}{}に\ruby{理想}{}を\ruby{秘}{}めて\\
        \ruby{真摯}{}の\ruby{道}{}を\ruby{歩}{}みゆかん
        
    \end{minipage}
    \begin{minipage}[c]{\blocksize}
        
        \vspace{\linespace}
        \item~\\
        % 3.
        \ruby{逍遙}{}の\ruby{詩静寂}{}に\ruby{透}{}り\\
        \ruby{曠野}{}を\ruby{一人}{}ゆく\ruby{吾佇}{}めば\\
        \ruby{日輪幽寂}{}に\ruby{手稲}{}の\ruby{端}{}にて\\
        \ruby{朱}{}に\ruby{染}{}まらん\ruby{哉原始}{}の\ruby{森}{}は
        
    \end{minipage}
    \begin{minipage}[c]{\blocksize}
        
        \vspace{\linespace}
        \item~\\
        % 4.
        \ruby{嗚呼寮友}{}よ\ruby{夕}{}の\ruby{瞑想}{}\\
        \ruby{己身}{}に\ruby{嘆}{}けども\ruby{憂愁}{}はやまず\\
        \ruby{白銀}{}の\ruby{季節寮舎}{}に\ruby{在}{}りて\\
        \ruby{熱}{}き\ruby{心}{}を\ruby{語}{}り\ruby{明}{}かせよ
        
    \end{minipage}
    \begin{minipage}[c]{\blocksize}
        
        \vspace{\linespace}
        \item~\\
        % 5.
        \ruby{光幽}{}けき\ruby{憧憬}{}の\ruby{故郷}{}\\
        \ruby{霞静}{}かに\ruby{流}{}れ\ruby{渡}{}りて\\
        \ruby{新緑}{}にみる\ruby{自然}{}の\ruby{黙示}{}\\
        \ruby{北溟}{}の\ruby{大地}{}は\ruby{我}{}が\ruby{故郷}{}か
    
    \end{minipage}
\end{enumerate} % 番号の箇条書き ここまで
%%%%% 歌詞 ここまで %%%%%
% end body

\end{document}

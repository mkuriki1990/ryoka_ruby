\documentclass[10pt,b5j]{tarticle} % B6 縦書き
% \documentclass[10pt,b5j]{tarticle} % B6 縦書き
\AtBeginDvi{\special{papersize=128mm,182mm}} % B6 用用紙サイズ
\usepackage{otf} % Unicode で字を入力するのに必要なパッケージ
\usepackage[size=b6j]{bxpapersize} % B6 用紙サイズを指定
\usepackage[dvipdfmx]{graphicx} % 画像を挿入するためのパッケージ
\usepackage[dvipdfmx]{color} % 色をつけるためのパッケージ
\usepackage{pxrubrica} % ルビを振るためのパッケージ
\usepackage{multicol} % 複数段組を作るためのパッケージ
\setlength{\topmargin}{14mm} % 上下方向のマージン
\addtolength{\topmargin}{-1in} % 
\setlength{\oddsidemargin}{11mm} % 左右方向のマージン
\addtolength{\oddsidemargin}{-1in} % 
\setlength{\textwidth}{154mm} % B6 用
\setlength{\textheight}{108mm} % B6 用
\setlength{\headsep}{0mm} % 
\setlength{\headheight}{0mm} % 
\setlength{\topskip}{0mm} % 
\setlength{\parskip}{0pt} % 
\def\labelenumi{\theenumi、} % 箇条書きのフォーマット
\parindent = 0pt % 段落下げしない

 % B6 用テンプレート読み込み

\begin{document}
% begin header
%%%%% タイトルと作者 ここから %%%%%
\begin{minipage}[c]{0.7\hsize} % タイトルは上から 7 割
    \begin{center}
    % begin title
        {\LARGE
            スキー部部歌 % タイトルを入れる
        }
        {\small 
             % 年などを入れる
        }
    % end title
    \end{center}
\end{minipage}
\begin{minipage}[c]{0.3\hsize} % 作歌作曲は上から 3 割
    \begin{flushright} % 下寄せにする
        % begin name
        相川正義君 作歌\\小野鉄之助君 作曲 % 作歌・作曲者
        % end name
    \end{flushright}
\end{minipage}
%%%%% タイトルと作者 ここまで %%%%%
% % end header

% begin length
\vspace{1.5em} % タイトル, 作者と歌詞の間に隙間を設ける
\newcommand{\linespace}{0.5em} % 行間の設定
\newcommand{\blocksize}{0.5\hsize} % 段組間の設定
\newcommand{\itemmargin}{6em} % 曲番の位置調整の長さ
% end length
% begin body
%%%%% 歌詞 ここから %%%%%
\begin{enumerate} % 番号の箇条書き ここから
    \setlength{\itemindent}{\itemmargin} % 曲番の位置調整
    \begin{minipage}[c]{\blocksize}
    
        \vspace{\linespace}
        \item~\\
        % 1.
        \ruby{見}{}よや\ruby{若人}{} \ruby{見}{}はるかす\\
        \ruby{北山脈}{}の \ruby{頂}{}に\\
        \ruby{巨人}{}の\ruby{息}{}は \ruby{凝}{}り\ruby{果}{}てて\\
        \ruby{冬厳}{}に \ruby{訪}{}れぬ
        
        \vspace{\linespace}
        \item~\\
        % 2.
        \ruby{冬}{}こそ\ruby{男性}{} \ruby{雄々}{}しかる\\
        \ruby{自然}{}の\ruby{息吹}{} \ruby{荒}{}くして\\
        \ruby{人無}{}き\ruby{山岳}{} いや\ruby{高}{}く\\
        \ruby{吾呼}{}ぶ\ruby{声}{}の なつかしき
        
        \vspace{\linespace}
        \item~\\
        % 3.
        \ruby{切}{}なく\ruby{昏}{}き \ruby{人}{}の\ruby{世}{}の\\
        \ruby{夢}{}む\ruby{杯}{} \ruby{打}{}ち\ruby{砕}{}き\\
        \ruby{吹雪}{}く\ruby{樹林}{}に \ruby{躍}{}り\ruby{入}{}り\\
        スキーを\ruby{炬火}{}と \ruby{燃}{}えしめん
        
        \vspace{\linespace}
        \item~\\
        % 4.
        \ruby{繚爛花}{}の \ruby{粧}{}は\\
        \ruby{凍土深}{}く あせたれど\\
        \ruby{雪}{}の\ruby{野征矢}{}と \ruby{馳}{}せ\ruby{違}{}ふ\\
        \ruby{頬紅}{}に \ruby{誇}{}りあり
        
        \vspace{\linespace}
        \item~\\
        % 5.
        \ruby{窮}{}まる\ruby{林}{} \ruby{幻想}{}に\\
        \ruby{打}{}ち\ruby{黙}{}しつつ \ruby{尋}{}め\ruby{行}{}けば\\
        \ruby{故郷恋}{}ふる \ruby{旅人}{}の\\
        \ruby{肩}{}に\ruby{降}{}り\ruby{散}{}る \ruby{雪崩}{}
        
        \vspace{\linespace}
        \item~\\
        % 6.
        その\ruby{頂}{}の \ruby{樹氷}{}かみ\\
        \ruby{今立}{}ち\ruby{返}{}る \ruby{麓}{}へと\\
        \ruby{指}{}さす\ruby{方}{}に \ruby{雪}{}は\ruby{映}{}え\\
        \ruby{聖}{}き\ruby{思}{}ひに \ruby{奢}{}るなり
        
        \vspace{\linespace}
        \item~\\
        % 7.
        シルバーシャンツェに \ruby{陽}{}はくれて\\
        \ruby{谺}{}ひびける \ruby{谿間}{}に\\
        \ruby{描}{}きつくせし スプールを\\
        \ruby{顧}{}みる\ruby{身}{}の \ruby{歓喜}{}よ
        
        \vspace{\linespace}
        \item~\\
        % 8.
        \ruby{窓}{}に\ruby{私語}{}めく \ruby{粉雪}{}に\\
        \ruby{雪煙湧}{}く \ruby{滑走}{}の\\
        \ruby{胸}{}とどろかす \ruby{男}{}の\ruby{子等}{}よ\\
        \ruby{明日}{}の\ruby{飛躍}{}に \ruby{幸}{}あれや
    
    \end{minipage}
\end{enumerate} % 番号の箇条書き ここまで
%%%%% 歌詞 ここまで %%%%%
% end body

\end{document}

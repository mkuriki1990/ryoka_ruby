\documentclass[10pt,b5j]{tarticle} % B6 縦書き
% \documentclass[10pt,b5j]{tarticle} % B6 縦書き
\AtBeginDvi{\special{papersize=128mm,182mm}} % B6 用用紙サイズ
\usepackage{otf} % Unicode で字を入力するのに必要なパッケージ
\usepackage[size=b6j]{bxpapersize} % B6 用紙サイズを指定
\usepackage[dvipdfmx]{graphicx} % 画像を挿入するためのパッケージ
\usepackage[dvipdfmx]{color} % 色をつけるためのパッケージ
\usepackage{pxrubrica} % ルビを振るためのパッケージ
\usepackage{plext} % 漢数字の enumerate を使うためのパッケージ
\usepackage{multicol} % 複数段組を作るためのパッケージ
\setlength{\topmargin}{14mm} % 上下方向のマージン
\addtolength{\topmargin}{-1in} % 
\setlength{\oddsidemargin}{11mm} % 左右方向のマージン
\addtolength{\oddsidemargin}{-1in} % 
\setlength{\textwidth}{154mm} % B6 用
\setlength{\textheight}{108mm} % B6 用
\setlength{\headsep}{0mm} % 
\setlength{\headheight}{0mm} % 
\setlength{\topskip}{0mm} % 
\setlength{\parskip}{0pt} % 
\def\theenumi{\Kanji{enumi}} % 箇条書きのフォーマットを漢数字に変更
\parindent = 0pt % 段落下げしない
\pagestyle{empty} % すべてのページ番号を消去
% \renewcommand{\baselinestretch}{0.9} % 行間の倍率
 % B6 用テンプレート読み込み

\begin{document}
% begin header
%%%%% タイトルと作者 ここから %%%%%
\begin{minipage}[c]{0.7\hsize} % タイトルは上から 7 割
    \begin{center}
    % begin title
        {\LARGE
            いつの日にか % タイトルを入れる
        }
        {\small 
            (昭和五十一年寮歌) % 年などを入れる
        }
    % end title
    \end{center}
\end{minipage}
\begin{minipage}[c]{0.3\hsize} % 作歌作曲は上から 3 割
    \begin{flushright} % 下寄せにする
        % begin name
        小島茂君 作歌\\真鍋利徳君 作曲 % 作歌・作曲者
        % end name
    \end{flushright}
\end{minipage}
%%%%% タイトルと作者 ここまで %%%%%
% (1,4,5 繰り返しなし)
% end header

% begin length
\vspace{1.5em} % タイトル, 作者と歌詞の間に隙間を設ける
\newcommand{\linespace}{0.5em} % 行間の設定
\newcommand{\blocksize}{0.5\hsize} % 段組間の設定
\newcommand{\itemmargin}{3em} % 曲番の位置調整の長さ
% end length
% begin body
%%%%% 歌詞 ここから %%%%%
\begin{enumerate} % 番号の箇条書き ここから
    \setlength{\itemindent}{\itemmargin} % 曲番の位置調整
    \begin{minipage}[c]{\blocksize}
    
        \vspace{\linespace}
        \item~\\
        % 1.
        \ruby{夜}{}は\ruby{巡}{}り\\
        \ruby{限}{}りなき\ruby{光}{}の\ruby{束}{}は\\
        \ruby{樹林}{}をつらぬきぬ\\
        \ruby{朝}{}の\ruby{静寂}{}の\ruby{中一人}{}にて\\
        \ruby{無為}{}の\ruby{思}{}いもち\ruby{嘆}{}き\ruby{憂}{}える\\
        もう\ruby{情熱}{}もなく\ruby{涙}{}ながる
        
    \end{minipage}
    \begin{minipage}[c]{\blocksize}
        
        \vspace{\linespace}
        \item~\\
        % 2.
        \ruby{何}{}を\ruby{求}{}め\\
        ほの\ruby{暗}{}き\ruby{大気}{}の\ruby{底}{}に\\
        \ruby{真摯}{}な\ruby{魂}{}は\\
        \ruby{一}{}つの\ruby{心}{}を\ruby{持}{}ちさまよいぬ\\
        もはや\ruby{言葉}{}なく\\
        \ruby{凍}{}てつきて\ruby{立}{}つポプラを\ruby{見}{}つめ\\
        \ruby{祈}{}りささぐ
        
    \end{minipage}
    \begin{minipage}[c]{\blocksize}
        
        \vspace{\linespace}
        \item~\\
        % 3.
        \ruby{大}{}き\ruby{精神}{}\\
        \ruby{物思}{}う\ruby{我}{}らに\\
        いまだあれどかすかなり\\
        \ruby{不毛}{}の\ruby{日々}{}はかわき\ruby{過}{}ぎ\ruby{去}{}りぬ\\
        なれどいつの\ruby{日}{}か\ruby{結}{}びつけなん\\
        \ruby{我}{}らが\ruby{命大}{}き\ruby{魂}{}へ
        
    \end{minipage}
    \begin{minipage}[c]{\blocksize}
        
        \vspace{\linespace}
        \item~\\
        % 4.
        \ruby{女性}{}の\ruby{清}{}き\ruby{美}{}しさ\\
        \ruby{真摯}{}な\ruby{理性}{}の\ruby{輝}{}きにさそわれて\\
        ほのかな\ruby{恋}{}の\ruby{想}{}い\ruby{胸}{}に\\
        なれど\ruby{結}{}びえず\\
        あまりに\ruby{深}{}き\ruby{心}{}のあがき\\
        この\ruby{暗}{}さに
        
    \end{minipage}
    \begin{minipage}[c]{\blocksize}
        
        \vspace{\linespace}
        \item~\\
        % 5.
        \ruby{深}{}き\ruby{森}{}のささやき\\
        \ruby{清冷}{}な\ruby{川}{}の\ruby{流}{}れに\ruby{聞}{}きいりて\\
        \ruby{清}{}らかさの\ruby{中我息}{}しなん\\
        \ruby{物}{}を\ruby{思}{}わなん\\
        \ruby{静}{}けさの\ruby{中}{}とけこみいりて\\
        いつの\ruby{日}{}にか
    
    \end{minipage}
\end{enumerate} % 番号の箇条書き ここまで
%%%%% 歌詞 ここまで %%%%%
% end body

\end{document}

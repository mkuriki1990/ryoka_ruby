\documentclass[10pt,b5j]{tarticle} % B6 縦書き
% \documentclass[10pt,b5j]{tarticle} % B6 縦書き
\AtBeginDvi{\special{papersize=128mm,182mm}} % B6 用用紙サイズ
\usepackage{otf} % Unicode で字を入力するのに必要なパッケージ
\usepackage[size=b6j]{bxpapersize} % B6 用紙サイズを指定
\usepackage[dvipdfmx]{graphicx} % 画像を挿入するためのパッケージ
\usepackage[dvipdfmx]{color} % 色をつけるためのパッケージ
\usepackage{pxrubrica} % ルビを振るためのパッケージ
\usepackage{multicol} % 複数段組を作るためのパッケージ
\setlength{\topmargin}{14mm} % 上下方向のマージン
\addtolength{\topmargin}{-1in} % 
\setlength{\oddsidemargin}{11mm} % 左右方向のマージン
\addtolength{\oddsidemargin}{-1in} % 
\setlength{\textwidth}{154mm} % B6 用
\setlength{\textheight}{108mm} % B6 用
\setlength{\headsep}{0mm} % 
\setlength{\headheight}{0mm} % 
\setlength{\topskip}{0mm} % 
\setlength{\parskip}{0pt} % 
\def\labelenumi{\theenumi、} % 箇条書きのフォーマット
\parindent = 0pt % 段落下げしない

 % B6 用テンプレート読み込み

\begin{document}
% begin header
%%%%% タイトルと作者 ここから %%%%%
\begin{minipage}[c]{0.7\hsize} % タイトルは上から 7 割
    \begin{center}
    % begin title
        {\LARGE
            朝葉末の % タイトルを入れる
        }
        {\small 
            (第三期卒業生贈桜星会歌) % 年などを入れる
        }
    % end title
    \end{center}
\end{minipage}
\begin{minipage}[c]{0.3\hsize} % 作歌作曲は上から 3 割
    \begin{flushright} % 下寄せにする
        % begin name
        加藤義夫君 作歌\\角倉邦彦君 作曲 % 作歌・作曲者
        % end name
    \end{flushright}
\end{minipage}
%%%%% タイトルと作者 ここまで %%%%%
% % end header

% begin body
\vspace{1.5em} % タイトル, 作者と歌詞の間に隙間を設ける
\newcommand{\linespace}{0.5em} % 行間の設定
\newcommand{\blocksize}{0.5\hsize} % 段組間の設定
%%%%% 歌詞 ここから %%%%%
% begin lilycs
\begin{enumerate} % 番号の箇条書き ここから
    \begin{minipage}[c]{\blocksize}
    
        \vspace{\linespace}
        \item
        % 1.
        \ruby{朝葉末}{}の\ruby{露}{}を\ruby{受}{}け\\
        \ruby{夕歸鳥}{}の\ruby{影宿}{}し\\
        \ruby{曙匂}{}ふ\ruby{石狩}{}に\\
        \ruby{玉}{}の\ruby{泉}{}と\ruby{湧}{}きしより\\
        \ruby{思}{}へば\ruby{茲}{}に\ruby{三歳}{}の\\
        \ruby{過}{}ぎにし\ruby{水路}{}を\ruby{偲}{}ぶ\ruby{哉}{}
        
        \vspace{\linespace}
        \item
        % 2.
        \ruby{大気}{}は\ruby{凍}{}り\ruby{雪}{}もやの\\
        \ruby{荒}{}れし\ruby{廣野}{}の\ruby{面}{}をこむ\\
        \ruby{時}{}しも\ruby{高}{}く\ruby{天界}{}に\\
        \ruby{光芒強}{}き\ruby{北極星}{}\\
        いさごと\ruby{光}{}る\ruby{星}{}くづは\\
        \ruby{我}{}をばめぐり\ruby{走}{}るなり
        
        \vspace{\linespace}
        \item
        % 3.
        かつらの\ruby{若芽色}{}も\ruby{濃}{}く\\
        \ruby{森}{}に\ruby{生氣}{}の\ruby{溢}{}る\ruby{時}{}\\
        \ruby{奇}{}しき\ruby{天地}{}の\ruby{靈受}{}けて\\
        \ruby{大和心}{}と\ruby{咲}{}き\ruby{出}{}でし\\
        \ruby{蝦夷}{}の\ruby{深山}{}の\ruby{山櫻}{}\\
        \ruby{我等}{}が\ruby{理想此處}{}にあり
        
        \vspace{\linespace}
        \item
        % 4.
        \ruby{雲漠々}{}に\ruby{水}{}ゆるぎ\\
        \ruby{大野}{}の\ruby{心我}{}にあり\\
        \ruby{眞理求}{}めて\ruby{息}{}まざる\\
        \ruby{久遠}{}の\ruby{望我}{}にあり\\
        \ruby{衆愚}{}の\ruby{聲}{}にまどはざる\\
        \ruby{我}{}に\ruby{男}{}の\ruby{子}{}の\ruby{覺悟}{}あり
        
        \vspace{\linespace}
        \item
        % 5.
        \ruby{消}{}ゆる\ruby{榮華}{}を\ruby{夢}{}に\ruby{見}{}て\\
        \ruby{虚}{}しき\ruby{名}{}をば\ruby{人}{}よ\ruby{追}{}へ\\
        \ruby{北}{}の\ruby{荒野}{}に\ruby{三百}{}の\\
        \ruby{健兒浮雲}{}を\ruby{嘲}{}りつ\\
        \ruby{永遠}{}に\ruby{變}{}らぬ\ruby{美土}{}に\\
        \ruby{注}{}ぎし\ruby{汗}{}の\ruby{寶}{}を\ruby{求}{}む
        
        \vspace{\linespace}
        \item
        % 6.
        \ruby{黄花}{}の\ruby{牧}{}に\ruby{新緑}{}の\\
        \ruby{森}{}に\ruby{鍛}{}へよ\ruby{鐵}{}の\ruby{腕}{}\\
        \ruby{紅葉彩}{}どる\ruby{野}{}に\ruby{山}{}に\\
        \ruby{吹雪}{}の\ruby{里}{}に\ruby{思想錬}{}れ\\
        \ruby{勉}{}めよ\ruby{奮}{}へ\ruby{我友}{}よ\\
        やがてぞ\ruby{起}{}たん\ruby{時}{}は\ruby{來}{}ん
    
    \end{minipage}
\end{enumerate} % 番号の箇条書き ここまで
% end lilycs
%%%%% 歌詞 ここまで %%%%%
% end body

\end{document}

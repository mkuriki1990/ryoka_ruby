\documentclass[10pt,b5j]{tarticle} % B6 縦書き
% \documentclass[10pt,b5j]{tarticle} % B6 縦書き
\AtBeginDvi{\special{papersize=128mm,182mm}} % B6 用用紙サイズ
\usepackage{otf} % Unicode で字を入力するのに必要なパッケージ
\usepackage[size=b6j]{bxpapersize} % B6 用紙サイズを指定
\usepackage[dvipdfmx]{graphicx} % 画像を挿入するためのパッケージ
\usepackage[dvipdfmx]{color} % 色をつけるためのパッケージ
\usepackage{pxrubrica} % ルビを振るためのパッケージ
\usepackage{multicol} % 複数段組を作るためのパッケージ
\setlength{\topmargin}{14mm} % 上下方向のマージン
\addtolength{\topmargin}{-1in} % 
\setlength{\oddsidemargin}{11mm} % 左右方向のマージン
\addtolength{\oddsidemargin}{-1in} % 
\setlength{\textwidth}{154mm} % B6 用
\setlength{\textheight}{108mm} % B6 用
\setlength{\headsep}{0mm} % 
\setlength{\headheight}{0mm} % 
\setlength{\topskip}{0mm} % 
\setlength{\parskip}{0pt} % 
\def\labelenumi{\theenumi、} % 箇条書きのフォーマット
\parindent = 0pt % 段落下げしない

 % B6 用テンプレート読み込み

\begin{document}
% begin header
%%%%% タイトルと作者 ここから %%%%%
\begin{minipage}[c]{0.7\hsize} % タイトルは上から 7 割
    \begin{center}
    % begin title
        {\LARGE
            水産放浪歌 % タイトルを入れる
        }
        {\small 
             % 年などを入れる
        }
    % end title
    \end{center}
\end{minipage}
\begin{minipage}[c]{0.3\hsize} % 作歌作曲は上から 3 割
    \begin{flushright} % 下寄せにする
        % begin name
         % 作歌・作曲者
        % end name
    \end{flushright}
\end{minipage}
%%%%% タイトルと作者 ここまで %%%%%
% (1,2,3 了あり)
% end header

% begin body
\vspace{1.5em} % タイトル, 作者と歌詞の間に隙間を設ける
\newcommand{\linespace}{0.5em} % 行間の設定
\newcommand{\blocksize}{0.5\hsize} % 段組間の設定
%%%%% 歌詞 ここから %%%%%
% begin lilycs
\begin{enumerate} % 番号の箇条書き ここから
    \begin{minipage}[c]{\blocksize}
    
        \vspace{\linespace}
        \item
        \ruby{富貴名門}{}の\ruby{女性}{}に\ruby{恋}{}するを\\
        \ruby{純情}{}の\ruby{恋}{}と\ruby{誰}{}が\ruby{言}{}うぞ。\\
        \ruby{暗鬼紅灯}{}の\ruby{巷}{}に\ruby{彷徨}{}う\ruby{女性}{}に\\
        \ruby{恋}{}するを\ruby{不情}{}の\ruby{恋}{}と\ruby{誰}{}が\ruby{言}{}うぞ。\\
        \ruby{雨降}{}らば\ruby{雨降}{}るもよし\\
        \ruby{風吹}{}かば\ruby{風吹}{}くもよし\\
        \ruby{月下}{}の\ruby{酒場}{}にて\ruby{媚}{}を\ruby{売}{}る\ruby{女性}{}にも\\
        \ruby{純情可憐}{}なる\ruby{者}{}あれ。\\
        \ruby{女}{}の\ruby{膝枕}{}にて\ruby{一夜}{}の\ruby{快楽}{}を\\
        \ruby{共}{}に\ruby{過}{}さずんば\\
        \ruby{人生夢}{}もなければ\ruby{恋}{}もなし。\\
        \ruby{響}{}く\ruby{雷鳴}{} \ruby{握}{}る\ruby{舵輪}{}\\
        \ruby{睨}{}むコンパス\ruby{六分儀}{}\\
        \ruby{吾}{}ら\ruby{海行}{}く\ruby{鴎鳥}{} さらば\ruby{歌}{}わん\ruby{哉}{}\\
        \ruby{吾}{}らが\ruby{水産放浪歌}{}
        
        \vspace{\linespace}
        \item
        % 1.
        \ruby{心猛}{}くも\ruby{鬼神}{}ならず\\
        \ruby{男}{}と\ruby{生}{}れて\ruby{情}{}はあれど\\
        \ruby{母}{}を\ruby{見捨}{}てて\ruby{浪越}{}えてゆく\\
        \ruby{友}{}よ\ruby{兄等}{}よ\ruby{何時}{}また\ruby{会}{}わん
        
        \vspace{\linespace}
        \item
        % 2.
        \ruby{朝日夕日}{}をデッキに\ruby{浴}{}びて\\
        \ruby{続}{}く\ruby{海原一筋道}{}を\\
        \ruby{大和男子}{}が\ruby{心}{}に\ruby{秘}{}めて\\
        \ruby{行}{}くや\ruby{万里}{}の\ruby{荒波越}{}えて
        
        \vspace{\linespace}
        \item
        % 3.
        \ruby{波}{}の\ruby{彼方}{}の\ruby{南氷洋}{}は\\
        \ruby{男多恨}{}の\ruby{身}{}の\ruby{捨}{}てどころ\\
        \ruby{胸}{}に\ruby{秘}{}めたる\ruby{大願}{}あれど\\
        \ruby{行}{}きて\ruby{帰}{}らじ\ruby{望}{}みは\ruby{待}{}たじ
        
        (※
        成立事情不明なるも蒙古放浪歌
        (仲田三孝作詞、川上義彦作曲)の
        換え歌と推定される。)

    
    \end{minipage}
\end{enumerate} % 番号の箇条書き ここまで
% end lilycs
%%%%% 歌詞 ここまで %%%%%
% end body

\end{document}

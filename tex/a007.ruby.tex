\documentclass[10pt,b5j]{tarticle} % B6 縦書き
% \documentclass[10pt,b5j]{tarticle} % B6 縦書き
\AtBeginDvi{\special{papersize=128mm,182mm}} % B6 用用紙サイズ
\usepackage{otf} % Unicode で字を入力するのに必要なパッケージ
\usepackage[size=b6j]{bxpapersize} % B6 用紙サイズを指定
\usepackage[dvipdfmx]{graphicx} % 画像を挿入するためのパッケージ
\usepackage[dvipdfmx]{color} % 色をつけるためのパッケージ
\usepackage{pxrubrica} % ルビを振るためのパッケージ
\usepackage{multicol} % 複数段組を作るためのパッケージ
\setlength{\topmargin}{14mm} % 上下方向のマージン
\addtolength{\topmargin}{-1in} % 
\setlength{\oddsidemargin}{11mm} % 左右方向のマージン
\addtolength{\oddsidemargin}{-1in} % 
\setlength{\textwidth}{154mm} % B6 用
\setlength{\textheight}{108mm} % B6 用
\setlength{\headsep}{0mm} % 
\setlength{\headheight}{0mm} % 
\setlength{\topskip}{0mm} % 
\setlength{\parskip}{0pt} % 
\def\labelenumi{\theenumi、} % 箇条書きのフォーマット
\parindent = 0pt % 段落下げしない

 % B6 用テンプレート読み込み

\begin{document}
% begin header
%%%%% タイトルと作者 ここから %%%%%
\begin{minipage}[c]{0.7\hsize} % タイトルは上から 7 割
    \begin{center}
    % begin title
        {\LARGE
            桑楡哺紅に % タイトルを入れる
        }
        {\small 
            (大正十四年桜星会優勝歌) % 年などを入れる
        }
    % end title
    \end{center}
\end{minipage}
\begin{minipage}[c]{0.3\hsize} % 作歌作曲は上から 3 割
    \begin{flushright} % 下寄せにする
        % begin name
        木村英男君 作歌\\宗知康君 作曲 % 作歌・作曲者
        % end name
    \end{flushright}
\end{minipage}
%%%%% タイトルと作者 ここまで %%%%%
% (1,2 了あり)
% end header

% begin body
\vspace{1.5em} % タイトル, 作者と歌詞の間に隙間を設ける
\newcommand{\linespace}{0.5em} % 行間の設定
\newcommand{\blocksize}{0.5\hsize} % 段組間の設定
%%%%% 歌詞 ここから %%%%%
% begin lilycs
\begin{enumerate} % 番号の箇条書き ここから
    \begin{minipage}[c]{\blocksize}
    
        \vspace{\linespace}
        \item
        % 1.
        \ruby{桑楡哺紅}{}に\ruby{彩}{}なせる\\
        われ\ruby{吾}{}が\ruby{戦友}{}の\ruby{血涙史}{}\\
        そは\ruby{繚原}{}の\ruby{火}{}と\ruby{燃}{}えて\\
        \ruby{今幽貎}{}の\ruby{曠野}{}に\ruby{狂}{}ひ\\
        \ruby{凝視}{}よ\ruby{感激}{}の\ruby{胸}{}と\ruby{胸}{}\\
        \ruby{結}{}び\ruby{輝}{}く\ruby{雙眸}{}を
        
        \vspace{\linespace}
        \item
        % 2.
        \ruby{五障}{}の\ruby{霞}{}はれ\ruby{難}{}き\\
        \ruby{酣春一時}{}の\ruby{綺花}{}に\ruby{酔}{}ふ\\
        \ruby{胡蝶蒼穹}{}ゆく\ruby{夢}{}しばし\\
        \ruby{飄帆軽}{}き\ruby{景雲}{}の\ruby{船}{}\\
        \ruby{浮}{}べん\ruby{戦士}{}が\ruby{情懐}{}を\\
        \ruby{讃}{}へ\ruby{唱}{}はん\ruby{光栄}{}の\ruby{優勝歌}{}
    
    \end{minipage}
\end{enumerate} % 番号の箇条書き ここまで
% end lilycs
%%%%% 歌詞 ここまで %%%%%
% end body

\end{document}

\documentclass[10pt,b5j]{tarticle} % B6 縦書き
% \documentclass[10pt,b5j]{tarticle} % B6 縦書き
\AtBeginDvi{\special{papersize=128mm,182mm}} % B6 用用紙サイズ
\usepackage{otf} % Unicode で字を入力するのに必要なパッケージ
\usepackage[size=b6j]{bxpapersize} % B6 用紙サイズを指定
\usepackage[dvipdfmx]{graphicx} % 画像を挿入するためのパッケージ
\usepackage[dvipdfmx]{color} % 色をつけるためのパッケージ
\usepackage{pxrubrica} % ルビを振るためのパッケージ
\usepackage{multicol} % 複数段組を作るためのパッケージ
\setlength{\topmargin}{14mm} % 上下方向のマージン
\addtolength{\topmargin}{-1in} % 
\setlength{\oddsidemargin}{11mm} % 左右方向のマージン
\addtolength{\oddsidemargin}{-1in} % 
\setlength{\textwidth}{154mm} % B6 用
\setlength{\textheight}{108mm} % B6 用
\setlength{\headsep}{0mm} % 
\setlength{\headheight}{0mm} % 
\setlength{\topskip}{0mm} % 
\setlength{\parskip}{0pt} % 
\def\labelenumi{\theenumi、} % 箇条書きのフォーマット
\parindent = 0pt % 段落下げしない

 % B6 用テンプレート読み込み

\begin{document}
% begin header
%%%%% タイトルと作者 ここから %%%%%
\begin{minipage}[c]{0.7\hsize} % タイトルは上から 7 割
    \begin{center}
    % begin title
        {\LARGE
            桑楡哺紅に % タイトルを入れる
        }
        {\small 
            (大正十四年桜星会優勝歌) % 年などを入れる
        }
    % end title
    \end{center}
\end{minipage}
\begin{minipage}[c]{0.3\hsize} % 作歌作曲は上から 3 割
    \begin{flushright} % 下寄せにする
        % begin name
        木村英男君 作歌\\宗知康君 作曲 % 作歌・作曲者
        % end name
    \end{flushright}
\end{minipage}
%%%%% タイトルと作者 ここまで %%%%%
% (1,2 了あり)
% end header

% begin length
\vspace{1.5em} % タイトル, 作者と歌詞の間に隙間を設ける
\newcommand{\linespace}{0.5em} % 行間の設定
\newcommand{\blocksize}{0.5\hsize} % 段組間の設定
\newcommand{\itemmargin}{3em} % 曲番の位置調整の長さ
% end length
% begin body
%%%%% 歌詞 ここから %%%%%
\begin{enumerate} % 番号の箇条書き ここから
    \setlength{\itemindent}{\itemmargin} % 曲番の位置調整
    \begin{minipage}[c]{\blocksize}
    
        \vspace{\linespace}
        \item~\\
        % 1.
        \ruby{桑}{くわ}\ruby{楡}{にれ}\ruby{哺紅}{}に\ruby{彩}{いろどり}なせる\\
        われ\ruby{吾}{わ}が\ruby{戦友}{せんゆう}の\ruby{血涙}{けつるい}\ruby{史}{し}\\
        そは\ruby{繚原}{}の\ruby{火}{ひ}と\ruby{燃}{も}えて\\
        \ruby{今}{いま}\ruby{幽}{かそけ}\ruby{貎}{}の\ruby{曠野}{あらの}に\ruby{狂}{きょう}ひ\\
        \ruby{凝視}{ぎょうし}よ\ruby{感激}{かんげき}の\ruby{胸}{むね}と\ruby{胸}{むね}\\
        \ruby{結}{むす}び\ruby{輝}{かがや}く\ruby{雙眸}{}を
        
    \end{minipage}
    \begin{minipage}[c]{\blocksize}
        
        \vspace{\linespace}
        \item~\\
        % 2.
        \ruby{五障}{ごしょう}の\ruby{霞}{かすみ}はれ\ruby{難}{がた}き\\
        \ruby{酣}{たけなわ}\ruby{春}{はる}\ruby{一}{いち}\ruby{時}{じ}の\ruby{綺}{あやぎぬ}\ruby{花}{はな}に\ruby{酔}{よい}ふ\\
        \ruby{胡蝶}{こちょう}\ruby{蒼穹}{そうきゅう}ゆく\ruby{夢}{ゆめ}しばし\\
        \ruby{飄帆}{}\ruby{軽}{かる}き\ruby{景}{けい}\ruby{雲}{くも}の\ruby{船}{ふね}\\
        \ruby{浮}{うか}べん\ruby{戦士}{せんし}が\ruby{情}{じょう}\ruby{懐}{ふところ}を\\
        \ruby{讃}{さん}へ\ruby{唱}{}はん\ruby{光栄}{こうえい}の\ruby{優勝}{ゆうしょう}\ruby{歌}{か}
    
    \end{minipage}
\end{enumerate} % 番号の箇条書き ここまで
%%%%% 歌詞 ここまで %%%%%
% end body

\end{document}

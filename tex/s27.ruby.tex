\documentclass[10pt,b5j]{tarticle} % B6 縦書き
% \documentclass[10pt,b5j]{tarticle} % B6 縦書き
\AtBeginDvi{\special{papersize=128mm,182mm}} % B6 用用紙サイズ
\usepackage{otf} % Unicode で字を入力するのに必要なパッケージ
\usepackage[size=b6j]{bxpapersize} % B6 用紙サイズを指定
\usepackage[dvipdfmx]{graphicx} % 画像を挿入するためのパッケージ
\usepackage[dvipdfmx]{color} % 色をつけるためのパッケージ
\usepackage{pxrubrica} % ルビを振るためのパッケージ
\usepackage{multicol} % 複数段組を作るためのパッケージ
\setlength{\topmargin}{14mm} % 上下方向のマージン
\addtolength{\topmargin}{-1in} % 
\setlength{\oddsidemargin}{11mm} % 左右方向のマージン
\addtolength{\oddsidemargin}{-1in} % 
\setlength{\textwidth}{154mm} % B6 用
\setlength{\textheight}{108mm} % B6 用
\setlength{\headsep}{0mm} % 
\setlength{\headheight}{0mm} % 
\setlength{\topskip}{0mm} % 
\setlength{\parskip}{0pt} % 
\def\labelenumi{\theenumi、} % 箇条書きのフォーマット
\parindent = 0pt % 段落下げしない

 % B6 用テンプレート読み込み

\begin{document}
% begin header
%%%%% タイトルと作者 ここから %%%%%
\begin{minipage}[c]{0.7\hsize} % タイトルは上から 7 割
    \begin{center}
    % begin title
        {\LARGE
            永遠の水のひろごり % タイトルを入れる
        }
        {\small 
            (昭和二十七年寮歌) % 年などを入れる
        }
    % end title
    \end{center}
\end{minipage}
\begin{minipage}[c]{0.3\hsize} % 作歌作曲は上から 3 割
    \begin{flushright} % 下寄せにする
        % begin name
        村上啓司君 作歌\\田畑実君 作曲 % 作歌・作曲者
        % end name
    \end{flushright}
\end{minipage}
%%%%% タイトルと作者 ここまで %%%%%
% (1 繰り返しなし)
% end header

% begin length
\vspace{1.5em} % タイトル, 作者と歌詞の間に隙間を設ける
\newcommand{\linespace}{0.5em} % 行間の設定
\newcommand{\blocksize}{0.5\hsize} % 段組間の設定
\newcommand{\itemmargin}{6em} % 曲番の位置調整の長さ
% end length
% begin body
%%%%% 歌詞 ここから %%%%%
\begin{enumerate} % 番号の箇条書き ここから
    \setlength{\itemindent}{\itemmargin} % 曲番の位置調整
    \begin{minipage}[c]{\blocksize}
    
        \vspace{\linespace}
        \item~\\
        % 1.
        \ruby{永遠}{}の\ruby{水}{}の\ruby{広}{}ごり\\
        \ruby{去}{}にし\ruby{全}{}ての\ruby{名残}{}りをしるす\\
        \ruby{陽}{}の\ruby{光水}{}の\ruby{面}{}にわたらず\\
        \ruby{厚}{}き\ruby{雲}{}の\ruby{低}{}くたれたり\\
        \ruby{大}{}いなる\ruby{水}{}と\ruby{強}{}き\ruby{風}{}との\\
        \ruby{須臾}{}なる\ruby{静}{}けさ\ruby{今}{}ぞ\ruby{破}{}れん\\
        \ruby{無限}{}の\ruby{過去}{}の\ruby{名残}{}りを\ruby{無}{}みと\\
        \ruby{今}{}こそ\ruby{吾等雄々}{}しく\ruby{立}{}たん
        
        \vspace{\linespace}
        \item~\\
        % 2.
        \ruby{再}{}びす\ruby{宣臂}{}の\ruby{叫}{}び\\
        \ruby{血}{}をもて\ruby{験}{}りし\ruby{訓}{}えを\ruby{忘}{}る\\
        \ruby{屈辱}{}の\ruby{条文}{}は\ruby{結}{}ばれ\\
        \ruby{時}{}の\ruby{声}{}の\ruby{高}{}く\ruby{顕}{}る\\
        \ruby{核崩壊}{}なる\ruby{強}{}き\ruby{力}{}は\\
        \ruby{生命}{}と\ruby{愛}{}とを\ruby{毀}{}ち\ruby{捨}{}てなん\\
        \ruby{再}{}び\ruby{過去}{}の\ruby{犯}{}ちせじと\\
        \ruby{今}{}こそ\ruby{吾等凛乎}{}と\ruby{起}{}たん
        
        \vspace{\linespace}
        \item~\\
        % 3.
        \ruby{北国}{}の\ruby{樹々}{}の\ruby{直}{}さよ\\
        \ruby{牧場}{}の\ruby{草}{}の\ruby{色}{}の\ruby{濃緑}{}さよ\\
        \ruby{永}{}き\ruby{冬厳}{}しき\ruby{試練}{}に\\
        \ruby{打}{}ち\ruby{耐}{}えたる\ruby{姿美}{}わし\\
        \ruby{潮風荒}{}べる\ruby{荒磯}{}にさえ\\
        \ruby{名}{}もなき\ruby{草木}{}の\ruby{生}{}をば\ruby{享受}{}ぬ\\
        \ruby{自然}{}の\ruby{真理}{}の\ruby{頌歌}{}を\ruby{唱}{}い\\
        \ruby{今}{}こそ\ruby{吾等深}{}く\ruby{究}{}めん
    
    \end{minipage}
\end{enumerate} % 番号の箇条書き ここまで
%%%%% 歌詞 ここまで %%%%%
% end body

\end{document}

\documentclass[10pt,b5j]{tarticle} % B6 縦書き
% \documentclass[10pt,b5j]{tarticle} % B6 縦書き
\AtBeginDvi{\special{papersize=128mm,182mm}} % B6 用用紙サイズ
\usepackage{otf} % Unicode で字を入力するのに必要なパッケージ
\usepackage[size=b6j]{bxpapersize} % B6 用紙サイズを指定
\usepackage[dvipdfmx]{graphicx} % 画像を挿入するためのパッケージ
\usepackage[dvipdfmx]{color} % 色をつけるためのパッケージ
\usepackage{pxrubrica} % ルビを振るためのパッケージ
\usepackage{multicol} % 複数段組を作るためのパッケージ
\setlength{\topmargin}{14mm} % 上下方向のマージン
\addtolength{\topmargin}{-1in} % 
\setlength{\oddsidemargin}{11mm} % 左右方向のマージン
\addtolength{\oddsidemargin}{-1in} % 
\setlength{\textwidth}{154mm} % B6 用
\setlength{\textheight}{108mm} % B6 用
\setlength{\headsep}{0mm} % 
\setlength{\headheight}{0mm} % 
\setlength{\topskip}{0mm} % 
\setlength{\parskip}{0pt} % 
\def\labelenumi{\theenumi、} % 箇条書きのフォーマット
\parindent = 0pt % 段落下げしない

 % B6 用テンプレート読み込み

\begin{document}
% begin header
%%%%% タイトルと作者 ここから %%%%%
\begin{minipage}[c]{0.7\hsize} % タイトルは上から 7 割
    \begin{center}
    % begin title
        {\LARGE
            山の四季 % タイトルを入れる
        }
        {\small 
            (山岳部部歌) % 年などを入れる
        }
    % end title
    \end{center}
\end{minipage}
\begin{minipage}[c]{0.3\hsize} % 作歌作曲は上から 3 割
    \begin{flushright} % 下寄せにする
        % begin name
        朝比奈英三君 作歌\\渡辺良一君 作曲 % 作歌・作曲者
        % end name
    \end{flushright}
\end{minipage}
%%%%% タイトルと作者 ここまで %%%%%
% % end header

% begin length
\vspace{1.5em} % タイトル, 作者と歌詞の間に隙間を設ける
\newcommand{\linespace}{0.5em} % 行間の設定
\newcommand{\blocksize}{0.5\hsize} % 段組間の設定
\newcommand{\itemmargin}{3em} % 曲番の位置調整の長さ
% end length
% begin body
%%%%% 歌詞 ここから %%%%%
\begin{enumerate} % 番号の箇条書き ここから
    \setlength{\itemindent}{\itemmargin} % 曲番の位置調整
    \begin{minipage}[c]{\blocksize}
    
        \vspace{\linespace}
        \item~\\
        % 1.
        ふぶきの\ruby{尾根}{}も \ruby{風止}{}みて\\
        \ruby{春}{}の\ruby{日}{}ざしの おとずれに\\
        \ruby{沢}{}のなだれも \ruby{静}{}まりて\\
        \ruby{雪}{}げの\ruby{沢}{}の \ruby{歌楽}{}し\\
        いざ\ruby{行}{}こう \ruby{我}{}が\ruby{友}{}よ\\
        \ruby{暑寒}{}の\ruby{尾根}{}に \ruby{芦別}{}に\\
        \ruby{北}{}の\ruby{山}{}の\\
        ざらめの\ruby{尾根}{}を \ruby{飛}{}ばそうよ
        
    \end{minipage}
    \begin{minipage}[c]{\blocksize}
        
        \vspace{\linespace}
        \item~\\
        % 2.
        \ruby{沢}{}を\ruby{登}{}りて いま\ruby{五日}{}\\
        ワラジも\ruby{足}{}に \ruby{親}{}しみぬ\\
        \ruby{三日三晩}{}の \ruby{籠城}{}も\\
        \ruby{過}{}ぎて\ruby{楽}{}しき \ruby{思}{}い\ruby{出}{}よ\\
        いざ\ruby{行}{}こう \ruby{我}{}が\ruby{友}{}よ\\
        \ruby{日高}{}の\ruby{山}{}に \ruby{夏}{}の\ruby{旅}{}に\\
        \ruby{北}{}の\ruby{山}{}の\\
        カールの\ruby{中}{}に \ruby{眠}{}ろうよ
        
    \end{minipage}
    \begin{minipage}[c]{\blocksize}
        
        \vspace{\linespace}
        \item~\\
        % 3.
        \ruby{山}{}は\ruby{紅葉}{}に \ruby{彩}{}られ\\
        \ruby{頂高}{}く \ruby{空澄}{}みぬ\\
        \ruby{新雪輝}{}く \ruby{山山}{}は\\
        いずれも\ruby{親}{}しき \ruby{友}{}だちよ\\
        いざ\ruby{行}{}こう \ruby{我}{}が\ruby{友}{}よ\\
        ニセイカウシュペにトムラウシに\\
        \ruby{北}{}の\ruby{山}{}の\\
        \ruby{沢}{}のたき\ruby{火}{}に \ruby{語}{}ろうよ
        
    \end{minipage}
    \begin{minipage}[c]{\blocksize}
        
        \vspace{\linespace}
        \item~\\
        % 4.
        \ruby{吹雪}{}も\ruby{止}{}んだ \ruby{朝}{}まだき\\
        \ruby{凍}{}ったテントを \ruby{起}{}き\ruby{出}{}でて\\
        はるかにのぞむ やせ\ruby{尾根}{}は\\
        \ruby{朝焼}{}け\ruby{燃}{}ゆる ペテガリだ\\
        いざ\ruby{行}{}こう \ruby{我}{}が\ruby{友}{}よ\\
        \ruby{氷}{}の\ruby{尾根}{}に アンザイレン\\
        \ruby{北}{}の\ruby{山}{}の\\
        \ruby{聖}{}き\ruby{頂}{} \ruby{目指}{}そうよ
    
    \end{minipage}
\end{enumerate} % 番号の箇条書き ここまで
%%%%% 歌詞 ここまで %%%%%
% end body

\end{document}

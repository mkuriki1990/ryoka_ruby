\documentclass[10pt,b5j]{tarticle} % B6 縦書き
% \documentclass[10pt,b5j]{tarticle} % B6 縦書き
\AtBeginDvi{\special{papersize=128mm,182mm}} % B6 用用紙サイズ
\usepackage{otf} % Unicode で字を入力するのに必要なパッケージ
\usepackage[size=b6j]{bxpapersize} % B6 用紙サイズを指定
\usepackage[dvipdfmx]{graphicx} % 画像を挿入するためのパッケージ
\usepackage[dvipdfmx]{color} % 色をつけるためのパッケージ
\usepackage{pxrubrica} % ルビを振るためのパッケージ
\usepackage{multicol} % 複数段組を作るためのパッケージ
\setlength{\topmargin}{14mm} % 上下方向のマージン
\addtolength{\topmargin}{-1in} % 
\setlength{\oddsidemargin}{11mm} % 左右方向のマージン
\addtolength{\oddsidemargin}{-1in} % 
\setlength{\textwidth}{154mm} % B6 用
\setlength{\textheight}{108mm} % B6 用
\setlength{\headsep}{0mm} % 
\setlength{\headheight}{0mm} % 
\setlength{\topskip}{0mm} % 
\setlength{\parskip}{0pt} % 
\def\labelenumi{\theenumi、} % 箇条書きのフォーマット
\parindent = 0pt % 段落下げしない

 % B6 用テンプレート読み込み

\begin{document}
% begin header
%%%%% タイトルと作者 ここから %%%%%
\begin{minipage}[c]{0.7\hsize} % タイトルは上から 7 割
    \begin{center}
    % begin title
        {\LARGE
            流るる光途 % タイトルを入れる
        }
        {\small 
            (大正七年桜星会歌) % 年などを入れる
        }
    % end title
    \end{center}
\end{minipage}
\begin{minipage}[c]{0.3\hsize} % 作歌作曲は上から 3 割
    \begin{flushright} % 下寄せにする
        % begin name
         % 作歌・作曲者
        % end name
    \end{flushright}
\end{minipage}
%%%%% タイトルと作者 ここまで %%%%%
% % end header

% begin length
\vspace{1.5em} % タイトル, 作者と歌詞の間に隙間を設ける
\newcommand{\linespace}{0.5em} % 行間の設定
\newcommand{\blocksize}{0.5\hsize} % 段組間の設定
\newcommand{\itemmargin}{6em} % 曲番の位置調整の長さ
% end length
% begin body
%%%%% 歌詞 ここから %%%%%
\begin{enumerate} % 番号の箇条書き ここから
    \setlength{\itemindent}{\itemmargin} % 曲番の位置調整
    \begin{minipage}[c]{\blocksize}
    
        \vspace{\linespace}
        \item~\\
        % 1.
        \ruby{流}{}るヽ\ruby{光途重}{}ね\ruby{來}{}て\\
        \ruby{星霜此處}{}に\ruby{四十年}{}\\
        \ruby{北斗}{}の\ruby{光眸}{}さす\ruby{所}{}\\
        \ruby{櫻}{}かざして\ruby{先人}{}の\\
        \ruby{樹立}{}し\ruby{歴史}{}を\ruby{偲}{}ぶ\ruby{時}{}\\
        \ruby{誰}{}か\ruby{血汐}{}の\ruby{湧}{}かざらむ
        
        \vspace{\linespace}
        \item~\\
        % 2.
        \ruby{咽}{}ぶ\ruby{悲憤}{}の\ruby{誓}{}より\\
        \ruby{早}{}や\ruby{七年}{}の\ruby{春}{}うつり\\
        \ruby{人}{}は\ruby{変遷}{}れど\ruby{三百}{}の\\
        \ruby{健兒不滅}{}の\ruby{意氣}{}を\ruby{持}{}す\\
        いでや\ruby{謳}{}はん\ruby{北州}{}の\\
        \ruby{精力}{}に\ruby{満}{}ちし\ruby{凱歌}{}を
        
        \vspace{\linespace}
        \item~\\
        % 3.
        \ruby{陽春}{}の\ruby{光}{}に\ruby{覆翼}{}まれ\\
        \ruby{嫩草萠}{}ゆる\ruby{北}{}の\ruby{郷}{}\\
        \ruby{手稲}{}の\ruby{麓健兒等}{}が\\
        \ruby{燃}{}ゆる\ruby{想}{}を\ruby{合唱}{}せば\\
        \ruby{牧場}{}の\ruby{彼方際涯}{}しらず\\
        \ruby{高鳴}{}たてヽ\ruby{響}{}きゆく
        
        \vspace{\linespace}
        \item~\\
        % 4.
        \ruby{豊平川}{}の\ruby{夏}{}の\ruby{夜}{}や\\
        \ruby{玉兎}{}の\ruby{踊}{}る\ruby{波}{}の\ruby{上}{}\\
        \ruby{自治}{}の\ruby{流}{}の\ruby{悠久}{}を\\
        \ruby{語}{}る\ruby{川邊}{}に\ruby{佇}{}めば\\
        ありし\ruby{往昔}{}を\ruby{追憶}{}へとや\\
        \ruby{古塔}{}に\ruby{響}{}く\ruby{時}{}の\ruby{音}{}
        
        \vspace{\linespace}
        \item~\\
        % 5.
        こヽ\ruby{石狩}{}の\ruby{大沃野}{}\\
        \ruby{静}{}けき\ruby{秋}{}のめぐり\ruby{來}{}て\\
        \ruby{天紺青}{}の\ruby{色}{}ふかく\\
        \ruby{地}{}は\ruby{豊穣}{}なる\ruby{平和境}{}\\
        \ruby{人}{}は\ruby{有情}{}の\ruby{美}{}しき\\
        \ruby{自然}{}の\ruby{愛}{}に\ruby{狎}{}るヽ\ruby{哉}{}
        
        \vspace{\linespace}
        \item~\\
        % 6.
        \ruby{萬里茫々雪}{}の\ruby{海}{}\\
        \ruby{白龍怒}{}り\ruby{風叫}{}ぶ\\
        \ruby{吹雪}{}にさめし\ruby{暁}{}や\\
        \ruby{迷}{}いの\ruby{雲}{}をおしひらき\\
        \ruby{常世}{}の\ruby{幸}{}を\ruby{惠}{}むなる\\
        おヽ\ruby{紅}{}の\ruby{朝日影}{}
        
        \vspace{\linespace}
        \item~\\
        % 7.
        \ruby{北辰冴}{}ゆる\ruby{夕}{}まぐれ\\
        ボーイズ ビイ アンビシャスの\\
        \ruby{崇高}{}き\ruby{教}{}を\ruby{胸}{}に\ruby{秘}{}め\\
        エルムの\ruby{梢}{}とことばの\\
        \ruby{自由}{}の\ruby{調聴}{}くところ\\
        \ruby{若}{}き\ruby{生命}{}を\ruby{誇}{}らばや
    
    \end{minipage}
\end{enumerate} % 番号の箇条書き ここまで
%%%%% 歌詞 ここまで %%%%%
% end body

\end{document}

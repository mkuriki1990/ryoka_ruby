\documentclass[10pt,b5j]{tarticle} % B6 縦書き
% \documentclass[10pt,b5j]{tarticle} % B6 縦書き
\AtBeginDvi{\special{papersize=128mm,182mm}} % B6 用用紙サイズ
\usepackage{otf} % Unicode で字を入力するのに必要なパッケージ
\usepackage[size=b6j]{bxpapersize} % B6 用紙サイズを指定
\usepackage[dvipdfmx]{graphicx} % 画像を挿入するためのパッケージ
\usepackage[dvipdfmx]{color} % 色をつけるためのパッケージ
\usepackage{pxrubrica} % ルビを振るためのパッケージ
\usepackage{multicol} % 複数段組を作るためのパッケージ
\setlength{\topmargin}{14mm} % 上下方向のマージン
\addtolength{\topmargin}{-1in} % 
\setlength{\oddsidemargin}{11mm} % 左右方向のマージン
\addtolength{\oddsidemargin}{-1in} % 
\setlength{\textwidth}{154mm} % B6 用
\setlength{\textheight}{108mm} % B6 用
\setlength{\headsep}{0mm} % 
\setlength{\headheight}{0mm} % 
\setlength{\topskip}{0mm} % 
\setlength{\parskip}{0pt} % 
\def\labelenumi{\theenumi、} % 箇条書きのフォーマット
\parindent = 0pt % 段落下げしない

 % B6 用テンプレート読み込み

\begin{document}
% begin header
%%%%% タイトルと作者 ここから %%%%%
\begin{minipage}[c]{0.7\hsize} % タイトルは上から 7 割
    \begin{center}
    % begin title
        {\LARGE
            六華ぞ窓に % タイトルを入れる
        }
        {\small 
            (平成7年度寮歌) % 年などを入れる
        }
    % end title
    \end{center}
\end{minipage}
\begin{minipage}[c]{0.3\hsize} % 作歌作曲は上から 3 割
    \begin{flushright} % 下寄せにする
        % begin name
        宇野直哉君 作歌\\永田将人君 作曲 % 作歌・作曲者
        % end name
    \end{flushright}
\end{minipage}
%%%%% タイトルと作者 ここまで %%%%%
% (1,5 繰り返しなし)
% end header

% begin body
\vspace{1.5em} % タイトル, 作者と歌詞の間に隙間を設ける
\newcommand{\linespace}{0.5em} % 行間の設定
\newcommand{\blocksize}{0.5\hsize} % 段組間の設定
%%%%% 歌詞 ここから %%%%%
% begin lilycs
\begin{enumerate} % 番号の箇条書き ここから
    \begin{minipage}[c]{\blocksize}
    
        \vspace{\linespace}
        \item
        % 1.
        \ruby{六華}{}ぞ\ruby{窓}{}に\ruby{刻}{}まれる\\
        \ruby{灯灯}{}ともされて\\
        \ruby{家家}{}の\ruby{街}{}に\ruby{散}{}るほど\\
        まみえんとすれば\\
        \ruby{迷走}{}の\ruby{士}{}と\ruby{初}{}なる\ruby{乙女}{}\\
        \ruby{鈍}{}き\ruby{銀}{}なる\ruby{空}{}の\ruby{下}{}\\
        \ruby{暖}{}かき\ruby{片隅求}{}むる\ruby{若人等}{}
        
        \vspace{\linespace}
        \item
        % 2.
        \ruby{時効}{}なき\ruby{戦争裂}{}かれたる\\
        \ruby{一会}{}の\ruby{愛}{}の\ruby{光芒}{}と\\
        \ruby{時代}{}に\ruby{澱}{}の\ruby{沈}{}むを\ruby{見}{}つつ\\
        \ruby{新興}{}の\ruby{今何}{}かを\ruby{思}{}う\\
        \ruby{世}{}にふる\ruby{柳}{}の\ruby{薄緑}{}\\
        \ruby{岸}{}に\ruby{萌}{}えただよいしだれて\\
        \ruby{音}{}もなく
        
        \vspace{\linespace}
        \item
        % 3.
        \ruby{白}{}き\ruby{岩肌}{}かいとなり\\
        \ruby{登}{}りて\ruby{伝}{}う\ruby{水}{}の\ruby{城}{}\\
        \ruby{折}{}しも\ruby{巌}{}の\ruby{潤}{}い\ruby{映}{}えて\\
        \ruby{光}{}の\ruby{花}{}の\ruby{冠受}{}くを\ruby{見}{}ゆ\\
        この\ruby{灼熱}{}よこの\ruby{碧水}{}よ\\
        たどりこし\ruby{我}{}らが\ruby{魂}{}まで\ruby{飛沫}{}せよ
        
        \vspace{\linespace}
        \item
        % 4.
        \ruby{別}{}るる\ruby{道}{}を\ruby{限}{}りとて\\
        \ruby{露}{}けき\ruby{草}{}にさし\ruby{入}{}るも\\
        \ruby{月日}{}に\ruby{添}{}えてえうち\ruby{紛}{}れず\\
        \ruby{思}{}い\ruby{乱}{}るる\ruby{面影}{}に\ruby{添}{}う\\
        \ruby{友}{}の\ruby{一言軽}{}からず\\
        \ruby{肝胆相照}{}らしき\ruby{月影燦然}{}と
        
        \vspace{\linespace}
        \item
        % 5.
        \ruby{残照長}{}く\ruby{尾}{}を\ruby{引}{}けば\\
        \ruby{安}{}らぎ\ruby{満}{}ちて\ruby{夜}{}の\ruby{声}{}\\
        さらば\ruby{我}{}らが\ruby{土中}{}の\ruby{碧}{}の\\
        その\ruby{重}{}みこそ\ruby{出会}{}いし\ruby{歓喜}{}\\
        \ruby{新}{}たな\ruby{一歩}{}しるしつつ\\
        \ruby{忘}{}るまじ\ruby{清}{}き\ruby{華}{}かなる\ruby{憧}{}れを
    
    \end{minipage}
\end{enumerate} % 番号の箇条書き ここまで
% end lilycs
%%%%% 歌詞 ここまで %%%%%
% end body

\end{document}

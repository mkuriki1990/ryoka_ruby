\documentclass[10pt,b5j]{tarticle} % B6 縦書き
% \documentclass[10pt,b5j]{tarticle} % B6 縦書き
\AtBeginDvi{\special{papersize=128mm,182mm}} % B6 用用紙サイズ
\usepackage{otf} % Unicode で字を入力するのに必要なパッケージ
\usepackage[size=b6j]{bxpapersize} % B6 用紙サイズを指定
\usepackage[dvipdfmx]{graphicx} % 画像を挿入するためのパッケージ
\usepackage[dvipdfmx]{color} % 色をつけるためのパッケージ
\usepackage{pxrubrica} % ルビを振るためのパッケージ
\usepackage{multicol} % 複数段組を作るためのパッケージ
\setlength{\topmargin}{14mm} % 上下方向のマージン
\addtolength{\topmargin}{-1in} % 
\setlength{\oddsidemargin}{11mm} % 左右方向のマージン
\addtolength{\oddsidemargin}{-1in} % 
\setlength{\textwidth}{154mm} % B6 用
\setlength{\textheight}{108mm} % B6 用
\setlength{\headsep}{0mm} % 
\setlength{\headheight}{0mm} % 
\setlength{\topskip}{0mm} % 
\setlength{\parskip}{0pt} % 
\def\labelenumi{\theenumi、} % 箇条書きのフォーマット
\parindent = 0pt % 段落下げしない

 % B6 用テンプレート読み込み

\begin{document}
% begin header
%%%%% タイトルと作者 ここから %%%%%
\begin{minipage}[c]{0.7\hsize} % タイトルは上から 7 割
    \begin{center}
    % begin title
        {\LARGE
            爪紅の黎明の風 % タイトルを入れる
        }
        {\small 
            (大正十五年開学五十周年記念寮歌) % 年などを入れる
        }
    % end title
    \end{center}
\end{minipage}
\begin{minipage}[c]{0.3\hsize} % 作歌作曲は上から 3 割
    \begin{flushright} % 下寄せにする
        % begin name
        井上哲郎君 作歌\\河口忠雄君 作曲 % 作歌・作曲者
        % end name
    \end{flushright}
\end{minipage}
%%%%% タイトルと作者 ここまで %%%%%
% (1 繰り返しなし)
% end header

% begin body
\vspace{1.5em} % タイトル, 作者と歌詞の間に隙間を設ける
\newcommand{\linespace}{0.5em} % 行間の設定
\newcommand{\blocksize}{0.5\hsize} % 段組間の設定
%%%%% 歌詞 ここから %%%%%
% begin lilycs
\begin{enumerate} % 番号の箇条書き ここから
    \begin{minipage}[c]{\blocksize}
    
        \vspace{\linespace}
        \item
        % 1.
        \ruby{爪紅}{}の\ruby{黎明}{}の\ruby{風}{}\\
        \ruby{白羽箙}{}へる\ruby{若武者}{}が\\
        \ruby{青春}{}うち\ruby{慕}{}ふ\ruby{風情}{}あり\\
        \ruby{赤}{}き\ruby{血潮}{}の\ruby{溢}{}れては\\
        \ruby{北溟}{}の\ruby{城花}{}も\ruby{散}{}る\\
        \ruby{香}{}ふ\ruby{二十}{}を\ruby{愛}{}しむ\ruby{哉}{}
        
        \vspace{\linespace}
        \item
        % 2.
        いとうら\ruby{若}{}き\ruby{觴}{}を\\
        \ruby{逆巻}{}く\ruby{潮}{}に\ruby{浮}{}かべつつ\\
        \ruby{宿命}{}の\ruby{羈絆解}{}きうけば\\
        \ruby{無量無限}{}の\ruby{陽光}{}に\\
        \ruby{真白}{}き\ruby{鳥}{}のゆく\ruby{如}{}く\\
        \ruby{北海}{}の\ruby{置}{}くの\ruby{流離}{}よ
        
        \vspace{\linespace}
        \item
        % 3.
        ああ\ruby{黒潮}{}や、さざれ\ruby{床}{}\\
        いるかの\ruby{夢}{}に\ruby{身}{}をひそめ\\
        \ruby{郷愁空}{}に\ruby{盃}{}もなく\\
        \ruby{熱}{}ある\ruby{友}{}を\ruby{求}{}めては\\
        \ruby{溢}{}るる\ruby{涙袖}{}うちて\\
        \ruby{吾等}{}が\ruby{寮歌}{}を\ruby{含}{}むなり
        
        \vspace{\linespace}
        \item
        % 4.
        \ruby{淡紅}{}の\ruby{花陰}{}に\\
        \ruby{裸形}{}の\ruby{友}{}も\ruby{集}{}ひして\\
        \ruby{生}{}くる\ruby{力}{}の\ruby{征矢}{}ひけば\\
        \ruby{牧羊神}{}も\ruby{醒}{}めつらむ\\
        \ruby{孤雲}{}の\ruby{彼方}{}はるけくも\\
        \ruby{胸}{}うちふるふ\ruby{希望}{}あり
        
        \vspace{\linespace}
        \item
        % 5.
        されど\ruby{悲恋}{}の\ruby{跉跰}{}は\\
        \ruby{浩蕩雲}{}にむせびけむ\\
        \ruby{断腸}{}を\ruby{撞}{}かむ\ruby{巨鐘}{}の\\
        \ruby{鐘楼}{}の\ruby{夢}{}やいかなれば\\
        \ruby{嘆}{}かひ\ruby{濡}{}るる\ruby{月魄}{}に\\
        \ruby{秘}{}めにし\ruby{曲}{}をつたへずや
        
        \vspace{\linespace}
        \item
        % 6.
        \ruby{嗟呼青雲}{}を\ruby{吟}{}じなば\\
        \ruby{月毛}{}の\ruby{駒}{}に\ruby{星止}{}めむ\\
        \ruby{秋水義}{}に\ruby{反}{}きては\\
        \ruby{破波}{}の\ruby{想堪}{}へがたく\\
        \ruby{酒盃}{}にむせぶ\ruby{白雲}{}の\\
        \ruby{乱}{}るる\ruby{酔歌}{}に\ruby{恨}{}みあり
        
        \vspace{\linespace}
        \item
        % 7.
        \ruby{大熊星}{}のさすほとり\\
        \ruby{快楽}{}の\ruby{濁舟}{}ひくく\ruby{見}{}て\\
        \ruby{舞}{}ひつ\ruby{歌}{}ひつ\ruby{白羊}{}の\\
        あこがれ\ruby{楡}{}の\ruby{駅路}{}に\\
        \ruby{自由}{}の\ruby{泉青春}{}を\\
        うち\ruby{連}{}れ\ruby{汲}{}まん\ruby{誇}{}り\ruby{哉}{}
    
    \end{minipage}
\end{enumerate} % 番号の箇条書き ここまで
% end lilycs
%%%%% 歌詞 ここまで %%%%%
% end body

\end{document}

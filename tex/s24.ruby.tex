\documentclass[10pt,b5j]{tarticle} % B6 縦書き
% \documentclass[10pt,b5j]{tarticle} % B6 縦書き
\AtBeginDvi{\special{papersize=128mm,182mm}} % B6 用用紙サイズ
\usepackage{otf} % Unicode で字を入力するのに必要なパッケージ
\usepackage[size=b6j]{bxpapersize} % B6 用紙サイズを指定
\usepackage[dvipdfmx]{graphicx} % 画像を挿入するためのパッケージ
\usepackage[dvipdfmx]{color} % 色をつけるためのパッケージ
\usepackage{pxrubrica} % ルビを振るためのパッケージ
\usepackage{multicol} % 複数段組を作るためのパッケージ
\setlength{\topmargin}{14mm} % 上下方向のマージン
\addtolength{\topmargin}{-1in} % 
\setlength{\oddsidemargin}{11mm} % 左右方向のマージン
\addtolength{\oddsidemargin}{-1in} % 
\setlength{\textwidth}{154mm} % B6 用
\setlength{\textheight}{108mm} % B6 用
\setlength{\headsep}{0mm} % 
\setlength{\headheight}{0mm} % 
\setlength{\topskip}{0mm} % 
\setlength{\parskip}{0pt} % 
\def\labelenumi{\theenumi、} % 箇条書きのフォーマット
\parindent = 0pt % 段落下げしない

 % B6 用テンプレート読み込み

\begin{document}
% begin header
%%%%% タイトルと作者 ここから %%%%%
\begin{minipage}[c]{0.7\hsize} % タイトルは上から 7 割
    \begin{center}
    % begin title
        {\LARGE
            彷徨へる心のままに % タイトルを入れる
        }
        {\small 
            (昭和二十四年寮歌) % 年などを入れる
        }
    % end title
    \end{center}
\end{minipage}
\begin{minipage}[c]{0.3\hsize} % 作歌作曲は上から 3 割
    \begin{flushright} % 下寄せにする
        % begin name
        池田基君 作歌\\伊藤嘉弘君 作曲 % 作歌・作曲者
        % end name
    \end{flushright}
\end{minipage}
%%%%% タイトルと作者 ここまで %%%%%
% (序,冬,結 繰り返しなし)
% end header

% begin length
\vspace{1.5em} % タイトル, 作者と歌詞の間に隙間を設ける
\newcommand{\linespace}{0.5em} % 行間の設定
\newcommand{\blocksize}{0.5\hsize} % 段組間の設定
\newcommand{\itemmargin}{6em} % 曲番の位置調整の長さ
% end length
% begin body
%%%%% 歌詞 ここから %%%%%
\begin{enumerate} % 番号の箇条書き ここから
    \setlength{\itemindent}{\itemmargin} % 曲番の位置調整
    \begin{minipage}[c]{\blocksize}
    
        \vspace{\linespace}
        \item~\\
        % \ruby{序}{}.
        \ruby{彷徨}{}へる\ruby{心}{}のままに\\
        \ruby{見返}{}りの\ruby{陵}{}を\ruby{登}{}れば\\
        \ruby{野}{}は\ruby{遙}{}か\ruby{去}{}にし\ruby{日}{}の\ruby{面影}{}\\
        \ruby{簫々}{}の\ruby{闇}{}にとけゆく\\
        \ruby{斯}{}くあるは\ruby{人}{}の\ruby{宿命}{}か\\
        \ruby{天地}{}に\ruby{星}{}の\ruby{飛}{}ぶなり
        
        \vspace{\linespace}
        \item~\\
        % \ruby{春}{}.
        \ruby{清冽}{}の\ruby{玉散}{}る\ruby{知性}{}\\
        \ruby{燃}{}え\ruby{狂}{}ふ\ruby{情熱}{}の\ruby{焰}{}\\
        \ruby{若}{}き\ruby{身}{}の\ruby{裏}{}に\ruby{留}{}めて\\
        \ruby{相剋}{}の\ruby{旅}{}を\ruby{逝}{}くなり\\
        \ruby{苦悩}{}しみに\ruby{頬}{}を\ruby{濡}{}らせば\\
        \ruby{春雨}{}も\ruby{楡影}{}つたふ
        
        \vspace{\linespace}
        \item~\\
        % \ruby{夏}{}.
        \ruby{初夏}{}の\ruby{野}{}に\ruby{陽炎}{}たてば\\
        \ruby{痛}{}ましき\ruby{魂}{}の\ruby{疵}{}の\\
        \ruby{陽}{}に\ruby{癒}{}えて\ruby{幸福}{}は\ruby{希望}{}は\\
        \ruby{微風}{}に\ruby{咲}{}き\ruby{出}{}づる\ruby{華}{}\\
        \ruby{育}{}くみし\ruby{白珠}{}の\ruby{水}{}\\
        \ruby{浜茄}{}の\ruby{赤}{}き\ruby{血潮}{}よ
        
        \vspace{\linespace}
        \item~\\
        % \ruby{秋}{}.
        \ruby{秋深}{}き\ruby{磯}{}に\ruby{佇}{}み\\
        \ruby{汐飛沫浴}{}びし\ruby{彼}{}の\ruby{時}{}\\
        \ruby{月影}{}に\ruby{宿命解}{}かんと\\
        \ruby{友垣}{}の\ruby{誓}{}ひし\ruby{言葉}{}\\
        \ruby{斯}{}く\ruby{故}{}に\ruby{千草}{}ふみしき\\
        \ruby{寥々}{}の\ruby{孤杖}{}を\ruby{運}{}ぶ
        
        \vspace{\linespace}
        \item~\\
        % \ruby{冬}{}.
        \ruby{雪}{}の\ruby{舞}{}ふ\ruby{砂丘薄}{}れて\\
        \ruby{光輝}{}なき\ruby{旧}{}りし\ruby{仕種}{}は\\
        \ruby{忘却}{}の\ruby{寄}{}する\ruby{汐音}{}に\\
        \ruby{消}{}え\ruby{去}{}りぬ\ruby{名残}{}の\ruby{水際}{}\\
        \ruby{叫}{}ぶには\ruby{余}{}りに\ruby{深}{}く\\
        \ruby{涙}{}には\ruby{余}{}りに\ruby{虚}{}し
        
        \vspace{\linespace}
        \item~\\
        % \ruby{結}{}.
        \ruby{三春秋}{}の\ruby{絢夢原始林影}{}に\\
        \ruby{散}{}り\ruby{果}{}てて\ruby{悲哀}{}を\ruby{秘}{}めつ\\
        \ruby{陵}{}を\ruby{去}{}る\ruby{遊子}{}の\ruby{瞳}{}\\
        \ruby{又燃}{}えぬ\ruby{愛情}{}と\ruby{決意}{}に\\
        \ruby{暁}{}の\ruby{新}{}たな\ruby{旅出}{}\\
        \ruby{永遠}{}に\ruby{時}{}は\ruby{流}{}れぬ
    
    \end{minipage}
\end{enumerate} % 番号の箇条書き ここまで
%%%%% 歌詞 ここまで %%%%%
% end body

\end{document}

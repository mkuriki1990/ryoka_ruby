\documentclass[10pt,b5j]{tarticle} % B6 縦書き
% \documentclass[10pt,b5j]{tarticle} % B6 縦書き
\AtBeginDvi{\special{papersize=128mm,182mm}} % B6 用用紙サイズ
\usepackage{otf} % Unicode で字を入力するのに必要なパッケージ
\usepackage[size=b6j]{bxpapersize} % B6 用紙サイズを指定
\usepackage[dvipdfmx]{graphicx} % 画像を挿入するためのパッケージ
\usepackage[dvipdfmx]{color} % 色をつけるためのパッケージ
\usepackage{pxrubrica} % ルビを振るためのパッケージ
\usepackage{plext} % 漢数字の enumerate を使うためのパッケージ
\usepackage{multicol} % 複数段組を作るためのパッケージ
\setlength{\topmargin}{14mm} % 上下方向のマージン
\addtolength{\topmargin}{-1in} % 
\setlength{\oddsidemargin}{11mm} % 左右方向のマージン
\addtolength{\oddsidemargin}{-1in} % 
\setlength{\textwidth}{154mm} % B6 用
\setlength{\textheight}{108mm} % B6 用
\setlength{\headsep}{0mm} % 
\setlength{\headheight}{0mm} % 
\setlength{\topskip}{0mm} % 
\setlength{\parskip}{0pt} % 
\def\theenumi{\Kanji{enumi}} % 箇条書きのフォーマットを漢数字に変更
\parindent = 0pt % 段落下げしない
\pagestyle{empty} % すべてのページ番号を消去
% \renewcommand{\baselinestretch}{0.9} % 行間の倍率
 % B6 用テンプレート読み込み

\begin{document}
% begin header
%%%%% タイトルと作者 ここから %%%%%
\begin{minipage}[c]{0.7\hsize} % タイトルは上から 7 割
    \begin{center}
    % begin title
        {\LARGE
            医専逍遥歌 % タイトルを入れる
        }
        {\small 
            (昭和十四年七月) % 年などを入れる
        }
    % end title
    \end{center}
\end{minipage}
\begin{minipage}[c]{0.3\hsize} % 作歌作曲は上から 3 割
    \begin{flushright} % 下寄せにする
        % begin name
        谷本恒一郎君 作歌\\向井弘君 作曲 % 作歌・作曲者
        % end name
    \end{flushright}
\end{minipage}
%%%%% タイトルと作者 ここまで %%%%%
% % end header

% begin length
\vspace{1.5em} % タイトル, 作者と歌詞の間に隙間を設ける
\newcommand{\linespace}{0.5em} % 行間の設定
\newcommand{\blocksize}{0.5\hsize} % 段組間の設定
\newcommand{\itemmargin}{3em} % 曲番の位置調整の長さ
% end length
% begin body
%%%%% 歌詞 ここから %%%%%
\begin{enumerate} % 番号の箇条書き ここから
    \setlength{\itemindent}{\itemmargin} % 曲番の位置調整
    \begin{minipage}[c]{\blocksize}
    
        \vspace{\linespace}
        \item~\\
        % 1.
        楡の梢に影さして\\
        此処北溟の天地にも\\
        岡の牧場に萠え出でし\\
        生命の群を思ふ時\\
        爛漫香る花の色
        
    \end{minipage}
    \begin{minipage}[c]{\blocksize}
        
        \vspace{\linespace}
        \item~\\
        % 2.
        谷間に香る鈴蘭の\\
        清き姿を求めつつ\\
        遥か山路を分け入れば\\
        故郷遠く陽は落ちて\\
        游子に離郷の涙あり
        
    \end{minipage}
    \begin{minipage}[c]{\blocksize}
        
        \vspace{\linespace}
        \item~\\
        % 3.
        流れてやまぬ石狩の\\
        自然の黙示眺む時\\
        生命の科学に挑みつつ\\
        此処に集ひし同胞の\\
        吾等の胸に燃ゆるかな
        
    \end{minipage}
    \begin{minipage}[c]{\blocksize}
        
        \vspace{\linespace}
        \item~\\
        % 4.
        ここ四歳の春秋や\\
        牧場の園の延齢草\\
        ポプラ並木に月は冴え\\
        原始の森は暮れゆきて\\
        行手遥かに北斗星
    
    \end{minipage}
\end{enumerate} % 番号の箇条書き ここまで
%%%%% 歌詞 ここまで %%%%%
% end body

\end{document}

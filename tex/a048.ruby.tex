\documentclass[10pt,b5j]{tarticle} % B6 縦書き
% \documentclass[10pt,b5j]{tarticle} % B6 縦書き
\AtBeginDvi{\special{papersize=128mm,182mm}} % B6 用用紙サイズ
\usepackage{otf} % Unicode で字を入力するのに必要なパッケージ
\usepackage[size=b6j]{bxpapersize} % B6 用紙サイズを指定
\usepackage[dvipdfmx]{graphicx} % 画像を挿入するためのパッケージ
\usepackage[dvipdfmx]{color} % 色をつけるためのパッケージ
\usepackage{pxrubrica} % ルビを振るためのパッケージ
\usepackage{multicol} % 複数段組を作るためのパッケージ
\setlength{\topmargin}{14mm} % 上下方向のマージン
\addtolength{\topmargin}{-1in} % 
\setlength{\oddsidemargin}{11mm} % 左右方向のマージン
\addtolength{\oddsidemargin}{-1in} % 
\setlength{\textwidth}{154mm} % B6 用
\setlength{\textheight}{108mm} % B6 用
\setlength{\headsep}{0mm} % 
\setlength{\headheight}{0mm} % 
\setlength{\topskip}{0mm} % 
\setlength{\parskip}{0pt} % 
\def\labelenumi{\theenumi、} % 箇条書きのフォーマット
\parindent = 0pt % 段落下げしない

 % B6 用テンプレート読み込み

\begin{document}
% begin header
%%%%% タイトルと作者 ここから %%%%%
\begin{minipage}[c]{0.7\hsize} % タイトルは上から 7 割
    \begin{center}
    % begin title
        {\LARGE
            医専逍遥歌 % タイトルを入れる
        }
        {\small 
            (昭和14年7月) % 年などを入れる
        }
    % end title
    \end{center}
\end{minipage}
\begin{minipage}[c]{0.3\hsize} % 作歌作曲は上から 3 割
    \begin{flushright} % 下寄せにする
        % begin name
        谷本恒一郎君 作歌\\向井弘君 作曲 % 作歌・作曲者
        % end name
    \end{flushright}
\end{minipage}
%%%%% タイトルと作者 ここまで %%%%%
% % end header

% begin body
\vspace{1.5em} % タイトル, 作者と歌詞の間に隙間を設ける
\newcommand{\linespace}{0.5em} % 行間の設定
\newcommand{\blocksize}{0.5\hsize} % 段組間の設定
%%%%% 歌詞 ここから %%%%%
% begin lilycs
\begin{enumerate} % 番号の箇条書き ここから
    \begin{minipage}[c]{\blocksize}
    
        \vspace{\linespace}
        \item
        % 1.
        \ruby{楡}{}の\ruby{梢}{}に\ruby{影}{}さして\\
        \ruby{此処北溟}{}の\ruby{天地}{}にも\\
        \ruby{岡}{}の\ruby{牧場}{}に\ruby{萠}{}え\ruby{出}{}でし\\
        \ruby{生命}{}の\ruby{群}{}を\ruby{思}{}ふ\ruby{時}{}\\
        \ruby{爛漫香}{}る\ruby{花}{}の\ruby{色}{}
        
        \vspace{\linespace}
        \item
        % 2.
        \ruby{谷間}{}に\ruby{香}{}る\ruby{鈴蘭}{}の\\
        \ruby{清}{}き\ruby{姿}{}を\ruby{求}{}めつつ\\
        \ruby{遥}{}か\ruby{山路}{}を\ruby{分}{}け\ruby{入}{}れば\\
        \ruby{故郷遠}{}く\ruby{陽}{}は\ruby{落}{}ちて\\
        \ruby{游子}{}に\ruby{離郷}{}の\ruby{涙}{}あり
        
        \vspace{\linespace}
        \item
        % 3.
        \ruby{流}{}れてやまぬ\ruby{石狩}{}の\\
        \ruby{自然}{}の\ruby{黙示眺}{}む\ruby{時}{}\\
        \ruby{生命}{}の\ruby{科学}{}に\ruby{挑}{}みつつ\\
        \ruby{此処}{}に\ruby{集}{}ひし\ruby{同胞}{}の\\
        \ruby{吾等}{}の\ruby{胸}{}に\ruby{燃}{}ゆるかな
        
        \vspace{\linespace}
        \item
        % 4.
        ここ\ruby{四歳}{}の\ruby{春秋}{}や\\
        \ruby{牧場}{}の\ruby{園}{}の\ruby{延齢草}{}\\
        ポプラ\ruby{並木}{}に\ruby{月}{}は\ruby{冴}{}え\\
        \ruby{原始}{}の\ruby{森}{}は\ruby{暮}{}れゆきて\\
        \ruby{行手遥}{}かに\ruby{北斗星}{}
    
    \end{minipage}
\end{enumerate} % 番号の箇条書き ここまで
% end lilycs
%%%%% 歌詞 ここまで %%%%%
% end body

\end{document}

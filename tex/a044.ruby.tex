\documentclass[10pt,b5j]{tarticle} % B6 縦書き
% \documentclass[10pt,b5j]{tarticle} % B6 縦書き
\AtBeginDvi{\special{papersize=128mm,182mm}} % B6 用用紙サイズ
\usepackage{otf} % Unicode で字を入力するのに必要なパッケージ
\usepackage[size=b6j]{bxpapersize} % B6 用紙サイズを指定
\usepackage[dvipdfmx]{graphicx} % 画像を挿入するためのパッケージ
\usepackage[dvipdfmx]{color} % 色をつけるためのパッケージ
\usepackage{pxrubrica} % ルビを振るためのパッケージ
\usepackage{multicol} % 複数段組を作るためのパッケージ
\setlength{\topmargin}{14mm} % 上下方向のマージン
\addtolength{\topmargin}{-1in} % 
\setlength{\oddsidemargin}{11mm} % 左右方向のマージン
\addtolength{\oddsidemargin}{-1in} % 
\setlength{\textwidth}{154mm} % B6 用
\setlength{\textheight}{108mm} % B6 用
\setlength{\headsep}{0mm} % 
\setlength{\headheight}{0mm} % 
\setlength{\topskip}{0mm} % 
\setlength{\parskip}{0pt} % 
\def\labelenumi{\theenumi、} % 箇条書きのフォーマット
\parindent = 0pt % 段落下げしない

 % B6 用テンプレート読み込み

\begin{document}
% begin header
%%%%% タイトルと作者 ここから %%%%%
\begin{minipage}[c]{0.7\hsize} % タイトルは上から 7 割
    \begin{center}
    % begin title
        {\LARGE
            昭和11年応援歌 % タイトルを入れる
        }
        {\small 
             % 年などを入れる
        }
    % end title
    \end{center}
\end{minipage}
\begin{minipage}[c]{0.3\hsize} % 作歌作曲は上から 3 割
    \begin{flushright} % 下寄せにする
        % begin name
        河邨文一郎君 作歌\\能登君 作曲 % 作歌・作曲者
        % end name
    \end{flushright}
\end{minipage}
%%%%% タイトルと作者 ここまで %%%%%
% % end header

% begin length
\vspace{1.5em} % タイトル, 作者と歌詞の間に隙間を設ける
\newcommand{\linespace}{0.5em} % 行間の設定
\newcommand{\blocksize}{0.5\hsize} % 段組間の設定
\newcommand{\itemmargin}{3em} % 曲番の位置調整の長さ
% end length
% begin body
%%%%% 歌詞 ここから %%%%%
\begin{enumerate} % 番号の箇条書き ここから
    \setlength{\itemindent}{\itemmargin} % 曲番の位置調整
    \begin{minipage}[c]{\blocksize}
    
        \vspace{\linespace}
        \item~\\
        % 1.
        \ruby{起}{た}つは\ruby{今}{こん}\\
        \ruby{揚}{あ}げよ\ruby{雄叫}{おたけ}び\\
        \ruby{回天}{かいてん}の\ruby{覇業}{はぎょう}\\
        \ruby{成}{な}すべき\ruby{時}{とき}ぞ\\
        \ruby{北方}{ほっぽう}の\ruby{剛}{つよし}\\
        \ruby{曠野}{あらの}を\ruby{蹴}{け}りて\\
        \ruby{図}{ず}\ruby{南}{みなみ}の\ruby{翼}{つばさ}\\
        \ruby{蒼穹}{そうきゅう}にはためく
        
    \end{minipage}
    \begin{minipage}[c]{\blocksize}
        
        \vspace{\linespace}
        \item~\\
        % 2.
        \ruby{起}{た}つは\ruby{今}{こん}\\
        \ruby{灑}{}げ\ruby{鉄血}{てっけつ}\\
        \ruby{山}{やま}\ruby{抜}{ぬ}く\ruby{力}{ちから}\\
        \ruby{腕}{うで}に\ruby{満}{み}ちぬ\\
        \ruby{精鋭}{せいえい}すぐりて\\
        \ruby{利剣}{りけん}を\ruby{執}{と}れば\\
        \ruby{相}{あい}\ruby{搏}{}つ\ruby{闘志}{とうし}\\
        \ruby{燃}{も}えて\ruby{逆}{ぎゃく}\ruby{捲}{ま}く
    
    \end{minipage}
\end{enumerate} % 番号の箇条書き ここまで
%%%%% 歌詞 ここまで %%%%%
% end body

\end{document}

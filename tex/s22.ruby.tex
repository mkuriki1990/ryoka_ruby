\documentclass[10pt,b5j]{tarticle} % B6 縦書き
% \documentclass[10pt,b5j]{tarticle} % B6 縦書き
\AtBeginDvi{\special{papersize=128mm,182mm}} % B6 用用紙サイズ
\usepackage{otf} % Unicode で字を入力するのに必要なパッケージ
\usepackage[size=b6j]{bxpapersize} % B6 用紙サイズを指定
\usepackage[dvipdfmx]{graphicx} % 画像を挿入するためのパッケージ
\usepackage[dvipdfmx]{color} % 色をつけるためのパッケージ
\usepackage{pxrubrica} % ルビを振るためのパッケージ
\usepackage{multicol} % 複数段組を作るためのパッケージ
\setlength{\topmargin}{14mm} % 上下方向のマージン
\addtolength{\topmargin}{-1in} % 
\setlength{\oddsidemargin}{11mm} % 左右方向のマージン
\addtolength{\oddsidemargin}{-1in} % 
\setlength{\textwidth}{154mm} % B6 用
\setlength{\textheight}{108mm} % B6 用
\setlength{\headsep}{0mm} % 
\setlength{\headheight}{0mm} % 
\setlength{\topskip}{0mm} % 
\setlength{\parskip}{0pt} % 
\def\labelenumi{\theenumi、} % 箇条書きのフォーマット
\parindent = 0pt % 段落下げしない

 % B6 用テンプレート読み込み

\begin{document}
% begin header
%%%%% タイトルと作者 ここから %%%%%
\begin{minipage}[c]{0.7\hsize} % タイトルは上から 7 割
    \begin{center}
    % begin title
        {\LARGE
            暁の渚離りて % タイトルを入れる
        }
        {\small 
            (昭和二十二年寮歌) % 年などを入れる
        }
    % end title
    \end{center}
\end{minipage}
\begin{minipage}[c]{0.3\hsize} % 作歌作曲は上から 3 割
    \begin{flushright} % 下寄せにする
        % begin name
        篠原昭壽君 作歌\\竹内五男君 作曲 % 作歌・作曲者
        % end name
    \end{flushright}
\end{minipage}
%%%%% タイトルと作者 ここまで %%%%%
% (1,2 了なし繰り返しあり)
% end header

% begin length
\vspace{1.5em} % タイトル, 作者と歌詞の間に隙間を設ける
\newcommand{\linespace}{0.5em} % 行間の設定
\newcommand{\blocksize}{0.5\hsize} % 段組間の設定
\newcommand{\itemmargin}{3em} % 曲番の位置調整の長さ
% end length
% begin body
%%%%% 歌詞 ここから %%%%%
\begin{enumerate} % 番号の箇条書き ここから
    \setlength{\itemindent}{\itemmargin} % 曲番の位置調整
    \begin{minipage}[c]{\blocksize}
    
        \vspace{\linespace}
        \item~\\
        % 1.
        \ruby{暁}{}の\ruby{渚離}{}りて\\
        ふるきもの\ruby{光}{}なきもの\\
        \ruby{底}{}ひなき\ruby{海}{}に\ruby{抛}{}れば\\
        いささけき\ruby{水輪}{}が\ruby{呼}{}ばふ\\
        \ruby{想}{}い\ruby{出}{}の\ruby{古}{}りし\ruby{仕草}{}に\\
        \ruby{告}{}ぐるなりいたき\ruby{別}{}れを
        
    \end{minipage}
    \begin{minipage}[c]{\blocksize}
        
        \vspace{\linespace}
        \item~\\
        % 2.
        \ruby{永遠}{}に\ruby{絶}{}ゆることなく\\
        ひたひたと\ruby{寄}{}する\ruby{波間}{}に\\
        \ruby{万象}{}のよみがへりしを\\
        はぐくみしなさけ\ruby{忘}{}れず\\
        \ruby{真実}{}の\ruby{旗幟}{}を\ruby{取}{}り\ruby{持}{}ち\\
        いゆくものひたあゆむもの
        
    \end{minipage}
    \begin{minipage}[c]{\blocksize}
        
        \vspace{\linespace}
        \item~\\
        % 3.
        さあれ\ruby{吾}{}が\ruby{幸}{}は\ruby{希望}{}は\\
        ふたたび\ruby{会}{}ふ\ruby{事}{}なしと\\
        \ruby{燃}{}ゆる\ruby{火}{}の\ruby{炎立}{}ちに\ruby{消}{}えぬ\\
        あるはただ\ruby{宿命}{}のみなる\\
        さだめ\ruby{故旅}{}を\ruby{行}{}くなり\\
        いたましきいのちと\ruby{云}{}はめ
        
    \end{minipage}
    \begin{minipage}[c]{\blocksize}
        
        \vspace{\linespace}
        \item~\\
        % 4.
        \ruby{小船}{}もて\ruby{浜伝}{}え\ruby{行}{}き\\
        \ruby{火}{}の\ruby{神}{}の\ruby{荒}{}ぶる\ruby{山}{}を\\
        \ruby{怖}{}れみてかへりみすれば\\
        たちまちに\ruby{幻惑}{}は\ruby{裂}{}け\\
        くれなゐの\ruby{血潮流}{}れて\\
        \ruby{天地}{}は\ruby{夕焼}{}けにけり
        
    \end{minipage}
    \begin{minipage}[c]{\blocksize}
        
        \vspace{\linespace}
        \item~\\
        % 5.
        \ruby{涯知}{}らぬ\ruby{海}{}さまよひて\\
        い\ruby{着}{}きしは\ruby{辛夷咲}{}く\ruby{丘}{}\\
        \ruby{友垣}{}とあつく\ruby{結}{}びて\\
        いたましき\ruby{宿命}{}とかむと\\
        ひたざまに\ruby{立}{}ちあへぐ\ruby{夜半}{}\\
        \ruby{静}{}かなり\ruby{星}{}は\ruby{降}{}りつつ
        
    \end{minipage}
    \begin{minipage}[c]{\blocksize}
        
        \vspace{\linespace}
        \item~\\
        % 6.
        \ruby{溢}{}れ\ruby{出}{}る\ruby{涙留}{}めて\\
        \ruby{丘高}{}く\ruby{秀}{}づる\ruby{草}{}の\\
        \ruby{友}{}よ\ruby{見}{}よ\ruby{紅}{}に\ruby{映}{}ゆるを\\
        \ruby{歓喜}{}に\ruby{充}{}てるそよぎを\\
        \ruby{春秋}{}は\ruby{移}{}りて\ruby{行}{}けど\\
        \ruby{睦}{}びつつ\ruby{耐}{}へてを\ruby{行}{}かな
    
    \end{minipage}
\end{enumerate} % 番号の箇条書き ここまで
%%%%% 歌詞 ここまで %%%%%
% end body

\end{document}

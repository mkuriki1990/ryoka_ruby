\documentclass[10pt,b5j]{tarticle} % B6 縦書き
% \documentclass[10pt,b5j]{tarticle} % B6 縦書き
\AtBeginDvi{\special{papersize=128mm,182mm}} % B6 用用紙サイズ
\usepackage{otf} % Unicode で字を入力するのに必要なパッケージ
\usepackage[size=b6j]{bxpapersize} % B6 用紙サイズを指定
\usepackage[dvipdfmx]{graphicx} % 画像を挿入するためのパッケージ
\usepackage[dvipdfmx]{color} % 色をつけるためのパッケージ
\usepackage{pxrubrica} % ルビを振るためのパッケージ
\usepackage{multicol} % 複数段組を作るためのパッケージ
\setlength{\topmargin}{14mm} % 上下方向のマージン
\addtolength{\topmargin}{-1in} % 
\setlength{\oddsidemargin}{11mm} % 左右方向のマージン
\addtolength{\oddsidemargin}{-1in} % 
\setlength{\textwidth}{154mm} % B6 用
\setlength{\textheight}{108mm} % B6 用
\setlength{\headsep}{0mm} % 
\setlength{\headheight}{0mm} % 
\setlength{\topskip}{0mm} % 
\setlength{\parskip}{0pt} % 
\def\labelenumi{\theenumi、} % 箇条書きのフォーマット
\parindent = 0pt % 段落下げしない

 % B6 用テンプレート読み込み

\begin{document}
% begin header
%%%%% タイトルと作者 ここから %%%%%
\begin{minipage}[c]{0.7\hsize} % タイトルは上から 7 割
    \begin{center}
    % begin title
        {\LARGE
            函館高等水産学校校歌 % タイトルを入れる
        }
        {\small 
            (昭和十二年) % 年などを入れる
        }
    % end title
    \end{center}
\end{minipage}
\begin{minipage}[c]{0.3\hsize} % 作歌作曲は上から 3 割
    \begin{flushright} % 下寄せにする
        % begin name
        土井晩翠君 作歌\\東京音楽学校君 作曲 % 作歌・作曲者
        % end name
    \end{flushright}
\end{minipage}
%%%%% タイトルと作者 ここまで %%%%%
% % end header

% begin body
\vspace{1.5em} % タイトル, 作者と歌詞の間に隙間を設ける
\newcommand{\linespace}{0.5em} % 行間の設定
\newcommand{\blocksize}{0.5\hsize} % 段組間の設定
%%%%% 歌詞 ここから %%%%%
% begin lilycs
\begin{enumerate} % 番号の箇条書き ここから
    \begin{minipage}[c]{\blocksize}
    
        \vspace{\linespace}
        \item
        % 1.
        \ruby{太平洋}{}と\ruby{日本海}{}\\
        \ruby{結}{}ぶ\ruby{海峡}{}ただなかに\\
        \ruby{臨}{}む\ruby{函館天然}{}の\\
        \ruby{恵}{}みの\ruby{勝地}{}ここにして\\
        \ruby{高等水産校}{}は\ruby{立}{}つ
        
        \vspace{\linespace}
        \item
        % 2.
        \ruby{四方海}{}なる\ruby{大日本}{}\\
        \ruby{水産特}{}にその\ruby{宝}{}\\
        ちさき\ruby{陵土}{}をいかにせん\\
        \ruby{暖流寒流}{}のりこして\\
        \ruby{健児雄}{}たけぶ\ruby{波}{}の\ruby{上}{}
        
        \vspace{\linespace}
        \item
        % 3.
        \ruby{見}{}よベーリング、アラスカは\\
        \ruby{北}{}に、\ruby{南}{}は\ruby{赤道}{}を\\
        \ruby{過}{}ぎて\ruby{自由}{}の\ruby{天}{}の\ruby{領}{}\\
        \ruby{大鵬}{}の\ruby{羽数}{}ならず\\
        \ruby{進}{}み\ruby{使命}{}に\ruby{盡}{}すべし
        
        \vspace{\linespace}
        \item
        % 4.
        \ruby{洋}{}の\ruby{東西}{}おしなべて\\
        \ruby{斯界}{}の\ruby{指導}{}われの\ruby{任}{}\\
        \ruby{高}{}き\ruby{理想}{}にあこがれの\\
        \ruby{健児}{}はぐくむ\ruby{学園}{}に\\
        \ruby{無窮}{}の\ruby{榮}{}あらしめよ
    
    \end{minipage}
\end{enumerate} % 番号の箇条書き ここまで
% end lilycs
%%%%% 歌詞 ここまで %%%%%
% end body

\end{document}

\documentclass[10pt,b5j]{tarticle} % B6 縦書き
% \documentclass[10pt,b5j]{tarticle} % B6 縦書き
\AtBeginDvi{\special{papersize=128mm,182mm}} % B6 用用紙サイズ
\usepackage{otf} % Unicode で字を入力するのに必要なパッケージ
\usepackage[size=b6j]{bxpapersize} % B6 用紙サイズを指定
\usepackage[dvipdfmx]{graphicx} % 画像を挿入するためのパッケージ
\usepackage[dvipdfmx]{color} % 色をつけるためのパッケージ
\usepackage{pxrubrica} % ルビを振るためのパッケージ
\usepackage{multicol} % 複数段組を作るためのパッケージ
\setlength{\topmargin}{14mm} % 上下方向のマージン
\addtolength{\topmargin}{-1in} % 
\setlength{\oddsidemargin}{11mm} % 左右方向のマージン
\addtolength{\oddsidemargin}{-1in} % 
\setlength{\textwidth}{154mm} % B6 用
\setlength{\textheight}{108mm} % B6 用
\setlength{\headsep}{0mm} % 
\setlength{\headheight}{0mm} % 
\setlength{\topskip}{0mm} % 
\setlength{\parskip}{0pt} % 
\def\labelenumi{\theenumi、} % 箇条書きのフォーマット
\parindent = 0pt % 段落下げしない

 % B6 用テンプレート読み込み

\begin{document}
% begin header
%%%%% タイトルと作者 ここから %%%%%
\begin{minipage}[c]{0.7\hsize} % タイトルは上から 7 割
    \begin{center}
    % begin title
        {\LARGE
            函館高等水産学校校歌 % タイトルを入れる
        }
        {\small 
            (昭和十二年) % 年などを入れる
        }
    % end title
    \end{center}
\end{minipage}
\begin{minipage}[c]{0.3\hsize} % 作歌作曲は上から 3 割
    \begin{flushright} % 下寄せにする
        % begin name
        土井晩翠君 作歌\\東京音楽学校君 作曲 % 作歌・作曲者
        % end name
    \end{flushright}
\end{minipage}
%%%%% タイトルと作者 ここまで %%%%%
% % end header

% begin length
\vspace{1.5em} % タイトル, 作者と歌詞の間に隙間を設ける
\newcommand{\linespace}{0.5em} % 行間の設定
\newcommand{\blocksize}{0.5\hsize} % 段組間の設定
\newcommand{\itemmargin}{3em} % 曲番の位置調整の長さ
% end length
% begin body
%%%%% 歌詞 ここから %%%%%
\begin{enumerate} % 番号の箇条書き ここから
    \setlength{\itemindent}{\itemmargin} % 曲番の位置調整
    \begin{minipage}[c]{\blocksize}
    
        \vspace{\linespace}
        \item~\\
        % 1.
        \ruby{太平洋}{たいへいよう}と\ruby{日本海}{にほんかい}\\
        \ruby{結}{むす}ぶ\ruby{海峡}{かいきょう}ただなかに\\
        \ruby{臨}{のぞ}む\ruby{函館}{はこだて}\ruby{天然}{てんねん}の\\
        \ruby{恵}{めぐ}みの\ruby{勝地}{しょうち}ここにして\\
        \ruby{高等}{こうとう}\ruby{水産}{すいさん}\ruby{校}{こう}は\ruby{立}{た}つ
        
    \end{minipage}
    \begin{minipage}[c]{\blocksize}
        
        \vspace{\linespace}
        \item~\\
        % 2.
        \ruby{四方}{しほう}\ruby{海}{うみ}なる\ruby{大}{だい}\ruby{日本}{にっぽん}\\
        \ruby{水産}{すいさん}\ruby{特}{とく}にその\ruby{宝}{たから}\\
        ちさき\ruby{陵}{りょう}\ruby{土}{ど}をいかにせん\\
        \ruby{暖流}{だんりゅう}\ruby{寒流}{かんりゅう}のりこして\\
        \ruby{健児}{けんじ}\ruby{雄}{ゆう}たけぶ\ruby{波}{は}の\ruby{上}{うえ}
        
    \end{minipage}
    \begin{minipage}[c]{\blocksize}
        
        \vspace{\linespace}
        \item~\\
        % 3.
        \ruby{見}{み}よベーリング、アラスカは\\
        \ruby{北}{きた}に、\ruby{南}{みなみ}は\ruby{赤道}{せきどう}を\\
        \ruby{過}{す}ぎて\ruby{自由}{じゆう}の\ruby{天}{てん}の\ruby{領}{りょう}\\
        \ruby{大鵬}{たいほう}の\ruby{羽}{はね}\ruby{数}{すう}ならず\\
        \ruby{進}{すす}み\ruby{使命}{しめい}に\ruby{盡}{}すべし
        
    \end{minipage}
    \begin{minipage}[c]{\blocksize}
        
        \vspace{\linespace}
        \item~\\
        % 4.
        \ruby{洋}{よう}の\ruby{東西}{とうざい}おしなべて\\
        \ruby{斯界}{しかい}の\ruby{指導}{しどう}われの\ruby{任}{にん}\\
        \ruby{高}{たか}き\ruby{理想}{りそう}にあこがれの\\
        \ruby{健児}{けんじ}はぐくむ\ruby{学園}{がくえん}に\\
        \ruby{無窮}{むきゅう}の\ruby{榮}{さかえ}あらしめよ
    
    \end{minipage}
\end{enumerate} % 番号の箇条書き ここまで
%%%%% 歌詞 ここまで %%%%%
% end body

\end{document}

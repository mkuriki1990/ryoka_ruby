\documentclass[10pt,b5j]{tarticle} % B6 縦書き
% \documentclass[10pt,b5j]{tarticle} % B6 縦書き
\AtBeginDvi{\special{papersize=128mm,182mm}} % B6 用用紙サイズ
\usepackage{otf} % Unicode で字を入力するのに必要なパッケージ
\usepackage[size=b6j]{bxpapersize} % B6 用紙サイズを指定
\usepackage[dvipdfmx]{graphicx} % 画像を挿入するためのパッケージ
\usepackage[dvipdfmx]{color} % 色をつけるためのパッケージ
\usepackage{pxrubrica} % ルビを振るためのパッケージ
\usepackage{multicol} % 複数段組を作るためのパッケージ
\setlength{\topmargin}{14mm} % 上下方向のマージン
\addtolength{\topmargin}{-1in} % 
\setlength{\oddsidemargin}{11mm} % 左右方向のマージン
\addtolength{\oddsidemargin}{-1in} % 
\setlength{\textwidth}{154mm} % B6 用
\setlength{\textheight}{108mm} % B6 用
\setlength{\headsep}{0mm} % 
\setlength{\headheight}{0mm} % 
\setlength{\topskip}{0mm} % 
\setlength{\parskip}{0pt} % 
\def\labelenumi{\theenumi、} % 箇条書きのフォーマット
\parindent = 0pt % 段落下げしない

 % B6 用テンプレート読み込み

\begin{document}
% begin header
%%%%% タイトルと作者 ここから %%%%%
\begin{minipage}[c]{0.7\hsize} % タイトルは上から 7 割
    \begin{center}
    % begin title
        {\LARGE
            ああ青春の歓喜を % タイトルを入れる
        }
        {\small 
            (大正十五年寮歌) % 年などを入れる
        }
    % end title
    \end{center}
\end{minipage}
\begin{minipage}[c]{0.3\hsize} % 作歌作曲は上から 3 割
    \begin{flushright} % 下寄せにする
        % begin name
        木村左京君 作歌\\牧野千代治君 作曲 % 作歌・作曲者
        % end name
    \end{flushright}
\end{minipage}
%%%%% タイトルと作者 ここまで %%%%%
% (1,2,5 了あり)
% end header

% begin length
\vspace{1.5em} % タイトル, 作者と歌詞の間に隙間を設ける
\newcommand{\linespace}{0.5em} % 行間の設定
\newcommand{\blocksize}{0.5\hsize} % 段組間の設定
\newcommand{\itemmargin}{3em} % 曲番の位置調整の長さ
% end length
% begin body
%%%%% 歌詞 ここから %%%%%
\begin{enumerate} % 番号の箇条書き ここから
    \setlength{\itemindent}{\itemmargin} % 曲番の位置調整
    \begin{minipage}[c]{\blocksize}
    
        \vspace{\linespace}
        \item~\\
        % 1.
        ああ\ruby{青春}{せいしゅん}の\ruby{歓喜}{かんき}を\\
        \ruby{宴}{うたげ}の\ruby{酔}{よい}ひと\ruby{言}{ひとこと}ふは\ruby{誰}{だれ}れ\\
        \ruby{我}{わが}が\ruby{行}{い}く\ruby{方}{ほう}の\ruby{遠}{とお}ければ\\
        しばしこの\ruby{舎}{しゃ}に\ruby{憩}{いこい}ひして\\
        \ruby{草}{くさ}を\ruby{茵}{しとね}の\ruby{旅枕}{たびまくら}\\
        \ruby{明日}{あした}の\ruby{旅路}{たびじ}を\ruby{夢}{ゆめ}に\ruby{見}{み}ん
        
    \end{minipage}
    \begin{minipage}[c]{\blocksize}
        
        \vspace{\linespace}
        \item~\\
        % 2.
        \ruby{曠野}{あらの}に\ruby{萠}{めぐむ}ゆる\ruby{若草}{わかくさ}の\\
        しらべゆかしき\ruby{喜}{よろこ}びを\\
        そよ\ruby{吹}{そよふ}く\ruby{風}{かぜ}に\ruby{寄}{やどりき}するとき\\
        うららかに\ruby{照}{て}る\ruby{春}{はる}の\ruby{日}{ひ}は\\
        \ruby{霞}{かすみ}の\ruby{奥}{おく}にまどろみて\\
        \ruby{光}{ひかり}の\ruby{波}{なみ}は\ruby{野}{の}に\ruby{充}{あ}てり
        
    \end{minipage}
    \begin{minipage}[c]{\blocksize}
        
        \vspace{\linespace}
        \item~\\
        % 3.
        \ruby{故郷}{こきょう}の\ruby{空}{そら}は\ruby{見}{み}えねども\\
        ただ\ruby{野}{の}は\ruby{広}{ひろ}く\ruby{路}{みち}\ruby{遠}{とお}し\\
        \ruby{彼方}{かなた}の\ruby{国}{くに}に\ruby{孜々}{しし}として\\
        \ruby{歩}{あゆ}みつづくる\ruby{行人}{こうじん}は\\
        \ruby{行手}{ゆくて}の\ruby{空}{そら}に\ruby{湧}{わ}き\ruby{出}{で}づる\\
        \ruby{光}{ひかり}の\ruby{雲}{くも}を\ruby{如何}{いかが}に\ruby{見}{み}る
        
    \end{minipage}
    \begin{minipage}[c]{\blocksize}
        
        \vspace{\linespace}
        \item~\\
        % 4.
        \ruby{望}{もち}の\ruby{光}{ひかり}\ruby{見}{み}えざれば\\
        \ruby{世}{よ}は\ruby{永劫}{えいごう}に\ruby{常闇}{とこやみ}か\\
        \ruby{我}{わ}が\ruby{清純}{せいじゅん}の\ruby{魂}{たましい}の\\
        \ruby{撓}{たわ}まぬ\ruby{旅}{たび}は\ruby{麗}{うるわ}しく\\
        \ruby{頑迷}{がんめい}の\ruby{徒}{と}も\ruby{起}{お}き\ruby{出}{で}でて\\
        \ruby{我}{わが}\ruby{等}{とう}の\ruby{群}{ぐん}に\ruby{加}{か}はらん
        
    \end{minipage}
    \begin{minipage}[c]{\blocksize}
        
        \vspace{\linespace}
        \item~\\
        % 5.
        あはれゆかしき\ruby{人}{ひと}の\ruby{世}{よ}や\\
        \ruby{夜}{よ}ふけの\ruby{街}{まち}を\ruby{歩}{あゆ}みつつ\\
        \ruby{遠}{とお}き\ruby{北斗}{ほくと}の\ruby{星}{ほし}を\ruby{呼}{よ}び\\
        \ruby{友}{とも}も\ruby{歌}{うた}えば\ruby{我}{わが}も\ruby{和}{わ}し\\
        \ruby{来}{く}るはここぞ\ruby{森}{もり}の\ruby{奥}{おく}\\
        \ruby{光}{ひかり}まばゆき\ruby{自治}{じち}の\ruby{燈}{}
    
    \end{minipage}
\end{enumerate} % 番号の箇条書き ここまで
%%%%% 歌詞 ここまで %%%%%
% end body

\end{document}

\documentclass[10pt,b5j]{tarticle} % B6 縦書き
% \documentclass[10pt,b5j]{tarticle} % B6 縦書き
\AtBeginDvi{\special{papersize=128mm,182mm}} % B6 用用紙サイズ
\usepackage{otf} % Unicode で字を入力するのに必要なパッケージ
\usepackage[size=b6j]{bxpapersize} % B6 用紙サイズを指定
\usepackage[dvipdfmx]{graphicx} % 画像を挿入するためのパッケージ
\usepackage[dvipdfmx]{color} % 色をつけるためのパッケージ
\usepackage{pxrubrica} % ルビを振るためのパッケージ
\usepackage{multicol} % 複数段組を作るためのパッケージ
\setlength{\topmargin}{14mm} % 上下方向のマージン
\addtolength{\topmargin}{-1in} % 
\setlength{\oddsidemargin}{11mm} % 左右方向のマージン
\addtolength{\oddsidemargin}{-1in} % 
\setlength{\textwidth}{154mm} % B6 用
\setlength{\textheight}{108mm} % B6 用
\setlength{\headsep}{0mm} % 
\setlength{\headheight}{0mm} % 
\setlength{\topskip}{0mm} % 
\setlength{\parskip}{0pt} % 
\def\labelenumi{\theenumi、} % 箇条書きのフォーマット
\parindent = 0pt % 段落下げしない

 % B6 用テンプレート読み込み

\begin{document}
% begin header
%%%%% タイトルと作者 ここから %%%%%
\begin{minipage}[c]{0.7\hsize} % タイトルは上から 7 割
    \begin{center}
    % begin title
        {\LARGE
            ああ青春の歓喜を % タイトルを入れる
        }
        {\small 
            (大正15年寮歌) % 年などを入れる
        }
    % end title
    \end{center}
\end{minipage}
\begin{minipage}[c]{0.3\hsize} % 作歌作曲は上から 3 割
    \begin{flushright} % 下寄せにする
        % begin name
        木村左京君 作歌\\牧野千代治君 作曲 % 作歌・作曲者
        % end name
    \end{flushright}
\end{minipage}
%%%%% タイトルと作者 ここまで %%%%%
% (1,2,5 了あり)
% end header

% begin body
\vspace{1.5em} % タイトル, 作者と歌詞の間に隙間を設ける
\newcommand{\linespace}{0.5em} % 行間の設定
\newcommand{\blocksize}{0.5\hsize} % 段組間の設定
%%%%% 歌詞 ここから %%%%%
% begin lilycs
\begin{enumerate} % 番号の箇条書き ここから
    \begin{minipage}[c]{\blocksize}
    
        \vspace{\linespace}
        \item
        % 1.
        ああ\ruby{青春}{}の\ruby{歓喜}{}を\\
        \ruby{宴}{}の\ruby{酔}{}ひと\ruby{言}{}ふは\ruby{誰}{}れ\\
        \ruby{我}{}が\ruby{行}{}く\ruby{方}{}の\ruby{遠}{}ければ\\
        しばしこの\ruby{舎}{}に\ruby{憩}{}ひして\\
        \ruby{草}{}を\ruby{茵}{}の\ruby{旅枕}{}\\
        \ruby{明日}{}の\ruby{旅路}{}を\ruby{夢}{}に\ruby{見}{}ん
        
        \vspace{\linespace}
        \item
        % 2.
        \ruby{曠野}{}に\ruby{萠}{}ゆる\ruby{若草}{}の\\
        しらべゆかしき\ruby{喜}{}びを\\
        そよ\ruby{吹}{}く\ruby{風}{}に\ruby{寄}{}するとき\\
        うららかに\ruby{照}{}る\ruby{春}{}の\ruby{日}{}は\\
        \ruby{霞}{}の\ruby{奥}{}にまどろみて\\
        \ruby{光}{}の\ruby{波}{}は\ruby{野}{}に\ruby{充}{}てり
        
        \vspace{\linespace}
        \item
        % 3.
        \ruby{故郷}{}の\ruby{空}{}は\ruby{見}{}えねども\\
        ただ\ruby{野}{}は\ruby{広}{}く\ruby{路遠}{}し\\
        \ruby{彼方}{}の\ruby{国}{}に\ruby{孜々}{}として\\
        \ruby{歩}{}みつづくる\ruby{行人}{}は\\
        \ruby{行手}{}の\ruby{空}{}に\ruby{湧}{}き\ruby{出}{}づる\\
        \ruby{光}{}の\ruby{雲}{}を\ruby{如何}{}に\ruby{見}{}る
        
        \vspace{\linespace}
        \item
        % 4.
        \ruby{望}{}の\ruby{光見}{}えざれば\\
        \ruby{世}{}は\ruby{永劫}{}に\ruby{常闇}{}か\\
        \ruby{我}{}が\ruby{清純}{}の\ruby{魂}{}の\\
        \ruby{撓}{}まぬ\ruby{旅}{}は\ruby{麗}{}しく\\
        \ruby{頑迷}{}の\ruby{徒}{}も\ruby{起}{}き\ruby{出}{}でて\\
        \ruby{我等}{}の\ruby{群}{}に\ruby{加}{}はらん
        
        \vspace{\linespace}
        \item
        % 5.
        あはれゆかしき\ruby{人}{}の\ruby{世}{}や\\
        \ruby{夜}{}ふけの\ruby{街}{}を\ruby{歩}{}みつつ\\
        \ruby{遠}{}き\ruby{北斗}{}の\ruby{星}{}を\ruby{呼}{}び\\
        \ruby{友}{}も\ruby{歌}{}えば\ruby{我}{}も\ruby{和}{}し\\
        \ruby{来}{}るはここぞ\ruby{森}{}の\ruby{奥}{}\\
        \ruby{光}{}まばゆき\ruby{自治}{}の\ruby{燈}{}
    
    \end{minipage}
\end{enumerate} % 番号の箇条書き ここまで
% end lilycs
%%%%% 歌詞 ここまで %%%%%
% end body

\end{document}

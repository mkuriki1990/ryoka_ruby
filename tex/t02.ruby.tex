\documentclass[10pt,b5j]{tarticle} % B6 縦書き
% \documentclass[10pt,b5j]{tarticle} % B6 縦書き
\AtBeginDvi{\special{papersize=128mm,182mm}} % B6 用用紙サイズ
\usepackage{otf} % Unicode で字を入力するのに必要なパッケージ
\usepackage[size=b6j]{bxpapersize} % B6 用紙サイズを指定
\usepackage[dvipdfmx]{graphicx} % 画像を挿入するためのパッケージ
\usepackage[dvipdfmx]{color} % 色をつけるためのパッケージ
\usepackage{pxrubrica} % ルビを振るためのパッケージ
\usepackage{multicol} % 複数段組を作るためのパッケージ
\setlength{\topmargin}{14mm} % 上下方向のマージン
\addtolength{\topmargin}{-1in} % 
\setlength{\oddsidemargin}{11mm} % 左右方向のマージン
\addtolength{\oddsidemargin}{-1in} % 
\setlength{\textwidth}{154mm} % B6 用
\setlength{\textheight}{108mm} % B6 用
\setlength{\headsep}{0mm} % 
\setlength{\headheight}{0mm} % 
\setlength{\topskip}{0mm} % 
\setlength{\parskip}{0pt} % 
\def\labelenumi{\theenumi、} % 箇条書きのフォーマット
\parindent = 0pt % 段落下げしない

 % B6 用テンプレート読み込み

\begin{document}
% begin header
%%%%% タイトルと作者 ここから %%%%%
\begin{minipage}[c]{0.7\hsize} % タイトルは上から 7 割
    \begin{center}
    % begin title
        {\LARGE
            幾世幾年 % タイトルを入れる
        }
        {\small 
            (大正二年寮歌) % 年などを入れる
        }
    % end title
    \end{center}
\end{minipage}
\begin{minipage}[c]{0.3\hsize} % 作歌作曲は上から 3 割
    \begin{flushright} % 下寄せにする
        % begin name
        木原均君 作歌\\柳沢秀雄君 作曲 % 作歌・作曲者
        % end name
    \end{flushright}
\end{minipage}
%%%%% タイトルと作者 ここまで %%%%%
% (1,5 了あり)
% end header

% begin body
\vspace{1.5em} % タイトル, 作者と歌詞の間に隙間を設ける
\newcommand{\linespace}{0.5em} % 行間の設定
\newcommand{\blocksize}{0.5\hsize} % 段組間の設定
%%%%% 歌詞 ここから %%%%%
% begin lilycs
\begin{enumerate} % 番号の箇条書き ここから
    \begin{minipage}[c]{\blocksize}
    
        \vspace{\linespace}
        \item
        % 1.
        \ruby{幾世幾年流}{}れけん\\
        \ruby{永劫隔}{}つ\ruby{後}{}までも\\
        \ruby{洋々声}{}なく\ruby{野}{}をこえて\\
        \ruby{銀河}{}に\ruby{似}{}たる\ruby{石狩}{}の\\
        \ruby{岸辺静}{}けき\ruby{夕}{}まぐれ\\
        \ruby{導}{}く\ruby{星}{}を\ruby{仰}{}がずや
        
        \vspace{\linespace}
        \item
        % 2.
        \ruby{巷}{}の\ruby{塵}{}の\ruby{跡}{}を\ruby{絶}{}ち\\
        \ruby{惰眠}{}をさます\ruby{雪嵐}{}\\
        \ruby{毘嵐万里}{}をかけりては\\
        \ruby{天地}{}もゆらぐすさまじさ\\
        \ruby{万象淋}{}しく\ruby{装}{}ひて\\
        \ruby{蕭々寒}{}き\ruby{冬景色}{}
        
        \vspace{\linespace}
        \item
        % 3.
        めぐる\ruby{月日}{}の\ruby{尾車}{}や\\
        さざめく\ruby{小河春告}{}げぬ\\
        あはれ\ruby{幸}{}ある\ruby{北}{}の\ruby{国}{}\\
        \ruby{緑}{}が\ruby{丘}{}に\ruby{打}{}ち\ruby{臥}{}して\\
        \ruby{薫}{}る\ruby{微風身}{}にうけて\\
        \ruby{常世}{}の\ruby{春}{}を\ruby{偲}{}べかし
        
        \vspace{\linespace}
        \item
        % 4.
        \ruby{清}{}き\ruby{真理}{}の\ruby{渚}{}より\\
        \ruby{無窮}{}を\ruby{照}{}らす\ruby{最高}{}の\\
        \ruby{天}{}つ\ruby{光明}{}を\ruby{探}{}り\ruby{得}{}て\\
        \ruby{迷}{}ひの\ruby{羈絆解}{}きほどき\\
        \ruby{闇}{}を\ruby{排}{}して\ruby{永遠}{}の\\
        \ruby{理想}{}の\ruby{郷}{}を\ruby{拓}{}く\ruby{可}{}し
        
        
        \vspace{\linespace}
        \item
        % 5.
        \ruby{一百意気}{}みつ\ruby{北蝦夷}{}の\\
        \ruby{健児}{}よいざや\ruby{奪}{}ひ\ruby{起}{}て\\
        \ruby{白}{}き\ruby{朔風}{}われにあり\\
        \ruby{曠野}{}に\ruby{練}{}へし\ruby{心身}{}も\\
        \ruby{歌}{}へ\ruby{壮}{}なる\ruby{勝歌}{}を\\
        \ruby{島根}{}に\ruby{高}{}く\ruby{勇}{}ましく
    
    \end{minipage}
\end{enumerate} % 番号の箇条書き ここまで
% end lilycs
%%%%% 歌詞 ここまで %%%%%
% end body

\end{document}

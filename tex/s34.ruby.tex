\documentclass[10pt,b5j]{tarticle} % B6 縦書き
% \documentclass[10pt,b5j]{tarticle} % B6 縦書き
\AtBeginDvi{\special{papersize=128mm,182mm}} % B6 用用紙サイズ
\usepackage{otf} % Unicode で字を入力するのに必要なパッケージ
\usepackage[size=b6j]{bxpapersize} % B6 用紙サイズを指定
\usepackage[dvipdfmx]{graphicx} % 画像を挿入するためのパッケージ
\usepackage[dvipdfmx]{color} % 色をつけるためのパッケージ
\usepackage{pxrubrica} % ルビを振るためのパッケージ
\usepackage{multicol} % 複数段組を作るためのパッケージ
\setlength{\topmargin}{14mm} % 上下方向のマージン
\addtolength{\topmargin}{-1in} % 
\setlength{\oddsidemargin}{11mm} % 左右方向のマージン
\addtolength{\oddsidemargin}{-1in} % 
\setlength{\textwidth}{154mm} % B6 用
\setlength{\textheight}{108mm} % B6 用
\setlength{\headsep}{0mm} % 
\setlength{\headheight}{0mm} % 
\setlength{\topskip}{0mm} % 
\setlength{\parskip}{0pt} % 
\def\labelenumi{\theenumi、} % 箇条書きのフォーマット
\parindent = 0pt % 段落下げしない

 % B6 用テンプレート読み込み

\begin{document}
% begin header
%%%%% タイトルと作者 ここから %%%%%
\begin{minipage}[c]{0.7\hsize} % タイトルは上から 7 割
    \begin{center}
    % begin title
        {\LARGE
            清き生命の % タイトルを入れる
        }
        {\small 
            (昭和34年寮歌) % 年などを入れる
        }
    % end title
    \end{center}
\end{minipage}
\begin{minipage}[c]{0.3\hsize} % 作歌作曲は上から 3 割
    \begin{flushright} % 下寄せにする
        % begin name
        吉野生壮君 作歌\\中川清吾君 作曲 % 作歌・作曲者
        % end name
    \end{flushright}
\end{minipage}
%%%%% タイトルと作者 ここまで %%%%%
% (1 繰り返しなし)
% end header

% begin body
\vspace{1.5em} % タイトル, 作者と歌詞の間に隙間を設ける
\newcommand{\linespace}{0.5em} % 行間の設定
\newcommand{\blocksize}{0.5\hsize} % 段組間の設定
%%%%% 歌詞 ここから %%%%%
% begin lilycs
\begin{enumerate} % 番号の箇条書き ここから
    \begin{minipage}[c]{\blocksize}
    
        \vspace{\linespace}
        \item
        % 1.
        \ruby{清}{}き\ruby{生命}{}の\ruby{初潮}{}に\\
        \ruby{大}{}なる\ruby{志}{}を\ruby{打}{}ち\ruby{樹}{}てむ\\
        \ruby{学}{}び\ruby{舎}{}の\ruby{緑消}{}えんとすれど\\
        \ruby{求}{}めて\ruby{止}{}まぬ\ruby{探求}{}の\\
        \ruby{心}{}は\ruby{常}{}に\ruby{変}{}る\ruby{事}{}なく
        
        \vspace{\linespace}
        \item
        % 2.
        \ruby{毅}{}き\ruby{生命}{}の\ruby{奔流}{}に\\
        \ruby{燃}{}ゆる\ruby{生命}{}を\ruby{貫}{}かむ\\
        \ruby{時代}{}の\ruby{波}{}は\ruby{荒}{}れ\ruby{狂}{}うとも\\
        \ruby{永遠}{}に\ruby{変}{}らぬ\ruby{誠実}{}の\\
        \ruby{青春}{}の\ruby{血}{}は\ruby{絶}{}ゆる\ruby{事}{}なく
        
        \vspace{\linespace}
        \item
        % 3.
        \ruby{直}{}き\ruby{生命}{}のただ\ruby{中}{}に\\
        \ruby{熱}{}き\ruby{念}{}をそそがなむ\\
        \ruby{頭上}{}に\ruby{黒}{}き\ruby{雲漂}{}えど\\
        \ruby{燃}{}えて\ruby{尽}{}きせぬ\ruby{創造}{}の\\
        \ruby{生命}{}の\ruby{光消}{}ゆる\ruby{事}{}なく
    
    \end{minipage}
\end{enumerate} % 番号の箇条書き ここまで
% end lilycs
%%%%% 歌詞 ここまで %%%%%
% end body

\end{document}

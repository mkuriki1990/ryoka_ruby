\documentclass[10pt,b5j]{tarticle} % B6 縦書き
% \documentclass[10pt,b5j]{tarticle} % B6 縦書き
\AtBeginDvi{\special{papersize=128mm,182mm}} % B6 用用紙サイズ
\usepackage{otf} % Unicode で字を入力するのに必要なパッケージ
\usepackage[size=b6j]{bxpapersize} % B6 用紙サイズを指定
\usepackage[dvipdfmx]{graphicx} % 画像を挿入するためのパッケージ
\usepackage[dvipdfmx]{color} % 色をつけるためのパッケージ
\usepackage{pxrubrica} % ルビを振るためのパッケージ
\usepackage{multicol} % 複数段組を作るためのパッケージ
\setlength{\topmargin}{14mm} % 上下方向のマージン
\addtolength{\topmargin}{-1in} % 
\setlength{\oddsidemargin}{11mm} % 左右方向のマージン
\addtolength{\oddsidemargin}{-1in} % 
\setlength{\textwidth}{154mm} % B6 用
\setlength{\textheight}{108mm} % B6 用
\setlength{\headsep}{0mm} % 
\setlength{\headheight}{0mm} % 
\setlength{\topskip}{0mm} % 
\setlength{\parskip}{0pt} % 
\def\labelenumi{\theenumi、} % 箇条書きのフォーマット
\parindent = 0pt % 段落下げしない

 % B6 用テンプレート読み込み

\begin{document}
% begin header
%%%%% タイトルと作者 ここから %%%%%
\begin{minipage}[c]{0.7\hsize} % タイトルは上から 7 割
    \begin{center}
    % begin title
        {\LARGE
            清き郷石狩の % タイトルを入れる
        }
        {\small 
            (昭和十五年桜星会三十周年記念歌) % 年などを入れる
        }
    % end title
    \end{center}
\end{minipage}
\begin{minipage}[c]{0.3\hsize} % 作歌作曲は上から 3 割
    \begin{flushright} % 下寄せにする
        % begin name
        岩崎五郎君 作歌\\呉泰治郎君 作曲 % 作歌・作曲者
        % end name
    \end{flushright}
\end{minipage}
%%%%% タイトルと作者 ここまで %%%%%
% % end header

% begin length
\vspace{1.5em} % タイトル, 作者と歌詞の間に隙間を設ける
\newcommand{\linespace}{0.5em} % 行間の設定
\newcommand{\blocksize}{0.5\hsize} % 段組間の設定
\newcommand{\itemmargin}{6em} % 曲番の位置調整の長さ
% end length
% begin body
%%%%% 歌詞 ここから %%%%%
\begin{enumerate} % 番号の箇条書き ここから
    \setlength{\itemindent}{\itemmargin} % 曲番の位置調整
    \begin{minipage}[c]{\blocksize}
    
        \vspace{\linespace}
        \item~\\
        % 1.
        \ruby{清}{}き\ruby{郷石狩}{}の\ruby{曠野}{}に\\
        うち\ruby{立}{}てし\ruby{先人}{}が\ruby{跡}{}\\
        \ruby{乾坤}{}に\ruby{時光流}{}れて\\
        \ruby{今}{}ぞなる\ruby{三十年}{}の\ruby{崇高}{}き\ruby{青史}{}よ\\
        \ruby{讃}{}へなん いざ\\
        \ruby{若}{}き\ruby{血潮}{} \ruby{燃}{}ゆる\ruby{理想}{}\\
        \ruby{世}{}を\ruby{覺醒}{}し\ruby{世}{}を\ruby{導}{}かん\\
        \ruby{傅統}{}の\ruby{楡鐘高}{}く\ruby{鳴}{}るなり
        
        \vspace{\linespace}
        \item~\\
        % 2.
        \ruby{黒}{}き\ruby{雲世}{}に\ruby{狂}{}へども\\
        \ruby{守}{}り\ruby{來}{}し\ruby{正義}{}の\ruby{精神}{}\\
        \ruby{青春}{}の\ruby{生命捧}{}げて\\
        \ruby{惠}{}ぬなり\ruby{幽遠}{}なる\ruby{眞理}{}の\ruby{秘奥}{}\\
        \ruby{高唱}{}はなん いざ\\
        \ruby{熱}{}き\ruby{感激}{} たぎる\ruby{憧憬}{}\\
        \ruby{美}{}しく\ruby{強}{}く\ruby{生}{}かばや\\
        \ruby{雄叫}{}びは\ruby{高}{}く\ruby{湧}{}くなり
        
        \vspace{\linespace}
        \item~\\
        % 3.
        \ruby{天地}{}に\ruby{暴風雨吠}{}ゆるも\\
        \ruby{東洋}{}に\ruby{夜}{}は\ruby{黎明}{}んとす\\
        \ruby{世界}{}を\ruby{救}{}ふ\ruby{大理想}{}もて\\
        うち\ruby{立}{}てん\ruby{永劫}{}の\ruby{平和}{}の\ruby{大旆}{}\\
        \ruby{叫}{}ばなん いざ\\
        \ruby{湧}{}ける\ruby{激情}{} あがる\ruby{歡喜}{}\\
        \ruby{楡}{}の\ruby{舎}{}の\ruby{健兒我等}{}は\\
        \ruby{生}{}ける\ruby{證}{}に\ruby{胸}{}は\ruby{湧}{}くなり
        
        \vspace{\linespace}
        \item~\\
        % 4.
        \ruby{悠久}{}の\ruby{時}{}の\ruby{移}{}ろひ\\
        \ruby{青春}{}のこの\ruby{瞬間}{}を\\
        \ruby{星辰澄}{}きエルムの\ruby{園}{}に\\
        \ruby{過}{}すなり\ruby{涯際}{}なき\ruby{神秘}{}の\ruby{懐中}{}に\\
        \ruby{仰}{}がなん いざ\\
        \ruby{清}{}き\ruby{生命}{} \ruby{高}{}き\ruby{意欲}{}\\
        \ruby{先人}{}の\ruby{遺}{}せし\ruby{教訓}{}\\
        \ruby{我等}{}が\ruby{魂強}{}く\ruby{打}{}つなり
    
    \end{minipage}
\end{enumerate} % 番号の箇条書き ここまで
%%%%% 歌詞 ここまで %%%%%
% end body

\end{document}

\documentclass[10pt,b5j]{tarticle} % B6 縦書き
% \documentclass[10pt,b5j]{tarticle} % B6 縦書き
\AtBeginDvi{\special{papersize=128mm,182mm}} % B6 用用紙サイズ
\usepackage{otf} % Unicode で字を入力するのに必要なパッケージ
\usepackage[size=b6j]{bxpapersize} % B6 用紙サイズを指定
\usepackage[dvipdfmx]{graphicx} % 画像を挿入するためのパッケージ
\usepackage[dvipdfmx]{color} % 色をつけるためのパッケージ
\usepackage{pxrubrica} % ルビを振るためのパッケージ
\usepackage{multicol} % 複数段組を作るためのパッケージ
\setlength{\topmargin}{14mm} % 上下方向のマージン
\addtolength{\topmargin}{-1in} % 
\setlength{\oddsidemargin}{11mm} % 左右方向のマージン
\addtolength{\oddsidemargin}{-1in} % 
\setlength{\textwidth}{154mm} % B6 用
\setlength{\textheight}{108mm} % B6 用
\setlength{\headsep}{0mm} % 
\setlength{\headheight}{0mm} % 
\setlength{\topskip}{0mm} % 
\setlength{\parskip}{0pt} % 
\def\labelenumi{\theenumi、} % 箇条書きのフォーマット
\parindent = 0pt % 段落下げしない

 % B6 用テンプレート読み込み

\begin{document}
% begin header
%%%%% タイトルと作者 ここから %%%%%
\begin{minipage}[c]{0.7\hsize} % タイトルは上から 7 割
    \begin{center}
    % begin title
        {\LARGE
            悠遠き日にあこがれて % タイトルを入れる
        }
        {\small 
            (昭和二十五年寮歌) % 年などを入れる
        }
    % end title
    \end{center}
\end{minipage}
\begin{minipage}[c]{0.3\hsize} % 作歌作曲は上から 3 割
    \begin{flushright} % 下寄せにする
        % begin name
        高倉和昭君 作歌\\金井倶光君 作曲 % 作歌・作曲者
        % end name
    \end{flushright}
\end{minipage}
%%%%% タイトルと作者 ここまで %%%%%
% (1 繰り返しなし)
% end header

% begin length
\vspace{1.5em} % タイトル, 作者と歌詞の間に隙間を設ける
\newcommand{\linespace}{0.5em} % 行間の設定
\newcommand{\blocksize}{0.5\hsize} % 段組間の設定
\newcommand{\itemmargin}{3em} % 曲番の位置調整の長さ
% end length
% begin body
%%%%% 歌詞 ここから %%%%%
\begin{enumerate} % 番号の箇条書き ここから
    \setlength{\itemindent}{\itemmargin} % 曲番の位置調整
    \begin{minipage}[c]{\blocksize}
    
        \vspace{\linespace}
        \item~\\
        % 1.
        \ruby{悠遠}{}き\ruby{日}{}にあこがれて\\
        \ruby{我}{}は\ruby{来}{}たりぬ\\
        \ruby{北国}{}の\ruby{詩}{}の\ruby{都}{}ぞ\\
        やはらかき\ruby{緑}{}の\ruby{芝生}{}\\
        \ruby{美}{}はしき\ruby{小川}{}の\ruby{畔}{}\\
        \ruby{清明}{}の\ruby{森蔭深}{}く\ruby{訪}{}ね\ruby{来}{}て\\
        \ruby{新}{}らしき\ruby{喜}{}びに\ruby{満}{}つ
        
    \end{minipage}
    \begin{minipage}[c]{\blocksize}
        
        \vspace{\linespace}
        \item~\\
        % 2.
        \ruby{讃}{}へなむ\ruby{石狩}{}の\\
        \ruby{曠野}{}に\ruby{打建}{}てし\\
        \ruby{雄大}{}なる\ruby{先人}{}が\ruby{足跡}{}\\
        \ruby{四十三回記念祭巡}{}りて\\
        \ruby{光栄}{}あれ\ruby{伝統}{}の\ruby{法燈}{}\\
        \ruby{星辰清}{}きエルムの\ruby{学園}{}に\ruby{甦}{}へりたる\\
        \ruby{鐘}{}の\ruby{音}{}は\ruby{高}{}く\ruby{鳴}{}るなり
        
    \end{minipage}
    \begin{minipage}[c]{\blocksize}
        
        \vspace{\linespace}
        \item~\\
        % 3.
        あかつきは\ruby{紫}{}の\\
        \ruby{夢}{}にけむれり\\
        \ruby{雪解}{}なる\ruby{陵}{}にのぼりて\\
        \ruby{恋}{}ひ\ruby{慕}{}ふ\ruby{意気}{}と\ruby{血汐}{}の\\
        \ruby{花香}{}る\ruby{青史}{}の\ruby{光栄}{}よ\\
        \ruby{二春}{}を\ruby{魂}{}の\ruby{故郷}{}に\ruby{契}{}りては\\
        \ruby{培}{}はん\ruby{尊}{}き\ruby{遺訓}{}
        
    \end{minipage}
    \begin{minipage}[c]{\blocksize}
        
        \vspace{\linespace}
        \item~\\
        % 4.
        \ruby{仰}{}ぎ\ruby{見}{}よ\ruby{秀}{}でたる\\
        \ruby{久遠}{}の\ruby{山河}{}\\
        \ruby{有給}{}の\ruby{時}{}の\ruby{移}{}ろひ\\
        \ruby{森蔭}{}に\ruby{心情}{}は\ruby{燃}{}えて\\
        \ruby{恵}{}むなり\ruby{真理}{}の\ruby{秘奥}{}\\
        \ruby{青春}{}の\ruby{高遠}{}き\ruby{理想}{}を\ruby{抱}{}きては\\
        \ruby{進}{}まなむ\ruby{厳}{}しかる\ruby{道}{}
    
    \end{minipage}
\end{enumerate} % 番号の箇条書き ここまで
%%%%% 歌詞 ここまで %%%%%
% end body

\end{document}

\documentclass[10pt,b5j]{tarticle} % B6 縦書き
% \documentclass[10pt,b5j]{tarticle} % B6 縦書き
\AtBeginDvi{\special{papersize=128mm,182mm}} % B6 用用紙サイズ
\usepackage{otf} % Unicode で字を入力するのに必要なパッケージ
\usepackage[size=b6j]{bxpapersize} % B6 用紙サイズを指定
\usepackage[dvipdfmx]{graphicx} % 画像を挿入するためのパッケージ
\usepackage[dvipdfmx]{color} % 色をつけるためのパッケージ
\usepackage{pxrubrica} % ルビを振るためのパッケージ
\usepackage{multicol} % 複数段組を作るためのパッケージ
\setlength{\topmargin}{14mm} % 上下方向のマージン
\addtolength{\topmargin}{-1in} % 
\setlength{\oddsidemargin}{11mm} % 左右方向のマージン
\addtolength{\oddsidemargin}{-1in} % 
\setlength{\textwidth}{154mm} % B6 用
\setlength{\textheight}{108mm} % B6 用
\setlength{\headsep}{0mm} % 
\setlength{\headheight}{0mm} % 
\setlength{\topskip}{0mm} % 
\setlength{\parskip}{0pt} % 
\def\labelenumi{\theenumi、} % 箇条書きのフォーマット
\parindent = 0pt % 段落下げしない

 % B6 用テンプレート読み込み

\begin{document}
% begin header
%%%%% タイトルと作者 ここから %%%%%
\begin{minipage}[c]{0.7\hsize} % タイトルは上から 7 割
    \begin{center}
    % begin title
        {\LARGE
            姫月に重ねて % タイトルを入れる
        }
        {\small 
            (平成二十六年度寮歌) % 年などを入れる
        }
    % end title
    \end{center}
\end{minipage}
\begin{minipage}[c]{0.3\hsize} % 作歌作曲は上から 3 割
    \begin{flushright} % 下寄せにする
        % begin name
        松元一平君 作歌\\寺尾佳隆君 作曲 % 作歌・作曲者
        % end name
    \end{flushright}
\end{minipage}
%%%%% タイトルと作者 ここまで %%%%%
% (1,2,3 繰り返しなし)
% end header

% begin length
\vspace{1.5em} % タイトル, 作者と歌詞の間に隙間を設ける
\newcommand{\linespace}{0.5em} % 行間の設定
\newcommand{\blocksize}{0.5\hsize} % 段組間の設定
\newcommand{\itemmargin}{6em} % 曲番の位置調整の長さ
% end length
% begin body
%%%%% 歌詞 ここから %%%%%
\begin{enumerate} % 番号の箇条書き ここから
    \setlength{\itemindent}{\itemmargin} % 曲番の位置調整
    \begin{minipage}[c]{\blocksize}
    
        \vspace{\linespace}
        \item~\\
        \ruby{観月過}{}ぎゆく\ruby{晩秋}{}の\ruby{夜}{}、\\
        \ruby{穹蒼}{}の\ruby{天空高}{}く\\
        \ruby{舞}{}ひたる\ruby{月}{}は\ruby{今宵満}{}つるかな。\\
        その\ruby{清輝}{}に\ruby{映}{}えし\ruby{姫}{}が\ruby{鏡水}{}は、\\
        \ruby{鹿}{}が\ruby{純瞳}{}に\ruby{宿}{}らむ。\\
        \ruby{月影}{}は\ruby{鹿}{}を\ruby{誘}{}ひ\\
        \ruby{来}{}たりしこの\ruby{神無月}{}に\\
        \ruby{何}{}をば\ruby{見}{}せむ。
        
        \vspace{\linespace}
        \item~\\
        % 1.
        \ruby{時移}{}ろひて \ruby{人世}{}は\ruby{変}{}われども\\
        \ruby{今宵}{}も\ruby{満月}{}は\ruby{我}{}らを\ruby{照}{}さむ\\
        \ruby{夜}{}の\ruby{邪帳}{}をはらはむと\\
        \ruby{流歩}{}む\ruby{汝}{}は\ruby{楡}{}に\ruby{似}{}たれど\\
        \ruby{風流}{}を\ruby{掴}{}まむ\ruby{芽}{}に\ruby{感}{}ず\\
        \ruby{風習}{}に\ruby{付和}{}せし\\
        \ruby{狗}{}と\ruby{成}{}らざらめや\\
        さて\ruby{映}{}りこむ \ruby{我}{}が\ruby{鏡瞳}{}に\\
        \ruby{風習}{}だに\ruby{愛}{}づる その\ruby{気概}{}
        
        \vspace{\linespace}
        \item~\\
        % 2.
        \ruby{清澄}{}みたる\ruby{想}{}ひ \ruby{知}{}る\ruby{由}{}もなく\\
        \ruby{今宵}{}の\ruby{三日月}{}は\ruby{川面}{}に\ruby{映}{}らむ\\
        かの\ruby{日}{}の\ruby{月影}{}とは\ruby{違}{}へども\\
        \ruby{人世}{}(よ)に\ruby{充}{}つ\ruby{解答}{}を\ruby{自}{}ずと\ruby{心得}{}\\
        \ruby{此}{}れは\ruby{汝}{}の\ruby{求望}{}にか\\
        \ruby{漲}{}る\ruby{想}{}ひ などか\ruby{劣}{}らむ\\
        さて\ruby{映}{}りこむ \ruby{我}{}が\ruby{鏡瞳}{}に\\
        \ruby{身}{}を\ruby{委}{}ねばや その\ruby{清流}{}
        
        \vspace{\linespace}
        \item~\\
        % 3.
        \ruby{静}{}と\ruby{唸}{}りし \ruby{雨澪}{}したたれば\\
        \ruby{今宵}{}も\ruby{我}{}は\ruby{朧月}{}を\ruby{仰}{}がむ\\
        \ruby{姫}{}が\ruby{麗姿}{}を\ruby{追憶}{}ふべく\\
        \ruby{汝}{}が\ruby{想}{}ひは\ruby{涙}{}と\ruby{落流}{}れ\\
        \ruby{透}{}かし\ruby{斜光}{}にさらさるる\\
        \ruby{閉}{}じなむ\ruby{凌雲}{}よ こひ\ruby{願}{}はくば\\
        さて\ruby{映}{}りこむ \ruby{我}{}が\ruby{鏡瞳}{}に\\
        \ruby{嗚呼汲}{}まれたし その\ruby{厭心}{}\\
        \ruby{悲}{}しかりけむ\ruby{晩秋}{}の\ruby{夜}{}は\\
        \ruby{月影映}{}えて\ruby{人影}{}も\ruby{追}{}ひ\ruby{得}{}じ
    
    \end{minipage}
\end{enumerate} % 番号の箇条書き ここまで
%%%%% 歌詞 ここまで %%%%%
% end body

\end{document}

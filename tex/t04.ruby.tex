\documentclass[10pt,b5j]{tarticle} % B6 縦書き
% \documentclass[10pt,b5j]{tarticle} % B6 縦書き
\AtBeginDvi{\special{papersize=128mm,182mm}} % B6 用用紙サイズ
\usepackage{otf} % Unicode で字を入力するのに必要なパッケージ
\usepackage[size=b6j]{bxpapersize} % B6 用紙サイズを指定
\usepackage[dvipdfmx]{graphicx} % 画像を挿入するためのパッケージ
\usepackage[dvipdfmx]{color} % 色をつけるためのパッケージ
\usepackage{pxrubrica} % ルビを振るためのパッケージ
\usepackage{multicol} % 複数段組を作るためのパッケージ
\setlength{\topmargin}{14mm} % 上下方向のマージン
\addtolength{\topmargin}{-1in} % 
\setlength{\oddsidemargin}{11mm} % 左右方向のマージン
\addtolength{\oddsidemargin}{-1in} % 
\setlength{\textwidth}{154mm} % B6 用
\setlength{\textheight}{108mm} % B6 用
\setlength{\headsep}{0mm} % 
\setlength{\headheight}{0mm} % 
\setlength{\topskip}{0mm} % 
\setlength{\parskip}{0pt} % 
\def\labelenumi{\theenumi、} % 箇条書きのフォーマット
\parindent = 0pt % 段落下げしない

 % B6 用テンプレート読み込み

\begin{document}
% begin header
%%%%% タイトルと作者 ここから %%%%%
\begin{minipage}[c]{0.7\hsize} % タイトルは上から 7 割
    \begin{center}
    % begin title
        {\LARGE
            時轍乾坤に % タイトルを入れる
        }
        {\small 
            (大正四年寮歌) % 年などを入れる
        }
    % end title
    \end{center}
\end{minipage}
\begin{minipage}[c]{0.3\hsize} % 作歌作曲は上から 3 割
    \begin{flushright} % 下寄せにする
        % begin name
        沢田退蔵君 作歌・作曲 % 作歌・作曲者
        % end name
    \end{flushright}
\end{minipage}
%%%%% タイトルと作者 ここまで %%%%%
% (1,2,5,6 繰り返しなし)
% end header

% begin length
\vspace{1.5em} % タイトル, 作者と歌詞の間に隙間を設ける
\newcommand{\linespace}{0.5em} % 行間の設定
\newcommand{\blocksize}{0.5\hsize} % 段組間の設定
\newcommand{\itemmargin}{3em} % 曲番の位置調整の長さ
% end length
% begin body
%%%%% 歌詞 ここから %%%%%
\begin{enumerate} % 番号の箇条書き ここから
    \setlength{\itemindent}{\itemmargin} % 曲番の位置調整
    \begin{minipage}[c]{\blocksize}
    
        \vspace{\linespace}
        \item~\\
        % 1.
        \ruby{時}{とき}\ruby{轍}{わだち}\ruby{乾坤}{けんこん}に\ruby{回}{まわ}り\ruby{来}{き}て\\
        \ruby{陽春}{ようしゅん}\ruby{駘蕩}{たいとう}のおぼろよひ\\
        \ruby{紫}{むらさき}\ruby{淡}{あわ}く\ruby{霞}{かすみ}\ruby{罩}{}め\\
        \ruby{自治}{じち}の\ruby{流}{なが}れは\ruby{永遠}{えいえん}に\\
        \ruby{若葉}{わかば}の\ruby{陰}{かげ}を\ruby{浮}{うか}べつつ\\
        \ruby{吾等}{われら}が\ruby{幸}{こう}を\ruby{祝}{しゅく}ふらん
        
    \end{minipage}
    \begin{minipage}[c]{\blocksize}
        
        \vspace{\linespace}
        \item~\\
        % 2.
        \ruby{胡馬北風}{こばほくふう}に\ruby{嘶}{いなな}きて\\
        \ruby{越鳥南枝}{えっちょうなんし}に\ruby{巣}{す}を\ruby{造}{つく}る\\
        \ruby{世}{よ}の\ruby{濁江}{にごりえ}に\ruby{逆}{ぎゃく}へる\\
        \ruby{棹}{さお}\ruby{歌}{か}の\ruby{声}{こえ}の\ruby{勇}{いさ}ましき\\
        \ruby{三}{さん}\ruby{星霜}{せいそう}の\ruby{春}{はる}のおきふしに\\
        \ruby{深}{ふか}き\ruby{感慨}{かんがい}のなからめや
        
    \end{minipage}
    \begin{minipage}[c]{\blocksize}
        
        \vspace{\linespace}
        \item~\\
        % 3.
        \ruby{紫}{むらさき}\ruby{扉}{とびら}を\ruby{出}{で}でて\ruby{霜}{しも}を\ruby{踏}{ふ}み\\
        \ruby{川流}{かわながれ}を\ruby{掬}{むす}ぶ\ruby{薪}{たきぎ}\ruby{樵}{こ}る\\
        \ruby{崇}{たかし}き\ruby{希望}{きぼう}の\ruby{若人}{わこうど}が\\
        \ruby{歓喜}{かんき}\ruby{憂苦}{ゆうく}を\ruby{共}{とも}にせし\\
        \ruby{友}{とも}\ruby{悌}{すなお}\ruby{凋}{しぼ}まぬ\ruby{松柏}{しょうはく}と\\
        \ruby{幾}{いく}\ruby{千}{せん}\ruby{代}{だい}かけて\ruby{変}{かわ}らざれ
        
    \end{minipage}
    \begin{minipage}[c]{\blocksize}
        
        \vspace{\linespace}
        \item~\\
        % 4.
        \ruby{彼}{かれ}の\ruby{邯鄲}{かんたん}の\ruby{仮}{かり}\ruby{枕}{まくら}\\
        \ruby{栄華}{えいが}の\ruby{夢}{ゆめ}も\ruby{半}{はん}にて\\
        \ruby{世}{よ}の\ruby{秋風}{あきかぜ}に\ruby{驚}{おどろ}かん\\
        \ruby{目}{め}ざす\ruby{真理}{しんり}の\ruby{高殿}{たかどの}は\\
        \ruby{遠}{とお}く\ruby{遙}{}けし\ruby{突進}{とっしん}めいざ\\
        \ruby{心}{こころ}の\ruby{駒}{こま}に\ruby{鞭打}{むちう}ちて
        
    \end{minipage}
    \begin{minipage}[c]{\blocksize}
        
        \vspace{\linespace}
        \item~\\
        % 5.
        ウラルの\ruby{彼方}{かなた}\ruby{風}{かぜ}\ruby{凄}{すご}く\\
        \ruby{陣}{じん}\ruby{雲}{くも}くらき\ruby{八街}{やちまた}は\\
        \ruby{鉄騎}{てっき}\ruby{百}{ひゃく}\ruby{万}{まん}\ruby{駆}{か}りつつ\\
        \ruby{正義}{せいぎ}の\ruby{光}{ひかり}\ruby{失}{しっ}する\ruby{時}{とき}\\
        \ruby{燃}{もゆる}ゆる\ruby{義憤}{ぎふん}を\ruby{胸}{むね}に\ruby{秘}{ひ}め\\
        \ruby{起}{た}て\ruby{自治寮}{じちりょう}の\ruby{健}{けん}\ruby{男児}{だんじ}
        
    \end{minipage}
    \begin{minipage}[c]{\blocksize}
        
        \vspace{\linespace}
        \item~\\
        % 6.
        \ruby{自由}{じゆう}の\ruby{旗}{はた}を\ruby{振}{ふ}り\ruby{翳}{}し\\
        \ruby{平和}{へいわ}の\ruby{盾}{たて}を\ruby{掻}{か}き\ruby{列}{つら}ね\\
        \ruby{吾等}{われら}\ruby{起}{た}つべき\ruby{時}{とき}は\ruby{来}{こ}ぬ\\
        \ruby{見}{み}よや\ruby{獅子王}{ししおう}\ruby{一}{いち}\ruby{吼}{}して\\
        \ruby{曠野}{あらの}\ruby{虎狼}{ころう}の\ruby{影}{かげ}もなし\\
        \ruby{祝}{しゅく}へ\ruby{今宵}{こよい}の\ruby{記念祭}{きねんさい}
    
    \end{minipage}
\end{enumerate} % 番号の箇条書き ここまで
%%%%% 歌詞 ここまで %%%%%
% end body

\end{document}

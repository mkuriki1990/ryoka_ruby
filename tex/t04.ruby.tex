\documentclass[10pt,b5j]{tarticle} % B6 縦書き
% \documentclass[10pt,b5j]{tarticle} % B6 縦書き
\AtBeginDvi{\special{papersize=128mm,182mm}} % B6 用用紙サイズ
\usepackage{otf} % Unicode で字を入力するのに必要なパッケージ
\usepackage[size=b6j]{bxpapersize} % B6 用紙サイズを指定
\usepackage[dvipdfmx]{graphicx} % 画像を挿入するためのパッケージ
\usepackage[dvipdfmx]{color} % 色をつけるためのパッケージ
\usepackage{pxrubrica} % ルビを振るためのパッケージ
\usepackage{multicol} % 複数段組を作るためのパッケージ
\setlength{\topmargin}{14mm} % 上下方向のマージン
\addtolength{\topmargin}{-1in} % 
\setlength{\oddsidemargin}{11mm} % 左右方向のマージン
\addtolength{\oddsidemargin}{-1in} % 
\setlength{\textwidth}{154mm} % B6 用
\setlength{\textheight}{108mm} % B6 用
\setlength{\headsep}{0mm} % 
\setlength{\headheight}{0mm} % 
\setlength{\topskip}{0mm} % 
\setlength{\parskip}{0pt} % 
\def\labelenumi{\theenumi、} % 箇条書きのフォーマット
\parindent = 0pt % 段落下げしない

 % B6 用テンプレート読み込み

\begin{document}
% begin header
%%%%% タイトルと作者 ここから %%%%%
\begin{minipage}[c]{0.7\hsize} % タイトルは上から 7 割
    \begin{center}
    % begin title
        {\LARGE
            時轍乾坤に % タイトルを入れる
        }
        {\small 
            (大正四年寮歌) % 年などを入れる
        }
    % end title
    \end{center}
\end{minipage}
\begin{minipage}[c]{0.3\hsize} % 作歌作曲は上から 3 割
    \begin{flushright} % 下寄せにする
        % begin name
        沢田退蔵君 作歌・作曲 % 作歌・作曲者
        % end name
    \end{flushright}
\end{minipage}
%%%%% タイトルと作者 ここまで %%%%%
% (1,2,5,6 繰り返しなし)
% end header

% begin length
\vspace{1.5em} % タイトル, 作者と歌詞の間に隙間を設ける
\newcommand{\linespace}{0.5em} % 行間の設定
\newcommand{\blocksize}{0.5\hsize} % 段組間の設定
\newcommand{\itemmargin}{3em} % 曲番の位置調整の長さ
% end length
% begin body
%%%%% 歌詞 ここから %%%%%
\begin{enumerate} % 番号の箇条書き ここから
    \setlength{\itemindent}{\itemmargin} % 曲番の位置調整
    \begin{minipage}[c]{\blocksize}
    
        \vspace{\linespace}
        \item~\\
        % 1.
        \ruby{時轍乾坤}{}に\ruby{回}{}り\ruby{来}{}て\\
        \ruby{陽春駘蕩}{}のおぼろよひ\\
        \ruby{紫淡}{}く\ruby{霞罩}{}め\\
        \ruby{自治}{}の\ruby{流}{}れは\ruby{永遠}{}に\\
        \ruby{若葉}{}の\ruby{陰}{}を\ruby{浮}{}べつつ\\
        \ruby{吾等}{}が\ruby{幸}{}を\ruby{祝}{}ふらん
        
    \end{minipage}
    \begin{minipage}[c]{\blocksize}
        
        \vspace{\linespace}
        \item~\\
        % 2.
        \ruby{胡馬北風}{}に\ruby{嘶}{}きて\\
        \ruby{越鳥南枝}{}に\ruby{巣}{}を\ruby{造}{}る\\
        \ruby{世}{}の\ruby{濁江}{}に\ruby{逆}{}へる\\
        \ruby{棹歌}{}の\ruby{声}{}の\ruby{勇}{}ましき\\
        \ruby{三星霜}{}の\ruby{春}{}のおきふしに\\
        \ruby{深}{}き\ruby{感慨}{}のなからめや
        
    \end{minipage}
    \begin{minipage}[c]{\blocksize}
        
        \vspace{\linespace}
        \item~\\
        % 3.
        \ruby{紫扉}{}を\ruby{出}{}でて\ruby{霜}{}を\ruby{踏}{}み\\
        \ruby{川流}{}を\ruby{掬}{}ぶ\ruby{薪樵}{}る\\
        \ruby{崇}{}き\ruby{希望}{}の\ruby{若人}{}が\\
        \ruby{歓喜憂苦}{}を\ruby{共}{}にせし\\
        \ruby{友悌凋}{}まぬ\ruby{松柏}{}と\\
        \ruby{幾千代}{}かけて\ruby{変}{}らざれ
        
    \end{minipage}
    \begin{minipage}[c]{\blocksize}
        
        \vspace{\linespace}
        \item~\\
        % 4.
        \ruby{彼}{}の\ruby{邯鄲}{}の\ruby{仮枕}{}\\
        \ruby{栄華}{}の\ruby{夢}{}も\ruby{半}{}にて\\
        \ruby{世}{}の\ruby{秋風}{}に\ruby{驚}{}かん\\
        \ruby{目}{}ざす\ruby{真理}{}の\ruby{高殿}{}は\\
        \ruby{遠}{}く\ruby{遙}{}けし\ruby{突進}{}めいざ\\
        \ruby{心}{}の\ruby{駒}{}に\ruby{鞭打}{}ちて
        
    \end{minipage}
    \begin{minipage}[c]{\blocksize}
        
        \vspace{\linespace}
        \item~\\
        % 5.
        ウラルの\ruby{彼方風凄}{}く\\
        \ruby{陣雲}{}くらき\ruby{八街}{}は\\
        \ruby{鉄騎百万駆}{}りつつ\\
        \ruby{正義}{}の\ruby{光失}{}する\ruby{時}{}\\
        \ruby{燃}{}ゆる\ruby{義憤}{}を\ruby{胸}{}に\ruby{秘}{}め\\
        \ruby{起}{}て\ruby{自治寮}{}の\ruby{健男児}{}
        
    \end{minipage}
    \begin{minipage}[c]{\blocksize}
        
        \vspace{\linespace}
        \item~\\
        % 6.
        \ruby{自由}{}の\ruby{旗}{}を\ruby{振}{}り\ruby{翳}{}し\\
        \ruby{平和}{}の\ruby{盾}{}を\ruby{掻}{}き\ruby{列}{}ね\\
        \ruby{吾等起}{}つべき\ruby{時}{}は\ruby{来}{}ぬ\\
        \ruby{見}{}よや\ruby{獅子王一吼}{}して\\
        \ruby{曠野虎狼}{}の\ruby{影}{}もなし\\
        \ruby{祝}{}へ\ruby{今宵}{}の\ruby{記念祭}{}
    
    \end{minipage}
\end{enumerate} % 番号の箇条書き ここまで
%%%%% 歌詞 ここまで %%%%%
% end body

\end{document}

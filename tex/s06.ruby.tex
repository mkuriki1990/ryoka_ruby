\documentclass[10pt,b5j]{tarticle} % B6 縦書き
% \documentclass[10pt,b5j]{tarticle} % B6 縦書き
\AtBeginDvi{\special{papersize=128mm,182mm}} % B6 用用紙サイズ
\usepackage{otf} % Unicode で字を入力するのに必要なパッケージ
\usepackage[size=b6j]{bxpapersize} % B6 用紙サイズを指定
\usepackage[dvipdfmx]{graphicx} % 画像を挿入するためのパッケージ
\usepackage[dvipdfmx]{color} % 色をつけるためのパッケージ
\usepackage{pxrubrica} % ルビを振るためのパッケージ
\usepackage{multicol} % 複数段組を作るためのパッケージ
\setlength{\topmargin}{14mm} % 上下方向のマージン
\addtolength{\topmargin}{-1in} % 
\setlength{\oddsidemargin}{11mm} % 左右方向のマージン
\addtolength{\oddsidemargin}{-1in} % 
\setlength{\textwidth}{154mm} % B6 用
\setlength{\textheight}{108mm} % B6 用
\setlength{\headsep}{0mm} % 
\setlength{\headheight}{0mm} % 
\setlength{\topskip}{0mm} % 
\setlength{\parskip}{0pt} % 
\def\labelenumi{\theenumi、} % 箇条書きのフォーマット
\parindent = 0pt % 段落下げしない

 % B6 用テンプレート読み込み

\begin{document}
% begin header
%%%%% タイトルと作者 ここから %%%%%
\begin{minipage}[c]{0.7\hsize} % タイトルは上から 7 割
    \begin{center}
    % begin title
        {\LARGE
            平和の光輝ける % タイトルを入れる
        }
        {\small 
            (昭和6年寮歌) % 年などを入れる
        }
    % end title
    \end{center}
\end{minipage}
\begin{minipage}[c]{0.3\hsize} % 作歌作曲は上から 3 割
    \begin{flushright} % 下寄せにする
        % begin name
        広瀬英三君 作歌\\金景洙君 作曲 % 作歌・作曲者
        % end name
    \end{flushright}
\end{minipage}
%%%%% タイトルと作者 ここまで %%%%%
% (1,5 了あり)
% end header

% begin body
\vspace{1.5em} % タイトル, 作者と歌詞の間に隙間を設ける
\newcommand{\linespace}{0.5em} % 行間の設定
\newcommand{\blocksize}{0.5\hsize} % 段組間の設定
%%%%% 歌詞 ここから %%%%%
% begin lilycs
\begin{enumerate} % 番号の箇条書き ここから
    \begin{minipage}[c]{\blocksize}
    
        \vspace{\linespace}
        \item
        % 1.
        \ruby{平和}{}の\ruby{光輝}{}ける\\
        \ruby{春未}{}だ\ruby{浅}{}き\ruby{曙}{}に\\
        \ruby{綾}{}なす\ruby{紫雲}{}を\ruby{分}{}け\ruby{出}{}でて\\
        \ruby{彩色}{}られ\ruby{行}{}く\ruby{青春}{}の\\
        \ruby{久遠}{}の\ruby{迷夢}{}を\ruby{求}{}めつつ\\
        \ruby{声高}{}らかに\ruby{歌}{}はなん
        
        \vspace{\linespace}
        \item
        % 2.
        \ruby{陽光燦然乱}{}れ\ruby{入}{}る\\
        \ruby{夏}{}の\ruby{窓辺}{}に\ruby{書}{}よめば\\
        \ruby{寮庭}{}に\ruby{年経}{}るアカシヤの\\
        \ruby{床}{}しき\ruby{薫香漂}{}ひて\\
        いつか\ruby{心懐}{}の\ruby{極}{}みなく\\
        \ruby{蝦夷}{}の\ruby{昔}{}にいたる\ruby{哉}{}
        
        \vspace{\linespace}
        \item
        % 3.
        \ruby{秋}{}も\ruby{闌}{}け\ruby{行}{}く\ruby{北溟}{}の\ruby{州}{}\\
        \ruby{白楊}{}の\ruby{華乱}{}れとぶ\\
        \ruby{聖}{}き\ruby{都}{}に\ruby{寂寥}{}の\\
        \ruby{静}{}かに\ruby{迫}{}る\ruby{此}{}の\ruby{夕}{}べ\\
        \ruby{思索}{}の\ruby{迪}{}を\ruby{恵}{}ぬれば\\
        \ruby{楡林}{}に\ruby{鐘}{}はなり\ruby{響}{}く
        
        \vspace{\linespace}
        \item
        % 4.
        \ruby{馬橇}{}の\ruby{鈴}{}の\ruby{音}{}も\ruby{絶}{}えし\\
        \ruby{雪}{}の\ruby{大路}{}を\ruby{歩}{}みつつ\\
        \ruby{声}{}をかぎりに\ruby{寮歌}{}うたふ\\
        \ruby{凍}{}れるものみな\ruby{揺}{}かして\\
        \ruby{星斗高}{}く\ruby{冴}{}ゆる\ruby{夜}{}の\\
        \ruby{大空}{}のかなたへ\ruby{消}{}えて\ruby{行}{}く
        
        \vspace{\linespace}
        \item
        % 5.
        \ruby{高}{}き「\ruby{理想}{}」と「\ruby{純情}{}」に\\
        たぎる\ruby{生命}{}を\ruby{託}{}しつつ\\
        \ruby{憧}{}れ\ruby{集}{}ふ\ruby{若人}{}の\\
        \ruby{情熱}{}のかがり\ruby{火打}{}ち\ruby{囲}{}み\\
        \ruby{月下}{}に\ruby{酌}{}むや\ruby{楡}{}の\ruby{宴}{}\\
        いざや\ruby{謳歌}{}へん\ruby{記念祭}{}
    
    \end{minipage}
\end{enumerate} % 番号の箇条書き ここまで
% end lilycs
%%%%% 歌詞 ここまで %%%%%
% end body

\end{document}

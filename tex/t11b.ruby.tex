\documentclass[10pt,b5j]{tarticle} % B6 縦書き
% \documentclass[10pt,b5j]{tarticle} % B6 縦書き
\AtBeginDvi{\special{papersize=128mm,182mm}} % B6 用用紙サイズ
\usepackage{otf} % Unicode で字を入力するのに必要なパッケージ
\usepackage[size=b6j]{bxpapersize} % B6 用紙サイズを指定
\usepackage[dvipdfmx]{graphicx} % 画像を挿入するためのパッケージ
\usepackage[dvipdfmx]{color} % 色をつけるためのパッケージ
\usepackage{pxrubrica} % ルビを振るためのパッケージ
\usepackage{multicol} % 複数段組を作るためのパッケージ
\setlength{\topmargin}{14mm} % 上下方向のマージン
\addtolength{\topmargin}{-1in} % 
\setlength{\oddsidemargin}{11mm} % 左右方向のマージン
\addtolength{\oddsidemargin}{-1in} % 
\setlength{\textwidth}{154mm} % B6 用
\setlength{\textheight}{108mm} % B6 用
\setlength{\headsep}{0mm} % 
\setlength{\headheight}{0mm} % 
\setlength{\topskip}{0mm} % 
\setlength{\parskip}{0pt} % 
\def\labelenumi{\theenumi、} % 箇条書きのフォーマット
\parindent = 0pt % 段落下げしない

 % B6 用テンプレート読み込み

\begin{document}
% begin header
%%%%% タイトルと作者 ここから %%%%%
\begin{minipage}[c]{0.7\hsize} % タイトルは上から 7 割
    \begin{center}
    % begin title
        {\LARGE
            かがやく路 % タイトルを入れる
        }
        {\small 
            (大正11年新寮記念寮歌) % 年などを入れる
        }
    % end title
    \end{center}
\end{minipage}
\begin{minipage}[c]{0.3\hsize} % 作歌作曲は上から 3 割
    \begin{flushright} % 下寄せにする
        % begin name
        服部光平君 作歌\\山本吉之助君 作曲 % 作歌・作曲者
        % end name
    \end{flushright}
\end{minipage}
%%%%% タイトルと作者 ここまで %%%%%
% (1,2,3,4 繰り返しなし)
% end header

% begin body
\vspace{1.5em} % タイトル, 作者と歌詞の間に隙間を設ける
\newcommand{\linespace}{0.5em} % 行間の設定
\newcommand{\blocksize}{0.5\hsize} % 段組間の設定
%%%%% 歌詞 ここから %%%%%
% begin lilycs
\begin{enumerate} % 番号の箇条書き ここから
    \begin{minipage}[c]{\blocksize}
    
        \vspace{\linespace}
        \item
        % 1.
        かがやく\ruby{路}{}のさすらひや\\
        \ruby{魂}{}の\ruby{聖}{}なる\ruby{石狩}{}の\\
        \ruby{色華}{}かなるあけぼのの\\
        \ruby{揺籃}{}に\ruby{歌}{}ふ\ruby{若人}{}は
        
        \vspace{\linespace}
        \item
        % 2.
        \ruby{夏}{}の\ruby{林}{}に\ruby{流}{}れわたる\\
        いのちの\ruby{野火}{}のおき\ruby{伏}{}の\\
        \ruby{愛}{}の\ruby{栄}{}えは\ruby{香盤}{}に\\
        \ruby{感激深}{}く\ruby{胸}{}をゆる
        
        \vspace{\linespace}
        \item
        % 3.
        \ruby{秋}{}の\ruby{狭霧}{}の\ruby{野}{}を\ruby{越}{}えて\\
        \ruby{時}{}の\ruby{進}{}みのみちすぢに\\
        \ruby{鐘}{}の\ruby{音聞}{}けば\ruby{今更}{}に\\
        あはれ\ruby{高鳴}{}る\ruby{吾生命}{}よ
        
        \vspace{\linespace}
        \item
        % 4.
        \ruby{永遠}{}になみうつ\ruby{白銀}{}の\\
        \ruby{神秘}{}を\ruby{語}{}る\ruby{冬}{}の\ruby{夜}{}に\\
        \ruby{空色}{}の\ruby{国星}{}の\ruby{国}{}\\
        \ruby{沈黙}{}に\ruby{曳}{}ける\ruby{追懐}{}よ
    
    \end{minipage}
\end{enumerate} % 番号の箇条書き ここまで
% end lilycs
%%%%% 歌詞 ここまで %%%%%
% end body

\end{document}

\documentclass[10pt,b5j]{tarticle} % B6 縦書き
% \documentclass[10pt,b5j]{tarticle} % B6 縦書き
\AtBeginDvi{\special{papersize=128mm,182mm}} % B6 用用紙サイズ
\usepackage{otf} % Unicode で字を入力するのに必要なパッケージ
\usepackage[size=b6j]{bxpapersize} % B6 用紙サイズを指定
\usepackage[dvipdfmx]{graphicx} % 画像を挿入するためのパッケージ
\usepackage[dvipdfmx]{color} % 色をつけるためのパッケージ
\usepackage{pxrubrica} % ルビを振るためのパッケージ
\usepackage{multicol} % 複数段組を作るためのパッケージ
\setlength{\topmargin}{14mm} % 上下方向のマージン
\addtolength{\topmargin}{-1in} % 
\setlength{\oddsidemargin}{11mm} % 左右方向のマージン
\addtolength{\oddsidemargin}{-1in} % 
\setlength{\textwidth}{154mm} % B6 用
\setlength{\textheight}{108mm} % B6 用
\setlength{\headsep}{0mm} % 
\setlength{\headheight}{0mm} % 
\setlength{\topskip}{0mm} % 
\setlength{\parskip}{0pt} % 
\def\labelenumi{\theenumi、} % 箇条書きのフォーマット
\parindent = 0pt % 段落下げしない

 % B6 用テンプレート読み込み

\begin{document}
% begin header
%%%%% タイトルと作者 ここから %%%%%
\begin{minipage}[c]{0.7\hsize} % タイトルは上から 7 割
    \begin{center}
    % begin title
        {\LARGE
            陽春新しき % タイトルを入れる
        }
        {\small 
            (昭和61年度寮歌) % 年などを入れる
        }
    % end title
    \end{center}
\end{minipage}
\begin{minipage}[c]{0.3\hsize} % 作歌作曲は上から 3 割
    \begin{flushright} % 下寄せにする
        % begin name
        原沢辰明君 作歌\\山森聡君 作曲 % 作歌・作曲者
        % end name
    \end{flushright}
\end{minipage}
%%%%% タイトルと作者 ここまで %%%%%
% (1,2,3,4 了なし繰り返しあり)
% end header

% begin body
\vspace{1.5em} % タイトル, 作者と歌詞の間に隙間を設ける
\newcommand{\linespace}{0.5em} % 行間の設定
\newcommand{\blocksize}{0.5\hsize} % 段組間の設定
%%%%% 歌詞 ここから %%%%%
% begin lilycs
\begin{enumerate} % 番号の箇条書き ここから
    \begin{minipage}[c]{\blocksize}
    
        \vspace{\linespace}
        \item
        % 1.
        \ruby{陽春新}{}しき\ruby{希望満}{}つ\\
        \ruby{恵迪寮}{}に\ruby{若}{}き\ruby{男子等}{}が\\
        \ruby{野心}{}も\ruby{赤}{}き\ruby{夕手稲}{}\\
        \ruby{嗚呼力}{}もて\ruby{進}{}まんか
        
        \vspace{\linespace}
        \item
        % 2.
        \ruby{盛夏短}{}くてストームに\\
        \ruby{太鼓音闇}{}に\ruby{消}{}えるかな\\
        \ruby{朝}{}の\ruby{日露}{}に\ruby{寮歌}{}の\ruby{声}{}\\
        \ruby{嗚呼轟}{}くかこの\ruby{石狩平野}{}
        
        \vspace{\linespace}
        \item
        % 3.
        \ruby{夕暮風}{}の\ruby{涼}{}しさに\\
        \ruby{楡}{}の\ruby{悲}{}しみ\ruby{知}{}れるかな\\
        \ruby{雁}{}より\ruby{暮}{}れる\ruby{原始林}{}\\
        \ruby{嗚呼我}{}が\ruby{憂}{}ひすずろかな
        
        \vspace{\linespace}
        \item
        % 4.
        \ruby{北溟粉雪}{}に\ruby{荒}{}ぶれど\\
        \ruby{詩}{}を\ruby{忘却}{}れぬ\ruby{若人}{}が\\
        \ruby{理想}{}の\ruby{存在求}{}めつつ\\
        \ruby{嗚呼}{}その\ruby{自治寮創造}{}くかな
        
        \vspace{\linespace}
        \item
        % 5.
        \ruby{淡}{}き\ruby{憧憬}{}に\ruby{焦}{}れ\ruby{来}{}る\\
        \ruby{拙}{}き\ruby{言葉操}{}りて\\
        \ruby{胸}{}の\ruby{内}{}を\ruby{打}{}ち\ruby{明}{}けし\\
        \ruby{嗚呼}{}この\ruby{青春}{}も\ruby{早}{}や\ruby{行}{}かん
        
        \vspace{\linespace}
        \item
        % 6.
        \ruby{宴}{}の\ruby{酔狂}{}も\ruby{静寂}{}まりて\\
        \ruby{沈黙}{}の\ruby{彼方微}{}かなる\\
        \ruby{郭公}{}の\ruby{啼声}{}の\ruby{清}{}らかさ\\
        \ruby{嗚呼}{}この\ruby{初夏}{}も\ruby{過}{}ぐるかな
        
        \vspace{\linespace}
        \item
        % 7.
        \ruby{北斗煌}{}く\ruby{晩秋夜}{}\\
        \ruby{望月写}{}す\ruby{支笏湖}{}の\ruby{波}{}\\
        \ruby{明日}{}の\ruby{旅路}{}を\ruby{思}{}いつつ\\
        \ruby{嗚呼涙}{}して\ruby{更}{}くる\ruby{夜}{}
        
        \vspace{\linespace}
        \item
        % 8.
        \ruby{疎々}{}たる\ruby{原始林}{}に\ruby{我一人}{}\\
        \ruby{白雪舞}{}う\ruby{木立烈風強}{}く\\
        \ruby{冷徹}{}たき\ruby{真理索}{}めんと\\
        \ruby{嗚呼声}{}もなく\ruby{迪}{}を\ruby{行}{}く
        
        \vspace{\linespace}
        \item
        % 9.
        \ruby{春}{}も\ruby{巡}{}れる\ruby{四度}{}に\\
        \ruby{若}{}き\ruby{明日}{}の\ruby{祝極}{}と\\
        \ruby{南風頻}{}りに\ruby{頬}{}を\ruby{打}{}つ\\
        \ruby{嗚呼}{}この\ruby{別離永却}{}からず
    
    \end{minipage}
\end{enumerate} % 番号の箇条書き ここまで
% end lilycs
%%%%% 歌詞 ここまで %%%%%
% end body

\end{document}

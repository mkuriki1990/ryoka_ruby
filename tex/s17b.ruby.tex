\documentclass[10pt,b5j]{tarticle} % B6 縦書き
% \documentclass[10pt,b5j]{tarticle} % B6 縦書き
\AtBeginDvi{\special{papersize=128mm,182mm}} % B6 用用紙サイズ
\usepackage{otf} % Unicode で字を入力するのに必要なパッケージ
\usepackage[size=b6j]{bxpapersize} % B6 用紙サイズを指定
\usepackage[dvipdfmx]{graphicx} % 画像を挿入するためのパッケージ
\usepackage[dvipdfmx]{color} % 色をつけるためのパッケージ
\usepackage{pxrubrica} % ルビを振るためのパッケージ
\usepackage{plext} % 漢数字の enumerate を使うためのパッケージ
\usepackage{multicol} % 複数段組を作るためのパッケージ
\setlength{\topmargin}{14mm} % 上下方向のマージン
\addtolength{\topmargin}{-1in} % 
\setlength{\oddsidemargin}{11mm} % 左右方向のマージン
\addtolength{\oddsidemargin}{-1in} % 
\setlength{\textwidth}{154mm} % B6 用
\setlength{\textheight}{108mm} % B6 用
\setlength{\headsep}{0mm} % 
\setlength{\headheight}{0mm} % 
\setlength{\topskip}{0mm} % 
\setlength{\parskip}{0pt} % 
\def\theenumi{\Kanji{enumi}} % 箇条書きのフォーマットを漢数字に変更
\parindent = 0pt % 段落下げしない
\pagestyle{empty} % すべてのページ番号を消去
% \renewcommand{\baselinestretch}{0.9} % 行間の倍率
 % B6 用テンプレート読み込み

\begin{document}
% begin header
%%%%% タイトルと作者 ここから %%%%%
\begin{minipage}[c]{0.7\hsize} % タイトルは上から 7 割
    \begin{center}
    % begin title
        {\LARGE
            あますなく拓きゆく道 % タイトルを入れる
        }
        {\small 
            (昭和十七年大東亜戦争頌歌) % 年などを入れる
        }
    % end title
    \end{center}
\end{minipage}
\begin{minipage}[c]{0.3\hsize} % 作歌作曲は上から 3 割
    \begin{flushright} % 下寄せにする
        % begin name
        切替辰哉君 作歌\\池田政晴君 作曲 % 作歌・作曲者
        % end name
    \end{flushright}
\end{minipage}
%%%%% タイトルと作者 ここまで %%%%%
% (1,7 繰り返しなし)
% end header

% begin length
\vspace{1.5em} % タイトル, 作者と歌詞の間に隙間を設ける
\newcommand{\linespace}{0.5em} % 行間の設定
\newcommand{\blocksize}{0.5\hsize} % 段組間の設定
\newcommand{\itemmargin}{3em} % 曲番の位置調整の長さ
% end length
% begin body
%%%%% 歌詞 ここから %%%%%
\begin{enumerate} % 番号の箇条書き ここから
    \setlength{\itemindent}{\itemmargin} % 曲番の位置調整
    \begin{minipage}[c]{\blocksize}
    
        \vspace{\linespace}
        \item~\\
        % 1.
        あますなく\ruby{拓}{}きゆく\ruby{道}{}\\
        \ruby{天雲}{}の\ruby{向伏}{}す\ruby{極}{}み\\
        \ruby{地}{}の\ruby{涯}{}ゆ、\ruby{征}{}かむ\ruby{御楯}{}と\\
        \ruby{大詔}{}もち、\ruby{我等日}{}の\ruby{族}{}\\
        \ruby{源泉}{}のごと\ruby{湧}{}きたたむ\\
        \ruby{誇}{}らかに\ruby{諸声}{}に\ruby{血潮流}{}さむ
        
    \end{minipage}
    \begin{minipage}[c]{\blocksize}
        
        \vspace{\linespace}
        \item~\\
        % 2.
        \ruby{悠久}{}の\ruby{天詔琴}{}\\
        \ruby{今}{}ぞ\ruby{時}{}、\ruby{轟}{}き\ruby{赴}{}きぬ\\
        \ruby{高光}{}り\ruby{剣}{}を\ruby{植}{}ゑて\\
        \ruby{荒魂}{}の\ruby{魂}{}にぞ\ruby{生}{}きむ\\
        \ruby{遷}{}るべく\ruby{遷}{}る\ruby{亜細亜}{}の\\
        \ruby{峻}{}しかる\ruby{大}{}いなる\\
        \ruby{秋}{}に\ruby{生}{}れし
        
    \end{minipage}
    \begin{minipage}[c]{\blocksize}
        
        \vspace{\linespace}
        \item~\\
        % 3.
        どよめきぬ\ruby{祖霊}{}の\ruby{行}{}\\
        \ruby{六合}{}に\ruby{頸}{}く\ruby{漲}{}ぎり\\
        \ruby{天津日}{}は\ruby{紅燃}{}ゆる\\
        \ruby{南方圏}{}の\ruby{洋路遙}{}けく\\
        \ruby{秀麗}{}しき\ruby{創成}{}の\ruby{神意}{}\\
        \ruby{重}{}く\ruby{負}{}ふに\ruby{務}{}めして\\
        \ruby{生命}{}たぎちむ
        
    \end{minipage}
    \begin{minipage}[c]{\blocksize}
        
        \vspace{\linespace}
        \item~\\
        % 4.
        \ruby{欣求}{}の\ruby{宇宙蝕変満}{}つも\\
        \ruby{東亜}{}の\ruby{空}{}、\ruby{復円光}{}らん\\
        \ruby{斯}{}くせずばやまぬ\ruby{宿命}{}と\\
        \ruby{十億}{}の\ruby{健剛}{}を\ruby{禱}{}みて\\
        \ruby{国挙}{}り\ruby{歩}{}みゆくなり\\
        \ruby{熱涙}{}もて\ruby{仰}{}がなむ\\
        \ruby{黎明}{}の\ruby{幸星}{}
        
    \end{minipage}
    \begin{minipage}[c]{\blocksize}
        
        \vspace{\linespace}
        \item~\\
        % 5.
        \ruby{帰}{}るなき\ruby{発程}{}に\ruby{起}{}つ\\
        \ruby{眸澄}{}める\ruby{我等若人}{}\\
        \ruby{皇国}{}の\ruby{道}{}に\ruby{挺身}{}まん\\
        \ruby{諸共}{}に\ruby{雄叫}{}びすれば\\
        \ruby{叫}{}び\ruby{和}{}す\ruby{新潮}{}の\ruby{声}{}\\
        \ruby{抒情清}{}か、\ruby{白鳥}{}の\\
        \ruby{海図}{}に\ruby{夢}{}む
        
    \end{minipage}
    \begin{minipage}[c]{\blocksize}
        
        \vspace{\linespace}
        \item~\\
        % 6.
        \ruby{厳}{}かの\ruby{時}{}の\ruby{流}{}れに\\
        \ruby{新}{}しき\ruby{力}{}よ\ruby{躍}{}れ\\
        \ruby{鮮}{}けき\ruby{翳}{}りの\ruby{中}{}に\\
        \ruby{新}{}しき\ruby{叫}{}よ\ruby{挙}{}がれ\\
        \ruby{胸臆朗}{}ら、\ruby{身}{}を\ruby{透}{}けて\ruby{佇}{}つ\\
        \ruby{揺}{}ぎなく、\ruby{鍛}{}へして\\
        \ruby{先駆}{}に\ruby{埋}{}めん
        
    \end{minipage}
    \begin{minipage}[c]{\blocksize}
        
        \vspace{\linespace}
        \item~\\
        % 7.
        ここぞ\ruby{茲}{}、いかで\ruby{忘}{}れむ\\
        \ruby{日}{}に\ruby{若}{}き、\ruby{恵迪}{}の\ruby{児}{}よ\\
        たどり\ruby{得}{}し\ruby{道}{}の\ruby{感喜}{}\\
        \ruby{溢}{}れつつ、ほの\ruby{認}{}めけむ\\
        \ruby{仰}{}ぎ\ruby{見}{}る\ruby{銀漢}{}のほとり\\
        \ruby{真実}{}もて、\ruby{弥生}{}ひに\\
        \ruby{継}{}ぎて\ruby{行}{}かなむ
    
    \end{minipage}
\end{enumerate} % 番号の箇条書き ここまで
%%%%% 歌詞 ここまで %%%%%
% end body

\end{document}

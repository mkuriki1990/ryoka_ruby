\documentclass[10pt,b5j]{tarticle} % B6 縦書き
% \documentclass[10pt,b5j]{tarticle} % B6 縦書き
\AtBeginDvi{\special{papersize=128mm,182mm}} % B6 用用紙サイズ
\usepackage{otf} % Unicode で字を入力するのに必要なパッケージ
\usepackage[size=b6j]{bxpapersize} % B6 用紙サイズを指定
\usepackage[dvipdfmx]{graphicx} % 画像を挿入するためのパッケージ
\usepackage[dvipdfmx]{color} % 色をつけるためのパッケージ
\usepackage{pxrubrica} % ルビを振るためのパッケージ
\usepackage{plext} % 漢数字の enumerate を使うためのパッケージ
\usepackage{multicol} % 複数段組を作るためのパッケージ
\setlength{\topmargin}{14mm} % 上下方向のマージン
\addtolength{\topmargin}{-1in} % 
\setlength{\oddsidemargin}{11mm} % 左右方向のマージン
\addtolength{\oddsidemargin}{-1in} % 
\setlength{\textwidth}{154mm} % B6 用
\setlength{\textheight}{108mm} % B6 用
\setlength{\headsep}{0mm} % 
\setlength{\headheight}{0mm} % 
\setlength{\topskip}{0mm} % 
\setlength{\parskip}{0pt} % 
\def\theenumi{\Kanji{enumi}} % 箇条書きのフォーマットを漢数字に変更
\parindent = 0pt % 段落下げしない
\pagestyle{empty} % すべてのページ番号を消去
% \renewcommand{\baselinestretch}{0.9} % 行間の倍率
 % B6 用テンプレート読み込み

\begin{document}
% begin header
%%%%% タイトルと作者 ここから %%%%%
\begin{minipage}[c]{0.7\hsize} % タイトルは上から 7 割
    \begin{center}
    % begin title
        {\LARGE
            無窮の空に % タイトルを入れる
        }
        {\small 
            (大正9年寮歌) % 年などを入れる
        }
    % end title
    \end{center}
\end{minipage}
\begin{minipage}[c]{0.3\hsize} % 作歌作曲は上から 3 割
    \begin{flushright} % 下寄せにする
        % begin name
        戸田早苗君 作歌\\藤田篤君 作曲 % 作歌・作曲者
        % end name
    \end{flushright}
\end{minipage}
%%%%% タイトルと作者 ここまで %%%%%
% (1,2,7 了あり)
% end header

% begin body
\vspace{1.5em} % タイトル, 作者と歌詞の間に隙間を設ける
\newcommand{\linespace}{0.5em} % 行間の設定
\newcommand{\blocksize}{0.5\hsize} % 段組間の設定
%%%%% 歌詞 ここから %%%%%
% begin lilycs
\begin{enumerate} % 番号の箇条書き ここから
    \begin{minipage}[c]{\blocksize}
    
        \vspace{\linespace}
        \item
        % 1.
        \ruby{無窮}{}の\ruby{空}{}に\ruby{黎明}{}の\\
        \ruby{崇高}{}き\ruby{姿天翔}{}り\\
        \ruby{新}{}しき\ruby{日}{}は\ruby{来}{}れりと\\
        \ruby{万象}{}の\ruby{歓声}{}ひびく\ruby{哉}{}
        
        \vspace{\linespace}
        \item
        % 2.
        \ruby{自由}{}の\ruby{陽光}{}かぐはしき\\
        \ruby{美花}{}さく\ruby{学園}{}に\ruby{集}{}ふとき\\
        \ruby{青春}{}の\ruby{日}{}にゆるされし\\
        \ruby{尊}{}きたから\ruby{失}{}はじ
        
        \vspace{\linespace}
        \item
        % 3.
        \ruby{強}{}き\ruby{響}{}きの\ruby{底深}{}く\\
        みなぎる\ruby{大地踏}{}みしめて\\
        \ruby{虚偽}{}の\ruby{世}{}を\ruby{破}{}らんと\\
        \ruby{燃}{}えたちさかる\ruby{我}{}が\ruby{力}{}
        
        \vspace{\linespace}
        \item
        % 4.
        \ruby{生}{}くる\ruby{喜悦讃}{}へつつ\\
        \ruby{深紅}{}の\ruby{幻影狂}{}ひては\\
        \ruby{陽炎}{}ゆらぐ\ruby{野}{}に\ruby{出}{}でて\\
        \ruby{心}{}のかぎり\ruby{歌}{}ひ\ruby{舞}{}ふ
        
        \vspace{\linespace}
        \item
        % 5.
        \ruby{人}{}のいのちの\ruby{際涯}{}なき\\
        \ruby{暗}{}き\ruby{疑惑}{}を\ruby{我胸}{}に\\
        \ruby{夕楡影}{}に\ruby{佇}{}めば\\
        \ruby{北斗}{}は\ruby{高}{}く\ruby{輝}{}けり
        
        \vspace{\linespace}
        \item
        % 6.
        \ruby{真理}{}の\ruby{宮殿}{}の\ruby{灯}{}を\\
        \ruby{憧憬}{}れ\ruby{仰}{}ぐ\ruby{友}{}どちが\\
        \ruby{語}{}らひつきぬ\ruby{感激}{}に\\
        \ruby{吹雪叫}{}ぶ\ruby{夜}{}の\ruby{更}{}けゆくを
        
        \vspace{\linespace}
        \item
        % 7.
        \ruby{三年}{}の\ruby{夢}{}は\ruby{淡}{}くとも\\
        \ruby{長}{}き\ruby{旅路}{}のみちすがら\\
        \ruby{神秘}{}の\ruby{森}{}に\ruby{迷}{}い\ruby{入}{}る\\
        \ruby{尚}{}き\ruby{生命}{}と\ruby{君知}{}るや
    
    \end{minipage}
\end{enumerate} % 番号の箇条書き ここまで
% end lilycs
%%%%% 歌詞 ここまで %%%%%
% end body

\end{document}

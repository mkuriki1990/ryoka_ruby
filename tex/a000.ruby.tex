\documentclass[10pt,b5j]{tarticle} % B6 縦書き
% \documentclass[10pt,b5j]{tarticle} % B6 縦書き
\AtBeginDvi{\special{papersize=128mm,182mm}} % B6 用用紙サイズ
\usepackage{otf} % Unicode で字を入力するのに必要なパッケージ
\usepackage[size=b6j]{bxpapersize} % B6 用紙サイズを指定
\usepackage[dvipdfmx]{graphicx} % 画像を挿入するためのパッケージ
\usepackage[dvipdfmx]{color} % 色をつけるためのパッケージ
\usepackage{pxrubrica} % ルビを振るためのパッケージ
\usepackage{multicol} % 複数段組を作るためのパッケージ
\setlength{\topmargin}{14mm} % 上下方向のマージン
\addtolength{\topmargin}{-1in} % 
\setlength{\oddsidemargin}{11mm} % 左右方向のマージン
\addtolength{\oddsidemargin}{-1in} % 
\setlength{\textwidth}{154mm} % B6 用
\setlength{\textheight}{108mm} % B6 用
\setlength{\headsep}{0mm} % 
\setlength{\headheight}{0mm} % 
\setlength{\topskip}{0mm} % 
\setlength{\parskip}{0pt} % 
\def\labelenumi{\theenumi、} % 箇条書きのフォーマット
\parindent = 0pt % 段落下げしない

 % B6 用テンプレート読み込み

\begin{document}
% begin header
%%%%% タイトルと作者 ここから %%%%%
\begin{minipage}[c]{0.7\hsize} % タイトルは上から 7 割
    \begin{center}
    % begin title
        {\LARGE
            永遠の幸 % タイトルを入れる
        }
        {\small 
            (札幌農学校校歌) % 年などを入れる
        }
    % end title
    \end{center}
\end{minipage}
\begin{minipage}[c]{0.3\hsize} % 作歌作曲は上から 3 割
    \begin{flushright} % 下寄せにする
        % begin name
        大和田建樹氏 校閲 有島武朗君 作歌\\納所弁次郎君 作曲 % 作歌・作曲者
        % end name
    \end{flushright}
\end{minipage}
%%%%% タイトルと作者 ここまで %%%%%
% (1 了あり)
% end header

% begin body
\vspace{1.5em} % タイトル, 作者と歌詞の間に隙間を設ける
\newcommand{\linespace}{0.5em} % 行間の設定
\newcommand{\blocksize}{0.5\hsize} % 段組間の設定
%%%%% 歌詞 ここから %%%%%
% begin lilycs
\begin{enumerate} % 番号の箇条書き ここから
    \begin{minipage}[c]{\blocksize}
    
        \vspace{\linespace}
        \item
        % 1.
        \ruby{永遠}{}の\ruby{幸}{}\\
        \ruby{朽}{}ちざる\ruby{誉}{}\\
        つねに\ruby{我等}{}がうへにあれ\\
        よるひる\ruby{育}{}て\\
        あけくれ\ruby{教}{}へ\\
        \ruby{人}{}となしし\ruby{我庭}{}に
        
        \vspace{\linespace}
        \item
        イザイザイザ\\
        うちつれて\\
        \ruby{進}{}むは\ruby{今}{}ぞ\\
        \ruby{豊平}{}の\ruby{川}{}\\
        \ruby{尽}{}せぬながれ\\
        \ruby{友}{}たれ\ruby{永}{}く\ruby{友}{}たれ
        
        \vspace{\linespace}
        \item
        % 2.
        \ruby{北斗}{}をつかん\\
        たかき\ruby{希望}{}は\\
        \ruby{時代}{}を\ruby{照}{}す\ruby{光}{}なり\\
        \ruby{深雪}{}を\ruby{凌}{}ぐ\\
        \ruby{潔}{}き\ruby{節操}{}は\\
        \ruby{国}{}を\ruby{守}{}る\ruby{力}{}なり
        
        \vspace{\linespace}
        \item
        イザイザイザ\\
        うちつれて\\
        \ruby{進}{}むは\ruby{今}{}ぞ\\
        \ruby{豊平}{}の\ruby{川}{}\\
        \ruby{尽}{}せぬながれ\\
        \ruby{友}{}たれ\ruby{永}{}く\ruby{友}{}たれ
        
        \vspace{\linespace}
        \item
        % 3.
        \ruby{山}{}は\ruby{裂}{}くとも\\
        \ruby{海}{}はあすとも\\
        \ruby{真理正義}{}おつべしや\\
        \ruby{不朽}{}を\ruby{求}{}め\\
        \ruby{意気相}{}ゆるす\\
        \ruby{我等丈夫此}{}にあり
        
        \vspace{\linespace}
        \item
        イザイザイザ\\
        うちつれて\\
        \ruby{進}{}むは\ruby{今}{}ぞ\\
        \ruby{豊平}{}の\ruby{川}{}\\
        \ruby{尽}{}せぬながれ\\
        \ruby{友}{}たれ\ruby{永}{}く\ruby{友}{}たれ
        
        (※
        有島武郎在学中の明治三十三年の作。
        大和田建樹(一八五六 - 一九一〇)は
        作詞の面で、
        納所弁次郎(一八六五 - 一九三六)は
        作曲の面で、共に近代日本唱歌史に
        大きな足跡を残した。)

    
    \end{minipage}
\end{enumerate} % 番号の箇条書き ここまで
% end lilycs
%%%%% 歌詞 ここまで %%%%%
% end body

\end{document}

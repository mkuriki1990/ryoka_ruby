\documentclass[10pt,b5j]{tarticle} % B6 縦書き
% \documentclass[10pt,b5j]{tarticle} % B6 縦書き
\AtBeginDvi{\special{papersize=128mm,182mm}} % B6 用用紙サイズ
\usepackage{otf} % Unicode で字を入力するのに必要なパッケージ
\usepackage[size=b6j]{bxpapersize} % B6 用紙サイズを指定
\usepackage[dvipdfmx]{graphicx} % 画像を挿入するためのパッケージ
\usepackage[dvipdfmx]{color} % 色をつけるためのパッケージ
\usepackage{pxrubrica} % ルビを振るためのパッケージ
\usepackage{multicol} % 複数段組を作るためのパッケージ
\setlength{\topmargin}{14mm} % 上下方向のマージン
\addtolength{\topmargin}{-1in} % 
\setlength{\oddsidemargin}{11mm} % 左右方向のマージン
\addtolength{\oddsidemargin}{-1in} % 
\setlength{\textwidth}{154mm} % B6 用
\setlength{\textheight}{108mm} % B6 用
\setlength{\headsep}{0mm} % 
\setlength{\headheight}{0mm} % 
\setlength{\topskip}{0mm} % 
\setlength{\parskip}{0pt} % 
\def\labelenumi{\theenumi、} % 箇条書きのフォーマット
\parindent = 0pt % 段落下げしない

 % B6 用テンプレート読み込み

\begin{document}
% begin header
%%%%% タイトルと作者 ここから %%%%%
\begin{minipage}[c]{0.7\hsize} % タイトルは上から 7 割
    \begin{center}
    % begin title
        {\LARGE
            タンネの氷柱 % タイトルを入れる
        }
        {\small 
            (昭和八年寮歌) % 年などを入れる
        }
    % end title
    \end{center}
\end{minipage}
\begin{minipage}[c]{0.3\hsize} % 作歌作曲は上から 3 割
    \begin{flushright} % 下寄せにする
        % begin name
        卜部清君 作歌\\白石祐義君 作曲 % 作歌・作曲者
        % end name
    \end{flushright}
\end{minipage}
%%%%% タイトルと作者 ここまで %%%%%
% (1,2,3,4,5 了あり)
% end header

% begin length
\vspace{1.5em} % タイトル, 作者と歌詞の間に隙間を設ける
\newcommand{\linespace}{0.5em} % 行間の設定
\newcommand{\blocksize}{0.5\hsize} % 段組間の設定
\newcommand{\itemmargin}{6em} % 曲番の位置調整の長さ
% end length
% begin body
%%%%% 歌詞 ここから %%%%%
\begin{enumerate} % 番号の箇条書き ここから
    \setlength{\itemindent}{\itemmargin} % 曲番の位置調整
    \begin{minipage}[c]{\blocksize}
    
        \vspace{\linespace}
        \item~\\
        % 1.
        タンネの\ruby{氷柱消}{}ゆる\ruby{頃}{}\\
        \ruby{胡蝶}{}は\ruby{眠}{}る\ruby{花}{}の\ruby{宿}{}\\
        \ruby{牧場}{}に\ruby{結}{}ぶ\ruby{夢遙}{}か\\
        \ruby{青}{}き\ruby{希望}{}の\ruby{雪峯}{}こえて\\
        \ruby{四海}{}に\ruby{羽振}{}る\ruby{若鵬}{}の\\
        \ruby{石狩}{}を\ruby{立}{}つ\ruby{意気}{}をみん
        
        \vspace{\linespace}
        \item~\\
        % 2.
        \ruby{朝里}{}の\ruby{丘}{}に\ruby{烏頭咲}{}けば\\
        \ruby{蝦夷}{}が\ruby{芙蓉}{}の\ruby{雪}{}とけて\\
        \ruby{千尋}{}の\ruby{懸崖}{}ゆくだけ\ruby{入}{}る\\
        \ruby{忍路}{}の\ruby{沖}{}の\ruby{真白帆}{}に\\
        \ruby{万里}{}の\ruby{波濤翔}{}らんと\\
        \ruby{白鷗}{}はしばし\ruby{憩}{}ふなり
        
        \vspace{\linespace}
        \item~\\
        % 3.
        \ruby{真紅}{}の\ruby{夕陽山}{}の\ruby{端}{}に\\
        もゆる\ruby{紅葉}{}をかざしたる\\
        \ruby{友}{}がゆくての\ruby{野}{}を\ruby{遠}{}く\\
        \ruby{幌馬車}{}の\ruby{影消}{}え\ruby{去}{}りぬ\\
        \ruby{蓬髪風}{}に\ruby{靡}{}けつつ\\
        \ruby{懐情}{}は\ruby{尽}{}きず\ruby{果}{}てもなく
        
        \vspace{\linespace}
        \item~\\
        % 4.
        \ruby{十勝}{}の\ruby{峰}{}に\ruby{捲}{}き\ruby{起}{}こる\\
        \ruby{吹雪怒}{}りて\ruby{咆}{}ゆる\ruby{夜}{}も\\
        \ruby{旭光東}{}に\ruby{色}{}めけば\\
        \ruby{熊追}{}う\ruby{愛奴}{}の\ruby{雄叫}{}びに\\
        \ruby{大雪原}{}の\ruby{霊光}{}や\\
        \ruby{無弦琴}{}の\ruby{音}{}ぞ\ruby{高}{}し
        
        \vspace{\linespace}
        \item~\\
        % 5.
        \ruby{懸}{}る\ruby{垂氷}{}に\ruby{月}{}くだけ\\
        \ruby{千々}{}の\ruby{瞑想}{}は\ruby{来}{}し\ruby{方}{}の\\
        \ruby{六十}{}の\ruby{秋}{}はしるくして\\
        \ruby{緑}{}に\ruby{浮}{}ぶ\ruby{白亜城}{}\\
        \ruby{苔}{}むす\ruby{楡鐘}{}の\ruby{哀調}{}きけ\\
        \ruby{若}{}き\ruby{力}{}を\ruby{求}{}むなり
    
    \end{minipage}
\end{enumerate} % 番号の箇条書き ここまで
%%%%% 歌詞 ここまで %%%%%
% end body

\end{document}

\documentclass[10pt,b5j]{tarticle} % B6 縦書き
% \documentclass[10pt,b5j]{tarticle} % B6 縦書き
\AtBeginDvi{\special{papersize=128mm,182mm}} % B6 用用紙サイズ
\usepackage{otf} % Unicode で字を入力するのに必要なパッケージ
\usepackage[size=b6j]{bxpapersize} % B6 用紙サイズを指定
\usepackage[dvipdfmx]{graphicx} % 画像を挿入するためのパッケージ
\usepackage[dvipdfmx]{color} % 色をつけるためのパッケージ
\usepackage{pxrubrica} % ルビを振るためのパッケージ
\usepackage{multicol} % 複数段組を作るためのパッケージ
\setlength{\topmargin}{14mm} % 上下方向のマージン
\addtolength{\topmargin}{-1in} % 
\setlength{\oddsidemargin}{11mm} % 左右方向のマージン
\addtolength{\oddsidemargin}{-1in} % 
\setlength{\textwidth}{154mm} % B6 用
\setlength{\textheight}{108mm} % B6 用
\setlength{\headsep}{0mm} % 
\setlength{\headheight}{0mm} % 
\setlength{\topskip}{0mm} % 
\setlength{\parskip}{0pt} % 
\def\labelenumi{\theenumi、} % 箇条書きのフォーマット
\parindent = 0pt % 段落下げしない

 % B6 用テンプレート読み込み

\begin{document}
% begin header
%%%%% タイトルと作者 ここから %%%%%
\begin{minipage}[c]{0.7\hsize} % タイトルは上から 7 割
    \begin{center}
    % begin title
        {\LARGE
            若き力 % タイトルを入れる
        }
        {\small 
            (平成八年度寮歌) % 年などを入れる
        }
    % end title
    \end{center}
\end{minipage}
\begin{minipage}[c]{0.3\hsize} % 作歌作曲は上から 3 割
    \begin{flushright} % 下寄せにする
        % begin name
        長谷川健君 作歌\\石井英一君 作曲 % 作歌・作曲者
        % end name
    \end{flushright}
\end{minipage}
%%%%% タイトルと作者 ここまで %%%%%
% (1,2,3 了あり)
% end header

% begin length
\vspace{1.5em} % タイトル, 作者と歌詞の間に隙間を設ける
\newcommand{\linespace}{0.5em} % 行間の設定
\newcommand{\blocksize}{0.5\hsize} % 段組間の設定
\newcommand{\itemmargin}{3em} % 曲番の位置調整の長さ
% end length
% begin body
%%%%% 歌詞 ここから %%%%%
\begin{enumerate} % 番号の箇条書き ここから
    \setlength{\itemindent}{\itemmargin} % 曲番の位置調整
    \begin{minipage}[c]{\blocksize}
    
        \vspace{\linespace}
        \item~\\
        % 1.
        \ruby{高鳴}{たかな}る\ruby{胸}{むね}の\ruby{血潮}{ちしお}\ruby{巻}{ま}く\\
        \ruby{熱}{あつ}い情\ruby{熱}{あつ}に\ruby{身}{み}をまかす\\
        ただその\ruby{意気}{いき}を\ruby{信}{しん}じつつ\\
        \ruby{堂々}{どうどう}\ruby{迪}{すすむ}を\ruby{拓}{ひら}きゆく\\
        \ruby{我}{が}は\ruby{泰山北斗}{たいざんほくと}の\ruby{身}{み}
        
    \end{minipage}
    \begin{minipage}[c]{\blocksize}
        
        \vspace{\linespace}
        \item~\\
        % 2.
        \ruby{我}{が}\ruby{\ruby{何}{なに}者}{なにもの}ぞ\ruby{何}{なに}かある\\
        \ruby{我}{わ}が\ruby{若}{わか}き\ruby{力}{きりょく}\ruby{奮起}{ふんき}せば\\
        \ruby{鎧袖一触}{がいしゅういっしょく}\ruby{地}{ち}に\ruby{砕}{くだ}き\\
        \ruby{天}{てん}にも\ruby{響}{ひび}け「\ruby{嗚呼}{ああ}バンカラ!」\\
        \ruby{憂}{うれ}い\ruby{忘}{わす}れよ\ruby{杯}{はい}を\ruby{酌}{く}め
        
    \end{minipage}
    \begin{minipage}[c]{\blocksize}
        
        \vspace{\linespace}
        \item~\\
        % 3.
        \ruby{希}{まれ}みは\ruby{高}{こう}しけつ\ruby{青}{あお}し\\
        \ruby{芙蓉}{ふよう}\ruby{万里}{ばんり}の\ruby{風}{かぜ}を\ruby{待}{ま}ち\\
        しばしおさめしこの\ruby{翼}{つばさ}\\
        \ruby{雄図}{ゆうと}を\ruby{胸}{むね}に\ruby{刻}{きざ}みたる\\
        \ruby{鴻鵠}{こうこく}の\ruby{志}{こころざし}をだれ\ruby{知}{し}るや
    
    \end{minipage}
\end{enumerate} % 番号の箇条書き ここまで
%%%%% 歌詞 ここまで %%%%%
% end body

\end{document}

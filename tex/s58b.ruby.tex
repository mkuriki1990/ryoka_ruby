\documentclass[10pt,b5j]{tarticle} % B6 縦書き
% \documentclass[10pt,b5j]{tarticle} % B6 縦書き
\AtBeginDvi{\special{papersize=128mm,182mm}} % B6 用用紙サイズ
\usepackage{otf} % Unicode で字を入力するのに必要なパッケージ
\usepackage[size=b6j]{bxpapersize} % B6 用紙サイズを指定
\usepackage[dvipdfmx]{graphicx} % 画像を挿入するためのパッケージ
\usepackage[dvipdfmx]{color} % 色をつけるためのパッケージ
\usepackage{pxrubrica} % ルビを振るためのパッケージ
\usepackage{multicol} % 複数段組を作るためのパッケージ
\setlength{\topmargin}{14mm} % 上下方向のマージン
\addtolength{\topmargin}{-1in} % 
\setlength{\oddsidemargin}{11mm} % 左右方向のマージン
\addtolength{\oddsidemargin}{-1in} % 
\setlength{\textwidth}{154mm} % B6 用
\setlength{\textheight}{108mm} % B6 用
\setlength{\headsep}{0mm} % 
\setlength{\headheight}{0mm} % 
\setlength{\topskip}{0mm} % 
\setlength{\parskip}{0pt} % 
\def\labelenumi{\theenumi、} % 箇条書きのフォーマット
\parindent = 0pt % 段落下げしない

 % B6 用テンプレート読み込み

\begin{document}
% begin header
%%%%% タイトルと作者 ここから %%%%%
\begin{minipage}[c]{0.7\hsize} % タイトルは上から 7 割
    \begin{center}
    % begin title
        {\LARGE
            北に恵めし % タイトルを入れる
        }
        {\small 
            (昭和五十八年新寮記念寮歌) % 年などを入れる
        }
    % end title
    \end{center}
\end{minipage}
\begin{minipage}[c]{0.3\hsize} % 作歌作曲は上から 3 割
    \begin{flushright} % 下寄せにする
        % begin name
        大崎益孝君 作歌\\竹中秀文君 作曲 % 作歌・作曲者
        % end name
    \end{flushright}
\end{minipage}
%%%%% タイトルと作者 ここまで %%%%%
% (1,2,3 了あり)
% end header

% begin body
\vspace{1.5em} % タイトル, 作者と歌詞の間に隙間を設ける
\newcommand{\linespace}{0.5em} % 行間の設定
\newcommand{\blocksize}{0.5\hsize} % 段組間の設定
%%%%% 歌詞 ここから %%%%%
% begin lilycs
\begin{enumerate} % 番号の箇条書き ここから
    \begin{minipage}[c]{\blocksize}
    
        \vspace{\linespace}
        \item
        % 1.
        \ruby{北}{}に\ruby{恵}{}めし\ruby{若}{}き\ruby{日}{}の\ruby{夢}{}\\
        いつかは\ruby{壊}{}れゆくものか\\
        すがしき\ruby{朝}{}の\ruby{光}{}と\ruby{風}{}は\\
        \ruby{原始}{}の\ruby{森}{}に\ruby{消}{}え\ruby{去}{}りぬ\\
        \ruby{今}{}こそ\ruby{我}{}も\ruby{旅立}{}ちの\ruby{時}{}\\
        \ruby{心}{}の\ruby{宿}{}よいざさらば
        
        \vspace{\linespace}
        \item
        % 2.
        \ruby{北}{}の\ruby{原野}{}を\ruby{流離}{}い\ruby{行}{}けば\\
        \ruby{淡}{}き\ruby{花影}{}さゆらぎぬ\\
        \ruby{今}{}も\ruby{変}{}わらぬその\ruby{涼風}{}に\\
        \ruby{昔}{}の\ruby{光偲}{}ばずや\\
        \ruby{流}{}れる\ruby{雲}{}に\ruby{孤}{}り\ruby{謳}{}えば\\
        \ruby{果}{}てなく\ruby{夢}{}は\ruby{何処}{}までも
        
        \vspace{\linespace}
        \item
        % 3.
        \ruby{北}{}を\ruby{望}{}みし\ruby{岬}{}に\ruby{立}{}てば\\
        うち\ruby{寄}{}す\ruby{波}{}は\ruby{静}{}かなり\\
        されど\ruby{遙}{}けき\ruby{今樺太}{}の\\
        \ruby{色}{}めく\ruby{空}{}を\ruby{憂}{}い\ruby{眺}{}ん\\
        \ruby{功利}{}し\ruby{多}{}きこの\ruby{人}{}の\ruby{世}{}に\\
        \ruby{誠}{}の\ruby{迪}{}を\ruby{貫}{}かん
    
    \end{minipage}
\end{enumerate} % 番号の箇条書き ここまで
% end lilycs
%%%%% 歌詞 ここまで %%%%%
% end body

\end{document}

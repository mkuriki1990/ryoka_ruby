\documentclass[10pt,b5j]{tarticle} % B6 縦書き
% \documentclass[10pt,b5j]{tarticle} % B6 縦書き
\AtBeginDvi{\special{papersize=128mm,182mm}} % B6 用用紙サイズ
\usepackage{otf} % Unicode で字を入力するのに必要なパッケージ
\usepackage[size=b6j]{bxpapersize} % B6 用紙サイズを指定
\usepackage[dvipdfmx]{graphicx} % 画像を挿入するためのパッケージ
\usepackage[dvipdfmx]{color} % 色をつけるためのパッケージ
\usepackage{pxrubrica} % ルビを振るためのパッケージ
\usepackage{multicol} % 複数段組を作るためのパッケージ
\setlength{\topmargin}{14mm} % 上下方向のマージン
\addtolength{\topmargin}{-1in} % 
\setlength{\oddsidemargin}{11mm} % 左右方向のマージン
\addtolength{\oddsidemargin}{-1in} % 
\setlength{\textwidth}{154mm} % B6 用
\setlength{\textheight}{108mm} % B6 用
\setlength{\headsep}{0mm} % 
\setlength{\headheight}{0mm} % 
\setlength{\topskip}{0mm} % 
\setlength{\parskip}{0pt} % 
\def\labelenumi{\theenumi、} % 箇条書きのフォーマット
\parindent = 0pt % 段落下げしない

 % B6 用テンプレート読み込み

\begin{document}
% begin header
%%%%% タイトルと作者 ここから %%%%%
\begin{minipage}[c]{0.7\hsize} % タイトルは上から 7 割
    \begin{center}
    % begin title
        {\LARGE
            北に恵めし % タイトルを入れる
        }
        {\small 
            (昭和五十八年新寮記念寮歌) % 年などを入れる
        }
    % end title
    \end{center}
\end{minipage}
\begin{minipage}[c]{0.3\hsize} % 作歌作曲は上から 3 割
    \begin{flushright} % 下寄せにする
        % begin name
        大崎益孝君 作歌\\竹中秀文君 作曲 % 作歌・作曲者
        % end name
    \end{flushright}
\end{minipage}
%%%%% タイトルと作者 ここまで %%%%%
% (1,2,3 了あり)
% end header

% begin length
\vspace{1.5em} % タイトル, 作者と歌詞の間に隙間を設ける
\newcommand{\linespace}{0.5em} % 行間の設定
\newcommand{\blocksize}{0.5\hsize} % 段組間の設定
\newcommand{\itemmargin}{3em} % 曲番の位置調整の長さ
% end length
% begin body
%%%%% 歌詞 ここから %%%%%
\begin{enumerate} % 番号の箇条書き ここから
    \setlength{\itemindent}{\itemmargin} % 曲番の位置調整
    \begin{minipage}[c]{\blocksize}
    
        \vspace{\linespace}
        \item~\\
        % 1.
        \ruby{北}{きた}に\ruby{恵}{めぐ}めし\ruby{若}{わか}き\ruby{日}{ひ}の\ruby{夢}{ゆめ}\\
        いつかは\ruby{壊}{こわ}れゆくものか\\
        すがしき\ruby{朝}{あさ}の\ruby{光}{ひかり}と\ruby{風}{かぜ}は\\
        \ruby{原始}{げんし}の\ruby{森}{もり}に\ruby{消}{き}え\ruby{去}{さ}りぬ\\
        \ruby{今}{いま}こそ\ruby{我}{わが}も\ruby{旅}{}\ruby{立}{たびだ}ちの\ruby{時}{とき}\\
        \ruby{心}{こころ}の\ruby{宿}{やど}よいざさらば
        
    \end{minipage}
    \begin{minipage}[c]{\blocksize}
        
        \vspace{\linespace}
        \item~\\
        % 2.
        \ruby{北}{きた}の\ruby{原野}{げんや}を\ruby{流離}{りゅうり}い\ruby{行}{い}けば\\
        \ruby{淡}{あわ}き\ruby{花影}{はなかげ}さゆらぎぬ\\
        \ruby{今}{いま}も\ruby{変}{か}わらぬその\ruby{涼風}{すずかぜ}に\\
        \ruby{昔}{むかし}の\ruby{光}{ひかり}\ruby{偲}{しの}ばずや\\
        \ruby{流}{なが}れる\ruby{雲}{くも}に\ruby{孤}{こ}り\ruby{謳}{うた}えば\\
        \ruby{果}{は}てなく\ruby{夢}{ゆめ}は\ruby{何処}{どこ}までも
        
    \end{minipage}
    \begin{minipage}[c]{\blocksize}
        
        \vspace{\linespace}
        \item~\\
        % 3.
        \ruby{北}{きた}を\ruby{望}{のぞ}みし\ruby{岬}{みさき}に\ruby{立}{た}てば\\
        うち\ruby{寄}{やどりき}す\ruby{波}{なみ}は\ruby{静}{しず}かなり\\
        されど\ruby{遙}{}けき\ruby{今}{こん}\ruby{樺太}{からふと}の\\
        \ruby{色}{いろ}めく\ruby{空}{そら}を\ruby{憂}{うれ}い\ruby{眺}{}ん\\
        \ruby{功利}{こうり}し\ruby{多}{おお}きこの\ruby{人}{ひと}の\ruby{世}{よ}に\\
        \ruby{誠}{まこと}の\ruby{迪}{すすむ}を\ruby{貫}{つらぬ}かん
    
    \end{minipage}
\end{enumerate} % 番号の箇条書き ここまで
%%%%% 歌詞 ここまで %%%%%
% end body

\end{document}

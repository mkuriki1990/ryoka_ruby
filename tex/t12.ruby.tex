\documentclass[10pt,b5j]{tarticle} % B6 縦書き
% \documentclass[10pt,b5j]{tarticle} % B6 縦書き
\AtBeginDvi{\special{papersize=128mm,182mm}} % B6 用用紙サイズ
\usepackage{otf} % Unicode で字を入力するのに必要なパッケージ
\usepackage[size=b6j]{bxpapersize} % B6 用紙サイズを指定
\usepackage[dvipdfmx]{graphicx} % 画像を挿入するためのパッケージ
\usepackage[dvipdfmx]{color} % 色をつけるためのパッケージ
\usepackage{pxrubrica} % ルビを振るためのパッケージ
\usepackage{plext} % 漢数字の enumerate を使うためのパッケージ
\usepackage{multicol} % 複数段組を作るためのパッケージ
\setlength{\topmargin}{14mm} % 上下方向のマージン
\addtolength{\topmargin}{-1in} % 
\setlength{\oddsidemargin}{11mm} % 左右方向のマージン
\addtolength{\oddsidemargin}{-1in} % 
\setlength{\textwidth}{154mm} % B6 用
\setlength{\textheight}{108mm} % B6 用
\setlength{\headsep}{0mm} % 
\setlength{\headheight}{0mm} % 
\setlength{\topskip}{0mm} % 
\setlength{\parskip}{0pt} % 
\def\theenumi{\Kanji{enumi}} % 箇条書きのフォーマットを漢数字に変更
\parindent = 0pt % 段落下げしない
\pagestyle{empty} % すべてのページ番号を消去
% \renewcommand{\baselinestretch}{0.9} % 行間の倍率
 % B6 用テンプレート読み込み

\begin{document}
% begin header
%%%%% タイトルと作者 ここから %%%%%
\begin{minipage}[c]{0.7\hsize} % タイトルは上から 7 割
    \begin{center}
    % begin title
        {\LARGE
            春雨に濡る % タイトルを入れる
        }
        {\small 
            (大正十二年寮歌) % 年などを入れる
        }
    % end title
    \end{center}
\end{minipage}
\begin{minipage}[c]{0.3\hsize} % 作歌作曲は上から 3 割
    \begin{flushright} % 下寄せにする
        % begin name
        高橋北雄君 作歌\\西田貫道君 作曲 % 作歌・作曲者
        % end name
    \end{flushright}
\end{minipage}
%%%%% タイトルと作者 ここまで %%%%%
% (1,2,3,4 了あり)
% end header

% begin length
\vspace{1.5em} % タイトル, 作者と歌詞の間に隙間を設ける
\newcommand{\linespace}{0.5em} % 行間の設定
\newcommand{\blocksize}{0.5\hsize} % 段組間の設定
\newcommand{\itemmargin}{3em} % 曲番の位置調整の長さ
% end length
% begin body
%%%%% 歌詞 ここから %%%%%
\begin{enumerate} % 番号の箇条書き ここから
    \setlength{\itemindent}{\itemmargin} % 曲番の位置調整
    \begin{minipage}[c]{\blocksize}
    
        \vspace{\linespace}
        \item~\\
        % 1.
        \ruby{春雨}{はるさめ}に\ruby{濡}{}るアカシヤ\ruby{花}{はな}\\
        \ruby{街路}{がいろ}の\ruby{灯}{あかり}はなやかに\\
        \ruby{地}{ち}は\ruby{銀鼠}{ぎんねず}にたそがるる\\
        \ruby{寂}{さび}かに\ruby{歩}{あゆ}む\ruby{若人}{わこうど}が\\
        \ruby{心}{こころ}にめざむ\ruby{爽}{}かの\\
        \ruby{灑}{}み\ruby{充}{あ}てる\ruby{力}{ちから}かな
        
    \end{minipage}
    \begin{minipage}[c]{\blocksize}
        
        \vspace{\linespace}
        \item~\\
        % 2.
        \ruby{夏}{なつ}の\ruby{入}{にゅう}\ruby{陽}{ひ}に\ruby{砂丘}{さきゅう}の\\
        \ruby{猟虎}{らっこ}の\ruby{骨}{ほね}に\ruby{鷗}{}\ruby{飛}{と}ぶ\\
        \ruby{融}{と}けざる\ruby{銀}{ぎん}の\ruby{山脈}{さんみゃく}は\\
        \ruby{碧}{へき}\ruby{薄}{うす}れゆく\ruby{空}{そら}にうく\\
        \ruby{名残}{なごり}の\ruby{光}{ひかり}\ruby{身}{み}にあびて\\
        \ruby{異郷}{いきょう}の\ruby{方}{ほう}を\ruby{思}{おも}ふかな
        
    \end{minipage}
    \begin{minipage}[c]{\blocksize}
        
        \vspace{\linespace}
        \item~\\
        % 3.
        \ruby{仄}{}\ruby{青}{}\ruby{白}{あおじろ}き\ruby{白}{あおじろ}樺や\\
        \ruby{落葉}{らくよう}ふむ\ruby{音}{おと}\ruby{寂}{さび}しくも\\
        \ruby{谷}{たに}また\ruby{谷}{たに}を\ruby{辿}{たど}り\ruby{行}{い}き\\
        \ruby{今宵}{こよい}は\ruby{淡}{あわ}き\ruby{夢}{}\ruby{見}{ゆめみ}んと\\
        \ruby{焚火}{たきび}を\ruby{囲}{かこ}み\ruby{歌}{か}ふ寮\ruby{歌}{か}\\
        \ruby{紫紺}{しこん}の\ruby{闇}{やみ}に\ruby{解}{と}けて\ruby{行}{い}く
        
    \end{minipage}
    \begin{minipage}[c]{\blocksize}
        
        \vspace{\linespace}
        \item~\\
        % 4.
        \ruby{青}{あお}き\ruby{空}{そら}\ruby{透}{す}き\ruby{銀}{ぎん}の\ruby{月}{つき}\\
        \ruby{石狩}{いしかり}の\ruby{河}{かわ}\ruby{波}{なみ}\ruby{光}{ひか}る\\
        \ruby{雪}{ゆき}の\ruby{野}{の}\ruby{限}{げん}は\ruby{靄}{}こめて\\
        \ruby{灯}{あかり}\ruby{漂}{}ふアイヌ\ruby{小屋}{こや}\\
        \ruby{琥珀}{こはく}の\ruby{酒}{さけ}を\ruby{汲}{く}み\ruby{交}{かわ}し\\
        \ruby{王者}{おうじゃ}の\ruby{誇}{ほこ}\ruby{偲}{しの}ぶかな
    
    \end{minipage}
\end{enumerate} % 番号の箇条書き ここまで
%%%%% 歌詞 ここまで %%%%%
% end body

\end{document}

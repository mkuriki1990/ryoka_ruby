\documentclass[10pt,b5j]{tarticle} % B6 縦書き
% \documentclass[10pt,b5j]{tarticle} % B6 縦書き
\AtBeginDvi{\special{papersize=128mm,182mm}} % B6 用用紙サイズ
\usepackage{otf} % Unicode で字を入力するのに必要なパッケージ
\usepackage[size=b6j]{bxpapersize} % B6 用紙サイズを指定
\usepackage[dvipdfmx]{graphicx} % 画像を挿入するためのパッケージ
\usepackage[dvipdfmx]{color} % 色をつけるためのパッケージ
\usepackage{pxrubrica} % ルビを振るためのパッケージ
\usepackage{multicol} % 複数段組を作るためのパッケージ
\setlength{\topmargin}{14mm} % 上下方向のマージン
\addtolength{\topmargin}{-1in} % 
\setlength{\oddsidemargin}{11mm} % 左右方向のマージン
\addtolength{\oddsidemargin}{-1in} % 
\setlength{\textwidth}{154mm} % B6 用
\setlength{\textheight}{108mm} % B6 用
\setlength{\headsep}{0mm} % 
\setlength{\headheight}{0mm} % 
\setlength{\topskip}{0mm} % 
\setlength{\parskip}{0pt} % 
\def\labelenumi{\theenumi、} % 箇条書きのフォーマット
\parindent = 0pt % 段落下げしない

 % B6 用テンプレート読み込み

\begin{document}
% begin header
%%%%% タイトルと作者 ここから %%%%%
\begin{minipage}[c]{0.7\hsize} % タイトルは上から 7 割
    \begin{center}
    % begin title
        {\LARGE
            春雨に濡る % タイトルを入れる
        }
        {\small 
            (大正十二年寮歌) % 年などを入れる
        }
    % end title
    \end{center}
\end{minipage}
\begin{minipage}[c]{0.3\hsize} % 作歌作曲は上から 3 割
    \begin{flushright} % 下寄せにする
        % begin name
        高橋北雄君 作歌\\西田貫道君 作曲 % 作歌・作曲者
        % end name
    \end{flushright}
\end{minipage}
%%%%% タイトルと作者 ここまで %%%%%
% (1,2,3,4 了あり)
% end header

% begin length
\vspace{1.5em} % タイトル, 作者と歌詞の間に隙間を設ける
\newcommand{\linespace}{0.5em} % 行間の設定
\newcommand{\blocksize}{0.5\hsize} % 段組間の設定
\newcommand{\itemmargin}{3em} % 曲番の位置調整の長さ
% end length
% begin body
%%%%% 歌詞 ここから %%%%%
\begin{enumerate} % 番号の箇条書き ここから
    \setlength{\itemindent}{\itemmargin} % 曲番の位置調整
    \begin{minipage}[c]{\blocksize}
    
        \vspace{\linespace}
        \item~\\
        % 1.
        \ruby{春雨}{}に\ruby{濡}{}るアカシヤ\ruby{花}{}\\
        \ruby{街路}{}の\ruby{灯}{}はなやかに\\
        \ruby{地}{}は\ruby{銀鼠}{}にたそがるる\\
        \ruby{寂}{}かに\ruby{歩}{}む\ruby{若人}{}が\\
        \ruby{心}{}にめざむ\ruby{爽}{}かの\\
        \ruby{灑}{}み\ruby{充}{}てる\ruby{力}{}かな
        
    \end{minipage}
    \begin{minipage}[c]{\blocksize}
        
        \vspace{\linespace}
        \item~\\
        % 2.
        \ruby{夏}{}の\ruby{入陽}{}に\ruby{砂丘}{}の\\
        \ruby{猟虎}{}の\ruby{骨}{}に\ruby{鷗飛}{}ぶ\\
        \ruby{融}{}けざる\ruby{銀}{}の\ruby{山脈}{}は\\
        \ruby{碧薄}{}れゆく\ruby{空}{}にうく\\
        \ruby{名残}{}の\ruby{光身}{}にあびて\\
        \ruby{異郷}{}の\ruby{方}{}を\ruby{思}{}ふかな
        
    \end{minipage}
    \begin{minipage}[c]{\blocksize}
        
        \vspace{\linespace}
        \item~\\
        % 3.
        \ruby{仄青白}{}き\ruby{白樺}{}や\\
        \ruby{落葉}{}ふむ\ruby{音寂}{}しくも\\
        \ruby{谷}{}また\ruby{谷}{}を\ruby{辿}{}り\ruby{行}{}き\\
        \ruby{今宵}{}は\ruby{淡}{}き\ruby{夢見}{}んと\\
        \ruby{焚火}{}を\ruby{囲}{}み\ruby{歌}{}ふ\ruby{寮歌}{}\\
        \ruby{紫紺}{}の\ruby{闇}{}に\ruby{解}{}けて\ruby{行}{}く
        
    \end{minipage}
    \begin{minipage}[c]{\blocksize}
        
        \vspace{\linespace}
        \item~\\
        % 4.
        \ruby{青}{}き\ruby{空透}{}き\ruby{銀}{}の\ruby{月}{}\\
        \ruby{石狩}{}の\ruby{河波光}{}る\\
        \ruby{雪}{}の\ruby{野限}{}は\ruby{靄}{}こめて\\
        \ruby{灯漂}{}ふアイヌ\ruby{小屋}{}\\
        \ruby{琥珀}{}の\ruby{酒}{}を\ruby{汲}{}み\ruby{交}{}し\\
        \ruby{王者}{}の\ruby{誇偲}{}ぶかな
    
    \end{minipage}
\end{enumerate} % 番号の箇条書き ここまで
%%%%% 歌詞 ここまで %%%%%
% end body

\end{document}

\documentclass[10pt,b5j]{tarticle} % B6 縦書き
% \documentclass[10pt,b5j]{tarticle} % B6 縦書き
\AtBeginDvi{\special{papersize=128mm,182mm}} % B6 用用紙サイズ
\usepackage{otf} % Unicode で字を入力するのに必要なパッケージ
\usepackage[size=b6j]{bxpapersize} % B6 用紙サイズを指定
\usepackage[dvipdfmx]{graphicx} % 画像を挿入するためのパッケージ
\usepackage[dvipdfmx]{color} % 色をつけるためのパッケージ
\usepackage{pxrubrica} % ルビを振るためのパッケージ
\usepackage{multicol} % 複数段組を作るためのパッケージ
\setlength{\topmargin}{14mm} % 上下方向のマージン
\addtolength{\topmargin}{-1in} % 
\setlength{\oddsidemargin}{11mm} % 左右方向のマージン
\addtolength{\oddsidemargin}{-1in} % 
\setlength{\textwidth}{154mm} % B6 用
\setlength{\textheight}{108mm} % B6 用
\setlength{\headsep}{0mm} % 
\setlength{\headheight}{0mm} % 
\setlength{\topskip}{0mm} % 
\setlength{\parskip}{0pt} % 
\def\labelenumi{\theenumi、} % 箇条書きのフォーマット
\parindent = 0pt % 段落下げしない

 % B6 用テンプレート読み込み

\begin{document}
% begin header
%%%%% タイトルと作者 ここから %%%%%
\begin{minipage}[c]{0.7\hsize} % タイトルは上から 7 割
    \begin{center}
    % begin title
        {\LARGE
            遙かなる迪 % タイトルを入れる
        }
        {\small 
            (平成十七年度寮歌) % 年などを入れる
        }
    % end title
    \end{center}
\end{minipage}
\begin{minipage}[c]{0.3\hsize} % 作歌作曲は上から 3 割
    \begin{flushright} % 下寄せにする
        % begin name
        加藤信泰君 作歌\\福岡萌君 作曲 % 作歌・作曲者
        % end name
    \end{flushright}
\end{minipage}
%%%%% タイトルと作者 ここまで %%%%%
% (1,2,3 了なし繰り返しあり)
% end header

% begin length
\vspace{1.5em} % タイトル, 作者と歌詞の間に隙間を設ける
\newcommand{\linespace}{0.5em} % 行間の設定
\newcommand{\blocksize}{0.5\hsize} % 段組間の設定
\newcommand{\itemmargin}{6em} % 曲番の位置調整の長さ
% end length
% begin body
%%%%% 歌詞 ここから %%%%%
\begin{enumerate} % 番号の箇条書き ここから
    \setlength{\itemindent}{\itemmargin} % 曲番の位置調整
    \begin{minipage}[c]{\blocksize}
    
        \vspace{\linespace}
        \item~\\
        % 1.
        \ruby{繁滋}{}なる\ruby{思}{}いを\ruby{秘}{}して\ruby{寮}{}の\\
        \ruby{門}{}をくぐりし\ruby{若人}{}は\\
        \ruby{意気試}{}され\ruby{育}{}まれ\\
        \ruby{熱}{}き\ruby{契}{}りの\ruby{友}{}を\ruby{得}{}ん\\
        \ruby{楡}{}の\ruby{若葉曜}{}くごとく\\
        \ruby{遙}{}かなる\ruby{迪}{}に\ruby{根}{}を\ruby{張}{}らん
        
        \vspace{\linespace}
        \item~\\
        % 2.
        \ruby{時}{}は\ruby{過}{}ぎ\\
        \ruby{大地}{}に\ruby{根}{}を\ruby{張}{}る\ruby{若芽}{}らは\\
        \ruby{思}{}い\ruby{託}{}され\ruby{懊悩}{}しつつ\\
        \ruby{切磋琢磨}{}し\ruby{歩}{}む\ruby{毎}{}\\
        \ruby{寮支}{}える\ruby{大樹}{}とならん\\
        \ruby{祭}{}りの\ruby{燈火燿}{}くごとく\\
        \ruby{遙}{}かなる\ruby{迪}{}を\ruby{継}{}ぎ\ruby{行}{}かん
        
        \vspace{\linespace}
        \item~\\
        % 3.
        \ruby{何時}{}の\ruby{日}{}か\\
        ここで\ruby{学}{}びしひとごとが\\
        かけがえのない\ruby{寶}{}とならん\\
        \ruby{別}{}るる\ruby{友}{}に\ruby{思}{}いを\ruby{託}{}し\\
        \ruby{旅立}{}つ\ruby{未来}{}は\ruby{暗}{}くとも\\
        \ruby{雪野}{}に\ruby{朝日耀}{}くごとく\\
        \ruby{遙}{}かなる\ruby{迪}{}に\ruby{出}{}で\ruby{行}{}かん

    
    \end{minipage}
\end{enumerate} % 番号の箇条書き ここまで
%%%%% 歌詞 ここまで %%%%%
% end body

\end{document}

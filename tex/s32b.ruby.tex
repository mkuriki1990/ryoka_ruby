\documentclass[10pt,b5j]{tarticle} % B6 縦書き
% \documentclass[10pt,b5j]{tarticle} % B6 縦書き
\AtBeginDvi{\special{papersize=128mm,182mm}} % B6 用用紙サイズ
\usepackage{otf} % Unicode で字を入力するのに必要なパッケージ
\usepackage[size=b6j]{bxpapersize} % B6 用紙サイズを指定
\usepackage[dvipdfmx]{graphicx} % 画像を挿入するためのパッケージ
\usepackage[dvipdfmx]{color} % 色をつけるためのパッケージ
\usepackage{pxrubrica} % ルビを振るためのパッケージ
\usepackage{multicol} % 複数段組を作るためのパッケージ
\setlength{\topmargin}{14mm} % 上下方向のマージン
\addtolength{\topmargin}{-1in} % 
\setlength{\oddsidemargin}{11mm} % 左右方向のマージン
\addtolength{\oddsidemargin}{-1in} % 
\setlength{\textwidth}{154mm} % B6 用
\setlength{\textheight}{108mm} % B6 用
\setlength{\headsep}{0mm} % 
\setlength{\headheight}{0mm} % 
\setlength{\topskip}{0mm} % 
\setlength{\parskip}{0pt} % 
\def\labelenumi{\theenumi、} % 箇条書きのフォーマット
\parindent = 0pt % 段落下げしない

 % B6 用テンプレート読み込み

\begin{document}
% begin header
%%%%% タイトルと作者 ここから %%%%%
\begin{minipage}[c]{0.7\hsize} % タイトルは上から 7 割
    \begin{center}
    % begin title
        {\LARGE
            花咲き散りて % タイトルを入れる
        }
        {\small 
            (昭和三十二年第五十回記念祭歌) % 年などを入れる
        }
    % end title
    \end{center}
\end{minipage}
\begin{minipage}[c]{0.3\hsize} % 作歌作曲は上から 3 割
    \begin{flushright} % 下寄せにする
        % begin name
        佐伯政英君 作歌\\小椋進君 作曲 % 作歌・作曲者
        % end name
    \end{flushright}
\end{minipage}
%%%%% タイトルと作者 ここまで %%%%%
% (1,2,3,4 繰り返しなし)
% end header

% begin length
\vspace{1.5em} % タイトル, 作者と歌詞の間に隙間を設ける
\newcommand{\linespace}{0.5em} % 行間の設定
\newcommand{\blocksize}{0.5\hsize} % 段組間の設定
\newcommand{\itemmargin}{6em} % 曲番の位置調整の長さ
% end length
% begin body
%%%%% 歌詞 ここから %%%%%
\begin{enumerate} % 番号の箇条書き ここから
    \setlength{\itemindent}{\itemmargin} % 曲番の位置調整
    \begin{minipage}[c]{\blocksize}
    
        \vspace{\linespace}
        \item~\\
        % 1.
        \ruby{花咲}{}き\ruby{散}{}りて\ruby{五十年}{}の\\
        \ruby{実}{}りの\ruby{秋}{}ぞ\ruby{幸多}{}く\\
        ここに\ruby{我等}{}が\ruby{記念祭}{}\\
        \ruby{寮友}{}よ\ruby{歌}{}えよいざやいざ
        
        \vspace{\linespace}
        \item~\\
        % 2.
        \ruby{楡}{}の\ruby{大樹}{}に\ruby{尋}{}ぬれば\\
        \ruby{誇}{}らしげにぞ\ruby{答}{}えたる\\
        \ruby{我}{}が\ruby{先人}{}の\ruby{青春}{}の\ruby{理想}{}\\
        \ruby{寮友}{}よ\ruby{讃}{}えよいざやいざ
        
        \vspace{\linespace}
        \item~\\
        % 3.
        \ruby{歌声原始林}{}にこだまして\\
        \ruby{爆笑夜空}{}をつんざけば\\
        \ruby{楡}{}の\ruby{精}{}さえ\ruby{踊}{}らなん\\
        \ruby{寮友}{}よ\ruby{踊}{}れよいざやいざ
        
        \vspace{\linespace}
        \item~\\
        % 4.
        \ruby{我等}{}が\ruby{誇}{}る\ruby{自治寮}{}に\\
        \ruby{五十回}{}めぐる\ruby{記念祭}{}\\
        さらに\ruby{栄}{}ゆく\ruby{此}{}の\ruby{寮}{}を\\
        \ruby{寮友}{}よ\ruby{讃}{}えよいざやいざ
    
    \end{minipage}
\end{enumerate} % 番号の箇条書き ここまで
%%%%% 歌詞 ここまで %%%%%
% end body

\end{document}

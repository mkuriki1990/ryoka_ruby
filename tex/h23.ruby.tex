\documentclass[10pt,b5j]{tarticle} % B6 縦書き
% \documentclass[10pt,b5j]{tarticle} % B6 縦書き
\AtBeginDvi{\special{papersize=128mm,182mm}} % B6 用用紙サイズ
\usepackage{otf} % Unicode で字を入力するのに必要なパッケージ
\usepackage[size=b6j]{bxpapersize} % B6 用紙サイズを指定
\usepackage[dvipdfmx]{graphicx} % 画像を挿入するためのパッケージ
\usepackage[dvipdfmx]{color} % 色をつけるためのパッケージ
\usepackage{pxrubrica} % ルビを振るためのパッケージ
\usepackage{plext} % 漢数字の enumerate を使うためのパッケージ
\usepackage{multicol} % 複数段組を作るためのパッケージ
\setlength{\topmargin}{14mm} % 上下方向のマージン
\addtolength{\topmargin}{-1in} % 
\setlength{\oddsidemargin}{11mm} % 左右方向のマージン
\addtolength{\oddsidemargin}{-1in} % 
\setlength{\textwidth}{154mm} % B6 用
\setlength{\textheight}{108mm} % B6 用
\setlength{\headsep}{0mm} % 
\setlength{\headheight}{0mm} % 
\setlength{\topskip}{0mm} % 
\setlength{\parskip}{0pt} % 
\def\theenumi{\Kanji{enumi}} % 箇条書きのフォーマットを漢数字に変更
\parindent = 0pt % 段落下げしない
\pagestyle{empty} % すべてのページ番号を消去
% \renewcommand{\baselinestretch}{0.9} % 行間の倍率
 % B6 用テンプレート読み込み

\begin{document}
% begin header
%%%%% タイトルと作者 ここから %%%%%
\begin{minipage}[c]{0.7\hsize} % タイトルは上から 7 割
    \begin{center}
    % begin title
        {\LARGE
            広がるはただ青き旅路ぞ % タイトルを入れる
        }
        {\small 
            (平成二十三年度寮歌) % 年などを入れる
        }
    % end title
    \end{center}
\end{minipage}
\begin{minipage}[c]{0.3\hsize} % 作歌作曲は上から 3 割
    \begin{flushright} % 下寄せにする
        % begin name
        安田龍平君 作歌\\我如古弥司君 作曲 % 作歌・作曲者
        % end name
    \end{flushright}
\end{minipage}
%%%%% タイトルと作者 ここまで %%%%%
% (了なし繰り返しあり)
% end header

% begin body
\vspace{1.5em} % タイトル, 作者と歌詞の間に隙間を設ける
\newcommand{\linespace}{0.5em} % 行間の設定
\newcommand{\blocksize}{0.5\hsize} % 段組間の設定
%%%%% 歌詞 ここから %%%%%
% begin lilycs
\begin{enumerate} % 番号の箇条書き ここから
    \begin{minipage}[c]{\blocksize}
    
        \vspace{\linespace}
        \item
        \ruby{春風吹}{}きゆく\ruby{原始}{}の\ruby{森}{}に\\
        \ruby{吾}{}れ\ruby{微睡}{}みて\ruby{酒宴}{}して\\
        \ruby{逍遥}{}すれども\ruby{其}{}の\ruby{歩}{}は\ruby{止}{}まず\\
        \ruby{危急}{}の\ruby{時代}{}にあればこそ\\
        \ruby{渦巻}{}く\ruby{疾風吾}{}が\ruby{雄}{}を\ruby{呼}{}び\\
        \ruby{怒濤}{}は\ruby{汝}{}れに\ruby{義}{}を\ruby{求}{}む\\
        \ruby{今}{}ぞ\ruby{吾等}{}が\ruby{誠}{}を\ruby{奮}{}い\\
        \ruby{高唱}{}いて\ruby{進}{}まん\ruby{青}{}き\ruby{旅路}{}を
        
        \vspace{\linespace}
        \item
        \ruby{星}{}は\ruby{昂々美稲超}{}えて\\
        \ruby{玉黍}{}を\ruby{食}{}む\ruby{旅鳥}{}や\\
        \ruby{染}{}まず\ruby{彷徨}{}う\ruby{其}{}が\ruby{白羽}{}に\\
        \ruby{斗星}{}と\ruby{大志}{}の\ruby{結}{}ぶ\ruby{瞬間}{}\\
        \ruby{広}{}がるはただ\ruby{青}{}き\ruby{旅路}{}ぞ
        
        \vspace{\linespace}
        \item
        \ruby{花}{}は\ruby{灼々壌撃}{}つ\ruby{酔}{}いを\\
        \ruby{君影草}{}の\ruby{鈴音}{}にきく\\
        さればこの\ruby{手}{}を\ruby{春陽高}{}く\\
        \ruby{翳}{}して\ruby{情熱}{}をうち\ruby{燃}{}やし\\
        \ruby{濃緑}{}に\ruby{萠}{}ゆ\ruby{白花}{}に\ruby{誇}{}らん
        
        \vspace{\linespace}
        \item
        \ruby{月}{}は\ruby{朧々輝光}{}は\ruby{幽}{}か\\
        \ruby{梢叢分}{}けて\ruby{河}{}に\ruby{落}{}つ\\
        \ruby{水面}{}に\ruby{透}{}くきみが\ruby{底}{}に\\
        \ruby{己}{}が\ruby{混濁}{}をうつし\ruby{見}{}て\\
        \ruby{孤月仰}{}ぐ\ruby{子}{}よ\ruby{誰}{}が\ruby{為}{}に\ruby{泣}{}く
        
        \vspace{\linespace}
        \item
        \ruby{雪}{}は\ruby{皚々大地軋}{}めて\\
        \ruby{氷嵐}{}まさに\ruby{街}{}を\ruby{呑}{}む\\
        \ruby{無明}{}の\ruby{曠野}{}に\ruby{巨熊眠}{}るも\\
        \ruby{弦}{}を\ruby{矜持}{}と\ruby{爪弾}{}けば\\
        \ruby{嗚呼黎明}{}に\ruby{吹雪}{}も\ruby{霧散}{}す
        
        \vspace{\linespace}
        \item
        \ruby{宙}{}は\ruby{悠々逍遥}{}の\ruby{果}{}て\\
        \ruby{芝草}{}を\ruby{枕}{}に\ruby{星}{}を\ruby{抱}{}く\\
        \ruby{有情}{}の\ruby{声}{}に\ruby{朋友和}{}す\ruby{寮歌}{}を\\
        \ruby{讃}{}えて\ruby{天宙}{}を\ruby{見仰}{}げば\\
        \ruby{広}{}がるはただ\ruby{青}{}き\ruby{旅路}{}ぞ
    
    \end{minipage}
\end{enumerate} % 番号の箇条書き ここまで
% end lilycs
%%%%% 歌詞 ここまで %%%%%
% end body

\end{document}

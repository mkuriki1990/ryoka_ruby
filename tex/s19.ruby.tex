\documentclass[10pt,b5j]{tarticle} % B6 縦書き
% \documentclass[10pt,b5j]{tarticle} % B6 縦書き
\AtBeginDvi{\special{papersize=128mm,182mm}} % B6 用用紙サイズ
\usepackage{otf} % Unicode で字を入力するのに必要なパッケージ
\usepackage[size=b6j]{bxpapersize} % B6 用紙サイズを指定
\usepackage[dvipdfmx]{graphicx} % 画像を挿入するためのパッケージ
\usepackage[dvipdfmx]{color} % 色をつけるためのパッケージ
\usepackage{pxrubrica} % ルビを振るためのパッケージ
\usepackage{multicol} % 複数段組を作るためのパッケージ
\setlength{\topmargin}{14mm} % 上下方向のマージン
\addtolength{\topmargin}{-1in} % 
\setlength{\oddsidemargin}{11mm} % 左右方向のマージン
\addtolength{\oddsidemargin}{-1in} % 
\setlength{\textwidth}{154mm} % B6 用
\setlength{\textheight}{108mm} % B6 用
\setlength{\headsep}{0mm} % 
\setlength{\headheight}{0mm} % 
\setlength{\topskip}{0mm} % 
\setlength{\parskip}{0pt} % 
\def\labelenumi{\theenumi、} % 箇条書きのフォーマット
\parindent = 0pt % 段落下げしない

 % B6 用テンプレート読み込み

\begin{document}
% begin header
%%%%% タイトルと作者 ここから %%%%%
\begin{minipage}[c]{0.7\hsize} % タイトルは上から 7 割
    \begin{center}
    % begin title
        {\LARGE
            雪解の楡陵の % タイトルを入れる
        }
        {\small 
            (昭和十九年寮歌) % 年などを入れる
        }
    % end title
    \end{center}
\end{minipage}
\begin{minipage}[c]{0.3\hsize} % 作歌作曲は上から 3 割
    \begin{flushright} % 下寄せにする
        % begin name
        鈴木信夫君 作歌\\竹山賢治君 作曲 % 作歌・作曲者
        % end name
    \end{flushright}
\end{minipage}
%%%%% タイトルと作者 ここまで %%%%%
% (1,2,3 了あり)
% end header

% begin body
\vspace{1.5em} % タイトル, 作者と歌詞の間に隙間を設ける
\newcommand{\linespace}{0.5em} % 行間の設定
\newcommand{\blocksize}{0.5\hsize} % 段組間の設定
%%%%% 歌詞 ここから %%%%%
% begin lilycs
\begin{enumerate} % 番号の箇条書き ここから
    \begin{minipage}[c]{\blocksize}
    
        \vspace{\linespace}
        \item
        % 1.
        \ruby{雪解}{}の\ruby{楡陵}{}の\ruby{一流}{}や\\
        \ruby{岸辺}{}に\ruby{憩}{}ふ\ruby{水鳥}{}の\\
        \ruby{孤影}{}ぞしばし\ruby{春}{}の\ruby{水面}{}\\
        ああ\ruby{石狩}{}の\ruby{天空晴}{}れて\\
        \ruby{轟}{}け\ruby{謳}{}ふ\ruby{恵迪}{}の\\
        \ruby{児等}{}が\ruby{生命}{}や\ruby{聖}{}からん
        
        \vspace{\linespace}
        \item
        % 2.
        \ruby{歓喜憂苦}{}を\ruby{共}{}にせむ\\
        \ruby{結}{}ぶ\ruby{契}{}の\ruby{盃}{}に\\
        \ruby{松}{}の\ruby{枝漏}{}るる\ruby{月影}{}や\\
        \ruby{人生意気}{}に\ruby{感}{}じてか\\
        \ruby{集}{}ひし\ruby{雁}{}の\ruby{行}{}く\ruby{手稲}{}\\
        \ruby{青雲}{}の\ruby{峯巍峨}{}の\ruby{峯}{}
        
        \vspace{\linespace}
        \item
        % 3.
        いざや\ruby{伝統}{}の\ruby{聖火}{}を\ruby{翳}{}し\\
        \ruby{先人}{}の\ruby{絢夢偲}{}びつつ\\
        \ruby{寮祭}{}の\ruby{庭}{}に\ruby{四十回}{}の\\
        \ruby{春風頬涙}{}を\ruby{乾}{}すなれば\\
        \ruby{散}{}りゆく\ruby{夜迷雲}{}のかげ\ruby{消}{}えて\\
        \ruby{声}{}を\ruby{限}{}りの\ruby{感激}{}かな
        
        \vspace{\linespace}
        \item
        % 4.
        \ruby{南}{}の\ruby{海}{}の\ruby{有明}{}に\\
        \ruby{燦}{}く\ruby{星辰}{}の\ruby{消}{}え\ruby{果}{}てて\\
        \ruby{散}{}りぬる\ruby{若桜}{}もあるぞかし\\
        いかで\ruby{我等}{}の\ruby{蹶起}{}ざらん\\
        \ruby{義憤}{}が\ruby{胸}{}にほのぼのと\\
        \ruby{染}{}め\ruby{映}{}えにしか\ruby{朝日影}{}
        
        \vspace{\linespace}
        \item
        % 5.
        \ruby{噫世}{}は\ruby{変遷}{}り\ruby{人変}{}り\\
        \ruby{舘}{}の\ruby{原始林}{}は\ruby{愁}{}へども\\
        \ruby{剛毅}{}の\ruby{大旆仰}{}ぎてし\\
        \ruby{熱血燃}{}ゆる\ruby{益良夫}{}が\\
        \ruby{皇国}{}の\ruby{道}{}に\ruby{挺身}{}まんと\\
        \ruby{誓}{}ひし\ruby{眸}{}に\ruby{光輝}{}あれ
    
    \end{minipage}
\end{enumerate} % 番号の箇条書き ここまで
% end lilycs
%%%%% 歌詞 ここまで %%%%%
% end body

\end{document}

\documentclass[10pt,b5j]{tarticle} % B6 縦書き
% \documentclass[10pt,b5j]{tarticle} % B6 縦書き
\AtBeginDvi{\special{papersize=128mm,182mm}} % B6 用用紙サイズ
\usepackage{otf} % Unicode で字を入力するのに必要なパッケージ
\usepackage[size=b6j]{bxpapersize} % B6 用紙サイズを指定
\usepackage[dvipdfmx]{graphicx} % 画像を挿入するためのパッケージ
\usepackage[dvipdfmx]{color} % 色をつけるためのパッケージ
\usepackage{pxrubrica} % ルビを振るためのパッケージ
\usepackage{multicol} % 複数段組を作るためのパッケージ
\setlength{\topmargin}{14mm} % 上下方向のマージン
\addtolength{\topmargin}{-1in} % 
\setlength{\oddsidemargin}{11mm} % 左右方向のマージン
\addtolength{\oddsidemargin}{-1in} % 
\setlength{\textwidth}{154mm} % B6 用
\setlength{\textheight}{108mm} % B6 用
\setlength{\headsep}{0mm} % 
\setlength{\headheight}{0mm} % 
\setlength{\topskip}{0mm} % 
\setlength{\parskip}{0pt} % 
\def\labelenumi{\theenumi、} % 箇条書きのフォーマット
\parindent = 0pt % 段落下げしない

 % B6 用テンプレート読み込み

\begin{document}
% begin header
%%%%% タイトルと作者 ここから %%%%%
\begin{minipage}[c]{0.7\hsize} % タイトルは上から 7 割
    \begin{center}
    % begin title
        {\LARGE
            昭和6年応援歌 % タイトルを入れる
        }
        {\small 
             % 年などを入れる
        }
    % end title
    \end{center}
\end{minipage}
\begin{minipage}[c]{0.3\hsize} % 作歌作曲は上から 3 割
    \begin{flushright} % 下寄せにする
        % begin name
        北大管弦楽部君 作歌・作曲 % 作歌・作曲者
        % end name
    \end{flushright}
\end{minipage}
%%%%% タイトルと作者 ここまで %%%%%
% % end header

% begin length
\vspace{1.5em} % タイトル, 作者と歌詞の間に隙間を設ける
\newcommand{\linespace}{0.5em} % 行間の設定
\newcommand{\blocksize}{0.5\hsize} % 段組間の設定
\newcommand{\itemmargin}{6em} % 曲番の位置調整の長さ
% end length
% begin body
%%%%% 歌詞 ここから %%%%%
\begin{enumerate} % 番号の箇条書き ここから
    \setlength{\itemindent}{\itemmargin} % 曲番の位置調整
    \begin{minipage}[c]{\blocksize}
    
        \vspace{\linespace}
        \item~\\
        % 1.
        \ruby{戦闘}{}の\ruby{暗雲今低}{}く\ruby{垂}{}れ\\
        \ruby{喊声凄}{}き\ruby{北州}{}の\ruby{蒼穹}{}\\
        \ruby{九百}{}の\ruby{健児}{}\\
        \ruby{胸}{}は\ruby{火}{}ともえて\\
        \ruby{熱涙}{}もて\ruby{謳}{}ふいざ\ruby{進}{}め\ruby{選手}{}\\
        \ruby{輝}{}く\ruby{覇権}{}おゝ!!\ruby{北大予科}{}\\
        \ruby{輝}{}く\ruby{覇権}{}おゝ!!\ruby{北大予科}{}
        
        \vspace{\linespace}
        \item~\\
        % 2.
        \ruby{桜星}{}の\ruby{旗見}{}よ\ruby{天}{}を\ruby{摩}{}し\\
        \ruby{北州}{}の\ruby{蒼穹勝闘}{}あがり\\
        \ruby{九百}{}の\ruby{健児}{}\\
        \ruby{血潮高鳴}{}る\\
        \ruby{栄}{}えある\ruby{吾}{}が\ruby{選手}{}よ\ruby{戦}{}は\ruby{勝}{}ちたり\\
        \ruby{輝}{}く\ruby{覇権}{}おゝ!!\ruby{北大予科}{}\\
        \ruby{輝}{}く\ruby{覇権}{}おゝ!!\ruby{北大予科}{}
    
    \end{minipage}
\end{enumerate} % 番号の箇条書き ここまで
%%%%% 歌詞 ここまで %%%%%
% end body

\end{document}

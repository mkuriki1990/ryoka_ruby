\documentclass[10pt,b5j]{tarticle} % B6 縦書き
% \documentclass[10pt,b5j]{tarticle} % B6 縦書き
\AtBeginDvi{\special{papersize=128mm,182mm}} % B6 用用紙サイズ
\usepackage{otf} % Unicode で字を入力するのに必要なパッケージ
\usepackage[size=b6j]{bxpapersize} % B6 用紙サイズを指定
\usepackage[dvipdfmx]{graphicx} % 画像を挿入するためのパッケージ
\usepackage[dvipdfmx]{color} % 色をつけるためのパッケージ
\usepackage{pxrubrica} % ルビを振るためのパッケージ
\usepackage{multicol} % 複数段組を作るためのパッケージ
\setlength{\topmargin}{14mm} % 上下方向のマージン
\addtolength{\topmargin}{-1in} % 
\setlength{\oddsidemargin}{11mm} % 左右方向のマージン
\addtolength{\oddsidemargin}{-1in} % 
\setlength{\textwidth}{154mm} % B6 用
\setlength{\textheight}{108mm} % B6 用
\setlength{\headsep}{0mm} % 
\setlength{\headheight}{0mm} % 
\setlength{\topskip}{0mm} % 
\setlength{\parskip}{0pt} % 
\def\labelenumi{\theenumi、} % 箇条書きのフォーマット
\parindent = 0pt % 段落下げしない

 % B6 用テンプレート読み込み

\begin{document}
% begin header
%%%%% タイトルと作者 ここから %%%%%
\begin{minipage}[c]{0.7\hsize} % タイトルは上から 7 割
    \begin{center}
    % begin title
        {\LARGE
            北海道帝国大学独立記念歌 % タイトルを入れる
        }
        {\small 
            (大正七年) % 年などを入れる
        }
    % end title
    \end{center}
\end{minipage}
\begin{minipage}[c]{0.3\hsize} % 作歌作曲は上から 3 割
    \begin{flushright} % 下寄せにする
        % begin name
         % 作歌・作曲者
        % end name
    \end{flushright}
\end{minipage}
%%%%% タイトルと作者 ここまで %%%%%
% % end header

% begin body
\vspace{1.5em} % タイトル, 作者と歌詞の間に隙間を設ける
\newcommand{\linespace}{0.5em} % 行間の設定
\newcommand{\blocksize}{0.5\hsize} % 段組間の設定
%%%%% 歌詞 ここから %%%%%
% begin lilycs
\begin{enumerate} % 番号の箇条書き ここから
    \begin{minipage}[c]{\blocksize}
    
        \vspace{\linespace}
        \item
        % 1.
        \ruby{都}{}の\ruby{花}{}を\ruby{吹}{}く\ruby{風}{}の\\
        \ruby{津輕}{}の\ruby{海}{}をこえくれば\\
        \ruby{石狩}{}の\ruby{野辺雪消}{}えて\\
        うら\ruby{若草}{}の\ruby{香}{}も\ruby{高}{}く\\
        \ruby{白雲空}{}に\ruby{行通}{}ひて\\
        \ruby{羊}{}の\ruby{夢}{}ぞ\ruby{長閑}{}なる
        
        \vspace{\linespace}
        \item
        % 2.
        さあれ\ruby{平和}{}の\ruby{夢}{}の\ruby{夢}{}\\
        \ruby{見}{}よ\ruby{西欧}{}の\ruby{空}{}の\ruby{様}{}\\
        \ruby{怪雲荒}{}び\ruby{暴風吠}{}え\\
        シベリヤの\ruby{春}{}の\ruby{色}{}もなく\\
        \ruby{狂風千里胡砂}{}を\ruby{捲}{}き\\
        \ruby{日本海}{}に\ruby{波高}{}し
        
        \vspace{\linespace}
        \item
        % 3.
        \ruby{今}{}ぞ\ruby{皇国多事}{}の\ruby{時}{}\\
        \ruby{北}{}の\ruby{守}{}の\ruby{北州}{}に\\
        \ruby{護国}{}の\ruby{子等}{}が\ruby{学}{}び\ruby{舎}{}の\\
        \ruby{弥}{}や\ruby{栄}{}えゆく\ruby{喜}{}を\\
        \ruby{心}{}に\ruby{永}{}くしるさんと\\
        \ruby{歌}{}ごゑ\ruby{高}{}き\ruby{春今宵}{}
        
        \vspace{\linespace}
        \item
        <「\ruby{藻岩}{}の\ruby{緑}{}」の\ruby{譜}{}による>
    
    \end{minipage}
\end{enumerate} % 番号の箇条書き ここまで
% end lilycs
%%%%% 歌詞 ここまで %%%%%
% end body

\end{document}

\documentclass[10pt,b5j]{tarticle} % B6 縦書き
% \documentclass[10pt,b5j]{tarticle} % B6 縦書き
\AtBeginDvi{\special{papersize=128mm,182mm}} % B6 用用紙サイズ
\usepackage{otf} % Unicode で字を入力するのに必要なパッケージ
\usepackage[size=b6j]{bxpapersize} % B6 用紙サイズを指定
\usepackage[dvipdfmx]{graphicx} % 画像を挿入するためのパッケージ
\usepackage[dvipdfmx]{color} % 色をつけるためのパッケージ
\usepackage{pxrubrica} % ルビを振るためのパッケージ
\usepackage{multicol} % 複数段組を作るためのパッケージ
\setlength{\topmargin}{14mm} % 上下方向のマージン
\addtolength{\topmargin}{-1in} % 
\setlength{\oddsidemargin}{11mm} % 左右方向のマージン
\addtolength{\oddsidemargin}{-1in} % 
\setlength{\textwidth}{154mm} % B6 用
\setlength{\textheight}{108mm} % B6 用
\setlength{\headsep}{0mm} % 
\setlength{\headheight}{0mm} % 
\setlength{\topskip}{0mm} % 
\setlength{\parskip}{0pt} % 
\def\labelenumi{\theenumi、} % 箇条書きのフォーマット
\parindent = 0pt % 段落下げしない

 % B6 用テンプレート読み込み

\begin{document}
% begin header
%%%%% タイトルと作者 ここから %%%%%
\begin{minipage}[c]{0.7\hsize} % タイトルは上から 7 割
    \begin{center}
    % begin title
        {\LARGE
            北海道帝国大学独立記念歌 % タイトルを入れる
        }
        {\small 
            (大正七年) % 年などを入れる
        }
    % end title
    \end{center}
\end{minipage}
\begin{minipage}[c]{0.3\hsize} % 作歌作曲は上から 3 割
    \begin{flushright} % 下寄せにする
        % begin name
         % 作歌・作曲者
        % end name
    \end{flushright}
\end{minipage}
%%%%% タイトルと作者 ここまで %%%%%
% % end header

% begin length
\vspace{1.5em} % タイトル, 作者と歌詞の間に隙間を設ける
\newcommand{\linespace}{0.5em} % 行間の設定
\newcommand{\blocksize}{0.5\hsize} % 段組間の設定
\newcommand{\itemmargin}{3em} % 曲番の位置調整の長さ
% end length
% begin body
%%%%% 歌詞 ここから %%%%%
\begin{enumerate} % 番号の箇条書き ここから
    \setlength{\itemindent}{\itemmargin} % 曲番の位置調整
    \begin{minipage}[c]{\blocksize}
    
        \vspace{\linespace}
        \item~\\
        % 1.
        \ruby{都}{と}の\ruby{花}{はな}を\ruby{吹}{ふ}く\ruby{風}{かぜ}の\\
        \ruby{津}{つ}\ruby{輕}{}の\ruby{海}{うみ}をこえくれば\\
        \ruby{石狩}{いしかり}の\ruby{野辺}{のべ}\ruby{雪}{ゆき}\ruby{消}{き}えて\\
        うら\ruby{若}{うらわか}\ruby{草}{くさ}の\ruby{香}{こう}も\ruby{高}{たか}く\\
        \ruby{白雲}{しらくも}\ruby{空}{そら}に\ruby{行}{くだり}\ruby{通}{どおり}ひて\\
        \ruby{羊}{ひつじ}の\ruby{夢}{ゆめ}ぞ\ruby{長閑}{のどか}なる
        
    \end{minipage}
    \begin{minipage}[c]{\blocksize}
        
        \vspace{\linespace}
        \item~\\
        % 2.
        さあれ\ruby{平和}{へいわ}の\ruby{夢}{ゆめ}の\ruby{夢}{ゆめ}\\
        \ruby{見}{み}よ\ruby{西欧}{せいおう}の\ruby{空}{そら}の\ruby{様}{よう}\\
        \ruby{怪}{かい}\ruby{雲}{くも}\ruby{荒}{すさ}び\ruby{暴風}{ぼうふう}\ruby{吠}{ほ}え\\
        シベリヤの\ruby{春}{はる}の\ruby{色}{いろ}もなく\\
        \ruby{狂風}{きょうふう}\ruby{千里}{せんり}\ruby{胡}{えびす}\ruby{砂}{すな}を\ruby{捲}{ま}き\\
        \ruby{日本海}{にほんかい}に\ruby{波}{なみ}\ruby{高}{たか}し
        
    \end{minipage}
    \begin{minipage}[c]{\blocksize}
        
        \vspace{\linespace}
        \item~\\
        % 3.
        \ruby{今}{いま}ぞ\ruby{皇国}{こうこく}\ruby{多事}{たじ}の\ruby{時}{とき}\\
        \ruby{北}{きた}の\ruby{守}{まもり}の\ruby{北}{きた}\ruby{州}{しゅう}に\\
        \ruby{護国}{ごこく}の\ruby{子}{こ}\ruby{等}{とう}が\ruby{学}{まな}び\ruby{舎}{や}の\\
        \ruby{弥}{わたる}や\ruby{栄}{さか}えゆく\ruby{喜}{き}を\\
        \ruby{心}{こころ}に\ruby{永}{なが}くしるさんと\\
        \ruby{歌}{うた}ごゑ\ruby{高}{たか}き\ruby{春}{はる}\ruby{今宵}{こよい}
        
    \end{minipage}
    \begin{minipage}[c]{\blocksize}
        
        \vspace{\linespace}
        \item~\\
        <「\ruby{藻}{も}\ruby{岩}{いわ}の\ruby{緑}{みどり}」の\ruby{譜}{ふ}による>
    
    \end{minipage}
\end{enumerate} % 番号の箇条書き ここまで
%%%%% 歌詞 ここまで %%%%%
% end body

\end{document}

\documentclass[10pt,b5j]{tarticle} % B6 縦書き
% \documentclass[10pt,b5j]{tarticle} % B6 縦書き
\AtBeginDvi{\special{papersize=128mm,182mm}} % B6 用用紙サイズ
\usepackage{otf} % Unicode で字を入力するのに必要なパッケージ
\usepackage[size=b6j]{bxpapersize} % B6 用紙サイズを指定
\usepackage[dvipdfmx]{graphicx} % 画像を挿入するためのパッケージ
\usepackage[dvipdfmx]{color} % 色をつけるためのパッケージ
\usepackage{pxrubrica} % ルビを振るためのパッケージ
\usepackage{multicol} % 複数段組を作るためのパッケージ
\setlength{\topmargin}{14mm} % 上下方向のマージン
\addtolength{\topmargin}{-1in} % 
\setlength{\oddsidemargin}{11mm} % 左右方向のマージン
\addtolength{\oddsidemargin}{-1in} % 
\setlength{\textwidth}{154mm} % B6 用
\setlength{\textheight}{108mm} % B6 用
\setlength{\headsep}{0mm} % 
\setlength{\headheight}{0mm} % 
\setlength{\topskip}{0mm} % 
\setlength{\parskip}{0pt} % 
\def\labelenumi{\theenumi、} % 箇条書きのフォーマット
\parindent = 0pt % 段落下げしない

 % B6 用テンプレート読み込み

\begin{document}
% begin header
%%%%% タイトルと作者 ここから %%%%%
\begin{minipage}[c]{0.7\hsize} % タイトルは上から 7 割
    \begin{center}
    % begin title
        {\LARGE
            北を恋う % タイトルを入れる
        }
        {\small 
            (北大創基百周年記念東京同窓会寄贈寮歌) % 年などを入れる
        }
    % end title
    \end{center}
\end{minipage}
\begin{minipage}[c]{0.3\hsize} % 作歌作曲は上から 3 割
    \begin{flushright} % 下寄せにする
        % begin name
        宍戸昌夫君 作歌・作曲 % 作歌・作曲者
        % end name
    \end{flushright}
\end{minipage}
%%%%% タイトルと作者 ここまで %%%%%
% % end header

% begin length
\vspace{1.5em} % タイトル, 作者と歌詞の間に隙間を設ける
\newcommand{\linespace}{0.5em} % 行間の設定
\newcommand{\blocksize}{0.5\hsize} % 段組間の設定
\newcommand{\itemmargin}{3em} % 曲番の位置調整の長さ
% end length
% begin body
%%%%% 歌詞 ここから %%%%%
\begin{enumerate} % 番号の箇条書き ここから
    \setlength{\itemindent}{\itemmargin} % 曲番の位置調整
    \begin{minipage}[c]{\blocksize}
    
        \vspace{\linespace}
        \item~\\
        % 1.
        \ruby{北}{きた}を\ruby{恋}{こ}う\ruby{憶}{}いはせちに\\
        \ruby{夢}{ゆめ}にのみ\ruby{訪}{おとな}うぞかなしき\\
        \ruby{大}{おお}いなる\ruby{志}{こころざし}\ruby{立}{だ}てにし\\
        \ruby{魂}{たましい}の\ruby{故郷}{こきょう}と\ruby{呼}{よ}ぶ\ruby{地}{ち}よ
        
    \end{minipage}
    \begin{minipage}[c]{\blocksize}
        
        \vspace{\linespace}
        \item~\\
        % 2.
        あはれ\ruby{彼}{かれ}のエルムの\ruby{樹}{き}\ruby{蔭}{かげ}\\
        \ruby{朝}{あさ}まだき\ruby{郭公}{かっこう}\ruby{聞}{き}きし\\
        \ruby{若}{わか}き\ruby{日}{ひ}の\ruby{情熱}{じょうねつ}と\ruby{愁思}{しゅうし}\\
        \ruby{胸}{むね}いたく\ruby{思}{おも}い\ruby{出}{で}ずなり
        
    \end{minipage}
    \begin{minipage}[c]{\blocksize}
        
        \vspace{\linespace}
        \item~\\
        % 3.
        \ruby{憧}{あこが}れし\ruby{理想}{りそう}の\ruby{光}{ひかり}\\
        \ruby{謳歌}{おうか}いしは\ruby{自由}{じゆう}の\ruby{讃歌}{さんか}\\
        \ruby{恵}{めぐみ}わん\ruby{迪}{すすむ}を\ruby{索}{さく}めて\\
        \ruby{彷徨}{ほうこう}いき\ruby{三}{さん}\ruby{年}{ねん}の\ruby{青春}{せいしゅん}を
        
    \end{minipage}
    \begin{minipage}[c]{\blocksize}
        
        \vspace{\linespace}
        \item~\\
        % 4.
        \ruby{星}{ほし}\ruby{斗}{と}\ruby{仰}{あお}ぎ\ruby{語}{かた}らいし\ruby{夜}{よる}\\
        \ruby{友情}{ゆうじょう}もて\ruby{睦}{むつ}みし\ruby{団欒}{だんらん}\\
        \ruby{濁}{にご}り\ruby{世}{よ}を\ruby{止}{と}め\ruby{揚}{あ}げつつ\\
        \ruby{一途}{いっと}に\ruby{真理}{しんり}\ruby{探}{さぐ}りき
        
    \end{minipage}
    \begin{minipage}[c]{\blocksize}
        
        \vspace{\linespace}
        \item~\\
        % 5.
        \ruby{学}{まな}び\ruby{舎}{や}の\ruby{青史}{せいし}\ruby{薫}{かお}りて\\
        \ruby{今}{いま}ここに\ruby{百}{ひゃく}\ruby{年}{ねん}\ruby{迎}{むかい}う\\
        \ruby{祝}{いわ}うべしこの\ruby{光栄}{こうえい}の\ruby{日}{ひ}を\\
        \ruby{嗣}{つ}ぎ\ruby{行}{い}かん\ruby{先人}{せんじん}の\ruby{偉業}{いぎょう}
        
    \end{minipage}
    \begin{minipage}[c]{\blocksize}
        
        \vspace{\linespace}
        \item~\\
        % 6.
        \ruby{嗚呼}{ああ}われら\ruby{恵迪}{けいてき}の\ruby{子}{こ}ら\\
        \ruby{起}{た}たんかなこの\ruby{秋}{あき}にして\\
        \ruby{比類}{ひるい}なき\ruby{教訓}{きょうくん}\ruby{守}{まも}りて\\
        \ruby{拓}{ひら}かなん\ruby{来}{き}たるべき\ruby{世代}{せだい}を
    
    \end{minipage}
\end{enumerate} % 番号の箇条書き ここまで
%%%%% 歌詞 ここまで %%%%%
% end body

\end{document}

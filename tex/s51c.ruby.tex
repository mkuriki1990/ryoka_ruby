\documentclass[10pt,b5j]{tarticle} % B6 縦書き
% \documentclass[10pt,b5j]{tarticle} % B6 縦書き
\AtBeginDvi{\special{papersize=128mm,182mm}} % B6 用用紙サイズ
\usepackage{otf} % Unicode で字を入力するのに必要なパッケージ
\usepackage[size=b6j]{bxpapersize} % B6 用紙サイズを指定
\usepackage[dvipdfmx]{graphicx} % 画像を挿入するためのパッケージ
\usepackage[dvipdfmx]{color} % 色をつけるためのパッケージ
\usepackage{pxrubrica} % ルビを振るためのパッケージ
\usepackage{multicol} % 複数段組を作るためのパッケージ
\setlength{\topmargin}{14mm} % 上下方向のマージン
\addtolength{\topmargin}{-1in} % 
\setlength{\oddsidemargin}{11mm} % 左右方向のマージン
\addtolength{\oddsidemargin}{-1in} % 
\setlength{\textwidth}{154mm} % B6 用
\setlength{\textheight}{108mm} % B6 用
\setlength{\headsep}{0mm} % 
\setlength{\headheight}{0mm} % 
\setlength{\topskip}{0mm} % 
\setlength{\parskip}{0pt} % 
\def\labelenumi{\theenumi、} % 箇条書きのフォーマット
\parindent = 0pt % 段落下げしない

 % B6 用テンプレート読み込み

\begin{document}
% begin header
%%%%% タイトルと作者 ここから %%%%%
\begin{minipage}[c]{0.7\hsize} % タイトルは上から 7 割
    \begin{center}
    % begin title
        {\LARGE
            北を恋う % タイトルを入れる
        }
        {\small 
            (北大創基百周年記念東京同窓会寄贈寮歌) % 年などを入れる
        }
    % end title
    \end{center}
\end{minipage}
\begin{minipage}[c]{0.3\hsize} % 作歌作曲は上から 3 割
    \begin{flushright} % 下寄せにする
        % begin name
        宍戸昌夫君 作歌・作曲 % 作歌・作曲者
        % end name
    \end{flushright}
\end{minipage}
%%%%% タイトルと作者 ここまで %%%%%
% % end header

% begin length
\vspace{1.5em} % タイトル, 作者と歌詞の間に隙間を設ける
\newcommand{\linespace}{0.5em} % 行間の設定
\newcommand{\blocksize}{0.5\hsize} % 段組間の設定
\newcommand{\itemmargin}{3em} % 曲番の位置調整の長さ
% end length
% begin body
%%%%% 歌詞 ここから %%%%%
\begin{enumerate} % 番号の箇条書き ここから
    \setlength{\itemindent}{\itemmargin} % 曲番の位置調整
    \begin{minipage}[c]{\blocksize}
    
        \vspace{\linespace}
        \item~\\
        % 1.
        \ruby{北}{}を\ruby{恋}{}う\ruby{憶}{}いはせちに\\
        \ruby{夢}{}にのみ\ruby{訪}{}うぞかなしき\\
        \ruby{大}{}いなる\ruby{志立}{}てにし\\
        \ruby{魂}{}の\ruby{故郷}{}と\ruby{呼}{}ぶ\ruby{地}{}よ
        
    \end{minipage}
    \begin{minipage}[c]{\blocksize}
        
        \vspace{\linespace}
        \item~\\
        % 2.
        あはれ\ruby{彼}{}のエルムの\ruby{樹蔭}{}\\
        \ruby{朝}{}まだき\ruby{郭公聞}{}きし\\
        \ruby{若}{}き\ruby{日}{}の\ruby{情熱}{}と\ruby{愁思}{}\\
        \ruby{胸}{}いたく\ruby{思}{}い\ruby{出}{}ずなり
        
    \end{minipage}
    \begin{minipage}[c]{\blocksize}
        
        \vspace{\linespace}
        \item~\\
        % 3.
        \ruby{憧}{}れし\ruby{理想}{}の\ruby{光}{}\\
        \ruby{謳歌}{}いしは\ruby{自由}{}の\ruby{讃歌}{}\\
        \ruby{恵}{}わん\ruby{迪}{}を\ruby{索}{}めて\\
        \ruby{彷徨}{}いき\ruby{三年}{}の\ruby{青春}{}を
        
    \end{minipage}
    \begin{minipage}[c]{\blocksize}
        
        \vspace{\linespace}
        \item~\\
        % 4.
        \ruby{星斗仰}{}ぎ\ruby{語}{}らいし\ruby{夜}{}\\
        \ruby{友情}{}もて\ruby{睦}{}みし\ruby{団欒}{}\\
        \ruby{濁}{}り\ruby{世}{}を\ruby{止}{}め\ruby{揚}{}げつつ\\
        \ruby{一途}{}に\ruby{真理探}{}りき
        
    \end{minipage}
    \begin{minipage}[c]{\blocksize}
        
        \vspace{\linespace}
        \item~\\
        % 5.
        \ruby{学}{}び\ruby{舎}{}の\ruby{青史薫}{}りて\\
        \ruby{今}{}ここに\ruby{百年迎}{}う\\
        \ruby{祝}{}うべしこの\ruby{光栄}{}の\ruby{日}{}を\\
        \ruby{嗣}{}ぎ\ruby{行}{}かん\ruby{先人}{}の\ruby{偉業}{}
        
    \end{minipage}
    \begin{minipage}[c]{\blocksize}
        
        \vspace{\linespace}
        \item~\\
        % 6.
        \ruby{嗚呼}{}われら\ruby{恵迪}{}の\ruby{子}{}ら\\
        \ruby{起}{}たんかなこの\ruby{秋}{}にして\\
        \ruby{比類}{}なき\ruby{教訓守}{}りて\\
        \ruby{拓}{}かなん\ruby{来}{}たるべき\ruby{世代}{}を
    
    \end{minipage}
\end{enumerate} % 番号の箇条書き ここまで
%%%%% 歌詞 ここまで %%%%%
% end body

\end{document}

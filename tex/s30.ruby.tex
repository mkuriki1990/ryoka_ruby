\documentclass[10pt,b5j]{tarticle} % B6 縦書き
% \documentclass[10pt,b5j]{tarticle} % B6 縦書き
\AtBeginDvi{\special{papersize=128mm,182mm}} % B6 用用紙サイズ
\usepackage{otf} % Unicode で字を入力するのに必要なパッケージ
\usepackage[size=b6j]{bxpapersize} % B6 用紙サイズを指定
\usepackage[dvipdfmx]{graphicx} % 画像を挿入するためのパッケージ
\usepackage[dvipdfmx]{color} % 色をつけるためのパッケージ
\usepackage{pxrubrica} % ルビを振るためのパッケージ
\usepackage{multicol} % 複数段組を作るためのパッケージ
\setlength{\topmargin}{14mm} % 上下方向のマージン
\addtolength{\topmargin}{-1in} % 
\setlength{\oddsidemargin}{11mm} % 左右方向のマージン
\addtolength{\oddsidemargin}{-1in} % 
\setlength{\textwidth}{154mm} % B6 用
\setlength{\textheight}{108mm} % B6 用
\setlength{\headsep}{0mm} % 
\setlength{\headheight}{0mm} % 
\setlength{\topskip}{0mm} % 
\setlength{\parskip}{0pt} % 
\def\labelenumi{\theenumi、} % 箇条書きのフォーマット
\parindent = 0pt % 段落下げしない

 % B6 用テンプレート読み込み

\begin{document}
% begin header
%%%%% タイトルと作者 ここから %%%%%
\begin{minipage}[c]{0.7\hsize} % タイトルは上から 7 割
    \begin{center}
    % begin title
        {\LARGE
            悲歌に血吐きし % タイトルを入れる
        }
        {\small 
            (昭和三十年寮歌) % 年などを入れる
        }
    % end title
    \end{center}
\end{minipage}
\begin{minipage}[c]{0.3\hsize} % 作歌作曲は上から 3 割
    \begin{flushright} % 下寄せにする
        % begin name
        柳田和朗君 作歌\\菅原幸雄君 作曲 % 作歌・作曲者
        % end name
    \end{flushright}
\end{minipage}
%%%%% タイトルと作者 ここまで %%%%%
% (序,結 了あり)
% end header

% begin length
\vspace{1.5em} % タイトル, 作者と歌詞の間に隙間を設ける
\newcommand{\linespace}{0.5em} % 行間の設定
\newcommand{\blocksize}{0.5\hsize} % 段組間の設定
\newcommand{\itemmargin}{3em} % 曲番の位置調整の長さ
% end length
% begin body
%%%%% 歌詞 ここから %%%%%
\begin{enumerate} % 番号の箇条書き ここから
    \setlength{\itemindent}{\itemmargin} % 曲番の位置調整
    \begin{minipage}[c]{\blocksize}
    
        \vspace{\linespace}
        \item~\\
        % \ruby{序}{}.
        \ruby{悲歌}{}に\ruby{血吐}{}きし\ruby{我}{}らもが\\
        \ruby{永劫不変}{}を\ruby{探求}{}めんと\\
        \ruby{遙々漂泊来}{}たりても\\
        \ruby{赤}{}き\ruby{浜茄子摘}{}みとりて\\
        \ruby{悪魔牛耳}{}り\ruby{詩吟}{}する\\
        \ruby{天下不仰}{}の\ruby{寂寥児}{}
        
    \end{minipage}
    \begin{minipage}[c]{\blocksize}
        
        \vspace{\linespace}
        \item~\\
        % \ruby{春}{}.
        \ruby{未知}{}の\ruby{世界}{}に\ruby{立}{}ち\ruby{薫}{}る\\
        \ruby{冬}{}の\ruby{名残}{}りか\ruby{歓喜}{}か\\
        \ruby{春爛漫}{}のただなかに\\
        \ruby{手稲}{}の\ruby{山}{}の\ruby{淡雪}{}の\\
        \ruby{雪解}{}が\ruby{衣}{}の\ruby{袖軽}{}ろき\\
        \ruby{門出}{}が\ruby{詩歌}{}を\ruby{讃歌}{}わんや
        
    \end{minipage}
    \begin{minipage}[c]{\blocksize}
        
        \vspace{\linespace}
        \item~\\
        % \ruby{夏}{}.
        \ruby{原始}{}の\ruby{森}{}に\ruby{深}{}く\ruby{入}{}り\\
        \ruby{朱碧混}{}じる\ruby{眩}{}さに\\
        \ruby{神秘無象}{}の\ruby{影}{}さして\\
        \ruby{郭公生命}{}の\ruby{顫律}{}で\\
        \ruby{若}{}き\ruby{誇}{}りに\ruby{酔}{}い\ruby{痴}{}れて\\
        \ruby{自由}{}の\ruby{頌歌歌}{}うなり
        
    \end{minipage}
    \begin{minipage}[c]{\blocksize}
        
        \vspace{\linespace}
        \item~\\
        % \ruby{秋}{}.
        \ruby{朝}{}の\ruby{白露}{}は\ruby{詩}{}を\ruby{吟}{}じ\\
        \ruby{夕陽紅}{}く\ruby{舞}{}い\ruby{乱}{}る\\
        \ruby{秋風高歌昂然}{}と\\
        \ruby{踏轟}{}ろかすストームの\\
        \ruby{孤袖}{}の\ruby{遊子大望}{}の\\
        \ruby{希望}{}に\ruby{宿}{}る\ruby{北極星}{}
        
    \end{minipage}
    \begin{minipage}[c]{\blocksize}
        
        \vspace{\linespace}
        \item~\\
        % \ruby{冬}{}.
        \ruby{雪崩}{}に\ruby{雪}{}を\ruby{血}{}で\ruby{染}{}めて\\
        \ruby{若}{}き\ruby{生命}{}を\ruby{捨}{}つるとも\\
        あこがれ\ruby{清浄}{}き\ruby{樹氷恋}{}い\\
        \ruby{奥山古}{}き\ruby{谷間小屋}{}\\
        \ruby{空想}{}の\ruby{羽}{}の\ruby{頂上}{}に\\
        \ruby{炉火囲}{}み\ruby{唱}{}う\ruby{歌}{}
        
    \end{minipage}
    \begin{minipage}[c]{\blocksize}
        
        \vspace{\linespace}
        \item~\\
        % \ruby{結}{}.
        \ruby{年古}{}る\ruby{樹々}{}は\ruby{皆朽}{}ちて\\
        \ruby{生}{}の\ruby{心}{}が\ruby{落葉}{}の\\
        \ruby{記憶}{}の\ruby{底}{}に\ruby{沈}{}みいで\\
        \ruby{悲哀}{}の\ruby{涙}{}ほとばしる\\
        \ruby{世}{}の\ruby{暗闇}{}にひそめども\\
        \ruby{去}{}る\ruby{二年}{}を\ruby{謳歌}{}えんや
    
    \end{minipage}
\end{enumerate} % 番号の箇条書き ここまで
%%%%% 歌詞 ここまで %%%%%
% end body

\end{document}

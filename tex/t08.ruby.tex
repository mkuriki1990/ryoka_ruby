\documentclass[10pt,b5j]{tarticle} % B6 縦書き
% \documentclass[10pt,b5j]{tarticle} % B6 縦書き
\AtBeginDvi{\special{papersize=128mm,182mm}} % B6 用用紙サイズ
\usepackage{otf} % Unicode で字を入力するのに必要なパッケージ
\usepackage[size=b6j]{bxpapersize} % B6 用紙サイズを指定
\usepackage[dvipdfmx]{graphicx} % 画像を挿入するためのパッケージ
\usepackage[dvipdfmx]{color} % 色をつけるためのパッケージ
\usepackage{pxrubrica} % ルビを振るためのパッケージ
\usepackage{multicol} % 複数段組を作るためのパッケージ
\setlength{\topmargin}{14mm} % 上下方向のマージン
\addtolength{\topmargin}{-1in} % 
\setlength{\oddsidemargin}{11mm} % 左右方向のマージン
\addtolength{\oddsidemargin}{-1in} % 
\setlength{\textwidth}{154mm} % B6 用
\setlength{\textheight}{108mm} % B6 用
\setlength{\headsep}{0mm} % 
\setlength{\headheight}{0mm} % 
\setlength{\topskip}{0mm} % 
\setlength{\parskip}{0pt} % 
\def\labelenumi{\theenumi、} % 箇条書きのフォーマット
\parindent = 0pt % 段落下げしない

 % B6 用テンプレート読み込み

\begin{document}
% begin header
%%%%% タイトルと作者 ここから %%%%%
\begin{minipage}[c]{0.7\hsize} % タイトルは上から 7 割
    \begin{center}
    % begin title
        {\LARGE
            暗雲低く % タイトルを入れる
        }
        {\small 
            (大正八年寮歌) % 年などを入れる
        }
    % end title
    \end{center}
\end{minipage}
\begin{minipage}[c]{0.3\hsize} % 作歌作曲は上から 3 割
    \begin{flushright} % 下寄せにする
        % begin name
        熊谷巌君 作歌\\置塩奇君 作曲 % 作歌・作曲者
        % end name
    \end{flushright}
\end{minipage}
%%%%% タイトルと作者 ここまで %%%%%
% (1,6 了あり)
% end header

% begin body
\vspace{1.5em} % タイトル, 作者と歌詞の間に隙間を設ける
\newcommand{\linespace}{0.5em} % 行間の設定
\newcommand{\blocksize}{0.5\hsize} % 段組間の設定
%%%%% 歌詞 ここから %%%%%
% begin lilycs
\begin{enumerate} % 番号の箇条書き ここから
    \begin{minipage}[c]{\blocksize}
    
        \vspace{\linespace}
        \item
        % 1.
        \ruby{暗雲低}{}く\ruby{乱}{}れてし\\
        \ruby{怨磋}{}の\ruby{声}{}も\ruby{収}{}まるや\\
        \ruby{逆巻}{}く\ruby{波}{}も\ruby{和}{}み\ruby{来}{}て\\
        \ruby{星影淡}{}き\ruby{東雲}{}に\\
        \ruby{平和}{}の\ruby{光朗々}{}と\\
        \ruby{碧緑}{}の\ruby{海}{}に\ruby{輝}{}きぬ
        
        \vspace{\linespace}
        \item
        % 2.
        さあれ\ruby{意}{}へば\ruby{泰平}{}が\\
        やがて\ruby{醸}{}さん\ruby{痴惰}{}の\ruby{夢}{}\\
        \ruby{人}{}は\ruby{安佚}{}を\ruby{偸}{}むとも\\
        \ruby{我}{}には\ruby{固}{}き\ruby{自覚}{}あり\\
        \ruby{人}{}は\ruby{驕奢}{}に\ruby{酔}{}ひしるも\\
        \ruby{我}{}には\ruby{尚武}{}の\ruby{気魄}{}あり
        
        \vspace{\linespace}
        \item
        % 3.
        \ruby{夢深}{}かりし\ruby{曙}{}の\\
        \ruby{霞}{}にまがふ\ruby{蝦夷}{}が\ruby{野}{}に\\
        \ruby{礎固}{}く\ruby{営}{}みて\\
        \ruby{巍峨}{}とそそれる\ruby{自由}{}の\ruby{城}{}\\
        \ruby{浮世}{}の\ruby{塵}{}を\ruby{低}{}く\ruby{睥}{}て\\
        \ruby{健児}{}の\ruby{意気}{}を\ruby{養}{}はん
        
        \vspace{\linespace}
        \item
        % 4.
        \ruby{孤城}{}に\ruby{張}{}るの\ruby{訪}{}れて\\
        \ruby{楡樹}{}の\ruby{匂}{}まだしくも\\
        \ruby{北斗}{}の\ruby{光燦}{}として\\
        \ruby{崇}{}き\ruby{黙示}{}を\ruby{与}{}ふらん\\
        \ruby{雪}{}の\ruby{色}{}にもたぐふべき\\
        \ruby{潔}{}き\ruby{節操}{}を\ruby{思}{}はずや
        
        \vspace{\linespace}
        \item
        % 5.
        \ruby{永遠}{}に\ruby{変}{}らぬ\ruby{希望}{}もて\\
        \ruby{理想}{}の\ruby{華}{}を\ruby{咲}{}かせんと\\
        \ruby{険}{}しき\ruby{世路}{}に\ruby{逆}{}ひつつ\\
        \ruby{歩}{}み\ruby{運}{}びし\ruby{先進}{}が\\
        \ruby{光栄}{}の\ruby{歴史}{}を\ruby{偲}{}ぶれば\\
        \ruby{思出多}{}き\ruby{十四年}{}
        
        \vspace{\linespace}
        \item
        % 6.
        いざや\ruby{勝利}{}の\ruby{盃}{}を\\
        \ruby{平和}{}の\ruby{女神}{}に\ruby{捧}{}げつつ\\
        \ruby{右手}{}に\ruby{正義}{}の\ruby{剣}{}を\ruby{執}{}り\\
        \ruby{左手}{}に\ruby{自由}{}の\ruby{楯}{}を\ruby{持}{}し\\
        \ruby{若}{}き\ruby{血潮}{}の\ruby{鳴}{}るがまま\\
        \ruby{祝}{}ひ\ruby{謳}{}わん\ruby{記念祭}{}
    
    \end{minipage}
\end{enumerate} % 番号の箇条書き ここまで
% end lilycs
%%%%% 歌詞 ここまで %%%%%
% end body

\end{document}

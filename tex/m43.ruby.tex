\documentclass[10pt,b5j]{tarticle} % B6 縦書き
% \documentclass[10pt,b5j]{tarticle} % B6 縦書き
\AtBeginDvi{\special{papersize=128mm,182mm}} % B6 用用紙サイズ
\usepackage{otf} % Unicode で字を入力するのに必要なパッケージ
\usepackage[size=b6j]{bxpapersize} % B6 用紙サイズを指定
\usepackage[dvipdfmx]{graphicx} % 画像を挿入するためのパッケージ
\usepackage[dvipdfmx]{color} % 色をつけるためのパッケージ
\usepackage{pxrubrica} % ルビを振るためのパッケージ
\usepackage{plext} % 漢数字の enumerate を使うためのパッケージ
\usepackage{multicol} % 複数段組を作るためのパッケージ
\setlength{\topmargin}{14mm} % 上下方向のマージン
\addtolength{\topmargin}{-1in} % 
\setlength{\oddsidemargin}{11mm} % 左右方向のマージン
\addtolength{\oddsidemargin}{-1in} % 
\setlength{\textwidth}{154mm} % B6 用
\setlength{\textheight}{108mm} % B6 用
\setlength{\headsep}{0mm} % 
\setlength{\headheight}{0mm} % 
\setlength{\topskip}{0mm} % 
\setlength{\parskip}{0pt} % 
\def\theenumi{\Kanji{enumi}} % 箇条書きのフォーマットを漢数字に変更
\parindent = 0pt % 段落下げしない
\pagestyle{empty} % すべてのページ番号を消去
% \renewcommand{\baselinestretch}{0.9} % 行間の倍率
 % B6 用テンプレート読み込み

\begin{document}
% begin header
%%%%% タイトルと作者 ここから %%%%%
\begin{minipage}[c]{0.7\hsize} % タイトルは上から 7 割
    \begin{center}
    % begin title
        {\LARGE
            帝都を北に % タイトルを入れる
        }
        {\small 
            (明治四十三年寮歌) % 年などを入れる
        }
    % end title
    \end{center}
\end{minipage}
\begin{minipage}[c]{0.3\hsize} % 作歌作曲は上から 3 割
    \begin{flushright} % 下寄せにする
        % begin name
        谷村愛之助君 作歌\\柳沢秀雄君 作曲 % 作歌・作曲者
        % end name
    \end{flushright}
\end{minipage}
%%%%% タイトルと作者 ここまで %%%%%
% (1,4,6 了あり)
% end header

% begin length
\vspace{1.5em} % タイトル, 作者と歌詞の間に隙間を設ける
\newcommand{\linespace}{0.5em} % 行間の設定
\newcommand{\blocksize}{0.5\hsize} % 段組間の設定
\newcommand{\itemmargin}{6em} % 曲番の位置調整の長さ
% end length
% begin body
%%%%% 歌詞 ここから %%%%%
\begin{enumerate} % 番号の箇条書き ここから
    \setlength{\itemindent}{\itemmargin} % 曲番の位置調整
    \begin{minipage}[c]{\blocksize}
    
        \vspace{\linespace}
        \item~\\
        % 1.
        \ruby{帝都}{}を\ruby{北}{}に\ruby{三百里}{}\\
        \ruby{津軽}{}の\ruby{海}{}を\ruby{越}{}え\ruby{来}{}れば\\
        \ruby{紅塵絶}{}えて\ruby{空潔}{}く\\
        \ruby{蕭々}{}として\ruby{水寒}{}し\\
        \ruby{大陸}{}の\ruby{精鍾}{}まりて\\
        \ruby{我北州}{}の\ruby{島}{}と\ruby{凝}{}る
        
        \vspace{\linespace}
        \item~\\
        % 2.
        \ruby{鯨群吼}{}ゆる\ruby{荒潮}{}に\\
        \ruby{落}{}つる\ruby{北斗}{}の\ruby{影冴}{}えて\\
        \ruby{斧鉞入}{}らざるや\ruby{森林}{}や\\
        \ruby{人跡絶}{}えし\ruby{大野原}{}\\
        \ruby{原始}{}の\ruby{儘}{}の\ruby{俤}{}を\\
        \ruby{我北州}{}の\ruby{島}{}に\ruby{見}{}る
        
        \vspace{\linespace}
        \item~\\
        % 3.
        \ruby{鈴蘭薫}{}る\ruby{春}{}の\ruby{野辺}{}\\
        \ruby{楡}{}の\ruby{下蔭草繁}{}る\\
        \ruby{霜葉燃}{}ゆる\ruby{蔦葛}{}\\
        \ruby{吹雪}{}は\ruby{叫}{}ぶ\ruby{冬}{}の\ruby{夜半}{}\\
        \ruby{四季}{}の\ruby{変遷興添}{}えて\\
        \ruby{眺}{}めは\ruby{飽}{}かぬ\ruby{姿}{}かな
        
        \vspace{\linespace}
        \item~\\
        % 4.
        \ruby{朝霧深}{}き\ruby{野}{}の\ruby{面}{}に\\
        \ruby{嘶}{}く\ruby{駒}{}の\ruby{跡追}{}えば\\
        \ruby{露}{}の\ruby{白玉散}{}り\ruby{乱}{}る\\
        \ruby{甘藍}{}の\ruby{畑}{}たそがれて\\
        プラウの\ruby{土}{}を\ruby{払}{}ふ\ruby{時}{}\\
        \ruby{農牧}{}の\ruby{幸謳}{}ふかな
        
        \vspace{\linespace}
        \item~\\
        % 5.
        \ruby{見}{}よ\ruby{文明}{}は\ruby{北進}{}す\\
        \ruby{古嚢}{}は\ruby{盛}{}らず\ruby{新酒}{}を\\
        \ruby{新文明}{}の\ruby{建設}{}は\\
        \ruby{濁}{}れる\ruby{都}{}にあらずして\\
        \ruby{渺}{}たる\ruby{大河}{}の\ruby{片辺}{}\\
        \ruby{地}{}は\ruby{広漠}{}の\ruby{沖積層}{}
        
        \vspace{\linespace}
        \item~\\
        % 6.
        \ruby{此聖都}{}を\ruby{永久}{}に\\
        \ruby{浮華軽佻}{}の\ruby{国}{}とせず\\
        \ruby{真摯素樸}{}の\ruby{郷}{}となし\\
        \ruby{我等}{}が\ruby{使命成}{}し\ruby{遂}{}げん\\
        \ruby{真理}{}の\ruby{秘奥探}{}る\ruby{可}{}く\\
        \ruby{道義}{}の\ruby{光照}{}す\ruby{可}{}く
    
    \end{minipage}
\end{enumerate} % 番号の箇条書き ここまで
%%%%% 歌詞 ここまで %%%%%
% end body

\end{document}

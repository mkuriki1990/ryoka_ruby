\documentclass[10pt,b5j]{tarticle} % B6 縦書き
% \documentclass[10pt,b5j]{tarticle} % B6 縦書き
\AtBeginDvi{\special{papersize=128mm,182mm}} % B6 用用紙サイズ
\usepackage{otf} % Unicode で字を入力するのに必要なパッケージ
\usepackage[size=b6j]{bxpapersize} % B6 用紙サイズを指定
\usepackage[dvipdfmx]{graphicx} % 画像を挿入するためのパッケージ
\usepackage[dvipdfmx]{color} % 色をつけるためのパッケージ
\usepackage{pxrubrica} % ルビを振るためのパッケージ
\usepackage{multicol} % 複数段組を作るためのパッケージ
\setlength{\topmargin}{14mm} % 上下方向のマージン
\addtolength{\topmargin}{-1in} % 
\setlength{\oddsidemargin}{11mm} % 左右方向のマージン
\addtolength{\oddsidemargin}{-1in} % 
\setlength{\textwidth}{154mm} % B6 用
\setlength{\textheight}{108mm} % B6 用
\setlength{\headsep}{0mm} % 
\setlength{\headheight}{0mm} % 
\setlength{\topskip}{0mm} % 
\setlength{\parskip}{0pt} % 
\def\labelenumi{\theenumi、} % 箇条書きのフォーマット
\parindent = 0pt % 段落下げしない

 % B6 用テンプレート読み込み

\begin{document}
% begin header
%%%%% タイトルと作者 ここから %%%%%
\begin{minipage}[c]{0.7\hsize} % タイトルは上から 7 割
    \begin{center}
    % begin title
        {\LARGE
            帝都を北に % タイトルを入れる
        }
        {\small 
            (明治四十三年寮歌) % 年などを入れる
        }
    % end title
    \end{center}
\end{minipage}
\begin{minipage}[c]{0.3\hsize} % 作歌作曲は上から 3 割
    \begin{flushright} % 下寄せにする
        % begin name
        谷村愛之助君 作歌\\柳沢秀雄君 作曲 % 作歌・作曲者
        % end name
    \end{flushright}
\end{minipage}
%%%%% タイトルと作者 ここまで %%%%%
% (1,4,6 了あり)
% end header

% begin length
\vspace{1.5em} % タイトル, 作者と歌詞の間に隙間を設ける
\newcommand{\linespace}{0.5em} % 行間の設定
\newcommand{\blocksize}{0.5\hsize} % 段組間の設定
\newcommand{\itemmargin}{3em} % 曲番の位置調整の長さ
% end length
% begin body
%%%%% 歌詞 ここから %%%%%
\begin{enumerate} % 番号の箇条書き ここから
    \setlength{\itemindent}{\itemmargin} % 曲番の位置調整
    \begin{minipage}[c]{\blocksize}
    
        \vspace{\linespace}
        \item~\\
        % 1.
        \ruby{帝都}{ていと}を\ruby{北}{きた}に\ruby{三}{さん}\ruby{百}{ひゃく}\ruby{里}{さと}\\
        \ruby{津軽}{つがる}の\ruby{海}{うみ}を\ruby{越}{こ}え\ruby{来}{く}れば\\
        \ruby{紅塵}{こうじん}\ruby{絶}{た}えて\ruby{空}{そら}\ruby{潔}{いさぎよ}く\\
        \ruby{蕭々}{しょうしょう}として\ruby{水}{みず}\ruby{寒}{さむ}し\\
        \ruby{大陸}{たいりく}の\ruby{精}{せい}\ruby{鍾}{}まりて\\
        \ruby{我}{われ}\ruby{北州}{ほくしゅう}の\ruby{島}{しま}と\ruby{凝}{こ}る
        
    \end{minipage}
    \begin{minipage}[c]{\blocksize}
        
        \vspace{\linespace}
        \item~\\
        % 2.
        \ruby{鯨}{くじら}\ruby{群}{ぐん}\ruby{吼}{}ゆる\ruby{荒潮}{あらしお}に\\
        \ruby{落}{おち}つる\ruby{北斗}{ほくと}の\ruby{影}{かげ}\ruby{冴}{さ}えて\\
        \ruby{斧鉞}{ふえつ}\ruby{入}{はい}らざるや\ruby{森林}{しんりん}や\\
        \ruby{人}{ひと}\ruby{跡絶}{とだ}えし\ruby{大野原}{おおのはら}\\
        \ruby{原始}{げんし}の\ruby{儘}{まま}の\ruby{俤}{おもかげ}を\\
        \ruby{我}{われ}\ruby{北州}{ほくしゅう}の\ruby{島}{しま}に\ruby{見}{み}る
        
    \end{minipage}
    \begin{minipage}[c]{\blocksize}
        
        \vspace{\linespace}
        \item~\\
        % 3.
        \ruby{鈴蘭}{すずらん}\ruby{薫}{かお}る\ruby{春}{はる}の\ruby{野辺}{のべ}\\
        \ruby{楡}{にれ}の\ruby{下蔭}{しもかげ}\ruby{草}{くさ}\ruby{繁}{しげ}る\\
        \ruby{霜}{しも}\ruby{葉}{は}\ruby{燃}{もゆる}ゆる\ruby{蔦葛}{つたかずら}\\
        \ruby{吹雪}{ふぶき}は\ruby{叫}{さけ}ぶ\ruby{冬}{ふゆ}の\ruby{夜半}{やはん}\\
        \ruby{四季}{しき}の\ruby{変遷}{へんせん}\ruby{興}{きょう}\ruby{添}{そ}えて\\
        \ruby{眺}{なが}めは\ruby{飽}{あ}かぬ\ruby{姿}{すがた}かな
        
    \end{minipage}
    \begin{minipage}[c]{\blocksize}
        
        \vspace{\linespace}
        \item~\\
        % 4.
        \ruby{朝霧}{あさぎり}\ruby{深}{ふかし}き\ruby{野}{きの}の\ruby{面}{めん}に\\
        \ruby{嘶}{いなな}く\ruby{駒}{こま}の\ruby{跡}{あと}\ruby{追}{お}えば\\
        \ruby{露}{ろ}の\ruby{白玉}{しらたま}\ruby{散}{ち}り\ruby{乱}{らん}る\\
        \ruby{甘藍}{かんらん}の\ruby{畑}{はたけ}たそがれて\\
        プラウの\ruby{土}{ど}を\ruby{払}{}ふ\ruby{時}{ふじ}\\
        \ruby{農}{のう}\ruby{牧}{まき}の\ruby{幸}{こう}\ruby{謳}{}ふかな
        
    \end{minipage}
    \begin{minipage}[c]{\blocksize}
        
        \vspace{\linespace}
        \item~\\
        % 5.
        \ruby{見}{み}よ\ruby{文明}{ぶんめい}は\ruby{北進}{ほくしん}す\\
        \ruby{古}{こ}\ruby{嚢}{}は\ruby{盛}{も}らず\ruby{新酒}{しんしゅ}を\\
        \ruby{新}{しん}\ruby{文明}{ぶんめい}の\ruby{建設}{けんせつ}は\\
        \ruby{濁}{にご}れる\ruby{都}{と}にあらずして\\
        \ruby{渺}{びょう}たる\ruby{大河}{たいが}の\ruby{片辺}{かたほとり}\\
        \ruby{地}{ち}は\ruby{広漠}{こうばく}の\ruby{沖積層}{ちゅうせきそう}
        
    \end{minipage}
    \begin{minipage}[c]{\blocksize}
        
        \vspace{\linespace}
        \item~\\
        % 6.
        \ruby{此}{この}\ruby{聖}{きよし}\ruby{都}{と}を\ruby{永久}{えいきゅう}に\\
        \ruby{浮華}{ふか}\ruby{軽佻}{けいちょう}の\ruby{国}{くに}とせず\\
        \ruby{真摯}{しんし}\ruby{素樸}{そぼく}の\ruby{郷}{さと}となし\\
        \ruby{我等}{われら}が\ruby{使命}{しめい}\ruby{成}{な}し\ruby{遂}{}げん\\
        \ruby{真理}{しんり}の\ruby{秘奥}{ひおう}\ruby{探}{さぐ}る\ruby{可}{か}く\\
        \ruby{道義}{どうぎ}の\ruby{光照}{みつてる}す\ruby{可}{か}く
    
    \end{minipage}
\end{enumerate} % 番号の箇条書き ここまで
%%%%% 歌詞 ここまで %%%%%
% end body

\end{document}

\documentclass[10pt,b5j]{tarticle} % B6 縦書き
% \documentclass[10pt,b5j]{tarticle} % B6 縦書き
\AtBeginDvi{\special{papersize=128mm,182mm}} % B6 用用紙サイズ
\usepackage{otf} % Unicode で字を入力するのに必要なパッケージ
\usepackage[size=b6j]{bxpapersize} % B6 用紙サイズを指定
\usepackage[dvipdfmx]{graphicx} % 画像を挿入するためのパッケージ
\usepackage[dvipdfmx]{color} % 色をつけるためのパッケージ
\usepackage{pxrubrica} % ルビを振るためのパッケージ
\usepackage{plext} % 漢数字の enumerate を使うためのパッケージ
\usepackage{multicol} % 複数段組を作るためのパッケージ
\setlength{\topmargin}{14mm} % 上下方向のマージン
\addtolength{\topmargin}{-1in} % 
\setlength{\oddsidemargin}{11mm} % 左右方向のマージン
\addtolength{\oddsidemargin}{-1in} % 
\setlength{\textwidth}{154mm} % B6 用
\setlength{\textheight}{108mm} % B6 用
\setlength{\headsep}{0mm} % 
\setlength{\headheight}{0mm} % 
\setlength{\topskip}{0mm} % 
\setlength{\parskip}{0pt} % 
\def\theenumi{\Kanji{enumi}} % 箇条書きのフォーマットを漢数字に変更
\parindent = 0pt % 段落下げしない
\pagestyle{empty} % すべてのページ番号を消去
% \renewcommand{\baselinestretch}{0.9} % 行間の倍率
 % B6 用テンプレート読み込み

\begin{document}
% begin header
%%%%% タイトルと作者 ここから %%%%%
\begin{minipage}[c]{0.7\hsize} % タイトルは上から 7 割
    \begin{center}
    % begin title
        {\LARGE
            昭和12年応援歌 % タイトルを入れる
        }
        {\small 
             % 年などを入れる
        }
    % end title
    \end{center}
\end{minipage}
\begin{minipage}[c]{0.3\hsize} % 作歌作曲は上から 3 割
    \begin{flushright} % 下寄せにする
        % begin name
        河邨文一郎君 作歌\\桜井允君 作曲 % 作歌・作曲者
        % end name
    \end{flushright}
\end{minipage}
%%%%% タイトルと作者 ここまで %%%%%
% % end header

% begin body
\vspace{1.5em} % タイトル, 作者と歌詞の間に隙間を設ける
\newcommand{\linespace}{0.5em} % 行間の設定
\newcommand{\blocksize}{0.5\hsize} % 段組間の設定
%%%%% 歌詞 ここから %%%%%
% begin lilycs
\begin{enumerate} % 番号の箇条書き ここから
    \begin{minipage}[c]{\blocksize}
    
        \vspace{\linespace}
        \item
        % 1.
        \ruby{大荒吼}{}ゆる\\
        \ruby{北方}{}の\\
        \ruby{雄覇}{}の\ruby{高眠}{}\\
        \ruby{永}{}からず\\
        \ruby{陣雲低}{}く\\
        \ruby{雄叫}{}びに\\
        \ruby{天柱摧}{}け\\
        \ruby{地維}{}は\ruby{裂}{}け\\
        \ruby{貔貅}{}の\ruby{若血}{}\\
        \ruby{高鳴}{}りて\\
        \ruby{旌旗}{}\\
        \ruby{南}{}を\ruby{指}{}すや\ruby{今}{}
        
        \vspace{\linespace}
        \item
        % 2.
        \ruby{荘図辺都}{}に\\
        \ruby{止}{}むべきや\\
        \ruby{萬里遥}{}かに\\
        \ruby{空翔}{}り\\
        \ruby{雷鼓轟}{}く\\
        \ruby{蒼溟}{}を\\
        \ruby{越}{}えて\ruby{翔飛}{}す\\
        \ruby{北}{}の\ruby{鷲}{}\\
        \ruby{風塵昏}{}く\\
        \ruby{雄剣}{}の\\
        \ruby{中原略}{}す\\
        \ruby{時}{}や\ruby{今}{}
        
        \vspace{\linespace}
        \item
        % 3.
        \ruby{意気幕天}{}の\\
        \ruby{強弩軍}{}\\
        \ruby{殺気地}{}を\ruby{席}{}く\\
        \ruby{晴戦}{}\\
        \ruby{男児}{}の\ruby{覇業}{}\\
        \ruby{遂}{}げざらば\\
        \ruby{玉}{}と\ruby{砕}{}けん\\
        \ruby{名}{}に\ruby{死}{}せん\\
        \ruby{素破}{}\\
        \ruby{乾坤}{}を\ruby{擲}{}ちて\\
        \ruby{血潮}{}に\ruby{叫}{}ぶ\\
        \ruby{時}{}や\ruby{今}{}\\
        \ruby{血潮}{}に\ruby{謳}{}う\\
        \ruby{時}{}や\ruby{今}{}
    
    \end{minipage}
\end{enumerate} % 番号の箇条書き ここまで
% end lilycs
%%%%% 歌詞 ここまで %%%%%
% end body

\end{document}

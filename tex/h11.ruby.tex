\documentclass[10pt,b5j]{tarticle} % B6 縦書き
% \documentclass[10pt,b5j]{tarticle} % B6 縦書き
\AtBeginDvi{\special{papersize=128mm,182mm}} % B6 用用紙サイズ
\usepackage{otf} % Unicode で字を入力するのに必要なパッケージ
\usepackage[size=b6j]{bxpapersize} % B6 用紙サイズを指定
\usepackage[dvipdfmx]{graphicx} % 画像を挿入するためのパッケージ
\usepackage[dvipdfmx]{color} % 色をつけるためのパッケージ
\usepackage{pxrubrica} % ルビを振るためのパッケージ
\usepackage{plext} % 漢数字の enumerate を使うためのパッケージ
\usepackage{multicol} % 複数段組を作るためのパッケージ
\setlength{\topmargin}{14mm} % 上下方向のマージン
\addtolength{\topmargin}{-1in} % 
\setlength{\oddsidemargin}{11mm} % 左右方向のマージン
\addtolength{\oddsidemargin}{-1in} % 
\setlength{\textwidth}{154mm} % B6 用
\setlength{\textheight}{108mm} % B6 用
\setlength{\headsep}{0mm} % 
\setlength{\headheight}{0mm} % 
\setlength{\topskip}{0mm} % 
\setlength{\parskip}{0pt} % 
\def\theenumi{\Kanji{enumi}} % 箇条書きのフォーマットを漢数字に変更
\parindent = 0pt % 段落下げしない
\pagestyle{empty} % すべてのページ番号を消去
% \renewcommand{\baselinestretch}{0.9} % 行間の倍率
 % B6 用テンプレート読み込み

\begin{document}
% begin header
%%%%% タイトルと作者 ここから %%%%%
\begin{minipage}[c]{0.7\hsize} % タイトルは上から 7 割
    \begin{center}
    % begin title
        {\LARGE
            精華の誓 % タイトルを入れる
        }
        {\small 
            (平成11年度寮歌) % 年などを入れる
        }
    % end title
    \end{center}
\end{minipage}
\begin{minipage}[c]{0.3\hsize} % 作歌作曲は上から 3 割
    \begin{flushright} % 下寄せにする
        % begin name
        荒木洋祐君 作歌\\小出隆広君 作曲 % 作歌・作曲者
        % end name
    \end{flushright}
\end{minipage}
%%%%% タイトルと作者 ここまで %%%%%
% (1,2 了あり)
% end header

% begin body
\vspace{1.5em} % タイトル, 作者と歌詞の間に隙間を設ける
\newcommand{\linespace}{0.5em} % 行間の設定
\newcommand{\blocksize}{0.5\hsize} % 段組間の設定
%%%%% 歌詞 ここから %%%%%
% begin lilycs
\begin{enumerate} % 番号の箇条書き ここから
    \begin{minipage}[c]{\blocksize}
    
        \vspace{\linespace}
        \item
        % 1.
        \ruby{雪舞}{}う\ruby{地平}{}にひときわ\ruby{映}{}える\\
        \ruby{六華}{}の\ruby{紋}{}ぞ\ruby{我}{}らが\ruby{砦}{}\\
        \ruby{野心}{}は\ruby{満}{}ちて\ruby{冬空焦}{}がし\\
        \ruby{樹間}{}の\ruby{風}{}は\ruby{情熱}{}を\ruby{運}{}ぶ\\
        \ruby{杯}{}に\ruby{写}{}る\ruby{未来}{}をみよう\\
        \ruby{夜明}{}かし\ruby{語}{}るこの\ruby{今}{}にこそ\\
        カペラの\ruby{叡智}{}オリオンの\ruby{武勇}{}\\
        \ruby{天}{}よ\ruby{闇}{}よ\ruby{我}{}らに\ruby{賜}{}え
        
        \vspace{\linespace}
        \item
        % 2.
        \ruby{国}{}を\ruby{覆}{}い\ruby{地球}{}をゆるがす\\
        \ruby{四百志士}{}の\ruby{夢}{}よ\ruby{醒}{}めよ\\
        \ruby{太平洋}{}にかかる\\
        \ruby{橋}{}にぞなれる\\
        \ruby{我}{}らが\ruby{行}{}く\ruby{手}{}に\ruby{光}{}りあり\\
        \ruby{寮}{}で\ruby{培}{}う\ruby{時間}{}を\ruby{糧}{}に\\
        いざうちつれて\ruby{歩}{}みだそう\\
        \ruby{北}{}の\ruby{都}{}に\ruby{世紀}{}はめぐり\\
        \ruby{清華}{}の\ruby{誓今}{}ここに
    
    \end{minipage}
\end{enumerate} % 番号の箇条書き ここまで
% end lilycs
%%%%% 歌詞 ここまで %%%%%
% end body

\end{document}

\documentclass[10pt,b5j]{tarticle} % B6 縦書き
% \documentclass[10pt,b5j]{tarticle} % B6 縦書き
\AtBeginDvi{\special{papersize=128mm,182mm}} % B6 用用紙サイズ
\usepackage{otf} % Unicode で字を入力するのに必要なパッケージ
\usepackage[size=b6j]{bxpapersize} % B6 用紙サイズを指定
\usepackage[dvipdfmx]{graphicx} % 画像を挿入するためのパッケージ
\usepackage[dvipdfmx]{color} % 色をつけるためのパッケージ
\usepackage{pxrubrica} % ルビを振るためのパッケージ
\usepackage{multicol} % 複数段組を作るためのパッケージ
\setlength{\topmargin}{14mm} % 上下方向のマージン
\addtolength{\topmargin}{-1in} % 
\setlength{\oddsidemargin}{11mm} % 左右方向のマージン
\addtolength{\oddsidemargin}{-1in} % 
\setlength{\textwidth}{154mm} % B6 用
\setlength{\textheight}{108mm} % B6 用
\setlength{\headsep}{0mm} % 
\setlength{\headheight}{0mm} % 
\setlength{\topskip}{0mm} % 
\setlength{\parskip}{0pt} % 
\def\labelenumi{\theenumi、} % 箇条書きのフォーマット
\parindent = 0pt % 段落下げしない

 % B6 用テンプレート読み込み

\begin{document}
% begin header
%%%%% タイトルと作者 ここから %%%%%
\begin{minipage}[c]{0.7\hsize} % タイトルは上から 7 割
    \begin{center}
    % begin title
        {\LARGE
            寒気身を刺す % タイトルを入れる
        }
        {\small 
            (昭和四十二年寮歌) % 年などを入れる
        }
    % end title
    \end{center}
\end{minipage}
\begin{minipage}[c]{0.3\hsize} % 作歌作曲は上から 3 割
    \begin{flushright} % 下寄せにする
        % begin name
        岡田雄三君 作歌\\森田弘彦君 作曲 % 作歌・作曲者
        % end name
    \end{flushright}
\end{minipage}
%%%%% タイトルと作者 ここまで %%%%%
% (1,2,3 了あり)
% end header

% begin length
\vspace{1.5em} % タイトル, 作者と歌詞の間に隙間を設ける
\newcommand{\linespace}{0.5em} % 行間の設定
\newcommand{\blocksize}{0.5\hsize} % 段組間の設定
\newcommand{\itemmargin}{3em} % 曲番の位置調整の長さ
% end length
% begin body
%%%%% 歌詞 ここから %%%%%
\begin{enumerate} % 番号の箇条書き ここから
    \setlength{\itemindent}{\itemmargin} % 曲番の位置調整
    \begin{minipage}[c]{\blocksize}
    
        \vspace{\linespace}
        \item~\\
        % 1.
        \ruby{寒気身}{}を\ruby{刺}{}す\ruby{北国}{}の\\
        \ruby{永遠}{}に\ruby{名}{}を\ruby{覇}{}す\ruby{恵迪寮}{}\\
        \ruby{四百野人}{}の\ruby{集}{}いしに\\
        \ruby{我等}{}が\ruby{理想何時}{}の\ruby{日}{}か\\
        \ruby{成}{}さざらむとぞ\ruby{意気高}{}し
        
    \end{minipage}
    \begin{minipage}[c]{\blocksize}
        
        \vspace{\linespace}
        \item~\\
        % 2.
        \ruby{窈窕多}{}し\ruby{札幌}{}に\\
        \ruby{弊衣破帽}{}の\ruby{身}{}なれども\\
        \ruby{一度歌}{}わば\ruby{蛮声}{}の\\
        \ruby{遠}{}く\ruby{手稲}{}に\ruby{木霊}{}して\\
        \ruby{嗚呼誰}{}か\ruby{知}{}る\ruby{吾}{}が\ruby{野心}{}
        
    \end{minipage}
    \begin{minipage}[c]{\blocksize}
        
        \vspace{\linespace}
        \item~\\
        % 3.
        \ruby{燃}{}ゆる\ruby{紅原始林}{}\\
        \ruby{尽}{}きぬ\ruby{想}{}いを\ruby{酒杯}{}に\\
        \ruby{酔}{}えば\ruby{肩取}{}り\ruby{乱舞}{}する\\
        \ruby{吾等}{}が\ruby{行先}{}に\ruby{光明}{}あり\\
        \ruby{楽}{}しからずや\ruby{此}{}の\ruby{饗宴}{}
        
    \end{minipage}
    \begin{minipage}[c]{\blocksize}
        
        \vspace{\linespace}
        \item~\\
        % 4.
        \ruby{蒼空}{}の\ruby{下佇}{}みて\\
        \ruby{木}{}の\ruby{葉身}{}に\ruby{降}{}る\ruby{秋}{}の\ruby{日}{}に\\
        \ruby{仮}{}いこの\ruby{身}{}は\ruby{一介}{}の\\
        \ruby{卑}{}しきものと\ruby{知}{}るとても\\
        \ruby{吾}{}が\ruby{野望}{}は\ruby{永遠}{}に
    
    \end{minipage}
\end{enumerate} % 番号の箇条書き ここまで
%%%%% 歌詞 ここまで %%%%%
% end body

\end{document}

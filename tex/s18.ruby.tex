\documentclass[10pt,b5j]{tarticle} % B6 縦書き
% \documentclass[10pt,b5j]{tarticle} % B6 縦書き
\AtBeginDvi{\special{papersize=128mm,182mm}} % B6 用用紙サイズ
\usepackage{otf} % Unicode で字を入力するのに必要なパッケージ
\usepackage[size=b6j]{bxpapersize} % B6 用紙サイズを指定
\usepackage[dvipdfmx]{graphicx} % 画像を挿入するためのパッケージ
\usepackage[dvipdfmx]{color} % 色をつけるためのパッケージ
\usepackage{pxrubrica} % ルビを振るためのパッケージ
\usepackage{plext} % 漢数字の enumerate を使うためのパッケージ
\usepackage{multicol} % 複数段組を作るためのパッケージ
\setlength{\topmargin}{14mm} % 上下方向のマージン
\addtolength{\topmargin}{-1in} % 
\setlength{\oddsidemargin}{11mm} % 左右方向のマージン
\addtolength{\oddsidemargin}{-1in} % 
\setlength{\textwidth}{154mm} % B6 用
\setlength{\textheight}{108mm} % B6 用
\setlength{\headsep}{0mm} % 
\setlength{\headheight}{0mm} % 
\setlength{\topskip}{0mm} % 
\setlength{\parskip}{0pt} % 
\def\theenumi{\Kanji{enumi}} % 箇条書きのフォーマットを漢数字に変更
\parindent = 0pt % 段落下げしない
\pagestyle{empty} % すべてのページ番号を消去
% \renewcommand{\baselinestretch}{0.9} % 行間の倍率
 % B6 用テンプレート読み込み

\begin{document}
% begin header
%%%%% タイトルと作者 ここから %%%%%
\begin{minipage}[c]{0.7\hsize} % タイトルは上から 7 割
    \begin{center}
    % begin title
        {\LARGE
            天地の奥に % タイトルを入れる
        }
        {\small 
            (昭和十八年寮歌) % 年などを入れる
        }
    % end title
    \end{center}
\end{minipage}
\begin{minipage}[c]{0.3\hsize} % 作歌作曲は上から 3 割
    \begin{flushright} % 下寄せにする
        % begin name
        橋爪秀雄君 作歌\\池田政晴君 作曲 % 作歌・作曲者
        % end name
    \end{flushright}
\end{minipage}
%%%%% タイトルと作者 ここまで %%%%%
% (1,2,6 了あり)
% end header

% begin body
\vspace{1.5em} % タイトル, 作者と歌詞の間に隙間を設ける
\newcommand{\linespace}{0.5em} % 行間の設定
\newcommand{\blocksize}{0.5\hsize} % 段組間の設定
%%%%% 歌詞 ここから %%%%%
% begin lilycs
\begin{enumerate} % 番号の箇条書き ここから
    \begin{minipage}[c]{\blocksize}
    
        \vspace{\linespace}
        \item
        % 1.
        \ruby{天地}{}の\ruby{奥}{}に\ruby{征}{}く\ruby{吾}{}や\\
        \ruby{弧杖無限}{}に\ruby{旅立}{}ちて\\
        \ruby{渓巒}{}はるか\ruby{訪}{}ね\ruby{来}{}し\\
        \ruby{楡陵}{}の\ruby{宿}{}や\ruby{三春}{}の\\
        \ruby{旅}{}にしあれどそは\ruby{深}{}き\\
        \ruby{噫魂}{}のふるさとか
        
        \vspace{\linespace}
        \item
        % 2.
        \ruby{四大}{}も\ruby{夢}{}む\ruby{幌}{}のさと\\
        \ruby{歌}{}の\ruby{心}{}を\ruby{温}{}ぬれば\\
        \ruby{馥}{}り\ruby{床}{}しきアカシヤの\\
        \ruby{花仄白}{}き\ruby{憂}{}あり\\
        \ruby{夏宵}{}の\ruby{霞靉}{}びきて\\
        \ruby{月皎々}{}の\ruby{滄海}{}をゆく
        
        \vspace{\linespace}
        \item
        % 3.
        \ruby{大空風}{}に\ruby{咽}{}ぶよひ\\
        \ruby{暮鐘}{}は\ruby{低}{}く\ruby{漂}{}ひて\\
        \ruby{荒野}{}は\ruby{凋落}{}の\ruby{悲歌}{}に\ruby{泣}{}く\\
        \ruby{栄枯}{}は\ruby{移}{}る\ruby{秋}{}の\ruby{日}{}の\\
        \ruby{秋思}{}の\ruby{歩}{}み\ruby{運}{}ぶ\ruby{夜半}{}\\
        \ruby{久遠}{}の\ruby{星}{}を\ruby{仰}{}がずや
        
        \vspace{\linespace}
        \item
        % 4.
        \ruby{高}{}き\ruby{理想}{}は\ruby{人}{}の\ruby{世}{}を\\
        \ruby{人}{}の\ruby{世}{}と\ruby{生}{}く\ruby{佗}{}しさに\\
        \ruby{坤球鳴}{}りて\ruby{吹雪}{}き\ruby{狂}{}ふ\\
        \ruby{孤高}{}の\ruby{峯}{}に\ruby{伏}{}する\ruby{今}{}\\
        \ruby{浮生}{}の\ruby{夢}{}は\ruby{消}{}え\ruby{果}{}てて\\
        \ruby{心虚}{}しき\ruby{歓喜}{}よ
        
        \vspace{\linespace}
        \item
        % 5.
        \ruby{北溟春}{}は\ruby{浅}{}けれど\\
        \ruby{森}{}かげ\ruby{清}{}く\ruby{黄花咲}{}き\\
        \ruby{雲雀}{}は\ruby{高}{}く\ruby{空}{}に\ruby{入}{}り\\
        \ruby{新生}{}の\ruby{合唱野}{}に\ruby{満}{}てり\\
        \ruby{古衣}{}を\ruby{重}{}ぬる\ruby{日}{}は\ruby{逝}{}いて\\
        \ruby{時乾坤}{}に\ruby{春}{}よ\ruby{立}{}つ
        
        \vspace{\linespace}
        \item
        % 6.
        いざ\ruby{浩歌}{}はなん\ruby{天壤}{}の\\
        \ruby{栄}{}ゆる\ruby{時}{}ぞ\ruby{益荒男}{}の\\
        \ruby{事}{}ふる\ruby{道}{}は\ruby{烈}{}しかる\\
        \ruby{今宵祭}{}の\ruby{聖}{}き\ruby{火}{}に\\
        \ruby{尊}{}き\ruby{誓}{}ひ\ruby{立}{}てよかし\\
        \ruby{興亡分}{}るる\ruby{秋}{}なれば
    
    \end{minipage}
\end{enumerate} % 番号の箇条書き ここまで
% end lilycs
%%%%% 歌詞 ここまで %%%%%
% end body

\end{document}

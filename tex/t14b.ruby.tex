\documentclass[10pt,b5j]{tarticle} % B6 縦書き
% \documentclass[10pt,b5j]{tarticle} % B6 縦書き
\AtBeginDvi{\special{papersize=128mm,182mm}} % B6 用用紙サイズ
\usepackage{otf} % Unicode で字を入力するのに必要なパッケージ
\usepackage[size=b6j]{bxpapersize} % B6 用紙サイズを指定
\usepackage[dvipdfmx]{graphicx} % 画像を挿入するためのパッケージ
\usepackage[dvipdfmx]{color} % 色をつけるためのパッケージ
\usepackage{pxrubrica} % ルビを振るためのパッケージ
\usepackage{multicol} % 複数段組を作るためのパッケージ
\setlength{\topmargin}{14mm} % 上下方向のマージン
\addtolength{\topmargin}{-1in} % 
\setlength{\oddsidemargin}{11mm} % 左右方向のマージン
\addtolength{\oddsidemargin}{-1in} % 
\setlength{\textwidth}{154mm} % B6 用
\setlength{\textheight}{108mm} % B6 用
\setlength{\headsep}{0mm} % 
\setlength{\headheight}{0mm} % 
\setlength{\topskip}{0mm} % 
\setlength{\parskip}{0pt} % 
\def\labelenumi{\theenumi、} % 箇条書きのフォーマット
\parindent = 0pt % 段落下げしない

 % B6 用テンプレート読み込み

\begin{document}
% begin header
%%%%% タイトルと作者 ここから %%%%%
\begin{minipage}[c]{0.7\hsize} % タイトルは上から 7 割
    \begin{center}
    % begin title
        {\LARGE
            大地はなごやかに % タイトルを入れる
        }
        {\small 
            (大正十四年開舎二十周年記念寮歌) % 年などを入れる
        }
    % end title
    \end{center}
\end{minipage}
\begin{minipage}[c]{0.3\hsize} % 作歌作曲は上から 3 割
    \begin{flushright} % 下寄せにする
        % begin name
        黒沢徹君 作歌\\三溝清美君 作曲 % 作歌・作曲者
        % end name
    \end{flushright}
\end{minipage}
%%%%% タイトルと作者 ここまで %%%%%
% (1,2,3 了あり)
% end header

% begin body
\vspace{1.5em} % タイトル, 作者と歌詞の間に隙間を設ける
\newcommand{\linespace}{0.5em} % 行間の設定
\newcommand{\blocksize}{0.5\hsize} % 段組間の設定
%%%%% 歌詞 ここから %%%%%
% begin lilycs
\begin{enumerate} % 番号の箇条書き ここから
    \begin{minipage}[c]{\blocksize}
    
        \vspace{\linespace}
        \item
        % 1.
        \ruby{大地}{}はなごやかにうるほひて\\
        \ruby{丘陵}{}の\ruby{傾斜}{}の\ruby{若草}{}や\\
        さゆらぐ\ruby{楡}{}の\ruby{嫩葉}{}にも\\
        \ruby{春新生}{}の\ruby{精気}{}は\ruby{溢}{}る\\
        \ruby{原始林}{}の\ruby{緑}{}に\ruby{流}{}れ\ruby{来}{}る\\
        \ruby{嗚呼青春}{}の\ruby{讃歌}{}
        
        \vspace{\linespace}
        \item
        % 2.
        \ruby{色紫}{}の\ruby{彩絹}{}に\\
        \ruby{染}{}めて\ruby{溶}{}けたる\ruby{朝霧}{}の\\
        \ruby{悠久}{}の\ruby{蒼穹}{}はるかにも\\
        \ruby{濃}{}き\ruby{水色}{}にうつろへば\\
        \ruby{白鳥高}{}く\ruby{海}{}に\ruby{飛}{}び\\
        \ruby{入江}{}の\ruby{波}{}に\ruby{夏陽}{}は\ruby{映}{}ゆる
        
        \vspace{\linespace}
        \item
        % 3.
        \ruby{連嶺紅}{}に\ruby{黄昏}{}れて\\
        \ruby{夕靄流}{}る\ruby{水沼}{}の\\
        \ruby{白}{}き\ruby{葦穂波}{}に\ruby{顫}{}ふ\ruby{月}{}\\
        \ruby{幽暗}{}の\ruby{草野}{}に\ruby{訪}{}づれば\\
        \ruby{仄}{}かに\ruby{響}{}く\ruby{胸}{}うちの\\
        \ruby{高遠}{}き\ruby{感激}{}に\ruby{逍遙}{}ふ\ruby{哉}{}
        
        \vspace{\linespace}
        \item
        % 4.
        \ruby{神秘}{}の\ruby{森林}{}に\ruby{群星}{}さえて\\
        \ruby{雪}{}の\ruby{曠野遠}{}く\ruby{静謐}{}なり\\
        \ruby{銀壺}{}にゆるる\ruby{灯}{}に\\
        \ruby{崇}{}き\ruby{教訓}{}を\ruby{胸}{}にして\\
        \ruby{心}{}の\ruby{憧憬郷}{}にまどゐする\\
        \ruby{若}{}き\ruby{人等}{}の\ruby{哀歓}{}よ
        
        \vspace{\linespace}
        \item
        % 5.
        \ruby{陽炎}{}ゆらぐ\ruby{春}{}の\ruby{日}{}に\\
        \ruby{落葉}{}しぐる\ruby{秋}{}の\ruby{夜}{}に\\
        \ruby{胸}{}に\ruby{高鳴}{}る\ruby{青春}{}の\\
        \ruby{若}{}き\ruby{誇}{}りを\ruby{歌}{}ひつつ\\
        \ruby{限}{}れる\ruby{生}{}の\ruby{瞬時}{}を\\
        \ruby{深}{}き\ruby{瞑想}{}に\ruby{過}{}さずや
    
    \end{minipage}
\end{enumerate} % 番号の箇条書き ここまで
% end lilycs
%%%%% 歌詞 ここまで %%%%%
% end body

\end{document}

\documentclass[10pt,b5j]{tarticle} % B6 縦書き
% \documentclass[10pt,b5j]{tarticle} % B6 縦書き
\AtBeginDvi{\special{papersize=128mm,182mm}} % B6 用用紙サイズ
\usepackage{otf} % Unicode で字を入力するのに必要なパッケージ
\usepackage[size=b6j]{bxpapersize} % B6 用紙サイズを指定
\usepackage[dvipdfmx]{graphicx} % 画像を挿入するためのパッケージ
\usepackage[dvipdfmx]{color} % 色をつけるためのパッケージ
\usepackage{pxrubrica} % ルビを振るためのパッケージ
\usepackage{multicol} % 複数段組を作るためのパッケージ
\setlength{\topmargin}{14mm} % 上下方向のマージン
\addtolength{\topmargin}{-1in} % 
\setlength{\oddsidemargin}{11mm} % 左右方向のマージン
\addtolength{\oddsidemargin}{-1in} % 
\setlength{\textwidth}{154mm} % B6 用
\setlength{\textheight}{108mm} % B6 用
\setlength{\headsep}{0mm} % 
\setlength{\headheight}{0mm} % 
\setlength{\topskip}{0mm} % 
\setlength{\parskip}{0pt} % 
\def\labelenumi{\theenumi、} % 箇条書きのフォーマット
\parindent = 0pt % 段落下げしない

 % B6 用テンプレート読み込み

\begin{document}
% begin header
%%%%% タイトルと作者 ここから %%%%%
\begin{minipage}[c]{0.7\hsize} % タイトルは上から 7 割
    \begin{center}
    % begin title
        {\LARGE
            湖に星の散るなり % タイトルを入れる
        }
        {\small 
            (昭和十六年寮歌) % 年などを入れる
        }
    % end title
    \end{center}
\end{minipage}
\begin{minipage}[c]{0.3\hsize} % 作歌作曲は上から 3 割
    \begin{flushright} % 下寄せにする
        % begin name
        切替辰哉君 作歌\\岡田和雄君 作曲 % 作歌・作曲者
        % end name
    \end{flushright}
\end{minipage}
%%%%% タイトルと作者 ここまで %%%%%
% (1 繰り返しなし)
% end header

% begin body
\vspace{1.5em} % タイトル, 作者と歌詞の間に隙間を設ける
\newcommand{\linespace}{0.5em} % 行間の設定
\newcommand{\blocksize}{0.5\hsize} % 段組間の設定
%%%%% 歌詞 ここから %%%%%
% begin lilycs
\begin{enumerate} % 番号の箇条書き ここから
    \begin{minipage}[c]{\blocksize}
    
        \vspace{\linespace}
        \item
        % 1.
        \ruby{湖}{}に\ruby{星}{}の\ruby{散}{}るなり\ruby{幽}{}けさよ\\
        \ruby{松}{}の\ruby{火燃}{}えて\\
        \ruby{漕}{}ぎ\ruby{出}{}づる\ruby{愛奴}{}の\ruby{漁舟}{}の\\
        \ruby{岸辺佇}{}ち\ruby{沁々眺}{}む\\
        \ruby{旅}{}の\ruby{日}{}ははや\ruby{暮}{}れゆきぬ\\
        \ruby{夢}{}に\ruby{酔}{}ひ\ruby{夢}{}にぞ\ruby{歎}{}かん\\
        \ruby{汚}{}れなき\ruby{心}{}を\ruby{慕}{}ふ\\
        \ruby{大}{}いなる\ruby{支笏}{}の\ruby{湖}{}よ\\
        \ruby{花若}{}く\ruby{我汝}{}が\ruby{許}{}に\\
        \ruby{希望満}{}ち\ruby{今宵宿}{}らん
        
        \vspace{\linespace}
        \item
        % 2.
        \ruby{轟}{}けるか\ruby{雄叫}{}びよ\ruby{創造}{}の\\
        \ruby{歴程一路新}{}しき\ruby{使命}{}に\ruby{捧}{}ぐ\\
        \ruby{幸}{}の\ruby{今日}{}にしあれば\\
        \ruby{忍苦}{}して\ruby{欣求}{}むるところ\\
        \ruby{得}{}べくして\ruby{得}{}べからざりし\\
        \ruby{秀麗}{}わしきまことの\ruby{道}{}ぞ\\
        \ruby{近}{}くして\ruby{遙}{}かなる\ruby{哉}{}\\
        \ruby{若}{}き\ruby{世}{}の\ruby{秩序}{}を\ruby{背負}{}ふ\\
        \ruby{洋々}{}の\ruby{日}{}と\ruby{倶}{}にゆかなむ
        
        \vspace{\linespace}
        \item
        % 3.
        \ruby{乾坤}{}に\ruby{伏}{}し\ruby{祈}{}るなり\ruby{栄光}{}あれ\\
        \ruby{祖国}{}の\ruby{生命決意}{}する\\
        \ruby{光眩}{}ゆく\ruby{手}{}に\ruby{取}{}りぬ\ruby{楡}{}の\ruby{嫩葉}{}\\
        \ruby{葉脈}{}の\ruby{強}{}きを\ruby{讃}{}ふ\\
        \ruby{草々}{}のたおれ\ruby{生}{}れて\\
        \ruby{春青}{}み\ruby{辛夷咲}{}くなり\\
        \ruby{逍遙}{}の\ruby{原始林蔭清}{}く\\
        \ruby{暢}{}び\ruby{行}{}かん\ruby{我}{}が\ruby{民族}{}の\\
        \ruby{逞}{}しき\ruby{息吹}{}き\ruby{感}{}じぬ
        
        \vspace{\linespace}
        \item
        % 4.
        \ruby{立}{}て\ruby{歩}{}め\ruby{光}{}の\ruby{中}{}を\ruby{国民}{}の\\
        \ruby{重}{}き\ruby{責任負}{}ひ\ruby{燦}{}めきの\\
        \ruby{星辰}{}は\ruby{語}{}らひ\ruby{微香}{}る\ruby{大地囁}{}きぬ\\
        \ruby{甦生}{}へる\ruby{征覇}{}のいくさ\ruby{祝歌}{}ふ\\
        \ruby{吾等}{}が\ruby{双頬}{}に\\
        \ruby{失}{}はじ\ruby{高}{}きが\ruby{矜持護}{}り\ruby{来}{}し\\
        \ruby{伝統}{}の\ruby{法火}{}\\
        \ruby{浄}{}らかに\ruby{燃}{}え\ruby{熾}{}る\ruby{刻}{}\\
        \ruby{継}{}ぎ\ruby{行}{}かな\ruby{来}{}ん\ruby{若人}{}に
    
    \end{minipage}
\end{enumerate} % 番号の箇条書き ここまで
% end lilycs
%%%%% 歌詞 ここまで %%%%%
% end body

\end{document}

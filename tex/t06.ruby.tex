\documentclass[10pt,b5j]{tarticle} % B6 縦書き
% \documentclass[10pt,b5j]{tarticle} % B6 縦書き
\AtBeginDvi{\special{papersize=128mm,182mm}} % B6 用用紙サイズ
\usepackage{otf} % Unicode で字を入力するのに必要なパッケージ
\usepackage[size=b6j]{bxpapersize} % B6 用紙サイズを指定
\usepackage[dvipdfmx]{graphicx} % 画像を挿入するためのパッケージ
\usepackage[dvipdfmx]{color} % 色をつけるためのパッケージ
\usepackage{pxrubrica} % ルビを振るためのパッケージ
\usepackage{plext} % 漢数字の enumerate を使うためのパッケージ
\usepackage{multicol} % 複数段組を作るためのパッケージ
\setlength{\topmargin}{14mm} % 上下方向のマージン
\addtolength{\topmargin}{-1in} % 
\setlength{\oddsidemargin}{11mm} % 左右方向のマージン
\addtolength{\oddsidemargin}{-1in} % 
\setlength{\textwidth}{154mm} % B6 用
\setlength{\textheight}{108mm} % B6 用
\setlength{\headsep}{0mm} % 
\setlength{\headheight}{0mm} % 
\setlength{\topskip}{0mm} % 
\setlength{\parskip}{0pt} % 
\def\theenumi{\Kanji{enumi}} % 箇条書きのフォーマットを漢数字に変更
\parindent = 0pt % 段落下げしない
\pagestyle{empty} % すべてのページ番号を消去
% \renewcommand{\baselinestretch}{0.9} % 行間の倍率
 % B6 用テンプレート読み込み

\begin{document}
% begin header
%%%%% タイトルと作者 ここから %%%%%
\begin{minipage}[c]{0.7\hsize} % タイトルは上から 7 割
    \begin{center}
    % begin title
        {\LARGE
            魔神の呪 % タイトルを入れる
        }
        {\small 
            (大正六年寮歌) % 年などを入れる
        }
    % end title
    \end{center}
\end{minipage}
\begin{minipage}[c]{0.3\hsize} % 作歌作曲は上から 3 割
    \begin{flushright} % 下寄せにする
        % begin name
        佐藤惣之助君 作歌\\植村泰二君 作曲 % 作歌・作曲者
        % end name
    \end{flushright}
\end{minipage}
%%%%% タイトルと作者 ここまで %%%%%
% (1,6 了あり)
% end header

% begin body
\vspace{1.5em} % タイトル, 作者と歌詞の間に隙間を設ける
\newcommand{\linespace}{0.5em} % 行間の設定
\newcommand{\blocksize}{0.5\hsize} % 段組間の設定
%%%%% 歌詞 ここから %%%%%
% begin lilycs
\begin{enumerate} % 番号の箇条書き ここから
    \begin{minipage}[c]{\blocksize}
    
        \vspace{\linespace}
        \item
        % 1.
        \ruby{魔神}{}の\ruby{呪}{}アルペンの\\
        \ruby{白雪永久}{}に\ruby{清}{}からず\\
        \ruby{見}{}よ\ruby{永劫}{}と\ruby{誓}{}ひけん\\
        \ruby{平和}{}の\ruby{春}{}は\ruby{短}{}くて\\
        \ruby{吹}{}く\ruby{凋落}{}の\ruby{秋風}{}に\\
        \ruby{正義}{}の\ruby{光影}{}くらし
        
        \vspace{\linespace}
        \item
        % 2.
        されど\ruby{儼然東洋}{}に\\
        その\ruby{義}{}と\ruby{侠}{}を\ruby{胸}{}にして\\
        \ruby{燦}{}たる\ruby{北斗北陲}{}の\\
        \ruby{強}{}と\ruby{仰}{}がれ\ruby{誇矜}{}りつつ\\
        \ruby{自治}{}を\ruby{精神}{}の\ruby{我寮}{}は\\
        \ruby{映華}{}ある\ruby{歴史十二年}{}
        
        \vspace{\linespace}
        \item
        % 3.
        \ruby{嗚呼北海}{}の\ruby{荒吹雪}{}\\
        \ruby{白箭膚}{}を\ruby{擘}{}くも\\
        \ruby{胸}{}の\ruby{狂瀾青春}{}の\\
        \ruby{血潮}{}に\ruby{如何}{}で\ruby{比}{}すべきぞ\\
        \ruby{力}{}の\ruby{緒琴高鳴}{}りて\\
        \ruby{紅燃}{}ゆる\ruby{悶}{}えあり
        
        \vspace{\linespace}
        \item
        % 4.
        \ruby{残陽西}{}に\ruby{茜}{}して\\
        \ruby{今日}{}も\ruby{暮}{}れ\ruby{行}{}く\ruby{手稲山}{}\\
        \ruby{雲}{}の\ruby{五彩}{}を\ruby{眺}{}めては\\
        \ruby{思}{}ひは\ruby{遠}{}く\ruby{渺茫}{}の\\
        \ruby{彼}{}の\ruby{海}{}を\ruby{越}{}え\ruby{山}{}を\ruby{越}{}え\\
        \ruby{雄図千里}{}ぞ\ruby{駆}{}りゆく
        
        \vspace{\linespace}
        \item
        % 5.
        \ruby{平和}{}の\ruby{流}{}れ\ruby{豊平}{}の\\
        \ruby{狭霧罩}{}めたる\ruby{朝}{}ぼらけ\\
        \ruby{東指}{}して\ruby{流}{}れ\ruby{行}{}く\\
        \ruby{淙々}{}の\ruby{音}{}を\ruby{我聴}{}けば\\
        \ruby{瀬々}{}の\ruby{河波声}{}あげて\\
        \ruby{唄}{}ふ「\ruby{自由}{}」の\ruby{二字}{}の\ruby{曲}{}
        
        \vspace{\linespace}
        \item
        % 6.
        \ruby{今宵楡影}{}に\ruby{団欒}{}して\\
        \ruby{月影}{}に\ruby{酌}{}む\ruby{自治}{}の\ruby{宴}{}\\
        \ruby{廻}{}る\ruby{盃夜}{}も\ruby{更}{}けて\\
        \ruby{北斗傾}{}く\ruby{玻璃}{}の\ruby{窓}{}\\
        いざ\ruby{吾}{}が\ruby{友}{}よ\ruby{熟睡}{}せむ\\
        \ruby{明日}{}は\ruby{人生}{}の\ruby{旅}{}なれば
    
    \end{minipage}
\end{enumerate} % 番号の箇条書き ここまで
% end lilycs
%%%%% 歌詞 ここまで %%%%%
% end body

\end{document}

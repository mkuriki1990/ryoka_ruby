\documentclass[10pt,b5j]{tarticle} % B6 縦書き
% \documentclass[10pt,b5j]{tarticle} % B6 縦書き
\AtBeginDvi{\special{papersize=128mm,182mm}} % B6 用用紙サイズ
\usepackage{otf} % Unicode で字を入力するのに必要なパッケージ
\usepackage[size=b6j]{bxpapersize} % B6 用紙サイズを指定
\usepackage[dvipdfmx]{graphicx} % 画像を挿入するためのパッケージ
\usepackage[dvipdfmx]{color} % 色をつけるためのパッケージ
\usepackage{pxrubrica} % ルビを振るためのパッケージ
\usepackage{multicol} % 複数段組を作るためのパッケージ
\setlength{\topmargin}{14mm} % 上下方向のマージン
\addtolength{\topmargin}{-1in} % 
\setlength{\oddsidemargin}{11mm} % 左右方向のマージン
\addtolength{\oddsidemargin}{-1in} % 
\setlength{\textwidth}{154mm} % B6 用
\setlength{\textheight}{108mm} % B6 用
\setlength{\headsep}{0mm} % 
\setlength{\headheight}{0mm} % 
\setlength{\topskip}{0mm} % 
\setlength{\parskip}{0pt} % 
\def\labelenumi{\theenumi、} % 箇条書きのフォーマット
\parindent = 0pt % 段落下げしない

 % B6 用テンプレート読み込み

\begin{document}
% begin header
%%%%% タイトルと作者 ここから %%%%%
\begin{minipage}[c]{0.7\hsize} % タイトルは上から 7 割
    \begin{center}
    % begin title
        {\LARGE
            噫妖雲は % タイトルを入れる
        }
        {\small 
            (昭和十年寮歌) % 年などを入れる
        }
    % end title
    \end{center}
\end{minipage}
\begin{minipage}[c]{0.3\hsize} % 作歌作曲は上から 3 割
    \begin{flushright} % 下寄せにする
        % begin name
        川村真君 作歌\\荻野辰夫君 作曲 % 作歌・作曲者
        % end name
    \end{flushright}
\end{minipage}
%%%%% タイトルと作者 ここまで %%%%%
% (1,2,3,6 了あり)
% end header

% begin length
\vspace{1.5em} % タイトル, 作者と歌詞の間に隙間を設ける
\newcommand{\linespace}{0.5em} % 行間の設定
\newcommand{\blocksize}{0.5\hsize} % 段組間の設定
\newcommand{\itemmargin}{3em} % 曲番の位置調整の長さ
% end length
% begin body
%%%%% 歌詞 ここから %%%%%
\begin{enumerate} % 番号の箇条書き ここから
    \setlength{\itemindent}{\itemmargin} % 曲番の位置調整
    \begin{minipage}[c]{\blocksize}
    
        \vspace{\linespace}
        \item~\\
        % 1.
        \ruby{噫妖雲}{}は\ruby{狂}{}へども\\
        \ruby{迪}{}を\ruby{恵}{}めし\ruby{若人等}{}\\
        \ruby{巍然四寮}{}に\ruby{立籠}{}もり\\
        \ruby{覚醒}{}の\ruby{歌高誦}{}ふかな
        
    \end{minipage}
    \begin{minipage}[c]{\blocksize}
        
        \vspace{\linespace}
        \item~\\
        % 2.
        \ruby{三年}{}の\ruby{契浅}{}からず\\
        \ruby{爛漫春}{}を\ruby{欺}{}けど\\
        \ruby{銀觴口辺}{}にうつろへば\\
        \ruby{名残}{}の\ruby{春}{}を\ruby{惜}{}むべし
        
    \end{minipage}
    \begin{minipage}[c]{\blocksize}
        
        \vspace{\linespace}
        \item~\\
        % 3.
        \ruby{羊}{}の\ruby{群}{}は\ruby{去}{}り\ruby{行}{}きて\\
        \ruby{角笛遠}{}くこだましぬ\\
        \ruby{夏草深}{}き\ruby{丘上}{}に\\
        \ruby{月三更}{}の\ruby{影冴}{}ゆる
        
    \end{minipage}
    \begin{minipage}[c]{\blocksize}
        
        \vspace{\linespace}
        \item~\\
        % 4.
        \ruby{不壊}{}の\ruby{生命}{}と\ruby{輝}{}きし\\
        \ruby{緑葉漸}{}く\ruby{紅葉}{}して\\
        \ruby{今玲瓏}{}の\ruby{谿谷}{}に\\
        \ruby{若}{}き\ruby{男}{}の\ruby{子}{}の\ruby{寮歌消}{}ゆる
        
    \end{minipage}
    \begin{minipage}[c]{\blocksize}
        
        \vspace{\linespace}
        \item~\\
        % 5.
        \ruby{颯々}{}の\ruby{風音寒}{}く\\
        \ruby{橇}{}の\ruby{音孤絃}{}の\ruby{月}{}を\ruby{呼}{}ぶ\\
        \ruby{窓}{}に\ruby{佇}{}む\ruby{多感}{}の\ruby{遊子}{}\\
        \ruby{今宵何}{}をか\ruby{思}{}ふらん
        
    \end{minipage}
    \begin{minipage}[c]{\blocksize}
        
        \vspace{\linespace}
        \item~\\
        % 6.
        \ruby{月影淡}{}き\ruby{楡}{}の\ruby{陵}{}\\
        \ruby{記念}{}の\ruby{祭終}{}るなり\\
        \ruby{篝火焚}{}きて\ruby{我}{}は\ruby{今}{}\\
        \ruby{静}{}かに\ruby{宵}{}を\ruby{誦}{}はなん
    
    \end{minipage}
\end{enumerate} % 番号の箇条書き ここまで
%%%%% 歌詞 ここまで %%%%%
% end body

\end{document}

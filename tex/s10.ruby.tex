\documentclass[10pt,b5j]{tarticle} % B6 縦書き
% \documentclass[10pt,b5j]{tarticle} % B6 縦書き
\AtBeginDvi{\special{papersize=128mm,182mm}} % B6 用用紙サイズ
\usepackage{otf} % Unicode で字を入力するのに必要なパッケージ
\usepackage[size=b6j]{bxpapersize} % B6 用紙サイズを指定
\usepackage[dvipdfmx]{graphicx} % 画像を挿入するためのパッケージ
\usepackage[dvipdfmx]{color} % 色をつけるためのパッケージ
\usepackage{pxrubrica} % ルビを振るためのパッケージ
\usepackage{plext} % 漢数字の enumerate を使うためのパッケージ
\usepackage{multicol} % 複数段組を作るためのパッケージ
\setlength{\topmargin}{14mm} % 上下方向のマージン
\addtolength{\topmargin}{-1in} % 
\setlength{\oddsidemargin}{11mm} % 左右方向のマージン
\addtolength{\oddsidemargin}{-1in} % 
\setlength{\textwidth}{154mm} % B6 用
\setlength{\textheight}{108mm} % B6 用
\setlength{\headsep}{0mm} % 
\setlength{\headheight}{0mm} % 
\setlength{\topskip}{0mm} % 
\setlength{\parskip}{0pt} % 
\def\theenumi{\Kanji{enumi}} % 箇条書きのフォーマットを漢数字に変更
\parindent = 0pt % 段落下げしない
\pagestyle{empty} % すべてのページ番号を消去
% \renewcommand{\baselinestretch}{0.9} % 行間の倍率
 % B6 用テンプレート読み込み

\begin{document}
% begin header
%%%%% タイトルと作者 ここから %%%%%
\begin{minipage}[c]{0.7\hsize} % タイトルは上から 7 割
    \begin{center}
    % begin title
        {\LARGE
            噫妖雲は % タイトルを入れる
        }
        {\small 
            (昭和十年寮歌) % 年などを入れる
        }
    % end title
    \end{center}
\end{minipage}
\begin{minipage}[c]{0.3\hsize} % 作歌作曲は上から 3 割
    \begin{flushright} % 下寄せにする
        % begin name
        川村真君 作歌\\荻野辰夫君 作曲 % 作歌・作曲者
        % end name
    \end{flushright}
\end{minipage}
%%%%% タイトルと作者 ここまで %%%%%
% (1,2,3,6 了あり)
% end header

% begin length
\vspace{1.5em} % タイトル, 作者と歌詞の間に隙間を設ける
\newcommand{\linespace}{0.5em} % 行間の設定
\newcommand{\blocksize}{0.5\hsize} % 段組間の設定
\newcommand{\itemmargin}{3em} % 曲番の位置調整の長さ
% end length
% begin body
%%%%% 歌詞 ここから %%%%%
\begin{enumerate} % 番号の箇条書き ここから
    \setlength{\itemindent}{\itemmargin} % 曲番の位置調整
    \begin{minipage}[c]{\blocksize}
    
        \vspace{\linespace}
        \item~\\
        % 1.
        \ruby{噫妖}{}\ruby{雲}{くも}は\ruby{狂}{きょう}へども\\
        \ruby{迪}{すすむ}を\ruby{恵}{めぐ}めし\ruby{若人}{わこうど}\ruby{等}{とう}\\
        \ruby{巍然}{ぎぜん}\ruby{四}{よん}\ruby{寮}{りょう}に\ruby{立}{たつ}\ruby{籠}{かご}もり\\
        \ruby{覚醒}{かくせい}の\ruby{歌}{うた}\ruby{高}{だか}\ruby{誦}{}ふかな
        
    \end{minipage}
    \begin{minipage}[c]{\blocksize}
        
        \vspace{\linespace}
        \item~\\
        % 2.
        \ruby{三年}{みとし}の\ruby{契}{ちぎり}\ruby{浅}{あさ}からず\\
        \ruby{爛漫}{らんまん}\ruby{春}{はる}を\ruby{欺}{あざむ}けど\\
        \ruby{銀}{ぎん}\ruby{觴口}{}\ruby{辺}{あたり}にうつろへば\\
        \ruby{名残}{なごり}の\ruby{春}{はる}を\ruby{惜}{}むべし
        
    \end{minipage}
    \begin{minipage}[c]{\blocksize}
        
        \vspace{\linespace}
        \item~\\
        % 3.
        \ruby{羊}{ひつじ}の\ruby{群}{ぐん}は\ruby{去}{さ}り\ruby{行}{ゆ}きて\\
        \ruby{角笛}{つのぶえ}\ruby{遠}{とお}くこだましぬ\\
        \ruby{夏}{なつ}\ruby{草深}{くさぶか}き\ruby{丘上}{おかうえ}に\\
        \ruby{月}{つき}\ruby{三更}{さんこう}の\ruby{影}{かげ}\ruby{冴}{さえ}ゆる
        
    \end{minipage}
    \begin{minipage}[c]{\blocksize}
        
        \vspace{\linespace}
        \item~\\
        % 4.
        \ruby{不壊}{ふえ}の\ruby{生命}{いのち}と\ruby{輝}{かがや}きし\\
        \ruby{緑葉}{りょくよう}\ruby{漸}{ようや}く\ruby{紅葉}{こうよう}して\\
        \ruby{今}{こん}\ruby{玲瓏}{れいろう}の\ruby{谿谷}{けいこく}に\\
        \ruby{若}{わか}き\ruby{男}{おとこ}の\ruby{子}{こ}の\ruby{寮歌}{りょうか}\ruby{消}{しょう}ゆる
        
    \end{minipage}
    \begin{minipage}[c]{\blocksize}
        
        \vspace{\linespace}
        \item~\\
        % 5.
        \ruby{颯々}{さっさつ}の\ruby{風音}{かざね}\ruby{寒}{さむ}く\\
        \ruby{橇}{そり}の\ruby{音}{おと}\ruby{孤}{こ}\ruby{絃}{いと}の\ruby{月}{つき}を\ruby{呼}{よ}ぶ\\
        \ruby{窓}{まど}に\ruby{佇}{たたず}む\ruby{多感}{たかん}の\ruby{遊子}{ゆうし}\\
        \ruby{今宵}{こよい}\ruby{何}{なに}をか\ruby{思}{おも}ふらん
        
    \end{minipage}
    \begin{minipage}[c]{\blocksize}
        
        \vspace{\linespace}
        \item~\\
        % 6.
        \ruby{月影}{つきかげ}\ruby{淡}{あわ}き\ruby{楡}{にれ}の\ruby{陵}{りょう}\\
        \ruby{記念}{きねん}の\ruby{祭}{まつり}\ruby{終}{おわ}るなり\\
        \ruby{篝火}{かがりび}\ruby{焚}{た}きて\ruby{我}{が}は\ruby{今}{いま}\\
        \ruby{静}{しず}かに\ruby{宵}{よい}を\ruby{誦}{}はなん
    
    \end{minipage}
\end{enumerate} % 番号の箇条書き ここまで
%%%%% 歌詞 ここまで %%%%%
% end body

\end{document}

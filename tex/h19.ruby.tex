\documentclass[10pt,b5j]{tarticle} % B6 縦書き
% \documentclass[10pt,b5j]{tarticle} % B6 縦書き
\AtBeginDvi{\special{papersize=128mm,182mm}} % B6 用用紙サイズ
\usepackage{otf} % Unicode で字を入力するのに必要なパッケージ
\usepackage[size=b6j]{bxpapersize} % B6 用紙サイズを指定
\usepackage[dvipdfmx]{graphicx} % 画像を挿入するためのパッケージ
\usepackage[dvipdfmx]{color} % 色をつけるためのパッケージ
\usepackage{pxrubrica} % ルビを振るためのパッケージ
\usepackage{plext} % 漢数字の enumerate を使うためのパッケージ
\usepackage{multicol} % 複数段組を作るためのパッケージ
\setlength{\topmargin}{14mm} % 上下方向のマージン
\addtolength{\topmargin}{-1in} % 
\setlength{\oddsidemargin}{11mm} % 左右方向のマージン
\addtolength{\oddsidemargin}{-1in} % 
\setlength{\textwidth}{154mm} % B6 用
\setlength{\textheight}{108mm} % B6 用
\setlength{\headsep}{0mm} % 
\setlength{\headheight}{0mm} % 
\setlength{\topskip}{0mm} % 
\setlength{\parskip}{0pt} % 
\def\theenumi{\Kanji{enumi}} % 箇条書きのフォーマットを漢数字に変更
\parindent = 0pt % 段落下げしない
\pagestyle{empty} % すべてのページ番号を消去
% \renewcommand{\baselinestretch}{0.9} % 行間の倍率
 % B6 用テンプレート読み込み

\begin{document}
% begin header
%%%%% タイトルと作者 ここから %%%%%
\begin{minipage}[c]{0.7\hsize} % タイトルは上から 7 割
    \begin{center}
    % begin title
        {\LARGE
            恵迪小唄 % タイトルを入れる
        }
        {\small 
            (平成十九年度寮歌) % 年などを入れる
        }
    % end title
    \end{center}
\end{minipage}
\begin{minipage}[c]{0.3\hsize} % 作歌作曲は上から 3 割
    \begin{flushright} % 下寄せにする
        % begin name
        井関俊介君 作歌\\八城雄太君 作曲 % 作歌・作曲者
        % end name
    \end{flushright}
\end{minipage}
%%%%% タイトルと作者 ここまで %%%%%
% (1,2,3,4 繰り返しなし)
% end header

% begin length
\vspace{1.5em} % タイトル, 作者と歌詞の間に隙間を設ける
\newcommand{\linespace}{0.5em} % 行間の設定
\newcommand{\blocksize}{0.5\hsize} % 段組間の設定
\newcommand{\itemmargin}{3em} % 曲番の位置調整の長さ
% end length
% begin body
%%%%% 歌詞 ここから %%%%%
\begin{enumerate} % 番号の箇条書き ここから
    \setlength{\itemindent}{\itemmargin} % 曲番の位置調整
    \begin{minipage}[c]{\blocksize}
    
        \vspace{\linespace}
        \item~\\
        % 1.
        \ruby{金}{}がないのが\ruby{最初}{}の\ruby{縁}{}で\\
        \ruby{入}{}ってみたのは\ruby{良}{}いけれど\\
        すみかはボロ\ruby{屋}{}に\\
        \ruby{得体}{}の\ruby{知}{}れぬ\\
        \ruby{上}{}の\ruby{年目}{}が\ruby{一絡}{}げ ヤレ\\
        \ruby{想}{}えば\ruby{遠}{}くへ\ruby{来}{}たもんだ
        
    \end{minipage}
    \begin{minipage}[c]{\blocksize}
        
        \vspace{\linespace}
        \item~\\
        % 2.
        \ruby{大志抱}{}きて\ruby{北都}{}へ\ruby{来}{}たが\\
        \ruby{気付}{}けば\ruby{朝寝}{}に\ruby{高}{}いびき\\
        \ruby{自分}{}は\ruby{違}{}うと\ruby{言}{}ってはみたが\\
        サァ \ruby{明日}{}からがんばるぞ ヤレ\\
        \ruby{朱}{}に\ruby{交}{}われば\ruby{朱}{}くなる
        
    \end{minipage}
    \begin{minipage}[c]{\blocksize}
        
        \vspace{\linespace}
        \item~\\
        % 3.
        \ruby{酒}{}を\ruby{飲}{}み\ruby{飲}{}み\ruby{話}{}もすれば\\
        \ruby{突然}{}ドンパと\ruby{突}{}っ\ruby{張}{}り\ruby{合}{}い\\
        \ruby{時}{}には\ruby{突}{}き\ruby{上}{}げ\ruby{時}{}には\ruby{日和}{}り\\
        \ruby{奴}{}より\ruby{俺}{}の\ruby{方}{}が\ruby{上}{} ヤレ\\
        \ruby{同}{}じ\ruby{団栗}{}せいくらべ
        
    \end{minipage}
    \begin{minipage}[c]{\blocksize}
        
        \vspace{\linespace}
        \item~\\
        % 4.
        \ruby{先}{}は\ruby{長}{}いと\ruby{思}{}っていても\\
        \ruby{時間}{}の\ruby{経}{}つのは\ruby{早}{}いもの\\
        \ruby{苦楽}{}を\ruby{共}{}に\ruby{住}{}んではいたが\\
        \ruby{避}{}けては\ruby{通}{}れぬ\ruby{別}{}れ\ruby{道}{} ヤレ\\
        \ruby{縁}{}は\ruby{異}{}なもの\ruby{味}{}なもの
    
    \end{minipage}
\end{enumerate} % 番号の箇条書き ここまで
%%%%% 歌詞 ここまで %%%%%
% end body

\end{document}

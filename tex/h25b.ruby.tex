\documentclass[10pt,b5j]{tarticle} % B6 縦書き
% \documentclass[10pt,b5j]{tarticle} % B6 縦書き
\AtBeginDvi{\special{papersize=128mm,182mm}} % B6 用用紙サイズ
\usepackage{otf} % Unicode で字を入力するのに必要なパッケージ
\usepackage[size=b6j]{bxpapersize} % B6 用紙サイズを指定
\usepackage[dvipdfmx]{graphicx} % 画像を挿入するためのパッケージ
\usepackage[dvipdfmx]{color} % 色をつけるためのパッケージ
\usepackage{pxrubrica} % ルビを振るためのパッケージ
\usepackage{plext} % 漢数字の enumerate を使うためのパッケージ
\usepackage{multicol} % 複数段組を作るためのパッケージ
\setlength{\topmargin}{14mm} % 上下方向のマージン
\addtolength{\topmargin}{-1in} % 
\setlength{\oddsidemargin}{11mm} % 左右方向のマージン
\addtolength{\oddsidemargin}{-1in} % 
\setlength{\textwidth}{154mm} % B6 用
\setlength{\textheight}{108mm} % B6 用
\setlength{\headsep}{0mm} % 
\setlength{\headheight}{0mm} % 
\setlength{\topskip}{0mm} % 
\setlength{\parskip}{0pt} % 
\def\theenumi{\Kanji{enumi}} % 箇条書きのフォーマットを漢数字に変更
\parindent = 0pt % 段落下げしない
\pagestyle{empty} % すべてのページ番号を消去
% \renewcommand{\baselinestretch}{0.9} % 行間の倍率
 % B6 用テンプレート読み込み

\begin{document}
% begin header
%%%%% タイトルと作者 ここから %%%%%
\begin{minipage}[c]{0.7\hsize} % タイトルは上から 7 割
    \begin{center}
    % begin title
        {\LARGE
            二つの春 % タイトルを入れる
        }
        {\small 
            (平成二十五年度寮歌) % 年などを入れる
        }
    % end title
    \end{center}
\end{minipage}
\begin{minipage}[c]{0.3\hsize} % 作歌作曲は上から 3 割
    \begin{flushright} % 下寄せにする
        % begin name
        丸田潤君 作歌・作曲 % 作歌・作曲者
        % end name
    \end{flushright}
\end{minipage}
%%%%% タイトルと作者 ここまで %%%%%
% (1,2 繰り返しなし)
% end header

% begin length
\vspace{1.5em} % タイトル, 作者と歌詞の間に隙間を設ける
\newcommand{\linespace}{0.5em} % 行間の設定
\newcommand{\blocksize}{0.5\hsize} % 段組間の設定
\newcommand{\itemmargin}{3em} % 曲番の位置調整の長さ
% end length
% begin body
%%%%% 歌詞 ここから %%%%%
\begin{enumerate} % 番号の箇条書き ここから
    \setlength{\itemindent}{\itemmargin} % 曲番の位置調整
    \begin{minipage}[c]{\blocksize}
    
        \vspace{\linespace}
        \item~\\
        % 1.
        \ruby{降}{お}りしく\ruby{雪}{ゆき}は\ruby{終}{おわ}るを\ruby{知}{し}らず、\\
        \ruby{未}{いま}だ\ruby{窓}{まど}\ruby{下}{か}には\ruby{白銀}{はくぎん}の\ruby{町}{まち}。\\
        されど\ruby{陽光}{ようこう}は\ruby{日毎}{ひごと}に\ruby{増}{ま}して、\\
        \ruby{春}{はる}の\ruby{訪}{おとず}れを\ruby{微}{かす}かに\ruby{予感}{よかん}う。\\
        \ruby{幾許}{いくばく}もせず\ruby{別離}{べつり}の\ruby{時}{とき}は\ruby{来}{き}て、\\
        \ruby{寮}{りょう}で\ruby{過}{す}ごせし\ruby{日}{ひ}は\ruby{想}{そう}\ruby{出}{で}となる。\\
        \ruby{雪上}{せつじょう}の\ruby{足跡}{あしあと} \ruby{融}{と}け\ruby{去}{さ}り\ruby{消}{しょう}ゆるよに、\\
        \ruby{巣立}{すだ}つ\ruby{若芽}{わかめ}も\ruby{晩冬}{ばんとう}と\ruby{共}{ととも}に\ruby{去}{さ}り\ruby{行}{い}く。
        
    \end{minipage}
    \begin{minipage}[c]{\blocksize}
        
        \vspace{\linespace}
        \item~\\
        % 2.
        \ruby{残雪}{ざんせつ}\ruby{融}{と}かす\ruby{春風}{しゅんぷう}\ruby{吹}{ふ}きて、\\
        \ruby{原始}{げんし}\ruby{林}{りん}\ruby{陰}{かげ}に\ruby{萌}{もえ}ゆる\ruby{新芽}{しんめ}は\ruby{踊}{おど}る。\\
        \ruby{生命}{せいめい}の\ruby{鐘声}{しょうせい}は\ruby{北都}{ほくと}を\ruby{巡}{めぐ}り、\\
        \ruby{長}{なが}き\ruby{寒}{かん}\ruby{冬}{ふゆ}の\ruby{影}{かげ}は\ruby{消}{き}え\ruby{往}{ゆ}く。\\
        \ruby{去}{さ}りし\ruby{寮}{りょう}\ruby{友}{とも}との\ruby{月日}{つきひ}\ruby{胸}{むね}にして、\\
        \ruby{新}{あら}たなる\ruby{一}{いち}\ruby{年}{ねん}の\ruby{扉}{とびら}を\ruby{開}{ひら}く。\\
        \ruby{若}{わか}き\ruby{我}{わが}\ruby{等}{とう}の\ruby{熱}{あつ}き\ruby{血}{ち}\ruby{滾}{たぎ}らせて、\\
        ひたすらに\ruby{只}{ただ}\ruby{青春}{せいしゅん}を\ruby{歩}{あゆ}みて\ruby{行}{い}かん。
    
    \end{minipage}
\end{enumerate} % 番号の箇条書き ここまで
%%%%% 歌詞 ここまで %%%%%
% end body

\end{document}

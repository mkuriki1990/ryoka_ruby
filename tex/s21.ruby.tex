\documentclass[10pt,b5j]{tarticle} % B6 縦書き
% \documentclass[10pt,b5j]{tarticle} % B6 縦書き
\AtBeginDvi{\special{papersize=128mm,182mm}} % B6 用用紙サイズ
\usepackage{otf} % Unicode で字を入力するのに必要なパッケージ
\usepackage[size=b6j]{bxpapersize} % B6 用紙サイズを指定
\usepackage[dvipdfmx]{graphicx} % 画像を挿入するためのパッケージ
\usepackage[dvipdfmx]{color} % 色をつけるためのパッケージ
\usepackage{pxrubrica} % ルビを振るためのパッケージ
\usepackage{multicol} % 複数段組を作るためのパッケージ
\setlength{\topmargin}{14mm} % 上下方向のマージン
\addtolength{\topmargin}{-1in} % 
\setlength{\oddsidemargin}{11mm} % 左右方向のマージン
\addtolength{\oddsidemargin}{-1in} % 
\setlength{\textwidth}{154mm} % B6 用
\setlength{\textheight}{108mm} % B6 用
\setlength{\headsep}{0mm} % 
\setlength{\headheight}{0mm} % 
\setlength{\topskip}{0mm} % 
\setlength{\parskip}{0pt} % 
\def\labelenumi{\theenumi、} % 箇条書きのフォーマット
\parindent = 0pt % 段落下げしない

 % B6 用テンプレート読み込み

\begin{document}
% begin header
%%%%% タイトルと作者 ここから %%%%%
\begin{minipage}[c]{0.7\hsize} % タイトルは上から 7 割
    \begin{center}
    % begin title
        {\LARGE
            時潮の波の % タイトルを入れる
        }
        {\small 
            (昭和21年寮歌) % 年などを入れる
        }
    % end title
    \end{center}
\end{minipage}
\begin{minipage}[c]{0.3\hsize} % 作歌作曲は上から 3 割
    \begin{flushright} % 下寄せにする
        % begin name
        渋谷富業君 作歌\\寺井幸夫君 作曲 % 作歌・作曲者
        % end name
    \end{flushright}
\end{minipage}
%%%%% タイトルと作者 ここまで %%%%%
% (序,1,結 繰り返しなし)
% end header

% begin body
\vspace{1.5em} % タイトル, 作者と歌詞の間に隙間を設ける
\newcommand{\linespace}{0.5em} % 行間の設定
\newcommand{\blocksize}{0.5\hsize} % 段組間の設定
%%%%% 歌詞 ここから %%%%%
% begin lilycs
\begin{enumerate} % 番号の箇条書き ここから
    \begin{minipage}[c]{\blocksize}
    
        \vspace{\linespace}
        \item
        % \ruby{序}{}.
        \ruby{厳}{}しかる\ruby{道}{}に\ruby{仕}{}へて\\
        \ruby{限}{}ある\ruby{玉緒惜}{}しむ\\
        げにさあれ\ruby{深}{}き\ruby{因縁}{}の\\
        \ruby{魂}{}ゆする\ruby{生命}{}の\ruby{饗宴}{}\\
        \ruby{汲}{}まざらめや\ruby{残}{}の\ruby{月}{}に\\
        \ruby{旅}{}の\ruby{朝早}{}くは\ruby{明}{}けぬ
        
        \vspace{\linespace}
        \item
        % 1.
        \ruby{時潮}{}の\ruby{波}{}の\ruby{寄}{}する\ruby{間}{}を\\
        \ruby{久遠}{}の\ruby{岸}{}に\ruby{佇}{}みて\\
        \ruby{不壊}{}の\ruby{真珠}{}を\ruby{漁}{}りする\\
        \ruby{嗚呼三星霜}{}の\ruby{光栄}{}よ\\
        \ruby{緑}{}の\ruby{星}{}を\ruby{夢}{}む\ruby{時}{}\\
        \ruby{疎梢}{}を\ruby{払}{}ふ\ruby{天籟}{}は\\
        \ruby{秘誦}{}の\ruby{啓示語}{}るなり
        
        \vspace{\linespace}
        \item
        % 2.
        \ruby{孤窓}{}に\ruby{流}{}る\ruby{星屑}{}に\\
        \ruby{無辺}{}の\ruby{調律訪}{}へば\\
        \ruby{測}{}りも\ruby{知}{}らに\ruby{底}{}つひゆ\\
        \ruby{言}{}の\ruby{葉洩}{}れて\ruby{伏}{}し\ruby{祈}{}る\\
        \ruby{奇}{}しく\ruby{貴}{}き\ruby{生命}{}をば\\
        \ruby{友情}{}を\ruby{讃}{}ふ\ruby{歌声}{}の\\
        \ruby{溶}{}け\ruby{行}{}く\ruby{方}{}に\ruby{馳}{}するかな
        
        \vspace{\linespace}
        \item
        % 3.
        \ruby{朽葉}{}ゆらぎて\ruby{湧}{}き\ruby{出}{}づる\\
        \ruby{楡}{}の\ruby{林}{}の\ruby{真清水}{}に\\
        \ruby{己}{}を\ruby{責}{}めて\ruby{泣}{}く\ruby{友}{}の\\
        \ruby{孤杖}{}を\ruby{運}{}ぶ\ruby{逍遙}{}や\\
        \ruby{遠}{}き\ruby{誓}{}ひの\ruby{日}{}を\ruby{偲}{}び\\
        \ruby{虚}{}しき\ruby{春}{}に\ruby{嘯}{}けば\\
        \ruby{淡}{}れし\ruby{影}{}の\ruby{寂寥}{}よ
        
        \vspace{\linespace}
        \item
        % 4.
        \ruby{宿命}{}の\ruby{道}{}を\ruby{行}{}く\ruby{身}{}にも\\
        \ruby{友}{}を\ruby{誇}{}らん\ruby{花筵}{}\\
        \ruby{銀燭頬涙}{}を\ruby{照}{}らす\ruby{宵}{}\\
        \ruby{沈黙}{}に\ruby{語}{}る\ruby{歓喜}{}よ\\
        \ruby{心}{}を\ruby{交}{}し\ruby{思}{}ひ\ruby{酌}{}み\\
        \ruby{団欒}{}にふるふ\ruby{共鳴}{}は\\
        \ruby{胸}{}の\ruby{小琴}{}を\ruby{掻}{}き\ruby{鳴}{}らす
        
        \vspace{\linespace}
        \item
        % 5.
        \ruby{北斗頭上}{}に\ruby{影冴}{}えて\\
        \ruby{神秘}{}の\ruby{息}{}に\ruby{吹}{}かれつつ\\
        \ruby{肩組}{}み\ruby{歌}{}ふ\ruby{旅}{}の\ruby{子}{}を\\
        \ruby{染}{}むる\ruby{伝統}{}の\ruby{篝火}{}よ\\
        \ruby{暮}{}るるに\ruby{早}{}き\ruby{青春}{}の\ruby{日}{}の\\
        \ruby{追懐}{}を\ruby{込}{}むる\ruby{此}{}の\ruby{盃}{}を\\
        \ruby{汲}{}まん\ruby{今宵}{}の\ruby{記念祭}{}
        
        \vspace{\linespace}
        \item
        % \ruby{結}{}.
        \ruby{近}{}きかな\ruby{楡陵}{}を\ruby{去}{}る\ruby{日}{}は\\
        \ruby{還}{}り\ruby{来}{}ぬ\ruby{足跡愛}{}しみて\\
        ひたぶると\ruby{打笑}{}む\ruby{時}{}ぞ\\
        \ruby{求}{}めつつ\ruby{得}{}べからざりし\\
        \ruby{秀邃}{}しき\ruby{真理}{}の\ruby{道}{}は\\
        はろかなり\ruby{我等}{}が\ruby{前途}{}\\
        \ruby{進}{}まざらめや
    
    \end{minipage}
\end{enumerate} % 番号の箇条書き ここまで
% end lilycs
%%%%% 歌詞 ここまで %%%%%
% end body

\end{document}

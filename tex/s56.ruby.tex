\documentclass[10pt,b5j]{tarticle} % B6 縦書き
% \documentclass[10pt,b5j]{tarticle} % B6 縦書き
\AtBeginDvi{\special{papersize=128mm,182mm}} % B6 用用紙サイズ
\usepackage{otf} % Unicode で字を入力するのに必要なパッケージ
\usepackage[size=b6j]{bxpapersize} % B6 用紙サイズを指定
\usepackage[dvipdfmx]{graphicx} % 画像を挿入するためのパッケージ
\usepackage[dvipdfmx]{color} % 色をつけるためのパッケージ
\usepackage{pxrubrica} % ルビを振るためのパッケージ
\usepackage{multicol} % 複数段組を作るためのパッケージ
\setlength{\topmargin}{14mm} % 上下方向のマージン
\addtolength{\topmargin}{-1in} % 
\setlength{\oddsidemargin}{11mm} % 左右方向のマージン
\addtolength{\oddsidemargin}{-1in} % 
\setlength{\textwidth}{154mm} % B6 用
\setlength{\textheight}{108mm} % B6 用
\setlength{\headsep}{0mm} % 
\setlength{\headheight}{0mm} % 
\setlength{\topskip}{0mm} % 
\setlength{\parskip}{0pt} % 
\def\labelenumi{\theenumi、} % 箇条書きのフォーマット
\parindent = 0pt % 段落下げしない

 % B6 用テンプレート読み込み

\begin{document}
% begin header
%%%%% タイトルと作者 ここから %%%%%
\begin{minipage}[c]{0.7\hsize} % タイトルは上から 7 割
    \begin{center}
    % begin title
        {\LARGE
            汝と我の % タイトルを入れる
        }
        {\small 
            (昭和五十六年寮歌) % 年などを入れる
        }
    % end title
    \end{center}
\end{minipage}
\begin{minipage}[c]{0.3\hsize} % 作歌作曲は上から 3 割
    \begin{flushright} % 下寄せにする
        % begin name
        山根誠君 作歌\\長谷部健君 作曲 % 作歌・作曲者
        % end name
    \end{flushright}
\end{minipage}
%%%%% タイトルと作者 ここまで %%%%%
% (1,3 繰り返しなし)
% end header

% begin body
\vspace{1.5em} % タイトル, 作者と歌詞の間に隙間を設ける
\newcommand{\linespace}{0.5em} % 行間の設定
\newcommand{\blocksize}{0.5\hsize} % 段組間の設定
%%%%% 歌詞 ここから %%%%%
% begin lilycs
\begin{enumerate} % 番号の箇条書き ここから
    \begin{minipage}[c]{\blocksize}
    
        \vspace{\linespace}
        \item
        % 1.
        よすがなき\ruby{姿}{}も\ruby{見}{}せぬ\ruby{郭公}{}を\\
        \ruby{探}{}せしは\ruby{誰}{}ぞ\ruby{汝}{}と\ruby{我}{}の\ruby{瞳}{}なり\\
        \ruby{草}{}いきれ\ruby{燃}{}えたつ\ruby{野}{}にて\ruby{戯}{}れぬ\\
        \ruby{獣}{}らは\ruby{誰}{}ぞ\ruby{汝}{}と\ruby{我}{}の\ruby{姿}{}なり\\
        \ruby{原始林}{}と\ruby{古屋}{}を\ruby{覆}{}いたる\\
        \ruby{邪}{}なものめぐる\ruby{世}{}に\\
        \ruby{正義}{}の\ruby{想}{}い\ruby{何処}{}にか\\
        \ruby{汝}{}と\ruby{我}{}の\ruby{胸}{}にあり
        
        \vspace{\linespace}
        \item
        % 2.
        \ruby{轟}{}ける\ruby{荒磯}{}の\ruby{波}{}のただ\ruby{中}{}を\\
        \ruby{漕}{}ぎゆくは\ruby{誰}{}ぞ\ruby{汝}{}と\ruby{我}{}の\ruby{腕}{}なり\\
        アカシアの\ruby{狭霧漂}{}う\ruby{道辻}{}を\\
        \ruby{疾}{}けゆくは\ruby{誰}{}ぞ\ruby{汝}{}と\ruby{我}{}の\ruby{跫}{}なり\\
        \ruby{移}{}ろい\ruby{巡}{}る\ruby{天地}{}を\\
        \ruby{己}{}が\ruby{父}{}とし\ruby{母}{}として\\
        のびゆく\ruby{命何処}{}にか\\
        \ruby{汝}{}と\ruby{我}{}の\ruby{胸}{}にあり
        
        \vspace{\linespace}
        \item
        % 3.
        \ruby{降}{}りつもる\ruby{雪}{}に\ruby{太古}{}の\ruby{巨象}{}を\\
        \ruby{描}{}きしは\ruby{誰}{}ぞ\ruby{汝}{}と\ruby{我}{}の\ruby{感傷}{}なり\\
        \ruby{夜}{}もすがら\ruby{思}{}い\ruby{乱}{}れる\ruby{若人}{}を\\
        \ruby{見}{}つめしは\ruby{誰}{}ぞ\\
        \ruby{汝}{}と\ruby{我}{}の\ruby{恵迪}{}なり\\
        \ruby{天宙駆}{}ける\ruby{参星}{}の\\
        \ruby{幽}{}けき\ruby{光}{}を\ruby{仰}{}ぎ\ruby{見}{}て\\
        \ruby{語}{}りしことば\ruby{何処}{}にか\\
        \ruby{汝}{}と\ruby{我}{}の\ruby{胸}{}にあり
    
    \end{minipage}
\end{enumerate} % 番号の箇条書き ここまで
% end lilycs
%%%%% 歌詞 ここまで %%%%%
% end body

\end{document}

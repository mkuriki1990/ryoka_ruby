\documentclass[10pt,b5j]{tarticle} % B6 縦書き
% \documentclass[10pt,b5j]{tarticle} % B6 縦書き
\AtBeginDvi{\special{papersize=128mm,182mm}} % B6 用用紙サイズ
\usepackage{otf} % Unicode で字を入力するのに必要なパッケージ
\usepackage[size=b6j]{bxpapersize} % B6 用紙サイズを指定
\usepackage[dvipdfmx]{graphicx} % 画像を挿入するためのパッケージ
\usepackage[dvipdfmx]{color} % 色をつけるためのパッケージ
\usepackage{pxrubrica} % ルビを振るためのパッケージ
\usepackage{multicol} % 複数段組を作るためのパッケージ
\setlength{\topmargin}{14mm} % 上下方向のマージン
\addtolength{\topmargin}{-1in} % 
\setlength{\oddsidemargin}{11mm} % 左右方向のマージン
\addtolength{\oddsidemargin}{-1in} % 
\setlength{\textwidth}{154mm} % B6 用
\setlength{\textheight}{108mm} % B6 用
\setlength{\headsep}{0mm} % 
\setlength{\headheight}{0mm} % 
\setlength{\topskip}{0mm} % 
\setlength{\parskip}{0pt} % 
\def\labelenumi{\theenumi、} % 箇条書きのフォーマット
\parindent = 0pt % 段落下げしない

 % B6 用テンプレート読み込み

\begin{document}
% begin header
%%%%% タイトルと作者 ここから %%%%%
\begin{minipage}[c]{0.7\hsize} % タイトルは上から 7 割
    \begin{center}
    % begin title
        {\LARGE
            汝と我の % タイトルを入れる
        }
        {\small 
            (昭和五十六年寮歌) % 年などを入れる
        }
    % end title
    \end{center}
\end{minipage}
\begin{minipage}[c]{0.3\hsize} % 作歌作曲は上から 3 割
    \begin{flushright} % 下寄せにする
        % begin name
        山根誠君 作歌\\長谷部健君 作曲 % 作歌・作曲者
        % end name
    \end{flushright}
\end{minipage}
%%%%% タイトルと作者 ここまで %%%%%
% (1,3 繰り返しなし)
% end header

% begin length
\vspace{1.5em} % タイトル, 作者と歌詞の間に隙間を設ける
\newcommand{\linespace}{0.5em} % 行間の設定
\newcommand{\blocksize}{0.5\hsize} % 段組間の設定
\newcommand{\itemmargin}{3em} % 曲番の位置調整の長さ
% end length
% begin body
%%%%% 歌詞 ここから %%%%%
\begin{enumerate} % 番号の箇条書き ここから
    \setlength{\itemindent}{\itemmargin} % 曲番の位置調整
    \begin{minipage}[c]{\blocksize}
    
        \vspace{\linespace}
        \item~\\
        % 1.
        よすがなき\ruby{姿}{すがた}も\ruby{見}{み}せぬ\ruby{郭公}{かっこう}を\\
        \ruby{探}{さが}せしは\ruby{誰}{だれ}ぞ\ruby{汝}{なんじ}と\ruby{我}{わが}の\ruby{瞳}{ひとみ}なり\\
        \ruby{草}{くさ}いきれ\ruby{燃}{も}えたつ\ruby{野}{の}にて\ruby{戯}{たわむ}れぬ\\
        \ruby{獣}{しし}らは\ruby{誰}{だれ}ぞ\ruby{汝}{なんじ}と\ruby{我}{わが}の\ruby{姿}{すがた}なり\\
        \ruby{原始}{げんし}\ruby{林}{りん}と\ruby{古屋}{ふるや}を\ruby{覆}{おお}いたる\\
        \ruby{邪}{よこしま}なものめぐる\ruby{世}{よ}に\\
        \ruby{正義}{せいぎ}の\ruby{想}{おも}い\ruby{何処}{どこ}にか\\
        \ruby{汝}{なんじ}と\ruby{我}{わが}の\ruby{胸}{むね}にあり
        
    \end{minipage}
    \begin{minipage}[c]{\blocksize}
        
        \vspace{\linespace}
        \item~\\
        % 2.
        \ruby{轟}{とどろき}ける\ruby{荒磯}{ありそ}の\ruby{波}{なみ}のただ\ruby{中}{ただなか}を\\
        \ruby{漕}{こ}ぎゆくは\ruby{誰}{だれ}ぞ\ruby{汝}{なんじ}と\ruby{我}{わが}の\ruby{腕}{うで}なり\\
        アカシアの\ruby{狭霧}{さぎり}\ruby{漂}{ただよ}う\ruby{道}{みち}\ruby{辻}{つじ}を\\
        \ruby{疾}{}けゆくは\ruby{誰}{だれ}ぞ\ruby{汝}{なんじ}と\ruby{我}{わが}の\ruby{跫}{}なり\\
        \ruby{移}{うつ}ろい\ruby{巡}{めぐ}る\ruby{天地}{てんち}を\\
        \ruby{己}{おのれ}が\ruby{父}{ちち}とし\ruby{母}{はは}として\\
        のびゆく\ruby{命}{いのち}\ruby{何処}{どこ}にか\\
        \ruby{汝}{なんじ}と\ruby{我}{わが}の\ruby{胸}{むね}にあり
        
    \end{minipage}
    \begin{minipage}[c]{\blocksize}
        
        \vspace{\linespace}
        \item~\\
        % 3.
        \ruby{降}{お}りつもる\ruby{雪}{ゆき}に\ruby{太古}{たいこ}の\ruby{巨象}{}を\\
        \ruby{描}{えが}きしは\ruby{誰}{だれ}ぞ\ruby{汝}{なんじ}と\ruby{我}{わが}の\ruby{感傷}{かんしょう}なり\\
        \ruby{夜}{よ}もすがら\ruby{思}{おも}い\ruby{乱}{みだ}れる\ruby{若人}{わこうど}を\\
        \ruby{見}{み}つめしは\ruby{誰}{だれ}ぞ\\
        \ruby{汝}{なんじ}と\ruby{我}{わが}の\ruby{恵}{めぐみ}\ruby{迪}{すすむ}なり\\
        \ruby{天}{てん}\ruby{宙}{ちゅう}\ruby{駆}{か}ける\ruby{参}{さん}\ruby{星}{ぼし}の\\
        \ruby{幽}{かそけ}けき\ruby{光}{こう}を\ruby{仰}{あお}ぎ\ruby{見}{み}て\\
        \ruby{語}{かた}りしことば\ruby{何処}{どこ}にか\\
        \ruby{汝}{なんじ}と\ruby{我}{わが}の\ruby{胸}{むね}にあり
    
    \end{minipage}
\end{enumerate} % 番号の箇条書き ここまで
%%%%% 歌詞 ここまで %%%%%
% end body

\end{document}

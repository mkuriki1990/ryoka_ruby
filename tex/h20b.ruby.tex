\documentclass[10pt,b5j]{tarticle} % B6 縦書き
% \documentclass[10pt,b5j]{tarticle} % B6 縦書き
\AtBeginDvi{\special{papersize=128mm,182mm}} % B6 用用紙サイズ
\usepackage{otf} % Unicode で字を入力するのに必要なパッケージ
\usepackage[size=b6j]{bxpapersize} % B6 用紙サイズを指定
\usepackage[dvipdfmx]{graphicx} % 画像を挿入するためのパッケージ
\usepackage[dvipdfmx]{color} % 色をつけるためのパッケージ
\usepackage{pxrubrica} % ルビを振るためのパッケージ
\usepackage{multicol} % 複数段組を作るためのパッケージ
\setlength{\topmargin}{14mm} % 上下方向のマージン
\addtolength{\topmargin}{-1in} % 
\setlength{\oddsidemargin}{11mm} % 左右方向のマージン
\addtolength{\oddsidemargin}{-1in} % 
\setlength{\textwidth}{154mm} % B6 用
\setlength{\textheight}{108mm} % B6 用
\setlength{\headsep}{0mm} % 
\setlength{\headheight}{0mm} % 
\setlength{\topskip}{0mm} % 
\setlength{\parskip}{0pt} % 
\def\labelenumi{\theenumi、} % 箇条書きのフォーマット
\parindent = 0pt % 段落下げしない

 % B6 用テンプレート読み込み

\begin{document}
% begin header
%%%%% タイトルと作者 ここから %%%%%
\begin{minipage}[c]{0.7\hsize} % タイトルは上から 7 割
    \begin{center}
    % begin title
        {\LARGE
            雲海貫く % タイトルを入れる
        }
        {\small 
            (平成二十年第百回記念祭歌) % 年などを入れる
        }
    % end title
    \end{center}
\end{minipage}
\begin{minipage}[c]{0.3\hsize} % 作歌作曲は上から 3 割
    \begin{flushright} % 下寄せにする
        % begin name
        石井翔君 作歌\\木川明音君 作曲 % 作歌・作曲者
        % end name
    \end{flushright}
\end{minipage}
%%%%% タイトルと作者 ここまで %%%%%
% (1,2,3,4 了なし繰り返しあり)
% end header

% begin body
\vspace{1.5em} % タイトル, 作者と歌詞の間に隙間を設ける
\newcommand{\linespace}{0.5em} % 行間の設定
\newcommand{\blocksize}{0.5\hsize} % 段組間の設定
%%%%% 歌詞 ここから %%%%%
% begin lilycs
\begin{enumerate} % 番号の箇条書き ここから
    \begin{minipage}[c]{\blocksize}
    
        \vspace{\linespace}
        \item
        % 1.
        \ruby{雲海貫}{}く\ruby{泰山}{}に\\
        \ruby{伍山}{}を\ruby{覇}{}せと\ruby{吼}{}える\ruby{熊}{}\\
        \ruby{俗野}{}に\ruby{満}{}てる\ruby{四面}{}の\ruby{楚歌}{}を\\
        \ruby{真理}{}のたけびで\ruby{吹}{}き\ruby{飛}{}ばす\\
        マツリダマツリダ\\
        マツリダヒグマ
        
        \vspace{\linespace}
        \item
        % 2.
        \ruby{混迷尽}{}きぬ\ruby{泥濘}{}に\\
        \ruby{曲学阿世}{}くだく\ruby{虎}{}\\
        \ruby{世}{}をまどわす\ruby{混沌}{}ぬえを\\
        \ruby{真理}{}の\ruby{瞳}{}で\ruby{睥睨}{}す\\
        マツリダマツリダ\\
        マツリダモウコ
        
        \vspace{\linespace}
        \item
        % 3.
        \ruby{濁流荒}{}ぶる\ruby{大飛泉}{}\\
        \ruby{己身一}{}つの\ruby{六六魚}{}\\
        \ruby{時}{}の\ruby{趨勢物}{}ともにせず\\
        \ruby{龍}{}に\ruby{転}{}ぜと\ruby{登}{}りゆく\\
        マツリダマツリダ\\
        マツリダオロチ
        
        \vspace{\linespace}
        \item
        % 4.
        \ruby{流}{}れに\ruby{流}{}れ\ruby{一百年}{}\\
        \ruby{祭}{}りに\ruby{祭}{}って\ruby{一世紀}{}\\
        \ruby{真理}{}を\ruby{求}{}む\ruby{若学徒}{}\\
        \ruby{今}{}ぞ\ruby{狂}{}いて\ruby{大宴}{}\\
        マツリダマツリダ\\
        マツリダゴッホ
    
    \end{minipage}
\end{enumerate} % 番号の箇条書き ここまで
% end lilycs
%%%%% 歌詞 ここまで %%%%%
% end body

\end{document}

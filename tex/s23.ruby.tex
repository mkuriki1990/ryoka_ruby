\documentclass[10pt,b5j]{tarticle} % B6 縦書き
% \documentclass[10pt,b5j]{tarticle} % B6 縦書き
\AtBeginDvi{\special{papersize=128mm,182mm}} % B6 用用紙サイズ
\usepackage{otf} % Unicode で字を入力するのに必要なパッケージ
\usepackage[size=b6j]{bxpapersize} % B6 用紙サイズを指定
\usepackage[dvipdfmx]{graphicx} % 画像を挿入するためのパッケージ
\usepackage[dvipdfmx]{color} % 色をつけるためのパッケージ
\usepackage{pxrubrica} % ルビを振るためのパッケージ
\usepackage{multicol} % 複数段組を作るためのパッケージ
\setlength{\topmargin}{14mm} % 上下方向のマージン
\addtolength{\topmargin}{-1in} % 
\setlength{\oddsidemargin}{11mm} % 左右方向のマージン
\addtolength{\oddsidemargin}{-1in} % 
\setlength{\textwidth}{154mm} % B6 用
\setlength{\textheight}{108mm} % B6 用
\setlength{\headsep}{0mm} % 
\setlength{\headheight}{0mm} % 
\setlength{\topskip}{0mm} % 
\setlength{\parskip}{0pt} % 
\def\labelenumi{\theenumi、} % 箇条書きのフォーマット
\parindent = 0pt % 段落下げしない

 % B6 用テンプレート読み込み

\begin{document}
% begin header
%%%%% タイトルと作者 ここから %%%%%
\begin{minipage}[c]{0.7\hsize} % タイトルは上から 7 割
    \begin{center}
    % begin title
        {\LARGE
            饗宴の杯に % タイトルを入れる
        }
        {\small 
            (昭和二十三年寮歌) % 年などを入れる
        }
    % end title
    \end{center}
\end{minipage}
\begin{minipage}[c]{0.3\hsize} % 作歌作曲は上から 3 割
    \begin{flushright} % 下寄せにする
        % begin name
        中坪清八君 作歌\\堀井洵君 作曲 % 作歌・作曲者
        % end name
    \end{flushright}
\end{minipage}
%%%%% タイトルと作者 ここまで %%%%%
% (1,6 了あり)
% end header

% begin length
\vspace{1.5em} % タイトル, 作者と歌詞の間に隙間を設ける
\newcommand{\linespace}{0.5em} % 行間の設定
\newcommand{\blocksize}{0.5\hsize} % 段組間の設定
\newcommand{\itemmargin}{6em} % 曲番の位置調整の長さ
% end length
% begin body
%%%%% 歌詞 ここから %%%%%
\begin{enumerate} % 番号の箇条書き ここから
    \setlength{\itemindent}{\itemmargin} % 曲番の位置調整
    \begin{minipage}[c]{\blocksize}
    
        \vspace{\linespace}
        \item~\\
        % 1.
        \ruby{饗宴}{}の\ruby{杯}{}に\ruby{淡}{}れゆく\\
        \ruby{手稲}{}の\ruby{峰}{}に\ruby{今}{}しばし\\
        \ruby{追憶止}{}めて\ruby{涙}{}する\\
        \ruby{逝}{}く\ruby{水}{}はやき\ruby{三春秋}{}の\\
        \ruby{絵巻}{}はやがて\ruby{尽}{}きざらん\\
        \ruby{優}{}しき\ruby{薫香遺}{}しつつ
        
        \vspace{\linespace}
        \item~\\
        % 2.
        \ruby{真理}{}の\ruby{道}{}の\ruby{彷徨}{}に\\
        \ruby{遊子}{}は\ruby{尋}{}めぬ\ruby{人性}{}を\\
        \ruby{真紅}{}に\ruby{輝}{}く\ruby{森蔭}{}に\\
        \ruby{榾火廻}{}りて\ruby{歌}{}へども\\
        \ruby{琥珀}{}の\ruby{酒}{}を\ruby{酌}{}みしかど\\
        \ruby{贏}{}しものは\ruby{何}{}ならん
        
        \vspace{\linespace}
        \item~\\
        % 3.
        \ruby{原始林}{}の\ruby{濃緑}{}のまどろみに\\
        \ruby{高夢}{}は\ruby{結}{}びぬ\ruby{先人}{}の\\
        \ruby{遺訓}{}の\ruby{蔭}{}に\ruby{泪}{}あり\\
        \ruby{孤雁一}{}たび\ruby{大地}{}に\ruby{啼}{}きて\\
        \ruby{驚}{}き\ruby{醒}{}むる\ruby{邯鄲}{}の\\
        \ruby{草野}{}に\ruby{夕陽}{}は\ruby{既}{}に\ruby{没}{}つ
        
        \vspace{\linespace}
        \item~\\
        % 4.
        \ruby{秋}{}の\ruby{哀愁}{}は\ruby{旅}{}の\ruby{子}{}に\\
        ひとしほ\ruby{沁}{}みる\ruby{夜半}{}の\ruby{月}{}\\
        \ruby{悲恋}{}の\ruby{苦悩胸}{}に\ruby{秘}{}め\\
        \ruby{北斗}{}の\ruby{光影}{}に\ruby{嘯}{}けば\\
        \ruby{若}{}き\ruby{情熱}{}の\ruby{高鳴}{}りて\\
        \ruby{凋落}{}の\ruby{世}{}に\ruby{響}{}くなり
        
        \vspace{\linespace}
        \item~\\
        % 5.
        \ruby{狂}{}ふ\ruby{吹雪}{}に\ruby{我}{}が\ruby{思索}{}\\
        \ruby{託}{}して\ruby{進}{}む\ruby{三百}{}の\\
        \ruby{児等}{}の\ruby{生命}{}はみはるかす\\
        \ruby{北溟}{}の\ruby{曠野}{}にこだまして\\
        \ruby{東}{}の\ruby{空}{}は\ruby{暁紅}{}に\ruby{染}{}み\\
        \ruby{高}{}き\ruby{理想}{}の\ruby{旭日}{}は\ruby{出}{}でぬ
        
        \vspace{\linespace}
        \item~\\
        % 6.
        \ruby{楡}{}の\ruby{鐘声}{}に\ruby{逝}{}く\ruby{青春}{}の\\
        \ruby{神秘}{}を\ruby{解}{}かん\ruby{花莚}{}\\
        \ruby{朝}{}はろけき\ruby{旅}{}を\ruby{行}{}く\\
        \ruby{郭公鳥}{}よ\ruby{永遠}{}に\\
        \ruby{黒百合咲}{}ける\ruby{石狩}{}の\\
        \ruby{汝}{}が\ruby{故郷}{}を\ruby{憶}{}えよや
    
    \end{minipage}
\end{enumerate} % 番号の箇条書き ここまで
%%%%% 歌詞 ここまで %%%%%
% end body

\end{document}

\documentclass[10pt,b5j]{tarticle} % B6 縦書き
% \documentclass[10pt,b5j]{tarticle} % B6 縦書き
\AtBeginDvi{\special{papersize=128mm,182mm}} % B6 用用紙サイズ
\usepackage{otf} % Unicode で字を入力するのに必要なパッケージ
\usepackage[size=b6j]{bxpapersize} % B6 用紙サイズを指定
\usepackage[dvipdfmx]{graphicx} % 画像を挿入するためのパッケージ
\usepackage[dvipdfmx]{color} % 色をつけるためのパッケージ
\usepackage{pxrubrica} % ルビを振るためのパッケージ
\usepackage{multicol} % 複数段組を作るためのパッケージ
\setlength{\topmargin}{14mm} % 上下方向のマージン
\addtolength{\topmargin}{-1in} % 
\setlength{\oddsidemargin}{11mm} % 左右方向のマージン
\addtolength{\oddsidemargin}{-1in} % 
\setlength{\textwidth}{154mm} % B6 用
\setlength{\textheight}{108mm} % B6 用
\setlength{\headsep}{0mm} % 
\setlength{\headheight}{0mm} % 
\setlength{\topskip}{0mm} % 
\setlength{\parskip}{0pt} % 
\def\labelenumi{\theenumi、} % 箇条書きのフォーマット
\parindent = 0pt % 段落下げしない

 % B6 用テンプレート読み込み

\begin{document}
% begin header
%%%%% タイトルと作者 ここから %%%%%
\begin{minipage}[c]{0.7\hsize} % タイトルは上から 7 割
    \begin{center}
    % begin title
        {\LARGE
            饗宴の杯に % タイトルを入れる
        }
        {\small 
            (昭和二十三年寮歌) % 年などを入れる
        }
    % end title
    \end{center}
\end{minipage}
\begin{minipage}[c]{0.3\hsize} % 作歌作曲は上から 3 割
    \begin{flushright} % 下寄せにする
        % begin name
        中坪清八君 作歌\\堀井洵君 作曲 % 作歌・作曲者
        % end name
    \end{flushright}
\end{minipage}
%%%%% タイトルと作者 ここまで %%%%%
% (1,6 了あり)
% end header

% begin length
\vspace{1.5em} % タイトル, 作者と歌詞の間に隙間を設ける
\newcommand{\linespace}{0.5em} % 行間の設定
\newcommand{\blocksize}{0.5\hsize} % 段組間の設定
\newcommand{\itemmargin}{3em} % 曲番の位置調整の長さ
% end length
% begin body
%%%%% 歌詞 ここから %%%%%
\begin{enumerate} % 番号の箇条書き ここから
    \setlength{\itemindent}{\itemmargin} % 曲番の位置調整
    \begin{minipage}[c]{\blocksize}
    
        \vspace{\linespace}
        \item~\\
        % 1.
        \ruby{饗宴}{きょうえん}の\ruby{杯}{はい}に\ruby{淡}{あわ}れゆく\\
        \ruby{手稲}{ていね}の\ruby{峰}{みね}に\ruby{今}{いま}しばし\\
        \ruby{追憶}{ついおく}\ruby{止}{と}めて\ruby{涙}{なみだ}する\\
        \ruby{逝}{ゆ}く\ruby{水}{みず}はやき\ruby{三}{さん}\ruby{春秋}{しゅんじゅう}の\\
        \ruby{絵巻}{えまき}はやがて\ruby{尽}{つ}きざらん\\
        \ruby{優}{やさ}しき\ruby{薫香}{くんこう}\ruby{遺}{のこ}しつつ
        
    \end{minipage}
    \begin{minipage}[c]{\blocksize}
        
        \vspace{\linespace}
        \item~\\
        % 2.
        \ruby{真理}{しんり}の\ruby{道}{みち}の\ruby{彷徨}{ほうこう}に\\
        \ruby{遊子}{ゆうし}は\ruby{尋}{ひろ}めぬ\ruby{人性}{じんせい}を\\
        \ruby{真紅}{しんく}に\ruby{輝}{かがや}く\ruby{森}{もり}\ruby{蔭}{かげ}に\\
        \ruby{榾}{ほた}\ruby{火}{ひ}\ruby{廻}{まわ}りて\ruby{歌}{うた}へども\\
        \ruby{琥珀}{こはく}の\ruby{酒}{さけ}を\ruby{酌}{く}みしかど\\
        \ruby{贏}{}しものは\ruby{何}{なに}ならん
        
    \end{minipage}
    \begin{minipage}[c]{\blocksize}
        
        \vspace{\linespace}
        \item~\\
        % 3.
        \ruby{原始}{げんし}\ruby{林}{りん}の\ruby{濃緑}{こみどり}のまどろみに\\
        \ruby{高}{こう}\ruby{夢}{ゆめ}は\ruby{結}{むす}びぬ\ruby{先人}{せんじん}の\\
        \ruby{遺訓}{いくん}の\ruby{蔭}{かげ}に\ruby{泪}{なみだ}あり\\
        \ruby{孤雁}{こがん}\ruby{一}{いち}たび\ruby{大地}{だいち}に\ruby{啼}{な}きて\\
        \ruby{驚}{おどろ}き\ruby{醒}{}むる\ruby{邯鄲}{かんたん}の\\
        \ruby{草野}{くさの}に\ruby{夕陽}{ゆうひ}は\ruby{既}{すで}に\ruby{没}{ぼつ}つ
        
    \end{minipage}
    \begin{minipage}[c]{\blocksize}
        
        \vspace{\linespace}
        \item~\\
        % 4.
        \ruby{秋}{あき}の\ruby{哀愁}{あいしゅう}は\ruby{旅}{たび}の\ruby{子}{こ}に\\
        ひとしほ\ruby{沁}{し}みる\ruby{夜半}{やはん}の\ruby{月}{つき}\\
        \ruby{悲恋}{ひれん}の\ruby{苦悩}{くのう}\ruby{胸}{むね}に\ruby{秘}{ひ}め\\
        \ruby{北斗}{ほくと}の\ruby{光}{ひかり}\ruby{影}{かげ}に\ruby{嘯}{うそぶ}けば\\
        \ruby{若}{わか}き\ruby{情熱}{じょうねつ}の\ruby{高}{}\ruby{鳴}{たかな}りて\\
        \ruby{凋落}{ちょうらく}の\ruby{世}{よ}に\ruby{響}{ひび}くなり
        
    \end{minipage}
    \begin{minipage}[c]{\blocksize}
        
        \vspace{\linespace}
        \item~\\
        % 5.
        \ruby{狂}{きょう}ふ\ruby{吹雪}{ふぶき}に\ruby{我}{わ}が\ruby{思索}{しさく}\\
        \ruby{託}{たく}して\ruby{進}{すす}む\ruby{三}{さん}\ruby{百}{ひゃく}の\\
        \ruby{児}{こ}\ruby{等}{とう}の\ruby{生命}{せいめい}はみはるかす\\
        \ruby{北}{きた}\ruby{溟}{}の\ruby{曠野}{あらの}にこだまして\\
        \ruby{東}{ひがし}の\ruby{空}{そら}は\ruby{暁紅}{ぎょうこう}に\ruby{染}{し}み\\
        \ruby{高}{たか}き\ruby{理想}{りそう}の\ruby{旭日}{きょくじつ}は\ruby{出}{で}でぬ
        
    \end{minipage}
    \begin{minipage}[c]{\blocksize}
        
        \vspace{\linespace}
        \item~\\
        % 6.
        \ruby{楡}{にれ}の\ruby{鐘声}{しょうせい}に\ruby{逝}{ゆ}く\ruby{青春}{せいしゅん}の\\
        \ruby{神秘}{しんぴ}を\ruby{解}{と}かん\ruby{花}{はな}\ruby{莚}{}\\
        \ruby{朝}{あさ}はろけき\ruby{旅}{たび}を\ruby{行}{い}く\\
        \ruby{郭公}{かっこう}\ruby{鳥}{とり}よ\ruby{永遠}{えいえん}に\\
        \ruby{黒}{くろ}\ruby{百}{ひゃく}\ruby{合}{ごう}\ruby{咲}{さき}ける\ruby{石狩}{いしかり}の\\
        \ruby{汝}{なんじ}が\ruby{故郷}{こきょう}を\ruby{憶}{おぼ}えよや
    
    \end{minipage}
\end{enumerate} % 番号の箇条書き ここまで
%%%%% 歌詞 ここまで %%%%%
% end body

\end{document}

\documentclass[10pt,b5j]{tarticle} % B6 縦書き
% \documentclass[10pt,b5j]{tarticle} % B6 縦書き
\AtBeginDvi{\special{papersize=128mm,182mm}} % B6 用用紙サイズ
\usepackage{otf} % Unicode で字を入力するのに必要なパッケージ
\usepackage[size=b6j]{bxpapersize} % B6 用紙サイズを指定
\usepackage[dvipdfmx]{graphicx} % 画像を挿入するためのパッケージ
\usepackage[dvipdfmx]{color} % 色をつけるためのパッケージ
\usepackage{pxrubrica} % ルビを振るためのパッケージ
\usepackage{multicol} % 複数段組を作るためのパッケージ
\setlength{\topmargin}{14mm} % 上下方向のマージン
\addtolength{\topmargin}{-1in} % 
\setlength{\oddsidemargin}{11mm} % 左右方向のマージン
\addtolength{\oddsidemargin}{-1in} % 
\setlength{\textwidth}{154mm} % B6 用
\setlength{\textheight}{108mm} % B6 用
\setlength{\headsep}{0mm} % 
\setlength{\headheight}{0mm} % 
\setlength{\topskip}{0mm} % 
\setlength{\parskip}{0pt} % 
\def\labelenumi{\theenumi、} % 箇条書きのフォーマット
\parindent = 0pt % 段落下げしない

 % B6 用テンプレート読み込み

\begin{document}
% begin header
%%%%% タイトルと作者 ここから %%%%%
\begin{minipage}[c]{0.7\hsize} % タイトルは上から 7 割
    \begin{center}
    % begin title
        {\LARGE
            浅緑燃ゆる % タイトルを入れる
        }
        {\small 
            (昭和二十二年第四十回記念祭歌) % 年などを入れる
        }
    % end title
    \end{center}
\end{minipage}
\begin{minipage}[c]{0.3\hsize} % 作歌作曲は上から 3 割
    \begin{flushright} % 下寄せにする
        % begin name
        山家貫之君 作歌\\堀井洵君 作曲 % 作歌・作曲者
        % end name
    \end{flushright}
\end{minipage}
%%%%% タイトルと作者 ここまで %%%%%
% (1,2,4,5 繰り返しなし)
% end header

% begin length
\vspace{1.5em} % タイトル, 作者と歌詞の間に隙間を設ける
\newcommand{\linespace}{0.5em} % 行間の設定
\newcommand{\blocksize}{0.5\hsize} % 段組間の設定
\newcommand{\itemmargin}{3em} % 曲番の位置調整の長さ
% end length
% begin body
%%%%% 歌詞 ここから %%%%%
\begin{enumerate} % 番号の箇条書き ここから
    \setlength{\itemindent}{\itemmargin} % 曲番の位置調整
    \begin{minipage}[c]{\blocksize}
    
        \vspace{\linespace}
        \item~\\
        % 1.
        \ruby{浅緑}{せんりょく}\ruby{燃}{もゆる}ゆる\ruby{北}{きた}の\ruby{曠里}{}\\
        \ruby{荒}{すさ}ぶ\ruby{嵐}{あらし}を\ruby{身}{み}に\ruby{受}{う}けて\\
        \ruby{神秘}{しんぴ}の\ruby{扉}{とびら}\ruby{開}{あ}け\ruby{放}{はな}ち\\
        \ruby{雄叫}{おたけ}び\ruby{高}{たか}く\ruby{濁世}{だくせい}に\\
        \ruby{叱咤}{しった}の\ruby{剣}{けん}を\ruby{振}{ふ}るふかな
        
    \end{minipage}
    \begin{minipage}[c]{\blocksize}
        
        \vspace{\linespace}
        \item~\\
        % 2.
        \ruby{沈黙}{ちんもく}の\ruby{楡林}{ゆりん}のほの\ruby{暗}{ほのぐら}く\\
        \ruby{友}{とも}と\ruby{高望}{たかのぞみ}を\ruby{語}{かた}りてし\\
        \ruby{三年}{みとし}の\ruby{夢}{ゆめ}は\ruby{淡}{あわ}くとも\\
        \ruby{羽}{はね}\ruby{搏}{}かんかな\ruby{大鳳}{たいほう}は\\
        アンデスの\ruby{嶺}{みね}\ruby{越}{こ}えゆかん
        
    \end{minipage}
    \begin{minipage}[c]{\blocksize}
        
        \vspace{\linespace}
        \item~\\
        % 3.
        ソロモンの\ruby{栄華}{えいが}すでになし\\
        \ruby{血涙}{けつるい}もて\ruby{築}{きず}きし\ruby{幾}{いく}\ruby{春秋}{しゅんじゅう}\\
        \ruby{花}{はな}を\ruby{褥}{しとね}に\ruby{仮睡}{かすい}めば\\
        \ruby{春}{はる}\ruby{駘蕩}{たいとう}の\ruby{微風}{びふう}の\ruby{香}{こう}に\\
        \ruby{私語}{しご}めく\ruby{永遠}{えいえん}の\ruby{理想}{りそう}かな
        
    \end{minipage}
    \begin{minipage}[c]{\blocksize}
        
        \vspace{\linespace}
        \item~\\
        % 4.
        \ruby{北斗}{ほくと}の\ruby{啓示}{けいじ}なほ\ruby{清}{きよ}く\\
        \ruby{今宵}{こよい}\ruby{四}{よん}\ruby{寮}{りょう}に\ruby{輝}{かがや}けば\\
        \ruby{猛}{たけし}き\ruby{遊}{ゆう}\ruby{児}{じ}の\ruby{熱血}{ねっけつ}は\\
        ナイルの\ruby{河}{かわ}のなほ\ruby{浩}{ひろし}く\\
        \ruby{乱}{みだ}れし\ruby{世}{よ}をば\ruby{呑}{の}みほさん
        
    \end{minipage}
    \begin{minipage}[c]{\blocksize}
        
        \vspace{\linespace}
        \item~\\
        % 5.
        \ruby{青史}{せいし}は\ruby{薫}{かお}る\ruby{七}{なな}\ruby{十}{じゅう}\ruby{星霜}{せいそう}の\\
        \ruby{崇高}{すうこう}き\ruby{歴史}{れきし}を\ruby{承継}{しょうけい}ぎて\\
        \ruby{明日}{あした}\ruby{創造}{そうぞう}の\ruby{首途}{しゅと}に\\
        \ruby{今日}{きょう}\ruby{四}{よん}\ruby{十}{じゅう}\ruby{回}{かい}の\ruby{記念祭}{きねんさい}\\
        \ruby{浩歌}{はるか}はんかな\ruby{吾}{わ}が\ruby{友}{とも}よ
    
    \end{minipage}
\end{enumerate} % 番号の箇条書き ここまで
%%%%% 歌詞 ここまで %%%%%
% end body

\end{document}

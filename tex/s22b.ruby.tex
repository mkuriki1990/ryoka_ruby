\documentclass[10pt,b5j]{tarticle} % B6 縦書き
% \documentclass[10pt,b5j]{tarticle} % B6 縦書き
\AtBeginDvi{\special{papersize=128mm,182mm}} % B6 用用紙サイズ
\usepackage{otf} % Unicode で字を入力するのに必要なパッケージ
\usepackage[size=b6j]{bxpapersize} % B6 用紙サイズを指定
\usepackage[dvipdfmx]{graphicx} % 画像を挿入するためのパッケージ
\usepackage[dvipdfmx]{color} % 色をつけるためのパッケージ
\usepackage{pxrubrica} % ルビを振るためのパッケージ
\usepackage{multicol} % 複数段組を作るためのパッケージ
\setlength{\topmargin}{14mm} % 上下方向のマージン
\addtolength{\topmargin}{-1in} % 
\setlength{\oddsidemargin}{11mm} % 左右方向のマージン
\addtolength{\oddsidemargin}{-1in} % 
\setlength{\textwidth}{154mm} % B6 用
\setlength{\textheight}{108mm} % B6 用
\setlength{\headsep}{0mm} % 
\setlength{\headheight}{0mm} % 
\setlength{\topskip}{0mm} % 
\setlength{\parskip}{0pt} % 
\def\labelenumi{\theenumi、} % 箇条書きのフォーマット
\parindent = 0pt % 段落下げしない

 % B6 用テンプレート読み込み

\begin{document}
% begin header
%%%%% タイトルと作者 ここから %%%%%
\begin{minipage}[c]{0.7\hsize} % タイトルは上から 7 割
    \begin{center}
    % begin title
        {\LARGE
            浅緑燃ゆる % タイトルを入れる
        }
        {\small 
            (昭和二十二年第四十回記念祭歌) % 年などを入れる
        }
    % end title
    \end{center}
\end{minipage}
\begin{minipage}[c]{0.3\hsize} % 作歌作曲は上から 3 割
    \begin{flushright} % 下寄せにする
        % begin name
        山家貫之君 作歌\\堀井洵君 作曲 % 作歌・作曲者
        % end name
    \end{flushright}
\end{minipage}
%%%%% タイトルと作者 ここまで %%%%%
% (1,2,4,5 繰り返しなし)
% end header

% begin body
\vspace{1.5em} % タイトル, 作者と歌詞の間に隙間を設ける
\newcommand{\linespace}{0.5em} % 行間の設定
\newcommand{\blocksize}{0.5\hsize} % 段組間の設定
%%%%% 歌詞 ここから %%%%%
% begin lilycs
\begin{enumerate} % 番号の箇条書き ここから
    \begin{minipage}[c]{\blocksize}
    
        \vspace{\linespace}
        \item
        % 1.
        \ruby{浅緑燃}{}ゆる\ruby{北}{}の\ruby{曠里}{}\\
        \ruby{荒}{}ぶ\ruby{嵐}{}を\ruby{身}{}に\ruby{受}{}けて\\
        \ruby{神秘}{}の\ruby{扉開}{}け\ruby{放}{}ち\\
        \ruby{雄叫}{}び\ruby{高}{}く\ruby{濁世}{}に\\
        \ruby{叱咤}{}の\ruby{剣}{}を\ruby{振}{}るふかな
        
        \vspace{\linespace}
        \item
        % 2.
        \ruby{沈黙}{}の\ruby{楡林}{}のほの\ruby{暗}{}く\\
        \ruby{友}{}と\ruby{高望}{}を\ruby{語}{}りてし\\
        \ruby{三年}{}の\ruby{夢}{}は\ruby{淡}{}くとも\\
        \ruby{羽搏}{}かんかな\ruby{大鳳}{}は\\
        アンデスの\ruby{嶺越}{}えゆかん
        
        \vspace{\linespace}
        \item
        % 3.
        ソロモンの\ruby{栄華}{}すでになし\\
        \ruby{血涙}{}もて\ruby{築}{}きし\ruby{幾春秋}{}\\
        \ruby{花}{}を\ruby{褥}{}に\ruby{仮睡}{}めば\\
        \ruby{春駘蕩}{}の\ruby{微風}{}の\ruby{香}{}に\\
        \ruby{私語}{}めく\ruby{永遠}{}の\ruby{理想}{}かな
        
        \vspace{\linespace}
        \item
        % 4.
        \ruby{北斗}{}の\ruby{啓示}{}なほ\ruby{清}{}く\\
        \ruby{今宵四寮}{}に\ruby{輝}{}けば\\
        \ruby{猛}{}き\ruby{遊児}{}の\ruby{熱血}{}は\\
        ナイルの\ruby{河}{}のなほ\ruby{浩}{}く\\
        \ruby{乱}{}れし\ruby{世}{}をば\ruby{呑}{}みほさん
        
        \vspace{\linespace}
        \item
        % 5.
        \ruby{青史}{}は\ruby{薫}{}る\ruby{七十星霜}{}の\\
        \ruby{崇高}{}き\ruby{歴史}{}を\ruby{承継}{}ぎて\\
        \ruby{明日創造}{}の\ruby{首途}{}に\\
        \ruby{今日四十回}{}の\ruby{記念祭}{}\\
        \ruby{浩歌}{}はんかな\ruby{吾}{}が\ruby{友}{}よ
    
    \end{minipage}
\end{enumerate} % 番号の箇条書き ここまで
% end lilycs
%%%%% 歌詞 ここまで %%%%%
% end body

\end{document}

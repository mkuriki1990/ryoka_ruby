\documentclass[10pt,b5j]{tarticle} % B6 縦書き
% \documentclass[10pt,b5j]{tarticle} % B6 縦書き
\AtBeginDvi{\special{papersize=128mm,182mm}} % B6 用用紙サイズ
\usepackage{otf} % Unicode で字を入力するのに必要なパッケージ
\usepackage[size=b6j]{bxpapersize} % B6 用紙サイズを指定
\usepackage[dvipdfmx]{graphicx} % 画像を挿入するためのパッケージ
\usepackage[dvipdfmx]{color} % 色をつけるためのパッケージ
\usepackage{pxrubrica} % ルビを振るためのパッケージ
\usepackage{multicol} % 複数段組を作るためのパッケージ
\setlength{\topmargin}{14mm} % 上下方向のマージン
\addtolength{\topmargin}{-1in} % 
\setlength{\oddsidemargin}{11mm} % 左右方向のマージン
\addtolength{\oddsidemargin}{-1in} % 
\setlength{\textwidth}{154mm} % B6 用
\setlength{\textheight}{108mm} % B6 用
\setlength{\headsep}{0mm} % 
\setlength{\headheight}{0mm} % 
\setlength{\topskip}{0mm} % 
\setlength{\parskip}{0pt} % 
\def\labelenumi{\theenumi、} % 箇条書きのフォーマット
\parindent = 0pt % 段落下げしない

 % B6 用テンプレート読み込み

\begin{document}
% begin header
%%%%% タイトルと作者 ここから %%%%%
\begin{minipage}[c]{0.7\hsize} % タイトルは上から 7 割
    \begin{center}
    % begin title
        {\LARGE
            偉大なる北溟の自然 % タイトルを入れる
        }
        {\small 
            (昭和三十九年寮歌) % 年などを入れる
        }
    % end title
    \end{center}
\end{minipage}
\begin{minipage}[c]{0.3\hsize} % 作歌作曲は上から 3 割
    \begin{flushright} % 下寄せにする
        % begin name
        司馬威彦君 作歌・作曲 % 作歌・作曲者
        % end name
    \end{flushright}
\end{minipage}
%%%%% タイトルと作者 ここまで %%%%%
% (序,1,結 了なし繰り返しあり)
% end header

% begin length
\vspace{1.5em} % タイトル, 作者と歌詞の間に隙間を設ける
\newcommand{\linespace}{0.5em} % 行間の設定
\newcommand{\blocksize}{0.5\hsize} % 段組間の設定
\newcommand{\itemmargin}{3em} % 曲番の位置調整の長さ
% end length
% begin body
%%%%% 歌詞 ここから %%%%%
\begin{enumerate} % 番号の箇条書き ここから
    \setlength{\itemindent}{\itemmargin} % 曲番の位置調整
    \begin{minipage}[c]{\blocksize}
    
        \vspace{\linespace}
        \item~\\
        % \ruby{序}{ついで}.
        \ruby{偉大}{いだい}なる\ruby{北}{きた}\ruby{溟}{}の\ruby{自然}{しぜん}は\\
        \ruby{我}{わ}が\ruby{眼前}{がんぜん}に\ruby{限}{かぎ}りなく\ruby{広}{ひろ}ごりて\\
        \ruby{野}{の}に\ruby{満}{まん}てる\ruby{清冽}{せいれつ}の\ruby{気}{き}は\\
        \ruby{雄々}{おお}しくも\ruby{気高}{けだか}き\ruby{情}{じょう}\ruby{懐}{ふところ}もて\\
        \ruby{嶮路}{}\ruby{遙}{}かに\ruby{辿}{たど}り\ruby{来}{きた}し\\
        \ruby{遊子}{ゆうし}が\ruby{胸}{むね}を\ruby{今}{いま}や\ruby{満}{みつる}しぬ
        
    \end{minipage}
    \begin{minipage}[c]{\blocksize}
        
        \vspace{\linespace}
        \item~\\
        % 1.
        \ruby{飄々}{ひょうひょう}の\ruby{北風}{きたかぜ}は\ruby{荒}{すさ}び\\
        \ruby{白銀}{はくぎん}の\ruby{華}{はな}\ruby{大地}{だいち}\ruby{覆}{おお}えど\\
        そははろかなる\ruby{古}{いにしえ}より\\
        \ruby{汚}{よご}れなき\ruby{美}{び}の\ruby{世界}{せかい}なれば\\
        \ruby{若人}{わこうど}はひたぶるの\\
        \ruby{愁}{うれ}いを\ruby{秘}{ひ}めて\\
        \ruby{異邦}{いほう}ゆ\ruby{憧憬}{どうけい}れ\ruby{集}{つど}いぬ
        
    \end{minipage}
    \begin{minipage}[c]{\blocksize}
        
        \vspace{\linespace}
        \item~\\
        % 2.
        いよよ\ruby{増}{ま}す\ruby{静寂}{せいじゃく}のなかに\\
        \ruby{永劫}{えいごう}の\ruby{影}{かげ}\ruby{宿}{やど}す\ruby{原始}{げんし}の\ruby{深森}{ふかもり}よ\\
        \ruby{先哲}{せんてつ}の\ruby{行路}{こうろ}を\ruby{慕}{した}いて\\
        \ruby{思索}{しさく}\ruby{胸}{むね}に\ruby{楡}{にれ}\ruby{陵}{りょう}を\ruby{歩}{あゆ}めば\\
        \ruby{仰}{あお}ぎみるエルムの\ruby{梢}{こずえ}に\\
        \ruby{萠}{めぐむ}え\ruby{出}{で}ん\ruby{若}{わか}き\ruby{情熱}{じょうねつ}は
        
    \end{minipage}
    \begin{minipage}[c]{\blocksize}
        
        \vspace{\linespace}
        \item~\\
        % 3.
        かりそめの\ruby{宿}{やど}にはあれど\\
        \ruby{忘}{わす}れ\ruby{得}{え}じ\ruby{若}{わか}き\ruby{日}{ひ}の\ruby{遍歴}{へんれき}\\
        \ruby{彷徨}{ほうこう}えば\ruby{夕陽}{ゆうひ}の\ruby{楡}{にれ}\ruby{陵}{りょう}に\\
        \ruby{宵闇}{よいやみ}はかそけくも\ruby{訪}{おとず}れ\\
        \ruby{睦}{むつ}みてし\ruby{真心}{まごころ}と\ruby{友情}{ゆうじょう}に\\
        \ruby{篝火}{かがりび}は\ruby{赤}{あか}く\ruby{燃}{も}えたり
        
    \end{minipage}
    \begin{minipage}[c]{\blocksize}
        
        \vspace{\linespace}
        \item~\\
        % 4.
        \ruby{輝}{かがや}ける\ruby{北国}{きたぐに}のたくみよ\\
        されど\ruby{優}{まさ}りて\ruby{美}{うつく}しき\ruby{自治}{じち}の\ruby{伝統}{でんとう}よ\\
        \ruby{斗}{と}い\ruby{苦}{く}\ruby{悩}{なや}み\ruby{寮}{りょう}\ruby{友}{とも}と\ruby{語}{かた}れば\\
        などて\ruby{疾}{と}く\ruby{過}{す}ぎ\ruby{行}{ゆ}く\ruby{二}{に}\ruby{年}{ねん}の\ruby{春}{はる}\\
        \ruby{願}{ねが}わなん\ruby{永久}{えいきゅう}の\ruby{栄}{は}えを\\
        \ruby{恵}{めぐみ}\ruby{迪}{すすむ}の\ruby{寮}{りょう}\ruby{故郷}{こきょう}の\ruby{上}{うえ}に
        
    \end{minipage}
    \begin{minipage}[c]{\blocksize}
        
        \vspace{\linespace}
        \item~\\
        % \ruby{結}{ゆい}.
        されど\ruby{視}{み}よ\ruby{我}{わが}\ruby{等}{とう}が\ruby{周囲}{しゅうい}を\\
        \ruby{邪悪}{じゃあく}なる\ruby{権力}{けんりょく}は\ruby{四方}{しほう}に\ruby{荒}{すさ}び\\
        \ruby{我}{わが}\ruby{等}{とう}が\ruby{愛}{いと}し\ruby{誇}{ほこ}らん\ruby{自治}{じち}の\ruby{砦}{とりで}に\\
        \ruby{暴逆}{}の\ruby{誠}{まこと}は\ruby{課}{か}されんとす\\
        されば\ruby{我}{わ}が\ruby{寮}{りょう}\ruby{友}{とも}よ\ruby{腕}{うで}むすびて\\
        \ruby{今}{いま}ぞ\ruby{正義}{せいぎ}の\ruby{旗}{はた}を\ruby{高}{たか}くかかげん
    
    \end{minipage}
\end{enumerate} % 番号の箇条書き ここまで
%%%%% 歌詞 ここまで %%%%%
% end body

\end{document}

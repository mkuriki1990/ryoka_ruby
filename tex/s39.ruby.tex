\documentclass[10pt,b5j]{tarticle} % B6 縦書き
% \documentclass[10pt,b5j]{tarticle} % B6 縦書き
\AtBeginDvi{\special{papersize=128mm,182mm}} % B6 用用紙サイズ
\usepackage{otf} % Unicode で字を入力するのに必要なパッケージ
\usepackage[size=b6j]{bxpapersize} % B6 用紙サイズを指定
\usepackage[dvipdfmx]{graphicx} % 画像を挿入するためのパッケージ
\usepackage[dvipdfmx]{color} % 色をつけるためのパッケージ
\usepackage{pxrubrica} % ルビを振るためのパッケージ
\usepackage{multicol} % 複数段組を作るためのパッケージ
\setlength{\topmargin}{14mm} % 上下方向のマージン
\addtolength{\topmargin}{-1in} % 
\setlength{\oddsidemargin}{11mm} % 左右方向のマージン
\addtolength{\oddsidemargin}{-1in} % 
\setlength{\textwidth}{154mm} % B6 用
\setlength{\textheight}{108mm} % B6 用
\setlength{\headsep}{0mm} % 
\setlength{\headheight}{0mm} % 
\setlength{\topskip}{0mm} % 
\setlength{\parskip}{0pt} % 
\def\labelenumi{\theenumi、} % 箇条書きのフォーマット
\parindent = 0pt % 段落下げしない

 % B6 用テンプレート読み込み

\begin{document}
% begin header
%%%%% タイトルと作者 ここから %%%%%
\begin{minipage}[c]{0.7\hsize} % タイトルは上から 7 割
    \begin{center}
    % begin title
        {\LARGE
            偉大なる北溟の自然 % タイトルを入れる
        }
        {\small 
            (昭和三十九年寮歌) % 年などを入れる
        }
    % end title
    \end{center}
\end{minipage}
\begin{minipage}[c]{0.3\hsize} % 作歌作曲は上から 3 割
    \begin{flushright} % 下寄せにする
        % begin name
        司馬威彦君 作歌・作曲 % 作歌・作曲者
        % end name
    \end{flushright}
\end{minipage}
%%%%% タイトルと作者 ここまで %%%%%
% (序,1,結 了なし繰り返しあり)
% end header

% begin length
\vspace{1.5em} % タイトル, 作者と歌詞の間に隙間を設ける
\newcommand{\linespace}{0.5em} % 行間の設定
\newcommand{\blocksize}{0.5\hsize} % 段組間の設定
\newcommand{\itemmargin}{6em} % 曲番の位置調整の長さ
% end length
% begin body
%%%%% 歌詞 ここから %%%%%
\begin{enumerate} % 番号の箇条書き ここから
    \setlength{\itemindent}{\itemmargin} % 曲番の位置調整
    \begin{minipage}[c]{\blocksize}
    
        \vspace{\linespace}
        \item~\\
        % \ruby{序}{}.
        \ruby{偉大}{}なる\ruby{北溟}{}の\ruby{自然}{}は\\
        \ruby{我}{}が\ruby{眼前}{}に\ruby{限}{}りなく\ruby{広}{}ごりて\\
        \ruby{野}{}に\ruby{満}{}てる\ruby{清冽}{}の\ruby{気}{}は\\
        \ruby{雄々}{}しくも\ruby{気高}{}き\ruby{情懐}{}もて\\
        \ruby{嶮路遙}{}かに\ruby{辿}{}り\ruby{来}{}し\\
        \ruby{遊子}{}が\ruby{胸}{}を\ruby{今}{}や\ruby{満}{}しぬ
        
        \vspace{\linespace}
        \item~\\
        % 1.
        \ruby{飄々}{}の\ruby{北風}{}は\ruby{荒}{}び\\
        \ruby{白銀}{}の\ruby{華大地覆}{}えど\\
        そははろかなる\ruby{古}{}より\\
        \ruby{汚}{}れなき\ruby{美}{}の\ruby{世界}{}なれば\\
        \ruby{若人}{}はひたぶるの\\
        \ruby{愁}{}いを\ruby{秘}{}めて\\
        \ruby{異邦}{}ゆ\ruby{憧憬}{}れ\ruby{集}{}いぬ
        
        \vspace{\linespace}
        \item~\\
        % 2.
        いよよ\ruby{増}{}す\ruby{静寂}{}のなかに\\
        \ruby{永劫}{}の\ruby{影宿}{}す\ruby{原始}{}の\ruby{深森}{}よ\\
        \ruby{先哲}{}の\ruby{行路}{}を\ruby{慕}{}いて\\
        \ruby{思索胸}{}に\ruby{楡陵}{}を\ruby{歩}{}めば\\
        \ruby{仰}{}ぎみるエルムの\ruby{梢}{}に\\
        \ruby{萠}{}え\ruby{出}{}ん\ruby{若}{}き\ruby{情熱}{}は
        
        \vspace{\linespace}
        \item~\\
        % 3.
        かりそめの\ruby{宿}{}にはあれど\\
        \ruby{忘}{}れ\ruby{得}{}じ\ruby{若}{}き\ruby{日}{}の\ruby{遍歴}{}\\
        \ruby{彷徨}{}えば\ruby{夕陽}{}の\ruby{楡陵}{}に\\
        \ruby{宵闇}{}はかそけくも\ruby{訪}{}れ\\
        \ruby{睦}{}みてし\ruby{真心}{}と\ruby{友情}{}に\\
        \ruby{篝火}{}は\ruby{赤}{}く\ruby{燃}{}えたり
        
        \vspace{\linespace}
        \item~\\
        % 4.
        \ruby{輝}{}ける\ruby{北国}{}のたくみよ\\
        されど\ruby{優}{}りて\ruby{美}{}しき\ruby{自治}{}の\ruby{伝統}{}よ\\
        \ruby{斗}{}い\ruby{苦悩}{}み\ruby{寮友}{}と\ruby{語}{}れば\\
        などて\ruby{疾}{}く\ruby{過}{}ぎ\ruby{行}{}く\ruby{二年}{}の\ruby{春}{}\\
        \ruby{願}{}わなん\ruby{永久}{}の\ruby{栄}{}えを\\
        \ruby{恵迪}{}の\ruby{寮故郷}{}の\ruby{上}{}に
        
        \vspace{\linespace}
        \item~\\
        % \ruby{結}{}.
        されど\ruby{視}{}よ\ruby{我等}{}が\ruby{周囲}{}を\\
        \ruby{邪悪}{}なる\ruby{権力}{}は\ruby{四方}{}に\ruby{荒}{}び\\
        \ruby{我等}{}が\ruby{愛}{}し\ruby{誇}{}らん\ruby{自治}{}の\ruby{砦}{}に\\
        \ruby{暴逆}{}の\ruby{誠}{}は\ruby{課}{}されんとす\\
        されば\ruby{我}{}が\ruby{寮友}{}よ\ruby{腕}{}むすびて\\
        \ruby{今}{}ぞ\ruby{正義}{}の\ruby{旗}{}を\ruby{高}{}くかかげん
    
    \end{minipage}
\end{enumerate} % 番号の箇条書き ここまで
%%%%% 歌詞 ここまで %%%%%
% end body

\end{document}

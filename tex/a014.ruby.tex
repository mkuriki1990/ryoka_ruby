\documentclass[10pt,b5j]{tarticle} % B6 縦書き
% \documentclass[10pt,b5j]{tarticle} % B6 縦書き
\AtBeginDvi{\special{papersize=128mm,182mm}} % B6 用用紙サイズ
\usepackage{otf} % Unicode で字を入力するのに必要なパッケージ
\usepackage[size=b6j]{bxpapersize} % B6 用紙サイズを指定
\usepackage[dvipdfmx]{graphicx} % 画像を挿入するためのパッケージ
\usepackage[dvipdfmx]{color} % 色をつけるためのパッケージ
\usepackage{pxrubrica} % ルビを振るためのパッケージ
\usepackage{multicol} % 複数段組を作るためのパッケージ
\setlength{\topmargin}{14mm} % 上下方向のマージン
\addtolength{\topmargin}{-1in} % 
\setlength{\oddsidemargin}{11mm} % 左右方向のマージン
\addtolength{\oddsidemargin}{-1in} % 
\setlength{\textwidth}{154mm} % B6 用
\setlength{\textheight}{108mm} % B6 用
\setlength{\headsep}{0mm} % 
\setlength{\headheight}{0mm} % 
\setlength{\topskip}{0mm} % 
\setlength{\parskip}{0pt} % 
\def\labelenumi{\theenumi、} % 箇条書きのフォーマット
\parindent = 0pt % 段落下げしない

 % B6 用テンプレート読み込み

\begin{document}
% begin header
%%%%% タイトルと作者 ここから %%%%%
\begin{minipage}[c]{0.7\hsize} % タイトルは上から 7 割
    \begin{center}
    % begin title
        {\LARGE
            北海道帝国大学文武会創基五十周年祝歌 % タイトルを入れる
        }
        {\small 
             % 年などを入れる
        }
    % end title
    \end{center}
\end{minipage}
\begin{minipage}[c]{0.3\hsize} % 作歌作曲は上から 3 割
    \begin{flushright} % 下寄せにする
        % begin name
        秋野豊太君 作歌\\河口忠雄君 作曲 % 作歌・作曲者
        % end name
    \end{flushright}
\end{minipage}
%%%%% タイトルと作者 ここまで %%%%%
% % end header

% begin length
\vspace{1.5em} % タイトル, 作者と歌詞の間に隙間を設ける
\newcommand{\linespace}{0.5em} % 行間の設定
\newcommand{\blocksize}{0.5\hsize} % 段組間の設定
\newcommand{\itemmargin}{3em} % 曲番の位置調整の長さ
% end length
% begin body
%%%%% 歌詞 ここから %%%%%
\begin{enumerate} % 番号の箇条書き ここから
    \setlength{\itemindent}{\itemmargin} % 曲番の位置調整
    \begin{minipage}[c]{\blocksize}
    
        \vspace{\linespace}
        \item~\\
        % 1.
        \ruby{楡}{にれ}の\ruby{森}{もり}\ruby{蔭}{かげ}\ruby{緑}{みどり}して\\
        \ruby{嫩草}{}\ruby{薫}{かお}る\ruby{学}{がく}\ruby{庭}{にわ}に\\
        \ruby{文}{ぶん}\ruby{華}{はな}\ruby{絢爛}{けんらん}\ruby{春}{はる}\ruby{來}{}れば\\
        \ruby{創}{そう}\ruby{基}{もと}\ruby{五}{ご}\ruby{十}{じゅう}の\ruby{鐘}{かね}ぞ\ruby{鳴}{な}る
        
    \end{minipage}
    \begin{minipage}[c]{\blocksize}
        
        \vspace{\linespace}
        \item~\\
        % 2.
        ビー・アンビシャス・ボーイズの\\
        \ruby{遠}{とお}き\ruby{遺訓}{いくん}も\ruby{新}{あたら}しく\\
        \ruby{温}{ぬる}ねて\ruby{友}{とも}よ\ruby{仰}{あお}ぎ\ruby{見}{み}ん\\
        \ruby{師}{し}の\ruby{俤}{おもかげ}の\ruby{尊}{みこと}としや
        
    \end{minipage}
    \begin{minipage}[c]{\blocksize}
        
        \vspace{\linespace}
        \item~\\
        % 3.
        \ruby{原始}{げんし}の\ruby{荒野}{あらの}\ruby{拓}{ひら}きつゝ\\
        \ruby{培}{}ひ\ruby{初}{はつ}めし\ruby{教}{きょう}\ruby{草}{くさ}\\
        \ruby{栄}{さか}えて\ruby{茲}{}に\ruby{最}{さい}と\ruby{高}{たか}き\\
        \ruby{学府}{がくふ}のほまれ\ruby{偉}{えら}なる\ruby{哉}{}
        
    \end{minipage}
    \begin{minipage}[c]{\blocksize}
        
        \vspace{\linespace}
        \item~\\
        % 4.
        あゝ\ruby{極}{ごく}\ruby{星}{ほし}は\ruby{燦}{}として\\
        \ruby{北}{きた}の\ruby{文化}{ぶんか}や\ruby{照}{て}り\ruby{映}{うつ}ゆる\\
        いざ\ruby{若人}{わこうど}よ\ruby{集}{しゅう}ひ\ruby{來}{}て\\
        \ruby{宵}{よい}の\ruby{灯}{あかり}かざし\ruby{祝}{しゅく}はずや
        
        
    \end{minipage}
    \begin{minipage}[c]{\blocksize}
        
        \vspace{\linespace}
        \item~\\
        \ruby{注}{ちゅう}\\
        \ruby{第}{だい}\ruby{二}{に}\ruby{節}{せつ}はクラーク\ruby{胸像}{きょうぞう}の\ruby{除幕}{じょまく}\ruby{式}{しき}を\\
        \ruby{祝}{いわ}う\ruby{意味}{いみ}で\ruby{作}{つく}られた。
    
    \end{minipage}
\end{enumerate} % 番号の箇条書き ここまで
%%%%% 歌詞 ここまで %%%%%
% end body

\end{document}

\documentclass[10pt,b5j]{tarticle} % B6 縦書き
% \documentclass[10pt,b5j]{tarticle} % B6 縦書き
\AtBeginDvi{\special{papersize=128mm,182mm}} % B6 用用紙サイズ
\usepackage{otf} % Unicode で字を入力するのに必要なパッケージ
\usepackage[size=b6j]{bxpapersize} % B6 用紙サイズを指定
\usepackage[dvipdfmx]{graphicx} % 画像を挿入するためのパッケージ
\usepackage[dvipdfmx]{color} % 色をつけるためのパッケージ
\usepackage{pxrubrica} % ルビを振るためのパッケージ
\usepackage{multicol} % 複数段組を作るためのパッケージ
\setlength{\topmargin}{14mm} % 上下方向のマージン
\addtolength{\topmargin}{-1in} % 
\setlength{\oddsidemargin}{11mm} % 左右方向のマージン
\addtolength{\oddsidemargin}{-1in} % 
\setlength{\textwidth}{154mm} % B6 用
\setlength{\textheight}{108mm} % B6 用
\setlength{\headsep}{0mm} % 
\setlength{\headheight}{0mm} % 
\setlength{\topskip}{0mm} % 
\setlength{\parskip}{0pt} % 
\def\labelenumi{\theenumi、} % 箇条書きのフォーマット
\parindent = 0pt % 段落下げしない

 % B6 用テンプレート読み込み

\begin{document}
% begin header
%%%%% タイトルと作者 ここから %%%%%
\begin{minipage}[c]{0.7\hsize} % タイトルは上から 7 割
    \begin{center}
    % begin title
        {\LARGE
            北海道帝国大学文武会創基五十周年祝歌 % タイトルを入れる
        }
        {\small 
             % 年などを入れる
        }
    % end title
    \end{center}
\end{minipage}
\begin{minipage}[c]{0.3\hsize} % 作歌作曲は上から 3 割
    \begin{flushright} % 下寄せにする
        % begin name
        秋野豊太君 作歌\\河口忠雄君 作曲 % 作歌・作曲者
        % end name
    \end{flushright}
\end{minipage}
%%%%% タイトルと作者 ここまで %%%%%
% % end header

% begin length
\vspace{1.5em} % タイトル, 作者と歌詞の間に隙間を設ける
\newcommand{\linespace}{0.5em} % 行間の設定
\newcommand{\blocksize}{0.5\hsize} % 段組間の設定
\newcommand{\itemmargin}{3em} % 曲番の位置調整の長さ
% end length
% begin body
%%%%% 歌詞 ここから %%%%%
\begin{enumerate} % 番号の箇条書き ここから
    \setlength{\itemindent}{\itemmargin} % 曲番の位置調整
    \begin{minipage}[c]{\blocksize}
    
        \vspace{\linespace}
        \item~\\
        % 1.
        \ruby{楡}{}の\ruby{森蔭緑}{}して\\
        \ruby{嫩草薫}{}る\ruby{学庭}{}に\\
        \ruby{文華絢爛春來}{}れば\\
        \ruby{創基五十}{}の\ruby{鐘}{}ぞ\ruby{鳴}{}る
        
    \end{minipage}
    \begin{minipage}[c]{\blocksize}
        
        \vspace{\linespace}
        \item~\\
        % 2.
        ビー・アンビシャス・ボーイズの\\
        \ruby{遠}{}き\ruby{遺訓}{}も\ruby{新}{}しく\\
        \ruby{温}{}ねて\ruby{友}{}よ\ruby{仰}{}ぎ\ruby{見}{}ん\\
        \ruby{師}{}の\ruby{俤}{}の\ruby{尊}{}としや
        
    \end{minipage}
    \begin{minipage}[c]{\blocksize}
        
        \vspace{\linespace}
        \item~\\
        % 3.
        \ruby{原始}{}の\ruby{荒野拓}{}きつゝ\\
        \ruby{培}{}ひ\ruby{初}{}めし\ruby{教草}{}\\
        \ruby{栄}{}えて\ruby{茲}{}に\ruby{最}{}と\ruby{高}{}き\\
        \ruby{学府}{}のほまれ\ruby{偉}{}なる\ruby{哉}{}
        
    \end{minipage}
    \begin{minipage}[c]{\blocksize}
        
        \vspace{\linespace}
        \item~\\
        % 4.
        あゝ\ruby{極星}{}は\ruby{燦}{}として\\
        \ruby{北}{}の\ruby{文化}{}や\ruby{照}{}り\ruby{映}{}ゆる\\
        いざ\ruby{若人}{}よ\ruby{集}{}ひ\ruby{來}{}て\\
        \ruby{宵}{}の\ruby{灯}{}かざし\ruby{祝}{}はずや
        
        
    \end{minipage}
    \begin{minipage}[c]{\blocksize}
        
        \vspace{\linespace}
        \item~\\
        \ruby{注}{}\\
        \ruby{第二節}{}はクラーク\ruby{胸像}{}の\ruby{除幕式}{}を\\
        \ruby{祝}{}う\ruby{意味}{}で\ruby{作}{}られた。
    
    \end{minipage}
\end{enumerate} % 番号の箇条書き ここまで
%%%%% 歌詞 ここまで %%%%%
% end body

\end{document}

\documentclass[10pt,b5j]{tarticle} % B6 縦書き
% \documentclass[10pt,b5j]{tarticle} % B6 縦書き
\AtBeginDvi{\special{papersize=128mm,182mm}} % B6 用用紙サイズ
\usepackage{otf} % Unicode で字を入力するのに必要なパッケージ
\usepackage[size=b6j]{bxpapersize} % B6 用紙サイズを指定
\usepackage[dvipdfmx]{graphicx} % 画像を挿入するためのパッケージ
\usepackage[dvipdfmx]{color} % 色をつけるためのパッケージ
\usepackage{pxrubrica} % ルビを振るためのパッケージ
\usepackage{multicol} % 複数段組を作るためのパッケージ
\setlength{\topmargin}{14mm} % 上下方向のマージン
\addtolength{\topmargin}{-1in} % 
\setlength{\oddsidemargin}{11mm} % 左右方向のマージン
\addtolength{\oddsidemargin}{-1in} % 
\setlength{\textwidth}{154mm} % B6 用
\setlength{\textheight}{108mm} % B6 用
\setlength{\headsep}{0mm} % 
\setlength{\headheight}{0mm} % 
\setlength{\topskip}{0mm} % 
\setlength{\parskip}{0pt} % 
\def\labelenumi{\theenumi、} % 箇条書きのフォーマット
\parindent = 0pt % 段落下げしない

 % B6 用テンプレート読み込み

\begin{document}
% begin header
%%%%% タイトルと作者 ここから %%%%%
\begin{minipage}[c]{0.7\hsize} % タイトルは上から 7 割
    \begin{center}
    % begin title
        {\LARGE
            春来にけらし % タイトルを入れる
        }
        {\small 
            (昭和十七年寮歌) % 年などを入れる
        }
    % end title
    \end{center}
\end{minipage}
\begin{minipage}[c]{0.3\hsize} % 作歌作曲は上から 3 割
    \begin{flushright} % 下寄せにする
        % begin name
        橋爪秀雄君 作歌\\杢子一雄君 作曲 % 作歌・作曲者
        % end name
    \end{flushright}
\end{minipage}
%%%%% タイトルと作者 ここまで %%%%%
% (1,2,3,4 繰り返しなし)
% end header

% begin length
\vspace{1.5em} % タイトル, 作者と歌詞の間に隙間を設ける
\newcommand{\linespace}{0.5em} % 行間の設定
\newcommand{\blocksize}{0.5\hsize} % 段組間の設定
\newcommand{\itemmargin}{3em} % 曲番の位置調整の長さ
% end length
% begin body
%%%%% 歌詞 ここから %%%%%
\begin{enumerate} % 番号の箇条書き ここから
    \setlength{\itemindent}{\itemmargin} % 曲番の位置調整
    \begin{minipage}[c]{\blocksize}
    
        \vspace{\linespace}
        \item~\\
        % 1.
        \ruby{春来}{}にけらし\ruby{白雪}{}の\\
        \ruby{厚}{}き\ruby{衣}{}や\ruby{重}{}からん\\
        \ruby{綾羅}{}の\ruby{糸}{}も\ruby{綻}{}ろびて\\
        \ruby{朧々深}{}き\ruby{五月闇}{}\\
        \ruby{楡影揺}{}めく\ruby{鼙鼓}{}の\ruby{音}{}に\\
        \ruby{夜霧}{}に\ruby{蒸}{}せる\ruby{緑酒汲}{}み\\
        \ruby{挙}{}りて\ruby{踊}{}る\ruby{楡}{}の\ruby{精}{}
        
    \end{minipage}
    \begin{minipage}[c]{\blocksize}
        
        \vspace{\linespace}
        \item~\\
        % 2.
        \ruby{草茅}{}しげき\ruby{原始林}{}かげに\\
        \ruby{聖}{}き\ruby{焔}{}を\ruby{囲}{}みつつ\\
        \ruby{若}{}き\ruby{情熱}{}は\ruby{求}{}むれど\\
        \ruby{人生誰}{}かよく\ruby{解}{}かん\\
        ただ\ruby{真}{}なる\ruby{愛}{}に\ruby{泣}{}く\\
        \ruby{寮友}{}の\ruby{姿}{}の\ruby{清}{}ければ\\
        \ruby{春宵}{}の\ruby{罪}{}と\ruby{誰}{}か\ruby{言}{}ふ
        
    \end{minipage}
    \begin{minipage}[c]{\blocksize}
        
        \vspace{\linespace}
        \item~\\
        % 3.
        \ruby{春秋糸}{}も\ruby{限}{}りなく\\
        \ruby{文月}{}の\ruby{夢}{}は\ruby{織女星}{}の\\
        あはれ\ruby{手稲}{}の\ruby{衣}{}かな\\
        \ruby{山}{}の\ruby{端深}{}くたそがれて\\
        \ruby{今宵銀河}{}の\ruby{祭日}{}の\\
        \ruby{永劫}{}の\ruby{空}{}を\ruby{眺}{}むれば\\
        \ruby{天空流}{}る\ruby{星一}{}つ
        
    \end{minipage}
    \begin{minipage}[c]{\blocksize}
        
        \vspace{\linespace}
        \item~\\
        % 4.
        \ruby{雨月}{}の\ruby{濁流滔々}{}と\\
        \ruby{豊川}{}に\ruby{聞}{}く\ruby{世}{}の\ruby{憂}{}\\
        \ruby{泥潦沈}{}み\ruby{真清水}{}の\\
        \ruby{流}{}るる\ruby{秋}{}は\ruby{見}{}ざるとも\\
        \ruby{墳墓}{}の\ruby{土}{}を\ruby{清}{}くせん\\
        \ruby{戦}{}の\ruby{庭}{}を\ruby{高}{}らかに\\
        \ruby{七}{}つの\ruby{海}{}の\ruby{潮音}{}よ
    
    \end{minipage}
\end{enumerate} % 番号の箇条書き ここまで
%%%%% 歌詞 ここまで %%%%%
% end body

\end{document}

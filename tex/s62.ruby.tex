\documentclass[10pt,b5j]{tarticle} % B6 縦書き
% \documentclass[10pt,b5j]{tarticle} % B6 縦書き
\AtBeginDvi{\special{papersize=128mm,182mm}} % B6 用用紙サイズ
\usepackage{otf} % Unicode で字を入力するのに必要なパッケージ
\usepackage[size=b6j]{bxpapersize} % B6 用紙サイズを指定
\usepackage[dvipdfmx]{graphicx} % 画像を挿入するためのパッケージ
\usepackage[dvipdfmx]{color} % 色をつけるためのパッケージ
\usepackage{pxrubrica} % ルビを振るためのパッケージ
\usepackage{plext} % 漢数字の enumerate を使うためのパッケージ
\usepackage{multicol} % 複数段組を作るためのパッケージ
\setlength{\topmargin}{14mm} % 上下方向のマージン
\addtolength{\topmargin}{-1in} % 
\setlength{\oddsidemargin}{11mm} % 左右方向のマージン
\addtolength{\oddsidemargin}{-1in} % 
\setlength{\textwidth}{154mm} % B6 用
\setlength{\textheight}{108mm} % B6 用
\setlength{\headsep}{0mm} % 
\setlength{\headheight}{0mm} % 
\setlength{\topskip}{0mm} % 
\setlength{\parskip}{0pt} % 
\def\theenumi{\Kanji{enumi}} % 箇条書きのフォーマットを漢数字に変更
\parindent = 0pt % 段落下げしない
\pagestyle{empty} % すべてのページ番号を消去
% \renewcommand{\baselinestretch}{0.9} % 行間の倍率
 % B6 用テンプレート読み込み

\begin{document}
% begin header
%%%%% タイトルと作者 ここから %%%%%
\begin{minipage}[c]{0.7\hsize} % タイトルは上から 7 割
    \begin{center}
    % begin title
        {\LARGE
            北斗遙かに % タイトルを入れる
        }
        {\small 
            (昭和六十二年度寮歌) % 年などを入れる
        }
    % end title
    \end{center}
\end{minipage}
\begin{minipage}[c]{0.3\hsize} % 作歌作曲は上から 3 割
    \begin{flushright} % 下寄せにする
        % begin name
        佐久間朗君 作歌\\吉田崇君 作曲 % 作歌・作曲者
        % end name
    \end{flushright}
\end{minipage}
%%%%% タイトルと作者 ここまで %%%%%
% (1,2,3,4 了あり)
% end header

% begin length
\vspace{1.5em} % タイトル, 作者と歌詞の間に隙間を設ける
\newcommand{\linespace}{0.5em} % 行間の設定
\newcommand{\blocksize}{0.5\hsize} % 段組間の設定
\newcommand{\itemmargin}{3em} % 曲番の位置調整の長さ
% end length
% begin body
%%%%% 歌詞 ここから %%%%%
\begin{enumerate} % 番号の箇条書き ここから
    \setlength{\itemindent}{\itemmargin} % 曲番の位置調整
    \begin{minipage}[c]{\blocksize}
    
        \vspace{\linespace}
        \item~\\
        % 1.
        \ruby{北斗遙}{}かに\ruby{広}{}がれる\\
        \ruby{波濤煌}{}く\ruby{水平線}{}\\
        \ruby{移}{}り\ruby{行}{}く\ruby{天水渡}{}る\ruby{朔風}{}\\
        \ruby{厳冬}{}の\ruby{記憶}{}を\ruby{留}{}めれど\\
        \ruby{新緑萌}{}す\ruby{曠野}{}には\\
        \ruby{若}{}き\ruby{生命}{}の\ruby{息吹}{}あり\\
        \ruby{嗚呼季節}{}の\ruby{芳香満}{}つ\\
        この\ruby{北}{}の\ruby{大地}{}に\\
        \ruby{新}{}たなる\ruby{夢}{}を\ruby{得}{}て\\
        \ruby{希望}{}かなえん
        
    \end{minipage}
    \begin{minipage}[c]{\blocksize}
        
        \vspace{\linespace}
        \item~\\
        % 2.
        \ruby{北斗清}{}かに\ruby{見}{}はるかす\\
        \ruby{紺碧}{}に\ruby{滲}{}む\ruby{大空}{}に\\
        \ruby{輝}{}く\ruby{光彩燦爛}{}と\\
        \ruby{短}{}き\ruby{盛夏}{}を\ruby{彩}{}りて\\
        \ruby{涼風}{}そよぐ\ruby{窓下}{}には\\
        \ruby{緑滴}{}る\ruby{原始林}{}\\
        \ruby{嗚呼季節}{}の\ruby{恵}{}み\ruby{満}{}つ\\
        この\ruby{地}{}の\ruby{大地}{}に\\
        \ruby{新}{}しき\ruby{情熱}{}もて\\
        \ruby{真理求}{}めん
        
    \end{minipage}
    \begin{minipage}[c]{\blocksize}
        
        \vspace{\linespace}
        \item~\\
        % 3.
        \ruby{北斗豊}{}かに\ruby{色}{}づける\\
        \ruby{黄金色}{}の\ruby{大沃野}{}\\
        \ruby{充足誘}{}う\ruby{黄昏}{}に\\
        \ruby{遠}{}く\ruby{彼方}{}を\ruby{見渡}{}せば\\
        \ruby{牧場}{}を\ruby{疾走}{}る\ruby{若駒}{}の\\
        \ruby{荒土蹴散}{}らすその\ruby{雄姿}{}\\
        \ruby{嗚呼季節}{}の\ruby{実}{}り\ruby{満}{}つ\\
        この\ruby{北}{}の\ruby{大地}{}に\\
        \ruby{新}{}しき\ruby{力得}{}て\\
        \ruby{正義貫}{}かん
        
    \end{minipage}
    \begin{minipage}[c]{\blocksize}
        
        \vspace{\linespace}
        \item~\\
        % 4.
        \ruby{北斗果}{}てなく\ruby{包}{}み\ruby{込}{}む\\
        \ruby{荒}{}び\ruby{飛}{}び\ruby{散}{}る\ruby{猛吹雪}{}\\
        \ruby{物皆埋}{}み\ruby{凍}{}てつかせ\\
        \ruby{我}{}らが\ruby{前途閉}{}ざせども\\
        ひたすら\ruby{拓}{}くその\ruby{迪}{}に\\
        \ruby{放歌笑声絶}{}ゆるなし\\
        \ruby{嗚呼季節}{}の\ruby{憂愁満}{}つ\\
        この\ruby{北}{}の\ruby{大地}{}に\\
        \ruby{新}{}しき\ruby{意識}{}もて\\
        \ruby{自治}{}を\ruby{築}{}かん
    
    \end{minipage}
\end{enumerate} % 番号の箇条書き ここまで
%%%%% 歌詞 ここまで %%%%%
% end body

\end{document}

\documentclass[10pt,b5j]{tarticle} % B6 縦書き
% \documentclass[10pt,b5j]{tarticle} % B6 縦書き
\AtBeginDvi{\special{papersize=128mm,182mm}} % B6 用用紙サイズ
\usepackage{otf} % Unicode で字を入力するのに必要なパッケージ
\usepackage[size=b6j]{bxpapersize} % B6 用紙サイズを指定
\usepackage[dvipdfmx]{graphicx} % 画像を挿入するためのパッケージ
\usepackage[dvipdfmx]{color} % 色をつけるためのパッケージ
\usepackage{pxrubrica} % ルビを振るためのパッケージ
\usepackage{multicol} % 複数段組を作るためのパッケージ
\setlength{\topmargin}{14mm} % 上下方向のマージン
\addtolength{\topmargin}{-1in} % 
\setlength{\oddsidemargin}{11mm} % 左右方向のマージン
\addtolength{\oddsidemargin}{-1in} % 
\setlength{\textwidth}{154mm} % B6 用
\setlength{\textheight}{108mm} % B6 用
\setlength{\headsep}{0mm} % 
\setlength{\headheight}{0mm} % 
\setlength{\topskip}{0mm} % 
\setlength{\parskip}{0pt} % 
\def\labelenumi{\theenumi、} % 箇条書きのフォーマット
\parindent = 0pt % 段落下げしない

 % B6 用テンプレート読み込み

\begin{document}
% begin header
%%%%% タイトルと作者 ここから %%%%%
\begin{minipage}[c]{0.7\hsize} % タイトルは上から 7 割
    \begin{center}
    % begin title
        {\LARGE
            時潮の流転 % タイトルを入れる
        }
        {\small 
            (昭和十四年寮歌) % 年などを入れる
        }
    % end title
    \end{center}
\end{minipage}
\begin{minipage}[c]{0.3\hsize} % 作歌作曲は上から 3 割
    \begin{flushright} % 下寄せにする
        % begin name
        望月真三君 作歌\\竹村伸一君 作曲 % 作歌・作曲者
        % end name
    \end{flushright}
\end{minipage}
%%%%% タイトルと作者 ここまで %%%%%
% (1,2,6 了あり)
% end header

% begin length
\vspace{1.5em} % タイトル, 作者と歌詞の間に隙間を設ける
\newcommand{\linespace}{0.5em} % 行間の設定
\newcommand{\blocksize}{0.5\hsize} % 段組間の設定
\newcommand{\itemmargin}{3em} % 曲番の位置調整の長さ
% end length
% begin body
%%%%% 歌詞 ここから %%%%%
\begin{enumerate} % 番号の箇条書き ここから
    \setlength{\itemindent}{\itemmargin} % 曲番の位置調整
    \begin{minipage}[c]{\blocksize}
    
        \vspace{\linespace}
        \item~\\
        % 1.
        \ruby{時}{とき}\ruby{潮}{しお}の\ruby{流転}{るてん}\ruby{淙々}{そうそう}と\\
        \ruby{四季}{しき}\ruby{乾坤}{けんこん}に\ruby{巡}{めぐ}り\ruby{立}{た}つ\\
        \ruby{去来}{きょらい}\ruby{常}{つね}なく\ruby{人}{ひと}\ruby{変}{かわ}り\\
        \ruby{有情}{うじょう}\ruby{無為}{むい}の\ruby{時}{とき}\ruby{鐘}{かね}の\ruby{音}{おと}に\\
        \ruby{孤城}{こじょう}の\ruby{爽春}{}は\ruby{未}{いま}だ\ruby{浅}{あさ}し
        
    \end{minipage}
    \begin{minipage}[c]{\blocksize}
        
        \vspace{\linespace}
        \item~\\
        % 2.
        \ruby{遠}{とお}く\ruby{流離}{りゅうり}の\ruby{春}{はる}に\ruby{来}{き}て\\
        \ruby{此}{}の\ruby{高楼}{こうろう}に\ruby{春愁}{しゅんしゅう}ひつつ\\
        \ruby{郭公}{かっこう}\ruby{鳥}{とり}の\ruby{鳴}{な}くさへも\\
        \ruby{多感}{たかん}の\ruby{児}{こ}\ruby{等}{とう}の\ruby{情}{じょう}\ruby{懐}{ふところ}\ruby{熱}{あつ}く\\
        \ruby{懐古}{かいこ}の\ruby{涙}{なみだ}\ruby{溢}{}るべし
        
    \end{minipage}
    \begin{minipage}[c]{\blocksize}
        
        \vspace{\linespace}
        \item~\\
        % 3.
        \ruby{真}{しん}\ruby{日}{び}\ruby{澄}{す}む\ruby{北}{きた}の\ruby{蒼穹}{そうきゅう}はるか\\
        \ruby{飛燕}{ひえん}ひとたび\ruby{音}{おん}に\ruby{鳴}{な}けば\\
        \ruby{桃李}{とうり}の\ruby{華}{はな}\ruby{影}{かげ}は\ruby{痩}{や}せゆきて\\
        あはれ\ruby{旅寝}{たびね}の\ruby{若}{わか}き\ruby{遊子}{ゆうし}よ\\
        \ruby{帰}{き}\ruby{南}{みなみ}の\ruby{郷愁}{きょうしゅう}しきりなり
        
    \end{minipage}
    \begin{minipage}[c]{\blocksize}
        
        \vspace{\linespace}
        \item~\\
        % 4.
        \ruby{夕陽}{ゆうひ}\ruby{西}{にし}に\ruby{落}{お}ち\ruby{行}{ゆ}けば\\
        \ruby{白樺}{しらかんば}\ruby{林}{りん}\ruby{朱}{しゅ}に\ruby{染}{し}み\\
        \ruby{暮秋}{ぼしゅう}の\ruby{颯}{}は\ruby{飄々}{ひょうひょう}と\\
        \ruby{時艱}{じかん}を\ruby{憂}{うれ}ふ\ruby{国}{くに}の\ruby{子}{こ}の\\
        \ruby{悲腸}{}の\ruby{声}{こえ}に\ruby{似}{に}たるかな
        
    \end{minipage}
    \begin{minipage}[c]{\blocksize}
        
        \vspace{\linespace}
        \item~\\
        % 5.
        \ruby{北斗}{ほくと}\ruby{地平}{ちへい}に\ruby{揺曳}{ようえい}ぐとき\\
        \ruby{天地}{てんち}の\ruby{四}{よん}\ruby{大}{だい}\ruby{霜}{しも}と\ruby{凝}{こ}り\\
        \ruby{四}{よん}\ruby{寮}{りょう}の\ruby{高}{こう}\ruby{夢}{ゆめ}も\ruby{凍}{い}てつきて\\
        ほがらほがらの\ruby{朝}{あさ}ぼらけ\\
        \ruby{帰雁}{きがん}の\ruby{孤影}{こえい}よ\ruby{月}{つき}に\ruby{飛}{と}ぶ
        
    \end{minipage}
    \begin{minipage}[c]{\blocksize}
        
        \vspace{\linespace}
        \item~\\
        % 6.
        \ruby{明日}{あした}\ruby{別}{わか}れ\ruby{行}{ゆ}く\ruby{旅人}{たびびと}の\\
        \ruby{春}{はる}の\ruby{夕}{ゆう}べの\ruby{宴}{うたげ}\ruby{遊}{}かな\\
        かへらぬ\ruby{絢}{あや}\ruby{夢}{ゆめ}をしのびつつ\\
        \ruby{生命}{せいめい}の\ruby{故郷}{こきょう}と\ruby{慨嘆}{がいたん}きしも\\
        すでに\ruby{三}{さん}\ruby{星霜}{せいそう}の\ruby{草枕}{くさまくら}
    
    \end{minipage}
\end{enumerate} % 番号の箇条書き ここまで
%%%%% 歌詞 ここまで %%%%%
% end body

\end{document}

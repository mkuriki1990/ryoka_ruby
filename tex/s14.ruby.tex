\documentclass[10pt,b5j]{tarticle} % B6 縦書き
% \documentclass[10pt,b5j]{tarticle} % B6 縦書き
\AtBeginDvi{\special{papersize=128mm,182mm}} % B6 用用紙サイズ
\usepackage{otf} % Unicode で字を入力するのに必要なパッケージ
\usepackage[size=b6j]{bxpapersize} % B6 用紙サイズを指定
\usepackage[dvipdfmx]{graphicx} % 画像を挿入するためのパッケージ
\usepackage[dvipdfmx]{color} % 色をつけるためのパッケージ
\usepackage{pxrubrica} % ルビを振るためのパッケージ
\usepackage{plext} % 漢数字の enumerate を使うためのパッケージ
\usepackage{multicol} % 複数段組を作るためのパッケージ
\setlength{\topmargin}{14mm} % 上下方向のマージン
\addtolength{\topmargin}{-1in} % 
\setlength{\oddsidemargin}{11mm} % 左右方向のマージン
\addtolength{\oddsidemargin}{-1in} % 
\setlength{\textwidth}{154mm} % B6 用
\setlength{\textheight}{108mm} % B6 用
\setlength{\headsep}{0mm} % 
\setlength{\headheight}{0mm} % 
\setlength{\topskip}{0mm} % 
\setlength{\parskip}{0pt} % 
\def\theenumi{\Kanji{enumi}} % 箇条書きのフォーマットを漢数字に変更
\parindent = 0pt % 段落下げしない
\pagestyle{empty} % すべてのページ番号を消去
% \renewcommand{\baselinestretch}{0.9} % 行間の倍率
 % B6 用テンプレート読み込み

\begin{document}
% begin header
%%%%% タイトルと作者 ここから %%%%%
\begin{minipage}[c]{0.7\hsize} % タイトルは上から 7 割
    \begin{center}
    % begin title
        {\LARGE
            時潮の流転 % タイトルを入れる
        }
        {\small 
            (昭和十四年寮歌) % 年などを入れる
        }
    % end title
    \end{center}
\end{minipage}
\begin{minipage}[c]{0.3\hsize} % 作歌作曲は上から 3 割
    \begin{flushright} % 下寄せにする
        % begin name
        望月真三君 作歌\\竹村伸一君 作曲 % 作歌・作曲者
        % end name
    \end{flushright}
\end{minipage}
%%%%% タイトルと作者 ここまで %%%%%
% (1,2,6 了あり)
% end header

% begin length
\vspace{1.5em} % タイトル, 作者と歌詞の間に隙間を設ける
\newcommand{\linespace}{0.5em} % 行間の設定
\newcommand{\blocksize}{0.5\hsize} % 段組間の設定
\newcommand{\itemmargin}{6em} % 曲番の位置調整の長さ
% end length
% begin body
%%%%% 歌詞 ここから %%%%%
\begin{enumerate} % 番号の箇条書き ここから
    \setlength{\itemindent}{\itemmargin} % 曲番の位置調整
    \begin{minipage}[c]{\blocksize}
    
        \vspace{\linespace}
        \item~\\
        % 1.
        \ruby{時潮}{}の\ruby{流転淙々}{}と\\
        \ruby{四季乾坤}{}に\ruby{巡}{}り\ruby{立}{}つ\\
        \ruby{去来常}{}なく\ruby{人変}{}り\\
        \ruby{有情無為}{}の\ruby{時鐘}{}の\ruby{音}{}に\\
        \ruby{孤城}{}の\ruby{爽春}{}は\ruby{未}{}だ\ruby{浅}{}し
        
        \vspace{\linespace}
        \item~\\
        % 2.
        \ruby{遠}{}く\ruby{流離}{}の\ruby{春}{}に\ruby{来}{}て\\
        \ruby{此}{}の\ruby{高楼}{}に\ruby{春愁}{}ひつつ\\
        \ruby{郭公鳥}{}の\ruby{鳴}{}くさへも\\
        \ruby{多感}{}の\ruby{児等}{}の\ruby{情懐熱}{}く\\
        \ruby{懐古}{}の\ruby{涙溢}{}るべし
        
        \vspace{\linespace}
        \item~\\
        % 3.
        \ruby{真日澄}{}む\ruby{北}{}の\ruby{蒼穹}{}はるか\\
        \ruby{飛燕}{}ひとたび\ruby{音}{}に\ruby{鳴}{}けば\\
        \ruby{桃李}{}の\ruby{華影}{}は\ruby{痩}{}せゆきて\\
        あはれ\ruby{旅寝}{}の\ruby{若}{}き\ruby{遊子}{}よ\\
        \ruby{帰南}{}の\ruby{郷愁}{}しきりなり
        
        \vspace{\linespace}
        \item~\\
        % 4.
        \ruby{夕陽西}{}に\ruby{落}{}ち\ruby{行}{}けば\\
        \ruby{白樺林朱}{}に\ruby{染}{}み\\
        \ruby{暮秋}{}の\ruby{颯}{}は\ruby{飄々}{}と\\
        \ruby{時艱}{}を\ruby{憂}{}ふ\ruby{国}{}の\ruby{子}{}の\\
        \ruby{悲腸}{}の\ruby{声}{}に\ruby{似}{}たるかな
        
        \vspace{\linespace}
        \item~\\
        % 5.
        \ruby{北斗地平}{}に\ruby{揺曳}{}ぐとき\\
        \ruby{天地}{}の\ruby{四大霜}{}と\ruby{凝}{}り\\
        \ruby{四寮}{}の\ruby{高夢}{}も\ruby{凍}{}てつきて\\
        ほがらほがらの\ruby{朝}{}ぼらけ\\
        \ruby{帰雁}{}の\ruby{孤影}{}よ\ruby{月}{}に\ruby{飛}{}ぶ
        
        \vspace{\linespace}
        \item~\\
        % 6.
        \ruby{明日別}{}れ\ruby{行}{}く\ruby{旅人}{}の\\
        \ruby{春}{}の\ruby{夕}{}べの\ruby{宴遊}{}かな\\
        かへらぬ\ruby{絢夢}{}をしのびつつ\\
        \ruby{生命}{}の\ruby{故郷}{}と\ruby{慨嘆}{}きしも\\
        すでに\ruby{三星霜}{}の\ruby{草枕}{}
    
    \end{minipage}
\end{enumerate} % 番号の箇条書き ここまで
%%%%% 歌詞 ここまで %%%%%
% end body

\end{document}

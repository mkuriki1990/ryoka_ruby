\documentclass[10pt,b5j]{tarticle} % B6 縦書き
% \documentclass[10pt,b5j]{tarticle} % B6 縦書き
\AtBeginDvi{\special{papersize=128mm,182mm}} % B6 用用紙サイズ
\usepackage{otf} % Unicode で字を入力するのに必要なパッケージ
\usepackage[size=b6j]{bxpapersize} % B6 用紙サイズを指定
\usepackage[dvipdfmx]{graphicx} % 画像を挿入するためのパッケージ
\usepackage[dvipdfmx]{color} % 色をつけるためのパッケージ
\usepackage{pxrubrica} % ルビを振るためのパッケージ
\usepackage{multicol} % 複数段組を作るためのパッケージ
\setlength{\topmargin}{14mm} % 上下方向のマージン
\addtolength{\topmargin}{-1in} % 
\setlength{\oddsidemargin}{11mm} % 左右方向のマージン
\addtolength{\oddsidemargin}{-1in} % 
\setlength{\textwidth}{154mm} % B6 用
\setlength{\textheight}{108mm} % B6 用
\setlength{\headsep}{0mm} % 
\setlength{\headheight}{0mm} % 
\setlength{\topskip}{0mm} % 
\setlength{\parskip}{0pt} % 
\def\labelenumi{\theenumi、} % 箇条書きのフォーマット
\parindent = 0pt % 段落下げしない

 % B6 用テンプレート読み込み

\begin{document}
% begin header
%%%%% タイトルと作者 ここから %%%%%
\begin{minipage}[c]{0.7\hsize} % タイトルは上から 7 割
    \begin{center}
    % begin title
        {\LARGE
            手をとりて美しき国を % タイトルを入れる
        }
        {\small 
            (昭和28年寮歌) % 年などを入れる
        }
    % end title
    \end{center}
\end{minipage}
\begin{minipage}[c]{0.3\hsize} % 作歌作曲は上から 3 割
    \begin{flushright} % 下寄せにする
        % begin name
        山本玉樹君 作歌\\三河勝彦君 作曲 % 作歌・作曲者
        % end name
    \end{flushright}
\end{minipage}
%%%%% タイトルと作者 ここまで %%%%%
% (1,2 繰り返しなし)
% end header

% begin body
\vspace{1.5em} % タイトル, 作者と歌詞の間に隙間を設ける
\newcommand{\linespace}{0.5em} % 行間の設定
\newcommand{\blocksize}{0.5\hsize} % 段組間の設定
%%%%% 歌詞 ここから %%%%%
% begin lilycs
\begin{enumerate} % 番号の箇条書き ここから
    \begin{minipage}[c]{\blocksize}
    
        \vspace{\linespace}
        \item
        % 1.
        \ruby{倒}{}れたる\ruby{友}{}の\ruby{姿}{}を\\
        \ruby{忘}{}るまじ\ruby{我}{}らが\ruby{胸}{}に\\
        \ruby{恐}{}ろしき\ruby{雲空}{}に\ruby{充}{}ち\\
        けがれたる\ruby{祖国}{}の\ruby{山河}{}に\\
        \ruby{新}{}しき\ruby{緑}{}の\ruby{息吹}{}が\\
        \ruby{若者}{}の\ruby{槌音}{}に\ruby{和}{}し\\
        もろ\ruby{人}{}の\ruby{幸深}{}めつつ\\
        この\ruby{町}{}にこだます\ruby{日}{}まで
        
        \vspace{\linespace}
        \item
        % 2.
        \ruby{湧}{}き\ruby{出}{}でよ\ruby{新}{}らしき\ruby{歌}{}\\
        \ruby{消}{}すまじ\ruby{自由}{}の\ruby{歌}{}を\\
        わだつみの\ruby{声}{}をばひめて\\
        \ruby{去}{}り\ruby{果}{}てし\ruby{若}{}き\ruby{生命}{}に\\
        たくましき\ruby{若}{}き\ruby{鼓動}{}が\\
        \ruby{美}{}しき\ruby{歌声}{}に\ruby{和}{}し\\
        \ruby{平和}{}なる\ruby{国}{}を\ruby{築}{}くと\\
        \ruby{海}{}こえてこだます\ruby{日}{}まで
    
    \end{minipage}
\end{enumerate} % 番号の箇条書き ここまで
% end lilycs
%%%%% 歌詞 ここまで %%%%%
% end body

\end{document}

\documentclass[10pt,b5j]{tarticle} % B6 縦書き
% \documentclass[10pt,b5j]{tarticle} % B6 縦書き
\AtBeginDvi{\special{papersize=128mm,182mm}} % B6 用用紙サイズ
\usepackage{otf} % Unicode で字を入力するのに必要なパッケージ
\usepackage[size=b6j]{bxpapersize} % B6 用紙サイズを指定
\usepackage[dvipdfmx]{graphicx} % 画像を挿入するためのパッケージ
\usepackage[dvipdfmx]{color} % 色をつけるためのパッケージ
\usepackage{pxrubrica} % ルビを振るためのパッケージ
\usepackage{multicol} % 複数段組を作るためのパッケージ
\setlength{\topmargin}{14mm} % 上下方向のマージン
\addtolength{\topmargin}{-1in} % 
\setlength{\oddsidemargin}{11mm} % 左右方向のマージン
\addtolength{\oddsidemargin}{-1in} % 
\setlength{\textwidth}{154mm} % B6 用
\setlength{\textheight}{108mm} % B6 用
\setlength{\headsep}{0mm} % 
\setlength{\headheight}{0mm} % 
\setlength{\topskip}{0mm} % 
\setlength{\parskip}{0pt} % 
\def\labelenumi{\theenumi、} % 箇条書きのフォーマット
\parindent = 0pt % 段落下げしない

 % B6 用テンプレート読み込み

\begin{document}
% begin header
%%%%% タイトルと作者 ここから %%%%%
\begin{minipage}[c]{0.7\hsize} % タイトルは上から 7 割
    \begin{center}
    % begin title
        {\LARGE
            嗚呼茫々の % タイトルを入れる
        }
        {\small 
            (昭和十一年寮歌) % 年などを入れる
        }
    % end title
    \end{center}
\end{minipage}
\begin{minipage}[c]{0.3\hsize} % 作歌作曲は上から 3 割
    \begin{flushright} % 下寄せにする
        % begin name
        宍戸昌夫君 作歌\\村岡五郎君 作曲 % 作歌・作曲者
        % end name
    \end{flushright}
\end{minipage}
%%%%% タイトルと作者 ここまで %%%%%
% (1 繰り返しなし)
% end header

% begin length
\vspace{1.5em} % タイトル, 作者と歌詞の間に隙間を設ける
\newcommand{\linespace}{0.5em} % 行間の設定
\newcommand{\blocksize}{0.5\hsize} % 段組間の設定
\newcommand{\itemmargin}{3em} % 曲番の位置調整の長さ
% end length
% begin body
%%%%% 歌詞 ここから %%%%%
\begin{enumerate} % 番号の箇条書き ここから
    \setlength{\itemindent}{\itemmargin} % 曲番の位置調整
    \begin{minipage}[c]{\blocksize}
    
        \vspace{\linespace}
        \item~\\
         \ruby{楡}{にれ}\ruby{陵}{りょう}\ruby{謳春}{}\ruby{賦}{ふ}\\
        \ruby{吾}{われ}\ruby{等}{とう}が\ruby{三}{さん}\ruby{年}{ねん}を\ruby{契}{ちぎ}る\ruby{絢爛}{けんらん}の\\
        その\ruby{饗宴}{きょうえん}はげに\ruby{過}{す}ぎ\ruby{易}{やす}し。\\
        \ruby{然}{しか}れども\ruby{見}{み}ずや\ruby{穹北}{}に\ruby{瞬}{まばた}く\ruby{星}{ほし}\ruby{斗}{と}\\
        \ruby{永久}{えいきゅう}に\ruby{曇}{くも}りなく、\ruby{雲}{くも}とまがふ\\
        \ruby{万朶}{ばんだ}の\ruby{桜花}{おうか}\ruby{久遠}{くおん}に\ruby{萎}{な}えざるを。\\
        \ruby{寮}{りょう}\ruby{友}{とも}よ\ruby{徒}{と}らに\ruby{明日}{あした}の\ruby{運命}{うんめい}を\\
        \ruby{歎}{}かんよりは\ruby{楡}{にれ}\ruby{林}{りん}に\ruby{篝火}{かがりび}を\ruby{焚}{た}きて、\\
        \ruby{去}{さ}りては\ruby{再}{ふたた}び\ruby{帰}{かえ}らざる\\
        \ruby{若}{わか}き\ruby{日}{ひ}の\ruby{感激}{かんげき}を\ruby{謳歌}{おうか}はん。
        
    \end{minipage}
    \begin{minipage}[c]{\blocksize}
        
        \vspace{\linespace}
        \item~\\
        % 1.
        \ruby{嗚呼}{ああ}\ruby{茫々}{ぼうぼう}の\ruby{大}{だい}\ruby{曠野}{あらの}\\
        \ruby{先人}{せんじん}ここに\ruby{芟}{}りて\\
        \ruby{建}{た}てし\ruby{自由}{じゆう}と\ruby{自治}{じち}の\ruby{城}{しろ}\\
        その\ruby{源}{みなもと}は\ruby{遠}{とお}くして\\
        \ruby{濁世}{だくせい}\ruby{叱咤}{しった}す\ruby{六}{ろく}\ruby{十}{じゅう}\ruby{年}{ねん}の\\
        \ruby{苔}{こけ}むす\ruby{青史}{せいし}\ruby{誇}{ほこ}りなん
        
    \end{minipage}
    \begin{minipage}[c]{\blocksize}
        
        \vspace{\linespace}
        \item~\\
        % 2.
        \ruby{老}{ろう}\ruby{桜}{さくら}の\ruby{蔭}{かげ}や\ruby{北辰}{ほくしん}の\ruby{下}{しも}\\
        \ruby{少時}{しょうじ}\ruby{旅寝}{たびね}の\ruby{若}{わか}き\ruby{子}{こ}が\\
        \ruby{自治}{じち}\ruby{燈}{}かかげ\ruby{聖}{きよし}\ruby{鐘}{かね}うちて\\
        \ruby{惰眠}{だみん}れる\ruby{魂}{たましい}を\ruby{覚醒}{かくせい}すべく\\
        \ruby{降魔}{ごうま}\ruby{剣}{けん}かざすとき\\
        \ruby{狂}{きょう}へる\ruby{飃}{}も\ruby{声}{こえ}ひそむ
        
    \end{minipage}
    \begin{minipage}[c]{\blocksize}
        
        \vspace{\linespace}
        \item~\\
        % 3.
        さはれ\ruby{今宵}{こよい}の\ruby{我}{わ}が\ruby{寮}{りょう}\\
        「\ruby{人生}{じんせい}\ruby{意気}{いき}」に\ruby{集}{つど}い\ruby{来}{きた}し\\
        \ruby{結}{むす}びてとけぬ\ruby{友垣}{ともがき}が\\
        \ruby{光明}{こうみょう}と\ruby{権威}{けんい}\ruby{謳}{}ふとき\\
        \ruby{星屑}{ほしくず}\ruby{原始}{げんし}\ruby{林}{りん}に\ruby{輝}{かがや}きて\\
        \ruby{流転}{るてん}の\ruby{相}{そう}を\ruby{示}{しめ}すなり
        
    \end{minipage}
    \begin{minipage}[c]{\blocksize}
        
        \vspace{\linespace}
        \item~\\
        % 4.
        ああ\ruby{感激}{かんげき}の\ruby{美酒}{びしゅ}は\\
        \ruby{廻}{まわ}りて\ruby{早}{はや}きその\ruby{三}{さん}\ruby{年}{ねん}\\
        \ruby{希望}{きぼう}の\ruby{光}{ひかり}\ruby{恵}{めぐ}めては\\
        \ruby{楡}{にれ}\ruby{林}{りん}にかはす\ruby{盃}{さかずき}に\\
        \ruby{啓示}{けいじ}の\ruby{翳}{かげ}を\ruby{泛}{}べつつ\\
        \ruby{男}{おとこ}の\ruby{子}{こ}の\ruby{眸}{ひとみ}に\ruby{涙}{なみだ}あり
    
    \end{minipage}
\end{enumerate} % 番号の箇条書き ここまで
%%%%% 歌詞 ここまで %%%%%
% end body

\end{document}

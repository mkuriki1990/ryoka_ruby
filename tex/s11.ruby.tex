\documentclass[10pt,b5j]{tarticle} % B6 縦書き
% \documentclass[10pt,b5j]{tarticle} % B6 縦書き
\AtBeginDvi{\special{papersize=128mm,182mm}} % B6 用用紙サイズ
\usepackage{otf} % Unicode で字を入力するのに必要なパッケージ
\usepackage[size=b6j]{bxpapersize} % B6 用紙サイズを指定
\usepackage[dvipdfmx]{graphicx} % 画像を挿入するためのパッケージ
\usepackage[dvipdfmx]{color} % 色をつけるためのパッケージ
\usepackage{pxrubrica} % ルビを振るためのパッケージ
\usepackage{multicol} % 複数段組を作るためのパッケージ
\setlength{\topmargin}{14mm} % 上下方向のマージン
\addtolength{\topmargin}{-1in} % 
\setlength{\oddsidemargin}{11mm} % 左右方向のマージン
\addtolength{\oddsidemargin}{-1in} % 
\setlength{\textwidth}{154mm} % B6 用
\setlength{\textheight}{108mm} % B6 用
\setlength{\headsep}{0mm} % 
\setlength{\headheight}{0mm} % 
\setlength{\topskip}{0mm} % 
\setlength{\parskip}{0pt} % 
\def\labelenumi{\theenumi、} % 箇条書きのフォーマット
\parindent = 0pt % 段落下げしない

 % B6 用テンプレート読み込み

\begin{document}
% begin header
%%%%% タイトルと作者 ここから %%%%%
\begin{minipage}[c]{0.7\hsize} % タイトルは上から 7 割
    \begin{center}
    % begin title
        {\LARGE
            嗚呼茫々の % タイトルを入れる
        }
        {\small 
            (昭和十一年寮歌) % 年などを入れる
        }
    % end title
    \end{center}
\end{minipage}
\begin{minipage}[c]{0.3\hsize} % 作歌作曲は上から 3 割
    \begin{flushright} % 下寄せにする
        % begin name
        宍戸昌夫君 作歌\\村岡五郎君 作曲 % 作歌・作曲者
        % end name
    \end{flushright}
\end{minipage}
%%%%% タイトルと作者 ここまで %%%%%
% (1 繰り返しなし)
% end header

% begin length
\vspace{1.5em} % タイトル, 作者と歌詞の間に隙間を設ける
\newcommand{\linespace}{0.5em} % 行間の設定
\newcommand{\blocksize}{0.5\hsize} % 段組間の設定
\newcommand{\itemmargin}{3em} % 曲番の位置調整の長さ
% end length
% begin body
%%%%% 歌詞 ここから %%%%%
\begin{enumerate} % 番号の箇条書き ここから
    \setlength{\itemindent}{\itemmargin} % 曲番の位置調整
    \begin{minipage}[c]{\blocksize}
    
        \vspace{\linespace}
        \item~\\
         \ruby{楡陵謳春賦}{}\\
        \ruby{吾等}{}が\ruby{三年}{}を\ruby{契}{}る\ruby{絢爛}{}の\\
        その\ruby{饗宴}{}はげに\ruby{過}{}ぎ\ruby{易}{}し。\\
        \ruby{然}{}れども\ruby{見}{}ずや\ruby{穹北}{}に\ruby{瞬}{}く\ruby{星斗}{}\\
        \ruby{永久}{}に\ruby{曇}{}りなく、\ruby{雲}{}とまがふ\\
        \ruby{万朶}{}の\ruby{桜花久遠}{}に\ruby{萎}{}えざるを。\\
        \ruby{寮友}{}よ\ruby{徒}{}らに\ruby{明日}{}の\ruby{運命}{}を\\
        \ruby{歎}{}かんよりは\ruby{楡林}{}に\ruby{篝火}{}を\ruby{焚}{}きて、\\
        \ruby{去}{}りては\ruby{再}{}び\ruby{帰}{}らざる\\
        \ruby{若}{}き\ruby{日}{}の\ruby{感激}{}を\ruby{謳歌}{}はん。
        
    \end{minipage}
    \begin{minipage}[c]{\blocksize}
        
        \vspace{\linespace}
        \item~\\
        % 1.
        \ruby{嗚呼茫々}{}の\ruby{大曠野}{}\\
        \ruby{先人}{}ここに\ruby{芟}{}りて\\
        \ruby{建}{}てし\ruby{自由}{}と\ruby{自治}{}の\ruby{城}{}\\
        その\ruby{源}{}は\ruby{遠}{}くして\\
        \ruby{濁世叱咤}{}す\ruby{六十年}{}の\\
        \ruby{苔}{}むす\ruby{青史誇}{}りなん
        
    \end{minipage}
    \begin{minipage}[c]{\blocksize}
        
        \vspace{\linespace}
        \item~\\
        % 2.
        \ruby{老桜}{}の\ruby{蔭}{}や\ruby{北辰}{}の\ruby{下}{}\\
        \ruby{少時旅寝}{}の\ruby{若}{}き\ruby{子}{}が\\
        \ruby{自治燈}{}かかげ\ruby{聖鐘}{}うちて\\
        \ruby{惰眠}{}れる\ruby{魂}{}を\ruby{覚醒}{}すべく\\
        \ruby{降魔剣}{}かざすとき\\
        \ruby{狂}{}へる\ruby{飃}{}も\ruby{声}{}ひそむ
        
    \end{minipage}
    \begin{minipage}[c]{\blocksize}
        
        \vspace{\linespace}
        \item~\\
        % 3.
        さはれ\ruby{今宵}{}の\ruby{我}{}が\ruby{寮}{}\\
        「\ruby{人生意気}{}」に\ruby{集}{}い\ruby{来}{}し\\
        \ruby{結}{}びてとけぬ\ruby{友垣}{}が\\
        \ruby{光明}{}と\ruby{権威謳}{}ふとき\\
        \ruby{星屑原始林}{}に\ruby{輝}{}きて\\
        \ruby{流転}{}の\ruby{相}{}を\ruby{示}{}すなり
        
    \end{minipage}
    \begin{minipage}[c]{\blocksize}
        
        \vspace{\linespace}
        \item~\\
        % 4.
        ああ\ruby{感激}{}の\ruby{美酒}{}は\\
        \ruby{廻}{}りて\ruby{早}{}きその\ruby{三年}{}\\
        \ruby{希望}{}の\ruby{光恵}{}めては\\
        \ruby{楡林}{}にかはす\ruby{盃}{}に\\
        \ruby{啓示}{}の\ruby{翳}{}を\ruby{泛}{}べつつ\\
        \ruby{男}{}の\ruby{子}{}の\ruby{眸}{}に\ruby{涙}{}あり
    
    \end{minipage}
\end{enumerate} % 番号の箇条書き ここまで
%%%%% 歌詞 ここまで %%%%%
% end body

\end{document}

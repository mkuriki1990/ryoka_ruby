\documentclass[10pt,b5j]{tarticle} % B6 縦書き
% \documentclass[10pt,b5j]{tarticle} % B6 縦書き
\AtBeginDvi{\special{papersize=128mm,182mm}} % B6 用用紙サイズ
\usepackage{otf} % Unicode で字を入力するのに必要なパッケージ
\usepackage[size=b6j]{bxpapersize} % B6 用紙サイズを指定
\usepackage[dvipdfmx]{graphicx} % 画像を挿入するためのパッケージ
\usepackage[dvipdfmx]{color} % 色をつけるためのパッケージ
\usepackage{pxrubrica} % ルビを振るためのパッケージ
\usepackage{multicol} % 複数段組を作るためのパッケージ
\setlength{\topmargin}{14mm} % 上下方向のマージン
\addtolength{\topmargin}{-1in} % 
\setlength{\oddsidemargin}{11mm} % 左右方向のマージン
\addtolength{\oddsidemargin}{-1in} % 
\setlength{\textwidth}{154mm} % B6 用
\setlength{\textheight}{108mm} % B6 用
\setlength{\headsep}{0mm} % 
\setlength{\headheight}{0mm} % 
\setlength{\topskip}{0mm} % 
\setlength{\parskip}{0pt} % 
\def\labelenumi{\theenumi、} % 箇条書きのフォーマット
\parindent = 0pt % 段落下げしない

 % B6 用テンプレート読み込み

\begin{document}
% begin header
%%%%% タイトルと作者 ここから %%%%%
\begin{minipage}[c]{0.7\hsize} % タイトルは上から 7 割
    \begin{center}
    % begin title
        {\LARGE
            芳香漂う % タイトルを入れる
        }
        {\small 
            (昭和四十二年第六十回記念祭歌) % 年などを入れる
        }
    % end title
    \end{center}
\end{minipage}
\begin{minipage}[c]{0.3\hsize} % 作歌作曲は上から 3 割
    \begin{flushright} % 下寄せにする
        % begin name
        稲田雅久君 作歌\\名田正信君 作曲 % 作歌・作曲者
        % end name
    \end{flushright}
\end{minipage}
%%%%% タイトルと作者 ここまで %%%%%
% (1,6 繰り返しなし)
% end header

% begin body
\vspace{1.5em} % タイトル, 作者と歌詞の間に隙間を設ける
\newcommand{\linespace}{0.5em} % 行間の設定
\newcommand{\blocksize}{0.5\hsize} % 段組間の設定
%%%%% 歌詞 ここから %%%%%
% begin lilycs
\begin{enumerate} % 番号の箇条書き ここから
    \begin{minipage}[c]{\blocksize}
    
        \vspace{\linespace}
        \item
        % 1.
        \ruby{芳香漂}{}うやわらかの\\
        \ruby{残}{}んの\ruby{春}{}の\ruby{夕間暮}{}\\
        \ruby{朧}{}にかかる\ruby{夕月}{}に\\
        \ruby{浮}{}かぶ\ruby{辛夷}{}の\ruby{花吹雪}{}\\
        ああ\ruby{鳴}{}り\ruby{止}{}みて\ruby{聞}{}えこぬ\\
        \ruby{色壮麗}{}の\ruby{鐘}{}の\ruby{音}{}は\\
        \ruby{六十路}{}の\ruby{夏}{}に\ruby{鳴}{}らざるや\\
        いま\ruby{黄昏}{}の\ruby{自治}{}の\ruby{庭}{}
        
        \vspace{\linespace}
        \item
        % 2.
        \ruby{細}{}き\ruby{羽音}{}も\ruby{秘}{}そやかの\\
        \ruby{蜉蝣闇}{}をかすめゆき\\
        \ruby{奔}{}る\ruby{流}{}れの\ruby{音}{}もなく\\
        まつよい\ruby{草}{}の\ruby{星}{}あかり\\
        ああ\ruby{死}{}に\ruby{絶}{}えて\ruby{泳}{}ぎこぬ\\
        \ruby{銀鱗}{}おどる\ruby{紅鮭}{}は\\
        \ruby{六十路}{}の\ruby{秋}{}に\ruby{溯}{}らずや\\
        いま\ruby{宵闇}{}の\ruby{自治}{}の\ruby{川}{}
        
        \vspace{\linespace}
        \item
        % 3.
        \ruby{風}{}に\ruby{棚引}{}く\ruby{軽}{}やかの\\
        \ruby{雲蒼空}{}の\ruby{朝}{}ぼらけ\\
        よぎる\ruby{秋津}{}の\ruby{影紅}{}く\\
        \ruby{残}{}んの\ruby{月}{}は\ruby{薄}{}れゆく\\
        ああ\ruby{舞}{}い\ruby{去}{}りて\ruby{渡}{}りこぬ\\
        \ruby{長}{}の\ruby{旅寝}{}の\ruby{雁}{}は\\
        \ruby{六十路}{}の\ruby{冬}{}に\ruby{還}{}らずや\\
        いま\ruby{有明}{}の\ruby{自治}{}の\ruby{原}{}
        
        \vspace{\linespace}
        \item
        % 4.
        \ruby{軒}{}に\ruby{麗}{}なる\ruby{銀}{}の\\
        \ruby{垂氷}{}に\ruby{映}{}る\ruby{灯}{}に\\
        \ruby{星影凍}{}みる\ruby{松}{}が\ruby{枝}{}を\\
        \ruby{散}{}るひとひらの\ruby{雪}{}の\ruby{花}{}\\
        ああ\ruby{枯}{}れ\ruby{果}{}てて\ruby{萠}{}しこぬ\\
        \ruby{野}{}も\ruby{狭}{}に\ruby{埋}{}もる\ruby{花}{}の\ruby{実}{}は\\
        \ruby{六十路}{}の\ruby{春}{}に\ruby{咲}{}かざるや\\
        いま\ruby{夜}{}も\ruby{更}{}けぬ\ruby{自治}{}の\ruby{舎}{}
        
        \vspace{\linespace}
        \item
        % 5.
        \ruby{露}{}に\ruby{滴}{}りぬ\ruby{生々}{}の\\
        \ruby{楡林}{}にねむる\ruby{夢醒}{}めて\\
        \ruby{牧場}{}におどる\ruby{朝}{}もやの\\
        さなかに\ruby{歌}{}う\ruby{夜明}{}の\ruby{鳥}{}\\
        \ruby{見}{}よ\ruby{紅}{}の\ruby{山}{}の\ruby{端}{}に\\
        \ruby{湧}{}き\ruby{立}{}つ\ruby{空}{}の\ruby{群雲}{}を\\
        つらぬきわたる\ruby{光}{}かな\\
        いま\ruby{六十歳}{}の\ruby{夜}{}は\ruby{明}{}けぬ
        
        \vspace{\linespace}
        \item
        % 6.
        \ruby{寮友}{}の\ruby{顔}{}に\ruby{篝火}{}の\\
        \ruby{炎}{}もわらう\ruby{記念祭}{}\\
        \ruby{歌}{}をうたわば\ruby{玉響}{}の\\
        \ruby{憂}{}さも\ruby{舞}{}い\ruby{飛}{}ぶ\ruby{火}{}の\ruby{粉}{}なり\\
        いざ\ruby{高}{}らかに\ruby{祭歌}{}\\
        はやる\ruby{太鼓}{}の\ruby{轟}{}きは\\
        \ruby{夜空}{}を\ruby{深}{}く\ruby{駆}{}け\ruby{抜}{}けて\\
        \ruby{北斗}{}に\ruby{和}{}する\ruby{生命}{}なり
    
    \end{minipage}
\end{enumerate} % 番号の箇条書き ここまで
% end lilycs
%%%%% 歌詞 ここまで %%%%%
% end body

\end{document}

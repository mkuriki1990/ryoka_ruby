\documentclass[10pt,b5j]{tarticle} % B6 縦書き
% \documentclass[10pt,b5j]{tarticle} % B6 縦書き
\AtBeginDvi{\special{papersize=128mm,182mm}} % B6 用用紙サイズ
\usepackage{otf} % Unicode で字を入力するのに必要なパッケージ
\usepackage[size=b6j]{bxpapersize} % B6 用紙サイズを指定
\usepackage[dvipdfmx]{graphicx} % 画像を挿入するためのパッケージ
\usepackage[dvipdfmx]{color} % 色をつけるためのパッケージ
\usepackage{pxrubrica} % ルビを振るためのパッケージ
\usepackage{plext} % 漢数字の enumerate を使うためのパッケージ
\usepackage{multicol} % 複数段組を作るためのパッケージ
\setlength{\topmargin}{14mm} % 上下方向のマージン
\addtolength{\topmargin}{-1in} % 
\setlength{\oddsidemargin}{11mm} % 左右方向のマージン
\addtolength{\oddsidemargin}{-1in} % 
\setlength{\textwidth}{154mm} % B6 用
\setlength{\textheight}{108mm} % B6 用
\setlength{\headsep}{0mm} % 
\setlength{\headheight}{0mm} % 
\setlength{\topskip}{0mm} % 
\setlength{\parskip}{0pt} % 
\def\theenumi{\Kanji{enumi}} % 箇条書きのフォーマットを漢数字に変更
\parindent = 0pt % 段落下げしない
\pagestyle{empty} % すべてのページ番号を消去
% \renewcommand{\baselinestretch}{0.9} % 行間の倍率
 % B6 用テンプレート読み込み

\begin{document}
% begin header
%%%%% タイトルと作者 ここから %%%%%
\begin{minipage}[c]{0.7\hsize} % タイトルは上から 7 割
    \begin{center}
    % begin title
        {\LARGE
            水泳部部歌 % タイトルを入れる
        }
        {\small 
             % 年などを入れる
        }
    % end title
    \end{center}
\end{minipage}
\begin{minipage}[c]{0.3\hsize} % 作歌作曲は上から 3 割
    \begin{flushright} % 下寄せにする
        % begin name
        築城武義君 作歌\\柳沢三郎君 作曲 % 作歌・作曲者
        % end name
    \end{flushright}
\end{minipage}
%%%%% タイトルと作者 ここまで %%%%%
% % end header

% begin length
\vspace{1.5em} % タイトル, 作者と歌詞の間に隙間を設ける
\newcommand{\linespace}{0.5em} % 行間の設定
\newcommand{\blocksize}{0.5\hsize} % 段組間の設定
\newcommand{\itemmargin}{3em} % 曲番の位置調整の長さ
% end length
% begin body
%%%%% 歌詞 ここから %%%%%
\begin{enumerate} % 番号の箇条書き ここから
    \setlength{\itemindent}{\itemmargin} % 曲番の位置調整
    \begin{minipage}[c]{\blocksize}
    
        \vspace{\linespace}
        \item~\\
        % 1.
        \ruby{春}{はる}\ruby{猶}{なお}\ruby{淺}{}し\ruby{北}{きた}\ruby{溟}{}の\ruby{州}{しゅう}\\
        \ruby{雪消}{ゆきげ}のましみづ\ruby{併}{あわ}せ\ruby{來}{}る\\
        \ruby{石狩川}{いしかりがわ}の\ruby{清流}{せいりゅう}にも\\
        \ruby{我等}{われら}が\ruby{力}{ちから}\ruby{譬}{}ふべし\\
        \ruby{健}{けん}き\ruby{心}{こころ}の\ruby{丈夫}{じょうぶ}は\\
        \ruby{脾肉}{ひにく}の\ruby{嘆}{}を\ruby{喞}{}つ\ruby{哉}{}
        
    \end{minipage}
    \begin{minipage}[c]{\blocksize}
        
        \vspace{\linespace}
        \item~\\
        % 2.
        \ruby{見}{み}よ\ruby{清澄}{せいちょう}の\ruby{北斗}{ほくと}の\ruby{蒼穹}{そうきゅう}\\
        \ruby{壯}{つよし}\ruby{厳}{げん}の\ruby{氣}{}は\ruby{充}{たかし}つるなり\\
        \ruby{意気}{いき}と\ruby{熱情}{ねつじょう}とに\ruby{培}{}はれ\\
        \ruby{我等}{われら}が\ruby{承}{う}けし\ruby{伝統}{でんとう}の\\
        \ruby{歴史}{れきし}は\ruby{經}{}りて\ruby{十}{じゅう}\ruby{五}{ご}\ruby{年}{ねん}
        
    \end{minipage}
    \begin{minipage}[c]{\blocksize}
        
        \vspace{\linespace}
        \item~\\
        % 3.
        \ruby{楡}{にれ}の\ruby{樹林}{じゅりん}に\ruby{斜陽}{しゃよう}は\ruby{射}{さ}して\\
        \ruby{郭公}{かっこう}の\ruby{啼聲}{}\ruby{哀}{あわ}れなり\\
        \ruby{飛沫}{しぶき}の\ruby{虹}{にじ}も\ruby{消}{しょう}ゆる\ruby{頃}{ころ}\\
        \ruby{営}{いとな}み\ruby{終}{おわり}へて\ruby{波}{なみ}\ruby{静}{しず}か\\
        \ruby{朋友}{ほうゆう}が\ruby{友情}{ゆうじょう}の\ruby{熱}{あつ}ければ\\
        \ruby{血涙}{けつるい}の\ruby{日}{ひ}の\ruby{夢}{ゆめ}\ruby{悲}{かな}し
        
    \end{minipage}
    \begin{minipage}[c]{\blocksize}
        
        \vspace{\linespace}
        \item~\\
        % 4.
        あゝアカシヤの\ruby{白}{しろ}\ruby{花}{はな}\ruby{咲}{さ}きて\\
        \ruby{石狩}{いしかり}の\ruby{原野}{げんや}に\ruby{風}{かぜ}\ruby{渡}{わた}る\\
        \ruby{大地}{だいち}に\ruby{立}{た}ちて\ruby{吾}{われ}\ruby{呼}{よ}べば\\
        \ruby{若}{わか}き\ruby{生命}{いのち}の\ruby{胸}{むね}に\ruby{充}{み}つ\\
        いでや\ruby{制覇}{せいは}の\ruby{剛}{つよし}\ruby{劍}{}を\ruby{把}{わ}り\\
        \ruby{津}{つ}\ruby{輕}{}の\ruby{海峡}{かいきょう}を\ruby{超}{こ}え\ruby{行}{い}かん
    
    \end{minipage}
\end{enumerate} % 番号の箇条書き ここまで
%%%%% 歌詞 ここまで %%%%%
% end body

\end{document}

\documentclass[10pt,b5j]{tarticle} % B6 縦書き
% \documentclass[10pt,b5j]{tarticle} % B6 縦書き
\AtBeginDvi{\special{papersize=128mm,182mm}} % B6 用用紙サイズ
\usepackage{otf} % Unicode で字を入力するのに必要なパッケージ
\usepackage[size=b6j]{bxpapersize} % B6 用紙サイズを指定
\usepackage[dvipdfmx]{graphicx} % 画像を挿入するためのパッケージ
\usepackage[dvipdfmx]{color} % 色をつけるためのパッケージ
\usepackage{pxrubrica} % ルビを振るためのパッケージ
\usepackage{multicol} % 複数段組を作るためのパッケージ
\setlength{\topmargin}{14mm} % 上下方向のマージン
\addtolength{\topmargin}{-1in} % 
\setlength{\oddsidemargin}{11mm} % 左右方向のマージン
\addtolength{\oddsidemargin}{-1in} % 
\setlength{\textwidth}{154mm} % B6 用
\setlength{\textheight}{108mm} % B6 用
\setlength{\headsep}{0mm} % 
\setlength{\headheight}{0mm} % 
\setlength{\topskip}{0mm} % 
\setlength{\parskip}{0pt} % 
\def\labelenumi{\theenumi、} % 箇条書きのフォーマット
\parindent = 0pt % 段落下げしない

 % B6 用テンプレート読み込み

\begin{document}
% begin header
%%%%% タイトルと作者 ここから %%%%%
\begin{minipage}[c]{0.7\hsize} % タイトルは上から 7 割
    \begin{center}
    % begin title
        {\LARGE
            水泳部部歌 % タイトルを入れる
        }
        {\small 
             % 年などを入れる
        }
    % end title
    \end{center}
\end{minipage}
\begin{minipage}[c]{0.3\hsize} % 作歌作曲は上から 3 割
    \begin{flushright} % 下寄せにする
        % begin name
        築城武義君 作歌\\柳沢三郎君 作曲 % 作歌・作曲者
        % end name
    \end{flushright}
\end{minipage}
%%%%% タイトルと作者 ここまで %%%%%
% % end header

% begin length
\vspace{1.5em} % タイトル, 作者と歌詞の間に隙間を設ける
\newcommand{\linespace}{0.5em} % 行間の設定
\newcommand{\blocksize}{0.5\hsize} % 段組間の設定
\newcommand{\itemmargin}{3em} % 曲番の位置調整の長さ
% end length
% begin body
%%%%% 歌詞 ここから %%%%%
\begin{enumerate} % 番号の箇条書き ここから
    \setlength{\itemindent}{\itemmargin} % 曲番の位置調整
    \begin{minipage}[c]{\blocksize}
    
        \vspace{\linespace}
        \item~\\
        % 1.
        \ruby{春猶淺}{}し\ruby{北溟}{}の\ruby{州}{}\\
        \ruby{雪消}{}のましみづ\ruby{併}{}せ\ruby{來}{}る\\
        \ruby{石狩川}{}の\ruby{清流}{}にも\\
        \ruby{我等}{}が\ruby{力譬}{}ふべし\\
        \ruby{健}{}き\ruby{心}{}の\ruby{丈夫}{}は\\
        \ruby{脾肉}{}の\ruby{嘆}{}を\ruby{喞}{}つ\ruby{哉}{}
        
    \end{minipage}
    \begin{minipage}[c]{\blocksize}
        
        \vspace{\linespace}
        \item~\\
        % 2.
        \ruby{見}{}よ\ruby{清澄}{}の\ruby{北斗}{}の\ruby{蒼穹}{}\\
        \ruby{壯厳}{}の\ruby{氣}{}は\ruby{充}{}つるなり\\
        \ruby{意気}{}と\ruby{熱情}{}とに\ruby{培}{}はれ\\
        \ruby{我等}{}が\ruby{承}{}けし\ruby{伝統}{}の\\
        \ruby{歴史}{}は\ruby{經}{}りて\ruby{十五年}{}
        
    \end{minipage}
    \begin{minipage}[c]{\blocksize}
        
        \vspace{\linespace}
        \item~\\
        % 3.
        \ruby{楡}{}の\ruby{樹林}{}に\ruby{斜陽}{}は\ruby{射}{}して\\
        \ruby{郭公}{}の\ruby{啼聲哀}{}れなり\\
        \ruby{飛沫}{}の\ruby{虹}{}も\ruby{消}{}ゆる\ruby{頃}{}\\
        \ruby{営}{}み\ruby{終}{}へて\ruby{波静}{}か\\
        \ruby{朋友}{}が\ruby{友情}{}の\ruby{熱}{}ければ\\
        \ruby{血涙}{}の\ruby{日}{}の\ruby{夢悲}{}し
        
    \end{minipage}
    \begin{minipage}[c]{\blocksize}
        
        \vspace{\linespace}
        \item~\\
        % 4.
        あゝアカシヤの\ruby{白花咲}{}きて\\
        \ruby{石狩}{}の\ruby{原野}{}に\ruby{風渡}{}る\\
        \ruby{大地}{}に\ruby{立}{}ちて\ruby{吾呼}{}べば\\
        \ruby{若}{}き\ruby{生命}{}の\ruby{胸}{}に\ruby{充}{}つ\\
        いでや\ruby{制覇}{}の\ruby{剛劍}{}を\ruby{把}{}り\\
        \ruby{津輕}{}の\ruby{海峡}{}を\ruby{超}{}え\ruby{行}{}かん
    
    \end{minipage}
\end{enumerate} % 番号の箇条書き ここまで
%%%%% 歌詞 ここまで %%%%%
% end body

\end{document}

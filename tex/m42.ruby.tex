\documentclass[10pt,b5j]{tarticle} % B6 縦書き
% \documentclass[10pt,b5j]{tarticle} % B6 縦書き
\AtBeginDvi{\special{papersize=128mm,182mm}} % B6 用用紙サイズ
\usepackage{otf} % Unicode で字を入力するのに必要なパッケージ
\usepackage[size=b6j]{bxpapersize} % B6 用紙サイズを指定
\usepackage[dvipdfmx]{graphicx} % 画像を挿入するためのパッケージ
\usepackage[dvipdfmx]{color} % 色をつけるためのパッケージ
\usepackage{pxrubrica} % ルビを振るためのパッケージ
\usepackage{multicol} % 複数段組を作るためのパッケージ
\setlength{\topmargin}{14mm} % 上下方向のマージン
\addtolength{\topmargin}{-1in} % 
\setlength{\oddsidemargin}{11mm} % 左右方向のマージン
\addtolength{\oddsidemargin}{-1in} % 
\setlength{\textwidth}{154mm} % B6 用
\setlength{\textheight}{108mm} % B6 用
\setlength{\headsep}{0mm} % 
\setlength{\headheight}{0mm} % 
\setlength{\topskip}{0mm} % 
\setlength{\parskip}{0pt} % 
\def\labelenumi{\theenumi、} % 箇条書きのフォーマット
\parindent = 0pt % 段落下げしない

 % B6 用テンプレート読み込み

\begin{document}
% begin header
%%%%% タイトルと作者 ここから %%%%%
\begin{minipage}[c]{0.7\hsize} % タイトルは上から 7 割
    \begin{center}
    % begin title
        {\LARGE
            希望の光 % タイトルを入れる
        }
        {\small 
            (明治四十二年寮歌) % 年などを入れる
        }
    % end title
    \end{center}
\end{minipage}
\begin{minipage}[c]{0.3\hsize} % 作歌作曲は上から 3 割
    \begin{flushright} % 下寄せにする
        % begin name
        加藤茂男君 作歌\\金原善知君 作曲 % 作歌・作曲者
        % end name
    \end{flushright}
\end{minipage}
%%%%% タイトルと作者 ここまで %%%%%
% (1,6 了あり)
% end header

% begin body
\vspace{1.5em} % タイトル, 作者と歌詞の間に隙間を設ける
\newcommand{\linespace}{0.5em} % 行間の設定
\newcommand{\blocksize}{0.5\hsize} % 段組間の設定
%%%%% 歌詞 ここから %%%%%
% begin lilycs
\begin{enumerate} % 番号の箇条書き ここから
    \begin{minipage}[c]{\blocksize}
    
        \vspace{\linespace}
        \item
        % 1.
        \ruby{希望}{}の\ruby{光仰}{}ぎつつ\\
        \ruby{思}{}へば\ruby{友}{}と\ruby{尋}{}ね\ruby{来}{}し\\
        \ruby{山}{}は\ruby{紅朝日子}{}の\\
        \ruby{燃}{}ゆる\ruby{姿}{}に\ruby{似}{}たる\ruby{哉}{}\\
        \ruby{嘶}{}く\ruby{駒}{}は\ruby{秋}{}に\ruby{肥}{}え\\
        \ruby{我等}{}が\ruby{門出栄}{}ありき
        
        \vspace{\linespace}
        \item
        % 2.
        ああ\ruby{冬寒}{}し\ruby{北国}{}の\\
        \ruby{大野}{}の\ruby{果}{}を\ruby{眺}{}むれば\\
        \ruby{雪}{}かあられか\ruby{空}{}たえて\\
        \ruby{限}{}りは\ruby{知}{}らず\ruby{暮}{}るとも\\
        \ruby{我等}{}が\ruby{胸}{}に\ruby{黙想}{}あり\\
        \ruby{星}{}の\ruby{光}{}に\ruby{啓示}{}あり
        
        \vspace{\linespace}
        \item
        % 3.
        \ruby{黙想}{}を\ruby{胸}{}に\ruby{結}{}ぶ\ruby{時}{}\\
        \ruby{啓示}{}を\ruby{空}{}に\ruby{望}{}む\ruby{時}{}\\
        \ruby{見}{}よ\ruby{下萠}{}ゆる\ruby{若草}{}の\\
        \ruby{息吹}{}さやかに\ruby{風薫}{}る\\
        \ruby{春}{}は\ruby{来}{}れり\ruby{春}{}は\ruby{来}{}ぬ\\
        \ruby{物皆此処}{}に\ruby{力}{}あり
        
        \vspace{\linespace}
        \item
        % 4.
        \ruby{春}{}の\ruby{光}{}の\ruby{照}{}る\ruby{所}{}\\
        \ruby{色}{}を\ruby{交}{}へて\ruby{咲}{}く\ruby{花}{}に\\
        \ruby{蝶舞}{}ひ\ruby{鳥}{}は\ruby{囀}{}りて\\
        \ruby{我等}{}が\ruby{血潮躍}{}るなり\\
        \ruby{斯}{}くて\ruby{見渡}{}す\ruby{行手}{}には\\
        \ruby{光蔽}{}はん\ruby{影}{}もなし
        
        \vspace{\linespace}
        \item
        % 5.
        \ruby{深}{}く\ruby{霞}{}に\ruby{鎖}{}されて\\
        \ruby{都}{}の\ruby{様}{}は\ruby{知}{}らねども\\
        \ruby{夕孤雁}{}の\ruby{声聞}{}けば\\
        \ruby{人太平}{}に\ruby{眠}{}るとや\\
        \ruby{吹雪}{}に\ruby{練}{}りし\ruby{双}{}の\ruby{腕}{}\\
        \ruby{鳴}{}るよ\ruby{常磐}{}の\ruby{夢醒}{}ませ
        
        \vspace{\linespace}
        \item
        % 6.
        \ruby{四年}{}の\ruby{昔人々}{}の\\
        \ruby{耘}{}り\ruby{建}{}てし\ruby{我}{}が\ruby{寮}{}に\\
        \ruby{春立}{}ち\ruby{還}{}る\ruby{時}{}よ\ruby{今}{}\\
        \ruby{希望}{}の\ruby{光新}{}なり\\
        さらば\ruby{起}{}て\ruby{友諸共}{}に\\
        \ruby{我等起}{}つべき\ruby{時}{}なれば
    
    \end{minipage}
\end{enumerate} % 番号の箇条書き ここまで
% end lilycs
%%%%% 歌詞 ここまで %%%%%
% end body

\end{document}

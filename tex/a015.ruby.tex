\documentclass[10pt,b5j]{tarticle} % B6 縦書き
% \documentclass[10pt,b5j]{tarticle} % B6 縦書き
\AtBeginDvi{\special{papersize=128mm,182mm}} % B6 用用紙サイズ
\usepackage{otf} % Unicode で字を入力するのに必要なパッケージ
\usepackage[size=b6j]{bxpapersize} % B6 用紙サイズを指定
\usepackage[dvipdfmx]{graphicx} % 画像を挿入するためのパッケージ
\usepackage[dvipdfmx]{color} % 色をつけるためのパッケージ
\usepackage{pxrubrica} % ルビを振るためのパッケージ
\usepackage{multicol} % 複数段組を作るためのパッケージ
\setlength{\topmargin}{14mm} % 上下方向のマージン
\addtolength{\topmargin}{-1in} % 
\setlength{\oddsidemargin}{11mm} % 左右方向のマージン
\addtolength{\oddsidemargin}{-1in} % 
\setlength{\textwidth}{154mm} % B6 用
\setlength{\textheight}{108mm} % B6 用
\setlength{\headsep}{0mm} % 
\setlength{\headheight}{0mm} % 
\setlength{\topskip}{0mm} % 
\setlength{\parskip}{0pt} % 
\def\labelenumi{\theenumi、} % 箇条書きのフォーマット
\parindent = 0pt % 段落下げしない

 % B6 用テンプレート読み込み

\begin{document}
% begin header
%%%%% タイトルと作者 ここから %%%%%
\begin{minipage}[c]{0.7\hsize} % タイトルは上から 7 割
    \begin{center}
    % begin title
        {\LARGE
            文武会歌 % タイトルを入れる
        }
        {\small 
            (昭和15年) % 年などを入れる
        }
    % end title
    \end{center}
\end{minipage}
\begin{minipage}[c]{0.3\hsize} % 作歌作曲は上から 3 割
    \begin{flushright} % 下寄せにする
        % begin name
         % 作歌・作曲者
        % end name
    \end{flushright}
\end{minipage}
%%%%% タイトルと作者 ここまで %%%%%
% % end header

% begin body
\vspace{1.5em} % タイトル, 作者と歌詞の間に隙間を設ける
\newcommand{\linespace}{0.5em} % 行間の設定
\newcommand{\blocksize}{0.5\hsize} % 段組間の設定
%%%%% 歌詞 ここから %%%%%
% begin lilycs
\begin{enumerate} % 番号の箇条書き ここから
    \begin{minipage}[c]{\blocksize}
    
        \vspace{\linespace}
        \item
        % 1.
        \ruby{鳳雛}{}は\ruby{集}{}い\ruby{來}{}たりぬ\\
        \ruby{玲瓏}{}の\ruby{楡}{}の\ruby{学園}{}よ\ruby{靄然}{}たり\\
        \ruby{精気}{}に\ruby{溢}{}れ\ruby{光明}{}に\ruby{満}{}ちて\\
        いざ\ruby{学友}{}よ \ruby{若}{}き\ruby{日}{}の\\
        \ruby{生命}{}の\ruby{限}{}り\ruby{団欒}{}ては\\
        \ruby{育}{}まん\ruby{哉文武}{}の\ruby{精華}{}
        
        \vspace{\linespace}
        \item
        % 2.
        \ruby{乾坤}{}に\ruby{時光移}{}ろひぬ\\
        \ruby{秀}{}でたる\ruby{久遠}{}の\ruby{山河}{}よ\ruby{厳然}{}たり\\
        \ruby{仰}{}ぐ\ruby{喜悦血潮}{}は\ruby{湧}{}きて\\
        いざ\ruby{学友}{}よ \ruby{若}{}き\ruby{日}{}の\\
        \ruby{魂}{}の\ruby{故郷}{}に\ruby{契}{}りては\\
        \ruby{培}{}はん\ruby{哉尊}{}き\ruby{遺訓}{}
        
        \vspace{\linespace}
        \item
        % 3.
        \ruby{清明}{}の\ruby{森影深}{}し\\
        \ruby{憧憬}{}の\ruby{象牙}{}の\ruby{塔}{}よ\ruby{巍然}{}たり\\
        \ruby{理想}{}は\ruby{高}{}く\ruby{情懐}{}に\ruby{燃}{}えて\\
        いざ\ruby{学友}{}よ \ruby{若}{}き\ruby{日}{}の\\
        \ruby{淨}{}き\ruby{思索}{}に\ruby{沈潜}{}ては\\
        \ruby{究}{}めん\ruby{哉眞理}{}の\ruby{秘奥}{}
    
    \end{minipage}
\end{enumerate} % 番号の箇条書き ここまで
% end lilycs
%%%%% 歌詞 ここまで %%%%%
% end body

\end{document}

\documentclass[10pt,b5j]{tarticle} % B6 縦書き
% \documentclass[10pt,b5j]{tarticle} % B6 縦書き
\AtBeginDvi{\special{papersize=128mm,182mm}} % B6 用用紙サイズ
\usepackage{otf} % Unicode で字を入力するのに必要なパッケージ
\usepackage[size=b6j]{bxpapersize} % B6 用紙サイズを指定
\usepackage[dvipdfmx]{graphicx} % 画像を挿入するためのパッケージ
\usepackage[dvipdfmx]{color} % 色をつけるためのパッケージ
\usepackage{pxrubrica} % ルビを振るためのパッケージ
\usepackage{multicol} % 複数段組を作るためのパッケージ
\setlength{\topmargin}{14mm} % 上下方向のマージン
\addtolength{\topmargin}{-1in} % 
\setlength{\oddsidemargin}{11mm} % 左右方向のマージン
\addtolength{\oddsidemargin}{-1in} % 
\setlength{\textwidth}{154mm} % B6 用
\setlength{\textheight}{108mm} % B6 用
\setlength{\headsep}{0mm} % 
\setlength{\headheight}{0mm} % 
\setlength{\topskip}{0mm} % 
\setlength{\parskip}{0pt} % 
\def\labelenumi{\theenumi、} % 箇条書きのフォーマット
\parindent = 0pt % 段落下げしない

 % B6 用テンプレート読み込み

\begin{document}
% begin header
%%%%% タイトルと作者 ここから %%%%%
\begin{minipage}[c]{0.7\hsize} % タイトルは上から 7 割
    \begin{center}
    % begin title
        {\LARGE
            柔道部東征歌 % タイトルを入れる
        }
        {\small 
             % 年などを入れる
        }
    % end title
    \end{center}
\end{minipage}
\begin{minipage}[c]{0.3\hsize} % 作歌作曲は上から 3 割
    \begin{flushright} % 下寄せにする
        % begin name
        河合九州夫君 作歌\\小峰三千男君 作曲 % 作歌・作曲者
        % end name
    \end{flushright}
\end{minipage}
%%%%% タイトルと作者 ここまで %%%%%
% % end header

% begin length
\vspace{1.5em} % タイトル, 作者と歌詞の間に隙間を設ける
\newcommand{\linespace}{0.5em} % 行間の設定
\newcommand{\blocksize}{0.5\hsize} % 段組間の設定
\newcommand{\itemmargin}{6em} % 曲番の位置調整の長さ
% end length
% begin body
%%%%% 歌詞 ここから %%%%%
\begin{enumerate} % 番号の箇条書き ここから
    \setlength{\itemindent}{\itemmargin} % 曲番の位置調整
    \begin{minipage}[c]{\blocksize}
    
        \vspace{\linespace}
        \item~\\
        % 1.
        \ruby{蓬風吼}{}ゆる\ruby{北海}{}の\\
        \ruby{岸辺}{}に\ruby{狂}{}う\ruby{波}{}の\ruby{花}{}\\
        \ruby{雲煙遠}{}く\ruby{流}{}れ\ruby{入}{}る\\
        \ruby{石狩河岸}{}に\ruby{根城}{}して\\
        \ruby{桜星}{}の\ruby{旗飜}{}し\\
        \ruby{立}{}てる\ruby{吾部}{}ぞ\ruby{力}{}あり
        
        \vspace{\linespace}
        \item~\\
        % 2.
        \ruby{桜花咲}{}く\ruby{日}{}の\ruby{国}{}の\\
        \ruby{猛}{}き\ruby{心}{}の益\ruby{荒男}{}が\\
        をのこさびする\ruby{高潮}{}に\\
        \ruby{不断}{}の\ruby{意力養}{}ひて\\
        \ruby{究}{}め\ruby{盡}{}せし\ruby{先人}{}の\\
        \ruby{教}{}へぞ\ruby{励}{}め\ruby{柔}{}の\ruby{道}{}
        
        \vspace{\linespace}
        \item~\\
        % 3.
        \ruby{号笛一声高鳴}{}りて\\
        \ruby{集}{}ひ\ruby{来}{}りし\ruby{貔貅軍}{}\\
        \ruby{熱砂吹}{}き\ruby{捲}{}く\ruby{夏}{}の\ruby{日}{}も\\
        \ruby{鈴}{}の\ruby{音氷}{}る\ruby{冬}{}の\ruby{夜}{}も\\
        \ruby{瘡痍}{}に\ruby{悩}{}みし\ruby{肉}{}よ\\
        \ruby{思}{}へばこゝに\ruby{幾年}{}ぞ
        
        \vspace{\linespace}
        \item~\\
        % 4.
        \ruby{今東海}{}に\ruby{雲湧}{}きて\\
        \ruby{雷鼓轟}{}き\ruby{風}{}すさぶ\\
        \ruby{覇者}{}の\ruby{偉業}{}を\ruby{果}{}さんと\\
        \ruby{蛟龍一度鋒}{}とれば\\
        あはれ\ruby{声}{}なき\ruby{影}{}もなき\\
        \ruby{彼}{}の\ruby{狂乱}{}の\ruby{牛羊等}{}
        
        \vspace{\linespace}
        \item~\\
        % 5.
        \ruby{輪影高}{}く\ruby{空澄}{}みて\\
        \ruby{朔風膚}{}を\ruby{裂}{}く\ruby{夕}{}\\
        \ruby{楡樹}{}の\ruby{下}{}に\ruby{帳}{}して\\
        \ruby{祝}{}ふ\ruby{首途}{}の\ruby{若武者}{}よ\\
        \ruby{交}{}す\ruby{栄}{}ある\ruby{酒盃}{}に\\
        \ruby{月下}{}の\ruby{誓}{}なさんかな
    
    \end{minipage}
\end{enumerate} % 番号の箇条書き ここまで
%%%%% 歌詞 ここまで %%%%%
% end body

\end{document}

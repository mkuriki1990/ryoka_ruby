\documentclass[10pt,b5j]{tarticle} % B6 縦書き
% \documentclass[10pt,b5j]{tarticle} % B6 縦書き
\AtBeginDvi{\special{papersize=128mm,182mm}} % B6 用用紙サイズ
\usepackage{otf} % Unicode で字を入力するのに必要なパッケージ
\usepackage[size=b6j]{bxpapersize} % B6 用紙サイズを指定
\usepackage[dvipdfmx]{graphicx} % 画像を挿入するためのパッケージ
\usepackage[dvipdfmx]{color} % 色をつけるためのパッケージ
\usepackage{pxrubrica} % ルビを振るためのパッケージ
\usepackage{multicol} % 複数段組を作るためのパッケージ
\setlength{\topmargin}{14mm} % 上下方向のマージン
\addtolength{\topmargin}{-1in} % 
\setlength{\oddsidemargin}{11mm} % 左右方向のマージン
\addtolength{\oddsidemargin}{-1in} % 
\setlength{\textwidth}{154mm} % B6 用
\setlength{\textheight}{108mm} % B6 用
\setlength{\headsep}{0mm} % 
\setlength{\headheight}{0mm} % 
\setlength{\topskip}{0mm} % 
\setlength{\parskip}{0pt} % 
\def\labelenumi{\theenumi、} % 箇条書きのフォーマット
\parindent = 0pt % 段落下げしない

 % B6 用テンプレート読み込み

\begin{document}
% begin header
%%%%% タイトルと作者 ここから %%%%%
\begin{minipage}[c]{0.7\hsize} % タイトルは上から 7 割
    \begin{center}
    % begin title
        {\LARGE
            若芽の出づる % タイトルを入れる
        }
        {\small 
            (平成三年度寮歌) % 年などを入れる
        }
    % end title
    \end{center}
\end{minipage}
\begin{minipage}[c]{0.3\hsize} % 作歌作曲は上から 3 割
    \begin{flushright} % 下寄せにする
        % begin name
        柴田一君 作歌・作曲 % 作歌・作曲者
        % end name
    \end{flushright}
\end{minipage}
%%%%% タイトルと作者 ここまで %%%%%
% (1,4 繰り返しなし)
% end header

% begin length
\vspace{1.5em} % タイトル, 作者と歌詞の間に隙間を設ける
\newcommand{\linespace}{0.5em} % 行間の設定
\newcommand{\blocksize}{0.5\hsize} % 段組間の設定
\newcommand{\itemmargin}{3em} % 曲番の位置調整の長さ
% end length
% begin body
%%%%% 歌詞 ここから %%%%%
\begin{enumerate} % 番号の箇条書き ここから
    \setlength{\itemindent}{\itemmargin} % 曲番の位置調整
    \begin{minipage}[c]{\blocksize}
    
        \vspace{\linespace}
        \item~\\
        % 1.
        \ruby{若芽}{}の\ruby{出}{}づる\ruby{早春}{}に\\
        \ruby{孤影}{}も\ruby{辞}{}せぬ\ruby{若人}{}の\\
        \ruby{尖風躰}{}を\ruby{貫}{}けば\\
        \ruby{漲}{}る\ruby{大志}{}の\ruby{息吹有}{}り
        
    \end{minipage}
    \begin{minipage}[c]{\blocksize}
        
        \vspace{\linespace}
        \item~\\
        % 2.
        \ruby{万物謳歌}{}う\ruby{盛夏}{}なれど\\
        \ruby{榮華}{}の\ruby{闇部忘}{}るまじ\\
        \ruby{凱風四界}{}を\ruby{覆}{}へども\\
        \ruby{鬼哭}{}の\ruby{嘆}{}きは\ruby{芯}{}を\ruby{凍}{}て
        
    \end{minipage}
    \begin{minipage}[c]{\blocksize}
        
        \vspace{\linespace}
        \item~\\
        % 3.
        \ruby{紅葉吠}{}ゆる\ruby{秋}{}の\ruby{窓}{}\\
        \ruby{落葉瓢}{}の\ruby{様}{}を\ruby{見}{}む\\
        \ruby{疾風怒濤}{}の\ruby{世}{}なればこそ\\
        \ruby{真理}{}の\ruby{迪}{}を\ruby{一筋}{}に
        
    \end{minipage}
    \begin{minipage}[c]{\blocksize}
        
        \vspace{\linespace}
        \item~\\
        % 4.
        \ruby{永雪猛}{}る\ruby{厳冬}{}は\\
        \ruby{心膽練磨}{}の\ruby{時節}{}かな\\
        \ruby{烈風大地}{}を\ruby{劈}{}けど\\
        \ruby{揺}{}るがぬ\ruby{我}{}らがこの\ruby{宿居}{}
    
    \end{minipage}
\end{enumerate} % 番号の箇条書き ここまで
%%%%% 歌詞 ここまで %%%%%
% end body

\end{document}

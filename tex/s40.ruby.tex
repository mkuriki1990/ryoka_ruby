\documentclass[10pt,b5j]{tarticle} % B6 縦書き
% \documentclass[10pt,b5j]{tarticle} % B6 縦書き
\AtBeginDvi{\special{papersize=128mm,182mm}} % B6 用用紙サイズ
\usepackage{otf} % Unicode で字を入力するのに必要なパッケージ
\usepackage[size=b6j]{bxpapersize} % B6 用紙サイズを指定
\usepackage[dvipdfmx]{graphicx} % 画像を挿入するためのパッケージ
\usepackage[dvipdfmx]{color} % 色をつけるためのパッケージ
\usepackage{pxrubrica} % ルビを振るためのパッケージ
\usepackage{plext} % 漢数字の enumerate を使うためのパッケージ
\usepackage{multicol} % 複数段組を作るためのパッケージ
\setlength{\topmargin}{14mm} % 上下方向のマージン
\addtolength{\topmargin}{-1in} % 
\setlength{\oddsidemargin}{11mm} % 左右方向のマージン
\addtolength{\oddsidemargin}{-1in} % 
\setlength{\textwidth}{154mm} % B6 用
\setlength{\textheight}{108mm} % B6 用
\setlength{\headsep}{0mm} % 
\setlength{\headheight}{0mm} % 
\setlength{\topskip}{0mm} % 
\setlength{\parskip}{0pt} % 
\def\theenumi{\Kanji{enumi}} % 箇条書きのフォーマットを漢数字に変更
\parindent = 0pt % 段落下げしない
\pagestyle{empty} % すべてのページ番号を消去
% \renewcommand{\baselinestretch}{0.9} % 行間の倍率
 % B6 用テンプレート読み込み

\begin{document}
% begin header
%%%%% タイトルと作者 ここから %%%%%
\begin{minipage}[c]{0.7\hsize} % タイトルは上から 7 割
    \begin{center}
    % begin title
        {\LARGE
            新しき陽は % タイトルを入れる
        }
        {\small 
            (昭和四十年寮歌) % 年などを入れる
        }
    % end title
    \end{center}
\end{minipage}
\begin{minipage}[c]{0.3\hsize} % 作歌作曲は上から 3 割
    \begin{flushright} % 下寄せにする
        % begin name
        金子公良君 作歌\\西雪弘光君 作曲 % 作歌・作曲者
        % end name
    \end{flushright}
\end{minipage}
%%%%% タイトルと作者 ここまで %%%%%
% (1,4 繰り返しなし)
% end header

% begin length
\vspace{1.5em} % タイトル, 作者と歌詞の間に隙間を設ける
\newcommand{\linespace}{0.5em} % 行間の設定
\newcommand{\blocksize}{0.5\hsize} % 段組間の設定
\newcommand{\itemmargin}{3em} % 曲番の位置調整の長さ
% end length
% begin body
%%%%% 歌詞 ここから %%%%%
\begin{enumerate} % 番号の箇条書き ここから
    \setlength{\itemindent}{\itemmargin} % 曲番の位置調整
    \begin{minipage}[c]{\blocksize}
    
        \vspace{\linespace}
        \item~\\
        % 1.
        \ruby{新}{}しき\ruby{陽}{}は\ruby{今昇}{}り\\
        \ruby{空}{}のはて\ruby{黎明}{}を\ruby{告}{}ぐ\\
        \ruby{黒}{}き\ruby{雲西}{}に\ruby{流}{}れん\\
        \ruby{吹}{}きすさぶ\ruby{嵐}{}をつきて\\
        \ruby{平和}{}をと\ruby{声}{}は\ruby{轟}{}く
        
    \end{minipage}
    \begin{minipage}[c]{\blocksize}
        
        \vspace{\linespace}
        \item~\\
        % 2.
        \ruby{逞}{}ましき\ruby{友}{}の\ruby{怒}{}りに\\
        \ruby{雄々}{}しくも\ruby{我等誓}{}いし\\
        \ruby{幸}{}の\ruby{世永遠}{}に\ruby{築}{}かん\\
        \ruby{広}{}き\ruby{地}{}に\ruby{鍬}{}ふりかざし\\
        \ruby{緑萠}{}え\ruby{水流}{}るまで
        
    \end{minipage}
    \begin{minipage}[c]{\blocksize}
        
        \vspace{\linespace}
        \item~\\
        % 3.
        \ruby{輝}{}やける\ruby{祖国}{}の\ruby{山河}{}に\\
        こだまする\ruby{我等}{}が\ruby{雄叫}{}び\\
        \ruby{一}{}すじの\ruby{光求}{}めて\\
        ひたすらにただひたすらに\\
        \ruby{腕組}{}みて\ruby{歩}{}み\ruby{進}{}まん
        
    \end{minipage}
    \begin{minipage}[c]{\blocksize}
        
        \vspace{\linespace}
        \item~\\
        % 4.
        \ruby{真実}{}の\ruby{鐘鳴}{}り\ruby{響}{}き\\
        \ruby{森影}{}にどよめきのわく\\
        \ruby{自治}{}の\ruby{旗風}{}にゆらめき\\
        \ruby{名}{}を\ruby{留}{}む\ruby{伝}{}え\ruby{守}{}りて\\
        \ruby{恵迪}{}は\ruby{今}{}よみがえる
    
    \end{minipage}
\end{enumerate} % 番号の箇条書き ここまで
%%%%% 歌詞 ここまで %%%%%
% end body

\end{document}

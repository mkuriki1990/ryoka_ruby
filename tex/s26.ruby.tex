\documentclass[10pt,b5j]{tarticle} % B6 縦書き
% \documentclass[10pt,b5j]{tarticle} % B6 縦書き
\AtBeginDvi{\special{papersize=128mm,182mm}} % B6 用用紙サイズ
\usepackage{otf} % Unicode で字を入力するのに必要なパッケージ
\usepackage[size=b6j]{bxpapersize} % B6 用紙サイズを指定
\usepackage[dvipdfmx]{graphicx} % 画像を挿入するためのパッケージ
\usepackage[dvipdfmx]{color} % 色をつけるためのパッケージ
\usepackage{pxrubrica} % ルビを振るためのパッケージ
\usepackage{multicol} % 複数段組を作るためのパッケージ
\setlength{\topmargin}{14mm} % 上下方向のマージン
\addtolength{\topmargin}{-1in} % 
\setlength{\oddsidemargin}{11mm} % 左右方向のマージン
\addtolength{\oddsidemargin}{-1in} % 
\setlength{\textwidth}{154mm} % B6 用
\setlength{\textheight}{108mm} % B6 用
\setlength{\headsep}{0mm} % 
\setlength{\headheight}{0mm} % 
\setlength{\topskip}{0mm} % 
\setlength{\parskip}{0pt} % 
\def\labelenumi{\theenumi、} % 箇条書きのフォーマット
\parindent = 0pt % 段落下げしない

 % B6 用テンプレート読み込み

\begin{document}
% begin header
%%%%% タイトルと作者 ここから %%%%%
\begin{minipage}[c]{0.7\hsize} % タイトルは上から 7 割
    \begin{center}
    % begin title
        {\LARGE
            新たなり天地 % タイトルを入れる
        }
        {\small 
            (昭和二十六年寮歌) % 年などを入れる
        }
    % end title
    \end{center}
\end{minipage}
\begin{minipage}[c]{0.3\hsize} % 作歌作曲は上から 3 割
    \begin{flushright} % 下寄せにする
        % begin name
        長尾久司君 作歌\\小林滋宗君 作曲 % 作歌・作曲者
        % end name
    \end{flushright}
\end{minipage}
%%%%% タイトルと作者 ここまで %%%%%
% (1,2,3 繰り返しなし)
% end header

% begin length
\vspace{1.5em} % タイトル, 作者と歌詞の間に隙間を設ける
\newcommand{\linespace}{0.5em} % 行間の設定
\newcommand{\blocksize}{0.5\hsize} % 段組間の設定
\newcommand{\itemmargin}{6em} % 曲番の位置調整の長さ
% end length
% begin body
%%%%% 歌詞 ここから %%%%%
\begin{enumerate} % 番号の箇条書き ここから
    \setlength{\itemindent}{\itemmargin} % 曲番の位置調整
    \begin{minipage}[c]{\blocksize}
    
        \vspace{\linespace}
        \item~\\
        % 1.
        \ruby{新}{}たなり\ruby{天地}{}\\
        \ruby{光}{}あり\ruby{北}{}の\ruby{学舎}{}\\
        \ruby{二年}{}を\ruby{心}{}に\ruby{契}{}る\\
        \ruby{若}{}き\ruby{日}{}の\ruby{生命}{}の\ruby{郷}{}に\\
        \ruby{誇}{}らなん\ruby{自治}{}と\ruby{自由}{}の\\
        \ruby{四十星霜}{}の\ruby{高}{}き\ruby{伝統}{}よ\\
        おごそかに\ruby{遺訓}{}をこめて\\
        \ruby{楡鐘}{}は\ruby{響}{}かん
        
        \vspace{\linespace}
        \item~\\
        % 2.
        \ruby{雄大}{}いなり\ruby{天地}{}\\
        \ruby{栄劫}{}の\ruby{時潮}{}の\ruby{流}{}れよ\\
        \ruby{悠久}{}の\ruby{神秘}{}をひめし\\
        うるわしき\ruby{石狩}{}の\ruby{野}{}に\\
        うたわなん\ruby{希望}{}のうたを\\
        \ruby{魂}{}ゆする\ruby{雄叫}{}びの\ruby{日}{}に\\
        あこがれと\ruby{正義}{}の\ruby{旗}{}を\\
        かざし\ruby{進}{}まん
        
        \vspace{\linespace}
        \item~\\
        % 3.
        きびしかる\ruby{天地}{}\\
        \ruby{野}{}にすさぶ\ruby{試練}{}の\ruby{嵐}{}\\
        \ruby{苦}{}しみを\ruby{越}{}えて\ruby{幸}{}あり\\
        たゆみなく\ruby{求}{}めて\ruby{得}{}たり\\
        \ruby{輝}{}ける\ruby{久遠}{}の\ruby{真理}{}\\
        よろこびの\ruby{若}{}き\ruby{力}{}に\\
        \ruby{創造}{}き\ruby{行}{}く\ruby{恵迪}{}の\ruby{寮}{}\\
        とわに\ruby{栄}{}えん
    
    \end{minipage}
\end{enumerate} % 番号の箇条書き ここまで
%%%%% 歌詞 ここまで %%%%%
% end body

\end{document}

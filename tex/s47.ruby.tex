\documentclass[10pt,b5j]{tarticle} % B6 縦書き
% \documentclass[10pt,b5j]{tarticle} % B6 縦書き
\AtBeginDvi{\special{papersize=128mm,182mm}} % B6 用用紙サイズ
\usepackage{otf} % Unicode で字を入力するのに必要なパッケージ
\usepackage[size=b6j]{bxpapersize} % B6 用紙サイズを指定
\usepackage[dvipdfmx]{graphicx} % 画像を挿入するためのパッケージ
\usepackage[dvipdfmx]{color} % 色をつけるためのパッケージ
\usepackage{pxrubrica} % ルビを振るためのパッケージ
\usepackage{multicol} % 複数段組を作るためのパッケージ
\setlength{\topmargin}{14mm} % 上下方向のマージン
\addtolength{\topmargin}{-1in} % 
\setlength{\oddsidemargin}{11mm} % 左右方向のマージン
\addtolength{\oddsidemargin}{-1in} % 
\setlength{\textwidth}{154mm} % B6 用
\setlength{\textheight}{108mm} % B6 用
\setlength{\headsep}{0mm} % 
\setlength{\headheight}{0mm} % 
\setlength{\topskip}{0mm} % 
\setlength{\parskip}{0pt} % 
\def\labelenumi{\theenumi、} % 箇条書きのフォーマット
\parindent = 0pt % 段落下げしない

 % B6 用テンプレート読み込み

\begin{document}
% begin header
%%%%% タイトルと作者 ここから %%%%%
\begin{minipage}[c]{0.7\hsize} % タイトルは上から 7 割
    \begin{center}
    % begin title
        {\LARGE
            楡陵に月は % タイトルを入れる
        }
        {\small 
            (昭和四十七年寮歌) % 年などを入れる
        }
    % end title
    \end{center}
\end{minipage}
\begin{minipage}[c]{0.3\hsize} % 作歌作曲は上から 3 割
    \begin{flushright} % 下寄せにする
        % begin name
        加藤秀弘君 作歌\\矢野哲憲君 作曲 % 作歌・作曲者
        % end name
    \end{flushright}
\end{minipage}
%%%%% タイトルと作者 ここまで %%%%%
% (1,6 了あり)
% end header

% begin length
\vspace{1.5em} % タイトル, 作者と歌詞の間に隙間を設ける
\newcommand{\linespace}{0.5em} % 行間の設定
\newcommand{\blocksize}{0.5\hsize} % 段組間の設定
\newcommand{\itemmargin}{6em} % 曲番の位置調整の長さ
% end length
% begin body
%%%%% 歌詞 ここから %%%%%
\begin{enumerate} % 番号の箇条書き ここから
    \setlength{\itemindent}{\itemmargin} % 曲番の位置調整
    \begin{minipage}[c]{\blocksize}
    
        \vspace{\linespace}
        \item~\\
        % 1.
        \ruby{楡陵}{}に\ruby{月}{}は\ruby{懸}{}れども\\
        \ruby{星霜深}{}き\ruby{原始林暗}{}し\\
        \ruby{蓁萋}{}ゆらぐ\ruby{風有}{}れど\\
        \ruby{思}{}い\ruby{分}{}かたん\ruby{術}{}も\ruby{無}{}し
        
        \vspace{\linespace}
        \item~\\
        % 2.
        \ruby{天空破}{}る\ruby{落雷}{}はあれど\\
        そびゆる\ruby{聳天樹}{}は\ruby{堂々}{}と\\
        \ruby{慟哭}{}の\ruby{声上}{}げらんと\\
        \ruby{意気揺籃}{}の\ruby{時}{}は\ruby{今}{}
        
        \vspace{\linespace}
        \item~\\
        % 3.
        \ruby{銀晶}{}ふるう\ruby{雪原}{}なれども\\
        \ruby{変}{}らぬ\ruby{沈黙奇}{}しきかな\\
        \ruby{黄鶴消}{}えて\ruby{姿無}{}し\\
        \ruby{蘇}{}える\ruby{春}{}まだ\ruby{遠}{}く
        
        \vspace{\linespace}
        \item~\\
        % 4.
        \ruby{鐘}{}の\ruby{音遠}{}く\ruby{聞}{}えども\\
        \ruby{雑踏}{}の\ruby{声}{}さざめきの\\
        \ruby{辛夷花咲}{}く\ruby{黎明}{}よ\\
        \ruby{石狩}{}の\ruby{野今何処}{}
        
        \vspace{\linespace}
        \item~\\
        % 5.
        \ruby{無尽}{}の\ruby{星}{}を\ruby{仰}{}げども\\
        \ruby{天}{}に\ruby{無双}{}の\ruby{北斗星}{}\\
        \ruby{白亜}{}の\ruby{城}{}に\ruby{覚醒}{}し\\
        \ruby{永遠}{}の\ruby{生命}{}を\ruby{誦}{}わなん
        
        \vspace{\linespace}
        \item~\\
        % 6.
        \ruby{未明}{}に\ruby{懸}{}る\ruby{白}{}き\ruby{月}{}\\
        \ruby{夢見}{}し\ruby{思}{}う\ruby{北溟}{}の\ruby{海}{}\\
        \ruby{憧}{}れ\ruby{来}{}しは\ruby{北溟}{}の\ruby{峰}{}\\
        \ruby{呼々我前途}{}の\ruby{行}{}く\ruby{果}{}は
    
    \end{minipage}
\end{enumerate} % 番号の箇条書き ここまで
%%%%% 歌詞 ここまで %%%%%
% end body

\end{document}

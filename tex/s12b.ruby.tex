\documentclass[10pt,b5j]{tarticle} % B6 縦書き
% \documentclass[10pt,b5j]{tarticle} % B6 縦書き
\AtBeginDvi{\special{papersize=128mm,182mm}} % B6 用用紙サイズ
\usepackage{otf} % Unicode で字を入力するのに必要なパッケージ
\usepackage[size=b6j]{bxpapersize} % B6 用紙サイズを指定
\usepackage[dvipdfmx]{graphicx} % 画像を挿入するためのパッケージ
\usepackage[dvipdfmx]{color} % 色をつけるためのパッケージ
\usepackage{pxrubrica} % ルビを振るためのパッケージ
\usepackage{multicol} % 複数段組を作るためのパッケージ
\setlength{\topmargin}{14mm} % 上下方向のマージン
\addtolength{\topmargin}{-1in} % 
\setlength{\oddsidemargin}{11mm} % 左右方向のマージン
\addtolength{\oddsidemargin}{-1in} % 
\setlength{\textwidth}{154mm} % B6 用
\setlength{\textheight}{108mm} % B6 用
\setlength{\headsep}{0mm} % 
\setlength{\headheight}{0mm} % 
\setlength{\topskip}{0mm} % 
\setlength{\parskip}{0pt} % 
\def\labelenumi{\theenumi、} % 箇条書きのフォーマット
\parindent = 0pt % 段落下げしない

 % B6 用テンプレート読み込み

\begin{document}
% begin header
%%%%% タイトルと作者 ここから %%%%%
\begin{minipage}[c]{0.7\hsize} % タイトルは上から 7 割
    \begin{center}
    % begin title
        {\LARGE
            春未だ浅き % タイトルを入れる
        }
        {\small 
            (昭和十二年第三十回記念祭歌) % 年などを入れる
        }
    % end title
    \end{center}
\end{minipage}
\begin{minipage}[c]{0.3\hsize} % 作歌作曲は上から 3 割
    \begin{flushright} % 下寄せにする
        % begin name
        平城鷹雄君 作歌\\宍戸昌夫君 作曲 % 作歌・作曲者
        % end name
    \end{flushright}
\end{minipage}
%%%%% タイトルと作者 ここまで %%%%%
% (1,2,3,4 了あり)
% end header

% begin length
\vspace{1.5em} % タイトル, 作者と歌詞の間に隙間を設ける
\newcommand{\linespace}{0.5em} % 行間の設定
\newcommand{\blocksize}{0.5\hsize} % 段組間の設定
\newcommand{\itemmargin}{3em} % 曲番の位置調整の長さ
% end length
% begin body
%%%%% 歌詞 ここから %%%%%
\begin{enumerate} % 番号の箇条書き ここから
    \setlength{\itemindent}{\itemmargin} % 曲番の位置調整
    \begin{minipage}[c]{\blocksize}
    
        \vspace{\linespace}
        \item~\\
        % 1.
        \ruby{春未}{}だ\ruby{浅}{}き\ruby{白楊}{}の\\
        \ruby{雪解}{}の\ruby{小路}{}たたずめば\\
        しばし\ruby{聞}{}けとて\ruby{私語}{}の\\
        \ruby{木}{}の\ruby{間}{}もれくる\ruby{夕嵐}{}
        
    \end{minipage}
    \begin{minipage}[c]{\blocksize}
        
        \vspace{\linespace}
        \item~\\
        % 2.
        あはく\ruby{足}{}げに\ruby{咲}{}き\ruby{出}{}でし\\
        おぼろおぼろの\ruby{水芭蕉}{}\\
        なつかしの\ruby{原始杜肩}{}とりて\\
        \ruby{榾火}{}をめぐり\ruby{歌}{}はなん
        
    \end{minipage}
    \begin{minipage}[c]{\blocksize}
        
        \vspace{\linespace}
        \item~\\
        % 3.
        \ruby{長髪頬}{}に\ruby{戯}{}むれて\\
        \ruby{昔変}{}らぬ\ruby{風}{}なれや\\
        \ruby{今}{}したたへん\ruby{三十回}{}の\\
        \ruby{青史}{}をかざす\ruby{記念祭}{}
        
    \end{minipage}
    \begin{minipage}[c]{\blocksize}
        
        \vspace{\linespace}
        \item~\\
        % 4.
        \ruby{美酒}{}の\ruby{夜}{}は\ruby{更}{}け\ruby{行}{}けど\\
        \ruby{尽}{}きぬ\ruby{男子}{}の\ruby{黒潮}{}を\\
        \ruby{契}{}の\ruby{杯}{}に\ruby{汲}{}み\ruby{交}{}はし\\
        \ruby{常緑}{}を\ruby{祝}{}ふ\ruby{自治}{}の\ruby{宴}{}
    
    \end{minipage}
\end{enumerate} % 番号の箇条書き ここまで
%%%%% 歌詞 ここまで %%%%%
% end body

\end{document}

\documentclass[10pt,b5j]{tarticle} % B6 縦書き
% \documentclass[10pt,b5j]{tarticle} % B6 縦書き
\AtBeginDvi{\special{papersize=128mm,182mm}} % B6 用用紙サイズ
\usepackage{otf} % Unicode で字を入力するのに必要なパッケージ
\usepackage[size=b6j]{bxpapersize} % B6 用紙サイズを指定
\usepackage[dvipdfmx]{graphicx} % 画像を挿入するためのパッケージ
\usepackage[dvipdfmx]{color} % 色をつけるためのパッケージ
\usepackage{pxrubrica} % ルビを振るためのパッケージ
\usepackage{multicol} % 複数段組を作るためのパッケージ
\setlength{\topmargin}{14mm} % 上下方向のマージン
\addtolength{\topmargin}{-1in} % 
\setlength{\oddsidemargin}{11mm} % 左右方向のマージン
\addtolength{\oddsidemargin}{-1in} % 
\setlength{\textwidth}{154mm} % B6 用
\setlength{\textheight}{108mm} % B6 用
\setlength{\headsep}{0mm} % 
\setlength{\headheight}{0mm} % 
\setlength{\topskip}{0mm} % 
\setlength{\parskip}{0pt} % 
\def\labelenumi{\theenumi、} % 箇条書きのフォーマット
\parindent = 0pt % 段落下げしない

 % B6 用テンプレート読み込み

\begin{document}
% begin header
%%%%% タイトルと作者 ここから %%%%%
\begin{minipage}[c]{0.7\hsize} % タイトルは上から 7 割
    \begin{center}
    % begin title
        {\LARGE
            穹蒼高く % タイトルを入れる
        }
        {\small 
            (大正5年南寮寮歌) % 年などを入れる
        }
    % end title
    \end{center}
\end{minipage}
\begin{minipage}[c]{0.3\hsize} % 作歌作曲は上から 3 割
    \begin{flushright} % 下寄せにする
        % begin name
        長崎次郎君 作歌\\黒住須賀夫君 作曲 % 作歌・作曲者
        % end name
    \end{flushright}
\end{minipage}
%%%%% タイトルと作者 ここまで %%%%%
% (1,2,3 了あり)
% end header

% begin body
\vspace{1.5em} % タイトル, 作者と歌詞の間に隙間を設ける
\newcommand{\linespace}{0.5em} % 行間の設定
\newcommand{\blocksize}{0.5\hsize} % 段組間の設定
%%%%% 歌詞 ここから %%%%%
% begin lilycs
\begin{enumerate} % 番号の箇条書き ここから
    \begin{minipage}[c]{\blocksize}
    
        \vspace{\linespace}
        \item
        % 1.
        \ruby{穹蒼高}{}く\ruby{夜}{}は\ruby{深}{}く\\
        \ruby{沈黙}{}の\ruby{杜}{}に\ruby{聳}{}えたつ\\
        \ruby{桂}{}の\ruby{梢指}{}すところ\\
        \ruby{北斗}{}の\ruby{冴}{}に\ruby{君}{}みずや\\
        「\ruby{吾}{}が\ruby{若人}{}よ\ruby{汝}{}が\ruby{野心}{}\\
        われにかも\ruby{似}{}て\ruby{崇}{}くあれ」
        
        \vspace{\linespace}
        \item
        % 2.
        \ruby{荒}{}ぶ\ruby{吹雪}{}のもだすとき\\
        \ruby{六片}{}の\ruby{花咲}{}くところ\\
        \ruby{皎}{}たる\ruby{天地塵絶}{}えて\\
        \ruby{塞}{}つる\ruby{力}{}を\ruby{君}{}よ\ruby{知}{}れ\\
        「\ruby{吾}{}が\ruby{若人}{}よ\ruby{北}{}の\ruby{曠野}{}に\\
        \ruby{身}{}を\ruby{練}{}り\ruby{魂}{}を\ruby{磨}{}かずや」
        
        \vspace{\linespace}
        \item
        % 3.
        \ruby{谷間}{}の\ruby{百合}{}の\ruby{香}{}のゆらぎ\\
        \ruby{楡}{}の\ruby{若葉}{}に\ruby{陽}{}はこぼる\\
        \ruby{春}{}の\ruby{息吹}{}に\ruby{渡}{}り\ruby{行}{}く\\
        \ruby{時鐘}{}の\ruby{響}{}きに\ruby{君}{}よ\ruby{聴}{}け\\
        「\ruby{吾}{}が\ruby{若人}{}よ\ruby{石狩}{}は\\
        \ruby{自由}{}の\ruby{郷土}{}ぞ\ruby{幸多}{}き」
        
        \vspace{\linespace}
        \item
        % 4.
        \ruby{百鳥歌}{}ひ\ruby{花}{}は\ruby{笑}{}む\\
        \ruby{美}{}しき\ruby{国}{}の\ruby{自治}{}の\ruby{家}{}に\\
        \ruby{十一}{}の\ruby{春今日来}{}る\\
        \ruby{祝歌}{}たかく\ruby{君歌}{}へ\\
        「\ruby{迪}{}に\ruby{恵}{}ふ\ruby{若人}{}の\\
        \ruby{住家}{}よ\ruby{永}{}に\ruby{栄}{}あれ」
        
        \vspace{\linespace}
        \item
        % 5.
        \ruby{崇}{}きのぞみを\ruby{星}{}に\ruby{懸}{}け\\
        \ruby{鐘}{}に\ruby{自由}{}を\ruby{学}{}びつつ\\
        \ruby{真理}{}を\ruby{求}{}むる\ruby{一百}{}の\\
        \ruby{健児}{}が\ruby{行手遠}{}けれど\\
        \ruby{吾若}{}き\ruby{力強}{}ければ\\
        \ruby{贏}{}む\ruby{秋}{}は\ruby{近}{}からむ\\
        など\ruby{贏}{}ざる\ruby{事}{}あらん
    
    \end{minipage}
\end{enumerate} % 番号の箇条書き ここまで
% end lilycs
%%%%% 歌詞 ここまで %%%%%
% end body

\end{document}

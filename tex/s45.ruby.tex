\documentclass[10pt,b5j]{tarticle} % B6 縦書き
% \documentclass[10pt,b5j]{tarticle} % B6 縦書き
\AtBeginDvi{\special{papersize=128mm,182mm}} % B6 用用紙サイズ
\usepackage{otf} % Unicode で字を入力するのに必要なパッケージ
\usepackage[size=b6j]{bxpapersize} % B6 用紙サイズを指定
\usepackage[dvipdfmx]{graphicx} % 画像を挿入するためのパッケージ
\usepackage[dvipdfmx]{color} % 色をつけるためのパッケージ
\usepackage{pxrubrica} % ルビを振るためのパッケージ
\usepackage{plext} % 漢数字の enumerate を使うためのパッケージ
\usepackage{multicol} % 複数段組を作るためのパッケージ
\setlength{\topmargin}{14mm} % 上下方向のマージン
\addtolength{\topmargin}{-1in} % 
\setlength{\oddsidemargin}{11mm} % 左右方向のマージン
\addtolength{\oddsidemargin}{-1in} % 
\setlength{\textwidth}{154mm} % B6 用
\setlength{\textheight}{108mm} % B6 用
\setlength{\headsep}{0mm} % 
\setlength{\headheight}{0mm} % 
\setlength{\topskip}{0mm} % 
\setlength{\parskip}{0pt} % 
\def\theenumi{\Kanji{enumi}} % 箇条書きのフォーマットを漢数字に変更
\parindent = 0pt % 段落下げしない
\pagestyle{empty} % すべてのページ番号を消去
% \renewcommand{\baselinestretch}{0.9} % 行間の倍率
 % B6 用テンプレート読み込み

\begin{document}
% begin header
%%%%% タイトルと作者 ここから %%%%%
\begin{minipage}[c]{0.7\hsize} % タイトルは上から 7 割
    \begin{center}
    % begin title
        {\LARGE
            秋逍遙 % タイトルを入れる
        }
        {\small 
            (昭和四十五年寮歌) % 年などを入れる
        }
    % end title
    \end{center}
\end{minipage}
\begin{minipage}[c]{0.3\hsize} % 作歌作曲は上から 3 割
    \begin{flushright} % 下寄せにする
        % begin name
        熊野芳明君 作歌\\吉田守男君 作曲 % 作歌・作曲者
        % end name
    \end{flushright}
\end{minipage}
%%%%% タイトルと作者 ここまで %%%%%
% (未明,払暁,初更,深更
% end header

% begin length
\vspace{1.5em} % タイトル, 作者と歌詞の間に隙間を設ける
\newcommand{\linespace}{0.5em} % 行間の設定
\newcommand{\blocksize}{0.5\hsize} % 段組間の設定
\newcommand{\itemmargin}{6em} % 曲番の位置調整の長さ
% end length
% begin body
%%%%% 歌詞 ここから %%%%%
\begin{enumerate} % 番号の箇条書き ここから
    \setlength{\itemindent}{\itemmargin} % 曲番の位置調整
    \begin{minipage}[c]{\blocksize}
    
        \vspace{\linespace}
        \item~\\
        % \ruby{未明}{}.
        \ruby{秋}{}に\ruby{秋添}{}う\ruby{時雨月}{}\\
        \ruby{曙星瞬}{}く\ruby{恋々}{}と\\
        されど\ruby{近}{}づく\ruby{蕭晨}{}に\\
        \ruby{幽愁}{}はつのるせつなくも\\
        \ruby{落涙}{}しばし\ruby{悄然}{}と
        
        \vspace{\linespace}
        \item~\\
        % \ruby{払暁}{}.
        \ruby{蕭晨}{}は\ruby{来}{}にけり\ruby{石狩野}{}\\
        \ruby{野菊}{}に\ruby{滴}{}る\ruby{血}{}の\ruby{雫}{}\\
        \ruby{木}{}の\ruby{葉}{}さやぎぬ\ruby{涼風}{}に\\
        \ruby{野}{}を\ruby{流離}{}えば\ruby{深}{}き\ruby{哀愁}{}\\
        \ruby{情}{}けの\ruby{露}{}を\ruby{探求}{}むなり
        
        \vspace{\linespace}
        \item~\\
        % \ruby{昼}{}.
        \ruby{遥}{}かに\ruby{煙}{}る\ruby{大平原}{}\\
        \ruby{蕭然秋}{}の\ruby{小糠雨}{}\\
        \ruby{原生林}{}の\ruby{錦}{}も\ruby{色寂}{}し\\
        \ruby{黒俊馬}{}の\ruby{長嘶}{}に\ruby{沈思破}{}れ\\
        \ruby{秋}{}の\ruby{情趣}{}を\ruby{知}{}る\ruby{二十}{}
        
        \vspace{\linespace}
        \item~\\
        % \ruby{落葉}{}.
        \ruby{時雨}{}もやみてあかねさす\\
        \ruby{赤紫雲}{}の\ruby{黄昏}{}に\\
        \ruby{夕陽返}{}し\ruby{珠玉}{}の\ruby{如}{}\\
        \ruby{蜻蛉}{}が\ruby{翅翎}{}に\ruby{我}{}が\ruby{久懐}{}\\
        \ruby{真情}{}の\ruby{友}{}へと\ruby{託}{}すかな
        
        \vspace{\linespace}
        \item~\\
        % \ruby{初更}{}.
        \ruby{釣瓶落}{}しの\ruby{秋}{}の\ruby{日}{}の\\
        \ruby{紫紺}{}の\ruby{闇}{}に\ruby{淡}{}く\ruby{浮}{}く\\
        \ruby{利鎌}{}の\ruby{秋月}{}はあな\ruby{悲}{}し\\
        きらめく\ruby{長庚}{}にただ\ruby{涙}{}\\
        \ruby{己}{}が\ruby{運命}{}か\ruby{斯}{}くあるが
        
        \vspace{\linespace}
        \item~\\
        % \ruby{深更}{}.
        \ruby{夢幻}{}か\ruby{人}{}の\ruby{夢}{}は\\
        \ruby{秋}{}の\ruby{百子夜}{}に\ruby{我悄然}{}\\
        \ruby{地平}{}の\ruby{彼方}{}へ\ruby{冴星空}{}を\\
        \ruby{過}{}りて\ruby{落}{}つる\ruby{流}{}れ\ruby{星}{}\\
        ただただ\ruby{涙}{}は\ruby{何故}{}か
    
    \end{minipage}
\end{enumerate} % 番号の箇条書き ここまで
%%%%% 歌詞 ここまで %%%%%
% end body

\end{document}

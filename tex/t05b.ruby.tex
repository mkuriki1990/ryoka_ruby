\documentclass[10pt,b5j]{tarticle} % B6 縦書き
% \documentclass[10pt,b5j]{tarticle} % B6 縦書き
\AtBeginDvi{\special{papersize=128mm,182mm}} % B6 用用紙サイズ
\usepackage{otf} % Unicode で字を入力するのに必要なパッケージ
\usepackage[size=b6j]{bxpapersize} % B6 用紙サイズを指定
\usepackage[dvipdfmx]{graphicx} % 画像を挿入するためのパッケージ
\usepackage[dvipdfmx]{color} % 色をつけるためのパッケージ
\usepackage{pxrubrica} % ルビを振るためのパッケージ
\usepackage{multicol} % 複数段組を作るためのパッケージ
\setlength{\topmargin}{14mm} % 上下方向のマージン
\addtolength{\topmargin}{-1in} % 
\setlength{\oddsidemargin}{11mm} % 左右方向のマージン
\addtolength{\oddsidemargin}{-1in} % 
\setlength{\textwidth}{154mm} % B6 用
\setlength{\textheight}{108mm} % B6 用
\setlength{\headsep}{0mm} % 
\setlength{\headheight}{0mm} % 
\setlength{\topskip}{0mm} % 
\setlength{\parskip}{0pt} % 
\def\labelenumi{\theenumi、} % 箇条書きのフォーマット
\parindent = 0pt % 段落下げしない

 % B6 用テンプレート読み込み

\begin{document}
% begin header
%%%%% タイトルと作者 ここから %%%%%
\begin{minipage}[c]{0.7\hsize} % タイトルは上から 7 割
    \begin{center}
    % begin title
        {\LARGE
            荒潮繞る % タイトルを入れる
        }
        {\small 
            (大正五年北寮寮歌) % 年などを入れる
        }
    % end title
    \end{center}
\end{minipage}
\begin{minipage}[c]{0.3\hsize} % 作歌作曲は上から 3 割
    \begin{flushright} % 下寄せにする
        % begin name
        桜井芳次郎君 作歌\\橋本吉郎君 作曲 % 作歌・作曲者
        % end name
    \end{flushright}
\end{minipage}
%%%%% タイトルと作者 ここまで %%%%%
% (1,2,3,4,5,6,7 了あり)
% end header

% begin body
\vspace{1.5em} % タイトル, 作者と歌詞の間に隙間を設ける
\newcommand{\linespace}{0.5em} % 行間の設定
\newcommand{\blocksize}{0.5\hsize} % 段組間の設定
%%%%% 歌詞 ここから %%%%%
% begin lilycs
\begin{enumerate} % 番号の箇条書き ここから
    \begin{minipage}[c]{\blocksize}
    
        \vspace{\linespace}
        \item
        % 1.
        \ruby{荒潮繞}{}る\ruby{北}{}の\ruby{郷}{}\\
        \ruby{絢爛}{}の\ruby{時}{}いと\ruby{高}{}く\\
        \ruby{看}{}よ\ruby{極光}{}に\ruby{照}{}らされて\\
        \ruby{夢}{}にまどろむ\ruby{春}{}の\ruby{精}{}
        
        \vspace{\linespace}
        \item
        % 2.
        \ruby{嗚呼感激}{}の\ruby{経営}{}を\\
        \ruby{矜}{}る\ruby{血潮}{}に\ruby{求}{}め\ruby{来}{}て\\
        \ruby{十一}{}の\ruby{年}{}の\ruby{旦暮}{}は\\
        \ruby{澄明}{}の\ruby{府霊清}{}し
        
        \vspace{\linespace}
        \item
        % 3.
        \ruby{夏}{}の\ruby{日悠然}{}に\ruby{石狩}{}の\\
        \ruby{浩蕩}{}の\ruby{水煌}{}めきて\\
        \ruby{流光高}{}く\ruby{際涯}{}なき\\
        \ruby{自然}{}の\ruby{業}{}を\ruby{畏}{}れずや
        
        \vspace{\linespace}
        \item
        % 4.
        \ruby{夕暮}{}れ\ruby{呼}{}ばふ\ruby{閑古鳥}{}\\
        \ruby{冥想}{}ここに\ruby{始}{}めよと\\
        \ruby{遠鳴}{}くなべも\ruby{紅葉}{}しつ\\
        \ruby{稜畳}{}として\ruby{唐錦}{}
        
        \vspace{\linespace}
        \item
        % 5.
        \ruby{北風胡沙}{}に\ruby{雪}{}を\ruby{捲}{}き\\
        \ruby{荒}{}れ\ruby{狂}{}ひたる\ruby{戦場}{}の\ruby{跡}{}\\
        \ruby{暮}{}れ\ruby{行}{}く\ruby{蛮霧}{}に\ruby{包}{}まれて\\
        \ruby{白銀}{}の\ruby{都今静}{}か
        
        \vspace{\linespace}
        \item
        % 6.
        \ruby{清}{}けき\ruby{永久}{}の\ruby{霊泉}{}の\\
        \ruby{至福}{}の\ruby{水}{}を\ruby{掬}{}ぶ\ruby{可}{}く\\
        \ruby{黄金}{}の\ruby{甕守}{}りつつ\\
        \ruby{調新}{}しく\ruby{唱}{}はなん
        
        \vspace{\linespace}
        \item
        % 7.
        \ruby{智慧}{}の\ruby{光}{}に\ruby{導}{}かれ\\
        \ruby{熱}{}の\ruby{磅礴}{}に\ruby{生立}{}ちて\\
        \ruby{潔}{}き\ruby{生活}{}の\ruby{道}{}すがら\\
        \ruby{曲勇}{}ましく\ruby{唱}{}はなむ
    
    \end{minipage}
\end{enumerate} % 番号の箇条書き ここまで
% end lilycs
%%%%% 歌詞 ここまで %%%%%
% end body

\end{document}

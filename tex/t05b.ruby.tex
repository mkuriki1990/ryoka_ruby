\documentclass[10pt,b5j]{tarticle} % B6 縦書き
% \documentclass[10pt,b5j]{tarticle} % B6 縦書き
\AtBeginDvi{\special{papersize=128mm,182mm}} % B6 用用紙サイズ
\usepackage{otf} % Unicode で字を入力するのに必要なパッケージ
\usepackage[size=b6j]{bxpapersize} % B6 用紙サイズを指定
\usepackage[dvipdfmx]{graphicx} % 画像を挿入するためのパッケージ
\usepackage[dvipdfmx]{color} % 色をつけるためのパッケージ
\usepackage{pxrubrica} % ルビを振るためのパッケージ
\usepackage{plext} % 漢数字の enumerate を使うためのパッケージ
\usepackage{multicol} % 複数段組を作るためのパッケージ
\setlength{\topmargin}{14mm} % 上下方向のマージン
\addtolength{\topmargin}{-1in} % 
\setlength{\oddsidemargin}{11mm} % 左右方向のマージン
\addtolength{\oddsidemargin}{-1in} % 
\setlength{\textwidth}{154mm} % B6 用
\setlength{\textheight}{108mm} % B6 用
\setlength{\headsep}{0mm} % 
\setlength{\headheight}{0mm} % 
\setlength{\topskip}{0mm} % 
\setlength{\parskip}{0pt} % 
\def\theenumi{\Kanji{enumi}} % 箇条書きのフォーマットを漢数字に変更
\parindent = 0pt % 段落下げしない
\pagestyle{empty} % すべてのページ番号を消去
% \renewcommand{\baselinestretch}{0.9} % 行間の倍率
 % B6 用テンプレート読み込み

\begin{document}
% begin header
%%%%% タイトルと作者 ここから %%%%%
\begin{minipage}[c]{0.7\hsize} % タイトルは上から 7 割
    \begin{center}
    % begin title
        {\LARGE
            荒潮繞る % タイトルを入れる
        }
        {\small 
            (大正五年北寮寮歌) % 年などを入れる
        }
    % end title
    \end{center}
\end{minipage}
\begin{minipage}[c]{0.3\hsize} % 作歌作曲は上から 3 割
    \begin{flushright} % 下寄せにする
        % begin name
        桜井芳次郎君 作歌\\橋本吉郎君 作曲 % 作歌・作曲者
        % end name
    \end{flushright}
\end{minipage}
%%%%% タイトルと作者 ここまで %%%%%
% (1,2,3,4,5,6,7 了あり)
% end header

% begin length
\vspace{1.5em} % タイトル, 作者と歌詞の間に隙間を設ける
\newcommand{\linespace}{0.5em} % 行間の設定
\newcommand{\blocksize}{0.5\hsize} % 段組間の設定
\newcommand{\itemmargin}{3em} % 曲番の位置調整の長さ
% end length
% begin body
%%%%% 歌詞 ここから %%%%%
\begin{enumerate} % 番号の箇条書き ここから
    \setlength{\itemindent}{\itemmargin} % 曲番の位置調整
    \begin{minipage}[c]{\blocksize}
    
        \vspace{\linespace}
        \item~\\
        % 1.
        \ruby{荒}{あら}\ruby{潮}{うしお}\ruby{繞}{にょう}る\ruby{北}{きた}の\ruby{郷}{さと}\\
        \ruby{絢爛}{けんらん}の\ruby{時}{とき}いと\ruby{高}{たか}く\\
        \ruby{看}{み}よ\ruby{極光}{きょっこう}に\ruby{照}{て}らされて\\
        \ruby{夢}{ゆめ}にまどろむ\ruby{春}{はる}の\ruby{精}{せい}
        
    \end{minipage}
    \begin{minipage}[c]{\blocksize}
        
        \vspace{\linespace}
        \item~\\
        % 2.
        \ruby{嗚呼}{ああ}\ruby{感激}{かんげき}の\ruby{経営}{けいえい}を\\
        \ruby{矜}{}る\ruby{血潮}{ちしお}に\ruby{求}{もと}め\ruby{来}{き}て\\
        \ruby{十}{じゅう}\ruby{一}{いち}の\ruby{年}{とし}の\ruby{旦暮}{たんぼ}は\\
        \ruby{澄明}{ちょうめい}の\ruby{府}{ふ}\ruby{霊}{れい}\ruby{清}{きよ}し
        
    \end{minipage}
    \begin{minipage}[c]{\blocksize}
        
        \vspace{\linespace}
        \item~\\
        % 3.
        \ruby{夏}{なつ}の\ruby{日}{ひ}\ruby{悠然}{ゆうぜん}に\ruby{石狩}{いしかり}の\\
        \ruby{浩}{ひろし}\ruby{蕩}{}の\ruby{水}{みず}\ruby{煌}{きら}めきて\\
        \ruby{流}{ながれ}\ruby{光}{ひかり}\ruby{高}{たか}く\ruby{際涯}{さいがい}なき\\
        \ruby{自然}{しぜん}の\ruby{業}{ごう}を\ruby{畏}{おそ}れずや
        
    \end{minipage}
    \begin{minipage}[c]{\blocksize}
        
        \vspace{\linespace}
        \item~\\
        % 4.
        \ruby{夕暮}{ゆうぐ}れ\ruby{呼}{よ}ばふ\ruby{閑古鳥}{かんこどり}\\
        \ruby{冥}{めい}\ruby{想}{そう}ここに\ruby{始}{はじ}めよと\\
        \ruby{遠}{とお}\ruby{鳴}{な}くなべも\ruby{紅葉}{こうよう}しつ\\
        \ruby{稜}{りょう}\ruby{畳}{たたみ}として\ruby{唐錦}{からにしき}
        
    \end{minipage}
    \begin{minipage}[c]{\blocksize}
        
        \vspace{\linespace}
        \item~\\
        % 5.
        \ruby{北風}{きたかぜ}\ruby{胡}{えびす}\ruby{沙}{いさご}に\ruby{雪}{ゆき}を\ruby{捲}{ま}き\\
        \ruby{荒}{あ}れ\ruby{狂}{きょう}ひたる\ruby{戦場}{せんじょう}の\ruby{跡}{あと}\\
        \ruby{暮}{く}れ\ruby{行}{い}く\ruby{蛮霧}{}に\ruby{包}{つつ}まれて\\
        \ruby{白銀}{はくぎん}の\ruby{都}{と}\ruby{今}{いま}\ruby{静}{しず}か
        
    \end{minipage}
    \begin{minipage}[c]{\blocksize}
        
        \vspace{\linespace}
        \item~\\
        % 6.
        \ruby{清}{さや}けき\ruby{永久}{えいきゅう}の\ruby{霊泉}{れいせん}の\\
        \ruby{至福}{しふく}の\ruby{水}{みず}を\ruby{掬}{むす}ぶ\ruby{可}{か}く\\
        \ruby{黄金}{おうごん}の\ruby{甕}{う}\ruby{守}{まも}りつつ\\
        \ruby{調}{ちょう}\ruby{新}{あたら}しく\ruby{唱}{}はなん
        
    \end{minipage}
    \begin{minipage}[c]{\blocksize}
        
        \vspace{\linespace}
        \item~\\
        % 7.
        \ruby{智慧}{ちえ}の\ruby{光}{ひかり}に\ruby{導}{みちび}かれ\\
        \ruby{熱}{ねつ}の\ruby{磅礴}{}に\ruby{生立}{おいた}ちて\\
        \ruby{潔}{いさぎよ}き\ruby{生活}{せいかつ}の\ruby{道}{みち}すがら\\
        \ruby{曲}{きょく}\ruby{勇}{いさ}ましく\ruby{唱}{}はなむ
    
    \end{minipage}
\end{enumerate} % 番号の箇条書き ここまで
%%%%% 歌詞 ここまで %%%%%
% end body

\end{document}

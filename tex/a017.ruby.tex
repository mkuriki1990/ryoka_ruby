\documentclass[10pt,b5j]{tarticle} % B6 縦書き
% \documentclass[10pt,b5j]{tarticle} % B6 縦書き
\AtBeginDvi{\special{papersize=128mm,182mm}} % B6 用用紙サイズ
\usepackage{otf} % Unicode で字を入力するのに必要なパッケージ
\usepackage[size=b6j]{bxpapersize} % B6 用紙サイズを指定
\usepackage[dvipdfmx]{graphicx} % 画像を挿入するためのパッケージ
\usepackage[dvipdfmx]{color} % 色をつけるためのパッケージ
\usepackage{pxrubrica} % ルビを振るためのパッケージ
\usepackage{multicol} % 複数段組を作るためのパッケージ
\setlength{\topmargin}{14mm} % 上下方向のマージン
\addtolength{\topmargin}{-1in} % 
\setlength{\oddsidemargin}{11mm} % 左右方向のマージン
\addtolength{\oddsidemargin}{-1in} % 
\setlength{\textwidth}{154mm} % B6 用
\setlength{\textheight}{108mm} % B6 用
\setlength{\headsep}{0mm} % 
\setlength{\headheight}{0mm} % 
\setlength{\topskip}{0mm} % 
\setlength{\parskip}{0pt} % 
\def\labelenumi{\theenumi、} % 箇条書きのフォーマット
\parindent = 0pt % 段落下げしない

 % B6 用テンプレート読み込み

\begin{document}
% begin header
%%%%% タイトルと作者 ここから %%%%%
\begin{minipage}[c]{0.7\hsize} % タイトルは上から 7 割
    \begin{center}
    % begin title
        {\LARGE
            MARCHING SONG % タイトルを入れる
        }
        {\small 
            (1915年) % 年などを入れる
        }
    % end title
    \end{center}
\end{minipage}
\begin{minipage}[c]{0.3\hsize} % 作歌作曲は上から 3 割
    \begin{flushright} % 下寄せにする
        % begin name
        Prof.Paul Rowland君 作歌・作曲 % 作歌・作曲者
        % end name
    \end{flushright}
\end{minipage}
%%%%% タイトルと作者 ここまで %%%%%
% % end header

% begin length
\vspace{1.5em} % タイトル, 作者と歌詞の間に隙間を設ける
\newcommand{\linespace}{0.5em} % 行間の設定
\newcommand{\blocksize}{0.5\hsize} % 段組間の設定
\newcommand{\itemmargin}{6em} % 曲番の位置調整の長さ
% end length
% begin body
%%%%% 歌詞 ここから %%%%%
\begin{enumerate} % 番号の箇条書き ここから
    \setlength{\itemindent}{\itemmargin} % 曲番の位置調整
    \begin{minipage}[c]{\blocksize}
    
        \vspace{\linespace}
        \item~\\
        % 1.
        Come all ye men of Sapporo,\\
          Come sing our far-fam'd college;\\
        Our hearts afire with high desire,\\
          We're here to share her knowledge:\\
        By Ishikari's mighty flood,\\
          With mountains firm defending,\\
            Give a cheer, cheer, cheer,\\
            Four our mother dear,\\
          Her thousand sons befriending.
        
        \vspace{\linespace}
        \item~\\
        % 2.
        Come winter snow and blizzards blow,\\
          We'll all enjoy you keenly,\\
        And gaily ski and skate with glee\\
          Beneath the elms so queenly;\\
        With tennis, wresling, fencing, ball,\\
          Our lusty sports are glorious,\\
            Then sing, sing\\
            Till the buildings ring\\
        With sounds of joy victorious.
        
        \vspace{\linespace}
        \item~\\
        % 3.
        So when our college days are o'er,\\
          And comrades true must sever,\\
        These mem'ries old shall make us bold\\
          To meet each new endeavor:\\
        Our motto, Be Ambitious, Boys,\\
          Our guide the North Star ever,\\
            Vive la Tohoku\\
            Noka Daigaku\\
        Forever and foever!
        
        
        \vspace{\linespace}
        \item~\\
        ローランド\ruby{先生}{}(アメリカ\ruby{人}{})は\ruby{大正}{}3\ruby{年}{}\\
        から6\ruby{年}{}まで\ruby{予科}{}で\ruby{英語}{}を\ruby{教}{}えられた。
    
    \end{minipage}
\end{enumerate} % 番号の箇条書き ここまで
%%%%% 歌詞 ここまで %%%%%
% end body

\end{document}

\documentclass[10pt,b5j]{tarticle} % B6 縦書き
% \documentclass[10pt,b5j]{tarticle} % B6 縦書き
\AtBeginDvi{\special{papersize=128mm,182mm}} % B6 用用紙サイズ
\usepackage{otf} % Unicode で字を入力するのに必要なパッケージ
\usepackage[size=b6j]{bxpapersize} % B6 用紙サイズを指定
\usepackage[dvipdfmx]{graphicx} % 画像を挿入するためのパッケージ
\usepackage[dvipdfmx]{color} % 色をつけるためのパッケージ
\usepackage{pxrubrica} % ルビを振るためのパッケージ
\usepackage{multicol} % 複数段組を作るためのパッケージ
\setlength{\topmargin}{14mm} % 上下方向のマージン
\addtolength{\topmargin}{-1in} % 
\setlength{\oddsidemargin}{11mm} % 左右方向のマージン
\addtolength{\oddsidemargin}{-1in} % 
\setlength{\textwidth}{154mm} % B6 用
\setlength{\textheight}{108mm} % B6 用
\setlength{\headsep}{0mm} % 
\setlength{\headheight}{0mm} % 
\setlength{\topskip}{0mm} % 
\setlength{\parskip}{0pt} % 
\def\labelenumi{\theenumi、} % 箇条書きのフォーマット
\parindent = 0pt % 段落下げしない

 % B6 用テンプレート読み込み

\begin{document}
% begin header
%%%%% タイトルと作者 ここから %%%%%
\begin{minipage}[c]{0.7\hsize} % タイトルは上から 7 割
    \begin{center}
    % begin title
        {\LARGE
            土木工科応援歌 % タイトルを入れる
        }
        {\small 
            (大正4年) % 年などを入れる
        }
    % end title
    \end{center}
\end{minipage}
\begin{minipage}[c]{0.3\hsize} % 作歌作曲は上から 3 割
    \begin{flushright} % 下寄せにする
        % begin name
         % 作歌・作曲者
        % end name
    \end{flushright}
\end{minipage}
%%%%% タイトルと作者 ここまで %%%%%
% % end header

% begin body
\vspace{1.5em} % タイトル, 作者と歌詞の間に隙間を設ける
\newcommand{\linespace}{0.5em} % 行間の設定
\newcommand{\blocksize}{0.5\hsize} % 段組間の設定
%%%%% 歌詞 ここから %%%%%
% begin lilycs
\begin{enumerate} % 番号の箇条書き ここから
    \begin{minipage}[c]{\blocksize}
    
        \vspace{\linespace}
        \item
        % 1.
        \ruby{起}{}てや\ruby{吾等}{}が\ruby{健男児}{}\\
        \ruby{若草}{}もゆる\ruby{石狩}{}に\\
        \ruby{妖雲暗}{}く\ruby{空低}{}く\\
        \ruby{吹}{}きまく\ruby{春}{}の\ruby{初風}{}に\\
        かざせる\ruby{工科}{}の\ruby{旗高}{}く\\
        \ruby{友}{}よ\ruby{矛}{}とれ\ruby{戦}{}はむ\\
        ラーラーラー・ワン ツー スリー・オール・オール・CED
        
        \vspace{\linespace}
        \item
        % 2.
        \ruby{混濁}{}の\ruby{浪逆}{}まきて\\
        \ruby{高}{}つ\ruby{瀬}{}なして\ruby{来}{}るとも\\
        \ruby{見}{}よ\ruby{雄}{}たけびの\ruby{只中}{}に\\
        \ruby{迅風}{}の\ruby{如}{}き\ruby{吾}{}が\ruby{撰手}{}\\
        \ruby{今}{}や\ruby{戦}{}たけなはぞ\\
        \ruby{友}{}よ\ruby{矛}{}とれ\ruby{戦}{}はむ
        
        \vspace{\linespace}
        \item
        % 3.
        \ruby{勝}{}たずはやまぬ\ruby{青春}{}の\\
        \ruby{若}{}き\ruby{血}{}しほの\ruby{充}{}ち\ruby{満}{}てる\\
        \ruby{友}{}よ\ruby{矛}{}とれ\ruby{戦}{}はむ\\
        \ruby{覇権}{}を\ruby{譲}{}る\ruby{事}{}なかれ\\
        \ruby{吾等工科}{}の\ruby{健男児}{}\\
        などて\ruby{覇権}{}を\ruby{譲}{}るべき
    
    \end{minipage}
\end{enumerate} % 番号の箇条書き ここまで
% end lilycs
%%%%% 歌詞 ここまで %%%%%
% end body

\end{document}

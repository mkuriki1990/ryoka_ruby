\documentclass[10pt,b5j]{tarticle} % B6 縦書き
% \documentclass[10pt,b5j]{tarticle} % B6 縦書き
\AtBeginDvi{\special{papersize=128mm,182mm}} % B6 用用紙サイズ
\usepackage{otf} % Unicode で字を入力するのに必要なパッケージ
\usepackage[size=b6j]{bxpapersize} % B6 用紙サイズを指定
\usepackage[dvipdfmx]{graphicx} % 画像を挿入するためのパッケージ
\usepackage[dvipdfmx]{color} % 色をつけるためのパッケージ
\usepackage{pxrubrica} % ルビを振るためのパッケージ
\usepackage{plext} % 漢数字の enumerate を使うためのパッケージ
\usepackage{multicol} % 複数段組を作るためのパッケージ
\setlength{\topmargin}{14mm} % 上下方向のマージン
\addtolength{\topmargin}{-1in} % 
\setlength{\oddsidemargin}{11mm} % 左右方向のマージン
\addtolength{\oddsidemargin}{-1in} % 
\setlength{\textwidth}{154mm} % B6 用
\setlength{\textheight}{108mm} % B6 用
\setlength{\headsep}{0mm} % 
\setlength{\headheight}{0mm} % 
\setlength{\topskip}{0mm} % 
\setlength{\parskip}{0pt} % 
\def\theenumi{\Kanji{enumi}} % 箇条書きのフォーマットを漢数字に変更
\parindent = 0pt % 段落下げしない
\pagestyle{empty} % すべてのページ番号を消去
% \renewcommand{\baselinestretch}{0.9} % 行間の倍率
 % B6 用テンプレート読み込み

\begin{document}
% begin header
%%%%% タイトルと作者 ここから %%%%%
\begin{minipage}[c]{0.7\hsize} % タイトルは上から 7 割
    \begin{center}
    % begin title
        {\LARGE
            楡は枯れず % タイトルを入れる
        }
        {\small 
            (昭和55年寮歌) % 年などを入れる
        }
    % end title
    \end{center}
\end{minipage}
\begin{minipage}[c]{0.3\hsize} % 作歌作曲は上から 3 割
    \begin{flushright} % 下寄せにする
        % begin name
        新井桂二君 作歌\\奥田和人君 作曲 % 作歌・作曲者
        % end name
    \end{flushright}
\end{minipage}
%%%%% タイトルと作者 ここまで %%%%%
% (1,2,3 了なし繰り返しあり)
% end header

% begin body
\vspace{1.5em} % タイトル, 作者と歌詞の間に隙間を設ける
\newcommand{\linespace}{0.5em} % 行間の設定
\newcommand{\blocksize}{0.5\hsize} % 段組間の設定
%%%%% 歌詞 ここから %%%%%
% begin lilycs
\begin{enumerate} % 番号の箇条書き ここから
    \begin{minipage}[c]{\blocksize}
    
        \vspace{\linespace}
        \item
        \ruby{北}{}に\ruby{生}{}まれし\ruby{者}{}たちよ\\
        \ruby{北}{}に\ruby{出会}{}いし\ruby{者}{}たちよ\\
        \ruby{北}{}に\ruby{奢}{}れる\ruby{者}{}たちよ\\
        \ruby{北}{}に\ruby{歌}{}える\ruby{者}{}たちよ\\
        \ruby{永遠}{}に\ruby{祈}{}りし\\
        \ruby{朝}{}は\ruby{未}{}だかなわず\\
        \ruby{百年}{}に\ruby{織}{}りたる\ruby{衣}{}は\\
        \ruby{当}{}に\ruby{引}{}き\ruby{裂}{}かれんとす\\
        \ruby{嗚呼願}{}わくば\ruby{二度糸}{}を\ruby{紡}{}ぎて\\
        \ruby{限}{}りなく\ruby{澄}{}みわたる\\
        \ruby{穹北}{}の\ruby{空}{}に\ruby{舞}{}わん
        
        \vspace{\linespace}
        \item
        % 1.
        \ruby{朝靄}{}けむる\ruby{今}{}ひとときの\\
        \ruby{熟寝}{}の\ruby{夢}{}の\ruby{幸}{}せよ\\
        \ruby{覚}{}めて\ruby{現}{}に\ruby{見渡}{}せば\\
        \ruby{美}{}は\ruby{崩}{}れゆく\ruby{北都}{}なり\\
        \ruby{天空常}{}に\ruby{雲抱}{}けども\\
        \ruby{楡}{}は\ruby{萌}{}えて\ruby{大地}{}をまねく
        
        \vspace{\linespace}
        \item
        % 2.
        \ruby{清冽}{}の\ruby{野}{}に\ruby{道}{}を\ruby{耕}{}し\\
        \ruby{荒野}{}に\ruby{明日}{}を\ruby{信}{}じつつ\\
        \ruby{彷徨}{}い\ruby{行}{}ける\ruby{寂}{}しさに\\
        \ruby{陽}{}は\ruby{傾}{}きて\ruby{我}{}を\ruby{見}{}る\\
        \ruby{虚}{}いゆける\ruby{時}{}にこそ\\
        \ruby{楡}{}は\ruby{映}{}えて\ruby{風}{}を\ruby{斬}{}る
        
        \vspace{\linespace}
        \item
        % 3.
        \ruby{北}{}の\ruby{自然}{}は\ruby{蝕}{}ばまれゆき\\
        \ruby{青葉}{}の\ruby{降}{}るや\ruby{青春}{}の\ruby{寮庭}{}\\
        \ruby{忘}{}るるなかれ\ruby{大願}{}を\\
        \ruby{胸}{}に\ruby{秘}{}めし\ruby{涙痕}{}を\\
        \ruby{時}{}は\ruby{人}{}はと\ruby{変}{}われども\\
        \ruby{楡}{}は\ruby{枯}{}れず\ruby{空}{}をさす
    
    \end{minipage}
\end{enumerate} % 番号の箇条書き ここまで
% end lilycs
%%%%% 歌詞 ここまで %%%%%
% end body

\end{document}

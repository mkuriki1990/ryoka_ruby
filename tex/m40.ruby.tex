\documentclass[10pt,b5j]{tarticle} % B6 縦書き
% \documentclass[10pt,b5j]{tarticle} % B6 縦書き
\AtBeginDvi{\special{papersize=128mm,182mm}} % B6 用用紙サイズ
\usepackage{otf} % Unicode で字を入力するのに必要なパッケージ
\usepackage[size=b6j]{bxpapersize} % B6 用紙サイズを指定
\usepackage[dvipdfmx]{graphicx} % 画像を挿入するためのパッケージ
\usepackage[dvipdfmx]{color} % 色をつけるためのパッケージ
\usepackage{pxrubrica} % ルビを振るためのパッケージ
\usepackage{plext} % 漢数字の enumerate を使うためのパッケージ
\usepackage{multicol} % 複数段組を作るためのパッケージ
\setlength{\topmargin}{14mm} % 上下方向のマージン
\addtolength{\topmargin}{-1in} % 
\setlength{\oddsidemargin}{11mm} % 左右方向のマージン
\addtolength{\oddsidemargin}{-1in} % 
\setlength{\textwidth}{154mm} % B6 用
\setlength{\textheight}{108mm} % B6 用
\setlength{\headsep}{0mm} % 
\setlength{\headheight}{0mm} % 
\setlength{\topskip}{0mm} % 
\setlength{\parskip}{0pt} % 
\def\theenumi{\Kanji{enumi}} % 箇条書きのフォーマットを漢数字に変更
\parindent = 0pt % 段落下げしない
\pagestyle{empty} % すべてのページ番号を消去
% \renewcommand{\baselinestretch}{0.9} % 行間の倍率
 % B6 用テンプレート読み込み

\begin{document}
% begin header
%%%%% タイトルと作者 ここから %%%%%
\begin{minipage}[c]{0.7\hsize} % タイトルは上から 7 割
    \begin{center}
    % begin title
        {\LARGE
            一帯ゆるき % タイトルを入れる
        }
        {\small 
            (明治40年寮歌) % 年などを入れる
        }
    % end title
    \end{center}
\end{minipage}
\begin{minipage}[c]{0.3\hsize} % 作歌作曲は上から 3 割
    \begin{flushright} % 下寄せにする
        % begin name
        田中義麿君 作歌\\高松正信君 作曲 % 作歌・作曲者
        % end name
    \end{flushright}
\end{minipage}
%%%%% タイトルと作者 ここまで %%%%%
% (1,2,3 了あり)
% end header

% begin body
\vspace{1.5em} % タイトル, 作者と歌詞の間に隙間を設ける
\newcommand{\linespace}{0.5em} % 行間の設定
\newcommand{\blocksize}{0.5\hsize} % 段組間の設定
%%%%% 歌詞 ここから %%%%%
% begin lilycs
\begin{enumerate} % 番号の箇条書き ここから
    \begin{minipage}[c]{\blocksize}
    
        \vspace{\linespace}
        \item
        % 1.
        \ruby{一帯}{}ゆるき\ruby{石狩}{}の\\
        \ruby{源遠}{}く\ruby{霞罩}{}め\\
        \ruby{五彩}{}を\ruby{染}{}むる\ruby{夕照}{}は\\
        \ruby{手稲}{}の\ruby{夏}{}の\ruby{栄}{}にして\\
        そこに\ruby{無限}{}の\ruby{恩寵}{}あり\\
        \ruby{是吾校}{}の\ruby{在}{}る\ruby{処}{}
        
        \vspace{\linespace}
        \item
        % 2.
        \ruby{胡沙吹}{}く\ruby{風}{}に\ruby{秋闌}{}けて\\
        \ruby{黄葉散}{}りしく\ruby{牧場千里}{}\\
        \ruby{満野}{}の\ruby{吹雪叱咤}{}する\\
        エルムの\ruby{姿壮}{}なれや\\
        そこに\ruby{無限}{}の\ruby{偉力}{}あり\\
        \ruby{是吾寮}{}の\ruby{在}{}る\ruby{処}{}
        
        \vspace{\linespace}
        \item
        % 3.
        \ruby{偲}{}へば\ruby{遠}{}き\ruby{三十年}{}の\\
        \ruby{榛莽}{}あしたの\ruby{日}{}を\ruby{蔽}{}ひ\\
        ゆふべの\ruby{月}{}に\ruby{羆熊吼}{}ゆる\\
        \ruby{北海}{}の\ruby{野}{}に\ruby{鋤入}{}れて\\
        \ruby{偉人}{}が\ruby{植}{}ゑし\ruby{桜花}{}\\
        \ruby{薫}{}は\ruby{高}{}し\ruby{千万古}{}
        
        \vspace{\linespace}
        \item
        % 4.
        \ruby{海}{}を\ruby{距}{}てて\ruby{南}{}の\\
        \ruby{空}{}の\ruby{彼方}{}を\ruby{眺}{}むれば\\
        \ruby{古人}{}の\ruby{道}{}は\ruby{跡}{}もなく\\
        \ruby{文明}{}の\ruby{徳}{}は\ruby{尚成}{}らず\\
        \ruby{溟濛天}{}に\ruby{漲}{}りて\\
        \ruby{帰鳥夕}{}に\ruby{彷徨}{}いぬ
        
        \vspace{\linespace}
        \item
        % 5.
        \ruby{颷々}{}として\ruby{風狂}{}ひ\\
        \ruby{北海}{}の\ruby{潮黒}{}むとき\\
        \ruby{電光凄}{}く\ruby{駛}{}りては\\
        \ruby{鬼啾々}{}の\ruby{声}{}すなり\\
        \ruby{破邪}{}の\ruby{剣}{}を\ruby{右手}{}にして\\
        \ruby{起}{}てるは\ruby{誰}{}ぞや\ruby{吾健児}{}
        
        \vspace{\linespace}
        \item
        % 6.
        \ruby{岩間}{}に\ruby{咽}{}ぶ\ruby{渓流}{}も\\
        \ruby{明日}{}は\ruby{黄河}{}に\ruby{波}{}うたむ\\
        \ruby{蟄竜遂}{}に\ruby{雲}{}を\ruby{呼}{}び\\
        \ruby{鳳雛}{}やがて\ruby{時}{}を\ruby{得}{}て\\
        \ruby{扶揺}{}に\ruby{搏}{}って\ruby{騰}{}りなば\\
        \ruby{魍魎遂}{}に\ruby{影}{}もなし
    
    \end{minipage}
\end{enumerate} % 番号の箇条書き ここまで
% end lilycs
%%%%% 歌詞 ここまで %%%%%
% end body

\end{document}

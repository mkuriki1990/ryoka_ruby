\documentclass[10pt,b5j]{tarticle} % B6 縦書き
% \documentclass[10pt,b5j]{tarticle} % B6 縦書き
\AtBeginDvi{\special{papersize=128mm,182mm}} % B6 用用紙サイズ
\usepackage{otf} % Unicode で字を入力するのに必要なパッケージ
\usepackage[size=b6j]{bxpapersize} % B6 用紙サイズを指定
\usepackage[dvipdfmx]{graphicx} % 画像を挿入するためのパッケージ
\usepackage[dvipdfmx]{color} % 色をつけるためのパッケージ
\usepackage{pxrubrica} % ルビを振るためのパッケージ
\usepackage{multicol} % 複数段組を作るためのパッケージ
\setlength{\topmargin}{14mm} % 上下方向のマージン
\addtolength{\topmargin}{-1in} % 
\setlength{\oddsidemargin}{11mm} % 左右方向のマージン
\addtolength{\oddsidemargin}{-1in} % 
\setlength{\textwidth}{154mm} % B6 用
\setlength{\textheight}{108mm} % B6 用
\setlength{\headsep}{0mm} % 
\setlength{\headheight}{0mm} % 
\setlength{\topskip}{0mm} % 
\setlength{\parskip}{0pt} % 
\def\labelenumi{\theenumi、} % 箇条書きのフォーマット
\parindent = 0pt % 段落下げしない

 % B6 用テンプレート読み込み

\begin{document}
% begin header
%%%%% タイトルと作者 ここから %%%%%
\begin{minipage}[c]{0.7\hsize} % タイトルは上から 7 割
    \begin{center}
    % begin title
        {\LARGE
            生命の旅路 % タイトルを入れる
        }
        {\small 
            (昭和二十年寮歌) % 年などを入れる
        }
    % end title
    \end{center}
\end{minipage}
\begin{minipage}[c]{0.3\hsize} % 作歌作曲は上から 3 割
    \begin{flushright} % 下寄せにする
        % begin name
        幸坂彪君 作歌\\新井忠雄君 作曲 % 作歌・作曲者
        % end name
    \end{flushright}
\end{minipage}
%%%%% タイトルと作者 ここまで %%%%%
% (1,3 了あり)
% end header

% begin body
\vspace{1.5em} % タイトル, 作者と歌詞の間に隙間を設ける
\newcommand{\linespace}{0.5em} % 行間の設定
\newcommand{\blocksize}{0.5\hsize} % 段組間の設定
%%%%% 歌詞 ここから %%%%%
% begin lilycs
\begin{enumerate} % 番号の箇条書き ここから
    \begin{minipage}[c]{\blocksize}
    
        \vspace{\linespace}
        \item
        % 1.
        \ruby{流転永世}{}の\ruby{旅衣}{}\\
        \ruby{四大}{}の\ruby{神秘尋}{}はんにも\\
        \ruby{若}{}き\ruby{生命}{}の\ruby{寂寥}{}に\\
        \ruby{遠}{}き\ruby{真理}{}の\ruby{暁星一}{}つ\\
        \ruby{起伏知}{}らに\ruby{慕}{}ひゆく\\
        \ruby{孤影簫々}{}の\ruby{荒野}{}に\ruby{消}{}えぬ
        
        \vspace{\linespace}
        \item
        % 2.
        \ruby{清}{}き\ruby{友情}{}を\ruby{先人}{}の\\
        \ruby{忍苦染}{}み\ruby{映}{}ゆ\ruby{楡}{}が\ruby{枝}{}に\\
        \ruby{懸}{}けて\ruby{団欒}{}す\ruby{一刻}{}の\\
        \ruby{玻璃}{}が\ruby{盃}{}の\ruby{面茜雲漂蕩}{}ぎ\\
        \ruby{胸琴触}{}れ\ruby{合唱}{}ふうつそみの\\
        \ruby{塵世}{}の\ruby{濁流}{}ひた\ruby{超}{}えて
        
        \vspace{\linespace}
        \item
        % 3.
        \ruby{寮窓辺}{}に\ruby{泣}{}くや\ruby{人性}{}の\\
        \ruby{運命}{}の\ruby{羈絆固}{}ければ\\
        \ruby{愛}{}と\ruby{誠}{}に\ruby{身}{}をせめつ\\
        \ruby{高謳}{}ふ\ruby{哉美}{}し\ruby{青春}{}の\\
        \ruby{剛毅}{}の\ruby{蔭}{}の\ruby{浄涙}{}をば\\
        \ruby{白珠碗}{}に\ruby{掬}{}ばなむ
        
        \vspace{\linespace}
        \item
        % 4.
        \ruby{秋闌}{}く\ruby{原始林}{}のうら\ruby{寂}{}びて\\
        \ruby{愛智}{}の\ruby{微光凄風}{}に\ruby{散}{}り\\
        \ruby{孤独}{}の\ruby{揺籃}{}に\ruby{熟睡}{}する\\
        \ruby{寮友}{}が\ruby{睫}{}に\ruby{恵迪}{}の\\
        \ruby{伝統}{}の\ruby{法燈}{}さゆらぎて\\
        \ruby{栄光}{}に\ruby{帆立}{}つ\ruby{吾寮}{}いま
        
        \vspace{\linespace}
        \item
        % 5.
        \ruby{生命}{}の\ruby{旅路厳粛}{}の\\
        \ruby{啓示}{}に\ruby{喘}{}ぐ\ruby{友垣}{}と\\
        \ruby{若}{}き\ruby{恩恵}{}の\ruby{聖火}{}に\ruby{狂}{}ひ\\
        \ruby{淋}{}しき\ruby{魂}{}を\ruby{睦}{}ぶとき\\
        \ruby{挽歌消}{}え\ruby{行}{}き\ruby{洋々}{}の\\
        \ruby{自由}{}の渚\ruby{濤声}{}とよむ
    
    \end{minipage}
\end{enumerate} % 番号の箇条書き ここまで
% end lilycs
%%%%% 歌詞 ここまで %%%%%
% end body

\end{document}

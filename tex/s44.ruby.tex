\documentclass[10pt,b5j]{tarticle} % B6 縦書き
% \documentclass[10pt,b5j]{tarticle} % B6 縦書き
\AtBeginDvi{\special{papersize=128mm,182mm}} % B6 用用紙サイズ
\usepackage{otf} % Unicode で字を入力するのに必要なパッケージ
\usepackage[size=b6j]{bxpapersize} % B6 用紙サイズを指定
\usepackage[dvipdfmx]{graphicx} % 画像を挿入するためのパッケージ
\usepackage[dvipdfmx]{color} % 色をつけるためのパッケージ
\usepackage{pxrubrica} % ルビを振るためのパッケージ
\usepackage{multicol} % 複数段組を作るためのパッケージ
\setlength{\topmargin}{14mm} % 上下方向のマージン
\addtolength{\topmargin}{-1in} % 
\setlength{\oddsidemargin}{11mm} % 左右方向のマージン
\addtolength{\oddsidemargin}{-1in} % 
\setlength{\textwidth}{154mm} % B6 用
\setlength{\textheight}{108mm} % B6 用
\setlength{\headsep}{0mm} % 
\setlength{\headheight}{0mm} % 
\setlength{\topskip}{0mm} % 
\setlength{\parskip}{0pt} % 
\def\labelenumi{\theenumi、} % 箇条書きのフォーマット
\parindent = 0pt % 段落下げしない

 % B6 用テンプレート読み込み

\begin{document}
% begin header
%%%%% タイトルと作者 ここから %%%%%
\begin{minipage}[c]{0.7\hsize} % タイトルは上から 7 割
    \begin{center}
    % begin title
        {\LARGE
            孤独に満てる % タイトルを入れる
        }
        {\small 
            (昭和四十四年寮歌) % 年などを入れる
        }
    % end title
    \end{center}
\end{minipage}
\begin{minipage}[c]{0.3\hsize} % 作歌作曲は上から 3 割
    \begin{flushright} % 下寄せにする
        % begin name
        山崎芳行君 作歌\\服部泰明君 作曲 % 作歌・作曲者
        % end name
    \end{flushright}
\end{minipage}
%%%%% タイトルと作者 ここまで %%%%%
% (1 了あり)
% end header

% begin length
\vspace{1.5em} % タイトル, 作者と歌詞の間に隙間を設ける
\newcommand{\linespace}{0.5em} % 行間の設定
\newcommand{\blocksize}{0.5\hsize} % 段組間の設定
\newcommand{\itemmargin}{6em} % 曲番の位置調整の長さ
% end length
% begin body
%%%%% 歌詞 ここから %%%%%
\begin{enumerate} % 番号の箇条書き ここから
    \setlength{\itemindent}{\itemmargin} % 曲番の位置調整
    \begin{minipage}[c]{\blocksize}
    
        \vspace{\linespace}
        \item~\\
        % 1.
        \ruby{孤独}{}に\ruby{満}{}てる\ruby{我}{}が\ruby{青春}{}に\\
        \ruby{何時}{}しか\ruby{遅春}{}も\ruby{訪}{}ずれぬ\\
        まだ\ruby{萠}{}えやらぬ\ruby{芝生}{}の\ruby{上}{}に\\
        \ruby{一片舞}{}い\ruby{散}{}る\ruby{桜花}{}\\
        \ruby{朝露}{}に\ruby{濡}{}れ\ruby{新}{}な\ruby{寮友}{}と\\
        \ruby{盃}{}かわす\ruby{楽}{}しさよ\\
        \ruby{嗚呼我一人}{}にあらずして\\
        \ruby{我}{}が\ruby{青春}{}は\ruby{寮友}{}とあり
        
        \vspace{\linespace}
        \item~\\
        % 2.
        \ruby{孤独}{}に\ruby{満}{}てる\ruby{旅人一人}{}\\
        \ruby{理想}{}を\ruby{求}{}めて\ruby{蝦夷}{}へ\ruby{来}{}ぬ\\
        その\ruby{彼}{}の\ruby{人}{}の\ruby{心知}{}れりや\\
        \ruby{原始}{}の\ruby{森}{}に\ruby{鳴}{}く\ruby{郭公}{}\\
        \ruby{寮友}{}と\ruby{別}{}れて\ruby{一月経}{}ちぬ\\
        \ruby{今日懐}{}かしき\ruby{便}{}りあり\\
        \ruby{嗚呼我一人}{}あらずして\\
        \ruby{我}{}が\ruby{青春}{}は\ruby{寮友}{}とあり
        
        \vspace{\linespace}
        \item~\\
        % 3.
        \ruby{孤独}{}に\ruby{満}{}てる\ruby{我}{}が\ruby{自治寮}{}に\\
        \ruby{早}{}くも\ruby{秋}{}の\ruby{気配}{}あり\\
        \ruby{夕陽}{}に\ruby{映}{}ゆるポプラの\ruby{並木}{}\\
        \ruby{憂愁風}{}に\ruby{枯}{}れ\ruby{葉飛}{}ぶ\\
        \ruby{再}{}び\ruby{会}{}いぬ\ruby{寮友}{}と\ruby{連}{}れ\ruby{立}{}ち\\
        \ruby{真理}{}の\ruby{国}{}を\ruby{彷徨}{}いぬ\\
        \ruby{嗚呼我一人}{}あらずして\\
        \ruby{我}{}が\ruby{青春}{}は\ruby{寮友}{}とあり
        
        \vspace{\linespace}
        \item~\\
        % 4.
        \ruby{孤独}{}に\ruby{満}{}てる\ruby{我}{}が\ruby{同胞}{}に\\
        \ruby{厳冬正}{}に\ruby{伸}{}し\ruby{掛}{}り\\
        \ruby{深雪}{}に\ruby{埋}{}むる\ruby{原始}{}の\ruby{森}{}へ\\
        \ruby{月光冴}{}かに\ruby{突差}{}しぬ\\
        \ruby{冷酒}{}を\ruby{飲}{}み\ruby{野心語}{}れば\\
        いとど\ruby{深}{}まる\ruby{友情}{}かな\\
        \ruby{嗚呼我一人}{}あらずして\\
        \ruby{我}{}が\ruby{青春}{}は\ruby{寮友}{}とあり
    
    \end{minipage}
\end{enumerate} % 番号の箇条書き ここまで
%%%%% 歌詞 ここまで %%%%%
% end body

\end{document}

\documentclass[10pt,b5j]{tarticle} % B6 縦書き
% \documentclass[10pt,b5j]{tarticle} % B6 縦書き
\AtBeginDvi{\special{papersize=128mm,182mm}} % B6 用用紙サイズ
\usepackage{otf} % Unicode で字を入力するのに必要なパッケージ
\usepackage[size=b6j]{bxpapersize} % B6 用紙サイズを指定
\usepackage[dvipdfmx]{graphicx} % 画像を挿入するためのパッケージ
\usepackage[dvipdfmx]{color} % 色をつけるためのパッケージ
\usepackage{pxrubrica} % ルビを振るためのパッケージ
\usepackage{multicol} % 複数段組を作るためのパッケージ
\setlength{\topmargin}{14mm} % 上下方向のマージン
\addtolength{\topmargin}{-1in} % 
\setlength{\oddsidemargin}{11mm} % 左右方向のマージン
\addtolength{\oddsidemargin}{-1in} % 
\setlength{\textwidth}{154mm} % B6 用
\setlength{\textheight}{108mm} % B6 用
\setlength{\headsep}{0mm} % 
\setlength{\headheight}{0mm} % 
\setlength{\topskip}{0mm} % 
\setlength{\parskip}{0pt} % 
\def\labelenumi{\theenumi、} % 箇条書きのフォーマット
\parindent = 0pt % 段落下げしない

 % B6 用テンプレート読み込み

\begin{document}
% begin header
%%%%% タイトルと作者 ここから %%%%%
\begin{minipage}[c]{0.7\hsize} % タイトルは上から 7 割
    \begin{center}
    % begin title
        {\LARGE
            樹梢霧海に % タイトルを入れる
        }
        {\small 
            (昭和四十三年寮歌) % 年などを入れる
        }
    % end title
    \end{center}
\end{minipage}
\begin{minipage}[c]{0.3\hsize} % 作歌作曲は上から 3 割
    \begin{flushright} % 下寄せにする
        % begin name
        新橋登君 作歌\\佐藤菊男君 作曲 % 作歌・作曲者
        % end name
    \end{flushright}
\end{minipage}
%%%%% タイトルと作者 ここまで %%%%%
% (1,4,転句 繰り返しなし)
% end header

% begin length
\vspace{1.5em} % タイトル, 作者と歌詞の間に隙間を設ける
\newcommand{\linespace}{0.5em} % 行間の設定
\newcommand{\blocksize}{0.5\hsize} % 段組間の設定
\newcommand{\itemmargin}{3em} % 曲番の位置調整の長さ
% end length
% begin body
%%%%% 歌詞 ここから %%%%%
\begin{enumerate} % 番号の箇条書き ここから
    \setlength{\itemindent}{\itemmargin} % 曲番の位置調整
    \begin{minipage}[c]{\blocksize}
    
        \vspace{\linespace}
        \item~\\
        % 1.
        \ruby{樹}{き}\ruby{梢}{こずえ}\ruby{霧}{きり}\ruby{海}{うみ}に\ruby{消}{き}え\ruby{入}{い}りて\\
        \ruby{北}{きた}\ruby{溟牙}{}\ruby{城}{じょう}の\ruby{夏}{なつ}の\ruby{宵}{よい}\\
        \ruby{難攻不落}{なんこうふらく}を\ruby{誇}{ほこ}りしも\\
        \ruby{時}{とき}\ruby{凋衰}{}の\ruby{風}{かぜ}\ruby{強}{つよ}し
        
    \end{minipage}
    \begin{minipage}[c]{\blocksize}
        
        \vspace{\linespace}
        \item~\\
        % 2.
        \ruby{伝統}{でんとう}の\ruby{石}{いし}に\ruby{佇}{たたず}みて\\
        \ruby{古昔}{こせき}の\ruby{意気}{いき}に\ruby{涙}{なみだ}する\\
        \ruby{秋}{あき}の\ruby{今宵}{こよい}の\ruby{宴}{うたげ}にも\\
        \ruby{貧}{ひん}\ruby{交行}{}の\ruby{風}{かぜ}\ruby{寒}{さむ}し
        
    \end{minipage}
    \begin{minipage}[c]{\blocksize}
        
        \vspace{\linespace}
        \item~\\
        % \ruby{転}{てん}\ruby{句}{く}.
        \ruby{楡}{にれ}\ruby{陵}{りょう}の\ruby{二}{に}\ruby{春}{はる}に\ruby{宿}{やど}せる\ruby{白露}{しらつゆ}の\\
        \ruby{生命}{せいめい}\ruby{短命}{たんめい}にして\ruby{吉}{きち}しとする\\
        さにあらば\ruby{吾}{われ}\ruby{等}{とう}が\ruby{友}{とも}よ\\
        \ruby{久遠}{くおん}なる\ruby{星}{ほし}に\\
        \ruby{崇}{たかし}\ruby{厳}{げん}に\ruby{大志}{たいし}を\ruby{告}{つ}げるべく\\
        \ruby{今}{こん}\ruby{高}{たか}らかに\ruby{誓}{ちか}いけん
        
    \end{minipage}
    \begin{minipage}[c]{\blocksize}
        
        \vspace{\linespace}
        \item~\\
        % 3.
        \ruby{白}{しろ}\ruby{雪}{ゆき}\ruby{深}{ふか}き\ruby{北国}{きたぐに}に\\
        \ruby{迪}{すすむ}をたずねる\ruby{旅人}{たびびと}よ\\
        \ruby{朔風}{さくふう}\ruby{如何}{いかが}に\ruby{荒}{あら}\ruby{吹}{}とも\\
        \ruby{真理}{しんり}の\ruby{郷}{さと}は\ruby{遠}{とお}からじ
        
    \end{minipage}
    \begin{minipage}[c]{\blocksize}
        
        \vspace{\linespace}
        \item~\\
        % 4.
        いざ\ruby{寮}{りょう}\ruby{友}{とも}ようたわなん\\
        あすの\ruby{生命}{せいめい}を\ruby{闘}{たたか}うと\\
        \ruby{万}{まん}\ruby{花}{はな}\ruby{乱}{らん}るる\ruby{春}{はる}の\ruby{日}{ひ}に\\
        \ruby{高}{こう}\ruby{遠}{とお}き\ruby{大望}{たいぼう}を\ruby{目}{}\ruby{指}{めざ}さんや
    
    \end{minipage}
\end{enumerate} % 番号の箇条書き ここまで
%%%%% 歌詞 ここまで %%%%%
% end body

\end{document}

\documentclass[10pt,b5j]{tarticle} % B6 縦書き
% \documentclass[10pt,b5j]{tarticle} % B6 縦書き
\AtBeginDvi{\special{papersize=128mm,182mm}} % B6 用用紙サイズ
\usepackage{otf} % Unicode で字を入力するのに必要なパッケージ
\usepackage[size=b6j]{bxpapersize} % B6 用紙サイズを指定
\usepackage[dvipdfmx]{graphicx} % 画像を挿入するためのパッケージ
\usepackage[dvipdfmx]{color} % 色をつけるためのパッケージ
\usepackage{pxrubrica} % ルビを振るためのパッケージ
\usepackage{multicol} % 複数段組を作るためのパッケージ
\setlength{\topmargin}{14mm} % 上下方向のマージン
\addtolength{\topmargin}{-1in} % 
\setlength{\oddsidemargin}{11mm} % 左右方向のマージン
\addtolength{\oddsidemargin}{-1in} % 
\setlength{\textwidth}{154mm} % B6 用
\setlength{\textheight}{108mm} % B6 用
\setlength{\headsep}{0mm} % 
\setlength{\headheight}{0mm} % 
\setlength{\topskip}{0mm} % 
\setlength{\parskip}{0pt} % 
\def\labelenumi{\theenumi、} % 箇条書きのフォーマット
\parindent = 0pt % 段落下げしない

 % B6 用テンプレート読み込み

\begin{document}
% begin header
%%%%% タイトルと作者 ここから %%%%%
\begin{minipage}[c]{0.7\hsize} % タイトルは上から 7 割
    \begin{center}
    % begin title
        {\LARGE
            樹梢霧海に % タイトルを入れる
        }
        {\small 
            (昭和四十三年寮歌) % 年などを入れる
        }
    % end title
    \end{center}
\end{minipage}
\begin{minipage}[c]{0.3\hsize} % 作歌作曲は上から 3 割
    \begin{flushright} % 下寄せにする
        % begin name
        新橋登君 作歌\\佐藤菊男君 作曲 % 作歌・作曲者
        % end name
    \end{flushright}
\end{minipage}
%%%%% タイトルと作者 ここまで %%%%%
% (1,4,転句 繰り返しなし)
% end header

% begin body
\vspace{1.5em} % タイトル, 作者と歌詞の間に隙間を設ける
\newcommand{\linespace}{0.5em} % 行間の設定
\newcommand{\blocksize}{0.5\hsize} % 段組間の設定
%%%%% 歌詞 ここから %%%%%
% begin lilycs
\begin{enumerate} % 番号の箇条書き ここから
    \begin{minipage}[c]{\blocksize}
    
        \vspace{\linespace}
        \item
        % 1.
        \ruby{樹梢霧海}{}に\ruby{消}{}え\ruby{入}{}りて\\
        \ruby{北溟牙城}{}の\ruby{夏}{}の\ruby{宵}{}\\
        \ruby{難攻不落}{}を\ruby{誇}{}りしも\\
        \ruby{時凋衰}{}の\ruby{風強}{}し
        
        \vspace{\linespace}
        \item
        % 2.
        \ruby{伝統}{}の\ruby{石}{}に\ruby{佇}{}みて\\
        \ruby{古昔}{}の\ruby{意気}{}に\ruby{涙}{}する\\
        \ruby{秋}{}の\ruby{今宵}{}の\ruby{宴}{}にも\\
        \ruby{貧交行}{}の\ruby{風寒}{}し
        
        \vspace{\linespace}
        \item
        % \ruby{転句}{}.
        \ruby{楡陵}{}の\ruby{二春}{}に\ruby{宿}{}せる\ruby{白露}{}の\\
        \ruby{生命短命}{}にして\ruby{吉}{}しとする\\
        さにあらば\ruby{吾等}{}が\ruby{友}{}よ\\
        \ruby{久遠}{}なる\ruby{星}{}に\\
        \ruby{崇厳}{}に\ruby{大志}{}を\ruby{告}{}げるべく\\
        \ruby{今高}{}らかに\ruby{誓}{}いけん
        
        \vspace{\linespace}
        \item
        % 3.
        \ruby{白雪深}{}き\ruby{北国}{}に\\
        \ruby{迪}{}をたずねる\ruby{旅人}{}よ\\
        \ruby{朔風如何}{}に\ruby{荒吹}{}とも\\
        \ruby{真理}{}の\ruby{郷}{}は\ruby{遠}{}からじ
        
        \vspace{\linespace}
        \item
        % 4.
        いざ\ruby{寮友}{}ようたわなん\\
        あすの\ruby{生命}{}を\ruby{闘}{}うと\\
        \ruby{万花乱}{}るる\ruby{春}{}の\ruby{日}{}に\\
        \ruby{高遠}{}き\ruby{大望}{}を\ruby{目指}{}さんや
    
    \end{minipage}
\end{enumerate} % 番号の箇条書き ここまで
% end lilycs
%%%%% 歌詞 ここまで %%%%%
% end body

\end{document}

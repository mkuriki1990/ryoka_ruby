\documentclass[10pt,b5j]{tarticle} % B6 縦書き
% \documentclass[10pt,b5j]{tarticle} % B6 縦書き
\AtBeginDvi{\special{papersize=128mm,182mm}} % B6 用用紙サイズ
\usepackage{otf} % Unicode で字を入力するのに必要なパッケージ
\usepackage[size=b6j]{bxpapersize} % B6 用紙サイズを指定
\usepackage[dvipdfmx]{graphicx} % 画像を挿入するためのパッケージ
\usepackage[dvipdfmx]{color} % 色をつけるためのパッケージ
\usepackage{pxrubrica} % ルビを振るためのパッケージ
\usepackage{plext} % 漢数字の enumerate を使うためのパッケージ
\usepackage{multicol} % 複数段組を作るためのパッケージ
\setlength{\topmargin}{14mm} % 上下方向のマージン
\addtolength{\topmargin}{-1in} % 
\setlength{\oddsidemargin}{11mm} % 左右方向のマージン
\addtolength{\oddsidemargin}{-1in} % 
\setlength{\textwidth}{154mm} % B6 用
\setlength{\textheight}{108mm} % B6 用
\setlength{\headsep}{0mm} % 
\setlength{\headheight}{0mm} % 
\setlength{\topskip}{0mm} % 
\setlength{\parskip}{0pt} % 
\def\theenumi{\Kanji{enumi}} % 箇条書きのフォーマットを漢数字に変更
\parindent = 0pt % 段落下げしない
\pagestyle{empty} % すべてのページ番号を消去
% \renewcommand{\baselinestretch}{0.9} % 行間の倍率
 % B6 用テンプレート読み込み

\begin{document}
% begin header
%%%%% タイトルと作者 ここから %%%%%
\begin{minipage}[c]{0.7\hsize} % タイトルは上から 7 割
    \begin{center}
    % begin title
        {\LARGE
            朔北に % タイトルを入れる
        }
        {\small 
            (昭和46年寮歌) % 年などを入れる
        }
    % end title
    \end{center}
\end{minipage}
\begin{minipage}[c]{0.3\hsize} % 作歌作曲は上から 3 割
    \begin{flushright} % 下寄せにする
        % begin name
        伊藤正朗君 作歌・作曲 % 作歌・作曲者
        % end name
    \end{flushright}
\end{minipage}
%%%%% タイトルと作者 ここまで %%%%%
% (1,2,3 繰り返しなし)
% end header

% begin body
\vspace{1.5em} % タイトル, 作者と歌詞の間に隙間を設ける
\newcommand{\linespace}{0.5em} % 行間の設定
\newcommand{\blocksize}{0.5\hsize} % 段組間の設定
%%%%% 歌詞 ここから %%%%%
% begin lilycs
\begin{enumerate} % 番号の箇条書き ここから
    \begin{minipage}[c]{\blocksize}
    
        \vspace{\linespace}
        \item
        % 1.
        \ruby{朔北}{}に\ruby{手稲颪}{}の\ruby{咆哮絶}{}えて\\
        \ruby{静寂}{}に\ruby{痛}{}し\ruby{遠汽笛}{}\\
        \ruby{凍}{}てつく\ruby{雪原}{}に\ruby{寒月}{}の\\
        \ruby{蒼}{}き\ruby{光}{}の\ruby{射}{}しそえば\\
        \ruby{聳天樹}{}の\ruby{影}{}は\ruby{猛}{}くして\\
        \ruby{虚空指}{}す\ruby{彼方宿}{}り\ruby{舎}{}の\\
        \ruby{灯}{}は\ruby{今宵}{}また\ruby{旅人}{}の\\
        \ruby{継}{}ぎ\ruby{培}{}いし\ruby{迪}{}を\ruby{諭}{}せり
        
        \vspace{\linespace}
        \item
        % 2.
        \ruby{朝焼}{}けて\ruby{南}{}に\ruby{風}{}の\ruby{起}{}つ\ruby{聞}{}かば\\
        \ruby{北}{}の\ruby{都}{}に\ruby{春近}{}く\\
        \ruby{雪融}{}け\ruby{水}{}の\ruby{溢}{}れては\\
        \ruby{豊水}{}の\ruby{岸塵高}{}し\\
        \ruby{黄}{}ばむ\ruby{空}{}ゆく\ruby{鳥}{}も\ruby{鳴}{}く\\
        \ruby{土}{}の\ruby{香}{}ぞする\ruby{野幌路}{}を\\
        \ruby{孤}{}りそぞろに\ruby{辿}{}る\ruby{日}{}は\\
        \ruby{異郷}{}の\ruby{旅}{}を\ruby{思}{}い\ruby{侘}{}ぶかな
        
        \vspace{\linespace}
        \item
        % 3.
        はろばろと\ruby{続}{}く\ruby{沃野}{}の\ruby{玉葱畠}{}\\
        \ruby{金}{}に\ruby{輝}{}く\ruby{北指}{}して\\
        \ruby{延}{}びる\ruby{鉄路}{}の\ruby{傍}{}に\\
        かの\ruby{石狩}{}の\ruby{文学碑}{}\\
        \ruby{濁}{}れる\ruby{川}{}に\ruby{臨}{}みては\\
        \ruby{沈}{}む\ruby{夏陽}{}に\ruby{涙}{}する\\
        \ruby{回顧百年忘}{}れずや\\
        この\ruby{地拓}{}きし\ruby{先人}{}の\ruby{夢}{}
    
    \end{minipage}
\end{enumerate} % 番号の箇条書き ここまで
% end lilycs
%%%%% 歌詞 ここまで %%%%%
% end body

\end{document}

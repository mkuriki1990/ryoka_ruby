\documentclass[10pt,b5j]{tarticle} % B6 縦書き
% \documentclass[10pt,b5j]{tarticle} % B6 縦書き
\AtBeginDvi{\special{papersize=128mm,182mm}} % B6 用用紙サイズ
\usepackage{otf} % Unicode で字を入力するのに必要なパッケージ
\usepackage[size=b6j]{bxpapersize} % B6 用紙サイズを指定
\usepackage[dvipdfmx]{graphicx} % 画像を挿入するためのパッケージ
\usepackage[dvipdfmx]{color} % 色をつけるためのパッケージ
\usepackage{pxrubrica} % ルビを振るためのパッケージ
\usepackage{plext} % 漢数字の enumerate を使うためのパッケージ
\usepackage{multicol} % 複数段組を作るためのパッケージ
\setlength{\topmargin}{14mm} % 上下方向のマージン
\addtolength{\topmargin}{-1in} % 
\setlength{\oddsidemargin}{11mm} % 左右方向のマージン
\addtolength{\oddsidemargin}{-1in} % 
\setlength{\textwidth}{154mm} % B6 用
\setlength{\textheight}{108mm} % B6 用
\setlength{\headsep}{0mm} % 
\setlength{\headheight}{0mm} % 
\setlength{\topskip}{0mm} % 
\setlength{\parskip}{0pt} % 
\def\theenumi{\Kanji{enumi}} % 箇条書きのフォーマットを漢数字に変更
\parindent = 0pt % 段落下げしない
\pagestyle{empty} % すべてのページ番号を消去
% \renewcommand{\baselinestretch}{0.9} % 行間の倍率
 % B6 用テンプレート読み込み

\begin{document}
% begin header
%%%%% タイトルと作者 ここから %%%%%
\begin{minipage}[c]{0.7\hsize} % タイトルは上から 7 割
    \begin{center}
    % begin title
        {\LARGE
            朔北に % タイトルを入れる
        }
        {\small 
            (昭和四十六年寮歌) % 年などを入れる
        }
    % end title
    \end{center}
\end{minipage}
\begin{minipage}[c]{0.3\hsize} % 作歌作曲は上から 3 割
    \begin{flushright} % 下寄せにする
        % begin name
        伊藤正朗君 作歌・作曲 % 作歌・作曲者
        % end name
    \end{flushright}
\end{minipage}
%%%%% タイトルと作者 ここまで %%%%%
% (1,2,3 繰り返しなし)
% end header

% begin length
\vspace{1.5em} % タイトル, 作者と歌詞の間に隙間を設ける
\newcommand{\linespace}{0.5em} % 行間の設定
\newcommand{\blocksize}{0.5\hsize} % 段組間の設定
\newcommand{\itemmargin}{3em} % 曲番の位置調整の長さ
% end length
% begin body
%%%%% 歌詞 ここから %%%%%
\begin{enumerate} % 番号の箇条書き ここから
    \setlength{\itemindent}{\itemmargin} % 曲番の位置調整
    \begin{minipage}[c]{\blocksize}
    
        \vspace{\linespace}
        \item~\\
        % 1.
        \ruby{朔北}{さくほく}に\ruby{手稲}{ていね}\ruby{颪}{おろし}の\ruby{咆哮}{ほうこう}\ruby{絶}{た}えて\\
        \ruby{静寂}{せいじゃく}に\ruby{痛}{いた}し\ruby{遠}{とお}\ruby{汽笛}{きてき}\\
        \ruby{凍}{い}てつく\ruby{雪原}{せつげん}に\ruby{寒月}{かんげつ}の\\
        \ruby{蒼}{あお}き\ruby{光}{ひかり}の\ruby{射}{さ}しそえば\\
        \ruby{聳天}{}\ruby{樹}{じゅ}の\ruby{影}{かげ}は\ruby{猛}{もう}くして\\
        \ruby{虚空}{こくう}\ruby{指}{さ}す\ruby{彼方}{かなた}\ruby{宿}{やど}り\ruby{舎}{しゃ}の\\
        \ruby{灯}{あかり}は\ruby{今宵}{こよい}また\ruby{旅}{またたび}\ruby{人}{じん}の\\
        \ruby{継}{つ}ぎ\ruby{培}{つちか}いし\ruby{迪}{すすむ}を\ruby{諭}{さと}せり
        
    \end{minipage}
    \begin{minipage}[c]{\blocksize}
        
        \vspace{\linespace}
        \item~\\
        % 2.
        \ruby{朝}{あさ}\ruby{焼}{や}けて\ruby{南}{みなみ}に\ruby{風}{かぜ}の\ruby{起}{た}つ\ruby{聞}{き}かば\\
        \ruby{北}{きた}の\ruby{都}{と}に\ruby{春}{はる}\ruby{近}{ちか}く\\
        \ruby{雪}{ゆき}\ruby{融}{と}け\ruby{水}{すい}の\ruby{溢}{あふ}れては\\
        \ruby{豊水}{ほうすい}の\ruby{岸}{きし}\ruby{塵}{ちり}\ruby{高}{たか}し\\
        \ruby{黄}{き}ばむ\ruby{空}{そら}ゆく\ruby{鳥}{とり}も\ruby{鳴}{な}く\\
        \ruby{土}{ど}の\ruby{香}{こう}ぞする\ruby{野幌}{のっぽろ}\ruby{路}{ろ}を\\
        \ruby{孤}{こ}りそぞろに\ruby{辿}{たど}る\ruby{日}{ひ}は\\
        \ruby{異郷}{いきょう}の\ruby{旅}{たび}を\ruby{思}{おも}い\ruby{侘}{わび}ぶかな
        
    \end{minipage}
    \begin{minipage}[c]{\blocksize}
        
        \vspace{\linespace}
        \item~\\
        % 3.
        はろばろと\ruby{続}{つづ}く\ruby{沃野}{よくや}の\ruby{玉葱}{たまねぎ}\ruby{畠}{はたけ}\\
        \ruby{金}{きん}に\ruby{輝}{かがや}く\ruby{北}{きた}\ruby{指}{さ}して\\
        \ruby{延}{の}びる\ruby{鉄路}{てつろ}の\ruby{傍}{はた}に\\
        かの\ruby{石狩}{いしかり}の\ruby{文学}{ぶんがく}\ruby{碑}{ひ}\\
        \ruby{濁}{にご}れる\ruby{川}{かわ}に\ruby{臨}{のぞ}みては\\
        \ruby{沈}{しず}む\ruby{夏陽}{なつひ}に\ruby{涙}{なみだ}する\\
        \ruby{回顧}{かいこ}\ruby{百}{ひゃく}\ruby{年}{ねん}\ruby{忘}{わす}れずや\\
        この\ruby{地}{ち}\ruby{拓}{ひら}きし\ruby{先人}{せんじん}の\ruby{夢}{ゆめ}
    
    \end{minipage}
\end{enumerate} % 番号の箇条書き ここまで
%%%%% 歌詞 ここまで %%%%%
% end body

\end{document}

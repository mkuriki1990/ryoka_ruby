\documentclass[10pt,b5j]{tarticle} % B6 縦書き
% \documentclass[10pt,b5j]{tarticle} % B6 縦書き
\AtBeginDvi{\special{papersize=128mm,182mm}} % B6 用用紙サイズ
\usepackage{otf} % Unicode で字を入力するのに必要なパッケージ
\usepackage[size=b6j]{bxpapersize} % B6 用紙サイズを指定
\usepackage[dvipdfmx]{graphicx} % 画像を挿入するためのパッケージ
\usepackage[dvipdfmx]{color} % 色をつけるためのパッケージ
\usepackage{pxrubrica} % ルビを振るためのパッケージ
\usepackage{multicol} % 複数段組を作るためのパッケージ
\setlength{\topmargin}{14mm} % 上下方向のマージン
\addtolength{\topmargin}{-1in} % 
\setlength{\oddsidemargin}{11mm} % 左右方向のマージン
\addtolength{\oddsidemargin}{-1in} % 
\setlength{\textwidth}{154mm} % B6 用
\setlength{\textheight}{108mm} % B6 用
\setlength{\headsep}{0mm} % 
\setlength{\headheight}{0mm} % 
\setlength{\topskip}{0mm} % 
\setlength{\parskip}{0pt} % 
\def\labelenumi{\theenumi、} % 箇条書きのフォーマット
\parindent = 0pt % 段落下げしない

 % B6 用テンプレート読み込み

\begin{document}
% begin header
%%%%% タイトルと作者 ここから %%%%%
\begin{minipage}[c]{0.7\hsize} % タイトルは上から 7 割
    \begin{center}
    % begin title
        {\LARGE
            津軽の海 % タイトルを入れる
        }
        {\small 
            (昭和9年寮歌) % 年などを入れる
        }
    % end title
    \end{center}
\end{minipage}
\begin{minipage}[c]{0.3\hsize} % 作歌作曲は上から 3 割
    \begin{flushright} % 下寄せにする
        % begin name
        星勇君 作歌\\白石祐義君 作曲 % 作歌・作曲者
        % end name
    \end{flushright}
\end{minipage}
%%%%% タイトルと作者 ここまで %%%%%
% (1,4,7 了あり)
% end header

% begin body
\vspace{1.5em} % タイトル, 作者と歌詞の間に隙間を設ける
\newcommand{\linespace}{0.5em} % 行間の設定
\newcommand{\blocksize}{0.5\hsize} % 段組間の設定
%%%%% 歌詞 ここから %%%%%
% begin lilycs
\begin{enumerate} % 番号の箇条書き ここから
    \begin{minipage}[c]{\blocksize}
    
        \vspace{\linespace}
        \item
        % 1.
        \ruby{津軽}{}の\ruby{海渦巻}{}ける\ruby{奥}{}\\
        オホツクの\ruby{寒潮咆哮}{}えて\\
        \ruby{雄健}{}き\ruby{名}{}ぞ\ruby{蝦夷}{}が\ruby{島根}{}に\\
        \ruby{年古}{}りし\ruby{恵迪}{}の\ruby{寮}{}\\
        \ruby{旅寝}{}とな\ruby{言}{}ひし\ruby{三年}{}を\\
        \ruby{揺籃}{}の\ruby{高夢}{}を\ruby{追}{}うなり
        
        \vspace{\linespace}
        \item
        % 2.
        \ruby{寂寥}{}の\ruby{歩行}{}はこびて\\
        \ruby{茂}{}みさぶる\ruby{森}{}に\ruby{仰臥}{}し\\
        \ruby{先人}{}の\ruby{詩}{}になぞらへ\\
        \ruby{陳腐}{}なる\ruby{歌}{}を\ruby{恥}{}ぢらふ\\
        ただ\ruby{仰}{}げ\ruby{自然}{}の\ruby{姿}{}\\
        そは\ruby{深}{}き\ruby{黙示}{}をきざむ
        
        \vspace{\linespace}
        \item
        % 3.
        \ruby{清明}{}の\ruby{水}{}に\ruby{浮}{}べる\\
        \ruby{宵月}{}の\ruby{影}{}はさやけし\\
        \ruby{酒觴}{}をめぐらしかさね\\
        \ruby{羆熊}{}の\ruby{声聞}{}くもしべなし\\
        たぎりゆく\ruby{若}{}き\ruby{血潮}{}に\\
        \ruby{限}{}りなき\ruby{感激}{}をしたふ
        
        \vspace{\linespace}
        \item
        % 4.
        \ruby{六十}{}にも\ruby{齢}{}うつろひ\\
        \ruby{集}{}ひたる\ruby{寮友}{}は\ruby{兄弟}{}\\
        \ruby{伝統}{}の\ruby{永遠}{}の\ruby{記念}{}と\\
        \ruby{感激}{}の\ruby{寮史}{}も\ruby{成}{}りぬ\\
        \ruby{情懐深}{}く\ruby{唯魂}{}が\\
        \ruby{魂}{}と\ruby{結}{}び\ruby{輝}{}く
        
        \vspace{\linespace}
        \item
        % 5.
        \ruby{恵迪}{}の\ruby{館}{}を\ruby{訪}{}ひし\\
        \ruby{竜田姫佐保神三}{}たび\\
        \ruby{若人}{}の\ruby{生命捧}{}げし\\
        \ruby{想}{}ひ\ruby{出}{}の\ruby{自由}{}の\ruby{宴遊}{}\\
        \ruby{永劫}{}に\ruby{若}{}き\ruby{一日}{}の\\
        \ruby{夢}{}とせむ\ruby{楡鐘}{}の\ruby{調}{}べを
        
        \vspace{\linespace}
        \item
        % 6.
        \ruby{黎明}{}は\ruby{曠野}{}の\ruby{際涯}{}\\
        \ruby{雄叫}{}びと\ruby{共}{}に\ruby{来}{}れり\\
        \ruby{満蒙}{}の\ruby{長夜}{}の\ruby{闇}{}も\\
        \ruby{晴}{}れんとす\ruby{起}{}てよ\ruby{寮友}{}\\
        \ruby{青春}{}の\ruby{象牙}{}の\ruby{塔}{}を\\
        いざ\ruby{出}{}でむ\ruby{時}{}は\ruby{到}{}れり
        
        \vspace{\linespace}
        \item
        % 7.
        \ruby{北溟}{}の\ruby{自治}{}の\ruby{牙城}{}を\\
        \ruby{蒼穹高}{}く\ruby{巣立}{}つ\ruby{寮友}{}\\
        \ruby{澆季}{}の\ruby{世救}{}はんは\ruby{汝}{}れ\\
        \ruby{済世}{}の\ruby{烽火}{}あぐべし\\
        \ruby{忘}{}れえぬ\ruby{恵迪}{}の\ruby{歌}{}\\
        \ruby{高唱}{}ひゆけ\ruby{正義}{}の\ruby{大道}{}を
    
    \end{minipage}
\end{enumerate} % 番号の箇条書き ここまで
% end lilycs
%%%%% 歌詞 ここまで %%%%%
% end body

\end{document}

\documentclass[10pt,b5j]{tarticle} % B6 縦書き
% \documentclass[10pt,b5j]{tarticle} % B6 縦書き
\AtBeginDvi{\special{papersize=128mm,182mm}} % B6 用用紙サイズ
\usepackage{otf} % Unicode で字を入力するのに必要なパッケージ
\usepackage[size=b6j]{bxpapersize} % B6 用紙サイズを指定
\usepackage[dvipdfmx]{graphicx} % 画像を挿入するためのパッケージ
\usepackage[dvipdfmx]{color} % 色をつけるためのパッケージ
\usepackage{pxrubrica} % ルビを振るためのパッケージ
\usepackage{plext} % 漢数字の enumerate を使うためのパッケージ
\usepackage{multicol} % 複数段組を作るためのパッケージ
\setlength{\topmargin}{14mm} % 上下方向のマージン
\addtolength{\topmargin}{-1in} % 
\setlength{\oddsidemargin}{11mm} % 左右方向のマージン
\addtolength{\oddsidemargin}{-1in} % 
\setlength{\textwidth}{154mm} % B6 用
\setlength{\textheight}{108mm} % B6 用
\setlength{\headsep}{0mm} % 
\setlength{\headheight}{0mm} % 
\setlength{\topskip}{0mm} % 
\setlength{\parskip}{0pt} % 
\def\theenumi{\Kanji{enumi}} % 箇条書きのフォーマットを漢数字に変更
\parindent = 0pt % 段落下げしない
\pagestyle{empty} % すべてのページ番号を消去
% \renewcommand{\baselinestretch}{0.9} % 行間の倍率
 % B6 用テンプレート読み込み

\begin{document}
% begin header
%%%%% タイトルと作者 ここから %%%%%
\begin{minipage}[c]{0.7\hsize} % タイトルは上から 7 割
    \begin{center}
    % begin title
        {\LARGE
            生命の争闘 % タイトルを入れる
        }
        {\small 
            (大正十年寮歌) % 年などを入れる
        }
    % end title
    \end{center}
\end{minipage}
\begin{minipage}[c]{0.3\hsize} % 作歌作曲は上から 3 割
    \begin{flushright} % 下寄せにする
        % begin name
        青野正男君 作歌\\小峰三千男君 作曲 % 作歌・作曲者
        % end name
    \end{flushright}
\end{minipage}
%%%%% タイトルと作者 ここまで %%%%%
% (1,6 了あり)
% end header

% begin length
\vspace{1.5em} % タイトル, 作者と歌詞の間に隙間を設ける
\newcommand{\linespace}{0.5em} % 行間の設定
\newcommand{\blocksize}{0.5\hsize} % 段組間の設定
\newcommand{\itemmargin}{6em} % 曲番の位置調整の長さ
% end length
% begin body
%%%%% 歌詞 ここから %%%%%
\begin{enumerate} % 番号の箇条書き ここから
    \setlength{\itemindent}{\itemmargin} % 曲番の位置調整
    \begin{minipage}[c]{\blocksize}
    
        \vspace{\linespace}
        \item~\\
        % 1.
        \ruby{生命}{}の\ruby{争闘敗}{}れじと\\
        \ruby{雪解}{}の\ruby{野辺}{}に\ruby{萠}{}え\ruby{出}{}でし\\
        \ruby{浅緑}{}なる\ruby{若草}{}の\\
        \ruby{伸展}{}ゆく\ruby{生命思}{}ふとき\\
        \ruby{若}{}き\ruby{力}{}のよろこびは\\
        \ruby{我等}{}が\ruby{胸}{}に\ruby{溢}{}るなり
        
        \vspace{\linespace}
        \item~\\
        % 2.
        \ruby{悲哀誘}{}ふ\ruby{郭公}{}の\\
        \ruby{声}{}を\ruby{聞}{}きつつ\ruby{逍遙}{}へば\\
        \ruby{今}{}は\ruby{小暗}{}き\ruby{木下闇}{}\\
        \ruby{黒百合咲}{}けど\ruby{春}{}いづこ\\
        うつろひやすき\ruby{若}{}き\ruby{日}{}を\\
        \ruby{盧生}{}の\ruby{夢}{}となすなかれ
        
        \vspace{\linespace}
        \item~\\
        % 3.
        \ruby{牧場}{}に\ruby{虫}{}の\ruby{音}{}も\ruby{淡}{}く\\
        \ruby{仰}{}げば\ruby{高}{}し\ruby{秋}{}の\ruby{空}{}\\
        \ruby{肥馬原頭}{}に\ruby{嘶}{}きて\\
        \ruby{雄渾}{}の\ruby{気}{}はあふれつつ\\
        \ruby{崇}{}き\ruby{理想}{}を\ruby{胸}{}にして\\
        \ruby{生}{}くる\ruby{喜悦謳}{}ふ\ruby{哉}{}
        
        \vspace{\linespace}
        \item~\\
        % 4.
        \ruby{眺}{}めはてなき\ruby{石狩}{}の\\
        \ruby{曠野}{}に\ruby{凋落}{}の\ruby{秋更}{}けて\\
        \ruby{寂}{}しく\ruby{暮}{}るる\ruby{手稲山}{}\\
        \ruby{今}{}うすれゆく\ruby{赤陽}{}に\\
        \ruby{想}{}ひぞ\ruby{馳}{}する\ruby{北欧州}{}\\
        \ruby{戦}{}禍の\ruby{跡}{}の\ruby{夕}{}まぐれ
        
        \vspace{\linespace}
        \item~\\
        % 5.
        \ruby{夕吹}{}く\ruby{風膚}{}にしみ\\
        \ruby{音}{}も\ruby{淋}{}しく\ruby{行}{}く\ruby{橇}{}の\\
        \ruby{大雪原}{}に\ruby{消}{}ゆるとき\\
        \ruby{寒月高}{}く\ruby{冴}{}ゆる\ruby{夜半}{}\\
        \ruby{哀愁}{}をこむる\ruby{若人}{}の\\
        \ruby{瞑想}{}ぞ\ruby{如何}{}に\ruby{深}{}からん
        
        \vspace{\linespace}
        \item~\\
        % 6.
        \ruby{嗚呼北州}{}の\ruby{春秋}{}に\\
        \ruby{自然}{}の\ruby{教訓学}{}びつつ\\
        \ruby{尚}{}き\ruby{生命}{}に\ruby{生}{}きなんと\\
        \ruby{精神}{}を\ruby{磨}{}く\ruby{友}{}どちよ\\
        \ruby{先人建}{}てし\ruby{自治寮}{}の\\
        \ruby{貴}{}き\ruby{歴史伝}{}へかし
    
    \end{minipage}
\end{enumerate} % 番号の箇条書き ここまで
%%%%% 歌詞 ここまで %%%%%
% end body

\end{document}

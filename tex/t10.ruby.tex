\documentclass[10pt,b5j]{tarticle} % B6 縦書き
% \documentclass[10pt,b5j]{tarticle} % B6 縦書き
\AtBeginDvi{\special{papersize=128mm,182mm}} % B6 用用紙サイズ
\usepackage{otf} % Unicode で字を入力するのに必要なパッケージ
\usepackage[size=b6j]{bxpapersize} % B6 用紙サイズを指定
\usepackage[dvipdfmx]{graphicx} % 画像を挿入するためのパッケージ
\usepackage[dvipdfmx]{color} % 色をつけるためのパッケージ
\usepackage{pxrubrica} % ルビを振るためのパッケージ
\usepackage{multicol} % 複数段組を作るためのパッケージ
\setlength{\topmargin}{14mm} % 上下方向のマージン
\addtolength{\topmargin}{-1in} % 
\setlength{\oddsidemargin}{11mm} % 左右方向のマージン
\addtolength{\oddsidemargin}{-1in} % 
\setlength{\textwidth}{154mm} % B6 用
\setlength{\textheight}{108mm} % B6 用
\setlength{\headsep}{0mm} % 
\setlength{\headheight}{0mm} % 
\setlength{\topskip}{0mm} % 
\setlength{\parskip}{0pt} % 
\def\labelenumi{\theenumi、} % 箇条書きのフォーマット
\parindent = 0pt % 段落下げしない

 % B6 用テンプレート読み込み

\begin{document}
% begin header
%%%%% タイトルと作者 ここから %%%%%
\begin{minipage}[c]{0.7\hsize} % タイトルは上から 7 割
    \begin{center}
    % begin title
        {\LARGE
            生命の争闘 % タイトルを入れる
        }
        {\small 
            (大正十年寮歌) % 年などを入れる
        }
    % end title
    \end{center}
\end{minipage}
\begin{minipage}[c]{0.3\hsize} % 作歌作曲は上から 3 割
    \begin{flushright} % 下寄せにする
        % begin name
        青野正男君 作歌\\小峰三千男君 作曲 % 作歌・作曲者
        % end name
    \end{flushright}
\end{minipage}
%%%%% タイトルと作者 ここまで %%%%%
% (1,6 了あり)
% end header

% begin length
\vspace{1.5em} % タイトル, 作者と歌詞の間に隙間を設ける
\newcommand{\linespace}{0.5em} % 行間の設定
\newcommand{\blocksize}{0.5\hsize} % 段組間の設定
\newcommand{\itemmargin}{3em} % 曲番の位置調整の長さ
% end length
% begin body
%%%%% 歌詞 ここから %%%%%
\begin{enumerate} % 番号の箇条書き ここから
    \setlength{\itemindent}{\itemmargin} % 曲番の位置調整
    \begin{minipage}[c]{\blocksize}
    
        \vspace{\linespace}
        \item~\\
        % 1.
        \ruby{生命}{いのち}の\ruby{争闘}{そうとう}\ruby{敗}{やぶ}れじと\\
        \ruby{雪解}{ゆきどけ}の\ruby{野辺}{のべ}に\ruby{萠}{めぐむ}え\ruby{出}{しゅつ}でし\\
        \ruby{浅緑}{せんりょく}なる\ruby{若草}{わかくさ}の\\
        \ruby{伸展}{しんてん}ゆく\ruby{生命}{いのち}\ruby{思}{おも}ふとき\\
        \ruby{若}{わか}き\ruby{力}{きりょく}のよろこびは\\
        \ruby{我等}{われら}が\ruby{胸}{むね}に\ruby{溢}{みつる}るなり
        
    \end{minipage}
    \begin{minipage}[c]{\blocksize}
        
        \vspace{\linespace}
        \item~\\
        % 2.
        \ruby{悲哀}{ひあい}\ruby{誘}{さそ}ふ\ruby{郭公}{かっこう}の\\
        \ruby{声}{こえ}を\ruby{聞}{き}きつつ\ruby{逍遙}{}へば\\
        \ruby{今}{いま}は\ruby{小暗}{おぐら}き\ruby{木下}{きのした}\ruby{闇}{やみ}\\
        \ruby{黒百合}{くろゆり}\ruby{咲}{さき}けど\ruby{春}{はる}いづこ\\
        うつろひやすき\ruby{若}{わか}き\ruby{日}{ひ}を\\
        \ruby{盧生}{ろせい}の\ruby{夢}{ゆめ}となすなかれ
        
    \end{minipage}
    \begin{minipage}[c]{\blocksize}
        
        \vspace{\linespace}
        \item~\\
        % 3.
        \ruby{牧場}{ぼくじょう}に\ruby{虫}{むし}の\ruby{音}{おと}も\ruby{淡}{あわ}く\\
        \ruby{仰}{あお}げば\ruby{高}{たか}し\ruby{秋}{あき}の\ruby{空}{そら}\\
        \ruby{肥}{こえ}\ruby{馬}{ば}\ruby{原頭}{げんとう}に\ruby{嘶}{いなな}きて\\
        \ruby{雄渾}{ゆうこん}の\ruby{気}{き}はあふれつつ\\
        \ruby{崇}{たかし}き\ruby{理想}{りそう}を\ruby{胸}{むね}にして\\
        \ruby{生}{なま}くる\ruby{喜悦}{きえつ}\ruby{謳}{}ふ\ruby{哉}{かな}
        
    \end{minipage}
    \begin{minipage}[c]{\blocksize}
        
        \vspace{\linespace}
        \item~\\
        % 4.
        \ruby{眺}{なが}めはてなき\ruby{石狩}{いしかり}の\\
        \ruby{曠野}{あらの}に\ruby{凋落}{ちょうらく}の\ruby{秋}{あき}\ruby{更}{ふ}けて\\
        \ruby{寂}{さび}しく\ruby{暮}{くれ}るる\ruby{手稲山}{ていねやま}\\
        \ruby{今}{いま}うすれゆく\ruby{赤}{あか}\ruby{陽}{よう}に\\
        \ruby{想}{そう}ひぞ\ruby{馳}{はせ}する\ruby{北欧}{ほくおう}\ruby{州}{しゅう}\\
        \ruby{戦}{せん}禍の\ruby{跡}{あと}の\ruby{夕}{ゆう}まぐれ
        
    \end{minipage}
    \begin{minipage}[c]{\blocksize}
        
        \vspace{\linespace}
        \item~\\
        % 5.
        \ruby{夕}{ゆう}\ruby{吹}{ふ}く\ruby{風}{かぜ}\ruby{膚}{はだ}にしみ\\
        \ruby{音}{おと}も\ruby{淋}{さび}しく\ruby{行}{い}く\ruby{橇}{そり}の\\
        \ruby{大}{だい}\ruby{雪原}{せつげん}に\ruby{消}{しょう}ゆるとき\\
        \ruby{寒月}{かんげつ}\ruby{高}{たか}く\ruby{冴}{さえ}ゆる\ruby{夜半}{やはん}\\
        \ruby{哀愁}{あいしゅう}をこむる\ruby{若人}{わこうど}の\\
        \ruby{瞑想}{めいそう}ぞ\ruby{如何}{いかが}に\ruby{深}{ふか}からん
        
    \end{minipage}
    \begin{minipage}[c]{\blocksize}
        
        \vspace{\linespace}
        \item~\\
        % 6.
        \ruby{嗚呼}{ああ}\ruby{北州}{ほくしゅう}の\ruby{春秋}{しゅんじゅう}に\\
        \ruby{自然}{しぜん}の\ruby{教訓}{きょうくん}\ruby{学}{まな}びつつ\\
        \ruby{尚}{なお}き\ruby{\ruby{生}{い}命}{いのち}に\ruby{生}{い}きなんと\\
        \ruby{精神}{せいしん}を\ruby{磨}{みが}く\ruby{友}{とも}どちよ\\
        \ruby{先人}{せんじん}\ruby{建}{だ}てし\ruby{自治寮}{じちりょう}の\\
        \ruby{貴}{とうと}き\ruby{歴史}{れきし}\ruby{伝}{でん}へかし
    
    \end{minipage}
\end{enumerate} % 番号の箇条書き ここまで
%%%%% 歌詞 ここまで %%%%%
% end body

\end{document}

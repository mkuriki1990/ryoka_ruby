\documentclass[10pt,b5j]{tarticle} % B6 縦書き
% \documentclass[10pt,b5j]{tarticle} % B6 縦書き
\AtBeginDvi{\special{papersize=128mm,182mm}} % B6 用用紙サイズ
\usepackage{otf} % Unicode で字を入力するのに必要なパッケージ
\usepackage[size=b6j]{bxpapersize} % B6 用紙サイズを指定
\usepackage[dvipdfmx]{graphicx} % 画像を挿入するためのパッケージ
\usepackage[dvipdfmx]{color} % 色をつけるためのパッケージ
\usepackage{pxrubrica} % ルビを振るためのパッケージ
\usepackage{plext} % 漢数字の enumerate を使うためのパッケージ
\usepackage{multicol} % 複数段組を作るためのパッケージ
\setlength{\topmargin}{14mm} % 上下方向のマージン
\addtolength{\topmargin}{-1in} % 
\setlength{\oddsidemargin}{11mm} % 左右方向のマージン
\addtolength{\oddsidemargin}{-1in} % 
\setlength{\textwidth}{154mm} % B6 用
\setlength{\textheight}{108mm} % B6 用
\setlength{\headsep}{0mm} % 
\setlength{\headheight}{0mm} % 
\setlength{\topskip}{0mm} % 
\setlength{\parskip}{0pt} % 
\def\theenumi{\Kanji{enumi}} % 箇条書きのフォーマットを漢数字に変更
\parindent = 0pt % 段落下げしない
\pagestyle{empty} % すべてのページ番号を消去
% \renewcommand{\baselinestretch}{0.9} % 行間の倍率
 % B6 用テンプレート読み込み

\begin{document}
% begin header
%%%%% タイトルと作者 ここから %%%%%
\begin{minipage}[c]{0.7\hsize} % タイトルは上から 7 割
    \begin{center}
    % begin title
        {\LARGE
            北海道帝国大学付属水産専門部歌 % タイトルを入れる
        }
        {\small 
            (昭和三年) % 年などを入れる
        }
    % end title
    \end{center}
\end{minipage}
\begin{minipage}[c]{0.3\hsize} % 作歌作曲は上から 3 割
    \begin{flushright} % 下寄せにする
        % begin name
        西村真琴君 作歌\\永井幸次君 作曲 % 作歌・作曲者
        % end name
    \end{flushright}
\end{minipage}
%%%%% タイトルと作者 ここまで %%%%%
% % end header

% begin length
\vspace{1.5em} % タイトル, 作者と歌詞の間に隙間を設ける
\newcommand{\linespace}{0.5em} % 行間の設定
\newcommand{\blocksize}{0.5\hsize} % 段組間の設定
\newcommand{\itemmargin}{6em} % 曲番の位置調整の長さ
% end length
% begin body
%%%%% 歌詞 ここから %%%%%
\begin{enumerate} % 番号の箇条書き ここから
    \setlength{\itemindent}{\itemmargin} % 曲番の位置調整
    \begin{minipage}[c]{\blocksize}
    
        \vspace{\linespace}
        \item~\\
        % 1.
        \ruby{洋}{}のひんがし\\
        \ruby{秋津島根}{}は\\
        \ruby{環海実}{}に\\
        \ruby{萬里}{}をこえて\\
        \ruby{天恵深}{}き\\
        \ruby{水産}{}の\ruby{国}{}\\
        \ruby{新日本}{}の\\
        \ruby{使命}{}に\ruby{活}{}きよ\\
        \ruby{時代}{}は\ruby{来}{}る\\
        \ruby{我等}{}が\ruby{時代}{}
        
        \vspace{\linespace}
        \item~\\
        % 2.
        \ruby{朝日}{}まばゆく\\
        かもめはとぶよ\\
        \ruby{平和}{}は\ruby{漲}{}る\\
        \ruby{大空}{}のもと\\
        \ruby{未来楽}{}しき\\
        \ruby{若人}{}のせて\\
        \ruby{理想}{}の\ruby{旗}{}の\\
        \ruby{意気揚々}{}と\\
        \ruby{静}{}かにすすむ\\
        わがおしょろ\ruby{丸}{}
        
        \vspace{\linespace}
        \item~\\
        % 3.
        かぶと\ruby{岩}{}に\\
        \ruby{寄}{}せてはかへる\\
        \ruby{波}{}は\ruby{砕}{}けて\\
        どんどとうつよ\\
        \ruby{堅忍不抜}{}\\
        \ruby{自然}{}の\ruby{態}{}を\\
        \ruby{我}{}が\ruby{実習}{}の\\
        あしたに\ruby{夕}{}に\\
        \ruby{眺}{}めて\ruby{艪}{}を\\
        \ruby{煉}{}りてぞゆかむ
        
        \vspace{\linespace}
        \item~\\
        % 4.
        \ruby{百河千流}{}\\
        \ruby{併}{}せて\ruby{呑}{}めど\\
        \ruby{海}{}は\ruby{鈍化}{}の\\
        \ruby{力}{}をもてり\\
        \ruby{大望}{}もゆる\\
        \ruby{我}{}がはらからよ\\
        ここに\ruby{因}{}みて\\
        \ruby{襟度}{}をひろめ\\
        \ruby{天下}{}にのべばや\\
        \ruby{水産魂}{}
    
    \end{minipage}
\end{enumerate} % 番号の箇条書き ここまで
%%%%% 歌詞 ここまで %%%%%
% end body

\end{document}

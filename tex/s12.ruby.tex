\documentclass[10pt,b5j]{tarticle} % B6 縦書き
% \documentclass[10pt,b5j]{tarticle} % B6 縦書き
\AtBeginDvi{\special{papersize=128mm,182mm}} % B6 用用紙サイズ
\usepackage{otf} % Unicode で字を入力するのに必要なパッケージ
\usepackage[size=b6j]{bxpapersize} % B6 用紙サイズを指定
\usepackage[dvipdfmx]{graphicx} % 画像を挿入するためのパッケージ
\usepackage[dvipdfmx]{color} % 色をつけるためのパッケージ
\usepackage{pxrubrica} % ルビを振るためのパッケージ
\usepackage{multicol} % 複数段組を作るためのパッケージ
\setlength{\topmargin}{14mm} % 上下方向のマージン
\addtolength{\topmargin}{-1in} % 
\setlength{\oddsidemargin}{11mm} % 左右方向のマージン
\addtolength{\oddsidemargin}{-1in} % 
\setlength{\textwidth}{154mm} % B6 用
\setlength{\textheight}{108mm} % B6 用
\setlength{\headsep}{0mm} % 
\setlength{\headheight}{0mm} % 
\setlength{\topskip}{0mm} % 
\setlength{\parskip}{0pt} % 
\def\labelenumi{\theenumi、} % 箇条書きのフォーマット
\parindent = 0pt % 段落下げしない

 % B6 用テンプレート読み込み

\begin{document}
% begin header
%%%%% タイトルと作者 ここから %%%%%
\begin{minipage}[c]{0.7\hsize} % タイトルは上から 7 割
    \begin{center}
    % begin title
        {\LARGE
            魂の故郷 % タイトルを入れる
        }
        {\small 
            (昭和十二年寮歌) % 年などを入れる
        }
    % end title
    \end{center}
\end{minipage}
\begin{minipage}[c]{0.3\hsize} % 作歌作曲は上から 3 割
    \begin{flushright} % 下寄せにする
        % begin name
        山崎善陽君 作歌\\平城鷹雄君 作曲 % 作歌・作曲者
        % end name
    \end{flushright}
\end{minipage}
%%%%% タイトルと作者 ここまで %%%%%
% (1,5 繰り返しなし)
% end header

% begin length
\vspace{1.5em} % タイトル, 作者と歌詞の間に隙間を設ける
\newcommand{\linespace}{0.5em} % 行間の設定
\newcommand{\blocksize}{0.5\hsize} % 段組間の設定
\newcommand{\itemmargin}{3em} % 曲番の位置調整の長さ
% end length
% begin body
%%%%% 歌詞 ここから %%%%%
\begin{enumerate} % 番号の箇条書き ここから
    \setlength{\itemindent}{\itemmargin} % 曲番の位置調整
    \begin{minipage}[c]{\blocksize}
    
        \vspace{\linespace}
        \item~\\
        % 1.
        \ruby{魂}{たましい}の\ruby{故郷}{こきょう}に\ruby{立}{た}つ\\
        \ruby{星}{ほし}\ruby{清}{きよ}き\ruby{楡}{にれ}の\ruby{園}{えん}よ\\
        \ruby{花}{はな}\ruby{芳}{よし}る\ruby{三春}{みはる}の\ruby{夢}{ゆめ}\\
        \ruby{感激}{かんげき}の\ruby{涙}{なみだ}あふれて\\
        \ruby{原始}{げんし}\ruby{林}{りん}\ruby{蔭}{かげ}に\ruby{盃}{さかずき}かはす\\
        \ruby{青春}{せいしゅん}きの\ruby{記念}{きねん}の\ruby{宴}{うたげ}\\
        \ruby{歌}{うた}ふなり\\
        \ruby{自治}{じち}と\ruby{自由}{じゆう}の\ruby{高}{たか}き\ruby{誇}{ほこ}を
        
    \end{minipage}
    \begin{minipage}[c]{\blocksize}
        
        \vspace{\linespace}
        \item~\\
        % 2.
        \ruby{六}{ろく}\ruby{十}{じゅう}\ruby{年}{ねん}の\ruby{青史}{せいし}は\ruby{薫}{かお}り\\
        \ruby{郭公}{かっこう}の\ruby{啼声}{}もはるかに\\
        \ruby{紺青}{こんじょう}の\ruby{入相}{いりあい}の\ruby{空}{そら}\\
        \ruby{魂}{たましい}は\ruby{虚空}{こくう}に\ruby{走}{はし}せて\\
        \ruby{住}{じゅう}\ruby{昔}{むかし}の\ruby{意気}{いき}を\ruby{慕}{}ふ\\
        \ruby{尽}{つ}きるなき\ruby{川}{かわ}のせせらぎ\\
        \ruby{夢}{ゆめ}ふかし\\
        \ruby{残}{ざん}\ruby{春}{はる}あはきポプラ\ruby{並木}{なみき}よ
        
    \end{minipage}
    \begin{minipage}[c]{\blocksize}
        
        \vspace{\linespace}
        \item~\\
        % 3.
        いで\ruby{湯}{いでゆ}\ruby{湧}{わ}く\ruby{郷}{さと}の\ruby{宴}{うたげ}は\\
        \ruby{夜}{よ}もすがら\ruby{感激}{かんげき}はてなき\\
        \ruby{絢爛}{けんらん}たる\ruby{瞬間}{しゅんかん}の\ruby{夢}{ゆめ}\\
        \ruby{落葉松}{からまつ}の\ruby{林}{はやし}\ruby{時}{}\ruby{雨}{しぐ}れて\\
        \ruby{颼}{}\ruby{々}{々}の\ruby{悲歌}{ひか}の\ruby{調}{しら}べは\\
        \ruby{楡}{にれ}\ruby{鐘}{かね}の\ruby{響}{ひびき}と\ruby{闇}{やみ}にきえゆく\\
        さびしらに\\
        \ruby{秋}{あき}\ruby{深}{ふか}みゆく\ruby{静寂}{せいじゃく}の\ruby{都}{と}
        
    \end{minipage}
    \begin{minipage}[c]{\blocksize}
        
        \vspace{\linespace}
        \item~\\
        % 4.
        \ruby{颷}{}\ruby{々}{々}の\ruby{暴風}{ぼうふう}おさまり\\
        \ruby{際涯}{さいがい}なき\ruby{雪}{ゆき}の\ruby{荒野}{あらの}に\\
        \ruby{皎}{こう}\ruby{々}{々}と\ruby{月光}{げっこう}\ruby{冴}{さえ}ゆる\\
        \ruby{橇}{そり}の\ruby{音}{おと}の\ruby{玻窓}{}にこほりて\\
        \ruby{限}{かぎ}りなき\ruby{瞑想}{めいそう}をさそふ\\
        \ruby{悠久}{ゆうきゅう}の\ruby{時}{とき}の\ruby{流転}{るてん}\\
        \ruby{人}{ひと}の\ruby{世}{よ}の\\
        \ruby{悲}{かな}しき\ruby{運命}{うんめい}ぞ\ruby{明日}{あした}の\ruby{旅路}{たびじ}は
        
    \end{minipage}
    \begin{minipage}[c]{\blocksize}
        
        \vspace{\linespace}
        \item~\\
        % 5.
        \ruby{曠野}{あらの}に\ruby{高}{こう}\ruby{嘯}{}ふ\ruby{恵}{めぐみ}\ruby{迪}{すすむ}の\ruby{健児}{けんじ}\\
        \ruby{毅然}{きぜん}たり\ruby{若}{わか}き\ruby{生命}{せいめい}よ\\
        \ruby{先人}{せんじん}の\ruby{崇}{たかし}き\ruby{訓戒}{くんかい}に\\
        \ruby{大}{おお}いなる\ruby{野心}{やしん}\ruby{育}{はぐく}む\\
        \ruby{慨世}{がいせい}の\ruby{憂}{う}はあれど\\
        ここ\ruby{暫}{しば}し\ruby{休息}{きゅうそく}もとめて\\
        いざ\ruby{寮}{りょう}\ruby{友}{とも}よ\\
        のこりの\ruby{春}{はる}を\ruby{惜}{お}しまざらめや
    
    \end{minipage}
\end{enumerate} % 番号の箇条書き ここまで
%%%%% 歌詞 ここまで %%%%%
% end body

\end{document}

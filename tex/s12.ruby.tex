\documentclass[10pt,b5j]{tarticle} % B6 縦書き
% \documentclass[10pt,b5j]{tarticle} % B6 縦書き
\AtBeginDvi{\special{papersize=128mm,182mm}} % B6 用用紙サイズ
\usepackage{otf} % Unicode で字を入力するのに必要なパッケージ
\usepackage[size=b6j]{bxpapersize} % B6 用紙サイズを指定
\usepackage[dvipdfmx]{graphicx} % 画像を挿入するためのパッケージ
\usepackage[dvipdfmx]{color} % 色をつけるためのパッケージ
\usepackage{pxrubrica} % ルビを振るためのパッケージ
\usepackage{multicol} % 複数段組を作るためのパッケージ
\setlength{\topmargin}{14mm} % 上下方向のマージン
\addtolength{\topmargin}{-1in} % 
\setlength{\oddsidemargin}{11mm} % 左右方向のマージン
\addtolength{\oddsidemargin}{-1in} % 
\setlength{\textwidth}{154mm} % B6 用
\setlength{\textheight}{108mm} % B6 用
\setlength{\headsep}{0mm} % 
\setlength{\headheight}{0mm} % 
\setlength{\topskip}{0mm} % 
\setlength{\parskip}{0pt} % 
\def\labelenumi{\theenumi、} % 箇条書きのフォーマット
\parindent = 0pt % 段落下げしない

 % B6 用テンプレート読み込み

\begin{document}
% begin header
%%%%% タイトルと作者 ここから %%%%%
\begin{minipage}[c]{0.7\hsize} % タイトルは上から 7 割
    \begin{center}
    % begin title
        {\LARGE
            魂の故郷 % タイトルを入れる
        }
        {\small 
            (昭和十二年寮歌) % 年などを入れる
        }
    % end title
    \end{center}
\end{minipage}
\begin{minipage}[c]{0.3\hsize} % 作歌作曲は上から 3 割
    \begin{flushright} % 下寄せにする
        % begin name
        山崎善陽君 作歌\\平城鷹雄君 作曲 % 作歌・作曲者
        % end name
    \end{flushright}
\end{minipage}
%%%%% タイトルと作者 ここまで %%%%%
% (1,5 繰り返しなし)
% end header

% begin length
\vspace{1.5em} % タイトル, 作者と歌詞の間に隙間を設ける
\newcommand{\linespace}{0.5em} % 行間の設定
\newcommand{\blocksize}{0.5\hsize} % 段組間の設定
\newcommand{\itemmargin}{6em} % 曲番の位置調整の長さ
% end length
% begin body
%%%%% 歌詞 ここから %%%%%
\begin{enumerate} % 番号の箇条書き ここから
    \setlength{\itemindent}{\itemmargin} % 曲番の位置調整
    \begin{minipage}[c]{\blocksize}
    
        \vspace{\linespace}
        \item~\\
        % 1.
        \ruby{魂}{}の\ruby{故郷}{}に\ruby{立}{}つ\\
        \ruby{星清}{}き\ruby{楡}{}の\ruby{園}{}よ\\
        \ruby{花芳}{}る\ruby{三春}{}の\ruby{夢}{}\\
        \ruby{感激}{}の\ruby{涙}{}あふれて\\
        \ruby{原始林蔭}{}に\ruby{盃}{}かはす\\
        \ruby{青春}{}きの\ruby{記念}{}の\ruby{宴}{}\\
        \ruby{歌}{}ふなり\\
        \ruby{自治}{}と\ruby{自由}{}の\ruby{高}{}き\ruby{誇}{}を
        
        \vspace{\linespace}
        \item~\\
        % 2.
        \ruby{六十年}{}の\ruby{青史}{}は\ruby{薫}{}り\\
        \ruby{郭公}{}の\ruby{啼声}{}もはるかに\\
        \ruby{紺青}{}の\ruby{入相}{}の\ruby{空}{}\\
        \ruby{魂}{}は\ruby{虚空}{}に\ruby{走}{}せて\\
        \ruby{住昔}{}の\ruby{意気}{}を\ruby{慕}{}ふ\\
        \ruby{尽}{}きるなき\ruby{川}{}のせせらぎ\\
        \ruby{夢}{}ふかし\\
        \ruby{残春}{}あはきポプラ\ruby{並木}{}よ
        
        \vspace{\linespace}
        \item~\\
        % 3.
        いで\ruby{湯湧}{}く\ruby{郷}{}の\ruby{宴}{}は\\
        \ruby{夜}{}もすがら\ruby{感激}{}はてなき\\
        \ruby{絢爛}{}たる\ruby{瞬間}{}の\ruby{夢}{}\\
        \ruby{落葉松}{}の\ruby{林時雨}{}れて\\
        \ruby{颼々}{}の\ruby{悲歌}{}の\ruby{調}{}べは\\
        \ruby{楡鐘}{}の\ruby{響}{}と\ruby{闇}{}にきえゆく\\
        さびしらに\\
        \ruby{秋深}{}みゆく\ruby{静寂}{}の\ruby{都}{}
        
        \vspace{\linespace}
        \item~\\
        % 4.
        \ruby{颷々}{}の\ruby{暴風}{}おさまり\\
        \ruby{際涯}{}なき\ruby{雪}{}の\ruby{荒野}{}に\\
        \ruby{皎々}{}と\ruby{月光冴}{}ゆる\\
        \ruby{橇}{}の\ruby{音}{}の\ruby{玻窓}{}にこほりて\\
        \ruby{限}{}りなき\ruby{瞑想}{}をさそふ\\
        \ruby{悠久}{}の\ruby{時}{}の\ruby{流転}{}\\
        \ruby{人}{}の\ruby{世}{}の\\
        \ruby{悲}{}しき\ruby{運命}{}ぞ\ruby{明日}{}の\ruby{旅路}{}は
        
        \vspace{\linespace}
        \item~\\
        % 5.
        \ruby{曠野}{}に\ruby{高嘯}{}ふ\ruby{恵迪}{}の\ruby{健児}{}\\
        \ruby{毅然}{}たり\ruby{若}{}き\ruby{生命}{}よ\\
        \ruby{先人}{}の\ruby{崇}{}き\ruby{訓戒}{}に\\
        \ruby{大}{}いなる\ruby{野心育}{}む\\
        \ruby{慨世}{}の\ruby{憂}{}はあれど\\
        ここ\ruby{暫}{}し\ruby{休息}{}もとめて\\
        いざ\ruby{寮友}{}よ\\
        のこりの\ruby{春}{}を\ruby{惜}{}しまざらめや
    
    \end{minipage}
\end{enumerate} % 番号の箇条書き ここまで
%%%%% 歌詞 ここまで %%%%%
% end body

\end{document}

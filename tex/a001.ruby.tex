\documentclass[10pt,b5j]{tarticle} % B6 縦書き
% \documentclass[10pt,b5j]{tarticle} % B6 縦書き
\AtBeginDvi{\special{papersize=128mm,182mm}} % B6 用用紙サイズ
\usepackage{otf} % Unicode で字を入力するのに必要なパッケージ
\usepackage[size=b6j]{bxpapersize} % B6 用紙サイズを指定
\usepackage[dvipdfmx]{graphicx} % 画像を挿入するためのパッケージ
\usepackage[dvipdfmx]{color} % 色をつけるためのパッケージ
\usepackage{pxrubrica} % ルビを振るためのパッケージ
\usepackage{multicol} % 複数段組を作るためのパッケージ
\setlength{\topmargin}{14mm} % 上下方向のマージン
\addtolength{\topmargin}{-1in} % 
\setlength{\oddsidemargin}{11mm} % 左右方向のマージン
\addtolength{\oddsidemargin}{-1in} % 
\setlength{\textwidth}{154mm} % B6 用
\setlength{\textheight}{108mm} % B6 用
\setlength{\headsep}{0mm} % 
\setlength{\headheight}{0mm} % 
\setlength{\topskip}{0mm} % 
\setlength{\parskip}{0pt} % 
\def\labelenumi{\theenumi、} % 箇条書きのフォーマット
\parindent = 0pt % 段落下げしない

 % B6 用テンプレート読み込み

\begin{document}
% begin header
%%%%% タイトルと作者 ここから %%%%%
\begin{minipage}[c]{0.7\hsize} % タイトルは上から 7 割
    \begin{center}
    % begin title
        {\LARGE
            あらうれし % タイトルを入れる
        }
        {\small 
            (大正元年桜星会歌) % 年などを入れる
        }
    % end title
    \end{center}
\end{minipage}
\begin{minipage}[c]{0.3\hsize} % 作歌作曲は上から 3 割
    \begin{flushright} % 下寄せにする
        % begin name
        横山芳介君 作歌\\柳沢秀雄君 作曲 % 作歌・作曲者
        % end name
    \end{flushright}
\end{minipage}
%%%%% タイトルと作者 ここまで %%%%%
% % end header

% begin body
\vspace{1.5em} % タイトル, 作者と歌詞の間に隙間を設ける
\newcommand{\linespace}{0.5em} % 行間の設定
\newcommand{\blocksize}{0.5\hsize} % 段組間の設定
%%%%% 歌詞 ここから %%%%%
% begin lilycs
\begin{enumerate} % 番号の箇条書き ここから
    \begin{minipage}[c]{\blocksize}
    
        \vspace{\linespace}
        \item
        % 1.
        あらうれし\\
        \ruby{我等}{}が\ruby{生命}{}は\ruby{若}{}ければ\\
        \ruby{熱}{}き\ruby{血潮}{} \ruby{強気氣力}{}に\\
        \ruby{漲}{}る\ruby{春日}{}の \ruby{光}{}ぞ\ruby{匂}{}う\\
        あはれ\ruby{吾}{}が\ruby{友}{}\\
        \ruby{櫻}{}と\ruby{星}{}に \ruby{明暮}{}を\\
        \ruby{契固}{}めて \ruby{共々}{}に\\
        \ruby{學}{}ぶはうれし \ruby{美}{}しき\ruby{國}{}
        
        \vspace{\linespace}
        \item
        % 2.
        あらたのし\\
        \ruby{我等}{}が\ruby{心}{}は\ruby{若}{}ければ\\
        \ruby{高}{}き\ruby{希望}{} \ruby{深}{}き\ruby{思想}{}に\\
        \ruby{湧}{}き\ruby{來}{}る\ruby{泉}{}は \ruby{汲}{}めども\ruby{盡}{}きず\\
        あはれ\ruby{吾}{}が\ruby{友}{}\\
        \ruby{學}{}びの\ruby{苑}{}は \ruby{常磐}{}なり\\
        \ruby{狂}{}ふ\ruby{波折}{}の \ruby{世}{}に\ruby{立}{}ちて\\
        \ruby{進}{}むはたのし \ruby{懐}{}しき\ruby{國}{}
    
    \end{minipage}
\end{enumerate} % 番号の箇条書き ここまで
% end lilycs
%%%%% 歌詞 ここまで %%%%%
% end body

\end{document}

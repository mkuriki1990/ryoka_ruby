\documentclass[10pt,b5j]{tarticle} % B6 縦書き
% \documentclass[10pt,b5j]{tarticle} % B6 縦書き
\AtBeginDvi{\special{papersize=128mm,182mm}} % B6 用用紙サイズ
\usepackage{otf} % Unicode で字を入力するのに必要なパッケージ
\usepackage[size=b6j]{bxpapersize} % B6 用紙サイズを指定
\usepackage[dvipdfmx]{graphicx} % 画像を挿入するためのパッケージ
\usepackage[dvipdfmx]{color} % 色をつけるためのパッケージ
\usepackage{pxrubrica} % ルビを振るためのパッケージ
\usepackage{multicol} % 複数段組を作るためのパッケージ
\setlength{\topmargin}{14mm} % 上下方向のマージン
\addtolength{\topmargin}{-1in} % 
\setlength{\oddsidemargin}{11mm} % 左右方向のマージン
\addtolength{\oddsidemargin}{-1in} % 
\setlength{\textwidth}{154mm} % B6 用
\setlength{\textheight}{108mm} % B6 用
\setlength{\headsep}{0mm} % 
\setlength{\headheight}{0mm} % 
\setlength{\topskip}{0mm} % 
\setlength{\parskip}{0pt} % 
\def\labelenumi{\theenumi、} % 箇条書きのフォーマット
\parindent = 0pt % 段落下げしない

 % B6 用テンプレート読み込み

\begin{document}
% begin header
%%%%% タイトルと作者 ここから %%%%%
\begin{minipage}[c]{0.7\hsize} % タイトルは上から 7 割
    \begin{center}
    % begin title
        {\LARGE
            いつの日か生命結ばん % タイトルを入れる
        }
        {\small 
            (昭和41年寮歌) % 年などを入れる
        }
    % end title
    \end{center}
\end{minipage}
\begin{minipage}[c]{0.3\hsize} % 作歌作曲は上から 3 割
    \begin{flushright} % 下寄せにする
        % begin name
        須藤洋一君 作歌\\吉川正文君 作曲 % 作歌・作曲者
        % end name
    \end{flushright}
\end{minipage}
%%%%% タイトルと作者 ここまで %%%%%
% (1,2,5,6 繰り返しなし)
% end header

% begin body
\vspace{1.5em} % タイトル, 作者と歌詞の間に隙間を設ける
\newcommand{\linespace}{0.5em} % 行間の設定
\newcommand{\blocksize}{0.5\hsize} % 段組間の設定
%%%%% 歌詞 ここから %%%%%
% begin lilycs
\begin{enumerate} % 番号の箇条書き ここから
    \begin{minipage}[c]{\blocksize}
    
        \vspace{\linespace}
        \item
        \ruby{重畳}{}たる\ruby{手稲}{}\\
        \ruby{藻岩}{}の\ruby{山脈}{}を\\
        \ruby{吾}{}が\ruby{宿舎}{}の\ruby{青垣}{}となし\\
        \ruby{鬱乎}{}たる\ruby{原始}{}の\ruby{叢林}{}を\\
        \ruby{吾}{}が\ruby{逍遥}{}の\ruby{小径}{}となす。\\
        \ruby{吾}{}が\ruby{寮友}{}よ\ruby{草原}{}に\ruby{出}{}でよ、\\
        \ruby{暗}{}き\ruby{孤城}{}より\ruby{出}{}でんかな。\\
        \ruby{深遠}{}き\ruby{蒼穹}{}あまりに\ruby{青}{}く、\\
        \ruby{輝}{}く\ruby{雪原}{}あまりに\ruby{白}{}し。\\
        さればよしその\ruby{身}{}は\\
        \ruby{平々凡々}{}ならんとも、\\
        \ruby{吾等}{}が\ruby{野望尽}{}くるを\ruby{知}{}らず。\\
        \ruby{静寂}{}を\ruby{破}{}る\ruby{蛮声}{}に、\\
        \ruby{吹雪鎮}{}むる\ruby{高吟}{}に\\
        \ruby{青春}{}の\ruby{意気託}{}しなん
        
        \vspace{\linespace}
        \item
        % 1.
        いつの\ruby{日}{}か\ruby{生命結}{}ばん\\
        \ruby{碧空高}{}き\ruby{楡}{}よポプラよ\\
        \ruby{黄金}{}なす\ruby{銀杏並木}{}よ\\
        \ruby{枯}{}れ\ruby{枯}{}れと\ruby{曠野}{}に\ruby{朔風吹}{}けば\\
        \ruby{荒涼}{}の\ruby{憂愁}{}よぎりぬ
        
        \vspace{\linespace}
        \item
        % 2.
        \ruby{島松}{}の\ruby{雪}{}の\ruby{路上}{}に\\
        \ruby{手}{}を\ruby{振}{}りし\ruby{遠}{}き\ruby{日}{}の\ruby{夢}{}\\
        \ruby{去}{}り\ruby{行}{}きぬ\ruby{偉大}{}なる\ruby{巨影}{}\\
        \ruby{君聞}{}くや\ruby{馬上}{}の\ruby{声}{}を\\
        \ruby{広}{}ごれる\ruby{石狩}{}の\ruby{原野}{}に
        
        \vspace{\linespace}
        \item
        % 3.
        \ruby{鶏}{}はまだ\ruby{長鳴}{}かずして\\
        \ruby{貪}{}れる\ruby{熟睡}{}をあとに\\
        \ruby{仄暗}{}き\ruby{叢林}{}に\ruby{佇立}{}てば\\
        \ruby{今}{}いずこ\ruby{青}{}き\ruby{野望}{}は\\
        \ruby{消}{}え\ruby{行}{}くや\ruby{先人}{}の\ruby{遺声}{}
        
        \vspace{\linespace}
        \item
        % 4.
        \ruby{蝦夷人}{}よ\ruby{今}{}こそ\ruby{瞑想}{}え\\
        \ruby{星辰}{}しるき\ruby{彼}{}の\ruby{冬空}{}に\\
        \ruby{雨翔}{}ける\ruby{天馬}{}の\ruby{行方}{}\\
        \ruby{吹雪}{}き\ruby{荒}{}ぶ\ruby{北風}{}をつぶてを\\
        \ruby{若駒}{}の\ruby{鞭}{}とはなさん
        
        \vspace{\linespace}
        \item
        % 5.
        \ruby{睦}{}み\ruby{来}{}て\ruby{親友}{}は\ruby{高唱}{}えど\\
        \ruby{舌苦}{}き\ruby{地酒}{}に\ruby{酔}{}い\ruby{痴}{}れ\\
        ストームに\ruby{身}{}は\ruby{狂乱}{}うとも\\
        \ruby{忘}{}れ\ruby{得}{}じ\ruby{果}{}てなき\ruby{旅路}{}\\
        この\ruby{惆悵誰}{}に\ruby{語}{}らん
        
        \vspace{\linespace}
        \item
        % 6.
        \ruby{暖}{}かき\ruby{光求}{}めて\\
        \ruby{彷徨}{}えり\ruby{冷}{}たき\ruby{野末}{}\\
        \ruby{北国}{}に\ruby{春来}{}りなば\\
        \ruby{若}{}き\ruby{日}{}の\ruby{稚}{}き\ruby{愁思}{}は\\
        \ruby{雪}{}の\ruby{如融}{}けて\ruby{流}{}れん
    
    \end{minipage}
\end{enumerate} % 番号の箇条書き ここまで
% end lilycs
%%%%% 歌詞 ここまで %%%%%
% end body

\end{document}

\documentclass[10pt,b5j]{tarticle} % B6 縦書き
% \documentclass[10pt,b5j]{tarticle} % B6 縦書き
\AtBeginDvi{\special{papersize=128mm,182mm}} % B6 用用紙サイズ
\usepackage{otf} % Unicode で字を入力するのに必要なパッケージ
\usepackage[size=b6j]{bxpapersize} % B6 用紙サイズを指定
\usepackage[dvipdfmx]{graphicx} % 画像を挿入するためのパッケージ
\usepackage[dvipdfmx]{color} % 色をつけるためのパッケージ
\usepackage{pxrubrica} % ルビを振るためのパッケージ
\usepackage{multicol} % 複数段組を作るためのパッケージ
\setlength{\topmargin}{14mm} % 上下方向のマージン
\addtolength{\topmargin}{-1in} % 
\setlength{\oddsidemargin}{11mm} % 左右方向のマージン
\addtolength{\oddsidemargin}{-1in} % 
\setlength{\textwidth}{154mm} % B6 用
\setlength{\textheight}{108mm} % B6 用
\setlength{\headsep}{0mm} % 
\setlength{\headheight}{0mm} % 
\setlength{\topskip}{0mm} % 
\setlength{\parskip}{0pt} % 
\def\labelenumi{\theenumi、} % 箇条書きのフォーマット
\parindent = 0pt % 段落下げしない

 % B6 用テンプレート読み込み

\begin{document}
% begin header
%%%%% タイトルと作者 ここから %%%%%
\begin{minipage}[c]{0.7\hsize} % タイトルは上から 7 割
    \begin{center}
    % begin title
        {\LARGE
            水産科応援歌 % タイトルを入れる
        }
        {\small 
            (明治四十二年) % 年などを入れる
        }
    % end title
    \end{center}
\end{minipage}
\begin{minipage}[c]{0.3\hsize} % 作歌作曲は上から 3 割
    \begin{flushright} % 下寄せにする
        % begin name
         % 作歌・作曲者
        % end name
    \end{flushright}
\end{minipage}
%%%%% タイトルと作者 ここまで %%%%%
% % end header

% begin body
\vspace{1.5em} % タイトル, 作者と歌詞の間に隙間を設ける
\newcommand{\linespace}{0.5em} % 行間の設定
\newcommand{\blocksize}{0.5\hsize} % 段組間の設定
%%%%% 歌詞 ここから %%%%%
% begin lilycs
\begin{enumerate} % 番号の箇条書き ここから
    \begin{minipage}[c]{\blocksize}
    
        \vspace{\linespace}
        \item
        % 1.
        \ruby{天下}{}の\ruby{粋}{}を\ruby{鍾}{}めたる\\
        \ruby{我}{}が\ruby{水産}{}の\ruby{健男児}{}\\
        \ruby{東風薫}{}る\ruby{今日}{}の\ruby{日}{}を\\
        \ruby{歌}{}へ\ruby{撰手}{}が\ruby{其}{}の\ruby{風姿}{}
        
        \vspace{\linespace}
        \item
        % 2.
        \ruby{春}{}、\ruby{南洋}{}の\ruby{曙}{}や\\
        \ruby{紅霞渺々立}{}ち\ruby{罩}{}めて\\
        \ruby{舟}{}の\ruby{行手}{}を\ruby{鎖}{}す\ruby{時}{}\\
        \ruby{健児}{}の\ruby{力}{}、\ruby{波}{}を\ruby{蹴}{}る
        
        \vspace{\linespace}
        \item
        % 3.
        \ruby{秋}{}、\ruby{北海}{}の\ruby{夕潮}{}や\\
        \ruby{銀星光}{}り\ruby{幽}{}かにて\\
        \ruby{船}{}の\ruby{行手}{}の\ruby{暗}{}き\ruby{時}{}\\
        \ruby{健児}{}の\ruby{勇氣鯨斬}{}る
        
        \vspace{\linespace}
        \item
        % 4.
        \ruby{白浪激}{}す\ruby{巌}{}の\ruby{上}{}\\
        \ruby{健児}{}が\ruby{叫}{}ぶ\ruby{凱歌}{}は\\
        \ruby{朝}{}?\ruby{躍}{}る\ruby{大洋}{}の\\
        \ruby{遠}{}き\ruby{紫雲}{}に\ruby{響}{}く\ruby{哉}{}
        
        \vspace{\linespace}
        \item
        % 5.
        \ruby{正義}{}の\ruby{血潮火}{}と\ruby{燃}{}ゆる\\
        \ruby{我}{}が\ruby{水産}{}の\ruby{健男児}{}\\
        \ruby{歌}{}ひ\ruby{果}{}てにしこの\ruby{夕}{}べ\\
        \ruby{歌}{}へ\ruby{撰手}{}が\ruby{其}{}の\ruby{勳巧}{}
        
        \vspace{\linespace}
        \item
        (「\ruby{勇敢}{}なる\ruby{水兵}{}」の\ruby{譜}{})
    
    \end{minipage}
\end{enumerate} % 番号の箇条書き ここまで
% end lilycs
%%%%% 歌詞 ここまで %%%%%
% end body

\end{document}

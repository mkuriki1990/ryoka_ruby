\documentclass[10pt,b5j]{tarticle} % B6 縦書き
% \documentclass[10pt,b5j]{tarticle} % B6 縦書き
\AtBeginDvi{\special{papersize=128mm,182mm}} % B6 用用紙サイズ
\usepackage{otf} % Unicode で字を入力するのに必要なパッケージ
\usepackage[size=b6j]{bxpapersize} % B6 用紙サイズを指定
\usepackage[dvipdfmx]{graphicx} % 画像を挿入するためのパッケージ
\usepackage[dvipdfmx]{color} % 色をつけるためのパッケージ
\usepackage{pxrubrica} % ルビを振るためのパッケージ
\usepackage{plext} % 漢数字の enumerate を使うためのパッケージ
\usepackage{multicol} % 複数段組を作るためのパッケージ
\setlength{\topmargin}{14mm} % 上下方向のマージン
\addtolength{\topmargin}{-1in} % 
\setlength{\oddsidemargin}{11mm} % 左右方向のマージン
\addtolength{\oddsidemargin}{-1in} % 
\setlength{\textwidth}{154mm} % B6 用
\setlength{\textheight}{108mm} % B6 用
\setlength{\headsep}{0mm} % 
\setlength{\headheight}{0mm} % 
\setlength{\topskip}{0mm} % 
\setlength{\parskip}{0pt} % 
\def\theenumi{\Kanji{enumi}} % 箇条書きのフォーマットを漢数字に変更
\parindent = 0pt % 段落下げしない
\pagestyle{empty} % すべてのページ番号を消去
% \renewcommand{\baselinestretch}{0.9} % 行間の倍率
 % B6 用テンプレート読み込み

\begin{document}
% begin header
%%%%% タイトルと作者 ここから %%%%%
\begin{minipage}[c]{0.7\hsize} % タイトルは上から 7 割
    \begin{center}
    % begin title
        {\LARGE
            林学実科応援歌 % タイトルを入れる
        }
        {\small 
            (大正4年) % 年などを入れる
        }
    % end title
    \end{center}
\end{minipage}
\begin{minipage}[c]{0.3\hsize} % 作歌作曲は上から 3 割
    \begin{flushright} % 下寄せにする
        % begin name
         % 作歌・作曲者
        % end name
    \end{flushright}
\end{minipage}
%%%%% タイトルと作者 ここまで %%%%%
% % end header

% begin body
\vspace{1.5em} % タイトル, 作者と歌詞の間に隙間を設ける
\newcommand{\linespace}{0.5em} % 行間の設定
\newcommand{\blocksize}{0.5\hsize} % 段組間の設定
%%%%% 歌詞 ここから %%%%%
% begin lilycs
\begin{enumerate} % 番号の箇条書き ここから
    \begin{minipage}[c]{\blocksize}
    
        \vspace{\linespace}
        \item
        % 1.
        \ruby{霞}{}は\ruby{深}{}し\ruby{鶉月}{}の\\
        \ruby{空尚寒}{}き\ruby{眺}{}めかな\\
        いざたて\ruby{林学健男子}{}\\
        \ruby{今}{}こそ\ruby{時}{}ぞ\ruby{奮}{}へ\ruby{友}{}\\
        \ruby{我等一戦此所}{}に\ruby{在}{}り\\
        \ruby{我等一戦此所}{}に\ruby{在}{}り\\
        フレー\ruby{林学}{} フレー\ruby{林学}{}\\
        フレー フレー フレー
        
        \vspace{\linespace}
        \item
        % 2.
        \ruby{涙}{}にむせび\ruby{血}{}をすすり\\
        \ruby{忍}{}ぶ\ruby{思}{}は\ruby{幾星霜}{}\\
        \ruby{今}{}ぞあふるる\ruby{熱血}{}の\\
        \ruby{胸}{}にもゆるをはた\ruby{如何}{}に\\
        はらせ\ruby{積}{}るその\ruby{恨}{}\\
        はらせ\ruby{積}{}るその\ruby{恨}{}
        
        \vspace{\linespace}
        \item
        % 3.
        \ruby{立}{}つべき\ruby{時}{}は\ruby{今}{}ぞ\ruby{今}{}\\
        \ruby{決死}{}の\ruby{意氣}{}よ\ruby{來}{}れ\ruby{友}{}\\
        \ruby{見}{}よや\ruby{男子}{}の\ruby{本領}{}を\\
        \ruby{聞}{}け\ruby{林学}{}の\ruby{其}{}のいさを\\
        \ruby{立}{}つべき\ruby{時}{}は\ruby{今}{}ぞ\ruby{今}{}\\
        \ruby{立}{}つべき\ruby{時}{}は\ruby{今}{}ぞ\ruby{今}{}
    
    \end{minipage}
\end{enumerate} % 番号の箇条書き ここまで
% end lilycs
%%%%% 歌詞 ここまで %%%%%
% end body

\end{document}

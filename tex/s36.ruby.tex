\documentclass[10pt,b5j]{tarticle} % B6 縦書き
% \documentclass[10pt,b5j]{tarticle} % B6 縦書き
\AtBeginDvi{\special{papersize=128mm,182mm}} % B6 用用紙サイズ
\usepackage{otf} % Unicode で字を入力するのに必要なパッケージ
\usepackage[size=b6j]{bxpapersize} % B6 用紙サイズを指定
\usepackage[dvipdfmx]{graphicx} % 画像を挿入するためのパッケージ
\usepackage[dvipdfmx]{color} % 色をつけるためのパッケージ
\usepackage{pxrubrica} % ルビを振るためのパッケージ
\usepackage{multicol} % 複数段組を作るためのパッケージ
\setlength{\topmargin}{14mm} % 上下方向のマージン
\addtolength{\topmargin}{-1in} % 
\setlength{\oddsidemargin}{11mm} % 左右方向のマージン
\addtolength{\oddsidemargin}{-1in} % 
\setlength{\textwidth}{154mm} % B6 用
\setlength{\textheight}{108mm} % B6 用
\setlength{\headsep}{0mm} % 
\setlength{\headheight}{0mm} % 
\setlength{\topskip}{0mm} % 
\setlength{\parskip}{0pt} % 
\def\labelenumi{\theenumi、} % 箇条書きのフォーマット
\parindent = 0pt % 段落下げしない

 % B6 用テンプレート読み込み

\begin{document}
% begin header
%%%%% タイトルと作者 ここから %%%%%
\begin{minipage}[c]{0.7\hsize} % タイトルは上から 7 割
    \begin{center}
    % begin title
        {\LARGE
            甦えれ白き辛夷よ % タイトルを入れる
        }
        {\small 
            (昭和三十六年寮歌) % 年などを入れる
        }
    % end title
    \end{center}
\end{minipage}
\begin{minipage}[c]{0.3\hsize} % 作歌作曲は上から 3 割
    \begin{flushright} % 下寄せにする
        % begin name
        小川徳人君 作歌\\脇地燗君 作曲 % 作歌・作曲者
        % end name
    \end{flushright}
\end{minipage}
%%%%% タイトルと作者 ここまで %%%%%
% (1 繰り返しなし)
% end header

% begin length
\vspace{1.5em} % タイトル, 作者と歌詞の間に隙間を設ける
\newcommand{\linespace}{0.5em} % 行間の設定
\newcommand{\blocksize}{0.5\hsize} % 段組間の設定
\newcommand{\itemmargin}{6em} % 曲番の位置調整の長さ
% end length
% begin body
%%%%% 歌詞 ここから %%%%%
\begin{enumerate} % 番号の箇条書き ここから
    \setlength{\itemindent}{\itemmargin} % 曲番の位置調整
    \begin{minipage}[c]{\blocksize}
    
        \vspace{\linespace}
        \item~\\
        % 1.
        \ruby{甦}{}えれ\ruby{白}{}き\ruby{辛夷}{}よ\\
        \ruby{吐息}{}なす\ruby{憂悶}{}の\ruby{日}{}も\\
        \ruby{寂莫}{}のまどろみも\ruby{去}{}り\\
        オホーツクの\ruby{水}{}やわらぎて\\
        \ruby{流氷}{}の\ruby{群軋}{}める\ruby{国}{}に\\
        \ruby{彷徨}{}のい\ruby{着}{}きしを\ruby{知}{}る\\
        \ruby{朽葉}{}ぬき\ruby{頭}{}もたげし\ruby{若}{}き\ruby{息吹}{}は\\
        わが\ruby{若}{}き\ruby{日}{}の\ruby{昏迷}{}を\ruby{掻}{}く
        
        \vspace{\linespace}
        \item~\\
        % 2.
        \ruby{濃霧}{}を\ruby{呑}{}み\ruby{大気}{}は\ruby{青}{}む\\
        \ruby{輝}{}ける\ruby{太陽}{}に\ruby{酔}{}い\ruby{痴}{}れて\\
        \ruby{高澄}{}の\ruby{日高}{}の\ruby{峠}{}を\\
        わだつみの\ruby{青}{}をば\ruby{追}{}わん\\
        ああ\ruby{慵}{}げき\ruby{虚}{}を\ruby{破}{}りて\\
        \ruby{筋骨}{}は\ruby{火照}{}に\ruby{燃}{}えぬ\\
        エゾマツの\ruby{深}{}き\ruby{樹林}{}を\ruby{渡}{}る\ruby{雄叫}{}び\\
        わが\ruby{若}{}き\ruby{日}{}の\ruby{胸}{}に\ruby{響}{}かん
        
        \vspace{\linespace}
        \item~\\
        % 3.
        \ruby{眼路渺茫}{}の\ruby{野末遙}{}けき\\
        \ruby{石礫}{}の\ruby{曠野}{}に\ruby{励}{}む\\
        \ruby{先達}{}の\ruby{真情}{}を\ruby{凝}{}らし\\
        \ruby{地}{}の\ruby{熟睡静}{}かに\ruby{温}{}む\\
        \ruby{真紅}{}をはきて\ruby{入日}{}たゆたい\\
        \ruby{颯々}{}とポプラは\ruby{鳴}{}れる\\
        \ruby{友}{}どちの\ruby{組}{}たる\ruby{肩}{}は\ruby{若}{}く\ruby{息}{}づく\\
        わがあすの\ruby{日}{}の\ruby{耕土}{}を\ruby{期}{}して
        
        \vspace{\linespace}
        \item~\\
        % 4.
        \ruby{白皚々}{}と\ruby{六華}{}は\ruby{咲}{}けど\\
        うす\ruby{月}{}は\ruby{雲}{}をどよませ\\
        \ruby{逆巻}{}の\ruby{吹雪}{}は\ruby{狂}{}う\\
        \ruby{邂逅}{}に\ruby{結}{}ぶ\ruby{灯火}{}\\
        \ruby{濃}{}き\ruby{鈍色}{}ににじみそめつも\\
        \ruby{手}{}をとりて\ruby{声}{}を\ruby{落}{}とさじ\\
        \ruby{明晰}{}な\ruby{眼}{}を\ruby{持}{}ちて\ruby{凝視}{}る\ruby{道}{}に\\
        わが\ruby{霹靂}{}の\ruby{痕}{}を\ruby{印}{}さん
    
    \end{minipage}
\end{enumerate} % 番号の箇条書き ここまで
%%%%% 歌詞 ここまで %%%%%
% end body

\end{document}

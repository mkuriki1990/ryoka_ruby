\documentclass[10pt,b5j]{tarticle} % B6 縦書き
% \documentclass[10pt,b5j]{tarticle} % B6 縦書き
\AtBeginDvi{\special{papersize=128mm,182mm}} % B6 用用紙サイズ
\usepackage{otf} % Unicode で字を入力するのに必要なパッケージ
\usepackage[size=b6j]{bxpapersize} % B6 用紙サイズを指定
\usepackage[dvipdfmx]{graphicx} % 画像を挿入するためのパッケージ
\usepackage[dvipdfmx]{color} % 色をつけるためのパッケージ
\usepackage{pxrubrica} % ルビを振るためのパッケージ
\usepackage{plext} % 漢数字の enumerate を使うためのパッケージ
\usepackage{multicol} % 複数段組を作るためのパッケージ
\setlength{\topmargin}{14mm} % 上下方向のマージン
\addtolength{\topmargin}{-1in} % 
\setlength{\oddsidemargin}{11mm} % 左右方向のマージン
\addtolength{\oddsidemargin}{-1in} % 
\setlength{\textwidth}{154mm} % B6 用
\setlength{\textheight}{108mm} % B6 用
\setlength{\headsep}{0mm} % 
\setlength{\headheight}{0mm} % 
\setlength{\topskip}{0mm} % 
\setlength{\parskip}{0pt} % 
\def\theenumi{\Kanji{enumi}} % 箇条書きのフォーマットを漢数字に変更
\parindent = 0pt % 段落下げしない
\pagestyle{empty} % すべてのページ番号を消去
% \renewcommand{\baselinestretch}{0.9} % 行間の倍率
 % B6 用テンプレート読み込み

\begin{document}
% begin header
%%%%% タイトルと作者 ここから %%%%%
\begin{minipage}[c]{0.7\hsize} % タイトルは上から 7 割
    \begin{center}
    % begin title
        {\LARGE
            甦えれ白き辛夷よ % タイトルを入れる
        }
        {\small 
            (昭和三十六年寮歌) % 年などを入れる
        }
    % end title
    \end{center}
\end{minipage}
\begin{minipage}[c]{0.3\hsize} % 作歌作曲は上から 3 割
    \begin{flushright} % 下寄せにする
        % begin name
        小川徳人君 作歌\\脇地燗君 作曲 % 作歌・作曲者
        % end name
    \end{flushright}
\end{minipage}
%%%%% タイトルと作者 ここまで %%%%%
% (1 繰り返しなし)
% end header

% begin length
\vspace{1.5em} % タイトル, 作者と歌詞の間に隙間を設ける
\newcommand{\linespace}{0.5em} % 行間の設定
\newcommand{\blocksize}{0.5\hsize} % 段組間の設定
\newcommand{\itemmargin}{3em} % 曲番の位置調整の長さ
% end length
% begin body
%%%%% 歌詞 ここから %%%%%
\begin{enumerate} % 番号の箇条書き ここから
    \setlength{\itemindent}{\itemmargin} % 曲番の位置調整
    \begin{minipage}[c]{\blocksize}
    
        \vspace{\linespace}
        \item~\\
        % 1.
        \ruby{甦}{よみがえ}えれ\ruby{白}{しろ}き\ruby{辛夷}{こぶし}よ\\
        \ruby{吐息}{といき}なす\ruby{憂悶}{ゆうもん}の\ruby{日}{ひ}も\\
        \ruby{寂}{さび}\ruby{莫}{}のまどろみも\ruby{去}{さ}り\\
        オホーツクの\ruby{水}{みず}やわらぎて\\
        \ruby{流氷}{りゅうひょう}の\ruby{群}{ぐん}\ruby{軋}{きし}める\ruby{国}{くに}に\\
        \ruby{彷徨}{ほうこう}のい\ruby{着}{ぎ}きしを\ruby{知}{し}る\\
        \ruby{朽葉}{くちば}ぬき\ruby{頭}{とう}もたげし\ruby{若}{わか}き\ruby{息吹}{いぶき}は\\
        わが\ruby{若}{わか}き\ruby{日}{ひ}の\ruby{昏迷}{こんめい}を\ruby{掻}{か}く
        
    \end{minipage}
    \begin{minipage}[c]{\blocksize}
        
        \vspace{\linespace}
        \item~\\
        % 2.
        \ruby{濃霧}{のうむ}を\ruby{呑}{の}み\ruby{大気}{たいき}は\ruby{青}{あお}む\\
        \ruby{輝}{かがや}ける\ruby{太陽}{たいよう}に\ruby{酔}{よ}い\ruby{痴}{し}れて\\
        \ruby{高}{こう}\ruby{澄}{きよし}の\ruby{日}{にち}\ruby{高}{こう}の\ruby{峠}{とうげ}を\\
        わだつみの\ruby{青}{あお}をば\ruby{追}{お}わん\\
        ああ\ruby{慵}{}げき\ruby{虚}{うろ}を\ruby{破}{やぶ}りて\\
        \ruby{筋骨}{きんこつ}は\ruby{火照}{ほて}に\ruby{燃}{も}えぬ\\
        エゾマツの\ruby{深}{ふか}き\ruby{樹林}{じゅりん}を\ruby{渡}{わた}る\ruby{雄叫}{おたけ}び\\
        わが\ruby{若}{わか}き\ruby{日}{ひ}の\ruby{胸}{むね}に\ruby{響}{ひび}かん
        
    \end{minipage}
    \begin{minipage}[c]{\blocksize}
        
        \vspace{\linespace}
        \item~\\
        % 3.
        \ruby{眼路}{めじ}\ruby{渺茫}{びょうぼう}の\ruby{野末}{のずえ}\ruby{遙}{}けき\\
        \ruby{石}{いし}\ruby{礫}{つぶて}の\ruby{曠野}{あらの}に\ruby{励}{はげ}む\\
        \ruby{先達}{せんだつ}の\ruby{真情}{しんじょう}を\ruby{凝}{こ}らし\\
        \ruby{地}{ち}の\ruby{熟睡}{じゅくすい}\ruby{静}{しず}かに\ruby{温}{ぬる}む\\
        \ruby{真紅}{しんく}をはきて\ruby{入日}{いりひ}たゆたい\\
        \ruby{颯々}{さっさつ}とポプラは\ruby{鳴}{な}れる\\
        \ruby{友}{とも}どちの\ruby{組}{ぐみ}たる\ruby{肩}{かた}は\ruby{若}{わか}く\ruby{息}{いき}づく\\
        わがあすの\ruby{日}{ひ}の\ruby{耕土}{こうど}を\ruby{期}{き}して
        
    \end{minipage}
    \begin{minipage}[c]{\blocksize}
        
        \vspace{\linespace}
        \item~\\
        % 4.
        \ruby{白}{しろ}\ruby{皚々}{がいがい}と\ruby{六}{ろく}\ruby{華}{はな}は\ruby{咲}{さ}けど\\
        うす\ruby{月}{つき}は\ruby{雲}{くも}をどよませ\\
        \ruby{逆}{ぎゃく}\ruby{巻}{まき}の\ruby{吹雪}{ふぶき}は\ruby{狂}{くる}う\\
        \ruby{邂逅}{かいこう}に\ruby{結}{むす}ぶ\ruby{灯火}{ともしび}\\
        \ruby{濃}{こ}き\ruby{鈍色}{にぶいろ}ににじみそめつも\\
        \ruby{手}{て}をとりて\ruby{声}{こえ}を\ruby{落}{お}とさじ\\
        \ruby{明晰}{めいせき}な\ruby{眼}{め}を\ruby{持}{も}ちて\ruby{凝視}{ぎょうし}る\ruby{道}{みち}に\\
        わが\ruby{霹靂}{}の\ruby{痕}{あと}を\ruby{印}{しる}さん
    
    \end{minipage}
\end{enumerate} % 番号の箇条書き ここまで
%%%%% 歌詞 ここまで %%%%%
% end body

\end{document}

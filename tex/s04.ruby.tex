\documentclass[10pt,b5j]{tarticle} % B6 縦書き
% \documentclass[10pt,b5j]{tarticle} % B6 縦書き
\AtBeginDvi{\special{papersize=128mm,182mm}} % B6 用用紙サイズ
\usepackage{otf} % Unicode で字を入力するのに必要なパッケージ
\usepackage[size=b6j]{bxpapersize} % B6 用紙サイズを指定
\usepackage[dvipdfmx]{graphicx} % 画像を挿入するためのパッケージ
\usepackage[dvipdfmx]{color} % 色をつけるためのパッケージ
\usepackage{pxrubrica} % ルビを振るためのパッケージ
\usepackage{multicol} % 複数段組を作るためのパッケージ
\setlength{\topmargin}{14mm} % 上下方向のマージン
\addtolength{\topmargin}{-1in} % 
\setlength{\oddsidemargin}{11mm} % 左右方向のマージン
\addtolength{\oddsidemargin}{-1in} % 
\setlength{\textwidth}{154mm} % B6 用
\setlength{\textheight}{108mm} % B6 用
\setlength{\headsep}{0mm} % 
\setlength{\headheight}{0mm} % 
\setlength{\topskip}{0mm} % 
\setlength{\parskip}{0pt} % 
\def\labelenumi{\theenumi、} % 箇条書きのフォーマット
\parindent = 0pt % 段落下げしない

 % B6 用テンプレート読み込み

\begin{document}
% begin header
%%%%% タイトルと作者 ここから %%%%%
\begin{minipage}[c]{0.7\hsize} % タイトルは上から 7 割
    \begin{center}
    % begin title
        {\LARGE
            黒潮鳴れる % タイトルを入れる
        }
        {\small 
            (昭和4年寮歌) % 年などを入れる
        }
    % end title
    \end{center}
\end{minipage}
\begin{minipage}[c]{0.3\hsize} % 作歌作曲は上から 3 割
    \begin{flushright} % 下寄せにする
        % begin name
        須田政美君 作歌\\森忠文君 作曲 % 作歌・作曲者
        % end name
    \end{flushright}
\end{minipage}
%%%%% タイトルと作者 ここまで %%%%%
% (1,2,3,6 繰り返しなし)
% end header

% begin body
\vspace{1.5em} % タイトル, 作者と歌詞の間に隙間を設ける
\newcommand{\linespace}{0.5em} % 行間の設定
\newcommand{\blocksize}{0.5\hsize} % 段組間の設定
%%%%% 歌詞 ここから %%%%%
% begin lilycs
\begin{enumerate} % 番号の箇条書き ここから
    \begin{minipage}[c]{\blocksize}
    
        \vspace{\linespace}
        \item
        % 1.
        \ruby{黒潮鳴}{}れる\ruby{滄海越}{}えて\\
        \ruby{際限無}{}き\ruby{春}{}を\ruby{北州}{}に\ruby{訪}{}ふ\\
        \ruby{原始}{}の\ruby{大森}{}に\ruby{八光揺}{}ぎ\\
        \ruby{若草}{}の\ruby{曠野}{}に\ruby{羊群遊}{}ぶ
        
        \vspace{\linespace}
        \item
        % 2.
        \ruby{情懐}{}は\ruby{朧月}{}に\ruby{仄}{}かに\ruby{薫}{}る\\
        アカシヤの\ruby{白花慕}{}ひて\ruby{歩}{}む\\
        \ruby{恋}{}ふる\ruby{往昔}{}の\ruby{静寂}{}けき\ruby{名残}{}り\\
        \ruby{古塔}{}にひびく\ruby{懐}{}しき\ruby{鐘}{}
        
        \vspace{\linespace}
        \item
        % 3.
        \ruby{紅光}{}うすくエルムに\ruby{映}{}えて\\
        \ruby{草笛}{}かそかに\ruby{牧場}{}にながる\\
        \ruby{漂泊}{}らひ\ruby{行}{}ける\ruby{白雲影仰}{}ぎ\\
        \ruby{無心}{}の\ruby{若人}{}らは\ruby{緑}{}に\ruby{臥}{}せり
        
        \vspace{\linespace}
        \item
        % 4.
        \ruby{果無}{}き\ruby{憧憬銀河}{}に\ruby{寄}{}せて\\
        \ruby{玻璃永劫}{}の\ruby{清}{}き\ruby{夜空}{}を\\
        \ruby{神秘}{}の\ruby{皓翼声}{}なく\ruby{衝}{}ちつ\\
        \ruby{我等}{}が\ruby{高夢}{}は\ruby{流}{}れゆくかな
        
        \vspace{\linespace}
        \item
        % 5.
        \ruby{淋}{}しき\ruby{風声}{}に\ruby{銀雪}{}は\ruby{乱}{}れつ\\
        \ruby{大空鳴}{}りて\ruby{渾瞑}{}く\ruby{暮}{}れゆく\\
        \ruby{燦}{}めく\ruby{灯影常春}{}の\ruby{謳歌}{}\\
        \ruby{血潮}{}と\ruby{共}{}に\ruby{尚湧}{}き\ruby{立}{}てり
        
        \vspace{\linespace}
        \item
        % 6.
        \ruby{久遠}{}の\ruby{絢夢}{}はうづもれゆきて\\
        \ruby{哀愁時}{}にしづかに\ruby{来}{}れど\\
        \ruby{雄}{}き「\ruby{自然}{}」と「\ruby{血潮}{}」の\ruby{人}{}は\\
        \ruby{楡陵}{}に\ruby{永}{}くうつくしく\ruby{立}{}つ
    
    \end{minipage}
\end{enumerate} % 番号の箇条書き ここまで
% end lilycs
%%%%% 歌詞 ここまで %%%%%
% end body

\end{document}

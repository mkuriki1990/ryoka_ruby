\documentclass[10pt,b5j]{tarticle} % B6 縦書き
% \documentclass[10pt,b5j]{tarticle} % B6 縦書き
\AtBeginDvi{\special{papersize=128mm,182mm}} % B6 用用紙サイズ
\usepackage{otf} % Unicode で字を入力するのに必要なパッケージ
\usepackage[size=b6j]{bxpapersize} % B6 用紙サイズを指定
\usepackage[dvipdfmx]{graphicx} % 画像を挿入するためのパッケージ
\usepackage[dvipdfmx]{color} % 色をつけるためのパッケージ
\usepackage{pxrubrica} % ルビを振るためのパッケージ
\usepackage{multicol} % 複数段組を作るためのパッケージ
\setlength{\topmargin}{14mm} % 上下方向のマージン
\addtolength{\topmargin}{-1in} % 
\setlength{\oddsidemargin}{11mm} % 左右方向のマージン
\addtolength{\oddsidemargin}{-1in} % 
\setlength{\textwidth}{154mm} % B6 用
\setlength{\textheight}{108mm} % B6 用
\setlength{\headsep}{0mm} % 
\setlength{\headheight}{0mm} % 
\setlength{\topskip}{0mm} % 
\setlength{\parskip}{0pt} % 
\def\labelenumi{\theenumi、} % 箇条書きのフォーマット
\parindent = 0pt % 段落下げしない

 % B6 用テンプレート読み込み

\begin{document}
% begin header
%%%%% タイトルと作者 ここから %%%%%
\begin{minipage}[c]{0.7\hsize} % タイトルは上から 7 割
    \begin{center}
    % begin title
        {\LARGE
            楡陵を去る日に % タイトルを入れる
        }
        {\small 
            (北大創基百周年記念東京同窓会寄贈寮歌) % 年などを入れる
        }
    % end title
    \end{center}
\end{minipage}
\begin{minipage}[c]{0.3\hsize} % 作歌作曲は上から 3 割
    \begin{flushright} % 下寄せにする
        % begin name
        小倉行雄君 作歌\\矢野哲憲君 作曲 % 作歌・作曲者
        % end name
    \end{flushright}
\end{minipage}
%%%%% タイトルと作者 ここまで %%%%%
% % end header

% begin length
\vspace{1.5em} % タイトル, 作者と歌詞の間に隙間を設ける
\newcommand{\linespace}{0.5em} % 行間の設定
\newcommand{\blocksize}{0.5\hsize} % 段組間の設定
\newcommand{\itemmargin}{3em} % 曲番の位置調整の長さ
% end length
% begin body
%%%%% 歌詞 ここから %%%%%
\begin{enumerate} % 番号の箇条書き ここから
    \setlength{\itemindent}{\itemmargin} % 曲番の位置調整
    \begin{minipage}[c]{\blocksize}
    
        \vspace{\linespace}
        \item~\\
        % 1.
        \ruby{楡陵}{}を\ruby{去}{}る\ruby{日}{}に\ruby{涙}{}せし\\
        \ruby{寮友}{}よ\ruby{何処}{}に\ruby{去}{}り\ruby{行}{}ける\\
        \ruby{百年}{}を\ruby{祝}{}う\ruby{記念祭}{}\\
        \ruby{寮友}{}よ\ruby{集}{}わん\ruby{楡}{}かげに
        
    \end{minipage}
    \begin{minipage}[c]{\blocksize}
        
        \vspace{\linespace}
        \item~\\
        % 2.
        \ruby{南溟}{}の\ruby{空北}{}の\ruby{果}{}て\\
        \ruby{散}{}りにし\ruby{寮友}{}も\ruby{集}{}いけむ\\
        \ruby{石狩}{}の\ruby{野}{}に\ruby{秋}{}たけし\\
        \ruby{今宵寮祭}{}のかゞり\ruby{火}{}に
        
    \end{minipage}
    \begin{minipage}[c]{\blocksize}
        
        \vspace{\linespace}
        \item~\\
        % 3.
        \ruby{原子}{}の\ruby{杜}{}は\ruby{拓}{}かれて\\
        \ruby{真白}{}き\ruby{帽}{}の\ruby{三条消}{}え\\
        \ruby{僅}{}かに\ruby{残}{}る\ruby{楡}{}ポプラ\\
        \ruby{郭公}{}の\ruby{声}{}あわれなり
        
    \end{minipage}
    \begin{minipage}[c]{\blocksize}
        
        \vspace{\linespace}
        \item~\\
        % 4.
        \ruby{生命}{}の\ruby{春}{}の\ruby{短}{}かさよ\\
        \ruby{藻岩}{}のみどり\ruby{紅葉}{}して\\
        \ruby{仰}{}ぐ\ruby{手稲}{}の\ruby{白雪}{}に\\
        あゝ\ruby{青春}{}の\ruby{日}{}よいづこ
        
    \end{minipage}
    \begin{minipage}[c]{\blocksize}
        
        \vspace{\linespace}
        \item~\\
        % 5.
        \ruby{旧}{}き\ruby{寮舎}{}を\ruby{訪}{}い\ruby{往}{}けば\\
        \ruby{若}{}き\ruby{男子}{}の\ruby{寮歌流}{}る\\
        \ruby{共}{}に\ruby{誦}{}びぬかの\ruby{寮歌}{}を\\
        \ruby{頬}{}を\ruby{流}{}るる\ruby{涙}{}かな
        
    \end{minipage}
    \begin{minipage}[c]{\blocksize}
        
        \vspace{\linespace}
        \item~\\
        % 6.
        \ruby{自治}{}の\ruby{燈}{}かかげ\ruby{百年}{}の\\
        \ruby{青史}{}をかざす\ruby{楡陵}{}の\ruby{宴}{}\\
        \ruby{迪}{}を\ruby{恵}{}めし\ruby{幾千}{}が\\
        こぞりて\ruby{謳歌}{}う\ruby{魂}{}の\ruby{寮歌}{}
    
    \end{minipage}
\end{enumerate} % 番号の箇条書き ここまで
%%%%% 歌詞 ここまで %%%%%
% end body

\end{document}

\documentclass[10pt,b5j]{tarticle} % B6 縦書き
% \documentclass[10pt,b5j]{tarticle} % B6 縦書き
\AtBeginDvi{\special{papersize=128mm,182mm}} % B6 用用紙サイズ
\usepackage{otf} % Unicode で字を入力するのに必要なパッケージ
\usepackage[size=b6j]{bxpapersize} % B6 用紙サイズを指定
\usepackage[dvipdfmx]{graphicx} % 画像を挿入するためのパッケージ
\usepackage[dvipdfmx]{color} % 色をつけるためのパッケージ
\usepackage{pxrubrica} % ルビを振るためのパッケージ
\usepackage{plext} % 漢数字の enumerate を使うためのパッケージ
\usepackage{multicol} % 複数段組を作るためのパッケージ
\setlength{\topmargin}{14mm} % 上下方向のマージン
\addtolength{\topmargin}{-1in} % 
\setlength{\oddsidemargin}{11mm} % 左右方向のマージン
\addtolength{\oddsidemargin}{-1in} % 
\setlength{\textwidth}{154mm} % B6 用
\setlength{\textheight}{108mm} % B6 用
\setlength{\headsep}{0mm} % 
\setlength{\headheight}{0mm} % 
\setlength{\topskip}{0mm} % 
\setlength{\parskip}{0pt} % 
\def\theenumi{\Kanji{enumi}} % 箇条書きのフォーマットを漢数字に変更
\parindent = 0pt % 段落下げしない
\pagestyle{empty} % すべてのページ番号を消去
% \renewcommand{\baselinestretch}{0.9} % 行間の倍率
 % B6 用テンプレート読み込み

\begin{document}
% begin header
%%%%% タイトルと作者 ここから %%%%%
\begin{minipage}[c]{0.7\hsize} % タイトルは上から 7 割
    \begin{center}
    % begin title
        {\LARGE
            花を褥 % タイトルを入れる
        }
        {\small 
            (大正七年寮歌) % 年などを入れる
        }
    % end title
    \end{center}
\end{minipage}
\begin{minipage}[c]{0.3\hsize} % 作歌作曲は上から 3 割
    \begin{flushright} % 下寄せにする
        % begin name
        松本五六君 作歌\\峰秀雄君 作曲 % 作歌・作曲者
        % end name
    \end{flushright}
\end{minipage}
%%%%% タイトルと作者 ここまで %%%%%
% (1 了あり)
% end header

% begin body
\vspace{1.5em} % タイトル, 作者と歌詞の間に隙間を設ける
\newcommand{\linespace}{0.5em} % 行間の設定
\newcommand{\blocksize}{0.5\hsize} % 段組間の設定
%%%%% 歌詞 ここから %%%%%
% begin lilycs
\begin{enumerate} % 番号の箇条書き ここから
    \begin{minipage}[c]{\blocksize}
    
        \vspace{\linespace}
        \item
        % 1.
        \ruby{花}{}を\ruby{褥}{}の\ruby{草枕}{}\\
        \ruby{霞}{}に\ruby{暮}{}るる\ruby{野辺}{}の\ruby{夢}{}\\
        ローマの\ruby{晨}{}ナイルの\ruby{夕}{}べ\\
        \ruby{栄華}{}よあはれ\ruby{夢}{}の\ruby{跡}{}\\
        \ruby{傾}{}く\ruby{月}{}に\ruby{猶心}{}せず\\
        \ruby{驕奢}{}に\ruby{酔}{}ひし\ruby{人々}{}の\\
        \ruby{惰睡}{}を\ruby{破}{}る\ruby{雄叫}{}や\\
        \ruby{健児義}{}を\ruby{取}{}る\ruby{北}{}の\ruby{国}{}
        
        \vspace{\linespace}
        \item
        % 2.
        \ruby{世}{}の\ruby{敗頽}{}に\ruby{神怒}{}り \\
        \ruby{南}{}の\ruby{洋}{}に\ruby{濤}{}さわぎ\\
        \ruby{腥風荒}{}さび\ruby{天日暗}{}く \\
        \ruby{欧亜}{}の\ruby{文華影消}{}えぬ\\
        \ruby{堯舜去}{}りて\ruby{妖雲霽}{}れず\\
        \ruby{江河氾濫}{}れて\ruby{末濁}{}る\\
        \ruby{暴虐無道幾年}{}ぞ\\
        \ruby{吾等立}{}つべき\ruby{時}{}ぞ\ruby{今}{}
        
        \vspace{\linespace}
        \item
        % 3.
        \ruby{煙霞曠}{}しき\ruby{石狩}{}の\\
        \ruby{荒野}{}に\ruby{立}{}ちて\ruby{嘯}{}けば\\
        \ruby{霜枯}{}れ\ruby{吹雪}{}く\ruby{原始}{}の\ruby{森}{}に\\
        エルゼの\ruby{歌}{}も\ruby{微}{}かなり\\
        \ruby{手稲}{}の\ruby{嶺}{}に\ruby{夕陽淡}{}く\\
        \ruby{宇宙}{}の\ruby{神秘畏}{}れみて\\
        \ruby{雄々}{}しき\ruby{自然}{}に\ruby{育}{}まれ\\
        \ruby{雲呼}{}び\ruby{沖天}{}に\ruby{翼搏}{}たん
        
        \vspace{\linespace}
        \item
        % 4.
        \ruby{春}{}の\ruby{女神}{}の\ruby{訪}{}れに\\
        \ruby{花}{}は\ruby{綻}{}び\ruby{鳥謡}{}ひ\\
        \ruby{翠}{}の\ruby{樹蔭}{}に\ruby{鈴蘭香}{}り\\
        \ruby{露}{}の\ruby{涼}{}しき\ruby{夏}{}の\ruby{朝}{}\\
        \ruby{時雨}{}に\ruby{漂}{}ふ\ruby{牧場}{}の\ruby{紅葉}{}\\
        \ruby{白雪晴}{}るる\ruby{冬}{}の\ruby{景}{}\\
        \ruby{書読}{}む\ruby{歳}{}は\ruby{豊平}{}の\\
        \ruby{時}{}の\ruby{流}{}れに\ruby{恵}{}あり
        
        \vspace{\linespace}
        \item
        % 5.
        \ruby{薫}{}る\ruby{春風}{}アカシヤの\\
        \ruby{情操床}{}しき\ruby{若人}{}が\\
        \ruby{崇}{}き\ruby{希望}{}の\ruby{象徴}{}と\ruby{仰}{}ぐ\\
        \ruby{聖}{}き\ruby{北斗}{}の\ruby{瞬}{}に\\
        \ruby{真理}{}の\ruby{道}{}の\ruby{暗示}{}を\ruby{索}{}め\\
        \ruby{純}{}しき\ruby{玉}{}の\ruby{緒一百}{}を\\
        \ruby{一}{}つに\ruby{懸}{}けて\ruby{結}{}びたる\\
        \ruby{自治}{}の\ruby{基礎動}{}きなし
        
        \vspace{\linespace}
        \item
        % 6.
        \ruby{烏兎流光}{}の\ruby{移}{}ろひて\\
        \ruby{昔}{}の\ruby{友}{}は\ruby{在}{}はさねど\\
        \ruby{十三年}{}の\ruby{光栄}{}ある\ruby{歴史}{}\\
        \ruby{護}{}り\ruby{伝}{}へて\ruby{極限無}{}し\\
        \ruby{自由}{}の\ruby{大旆正義}{}の\ruby{剣}{}\\
        \ruby{天下}{}の\ruby{民}{}を\ruby{済}{}ふべし\\
        \ruby{戦}{}の\ruby{場}{}の\ruby{首途}{}とて\\
        \ruby{宴}{}の\ruby{盃}{}いざ\ruby{汲}{}まん
    
    \end{minipage}
\end{enumerate} % 番号の箇条書き ここまで
% end lilycs
%%%%% 歌詞 ここまで %%%%%
% end body

\end{document}

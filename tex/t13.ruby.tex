\documentclass[10pt,b5j]{tarticle} % B6 縦書き
% \documentclass[10pt,b5j]{tarticle} % B6 縦書き
\AtBeginDvi{\special{papersize=128mm,182mm}} % B6 用用紙サイズ
\usepackage{otf} % Unicode で字を入力するのに必要なパッケージ
\usepackage[size=b6j]{bxpapersize} % B6 用紙サイズを指定
\usepackage[dvipdfmx]{graphicx} % 画像を挿入するためのパッケージ
\usepackage[dvipdfmx]{color} % 色をつけるためのパッケージ
\usepackage{pxrubrica} % ルビを振るためのパッケージ
\usepackage{plext} % 漢数字の enumerate を使うためのパッケージ
\usepackage{multicol} % 複数段組を作るためのパッケージ
\setlength{\topmargin}{14mm} % 上下方向のマージン
\addtolength{\topmargin}{-1in} % 
\setlength{\oddsidemargin}{11mm} % 左右方向のマージン
\addtolength{\oddsidemargin}{-1in} % 
\setlength{\textwidth}{154mm} % B6 用
\setlength{\textheight}{108mm} % B6 用
\setlength{\headsep}{0mm} % 
\setlength{\headheight}{0mm} % 
\setlength{\topskip}{0mm} % 
\setlength{\parskip}{0pt} % 
\def\theenumi{\Kanji{enumi}} % 箇条書きのフォーマットを漢数字に変更
\parindent = 0pt % 段落下げしない
\pagestyle{empty} % すべてのページ番号を消去
% \renewcommand{\baselinestretch}{0.9} % 行間の倍率
 % B6 用テンプレート読み込み

\begin{document}
% begin header
%%%%% タイトルと作者 ここから %%%%%
\begin{minipage}[c]{0.7\hsize} % タイトルは上から 7 割
    \begin{center}
    % begin title
        {\LARGE
            茫々はるか % タイトルを入れる
        }
        {\small 
            (大正13年寮歌) % 年などを入れる
        }
    % end title
    \end{center}
\end{minipage}
\begin{minipage}[c]{0.3\hsize} % 作歌作曲は上から 3 割
    \begin{flushright} % 下寄せにする
        % begin name
        高野芳雄君 作歌\\神島辰雄君 作曲 % 作歌・作曲者
        % end name
    \end{flushright}
\end{minipage}
%%%%% タイトルと作者 ここまで %%%%%
% (1,2,3 了あり)
% end header

% begin body
\vspace{1.5em} % タイトル, 作者と歌詞の間に隙間を設ける
\newcommand{\linespace}{0.5em} % 行間の設定
\newcommand{\blocksize}{0.5\hsize} % 段組間の設定
%%%%% 歌詞 ここから %%%%%
% begin lilycs
\begin{enumerate} % 番号の箇条書き ここから
    \begin{minipage}[c]{\blocksize}
    
        \vspace{\linespace}
        \item
        % 1.
        \ruby{茫々}{}はるかに\ruby{緑}{}に\ruby{炎}{}えて\\
        \ruby{石狩原頭美}{}の\ruby{香}{}に\ruby{酔}{}えば\\
        \ruby{高鳴}{}りあふるる\ruby{若人}{}の\ruby{血}{}や\\
        ああこの\ruby{霊}{}の\ruby{憧}{}れの\ruby{地}{}に\\
        \ruby{曙光}{}に\ruby{輝}{}く\ruby{黎明告}{}ぐる\\
        \ruby{鐘}{}を\ruby{撞}{}かばや
        
        \vspace{\linespace}
        \item
        % 2.
        \ruby{真紅}{}に\ruby{熱}{}せる\ruby{入陽}{}は\ruby{沈}{}み\\
        \ruby{黙思}{}の\ruby{歩}{}みを\ruby{運}{}ぶ\ruby{夕宵}{}は\\
        エルムの\ruby{繁}{}みの\ruby{梢透}{}かして\\
        \ruby{夕映流}{}るる\ruby{黄色}{}の\ruby{彩}{}に\\
        \ruby{生命}{}の\ruby{窓}{}をば\ruby{疾}{}く\ruby{開}{}け\ruby{放}{}ち\\
        \ruby{霊気吸}{}はずや
        
        \vspace{\linespace}
        \item
        % 3.
        \ruby{地平}{}の\ruby{際涯}{}によしや\ruby{吾等}{}の\\
        \ruby{感激}{}は\ruby{沈}{}めど\ruby{彼方}{}はるかに\\
        \ruby{思索}{}の\ruby{曠野}{}は\ruby{尽}{}せぬなれば\\
        \ruby{石狩河岸}{}に\ruby{友}{}よ\ruby{佇}{}み\\
        \ruby{野生}{}の\ruby{律}{}べの\ruby{秘奥}{}を\ruby{求}{}め\\
        \ruby{真理}{}を\ruby{聴}{}かん
        
        \vspace{\linespace}
        \item
        % 4.
        \ruby{寒風荒}{}びて\ruby{吹雪吹}{}く\ruby{夜}{}も\\
        \ruby{楡林}{}に\ruby{洩}{}れたる\ruby{四寮}{}の\ruby{燭光}{}\\
        \ruby{生命}{}ぞまたたき\ruby{青春}{}の\ruby{日}{}の\\
        \ruby{灯累}{}りて\ruby{永遠}{}に\ruby{輝}{}く\\
        ああ\ruby{其}{}の\ruby{灯}{}かげに\ruby{霊}{}と\ruby{血潮}{}の\\
        \ruby{籠}{}められしかな
        
        \vspace{\linespace}
        \item
        % 5.
        \ruby{昔}{}を\ruby{偲}{}べば\ruby{吾等}{}が\ruby{寮}{}は\\
        \ruby{原始}{}の\ruby{茂森}{}に\ruby{生}{}める\ruby{自然児}{}\\
        \ruby{不断}{}の\ruby{船路}{}に\ruby{彼岸}{}めがけて\\
        \ruby{自治}{}への\ruby{歩}{}みは\ruby{十九星霜}{}\\
        \ruby{自由}{}の\ruby{栄}{}に\ruby{友}{}よ\ruby{奏}{}でん\\
        \ruby{平和}{}の\ruby{序曲}{}
    
    \end{minipage}
\end{enumerate} % 番号の箇条書き ここまで
% end lilycs
%%%%% 歌詞 ここまで %%%%%
% end body

\end{document}

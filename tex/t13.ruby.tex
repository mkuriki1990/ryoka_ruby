\documentclass[10pt,b5j]{tarticle} % B6 縦書き
% \documentclass[10pt,b5j]{tarticle} % B6 縦書き
\AtBeginDvi{\special{papersize=128mm,182mm}} % B6 用用紙サイズ
\usepackage{otf} % Unicode で字を入力するのに必要なパッケージ
\usepackage[size=b6j]{bxpapersize} % B6 用紙サイズを指定
\usepackage[dvipdfmx]{graphicx} % 画像を挿入するためのパッケージ
\usepackage[dvipdfmx]{color} % 色をつけるためのパッケージ
\usepackage{pxrubrica} % ルビを振るためのパッケージ
\usepackage{multicol} % 複数段組を作るためのパッケージ
\setlength{\topmargin}{14mm} % 上下方向のマージン
\addtolength{\topmargin}{-1in} % 
\setlength{\oddsidemargin}{11mm} % 左右方向のマージン
\addtolength{\oddsidemargin}{-1in} % 
\setlength{\textwidth}{154mm} % B6 用
\setlength{\textheight}{108mm} % B6 用
\setlength{\headsep}{0mm} % 
\setlength{\headheight}{0mm} % 
\setlength{\topskip}{0mm} % 
\setlength{\parskip}{0pt} % 
\def\labelenumi{\theenumi、} % 箇条書きのフォーマット
\parindent = 0pt % 段落下げしない

 % B6 用テンプレート読み込み

\begin{document}
% begin header
%%%%% タイトルと作者 ここから %%%%%
\begin{minipage}[c]{0.7\hsize} % タイトルは上から 7 割
    \begin{center}
    % begin title
        {\LARGE
            茫々はるか % タイトルを入れる
        }
        {\small 
            (大正十三年寮歌) % 年などを入れる
        }
    % end title
    \end{center}
\end{minipage}
\begin{minipage}[c]{0.3\hsize} % 作歌作曲は上から 3 割
    \begin{flushright} % 下寄せにする
        % begin name
        高野芳雄君 作歌\\神島辰雄君 作曲 % 作歌・作曲者
        % end name
    \end{flushright}
\end{minipage}
%%%%% タイトルと作者 ここまで %%%%%
% (1,2,3 了あり)
% end header

% begin length
\vspace{1.5em} % タイトル, 作者と歌詞の間に隙間を設ける
\newcommand{\linespace}{0.5em} % 行間の設定
\newcommand{\blocksize}{0.5\hsize} % 段組間の設定
\newcommand{\itemmargin}{3em} % 曲番の位置調整の長さ
% end length
% begin body
%%%%% 歌詞 ここから %%%%%
\begin{enumerate} % 番号の箇条書き ここから
    \setlength{\itemindent}{\itemmargin} % 曲番の位置調整
    \begin{minipage}[c]{\blocksize}
    
        \vspace{\linespace}
        \item~\\
        % 1.
        \ruby{茫々}{ぼうぼう}はるかに\ruby{緑}{みどり}に\ruby{炎}{ほのお}えて\\
        \ruby{石狩}{いしかり}\ruby{原頭}{げんとう}\ruby{美}{び}の\ruby{香}{こう}に\ruby{酔}{よ}えば\\
        \ruby{高鳴}{たかな}りあふるる\ruby{若人}{わこうど}の\ruby{血}{ち}や\\
        ああこの\ruby{霊}{れい}の\ruby{憧}{あこが}れの\ruby{地}{ち}に\\
        \ruby{曙光}{しょこう}に\ruby{輝}{かがや}く\ruby{黎明}{れいめい}\ruby{告}{つげ}ぐる\\
        \ruby{鐘}{かね}を\ruby{撞}{つ}かばや
        
    \end{minipage}
    \begin{minipage}[c]{\blocksize}
        
        \vspace{\linespace}
        \item~\\
        % 2.
        \ruby{真紅}{しんく}に\ruby{熱}{ねつ}せる\ruby{入}{にゅう}\ruby{陽}{ひ}は\ruby{沈}{しず}み\\
        \ruby{黙}{だま}\ruby{思}{おもい}の\ruby{歩}{あゆ}みを\ruby{運}{はこ}ぶ\ruby{夕}{ゆう}\ruby{宵}{よい}は\\
        エルムの\ruby{繁}{しげ}みの\ruby{梢}{こずえ}\ruby{透}{す}かして\\
        \ruby{夕映}{ゆうばえ}\ruby{流}{りゅう}るる\ruby{黄色}{きいろ}の\ruby{彩}{いろどり}に\\
        \ruby{生命}{せいめい}の\ruby{窓}{まど}をば\ruby{疾}{と}く\ruby{開}{あ}け\ruby{放}{}ち\\
        \ruby{霊気}{れいき}\ruby{吸}{}はずや
        
    \end{minipage}
    \begin{minipage}[c]{\blocksize}
        
        \vspace{\linespace}
        \item~\\
        % 3.
        \ruby{地平}{ちへい}の\ruby{際涯}{さいがい}によしや\ruby{吾}{われ}\ruby{等}{とう}の\\
        \ruby{感激}{かんげき}は\ruby{沈}{しず}めど\ruby{彼方}{かなた}はるかに\\
        \ruby{思索}{しさく}の\ruby{曠野}{あらの}は\ruby{尽}{つく}せぬなれば\\
        \ruby{石狩}{いしかり}\ruby{河岸}{かわぎし}に\ruby{友}{とも}よ\ruby{佇}{たたず}み\\
        \ruby{野生}{やせい}の\ruby{律}{りつ}べの\ruby{秘奥}{ひおう}を\ruby{求}{もと}め\\
        \ruby{真理}{しんり}を\ruby{聴}{き}かん
        
    \end{minipage}
    \begin{minipage}[c]{\blocksize}
        
        \vspace{\linespace}
        \item~\\
        % 4.
        \ruby{寒風}{かんぷう}\ruby{荒}{すさ}びて\ruby{\ruby{吹}{ふ}雪}{ふぶき}\ruby{吹}{ふ}く\ruby{夜}{よる}も\\
        \ruby{楡林}{ゆりん}に\ruby{洩}{も}れたる\ruby{四}{よん}\ruby{寮}{りょう}の\ruby{燭光}{しょっこう}\\
        \ruby{生命}{せいめい}ぞまたたき\ruby{青春}{せいしゅん}の\ruby{日}{ひ}の\\
        \ruby{灯}{あかり}\ruby{累}{るい}りて\ruby{永遠}{えいえん}に\ruby{輝}{かがや}く\\
        ああ\ruby{其}{そ}の\ruby{灯}{あかり}かげに\ruby{霊}{れい}と\ruby{血潮}{ちしお}の\\
        \ruby{籠}{かご}められしかな
        
    \end{minipage}
    \begin{minipage}[c]{\blocksize}
        
        \vspace{\linespace}
        \item~\\
        % 5.
        \ruby{昔}{むかし}を\ruby{偲}{しの}べば\ruby{吾}{われ}\ruby{等}{とう}が\ruby{寮}{りょう}は\\
        \ruby{原始}{げんし}の\ruby{茂森}{しげもり}に\ruby{生}{う}める\ruby{自然}{しぜん}\ruby{児}{じ}\\
        \ruby{不断}{ふだん}の\ruby{船路}{ふなじ}に\ruby{彼岸}{ひがん}めがけて\\
        \ruby{自治}{じち}への\ruby{歩}{あゆ}みは\ruby{十九}{じゅっく}\ruby{星霜}{せいそう}\\
        \ruby{自由}{じゆう}の\ruby{栄}{さかえ}に\ruby{友}{とも}よ\ruby{奏}{かな}でん\\
        \ruby{平和}{へいわ}の\ruby{序曲}{じょきょく}
    
    \end{minipage}
\end{enumerate} % 番号の箇条書き ここまで
%%%%% 歌詞 ここまで %%%%%
% end body

\end{document}

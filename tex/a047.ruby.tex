\documentclass[10pt,b5j]{tarticle} % B6 縦書き
% \documentclass[10pt,b5j]{tarticle} % B6 縦書き
\AtBeginDvi{\special{papersize=128mm,182mm}} % B6 用用紙サイズ
\usepackage{otf} % Unicode で字を入力するのに必要なパッケージ
\usepackage[size=b6j]{bxpapersize} % B6 用紙サイズを指定
\usepackage[dvipdfmx]{graphicx} % 画像を挿入するためのパッケージ
\usepackage[dvipdfmx]{color} % 色をつけるためのパッケージ
\usepackage{pxrubrica} % ルビを振るためのパッケージ
\usepackage{multicol} % 複数段組を作るためのパッケージ
\setlength{\topmargin}{14mm} % 上下方向のマージン
\addtolength{\topmargin}{-1in} % 
\setlength{\oddsidemargin}{11mm} % 左右方向のマージン
\addtolength{\oddsidemargin}{-1in} % 
\setlength{\textwidth}{154mm} % B6 用
\setlength{\textheight}{108mm} % B6 用
\setlength{\headsep}{0mm} % 
\setlength{\headheight}{0mm} % 
\setlength{\topskip}{0mm} % 
\setlength{\parskip}{0pt} % 
\def\labelenumi{\theenumi、} % 箇条書きのフォーマット
\parindent = 0pt % 段落下げしない

 % B6 用テンプレート読み込み

\begin{document}
% begin header
%%%%% タイトルと作者 ここから %%%%%
\begin{minipage}[c]{0.7\hsize} % タイトルは上から 7 割
    \begin{center}
    % begin title
        {\LARGE
            医専振興歌 % タイトルを入れる
        }
        {\small 
            (昭和十四年七月) % 年などを入れる
        }
    % end title
    \end{center}
\end{minipage}
\begin{minipage}[c]{0.3\hsize} % 作歌作曲は上から 3 割
    \begin{flushright} % 下寄せにする
        % begin name
        高見浩也君 作歌\\向井弘君 作曲 % 作歌・作曲者
        % end name
    \end{flushright}
\end{minipage}
%%%%% タイトルと作者 ここまで %%%%%
% % end header

% begin length
\vspace{1.5em} % タイトル, 作者と歌詞の間に隙間を設ける
\newcommand{\linespace}{0.5em} % 行間の設定
\newcommand{\blocksize}{0.5\hsize} % 段組間の設定
\newcommand{\itemmargin}{3em} % 曲番の位置調整の長さ
% end length
% begin body
%%%%% 歌詞 ここから %%%%%
\begin{enumerate} % 番号の箇条書き ここから
    \setlength{\itemindent}{\itemmargin} % 曲番の位置調整
    \begin{minipage}[c]{\blocksize}
    
        \vspace{\linespace}
        \item~\\
        % 1.
        \ruby{東雲}{しののめ}\ruby{遥}{はる}か\ruby{北海}{ほっかい}の\\
        \ruby{朝日}{あさひ}に\ruby{燃}{もゆる}ゆる\ruby{石狩}{いしかり}に\\
        \ruby{呱々}{ここ}の\ruby{声}{こえ}をあげしより\\
        \ruby{進}{すす}んでやまぬ\ruby{若人}{わこうど}よ\\
        \ruby{示}{しめ}せ!\\
        \ruby{東亜}{とうあ}にその\ruby{意}{い}\ruby{氣}{}を!
        
    \end{minipage}
    \begin{minipage}[c]{\blocksize}
        
        \vspace{\linespace}
        \item~\\
        % 2.
        \ruby{鈴蘭}{すずらん}\ruby{薫}{かお}る\ruby{学}{まな}び\ruby{舎}{や}に\\
        \ruby{萠}{めぐむ}ゆる\ruby{緑}{みどり}は\ruby{春}{はる}の\ruby{呼吸}{こきゅう}\\
        \ruby{白樺}{しらかんば}\ruby{聳}{}ゆる\ruby{山々}{やまやま}を\\
        \ruby{彩}{いろど}る\ruby{錦}{にしき}は\ruby{秋}{あき}の\ruby{衣}{ころも}\\
        \ruby{磨}{みが}け!\\
        \ruby{学徒}{がくと}よその\ruby{腕}{うで}!
        
    \end{minipage}
    \begin{minipage}[c]{\blocksize}
        
        \vspace{\linespace}
        \item~\\
        % 3.
        \ruby{理想}{りそう}は\ruby{高}{たか}し\ruby{北斗星}{ほくとせい}\\
        \ruby{虚空}{こくう}\ruby{遥}{はる}かに\ruby{瞬}{まばた}いて\\
        \ruby{医専}{いせん}\ruby{健児}{けんじ}の\ruby{行}{い}く\ruby{処}{しょ}\\
        \ruby{如何}{いか}なる\ruby{山}{やま}も\ruby{踏}{ふ}み\ruby{越}{}えて\\
        \ruby{迎}{むかい}へよ!\\
        \ruby{栄}{さかえ}ある\ruby{大}{だい}\ruby{歴史}{れきし}!
    
    \end{minipage}
\end{enumerate} % 番号の箇条書き ここまで
%%%%% 歌詞 ここまで %%%%%
% end body

\end{document}

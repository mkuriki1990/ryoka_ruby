\documentclass[10pt,b5j]{tarticle} % B6 縦書き
% \documentclass[10pt,b5j]{tarticle} % B6 縦書き
\AtBeginDvi{\special{papersize=128mm,182mm}} % B6 用用紙サイズ
\usepackage{otf} % Unicode で字を入力するのに必要なパッケージ
\usepackage[size=b6j]{bxpapersize} % B6 用紙サイズを指定
\usepackage[dvipdfmx]{graphicx} % 画像を挿入するためのパッケージ
\usepackage[dvipdfmx]{color} % 色をつけるためのパッケージ
\usepackage{pxrubrica} % ルビを振るためのパッケージ
\usepackage{multicol} % 複数段組を作るためのパッケージ
\setlength{\topmargin}{14mm} % 上下方向のマージン
\addtolength{\topmargin}{-1in} % 
\setlength{\oddsidemargin}{11mm} % 左右方向のマージン
\addtolength{\oddsidemargin}{-1in} % 
\setlength{\textwidth}{154mm} % B6 用
\setlength{\textheight}{108mm} % B6 用
\setlength{\headsep}{0mm} % 
\setlength{\headheight}{0mm} % 
\setlength{\topskip}{0mm} % 
\setlength{\parskip}{0pt} % 
\def\labelenumi{\theenumi、} % 箇条書きのフォーマット
\parindent = 0pt % 段落下げしない

 % B6 用テンプレート読み込み

\begin{document}
% begin header
%%%%% タイトルと作者 ここから %%%%%
\begin{minipage}[c]{0.7\hsize} % タイトルは上から 7 割
    \begin{center}
    % begin title
        {\LARGE
            花繚乱の % タイトルを入れる
        }
        {\small 
            (昭和三十二年寮歌) % 年などを入れる
        }
    % end title
    \end{center}
\end{minipage}
\begin{minipage}[c]{0.3\hsize} % 作歌作曲は上から 3 割
    \begin{flushright} % 下寄せにする
        % begin name
        前島一淑君 作歌・作曲 % 作歌・作曲者
        % end name
    \end{flushright}
\end{minipage}
%%%%% タイトルと作者 ここまで %%%%%
% (1,2,3,4,5 了あり)
% end header

% begin body
\vspace{1.5em} % タイトル, 作者と歌詞の間に隙間を設ける
\newcommand{\linespace}{0.5em} % 行間の設定
\newcommand{\blocksize}{0.5\hsize} % 段組間の設定
%%%%% 歌詞 ここから %%%%%
% begin lilycs
\begin{enumerate} % 番号の箇条書き ここから
    \begin{minipage}[c]{\blocksize}
    
        \vspace{\linespace}
        \item
        % 1.
        \ruby{花繚乱}{}の\ruby{夢}{}に\ruby{酔}{}い\\
        \ruby{地}{}の\ruby{囁}{}きの\ruby{音}{}に\ruby{伏}{}せば\\
        \ruby{草淑々}{}の\ruby{声}{}すなり
        
        \vspace{\linespace}
        \item
        % 2.
        \ruby{夜光流}{}るる\ruby{芝草}{}や\\
        \ruby{辛夷}{}の\ruby{花}{}の\ruby{香}{}に\ruby{迷}{}う\\
        \ruby{遠}{}き\ruby{憧}{}れ\ruby{逝}{}にし\ruby{日}{}よ
        
        \vspace{\linespace}
        \item
        % 3.
        \ruby{窓辺}{}に\ruby{招}{}く\ruby{幻}{}の\\
        \ruby{影}{}にあくがれ\ruby{彷徨}{}えば\\
        \ruby{森}{}に\ruby{桂}{}の\ruby{火}{}は\ruby{燃}{}えぬ
        
        \vspace{\linespace}
        \item
        % 4.
        \ruby{今紅}{}の\ruby{篝火}{}よ\\
        \ruby{裸形}{}の\ruby{友}{}は\ruby{肩組}{}みて\\
        \ruby{去}{}り\ruby{行}{}く\ruby{青春}{}を\\
        \ruby{惜}{}しむかな
        
        \vspace{\linespace}
        \item
        % 5.
        \ruby{静寂甦}{}りぬ\ruby{春}{}の\ruby{宵}{}\\
        \ruby{銀漢}{}の\ruby{下希望}{}なる\\
        \ruby{支笏}{}の\ruby{湖}{}に\ruby{星}{}は\ruby{飛}{}ぶ
    
    \end{minipage}
\end{enumerate} % 番号の箇条書き ここまで
% end lilycs
%%%%% 歌詞 ここまで %%%%%
% end body

\end{document}

\documentclass[10pt,b5j]{tarticle} % B6 縦書き
% \documentclass[10pt,b5j]{tarticle} % B6 縦書き
\AtBeginDvi{\special{papersize=128mm,182mm}} % B6 用用紙サイズ
\usepackage{otf} % Unicode で字を入力するのに必要なパッケージ
\usepackage[size=b6j]{bxpapersize} % B6 用紙サイズを指定
\usepackage[dvipdfmx]{graphicx} % 画像を挿入するためのパッケージ
\usepackage[dvipdfmx]{color} % 色をつけるためのパッケージ
\usepackage{pxrubrica} % ルビを振るためのパッケージ
\usepackage{plext} % 漢数字の enumerate を使うためのパッケージ
\usepackage{multicol} % 複数段組を作るためのパッケージ
\setlength{\topmargin}{14mm} % 上下方向のマージン
\addtolength{\topmargin}{-1in} % 
\setlength{\oddsidemargin}{11mm} % 左右方向のマージン
\addtolength{\oddsidemargin}{-1in} % 
\setlength{\textwidth}{154mm} % B6 用
\setlength{\textheight}{108mm} % B6 用
\setlength{\headsep}{0mm} % 
\setlength{\headheight}{0mm} % 
\setlength{\topskip}{0mm} % 
\setlength{\parskip}{0pt} % 
\def\theenumi{\Kanji{enumi}} % 箇条書きのフォーマットを漢数字に変更
\parindent = 0pt % 段落下げしない
\pagestyle{empty} % すべてのページ番号を消去
% \renewcommand{\baselinestretch}{0.9} % 行間の倍率
 % B6 用テンプレート読み込み

\begin{document}
% begin header
%%%%% タイトルと作者 ここから %%%%%
\begin{minipage}[c]{0.7\hsize} % タイトルは上から 7 割
    \begin{center}
    % begin title
        {\LARGE
            古城の春は % タイトルを入れる
        }
        {\small 
            (昭和七年寮歌) % 年などを入れる
        }
    % end title
    \end{center}
\end{minipage}
\begin{minipage}[c]{0.3\hsize} % 作歌作曲は上から 3 割
    \begin{flushright} % 下寄せにする
        % begin name
        大槻均君 作歌\\中村小弥太君 作曲 % 作歌・作曲者
        % end name
    \end{flushright}
\end{minipage}
%%%%% タイトルと作者 ここまで %%%%%
% (1,2,5 繰り返しなし)
% end header

% begin length
\vspace{1.5em} % タイトル, 作者と歌詞の間に隙間を設ける
\newcommand{\linespace}{0.5em} % 行間の設定
\newcommand{\blocksize}{0.5\hsize} % 段組間の設定
\newcommand{\itemmargin}{3em} % 曲番の位置調整の長さ
% end length
% begin body
%%%%% 歌詞 ここから %%%%%
\begin{enumerate} % 番号の箇条書き ここから
    \setlength{\itemindent}{\itemmargin} % 曲番の位置調整
    \begin{minipage}[c]{\blocksize}
    
        \vspace{\linespace}
        \item~\\
        % 1.
        \ruby{古城}{こじょう}の\ruby{春}{はる}は\ruby{老}{お}い\ruby{易}{やす}く\\
        \ruby{延}{のべ}\ruby{齢}{よわい}\ruby{草}{そう}の\ruby{名}{な}に\ruby{問}{とい}へど\\
        \ruby{流転}{るてん}の\ruby{法}{ほう}は\ruby{断}{た}ち\ruby{難}{むずか}し\\
        \ruby{友}{とも}よエルムの\ruby{鐘}{かね}を\ruby{聴}{き}け\\
        \ruby{再建}{さいけん}の\ruby{秋}{あき}\ruby{程}{ほど}なけん\\
        ペルアスペラと\ruby{鳴}{な}り\ruby{響}{ひび}く
        
    \end{minipage}
    \begin{minipage}[c]{\blocksize}
        
        \vspace{\linespace}
        \item~\\
        % 2.
        \ruby{今}{こん}\ruby{移}{うつ}り\ruby{来}{きた}し\ruby{原始}{げんし}\ruby{林}{りん}の\ruby{蔭}{かげ}\\
        \ruby{宿}{やど}るは\ruby{未}{いま}だ\ruby{浅}{あさ}けれど\\
        \ruby{契}{ちぎり}は\ruby{深}{ふか}き\ruby{三}{さん}\ruby{百}{ひゃく}の\\
        \ruby{心}{こころ}を\ruby{交}{か}わすこの\ruby{宴}{うたげ}\\
        \ruby{暁}{あかつき}かけていざ\ruby{撞}{つ}かん\\
        アドアストラの\ruby{自治}{じち}の\ruby{鐘}{かね}
        
    \end{minipage}
    \begin{minipage}[c]{\blocksize}
        
        \vspace{\linespace}
        \item~\\
        % 3.
        \ruby{妖雲}{よううん}\ruby{西}{にし}に\ruby{漾}{}へど\\
        \ruby{視}{み}よ\ruby{落日}{らくじつ}の\ruby{悠々}{ゆうゆう}と\\
        \ruby{大地}{だいち}を\ruby{旋}{}り\ruby{淪}{}むかな\\
        \ruby{眠}{ねむ}る\ruby{此}{}の\ruby{城}{しろ}\ruby{吾}{われ}も\ruby{亦}{また}\\
        \ruby{醒}{さ}めての\ruby{生命}{せいめい}\ruby{培}{}はん\\
        \ruby{四大}{しだい}の\ruby{荒}{すさ}び\ruby{明日}{あした}あれば
        
    \end{minipage}
    \begin{minipage}[c]{\blocksize}
        
        \vspace{\linespace}
        \item~\\
        % 4.
        \ruby{厳寒}{げんかん}\ruby{凍}{こお}る\ruby{極北}{きょくほく}に\\
        \ruby{霧}{きり}\ruby{立}{た}ち\ruby{騒}{さわ}ぐ\ruby{曙}{あけぼの}の\\
        \ruby{光}{ひかり}を\ruby{担}{にな}うて\ruby{起}{た}たんとき\\
        \ruby{際涯}{さいがい}もなく\ruby{寄}{よ}せ\ruby{返}{かえ}す\\
        \ruby{世紀}{せいき}の\ruby{波}{なみ}\ruby{濤}{}は\ruby{狂}{きょう}へども\\
        \ruby{既倒}{きとう}にかへす\ruby{力}{ちから}あり
        
    \end{minipage}
    \begin{minipage}[c]{\blocksize}
        
        \vspace{\linespace}
        \item~\\
        % 5.
        \ruby{竜}{りゅう}\ruby{舵}{かじ}\ruby{岸}{がん}\ruby{打}{う}つ\ruby{大洋}{たいよう}の\\
        \ruby{今}{こん}\ruby{人生}{じんせい}の\ruby{船出}{ふなで}かな\\
        \ruby{白帆}{しらほ}\ruby{高}{たか}くはためきて\\
        \ruby{正気}{しょうき}をはらむ\ruby{若人}{わこうど}の\\
        \ruby{理想}{りそう}の\ruby{船}{ふね}は\ruby{不壊}{ふえ}にして\\
        さかまく\ruby{苦海}{くかい}を\ruby{永遠}{えいえん}に\ruby{航}{こう}く
    
    \end{minipage}
\end{enumerate} % 番号の箇条書き ここまで
%%%%% 歌詞 ここまで %%%%%
% end body

\end{document}

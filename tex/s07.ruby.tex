\documentclass[10pt,b5j]{tarticle} % B6 縦書き
% \documentclass[10pt,b5j]{tarticle} % B6 縦書き
\AtBeginDvi{\special{papersize=128mm,182mm}} % B6 用用紙サイズ
\usepackage{otf} % Unicode で字を入力するのに必要なパッケージ
\usepackage[size=b6j]{bxpapersize} % B6 用紙サイズを指定
\usepackage[dvipdfmx]{graphicx} % 画像を挿入するためのパッケージ
\usepackage[dvipdfmx]{color} % 色をつけるためのパッケージ
\usepackage{pxrubrica} % ルビを振るためのパッケージ
\usepackage{multicol} % 複数段組を作るためのパッケージ
\setlength{\topmargin}{14mm} % 上下方向のマージン
\addtolength{\topmargin}{-1in} % 
\setlength{\oddsidemargin}{11mm} % 左右方向のマージン
\addtolength{\oddsidemargin}{-1in} % 
\setlength{\textwidth}{154mm} % B6 用
\setlength{\textheight}{108mm} % B6 用
\setlength{\headsep}{0mm} % 
\setlength{\headheight}{0mm} % 
\setlength{\topskip}{0mm} % 
\setlength{\parskip}{0pt} % 
\def\labelenumi{\theenumi、} % 箇条書きのフォーマット
\parindent = 0pt % 段落下げしない

 % B6 用テンプレート読み込み

\begin{document}
% begin header
%%%%% タイトルと作者 ここから %%%%%
\begin{minipage}[c]{0.7\hsize} % タイトルは上から 7 割
    \begin{center}
    % begin title
        {\LARGE
            古城の春は % タイトルを入れる
        }
        {\small 
            (昭和7年寮歌) % 年などを入れる
        }
    % end title
    \end{center}
\end{minipage}
\begin{minipage}[c]{0.3\hsize} % 作歌作曲は上から 3 割
    \begin{flushright} % 下寄せにする
        % begin name
        大槻均君 作歌\\中村小弥太君 作曲 % 作歌・作曲者
        % end name
    \end{flushright}
\end{minipage}
%%%%% タイトルと作者 ここまで %%%%%
% (1,2,5 繰り返しなし)
% end header

% begin body
\vspace{1.5em} % タイトル, 作者と歌詞の間に隙間を設ける
\newcommand{\linespace}{0.5em} % 行間の設定
\newcommand{\blocksize}{0.5\hsize} % 段組間の設定
%%%%% 歌詞 ここから %%%%%
% begin lilycs
\begin{enumerate} % 番号の箇条書き ここから
    \begin{minipage}[c]{\blocksize}
    
        \vspace{\linespace}
        \item
        % 1.
        \ruby{古城}{}の\ruby{春}{}は\ruby{老}{}い\ruby{易}{}く\\
        \ruby{延齢草}{}の\ruby{名}{}に\ruby{問}{}へど\\
        \ruby{流転}{}の\ruby{法}{}は\ruby{断}{}ち\ruby{難}{}し\\
        \ruby{友}{}よエルムの\ruby{鐘}{}を\ruby{聴}{}け\\
        \ruby{再建}{}の\ruby{秋程}{}なけん\\
        ペルアスペラと\ruby{鳴}{}り\ruby{響}{}く
        
        \vspace{\linespace}
        \item
        % 2.
        \ruby{今移}{}り\ruby{来}{}し\ruby{原始林}{}の\ruby{蔭}{}\\
        \ruby{宿}{}るは\ruby{未}{}だ\ruby{浅}{}けれど\\
        \ruby{契}{}は\ruby{深}{}き\ruby{三百}{}の\\
        \ruby{心}{}を\ruby{交}{}わすこの\ruby{宴}{}\\
        \ruby{暁}{}かけていざ\ruby{撞}{}かん\\
        アドアストラの\ruby{自治}{}の\ruby{鐘}{}
        
        \vspace{\linespace}
        \item
        % 3.
        \ruby{妖雲西}{}に\ruby{漾}{}へど\\
        \ruby{視}{}よ\ruby{落日}{}の\ruby{悠々}{}と\\
        \ruby{大地}{}を\ruby{旋}{}り\ruby{淪}{}むかな\\
        \ruby{眠}{}る\ruby{此}{}の\ruby{城吾}{}も\ruby{亦}{}\\
        \ruby{醒}{}めての\ruby{生命培}{}はん\\
        \ruby{四大}{}の\ruby{荒}{}び\ruby{明日}{}あれば
        
        \vspace{\linespace}
        \item
        % 4.
        \ruby{厳寒凍}{}る\ruby{極北}{}に\\
        \ruby{霧立}{}ち\ruby{騒}{}ぐ\ruby{曙}{}の\\
        \ruby{光}{}を\ruby{担}{}うて\ruby{起}{}たんとき\\
        \ruby{際涯}{}もなく\ruby{寄}{}せ\ruby{返}{}す\\
        \ruby{世紀}{}の\ruby{波濤}{}は\ruby{狂}{}へども\\
        \ruby{既倒}{}にかへす\ruby{力}{}あり
        
        \vspace{\linespace}
        \item
        % 5.
        \ruby{竜舵岸打}{}つ\ruby{大洋}{}の\\
        \ruby{今人生}{}の\ruby{船出}{}かな\\
        \ruby{白帆高}{}くはためきて\\
        \ruby{正気}{}をはらむ\ruby{若人}{}の\\
        \ruby{理想}{}の\ruby{船}{}は\ruby{不壊}{}にして\\
        さかまく\ruby{苦海}{}を\ruby{永遠}{}に\ruby{航}{}く
    
    \end{minipage}
\end{enumerate} % 番号の箇条書き ここまで
% end lilycs
%%%%% 歌詞 ここまで %%%%%
% end body

\end{document}

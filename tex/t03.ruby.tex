\documentclass[10pt,b5j]{tarticle} % B6 縦書き
% \documentclass[10pt,b5j]{tarticle} % B6 縦書き
\AtBeginDvi{\special{papersize=128mm,182mm}} % B6 用用紙サイズ
\usepackage{otf} % Unicode で字を入力するのに必要なパッケージ
\usepackage[size=b6j]{bxpapersize} % B6 用紙サイズを指定
\usepackage[dvipdfmx]{graphicx} % 画像を挿入するためのパッケージ
\usepackage[dvipdfmx]{color} % 色をつけるためのパッケージ
\usepackage{pxrubrica} % ルビを振るためのパッケージ
\usepackage{multicol} % 複数段組を作るためのパッケージ
\setlength{\topmargin}{14mm} % 上下方向のマージン
\addtolength{\topmargin}{-1in} % 
\setlength{\oddsidemargin}{11mm} % 左右方向のマージン
\addtolength{\oddsidemargin}{-1in} % 
\setlength{\textwidth}{154mm} % B6 用
\setlength{\textheight}{108mm} % B6 用
\setlength{\headsep}{0mm} % 
\setlength{\headheight}{0mm} % 
\setlength{\topskip}{0mm} % 
\setlength{\parskip}{0pt} % 
\def\labelenumi{\theenumi、} % 箇条書きのフォーマット
\parindent = 0pt % 段落下げしない

 % B6 用テンプレート読み込み

\begin{document}
% begin header
%%%%% タイトルと作者 ここから %%%%%
\begin{minipage}[c]{0.7\hsize} % タイトルは上から 7 割
    \begin{center}
    % begin title
        {\LARGE
            我が運命こそ % タイトルを入れる
        }
        {\small 
            (大正三年寮歌) % 年などを入れる
        }
    % end title
    \end{center}
\end{minipage}
\begin{minipage}[c]{0.3\hsize} % 作歌作曲は上から 3 割
    \begin{flushright} % 下寄せにする
        % begin name
        樋口桜五君 作歌\\赤木顕次君 作曲 % 作歌・作曲者
        % end name
    \end{flushright}
\end{minipage}
%%%%% タイトルと作者 ここまで %%%%%
% (1,2,3,4 了あり)
% end header

% begin length
\vspace{1.5em} % タイトル, 作者と歌詞の間に隙間を設ける
\newcommand{\linespace}{0.5em} % 行間の設定
\newcommand{\blocksize}{0.5\hsize} % 段組間の設定
\newcommand{\itemmargin}{3em} % 曲番の位置調整の長さ
% end length
% begin body
%%%%% 歌詞 ここから %%%%%
\begin{enumerate} % 番号の箇条書き ここから
    \setlength{\itemindent}{\itemmargin} % 曲番の位置調整
    \begin{minipage}[c]{\blocksize}
    
        \vspace{\linespace}
        \item~\\
        % 1.
        \ruby{我}{}が\ruby{運命}{}こそ\ruby{青渦}{}わける\\
        \ruby{千}{}ひろの\ruby{海}{}の\ruby{真珠取}{}り\\
        \ruby{美想}{}にあこがるる\ruby{身}{}は\\
        \ruby{驕楽}{}の\ruby{春}{}に\ruby{酔}{}ひしれて\\
        \ruby{戯}{}る\ruby{人}{}を\ruby{夢}{}とはみつつ\\
        \ruby{逆}{}まく\ruby{波}{}を\ruby{闡}{}きゆく
        
    \end{minipage}
    \begin{minipage}[c]{\blocksize}
        
        \vspace{\linespace}
        \item~\\
        % 2.
        \ruby{永遠}{}に\ruby{華}{}さく\ruby{水底}{}ふかく\\
        \ruby{神秘}{}の\ruby{巌}{}に\ruby{嫦娥}{}の\\
        \ruby{露}{}のしづくの\ruby{真珠}{}またま\\
        \ruby{掌}{}に\ruby{獲}{}し\ruby{光栄}{}と\ruby{喜悦}{}と\\
        \ruby{七重}{}の\ruby{潮}{}の\ruby{妙音}{}にひびく\\
        \ruby{美珠}{}こそわれの\ruby{生命}{}なれ
        
    \end{minipage}
    \begin{minipage}[c]{\blocksize}
        
        \vspace{\linespace}
        \item~\\
        % 3.
        \ruby{薫}{}る\ruby{樹陰}{}に\ruby{花仄}{}みえて\\
        \ruby{朧}{}おぼろの\ruby{春}{}の\ruby{宵}{}\\
        \ruby{一壺}{}の\ruby{酒}{}の\ruby{汲}{}む\ruby{夢淡}{}く\\
        \ruby{心}{}の\ruby{酔}{}に\ruby{舞歌}{}を\\
        \ruby{社会高}{}くしらべ\ruby{祝}{}はむ\\
        \ruby{君瑞祥}{}の\ruby{歳}{}なれや
        
    \end{minipage}
    \begin{minipage}[c]{\blocksize}
        
        \vspace{\linespace}
        \item~\\
        % 4.
        \ruby{彩雲低}{}く\ruby{恵}{}の\ruby{家}{}に\\
        \ruby{幸漂蕩}{}ひてゆく\ruby{水}{}や\\
        \ruby{姿}{}うるほす\ruby{柳}{}の\ruby{萠黄}{}\\
        \ruby{契}{}りゆかしき\ruby{春鳥}{}の\\
        \ruby{団欒}{}の\ruby{音}{}をばうつし\ruby{伝}{}へむ\\
        \ruby{遠}{}くはるけき\ruby{師}{}の\ruby{君}{}に
    
    \end{minipage}
\end{enumerate} % 番号の箇条書き ここまで
%%%%% 歌詞 ここまで %%%%%
% end body

\end{document}

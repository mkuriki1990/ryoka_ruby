\documentclass[10pt,b5j]{tarticle} % B6 縦書き
% \documentclass[10pt,b5j]{tarticle} % B6 縦書き
\AtBeginDvi{\special{papersize=128mm,182mm}} % B6 用用紙サイズ
\usepackage{otf} % Unicode で字を入力するのに必要なパッケージ
\usepackage[size=b6j]{bxpapersize} % B6 用紙サイズを指定
\usepackage[dvipdfmx]{graphicx} % 画像を挿入するためのパッケージ
\usepackage[dvipdfmx]{color} % 色をつけるためのパッケージ
\usepackage{pxrubrica} % ルビを振るためのパッケージ
\usepackage{plext} % 漢数字の enumerate を使うためのパッケージ
\usepackage{multicol} % 複数段組を作るためのパッケージ
\setlength{\topmargin}{14mm} % 上下方向のマージン
\addtolength{\topmargin}{-1in} % 
\setlength{\oddsidemargin}{11mm} % 左右方向のマージン
\addtolength{\oddsidemargin}{-1in} % 
\setlength{\textwidth}{154mm} % B6 用
\setlength{\textheight}{108mm} % B6 用
\setlength{\headsep}{0mm} % 
\setlength{\headheight}{0mm} % 
\setlength{\topskip}{0mm} % 
\setlength{\parskip}{0pt} % 
\def\theenumi{\Kanji{enumi}} % 箇条書きのフォーマットを漢数字に変更
\parindent = 0pt % 段落下げしない
\pagestyle{empty} % すべてのページ番号を消去
% \renewcommand{\baselinestretch}{0.9} % 行間の倍率
 % B6 用テンプレート読み込み

\begin{document}
% begin header
%%%%% タイトルと作者 ここから %%%%%
\begin{minipage}[c]{0.7\hsize} % タイトルは上から 7 割
    \begin{center}
    % begin title
        {\LARGE
            我が運命こそ % タイトルを入れる
        }
        {\small 
            (大正三年寮歌) % 年などを入れる
        }
    % end title
    \end{center}
\end{minipage}
\begin{minipage}[c]{0.3\hsize} % 作歌作曲は上から 3 割
    \begin{flushright} % 下寄せにする
        % begin name
        樋口桜五君 作歌\\赤木顕次君 作曲 % 作歌・作曲者
        % end name
    \end{flushright}
\end{minipage}
%%%%% タイトルと作者 ここまで %%%%%
% (1,2,3,4 了あり)
% end header

% begin length
\vspace{1.5em} % タイトル, 作者と歌詞の間に隙間を設ける
\newcommand{\linespace}{0.5em} % 行間の設定
\newcommand{\blocksize}{0.5\hsize} % 段組間の設定
\newcommand{\itemmargin}{3em} % 曲番の位置調整の長さ
% end length
% begin body
%%%%% 歌詞 ここから %%%%%
\begin{enumerate} % 番号の箇条書き ここから
    \setlength{\itemindent}{\itemmargin} % 曲番の位置調整
    \begin{minipage}[c]{\blocksize}
    
        \vspace{\linespace}
        \item~\\
        % 1.
        \ruby{我}{わ}が\ruby{運命}{うんめい}こそ\ruby{青}{あお}\ruby{渦}{うず}わける\\
        \ruby{千}{ち}ひろの\ruby{海}{うみ}の\ruby{真珠}{しんじゅ}\ruby{取}{と}り\\
        \ruby{美想}{みこと}にあこがるる\ruby{身}{み}は\\
        \ruby{驕}{おご}\ruby{楽}{らく}の\ruby{春}{はる}に\ruby{酔}{}ひしれて\\
        \ruby{戯}{}る\ruby{人}{ひと}を\ruby{夢}{ゆめ}とはみつつ\\
        \ruby{逆}{ぎゃく}まく\ruby{波}{なみ}を\ruby{闡}{}きゆく
        
    \end{minipage}
    \begin{minipage}[c]{\blocksize}
        
        \vspace{\linespace}
        \item~\\
        % 2.
        \ruby{永遠}{えいえん}に\ruby{華}{はな}さく\ruby{水}{さくすい}\ruby{底}{そこ}ふかく\\
        \ruby{神秘}{しんぴ}の\ruby{巌}{いわお}に\ruby{嫦娥}{じょうが}の\\
        \ruby{露}{ろ}のしづくの\ruby{真珠}{しんじゅ}またま\\
        \ruby{掌}{てのひら}に\ruby{獲}{え}し\ruby{光栄}{こうえい}と\ruby{喜悦}{きえつ}と\\
        \ruby{七重}{ななえ}の\ruby{潮}{しお}の\ruby{妙音}{みょうおん}にひびく\\
        \ruby{美珠}{みじゅ}こそわれの\ruby{生命}{いのち}なれ
        
    \end{minipage}
    \begin{minipage}[c]{\blocksize}
        
        \vspace{\linespace}
        \item~\\
        % 3.
        \ruby{薫}{かお}る\ruby{樹陰}{じゅいん}に\ruby{花}{はな}\ruby{仄}{}みえて\\
        \ruby{朧}{おぼろ}おぼろの\ruby{春}{はる}の\ruby{宵}{よい}\\
        \ruby{一}{いち}\ruby{壺}{つぼ}の\ruby{酒}{さけ}の\ruby{汲}{く}む\ruby{夢}{ゆめ}\ruby{淡}{あわ}く\\
        \ruby{心}{こころ}の\ruby{酔}{よい}に\ruby{舞歌}{まいか}を\\
        \ruby{社会}{しゃかい}\ruby{高}{たか}くしらべ\ruby{祝}{しゅく}はむ\\
        \ruby{君}{きみ}\ruby{瑞祥}{ずいしょう}の\ruby{歳}{とし}なれや
        
    \end{minipage}
    \begin{minipage}[c]{\blocksize}
        
        \vspace{\linespace}
        \item~\\
        % 4.
        \ruby{彩雲}{さいうん}\ruby{低}{ひく}く\ruby{恵}{めぐみ}の\ruby{家}{いえ}に\\
        \ruby{幸}{こう}\ruby{漂蕩}{ひょうとう}ひてゆく\ruby{水}{みず}や\\
        \ruby{姿}{すがた}うるほす\ruby{柳}{やなぎ}の\ruby{萠}{めぐむ}\ruby{黄}{き}\\
        \ruby{契}{ちぎ}りゆかしき\ruby{春}{はる}\ruby{鳥}{とり}の\\
        \ruby{団欒}{だんらん}の\ruby{音}{おと}をばうつし\ruby{伝}{でん}へむ\\
        \ruby{遠}{とお}くはるけき\ruby{師}{し}の\ruby{君}{きみ}に
    
    \end{minipage}
\end{enumerate} % 番号の箇条書き ここまで
%%%%% 歌詞 ここまで %%%%%
% end body

\end{document}

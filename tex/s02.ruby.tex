\documentclass[10pt,b5j]{tarticle} % B6 縦書き
% \documentclass[10pt,b5j]{tarticle} % B6 縦書き
\AtBeginDvi{\special{papersize=128mm,182mm}} % B6 用用紙サイズ
\usepackage{otf} % Unicode で字を入力するのに必要なパッケージ
\usepackage[size=b6j]{bxpapersize} % B6 用紙サイズを指定
\usepackage[dvipdfmx]{graphicx} % 画像を挿入するためのパッケージ
\usepackage[dvipdfmx]{color} % 色をつけるためのパッケージ
\usepackage{pxrubrica} % ルビを振るためのパッケージ
\usepackage{multicol} % 複数段組を作るためのパッケージ
\setlength{\topmargin}{14mm} % 上下方向のマージン
\addtolength{\topmargin}{-1in} % 
\setlength{\oddsidemargin}{11mm} % 左右方向のマージン
\addtolength{\oddsidemargin}{-1in} % 
\setlength{\textwidth}{154mm} % B6 用
\setlength{\textheight}{108mm} % B6 用
\setlength{\headsep}{0mm} % 
\setlength{\headheight}{0mm} % 
\setlength{\topskip}{0mm} % 
\setlength{\parskip}{0pt} % 
\def\labelenumi{\theenumi、} % 箇条書きのフォーマット
\parindent = 0pt % 段落下げしない

 % B6 用テンプレート読み込み

\begin{document}
% begin header
%%%%% タイトルと作者 ここから %%%%%
\begin{minipage}[c]{0.7\hsize} % タイトルは上から 7 割
    \begin{center}
    % begin title
        {\LARGE
            蒼空高く翔らんと % タイトルを入れる
        }
        {\small 
            (昭和二年寮歌) % 年などを入れる
        }
    % end title
    \end{center}
\end{minipage}
\begin{minipage}[c]{0.3\hsize} % 作歌作曲は上から 3 割
    \begin{flushright} % 下寄せにする
        % begin name
        土井恒喜君 作歌\\長谷川吉郎君 作曲 % 作歌・作曲者
        % end name
    \end{flushright}
\end{minipage}
%%%%% タイトルと作者 ここまで %%%%%
% (1,2,3,4,5,6 了あり)
% end header

% begin length
\vspace{1.5em} % タイトル, 作者と歌詞の間に隙間を設ける
\newcommand{\linespace}{0.5em} % 行間の設定
\newcommand{\blocksize}{0.5\hsize} % 段組間の設定
\newcommand{\itemmargin}{3em} % 曲番の位置調整の長さ
% end length
% begin body
%%%%% 歌詞 ここから %%%%%
\begin{enumerate} % 番号の箇条書き ここから
    \setlength{\itemindent}{\itemmargin} % 曲番の位置調整
    \begin{minipage}[c]{\blocksize}
    
        \vspace{\linespace}
        \item~\\
        % 1.
        \ruby{蒼空高}{}く\ruby{翔}{}らむと\\
        \ruby{暫}{}しやすらふ\ruby{楡}{}の\ruby{蔭}{}\\
        \ruby{力}{}は\ruby{胸}{}に\ruby{溢}{}れつつ\\
        \ruby{翼}{}つくろふ\ruby{思}{}かな
        
    \end{minipage}
    \begin{minipage}[c]{\blocksize}
        
        \vspace{\linespace}
        \item~\\
        % 2.
        \ruby{朝曠野}{}の\ruby{露}{}を\ruby{吸}{}ひ\\
        \ruby{夕北斗}{}の\ruby{囁}{}きに\\
        \ruby{驚}{}き\ruby{瞠}{}る\ruby{幼鵬}{}の\\
        \ruby{清}{}き\ruby{眸君見}{}ずや
        
    \end{minipage}
    \begin{minipage}[c]{\blocksize}
        
        \vspace{\linespace}
        \item~\\
        % 3.
        うら\ruby{若}{}き\ruby{日}{}の\ruby{悦}{}びを\\
        はかなきものと\ruby{誰}{}かいふ\\
        \ruby{理想}{}の\ruby{潮湧}{}き\ruby{出}{}づる\\
        \ruby{生命}{}の\ruby{海}{}の\ruby{高鳴}{}るを
        
    \end{minipage}
    \begin{minipage}[c]{\blocksize}
        
        \vspace{\linespace}
        \item~\\
        % 4.
        \ruby{若}{}きに\ruby{芽}{}ぐむ\ruby{数々}{}の\\
        \ruby{深}{}き\ruby{苦悩}{}は\ruby{身}{}にあれど\\
        \ruby{迪}{}を\ruby{恵}{}ねて\ruby{辿}{}りゆく\\
        \ruby{遊子}{}の\ruby{真意君知}{}るや
        
    \end{minipage}
    \begin{minipage}[c]{\blocksize}
        
        \vspace{\linespace}
        \item~\\
        % 5.
        \ruby{茫々千里石狩}{}の\\
        \ruby{野}{}は\ruby{澄}{}みわたる\ruby{銀}{}の\\
        \ruby{雪}{}さんらんと\ruby{散}{}るところ\\
        われらが\ruby{魂}{}の\ruby{故郷}{}かな
        
    \end{minipage}
    \begin{minipage}[c]{\blocksize}
        
        \vspace{\linespace}
        \item~\\
        % 6.
        \ruby{若}{}き\ruby{勇者}{}よオキクルミ\\
        \ruby{熊}{}をはふりて\ruby{饗宴}{}せし\\
        \ruby{短檠}{}すでに\ruby{光消}{}え\\
        \ruby{東}{}の\ruby{空}{}はかぎろひぬ
        
    \end{minipage}
    \begin{minipage}[c]{\blocksize}
        
        \vspace{\linespace}
        \item~\\
        % 7.
        \ruby{花咲}{}き\ruby{散}{}りて\ruby{五十年}{}\\
        \ruby{寮庭}{}の\ruby{桂}{}も\ruby{年}{}ふりぬ\\
        \ruby{先人}{}の\ruby{影}{}とほけれど\\
        \ruby{遺訓}{}や\ruby{永久}{}に\ruby{薫}{}るらん
        
    \end{minipage}
    \begin{minipage}[c]{\blocksize}
        
        \vspace{\linespace}
        \item~\\
        % 8.
        \ruby{北溟城}{}の\ruby{生活}{}に\\
        \ruby{桜}{}と\ruby{星}{}の\ruby{旗}{}かざし\\
        \ruby{相寄}{}りむすぶ\ruby{三百}{}の\\
        \ruby{志}{}は\ruby{高}{}きわれらかな
        
    \end{minipage}
    \begin{minipage}[c]{\blocksize}
        
        \vspace{\linespace}
        \item~\\
        % 9.
        こよひ\ruby{手稲}{}に\ruby{日}{}は\ruby{落}{}ちて\\
        \ruby{新月細}{}くかがやけば\\
        \ruby{青}{}き\ruby{煙}{}のそが\ruby{中}{}に\\
        ほがらかになる\ruby{楡}{}の\ruby{鐘}{}
        
    \end{minipage}
    \begin{minipage}[c]{\blocksize}
        
        \vspace{\linespace}
        \item~\\
        % 10.
        ああ\ruby{碧落}{}に\ruby{永劫}{}の\\
        \ruby{北斗}{}の\ruby{光}{}かげさえて\\
        \ruby{清}{}き\ruby{三年}{}の\ruby{思出}{}の\\
        \ruby{銀觴}{}の\ruby{酒}{}つきざらん
    
    \end{minipage}
\end{enumerate} % 番号の箇条書き ここまで
%%%%% 歌詞 ここまで %%%%%
% end body

\end{document}

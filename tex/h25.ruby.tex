\documentclass[10pt,b5j]{tarticle} % B6 縦書き
% \documentclass[10pt,b5j]{tarticle} % B6 縦書き
\AtBeginDvi{\special{papersize=128mm,182mm}} % B6 用用紙サイズ
\usepackage{otf} % Unicode で字を入力するのに必要なパッケージ
\usepackage[size=b6j]{bxpapersize} % B6 用紙サイズを指定
\usepackage[dvipdfmx]{graphicx} % 画像を挿入するためのパッケージ
\usepackage[dvipdfmx]{color} % 色をつけるためのパッケージ
\usepackage{pxrubrica} % ルビを振るためのパッケージ
\usepackage{multicol} % 複数段組を作るためのパッケージ
\setlength{\topmargin}{14mm} % 上下方向のマージン
\addtolength{\topmargin}{-1in} % 
\setlength{\oddsidemargin}{11mm} % 左右方向のマージン
\addtolength{\oddsidemargin}{-1in} % 
\setlength{\textwidth}{154mm} % B6 用
\setlength{\textheight}{108mm} % B6 用
\setlength{\headsep}{0mm} % 
\setlength{\headheight}{0mm} % 
\setlength{\topskip}{0mm} % 
\setlength{\parskip}{0pt} % 
\def\labelenumi{\theenumi、} % 箇条書きのフォーマット
\parindent = 0pt % 段落下げしない

 % B6 用テンプレート読み込み

\begin{document}
% begin header
%%%%% タイトルと作者 ここから %%%%%
\begin{minipage}[c]{0.7\hsize} % タイトルは上から 7 割
    \begin{center}
    % begin title
        {\LARGE
            北溟の我らぞ % タイトルを入れる
        }
        {\small 
            (平成二十五年新々寮三十周年記念寮歌) % 年などを入れる
        }
    % end title
    \end{center}
\end{minipage}
\begin{minipage}[c]{0.3\hsize} % 作歌作曲は上から 3 割
    \begin{flushright} % 下寄せにする
        % begin name
        森貝聡恵君 作歌\\菊池玄之介君 作曲 % 作歌・作曲者
        % end name
    \end{flushright}
\end{minipage}
%%%%% タイトルと作者 ここまで %%%%%
% (1,2,3 繰り返しなし)
% end header

% begin body
\vspace{1.5em} % タイトル, 作者と歌詞の間に隙間を設ける
\newcommand{\linespace}{0.5em} % 行間の設定
\newcommand{\blocksize}{0.5\hsize} % 段組間の設定
%%%%% 歌詞 ここから %%%%%
% begin lilycs
\begin{enumerate} % 番号の箇条書き ここから
    \begin{minipage}[c]{\blocksize}
    
        \vspace{\linespace}
        \item
        % 1.
        \ruby{君何故来}{}たるこの\ruby{北溟}{}の\ruby{地}{}に\\
        かの\ruby{師}{}の\ruby{教}{}へ\ruby{受}{}け\ruby{継}{}ぎし\ruby{地}{}に\\
        \ruby{貴}{}き\ruby{野心}{}ゆめ\ruby{忘}{}るまじ\\
        ひたと\ruby{気高}{}く\\
        いざ\ruby{踏}{}み\ruby{出}{}さむ\ruby{新}{}たなる\ruby{夢}{}へ
        
        \vspace{\linespace}
        \item
        % 2.
        \ruby{野心}{}は\ruby{強}{}く\ruby{熱}{}くとも\\
        \ruby{道草恋}{}しき\ruby{時}{}もあるかな\\
        \ruby{青}{}き\ruby{春}{}の\ruby{夜友}{}らが\ruby{宴}{}\\
        \ruby{語}{}り\ruby{合}{}おうぞ\\
        \ruby{我}{}らが\ruby{大}{}き\ruby{大}{}き\ruby{夢}{}の\ruby{芽}{}
        
        \vspace{\linespace}
        \item
        % 3.
        \ruby{我}{}らが\ruby{未来}{}はいかなるものか\\
        \ruby{曇}{}り\ruby{澱}{}みし\ruby{過去}{}ではあれど\\
        これより\ruby{先}{}は\ruby{我}{}らが\ruby{拓}{}かむ\\
        \ruby{先達}{}に\ruby{続}{}け\\
        \ruby{大和}{}の\ruby{栄}{}えをば\ruby{担}{}うは\ruby{我}{}らぞ
    
    \end{minipage}
\end{enumerate} % 番号の箇条書き ここまで
% end lilycs
%%%%% 歌詞 ここまで %%%%%
% end body

\end{document}

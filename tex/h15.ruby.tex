\documentclass[10pt,b5j]{tarticle} % B6 縦書き
% \documentclass[10pt,b5j]{tarticle} % B6 縦書き
\AtBeginDvi{\special{papersize=128mm,182mm}} % B6 用用紙サイズ
\usepackage{otf} % Unicode で字を入力するのに必要なパッケージ
\usepackage[size=b6j]{bxpapersize} % B6 用紙サイズを指定
\usepackage[dvipdfmx]{graphicx} % 画像を挿入するためのパッケージ
\usepackage[dvipdfmx]{color} % 色をつけるためのパッケージ
\usepackage{pxrubrica} % ルビを振るためのパッケージ
\usepackage{multicol} % 複数段組を作るためのパッケージ
\setlength{\topmargin}{14mm} % 上下方向のマージン
\addtolength{\topmargin}{-1in} % 
\setlength{\oddsidemargin}{11mm} % 左右方向のマージン
\addtolength{\oddsidemargin}{-1in} % 
\setlength{\textwidth}{154mm} % B6 用
\setlength{\textheight}{108mm} % B6 用
\setlength{\headsep}{0mm} % 
\setlength{\headheight}{0mm} % 
\setlength{\topskip}{0mm} % 
\setlength{\parskip}{0pt} % 
\def\labelenumi{\theenumi、} % 箇条書きのフォーマット
\parindent = 0pt % 段落下げしない

 % B6 用テンプレート読み込み

\begin{document}
% begin header
%%%%% タイトルと作者 ここから %%%%%
\begin{minipage}[c]{0.7\hsize} % タイトルは上から 7 割
    \begin{center}
    % begin title
        {\LARGE
            ああグッと % タイトルを入れる
        }
        {\small 
            (平成15年度寮歌) % 年などを入れる
        }
    % end title
    \end{center}
\end{minipage}
\begin{minipage}[c]{0.3\hsize} % 作歌作曲は上から 3 割
    \begin{flushright} % 下寄せにする
        % begin name
        井口拓君 作歌\\持田翼君 作曲 % 作歌・作曲者
        % end name
    \end{flushright}
\end{minipage}
%%%%% タイトルと作者 ここまで %%%%%
% (1,2,3,9,10 了なし繰り返しあり)
% end header

% begin body
\vspace{1.5em} % タイトル, 作者と歌詞の間に隙間を設ける
\newcommand{\linespace}{0.5em} % 行間の設定
\newcommand{\blocksize}{0.5\hsize} % 段組間の設定
%%%%% 歌詞 ここから %%%%%
% begin lilycs
\begin{enumerate} % 番号の箇条書き ここから
    \begin{minipage}[c]{\blocksize}
    
        \vspace{\linespace}
        \item
        % 1.
        もしも\ruby{海}{}が\ruby{酒}{}ならば\\
        お\ruby{前}{}は\ruby{魚}{}になるという\\
        \ruby{俺}{}は\ruby{渚}{}の\ruby{貝}{}になる\\
        \ruby{波}{}が\ruby{来}{}るたび\ruby{酒}{}を\ruby{飲}{}む
        
        \vspace{\linespace}
        \item
        % 2.
        つまみはそうさ\ruby{俺}{}の\ruby{脳}{}\\
        \ruby{酒}{}にとろけた\ruby{脳}{}みそさ\\
        \ruby{代}{}わりにお\ruby{前}{}を\ruby{盃}{}に\\
        \ruby{空}{}の\ruby{頭蓋}{}に\ruby{酒}{}を\ruby{注}{}ぐ
        
        \vspace{\linespace}
        \item
        % 3.
        \ruby{明日}{}は\ruby{泥土}{}に\ruby{墜}{}ちるとも\\
        \ruby{今}{}は\ruby{昇}{}らんはしご\ruby{酒}{}\\
        \ruby{美}{}しの\ruby{盃}{}を\ruby{重}{}ねては\\
        その\ruby{身月}{}にも\ruby{届}{}くべし
        
        \vspace{\linespace}
        \item
        % 4.
        \ruby{盃}{}もめぐりて\ruby{今}{}や\ruby{今}{}\\
        \ruby{魑魅魍魎}{}が\ruby{顔}{}を\ruby{出}{}す\\
        ヤマタノオロチ\ruby{現}{}れる\\
        \ruby{大}{}トラ\ruby{小}{}トラ\ruby{管}{}を\ruby{巻}{}く
        
        \vspace{\linespace}
        \item
        % 5.
        \ruby{更}{}け\ruby{行}{}く\ruby{夜}{}に\ruby{浮}{}かぶ\ruby{月}{}\\
        \ruby{窓辺}{}にうつる\ruby{影}{}は\ruby{今}{}\\
        \ruby{何}{}をし\ruby{何}{}をされるのか\\
        \ruby{月}{}は\ruby{黙}{}って\ruby{見}{}るばかり
        
        \vspace{\linespace}
        \item
        % 6.
        \ruby{中天高}{}く\ruby{日}{}は\ruby{昇}{}り\\
        \ruby{今日}{}もマグロの\ruby{大漁旗}{}\\
        \ruby{死屍累々}{}の\ruby{戦場}{}に\\
        \ruby{兵}{}どもが\ruby{夢}{}の\ruby{跡}{}
        
        \vspace{\linespace}
        \item
        % 7.
        \ruby{天}{}の\ruby{夢}{}から\ruby{落}{}っこちて\\
        \ruby{今日}{}は\ruby{地}{}を\ruby{這}{}う\ruby{宿酔}{}\\
        「なぜ\ruby{繰}{}り\ruby{返}{}す\ruby{過}{}ちを」\\
        \ruby{空}{}しく\ruby{響}{}くいつもの\ruby{問}{}い
        
        \vspace{\linespace}
        \item
        % 8.
        \ruby{積}{}んでは\ruby{崩}{}す\ruby{盃}{}は\\
        \ruby{賽}{}の\ruby{河原}{}の\ruby{石積}{}みか\\
        それでもいつか\ruby{天}{}に\ruby{着}{}く\\
        その\ruby{日}{}を\ruby{信}{}じ\ruby{盃}{}に\ruby{酌}{}む
        
        \vspace{\linespace}
        \item
        % 9.
        とかく\ruby{憂}{}の\ruby{多}{}い\ruby{世}{}を\\
        されば\ruby{払}{}えよ\ruby{玉帚}{}\\
        \ruby{積}{}もる\ruby{芥}{}の\ruby{流}{}れては\\
        \ruby{自}{}ずと\ruby{心開}{}くべし
        
        \vspace{\linespace}
        \item
        % 10.
        たとえ\ruby{百年生}{}きたとて\\
        わずかに\ruby{三万六千日}{}\\
        されば\ruby{尽}{}くさんこの\ruby{盃}{}を\\
        \ruby{一日必}{}ず\ruby{三百杯}{}
    
    \end{minipage}
\end{enumerate} % 番号の箇条書き ここまで
% end lilycs
%%%%% 歌詞 ここまで %%%%%
% end body

\end{document}

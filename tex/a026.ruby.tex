\documentclass[10pt,b5j]{tarticle} % B6 縦書き
% \documentclass[10pt,b5j]{tarticle} % B6 縦書き
\AtBeginDvi{\special{papersize=128mm,182mm}} % B6 用用紙サイズ
\usepackage{otf} % Unicode で字を入力するのに必要なパッケージ
\usepackage[size=b6j]{bxpapersize} % B6 用紙サイズを指定
\usepackage[dvipdfmx]{graphicx} % 画像を挿入するためのパッケージ
\usepackage[dvipdfmx]{color} % 色をつけるためのパッケージ
\usepackage{pxrubrica} % ルビを振るためのパッケージ
\usepackage{multicol} % 複数段組を作るためのパッケージ
\setlength{\topmargin}{14mm} % 上下方向のマージン
\addtolength{\topmargin}{-1in} % 
\setlength{\oddsidemargin}{11mm} % 左右方向のマージン
\addtolength{\oddsidemargin}{-1in} % 
\setlength{\textwidth}{154mm} % B6 用
\setlength{\textheight}{108mm} % B6 用
\setlength{\headsep}{0mm} % 
\setlength{\headheight}{0mm} % 
\setlength{\topskip}{0mm} % 
\setlength{\parskip}{0pt} % 
\def\labelenumi{\theenumi、} % 箇条書きのフォーマット
\parindent = 0pt % 段落下げしない

 % B6 用テンプレート読み込み

\begin{document}
% begin header
%%%%% タイトルと作者 ここから %%%%%
\begin{minipage}[c]{0.7\hsize} % タイトルは上から 7 割
    \begin{center}
    % begin title
        {\LARGE
            春三月の(茨戸の歌) % タイトルを入れる
        }
        {\small 
            (ボート部部歌) % 年などを入れる
        }
    % end title
    \end{center}
\end{minipage}
\begin{minipage}[c]{0.3\hsize} % 作歌作曲は上から 3 割
    \begin{flushright} % 下寄せにする
        % begin name
        木原慎一君 作歌・作曲 % 作歌・作曲者
        % end name
    \end{flushright}
\end{minipage}
%%%%% タイトルと作者 ここまで %%%%%
% % end header

% begin body
\vspace{1.5em} % タイトル, 作者と歌詞の間に隙間を設ける
\newcommand{\linespace}{0.5em} % 行間の設定
\newcommand{\blocksize}{0.5\hsize} % 段組間の設定
%%%%% 歌詞 ここから %%%%%
% begin lilycs
\begin{enumerate} % 番号の箇条書き ここから
    \begin{minipage}[c]{\blocksize}
    
        \vspace{\linespace}
        \item
        % 1.
        \ruby{春三月}{}の\ruby{蝦夷島}{}\\
        \ruby{長}{}き\ruby{眠}{}りにとざされし\\
        \ruby{茨戸河畔}{}の\ruby{雪}{}とけて\\
        とく\ruby{待}{}ちわびし\ruby{水}{}の\ruby{子}{}の\\
        \ruby{喜}{}び\ruby{笑}{}ふ\ruby{声}{}すなり
        
        \vspace{\linespace}
        \item
        % 2.
        \ruby{岸}{}の\ruby{辺近}{}く\ruby{郭公}{}の\\
        \ruby{啼}{}く\ruby{音}{}うれしく\ruby{聞}{}き\ruby{初}{}めね\\
        \ruby{漕}{}ぎ\ruby{来}{}し\ruby{方}{}を\ruby{眺}{}むれば\\
        \ruby{霞}{}にとける\ruby{野}{}の\ruby{煙}{}\\
        \ruby{水郷}{}の\ruby{春}{}の\ruby{昼閑}{}か
        
        \vspace{\linespace}
        \item
        % 3.
        \ruby{岩燕}{}は\ruby{去}{}りて\ruby{風熱}{}き\\
        \ruby{夏}{}たけなはの\ruby{候}{}となる\\
        \ruby{運河一発引}{}き\ruby{抜}{}きて\\
        しばし\ruby{憩}{}はむ\ruby{土手}{}の\ruby{上}{}\\
        \ruby{羊}{}も\ruby{寄}{}りて\ruby{草}{}を\ruby{食}{}む
        
        \vspace{\linespace}
        \item
        % 4.
        いつか\ruby{炎暑}{}の\ruby{日}{}はゆきて\\
        \ruby{光}{}のどけき\ruby{茨戸河}{}\\
        \ruby{青}{}き\ruby{水}{}の\ruby{面}{}に\ruby{波立}{}たず\\
        こよなき\ruby{季節訪}{}れぬ\\
        \ruby{心}{}ゆくまで\ruby{漕}{}がむかな
        
        \vspace{\linespace}
        \item
        % 5.
        \ruby{手稲}{}は\ruby{紅}{}く\ruby{空高}{}く\\
        \ruby{秋}{}の\ruby{気深}{}くなりにけり\\
        かい\ruby{先近}{}くぼらはねて\\
        \ruby{夕練習終}{}へるころ\\
        \ruby{陽}{}はくれないに\ruby{没}{}したり
        
        \vspace{\linespace}
        \item
        % 6.
        \ruby{河霧深}{}くたちこめて\\
        \ruby{霜結}{}ぶ\ruby{朝艇出}{}す\\
        みぎわの\ruby{木々}{}は\ruby{枯}{}れはてて\\
        \ruby{冬}{}もま\ruby{近}{}となりぬれば\\
        \ruby{惜}{}しみて\ruby{漕}{}がむ\ruby{残}{}る\ruby{日々}{}
        
        \vspace{\linespace}
        \item
        % 7.
        \ruby{北風}{}すさび\ruby{雪}{}は\ruby{舞}{}ひ\\
        ふぶきに\ruby{暮}{}れる\ruby{冬}{}の\ruby{河}{}\\
        \ruby{今日}{}ぞわれらが\ruby{漕}{}ぎ\ruby{納}{}め\\
        いざわが\ruby{友}{}よ\ruby{胸深}{}く\\
        また\ruby{来}{}む\ruby{年}{}の\ruby{幸思}{}へ
    
    \end{minipage}
\end{enumerate} % 番号の箇条書き ここまで
% end lilycs
%%%%% 歌詞 ここまで %%%%%
% end body

\end{document}

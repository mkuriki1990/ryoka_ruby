\documentclass[10pt,b5j]{tarticle} % B6 縦書き
% \documentclass[10pt,b5j]{tarticle} % B6 縦書き
\AtBeginDvi{\special{papersize=128mm,182mm}} % B6 用用紙サイズ
\usepackage{otf} % Unicode で字を入力するのに必要なパッケージ
\usepackage[size=b6j]{bxpapersize} % B6 用紙サイズを指定
\usepackage[dvipdfmx]{graphicx} % 画像を挿入するためのパッケージ
\usepackage[dvipdfmx]{color} % 色をつけるためのパッケージ
\usepackage{pxrubrica} % ルビを振るためのパッケージ
\usepackage{multicol} % 複数段組を作るためのパッケージ
\setlength{\topmargin}{14mm} % 上下方向のマージン
\addtolength{\topmargin}{-1in} % 
\setlength{\oddsidemargin}{11mm} % 左右方向のマージン
\addtolength{\oddsidemargin}{-1in} % 
\setlength{\textwidth}{154mm} % B6 用
\setlength{\textheight}{108mm} % B6 用
\setlength{\headsep}{0mm} % 
\setlength{\headheight}{0mm} % 
\setlength{\topskip}{0mm} % 
\setlength{\parskip}{0pt} % 
\def\labelenumi{\theenumi、} % 箇条書きのフォーマット
\parindent = 0pt % 段落下げしない

 % B6 用テンプレート読み込み

\begin{document}
% begin header
%%%%% タイトルと作者 ここから %%%%%
\begin{minipage}[c]{0.7\hsize} % タイトルは上から 7 割
    \begin{center}
    % begin title
        {\LARGE
            昭和9年応援歌 % タイトルを入れる
        }
        {\small 
             % 年などを入れる
        }
    % end title
    \end{center}
\end{minipage}
\begin{minipage}[c]{0.3\hsize} % 作歌作曲は上から 3 割
    \begin{flushright} % 下寄せにする
        % begin name
        末松和彦君 作歌\\菊地修二君 作曲 % 作歌・作曲者
        % end name
    \end{flushright}
\end{minipage}
%%%%% タイトルと作者 ここまで %%%%%
% % end header

% begin length
\vspace{1.5em} % タイトル, 作者と歌詞の間に隙間を設ける
\newcommand{\linespace}{0.5em} % 行間の設定
\newcommand{\blocksize}{0.5\hsize} % 段組間の設定
\newcommand{\itemmargin}{3em} % 曲番の位置調整の長さ
% end length
% begin body
%%%%% 歌詞 ここから %%%%%
\begin{enumerate} % 番号の箇条書き ここから
    \setlength{\itemindent}{\itemmargin} % 曲番の位置調整
    \begin{minipage}[c]{\blocksize}
    
        \vspace{\linespace}
        \item~\\
        \ruby{戦雲}{せんうん}\ruby{低}{ひく}し\\
        \ruby{北張}{きたはり}の\ruby{牙城}{がじょう}\\
        \ruby{烽火}{ほうか}\ruby{揚}{あ}がれり\\
        \ruby{大地}{だいち}を\ruby{蹴}{け}って\\
        \ruby{集}{つど}える\ruby{友}{とも}が\ruby{意}{い}\ruby{氣天}{}を\ruby{衝}{つ}き\\
        \ruby{見}{み}よ\ruby{颯爽}{さっそう}と\\
        \ruby{白}{しろ}\ruby{旌}{}\ruby{揺}{ゆる}ぐ
        
    \end{minipage}
    \begin{minipage}[c]{\blocksize}
        
        \vspace{\linespace}
        \item~\\
        % 2.
        \ruby{乾坤}{けんこん}の\ruby{闘志}{とうし}を\ruby{我}{われ}\ruby{培}{}へり\\
        \ruby{中原}{なかはら}\ruby{定}{さだ}む\\
        \ruby{覇業}{はぎょう}と\ruby{共}{とも}に\\
        \ruby{魂魄}{こんぱく}\ruby{碎}{くだ}け\\
        \ruby{五体}{ごたい}は\ruby{散}{ち}れど\\
        \ruby{迸}{ほとばし}る\ruby{血}{ち}もて\ruby{凱歌}{がいか}に\ruby{酔}{よい}はむ
    
    \end{minipage}
\end{enumerate} % 番号の箇条書き ここまで
%%%%% 歌詞 ここまで %%%%%
% end body

\end{document}

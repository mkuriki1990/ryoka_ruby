\documentclass[10pt,b5j]{tarticle} % B6 縦書き
% \documentclass[10pt,b5j]{tarticle} % B6 縦書き
\AtBeginDvi{\special{papersize=128mm,182mm}} % B6 用用紙サイズ
\usepackage{otf} % Unicode で字を入力するのに必要なパッケージ
\usepackage[size=b6j]{bxpapersize} % B6 用紙サイズを指定
\usepackage[dvipdfmx]{graphicx} % 画像を挿入するためのパッケージ
\usepackage[dvipdfmx]{color} % 色をつけるためのパッケージ
\usepackage{pxrubrica} % ルビを振るためのパッケージ
\usepackage{multicol} % 複数段組を作るためのパッケージ
\setlength{\topmargin}{14mm} % 上下方向のマージン
\addtolength{\topmargin}{-1in} % 
\setlength{\oddsidemargin}{11mm} % 左右方向のマージン
\addtolength{\oddsidemargin}{-1in} % 
\setlength{\textwidth}{154mm} % B6 用
\setlength{\textheight}{108mm} % B6 用
\setlength{\headsep}{0mm} % 
\setlength{\headheight}{0mm} % 
\setlength{\topskip}{0mm} % 
\setlength{\parskip}{0pt} % 
\def\labelenumi{\theenumi、} % 箇条書きのフォーマット
\parindent = 0pt % 段落下げしない

 % B6 用テンプレート読み込み

\begin{document}
% begin header
%%%%% タイトルと作者 ここから %%%%%
\begin{minipage}[c]{0.7\hsize} % タイトルは上から 7 割
    \begin{center}
    % begin title
        {\LARGE
            予科応援歌 % タイトルを入れる
        }
        {\small 
            (大正四年) % 年などを入れる
        }
    % end title
    \end{center}
\end{minipage}
\begin{minipage}[c]{0.3\hsize} % 作歌作曲は上から 3 割
    \begin{flushright} % 下寄せにする
        % begin name
         % 作歌・作曲者
        % end name
    \end{flushright}
\end{minipage}
%%%%% タイトルと作者 ここまで %%%%%
% % end header

% begin length
\vspace{1.5em} % タイトル, 作者と歌詞の間に隙間を設ける
\newcommand{\linespace}{0.5em} % 行間の設定
\newcommand{\blocksize}{0.5\hsize} % 段組間の設定
\newcommand{\itemmargin}{6em} % 曲番の位置調整の長さ
% end length
% begin body
%%%%% 歌詞 ここから %%%%%
\begin{enumerate} % 番号の箇条書き ここから
    \setlength{\itemindent}{\itemmargin} % 曲番の位置調整
    \begin{minipage}[c]{\blocksize}
    
        \vspace{\linespace}
        \item~\\
        % 1.
        \ruby{楡樹}{}の\ruby{都春}{}たけて\\
        \ruby{風新緑}{}に\ruby{薫}{}るとき\\
        \ruby{勇士三百旗鼓堂々}{}と\\
        \ruby{喊聲}{}あぐるはれ\ruby{戦}{}\\
        フレ\ruby{予科}{} フレ\ruby{予科}{}\\
        フレ フレ フレ
        
        \vspace{\linespace}
        \item~\\
        % 2.
        \ruby{思}{}へば\ruby{昔豊公}{}が\\
        \ruby{貔貅十万海}{}を\ruby{越}{}え\\
        \ruby{韓八道}{}を\ruby{血潮}{}に\ruby{染}{}めし\\
        \ruby{滔天}{}の\ruby{意氣我}{}にあり\\
        \ruby{走}{}れ\ruby{鷺山}{} \ruby{走}{}れ\ruby{平岡}{}\\
        \ruby{走}{}れ \ruby{走}{}れ \ruby{走}{}れ
        
        \vspace{\linespace}
        \item~\\
        % 3.
        \ruby{振}{}へ\ruby{若人今}{}ぞ\ruby{今}{}\\
        \ruby{汝}{}が\ruby{雄心}{}のをどるとき\\
        \ruby{榮}{}ある\ruby{勝利}{}の\ruby{凱歌}{}と\ruby{共}{}に\\
        \ruby{誉}{}は\ruby{高}{}し\ruby{優勝旗}{}\\
        \ruby{勝}{}った\ruby{予科}{} \ruby{勝}{}った\ruby{予科}{}\\
        \ruby{勝}{}った \ruby{勝}{}った \ruby{勝}{}った
    
    \end{minipage}
\end{enumerate} % 番号の箇条書き ここまで
%%%%% 歌詞 ここまで %%%%%
% end body

\end{document}

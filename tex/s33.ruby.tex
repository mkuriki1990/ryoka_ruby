\documentclass[10pt,b5j]{tarticle} % B6 縦書き
% \documentclass[10pt,b5j]{tarticle} % B6 縦書き
\AtBeginDvi{\special{papersize=128mm,182mm}} % B6 用用紙サイズ
\usepackage{otf} % Unicode で字を入力するのに必要なパッケージ
\usepackage[size=b6j]{bxpapersize} % B6 用紙サイズを指定
\usepackage[dvipdfmx]{graphicx} % 画像を挿入するためのパッケージ
\usepackage[dvipdfmx]{color} % 色をつけるためのパッケージ
\usepackage{pxrubrica} % ルビを振るためのパッケージ
\usepackage{multicol} % 複数段組を作るためのパッケージ
\setlength{\topmargin}{14mm} % 上下方向のマージン
\addtolength{\topmargin}{-1in} % 
\setlength{\oddsidemargin}{11mm} % 左右方向のマージン
\addtolength{\oddsidemargin}{-1in} % 
\setlength{\textwidth}{154mm} % B6 用
\setlength{\textheight}{108mm} % B6 用
\setlength{\headsep}{0mm} % 
\setlength{\headheight}{0mm} % 
\setlength{\topskip}{0mm} % 
\setlength{\parskip}{0pt} % 
\def\labelenumi{\theenumi、} % 箇条書きのフォーマット
\parindent = 0pt % 段落下げしない

 % B6 用テンプレート読み込み

\begin{document}
% begin header
%%%%% タイトルと作者 ここから %%%%%
\begin{minipage}[c]{0.7\hsize} % タイトルは上から 7 割
    \begin{center}
    % begin title
        {\LARGE
            吾れ憧れし % タイトルを入れる
        }
        {\small 
            (昭和33年寮歌) % 年などを入れる
        }
    % end title
    \end{center}
\end{minipage}
\begin{minipage}[c]{0.3\hsize} % 作歌作曲は上から 3 割
    \begin{flushright} % 下寄せにする
        % begin name
        佐伯政英君 作歌\\佐藤一正君 作曲 % 作歌・作曲者
        % end name
    \end{flushright}
\end{minipage}
%%%%% タイトルと作者 ここまで %%%%%
% (1,2,3,4 了あり)
% end header

% begin body
\vspace{1.5em} % タイトル, 作者と歌詞の間に隙間を設ける
\newcommand{\linespace}{0.5em} % 行間の設定
\newcommand{\blocksize}{0.5\hsize} % 段組間の設定
%%%%% 歌詞 ここから %%%%%
% begin lilycs
\begin{enumerate} % 番号の箇条書き ここから
    \begin{minipage}[c]{\blocksize}
    
        \vspace{\linespace}
        \item
        % 1.
        \ruby{吾憧}{}れし\ruby{美}{}の\ruby{国}{}の\\
        \ruby{春}{}は\ruby{名}{}のみの\ruby{春}{}なれど\\
        \ruby{雪解}{}の\ruby{水}{}に\ruby{甦}{}る\\
        \ruby{野面}{}に\ruby{充}{}ち\ruby{満}{}つ\ruby{生命}{}あり
        
        \vspace{\linespace}
        \item
        % 2.
        \ruby{遠}{}くふるさと\ruby{離}{}れ\ruby{来}{}し\\
        \ruby{寮友}{}と\ruby{睦}{}の\ruby{杯酌}{}めば\\
        \ruby{今日}{}も\ruby{手稲山}{}に\ruby{夕映}{}えて\\
        \ruby{鐘声}{}はろかに\ruby{快}{}よし
        
        \vspace{\linespace}
        \item
        % 3.
        \ruby{楡}{}の\ruby{木蔭}{}に\ruby{憩}{}せば\\
        \ruby{紫紺}{}の\ruby{峰}{}をこえゆきて\\
        \ruby{父母}{}いかに\ruby{君}{}いかに\\
        つきるを\ruby{知}{}らぬ\ruby{吾}{}が\ruby{懐}{}い
        
        \vspace{\linespace}
        \item
        % 4.
        ただ\ruby{茫漠}{}の\ruby{大平野}{}\\
        \ruby{静寂}{}の\ruby{夜}{}は\ruby{更}{}けゆきて\\
        \ruby{囲}{}む\ruby{焚火}{}も\ruby{暗}{}に\ruby{消}{}え\\
        \ruby{夜空彩}{}る\ruby{北斗星}{}
    
    \end{minipage}
\end{enumerate} % 番号の箇条書き ここまで
% end lilycs
%%%%% 歌詞 ここまで %%%%%
% end body

\end{document}

\documentclass[10pt,b5j]{tarticle} % B6 縦書き
% \documentclass[10pt,b5j]{tarticle} % B6 縦書き
\AtBeginDvi{\special{papersize=128mm,182mm}} % B6 用用紙サイズ
\usepackage{otf} % Unicode で字を入力するのに必要なパッケージ
\usepackage[size=b6j]{bxpapersize} % B6 用紙サイズを指定
\usepackage[dvipdfmx]{graphicx} % 画像を挿入するためのパッケージ
\usepackage[dvipdfmx]{color} % 色をつけるためのパッケージ
\usepackage{pxrubrica} % ルビを振るためのパッケージ
\usepackage{multicol} % 複数段組を作るためのパッケージ
\setlength{\topmargin}{14mm} % 上下方向のマージン
\addtolength{\topmargin}{-1in} % 
\setlength{\oddsidemargin}{11mm} % 左右方向のマージン
\addtolength{\oddsidemargin}{-1in} % 
\setlength{\textwidth}{154mm} % B6 用
\setlength{\textheight}{108mm} % B6 用
\setlength{\headsep}{0mm} % 
\setlength{\headheight}{0mm} % 
\setlength{\topskip}{0mm} % 
\setlength{\parskip}{0pt} % 
\def\labelenumi{\theenumi、} % 箇条書きのフォーマット
\parindent = 0pt % 段落下げしない

 % B6 用テンプレート読み込み

\begin{document}
% begin header
%%%%% タイトルと作者 ここから %%%%%
\begin{minipage}[c]{0.7\hsize} % タイトルは上から 7 割
    \begin{center}
    % begin title
        {\LARGE
            太虚の齢 % タイトルを入れる
        }
        {\small 
            (明治四十一年寮歌) % 年などを入れる
        }
    % end title
    \end{center}
\end{minipage}
\begin{minipage}[c]{0.3\hsize} % 作歌作曲は上から 3 割
    \begin{flushright} % 下寄せにする
        % begin name
        田中義麿君 作歌\\早川直瀬君・前川徳次郎君 作曲 % 作歌・作曲者
        % end name
    \end{flushright}
\end{minipage}
%%%%% タイトルと作者 ここまで %%%%%
% (1,2,3,6 了あり)
% end header

% begin length
\vspace{1.5em} % タイトル, 作者と歌詞の間に隙間を設ける
\newcommand{\linespace}{0.5em} % 行間の設定
\newcommand{\blocksize}{0.5\hsize} % 段組間の設定
\newcommand{\itemmargin}{3em} % 曲番の位置調整の長さ
% end length
% begin body
%%%%% 歌詞 ここから %%%%%
\begin{enumerate} % 番号の箇条書き ここから
    \setlength{\itemindent}{\itemmargin} % 曲番の位置調整
    \begin{minipage}[c]{\blocksize}
    
        \vspace{\linespace}
        \item~\\
        % 1.
        \ruby{太虚}{}の\ruby{齢}{}は\ruby{知}{}らねども\\
        \ruby{興廃}{}うつる\ruby{人}{}の\ruby{世}{}の\\
        \ruby{文化}{}の\ruby{跡}{}は\ruby{四千年}{}\\
        ありし\ruby{往昔}{}を\ruby{温}{}ね\ruby{来}{}て\\
        \ruby{吾}{}が\ruby{世}{}の\ruby{状態}{}を\ruby{眺}{}むれば\\
        \ruby{希望栄}{}ある\ruby{前途}{}かな
        
    \end{minipage}
    \begin{minipage}[c]{\blocksize}
        
        \vspace{\linespace}
        \item~\\
        % 2.
        \ruby{嘗}{}てナイルの\ruby{河水}{}に\\
        \ruby{偉影涵}{}せし\ruby{金字塔}{}\\
        アテネの\ruby{春}{}も\ruby{夢}{}なれや\\
        ローマの\ruby{紅紫}{}また\ruby{散}{}りて\\
        \ruby{欧米}{}の\ruby{空今正}{}に\\
        \ruby{文化}{}の\ruby{花}{}ぞ\ruby{盛}{}なる
        
    \end{minipage}
    \begin{minipage}[c]{\blocksize}
        
        \vspace{\linespace}
        \item~\\
        % 3.
        \ruby{偉大}{}ならずや\ruby{雪潔}{}き\\
        ヒマラヤ\ruby{山下風薫}{}り\\
        \ruby{四百余州}{}に\ruby{吹}{}き\ruby{入}{}れば\\
        \ruby{聖賢雲}{}と\ruby{叢起}{}して\\
        \ruby{深}{}き\ruby{思想}{}は\ruby{東洋}{}の\\
        \ruby{青史不朽}{}の\ruby{誇}{}あり
        
    \end{minipage}
    \begin{minipage}[c]{\blocksize}
        
        \vspace{\linespace}
        \item~\\
        % 4.
        \ruby{今東海}{}の\ruby{一孤島}{}\\
        \ruby{文化}{}の\ruby{潮寄}{}せ\ruby{来}{}り\\
        \ruby{東西}{}の\ruby{岸}{}を\ruby{洗}{}ひつつ\\
        \ruby{高}{}き\ruby{響}{}を\ruby{伝}{}ふなり\\
        \ruby{孤島}{}にこもる\ruby{国民}{}の\\
        \ruby{使命}{}などかは\ruby{軽}{}からん
        
    \end{minipage}
    \begin{minipage}[c]{\blocksize}
        
        \vspace{\linespace}
        \item~\\
        % 5.
        \ruby{既}{}に\ruby{天地}{}の\ruby{利}{}は\ruby{獲}{}たり\\
        \ruby{人和豈}{}それなからんや\\
        \ruby{満韓}{}の\ruby{原遺利多}{}く\\
        アルゼンタイン\ruby{野}{}は\ruby{広}{}し\\
        \ruby{故人}{}の\ruby{教訓聴}{}かざるや\\
        「ビーアンビシァス\\
        ボーイズ」と
        
    \end{minipage}
    \begin{minipage}[c]{\blocksize}
        
        \vspace{\linespace}
        \item~\\
        % 6.
        \ruby{猛}{}き\ruby{心}{}の\ruby{往}{}くところ\\
        \ruby{虎狼鮫鰐}{}ものならず\\
        テキサス\ruby{鍬}{}を\ruby{入}{}るる\ruby{可}{}く\\
        シベリヤ\ruby{斧}{}を\ruby{振}{}ふ\ruby{可}{}し\\
        \ruby{故人}{}の\ruby{教訓膺}{}にせよ\\
        「ビーアンビシァス\\
        ボーイズ」と
    
    \end{minipage}
\end{enumerate} % 番号の箇条書き ここまで
%%%%% 歌詞 ここまで %%%%%
% end body

\end{document}

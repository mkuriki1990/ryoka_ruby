\documentclass[10pt,b5j]{tarticle} % B6 縦書き
% \documentclass[10pt,b5j]{tarticle} % B6 縦書き
\AtBeginDvi{\special{papersize=128mm,182mm}} % B6 用用紙サイズ
\usepackage{otf} % Unicode で字を入力するのに必要なパッケージ
\usepackage[size=b6j]{bxpapersize} % B6 用紙サイズを指定
\usepackage[dvipdfmx]{graphicx} % 画像を挿入するためのパッケージ
\usepackage[dvipdfmx]{color} % 色をつけるためのパッケージ
\usepackage{pxrubrica} % ルビを振るためのパッケージ
\usepackage{multicol} % 複数段組を作るためのパッケージ
\setlength{\topmargin}{14mm} % 上下方向のマージン
\addtolength{\topmargin}{-1in} % 
\setlength{\oddsidemargin}{11mm} % 左右方向のマージン
\addtolength{\oddsidemargin}{-1in} % 
\setlength{\textwidth}{154mm} % B6 用
\setlength{\textheight}{108mm} % B6 用
\setlength{\headsep}{0mm} % 
\setlength{\headheight}{0mm} % 
\setlength{\topskip}{0mm} % 
\setlength{\parskip}{0pt} % 
\def\labelenumi{\theenumi、} % 箇条書きのフォーマット
\parindent = 0pt % 段落下げしない

 % B6 用テンプレート読み込み

\begin{document}
% begin header
%%%%% タイトルと作者 ここから %%%%%
\begin{minipage}[c]{0.7\hsize} % タイトルは上から 7 割
    \begin{center}
    % begin title
        {\LARGE
            太虚の齢 % タイトルを入れる
        }
        {\small 
            (明治四十一年寮歌) % 年などを入れる
        }
    % end title
    \end{center}
\end{minipage}
\begin{minipage}[c]{0.3\hsize} % 作歌作曲は上から 3 割
    \begin{flushright} % 下寄せにする
        % begin name
        田中義麿君 作歌\\早川直瀬君・前川徳次郎君 作曲 % 作歌・作曲者
        % end name
    \end{flushright}
\end{minipage}
%%%%% タイトルと作者 ここまで %%%%%
% (1,2,3,6 了あり)
% end header

% begin length
\vspace{1.5em} % タイトル, 作者と歌詞の間に隙間を設ける
\newcommand{\linespace}{0.5em} % 行間の設定
\newcommand{\blocksize}{0.5\hsize} % 段組間の設定
\newcommand{\itemmargin}{3em} % 曲番の位置調整の長さ
% end length
% begin body
%%%%% 歌詞 ここから %%%%%
\begin{enumerate} % 番号の箇条書き ここから
    \setlength{\itemindent}{\itemmargin} % 曲番の位置調整
    \begin{minipage}[c]{\blocksize}
    
        \vspace{\linespace}
        \item~\\
        % 1.
        \ruby{太虚}{たいきょ}の\ruby{齢}{よわい}は\ruby{知}{し}らねども\\
        \ruby{興廃}{こうはい}うつる\ruby{人}{ひと}の\ruby{世}{よ}の\\
        \ruby{文化}{ぶんか}の\ruby{跡}{あと}は\ruby{四}{よん}\ruby{千}{せん}\ruby{年}{ねん}\\
        ありし\ruby{往昔}{おうせき}を\ruby{温}{ぬる}ね\ruby{来}{き}て\\
        \ruby{吾}{わ}が\ruby{世}{よ}の\ruby{状態}{じょうたい}を\ruby{眺}{}むれば\\
        \ruby{希望}{きぼう}\ruby{栄}{さかえ}ある\ruby{前途}{ぜんと}かな
        
    \end{minipage}
    \begin{minipage}[c]{\blocksize}
        
        \vspace{\linespace}
        \item~\\
        % 2.
        \ruby{嘗}{かつ}てナイルの\ruby{河水}{こうすい}に\\
        \ruby{偉}{えら}\ruby{影}{かげ}\ruby{涵}{}せし\ruby{金字塔}{きんじとう}\\
        アテネの\ruby{春}{はる}も\ruby{夢}{ゆめ}なれや\\
        ローマの\ruby{紅紫}{こうし}また\ruby{散}{ち}りて\\
        \ruby{欧米}{おうべい}の\ruby{空}{そら}\ruby{今}{いま}\ruby{正}{まさ}に\\
        \ruby{文化}{ぶんか}の\ruby{花}{はな}ぞ\ruby{盛}{もり}なる
        
    \end{minipage}
    \begin{minipage}[c]{\blocksize}
        
        \vspace{\linespace}
        \item~\\
        % 3.
        \ruby{偉大}{いだい}ならずや\ruby{雪}{ゆき}\ruby{潔}{いさぎよ}き\\
        ヒマラヤ\ruby{山下}{やました}\ruby{風}{ふう}\ruby{薫}{かお}り\\
        \ruby{四百余州}{しひゃくよしゅう}に\ruby{吹}{ふ}き\ruby{入}{はい}れば\\
        \ruby{聖賢}{せいけん}\ruby{雲}{くも}と\ruby{叢}{くさむら}\ruby{起}{おこ}して\\
        \ruby{深}{ふか}き\ruby{思想}{しそう}は\ruby{東洋}{とうよう}の\\
        \ruby{青史}{せいし}\ruby{不朽}{ふきゅう}の\ruby{誇}{ほこ}あり
        
    \end{minipage}
    \begin{minipage}[c]{\blocksize}
        
        \vspace{\linespace}
        \item~\\
        % 4.
        \ruby{今}{こん}\ruby{東海}{とうかい}の\ruby{一}{いち}\ruby{孤島}{ことう}\\
        \ruby{文化}{ぶんか}の\ruby{潮}{しお}\ruby{寄}{よ}せ\ruby{来}{きた}り\\
        \ruby{東西}{とうざい}の\ruby{岸}{きし}を\ruby{洗}{あらい}ひつつ\\
        \ruby{高}{たか}き\ruby{響}{ひびき}を\ruby{伝}{つて}ふなり\\
        \ruby{孤島}{ことう}にこもる\ruby{国民}{こくみん}の\\
        \ruby{使命}{しめい}などかは\ruby{軽}{かる}からん
        
    \end{minipage}
    \begin{minipage}[c]{\blocksize}
        
        \vspace{\linespace}
        \item~\\
        % 5.
        \ruby{既}{すで}に\ruby{天地}{てんち}の\ruby{利}{り}は\ruby{獲}{え}たり\\
        \ruby{人}{ひと}\ruby{和}{わ}\ruby{豈}{あに}それなからんや\\
        \ruby{満}{まん}\ruby{韓}{かん}の\ruby{原}{げん}\ruby{遺}{のこ}\ruby{利}{り}\ruby{多}{おお}く\\
        アルゼンタイン\ruby{野}{の}は\ruby{広}{ひろ}し\\
        \ruby{故人}{こじん}の\ruby{教訓}{きょうくん}\ruby{聴}{き}かざるや\\
        「ビーアンビシァス\\
        ボーイズ」と
        
    \end{minipage}
    \begin{minipage}[c]{\blocksize}
        
        \vspace{\linespace}
        \item~\\
        % 6.
        \ruby{猛}{もう}き\ruby{心}{しん}の\ruby{往}{ゆ}くところ\\
        \ruby{虎狼}{ころう}\ruby{鮫}{さめ}\ruby{鰐}{わに}ものならず\\
        テキサス\ruby{鍬}{くわ}を\ruby{入}{はい}るる\ruby{可}{か}く\\
        シベリヤ\ruby{斧}{おの}を\ruby{振}{ふ}ふ\ruby{可}{か}し\\
        \ruby{故人}{こじん}の\ruby{教訓}{きょうくん}\ruby{膺}{}にせよ\\
        「ビーアンビシァス\\
        ボーイズ」と
    
    \end{minipage}
\end{enumerate} % 番号の箇条書き ここまで
%%%%% 歌詞 ここまで %%%%%
% end body

\end{document}

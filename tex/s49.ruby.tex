\documentclass[10pt,b5j]{tarticle} % B6 縦書き
% \documentclass[10pt,b5j]{tarticle} % B6 縦書き
\AtBeginDvi{\special{papersize=128mm,182mm}} % B6 用用紙サイズ
\usepackage{otf} % Unicode で字を入力するのに必要なパッケージ
\usepackage[size=b6j]{bxpapersize} % B6 用紙サイズを指定
\usepackage[dvipdfmx]{graphicx} % 画像を挿入するためのパッケージ
\usepackage[dvipdfmx]{color} % 色をつけるためのパッケージ
\usepackage{pxrubrica} % ルビを振るためのパッケージ
\usepackage{plext} % 漢数字の enumerate を使うためのパッケージ
\usepackage{multicol} % 複数段組を作るためのパッケージ
\setlength{\topmargin}{14mm} % 上下方向のマージン
\addtolength{\topmargin}{-1in} % 
\setlength{\oddsidemargin}{11mm} % 左右方向のマージン
\addtolength{\oddsidemargin}{-1in} % 
\setlength{\textwidth}{154mm} % B6 用
\setlength{\textheight}{108mm} % B6 用
\setlength{\headsep}{0mm} % 
\setlength{\headheight}{0mm} % 
\setlength{\topskip}{0mm} % 
\setlength{\parskip}{0pt} % 
\def\theenumi{\Kanji{enumi}} % 箇条書きのフォーマットを漢数字に変更
\parindent = 0pt % 段落下げしない
\pagestyle{empty} % すべてのページ番号を消去
% \renewcommand{\baselinestretch}{0.9} % 行間の倍率
 % B6 用テンプレート読み込み

\begin{document}
% begin header
%%%%% タイトルと作者 ここから %%%%%
\begin{minipage}[c]{0.7\hsize} % タイトルは上から 7 割
    \begin{center}
    % begin title
        {\LARGE
            北の都は % タイトルを入れる
        }
        {\small 
            (昭和49年寮歌) % 年などを入れる
        }
    % end title
    \end{center}
\end{minipage}
\begin{minipage}[c]{0.3\hsize} % 作歌作曲は上から 3 割
    \begin{flushright} % 下寄せにする
        % begin name
        大森秀治君 作歌・作曲 % 作歌・作曲者
        % end name
    \end{flushright}
\end{minipage}
%%%%% タイトルと作者 ここまで %%%%%
% (1,2,6,7 了あり)
% end header

% begin body
\vspace{1.5em} % タイトル, 作者と歌詞の間に隙間を設ける
\newcommand{\linespace}{0.5em} % 行間の設定
\newcommand{\blocksize}{0.5\hsize} % 段組間の設定
%%%%% 歌詞 ここから %%%%%
% begin lilycs
\begin{enumerate} % 番号の箇条書き ここから
    \begin{minipage}[c]{\blocksize}
    
        \vspace{\linespace}
        \item
        % 1.
        \ruby{北}{}の\ruby{都}{}は\ruby{開発}{}かれて\\
        \ruby{喪失}{}われゆく\ruby{大自然}{}\\
        \ruby{寮}{}の\ruby{姿}{}も\ruby{変}{}われども\\
        \ruby{恵迪}{}の\ruby{名}{}は\ruby{永遠}{}に
        
        \vspace{\linespace}
        \item
        % 2.
        \ruby{残雪溶}{}けて\ruby{東風吹}{}かば\\
        \ruby{大地}{}は\ruby{黒々}{}と\ruby{輝}{}けど\\
        \ruby{川流絶}{}えて\ruby{水}{}は\ruby{涸}{}れ\\
        \ruby{湿原}{}に\ruby{咲}{}く\ruby{花影}{}なし
        
        \vspace{\linespace}
        \item
        % 3.
        \ruby{緑葉}{}さわぐ\ruby{楡}{}の\ruby{森}{}\\
        \ruby{昔日}{}の\ruby{影}{}すでになく\\
        \ruby{短}{}き\ruby{盛夏}{}の\ruby{夕陽}{}を\ruby{浴}{}びて\\
        ただ\ruby{寥々}{}と\ruby{佇立}{}まう
        
        \vspace{\linespace}
        \item
        % 4.
        \ruby{虚空逍遥}{}う\ruby{月}{}の\ruby{影}{}\\
        \ruby{蒼白}{}く\ruby{映}{}ゆ\ruby{原始森}{}の\ruby{木々}{}\\
        \ruby{秋風}{}にうたれて\ruby{舞}{}う\ruby{落葉}{}\\
        \ruby{早雪}{}までのこの\ruby{眺望}{}
        
        \vspace{\linespace}
        \item
        % 5.
        \ruby{白雪烈風}{}に\ruby{舞}{}い\ruby{上}{}がり\\
        \ruby{疎々}{}たる\ruby{杜}{}を\ruby{吹}{}き\ruby{抜}{}けぬ\\
        \ruby{樹影}{}に\ruby{黒}{}き\ruby{鴉鳥}{}\\
        \ruby{寂莫}{}として\ruby{声}{}もなし
        
        \vspace{\linespace}
        \item
        % 6.
        \ruby{警醒}{}の\ruby{鐘鳴}{}らせども\\
        \ruby{迷夢}{}の\ruby{夜}{}は\ruby{未}{}だ\ruby{明}{}けず\\
        \ruby{行方}{}も\ruby{知}{}れぬ\ruby{朔風}{}に\\
        \ruby{心}{}の\ruby{痛}{}みつのるかな
        
        \vspace{\linespace}
        \item
        % 7.
        \ruby{北}{}に\ruby{旅}{}してこの\ruby{宿}{}に\\
        \ruby{仮寝}{}の\ruby{夢}{}を\ruby{貪}{}りて\\
        \ruby{過}{}ぎし\ruby{歳月早二年}{}\\
        \ruby{懐}{}かしさ\ruby{満}{}つこの\ruby{団居}{}
    
    \end{minipage}
\end{enumerate} % 番号の箇条書き ここまで
% end lilycs
%%%%% 歌詞 ここまで %%%%%
% end body

\end{document}

\documentclass[10pt,b5j]{tarticle} % B6 縦書き
% \documentclass[10pt,b5j]{tarticle} % B6 縦書き
\AtBeginDvi{\special{papersize=128mm,182mm}} % B6 用用紙サイズ
\usepackage{otf} % Unicode で字を入力するのに必要なパッケージ
\usepackage[size=b6j]{bxpapersize} % B6 用紙サイズを指定
\usepackage[dvipdfmx]{graphicx} % 画像を挿入するためのパッケージ
\usepackage[dvipdfmx]{color} % 色をつけるためのパッケージ
\usepackage{pxrubrica} % ルビを振るためのパッケージ
\usepackage{multicol} % 複数段組を作るためのパッケージ
\setlength{\topmargin}{14mm} % 上下方向のマージン
\addtolength{\topmargin}{-1in} % 
\setlength{\oddsidemargin}{11mm} % 左右方向のマージン
\addtolength{\oddsidemargin}{-1in} % 
\setlength{\textwidth}{154mm} % B6 用
\setlength{\textheight}{108mm} % B6 用
\setlength{\headsep}{0mm} % 
\setlength{\headheight}{0mm} % 
\setlength{\topskip}{0mm} % 
\setlength{\parskip}{0pt} % 
\def\labelenumi{\theenumi、} % 箇条書きのフォーマット
\parindent = 0pt % 段落下げしない

 % B6 用テンプレート読み込み

\begin{document}
% begin header
%%%%% タイトルと作者 ここから %%%%%
\begin{minipage}[c]{0.7\hsize} % タイトルは上から 7 割
    \begin{center}
    % begin title
        {\LARGE
            星の舟唄 % タイトルを入れる
        }
        {\small 
            (平成二十年度寮歌) % 年などを入れる
        }
    % end title
    \end{center}
\end{minipage}
\begin{minipage}[c]{0.3\hsize} % 作歌作曲は上から 3 割
    \begin{flushright} % 下寄せにする
        % begin name
        黒瀬智子君 作歌・作曲 % 作歌・作曲者
        % end name
    \end{flushright}
\end{minipage}
%%%%% タイトルと作者 ここまで %%%%%
% (1,2,3,4,5 繰り返しなし)
% end header

% begin body
\vspace{1.5em} % タイトル, 作者と歌詞の間に隙間を設ける
\newcommand{\linespace}{0.5em} % 行間の設定
\newcommand{\blocksize}{0.5\hsize} % 段組間の設定
%%%%% 歌詞 ここから %%%%%
% begin lilycs
\begin{enumerate} % 番号の箇条書き ここから
    \begin{minipage}[c]{\blocksize}
    
        \vspace{\linespace}
        \item
        % 1.
        \ruby{雪}{}どけ\ruby{五月晴}{}れ\ruby{短}{}い\ruby{夏}{}の\ruby{日々}{}\\
        \ruby{黄金}{}のいちょう\ruby{並木}{}\\
        くぐれば\ruby{木枯}{}らし\\
        あしたも\ruby{同}{}じ\ruby{夕日}{}が\\
        \ruby{沈}{}むだろう\\
        \ruby{青春}{}は\ruby{退屈}{}だと\ruby{誰}{}か\ruby{歌}{}う
        
        \vspace{\linespace}
        \item
        % 2.
        まどろむ\ruby{子守唄人生}{}の\ruby{哲学}{}\\
        \ruby{雲}{}にかくれて\ruby{消}{}える\\
        \ruby{木}{}もれびの\ruby{夢}{}\\
        \ruby{眠}{}りをさまようまぶた\ruby{開}{}けば\\
        まこと\ruby{学成}{}りがたし\\
        \ruby{月}{}が\ruby{笑}{}う
        
        \vspace{\linespace}
        \item
        % 3.
        \ruby{悠々暮}{}らすこの\ruby{若}{}さを\\
        \ruby{持}{}て\ruby{余}{}し\\
        \ruby{港}{}にたどりつく\\
        さだめなき\ruby{小舟}{}\\
        \ruby{目}{}じるし\ruby{一}{}つの\ruby{星}{}\\
        \ruby{追}{}いかければ\\
        \ruby{流星雨}{}のごとく\ruby{目}{}をくらます
        
        \vspace{\linespace}
        \item
        % 4.
        あまたの\ruby{先人}{}が\ruby{説}{}く\ruby{壮大真理}{}\\
        この\ruby{脳}{}はそ\ruby{知}{}らねども\\
        \ruby{目}{}の\ruby{前}{}にあるは\\
        \ruby{瞳}{}の\ruby{暁}{}うつくしき\ruby{人}{}\\
        \ruby{千}{}の\ruby{論説}{}より\ruby{多}{}くを\ruby{語}{}る
        
        \vspace{\linespace}
        \item
        % 5.
        つつましい\ruby{志}{}が\ruby{正}{}しき\ruby{答}{}えか\\
        \ruby{道草}{}のかたわらに\\
        \ruby{咲}{}く\ruby{花}{}もある\\
        \ruby{学}{}べよ\ruby{遊}{}べよ\\
        \ruby{恋}{}せよ\ruby{舟}{}は\\
        \ruby{風}{}が\ruby{導}{}くままに\\
        \ruby{青}{}き\ruby{帆}{}を\ruby{張}{}る
    
    \end{minipage}
\end{enumerate} % 番号の箇条書き ここまで
% end lilycs
%%%%% 歌詞 ここまで %%%%%
% end body

\end{document}

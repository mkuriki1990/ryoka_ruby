\documentclass[10pt,b5j]{tarticle} % B6 縦書き
% \documentclass[10pt,b5j]{tarticle} % B6 縦書き
\AtBeginDvi{\special{papersize=128mm,182mm}} % B6 用用紙サイズ
\usepackage{otf} % Unicode で字を入力するのに必要なパッケージ
\usepackage[size=b6j]{bxpapersize} % B6 用紙サイズを指定
\usepackage[dvipdfmx]{graphicx} % 画像を挿入するためのパッケージ
\usepackage[dvipdfmx]{color} % 色をつけるためのパッケージ
\usepackage{pxrubrica} % ルビを振るためのパッケージ
\usepackage{multicol} % 複数段組を作るためのパッケージ
\setlength{\topmargin}{14mm} % 上下方向のマージン
\addtolength{\topmargin}{-1in} % 
\setlength{\oddsidemargin}{11mm} % 左右方向のマージン
\addtolength{\oddsidemargin}{-1in} % 
\setlength{\textwidth}{154mm} % B6 用
\setlength{\textheight}{108mm} % B6 用
\setlength{\headsep}{0mm} % 
\setlength{\headheight}{0mm} % 
\setlength{\topskip}{0mm} % 
\setlength{\parskip}{0pt} % 
\def\labelenumi{\theenumi、} % 箇条書きのフォーマット
\parindent = 0pt % 段落下げしない

 % B6 用テンプレート読み込み

\begin{document}
% begin header
%%%%% タイトルと作者 ここから %%%%%
\begin{minipage}[c]{0.7\hsize} % タイトルは上から 7 割
    \begin{center}
    % begin title
        {\LARGE
            寮友よ永遠に謳歌わん % タイトルを入れる
        }
        {\small 
            (昭和五十七年閉寮記念寮歌) % 年などを入れる
        }
    % end title
    \end{center}
\end{minipage}
\begin{minipage}[c]{0.3\hsize} % 作歌作曲は上から 3 割
    \begin{flushright} % 下寄せにする
        % begin name
        植木貴昭君 作歌\\串田厚司君 作曲 % 作歌・作曲者
        % end name
    \end{flushright}
\end{minipage}
%%%%% タイトルと作者 ここまで %%%%%
% (1,2,3 繰り返しなし)
% end header

% begin length
\vspace{1.5em} % タイトル, 作者と歌詞の間に隙間を設ける
\newcommand{\linespace}{0.5em} % 行間の設定
\newcommand{\blocksize}{0.5\hsize} % 段組間の設定
\newcommand{\itemmargin}{6em} % 曲番の位置調整の長さ
% end length
% begin body
%%%%% 歌詞 ここから %%%%%
\begin{enumerate} % 番号の箇条書き ここから
    \setlength{\itemindent}{\itemmargin} % 曲番の位置調整
    \begin{minipage}[c]{\blocksize}
    
        \vspace{\linespace}
        \item~\\
        % 1.
        \ruby{早緑}{}の\ruby{道駆}{}けし\ruby{我}{}\\
        \ruby{小川}{}に\ruby{映}{}る\ruby{延齢}{}の\ruby{花}{}\\
        \ruby{今}{}この\ruby{時}{}の\ruby{憧憬}{}に\\
        はるか\ruby{千嶂仰}{}ぎ\ruby{見}{}ん\\
        \ruby{心}{}の\ruby{静庵}{}ここにあり\\
        \ruby{我}{}が\ruby{夢馳}{}せし\ruby{夕暮}{}れに\\
        \ruby{明日}{}の\ruby{旅路}{}を\ruby{想}{}いなん
        
        \vspace{\linespace}
        \item~\\
        % 2.
        \ruby{北陵}{}の\ruby{夏歩}{}む\ruby{我}{}\\
        \ruby{今咲}{}きそろううす\ruby{影}{}のリラ\\
        \ruby{熱}{}き\ruby{涙}{}のほとばしり\\
        \ruby{正義}{}の\ruby{道}{}を\ruby{貫}{}かん\\
        \ruby{我}{}らが\ruby{誇}{}る\ruby{自治}{}の\ruby{魂}{}\\
        \ruby{清雅}{}にはゆる\ruby{星}{}よりも\\
        \ruby{深遠}{}にして\ruby{無限}{}なれ
        
        \vspace{\linespace}
        \item~\\
        % 3.
        \ruby{吹雪}{}の\ruby{中}{}に\ruby{立}{}てし\ruby{我}{}\\
        \ruby{原始}{}の\ruby{森}{}に\ruby{先人}{}のかげ\\
        \ruby{盃}{}かわす\ruby{寮友}{}と\\
        \ruby{過}{}ごせし\ruby{日々}{}の\ruby{感激}{}よ\\
        \ruby{我等}{}が\ruby{道}{}のしるべなり\\
        \ruby{我}{}が\ruby{春遠}{}き\ruby{北都}{}にも\\
        \ruby{誓}{}いの\ruby{絆永遠}{}に
    
    \end{minipage}
\end{enumerate} % 番号の箇条書き ここまで
%%%%% 歌詞 ここまで %%%%%
% end body

\end{document}

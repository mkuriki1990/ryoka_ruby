\documentclass[10pt,b5j]{tarticle} % B6 縦書き
% \documentclass[10pt,b5j]{tarticle} % B6 縦書き
\AtBeginDvi{\special{papersize=128mm,182mm}} % B6 用用紙サイズ
\usepackage{otf} % Unicode で字を入力するのに必要なパッケージ
\usepackage[size=b6j]{bxpapersize} % B6 用紙サイズを指定
\usepackage[dvipdfmx]{graphicx} % 画像を挿入するためのパッケージ
\usepackage[dvipdfmx]{color} % 色をつけるためのパッケージ
\usepackage{pxrubrica} % ルビを振るためのパッケージ
\usepackage{multicol} % 複数段組を作るためのパッケージ
\setlength{\topmargin}{14mm} % 上下方向のマージン
\addtolength{\topmargin}{-1in} % 
\setlength{\oddsidemargin}{11mm} % 左右方向のマージン
\addtolength{\oddsidemargin}{-1in} % 
\setlength{\textwidth}{154mm} % B6 用
\setlength{\textheight}{108mm} % B6 用
\setlength{\headsep}{0mm} % 
\setlength{\headheight}{0mm} % 
\setlength{\topskip}{0mm} % 
\setlength{\parskip}{0pt} % 
\def\labelenumi{\theenumi、} % 箇条書きのフォーマット
\parindent = 0pt % 段落下げしない

 % B6 用テンプレート読み込み

\begin{document}
% begin header
%%%%% タイトルと作者 ここから %%%%%
\begin{minipage}[c]{0.7\hsize} % タイトルは上から 7 割
    \begin{center}
    % begin title
        {\LARGE
            津軽の滄海の % タイトルを入れる
        }
        {\small 
            (昭和十三年寮歌) % 年などを入れる
        }
    % end title
    \end{center}
\end{minipage}
\begin{minipage}[c]{0.3\hsize} % 作歌作曲は上から 3 割
    \begin{flushright} % 下寄せにする
        % begin name
        二階堂孝一君 作歌\\高橋寛君 作曲 % 作歌・作曲者
        % end name
    \end{flushright}
\end{minipage}
%%%%% タイトルと作者 ここまで %%%%%
% (1,2,3 了あり)
% end header

% begin length
\vspace{1.5em} % タイトル, 作者と歌詞の間に隙間を設ける
\newcommand{\linespace}{0.5em} % 行間の設定
\newcommand{\blocksize}{0.5\hsize} % 段組間の設定
\newcommand{\itemmargin}{3em} % 曲番の位置調整の長さ
% end length
% begin body
%%%%% 歌詞 ここから %%%%%
\begin{enumerate} % 番号の箇条書き ここから
    \setlength{\itemindent}{\itemmargin} % 曲番の位置調整
    \begin{minipage}[c]{\blocksize}
    
        \vspace{\linespace}
        \item~\\
        % 1.
        \ruby{津軽}{}の\ruby{滄海}{}の\ruby{渦潮}{}わけて\\
        \ruby{雄大}{}き\ruby{想}{}ひを\ruby{北斗}{}に\ruby{馳}{}する\\
        \ruby{若}{}き\ruby{情懐}{}は\ruby{北溟}{}の\ruby{自然}{}に\\
        \ruby{抱擁}{}かれて\ruby{今野心培}{}ふ
        
    \end{minipage}
    \begin{minipage}[c]{\blocksize}
        
        \vspace{\linespace}
        \item~\\
        % 2.
        アカシヤの\ruby{白花散}{}り\ruby{敷}{}く\ruby{夕}{}べ\\
        \ruby{白銀}{}の\ruby{月仄}{}かに\ruby{浮}{}ぶ\\
        \ruby{牧場添}{}いの\ruby{野路逍遙}{}ひゆけば\\
        \ruby{羊}{}の\ruby{群}{}は\ruby{声}{}なく\ruby{去}{}りぬ
        
    \end{minipage}
    \begin{minipage}[c]{\blocksize}
        
        \vspace{\linespace}
        \item~\\
        % 3.
        \ruby{石狩}{}の\ruby{平野}{}に\ruby{爽夏訪}{}れて\\
        \ruby{原始}{}の\ruby{大森}{}は\ruby{緑影}{}も\ruby{小暗}{}し\\
        \ruby{郭公}{}の\ruby{朗声静寂}{}に\ruby{徹}{}り\\
        \ruby{清涼}{}しき\ruby{朝}{}の\ruby{熟睡}{}を\ruby{破}{}る
        
    \end{minipage}
    \begin{minipage}[c]{\blocksize}
        
        \vspace{\linespace}
        \item~\\
        % 4.
        \ruby{豊穣}{}の\ruby{秋}{}の\ruby{讃歌}{}を\ruby{奏}{}で\\
        ポプラの\ruby{高梢}{}さやかに\ruby{揺}{}ぐ\\
        \ruby{北漠}{}の\ruby{蒼穹紺碧}{}に\ruby{透}{}き\\
        \ruby{生}{}の\ruby{歓喜我}{}が\ruby{胸懐}{}に\ruby{充溢}{}つ
        
    \end{minipage}
    \begin{minipage}[c]{\blocksize}
        
        \vspace{\linespace}
        \item~\\
        % 5.
        \ruby{飄々}{}の\ruby{風声疎林}{}に\ruby{沈潜}{}み\\
        \ruby{無眼}{}の\ruby{静寂天地}{}に\ruby{充満}{}てり\\
        \ruby{寒月}{}は\ruby{鋭利}{}く\ruby{虚空}{}を\ruby{截}{}りて\\
        \ruby{我}{}が\ruby{行}{}く\ruby{孤影}{}よ\ruby{霜仁凍}{}りぬ
        
    \end{minipage}
    \begin{minipage}[c]{\blocksize}
        
        \vspace{\linespace}
        \item~\\
        % 6.
        \ruby{白銀}{}の\ruby{六華荘厳}{}に\ruby{咲}{}く\\
        \ruby{山嶺奥深}{}く\ruby{彷徨}{}れ\ruby{行}{}けば\\
        ああ\ruby{壮麗}{}の\ruby{樹氷}{}の\ruby{森}{}よ\\
        \ruby{冬}{}の\ruby{神秘}{}に\ruby{我}{}が\ruby{胸戦傈}{}ふ
        
    \end{minipage}
    \begin{minipage}[c]{\blocksize}
        
        \vspace{\linespace}
        \item~\\
        % 7.
        さあれ\ruby{戦塵東亜}{}を\ruby{閉鎖}{}し\\
        \ruby{全支}{}の\ruby{空}{}に\ruby{硝煙昏冥}{}し\\
        \ruby{大陸飛翔}{}る\ruby{荒鷲想}{}へば\\
        \ruby{雄心湧}{}きて\ruby{若}{}き\ruby{熱血滾}{}る
        
    \end{minipage}
    \begin{minipage}[c]{\blocksize}
        
        \vspace{\linespace}
        \item~\\
        % 8.
        \ruby{先人}{}の\ruby{絢夢残}{}れる\ruby{原始林}{}に\\
        \ruby{寮祭}{}の\ruby{犠牲}{}の\ruby{火柱廻}{}りて\\
        いざ\ruby{寮友}{}どちよ\ruby{永久}{}に\ruby{謳歌}{}はん\\
        \ruby{意気}{}と\ruby{血潮}{}の\ruby{三年}{}の\ruby{契}{}り
    
    \end{minipage}
\end{enumerate} % 番号の箇条書き ここまで
%%%%% 歌詞 ここまで %%%%%
% end body

\end{document}
